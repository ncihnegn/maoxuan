
\title{统一战线中的独立自主问题}
\date{一九三八年十一月五日}
\thanks{这是毛泽东在中国共产党第六届中央委员会扩大的第六次全体会议上所作的结论的一部分。结论是在一九三八年十一月五日和六日作的,这一部分是在五日讲的。统一战线中的独立自主问题,是当时毛泽东同陈绍禹在抗日民族统一战线问题上意见分歧的突出问题之一。这在本质上就是统一战线中无产阶级领导权的问题。关于这种意见分歧,毛泽东在一九四七年十二月二十五日的报告《目前形势和我们的任务》中曾作了以下简要的总结:“抗日战争时期,我党反对了和这种投降主义思想(编者按:指第一次国内革命战争时期陈独秀的投降主义思想)相类似的思想,即是对于国民党的反人民政策让步,信任国民党超过信任人民群众,不敢放手发动群众斗争,不敢在日本占领地区扩大解放区和扩大人民的军队,将抗日战争的领导权送给国民党。我党对于这样一种软弱无能的腐朽的违背马克思列宁主义原则的思想,进行了坚决的斗争,坚决地执行了‘发展进步势力,争取中间势力,孤立顽固势力’的政治路线,坚决地扩大了解放区和人民解放军。这样,就不但保证了我党在日本帝国主义侵略时期能够战胜日本帝国主义,而且保证了我党在日本投降以后蒋介石举行反革命战争时期,能够顺利地不受损失地转变到用人民革命战争反对蒋介石反革命战争的轨道上,并在短时期内取得了伟大的胜利。这些历史教训,全党同志都要牢记。”}
\maketitle


\section{帮助和让步应该是积极的,不应该是消极的}

为了长期合作,统一战线中的各党派实行互助互让是必需的,但应该是积极的,不是消极的。我们必须巩固和扩大我党我军,同时也应赞助友党友军的巩固和扩大;人民要求政府满足自己的政治经济要求,同时给政府以一切可能的利于抗日的援助;工人要求厂主改良待遇,同时积极作工以利抗日;地主应该减租减息,同时农民应该交租交息,团结对外。这些都是互助的原则和方针,是积极的方针,不是消极的片面的方针。互让也是如此。彼此不挖墙脚,彼此不在对方党政军内组织秘密支部;在我们方面,就是不在国民党及其政府、军队内组织秘密支部,使国民党安心,利于抗日。“有所不为而后可以有为”\mnote{1},正是这种情形。没有红军的改编,红色区域的改制,暴动政策的取消,就不能实现全国的抗日战争。让了前者就得了后者,消极的步骤达到了积极的目的。“为了更好的一跃而后退”\mnote{2},正是列宁主义。把让步看作纯消极的东西,不是马克思列宁主义所许可的。纯消极的让步是有过的,那就是第二国际的劳资合作论\mnote{3},把一个阶级一个革命都让掉了。中国前有陈独秀\mnote{4},后有张国焘\mnote{5},都是投降主义者;我们应该大大地反对投降主义。我们的让步、退守、防御或停顿,不论是向同盟者或向敌人,都是当作整个革命政策的一部分看的,是联系于总的革命路线而当作不可缺少的一环看的,是当作曲线运动的一个片断看的。一句话,是积极的。

\section{民族斗争和阶级斗争的一致性}

用长期合作支持长期战争,就是说使阶级斗争服从于今天抗日的民族斗争,这是统一战线的根本原则。在此原则下,保存党派和阶级的独立性,保存统一战线中的独立自主;不是因合作和统一而牺牲党派和阶级的必要权利,而是相反,坚持党派和阶级的一定限度的权利;这才有利于合作,也才有所谓合作。否则就是将合作变成了混一,必然牺牲统一战线。在民族斗争中,阶级斗争是以民族斗争的形式出现的,这种形式,表现了两者的一致性。一方面,阶级的政治经济要求在一定的历史时期内以不破裂合作为条件;又一方面,一切阶级斗争的要求都应以民族斗争的需要(为着抗日)为出发点。这样便把统一战线中的统一性和独立性、民族斗争和阶级斗争,一致起来了。

\section{“一切经过统一战线”是不对的}

国民党是当权的党,它至今不许有统一战线的组织形式。刘少奇同志说的很对,如果所谓“一切经过”就是经过蒋介石和阎锡山,那只是片面的服从,无所谓“经过统一战线”。在敌后,只有根据国民党已经许可的东西(例如《抗战建国纲领》\mnote{6}),独立自主地去做,无法“一切经过”。或者估计国民党可能许可的,先斩后奏。例如设置行政专员,派兵去山东之类,先“经过”则行不通。听说法国共产党曾经提出过这个口号,那大概是因为法国有了各党的共同委员会,而对于共同决定的纲领,社会党方面不愿照做,依然干他们自己的,故共产党有提此口号以限制社会党之必要,并不是提此口号以束缚自己。中国的情形是国民党剥夺各党派的平等权利,企图指挥各党听它一党的命令。我们提这个口号,如果是要求国民党“一切”都要“经过”我们同意,是做不到的,滑稽的。如果想把我们所要做的“一切”均事先取得国民党同意,那末,它不同意怎么办?国民党的方针是限制我们发展,我们提出这个口号,只是自己把自己的手脚束缚起来,是完全不应该的。在现时,有些应该先得国民党同意,例如将三个师的番号扩编为三个军的番号,这叫做先奏后斩。有些则造成既成事实再告诉它,例如发展二十余万军队,这叫做先斩后奏。有些则暂时斩而不奏,估计它现时不会同意,例如召集边区议会之类。有些则暂时不斩不奏,例如那些如果做了就要妨碍大局的事情。总之,我们一定不要破裂统一战线,但又决不可自己束缚自己的手脚,因此不应提出“一切经过统一战线”的口号。“一切服从统一战线”,如果解释为“一切服从”蒋介石和阎锡山,那也是错误的。我们的方针是统一战线中的独立自主,既统一,又独立。


\begin{maonote}
\mnitem{1}见《孟子·离娄下》。原文是:“人有不为也,而后可以有为。”
\mnitem{2}见列宁《黑格尔〈哲学史讲演录〉一书摘要》(《列宁全集》第55卷,人民出版社1990年版,第239页)。
\mnitem{3}“劳资合作论”,是第二国际主张在资本主义国家内,无产阶级与资产阶级合作,反对用革命手段推翻资产阶级统治以建立无产阶级专政的一种反动理论。
\mnitem{4}见本书第一卷\mxnote{中国革命战争的战略问题}{4}。
\mnitem{5}见本书第一卷\mxnote{论反对日本帝国主义的策略}{24}。
\mnitem{6}见本卷\mxnote{陕甘宁边区政府、第八路军后方留守处布告}{3}。
\end{maonote}
