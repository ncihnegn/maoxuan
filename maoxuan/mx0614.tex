
\title{记者头脑要冷静}
\date{一九五八年十一月二十一日}
\thanks{这是毛泽东同志同新华通讯社社长、《人民日报》总编辑吴冷西谈话的要点。}
\maketitle


做报纸工作的,做记者工作的,对遇到的问题要有分析,要有正确的看法、正确的态度。

矛盾有正面,有侧面。看问题一定要看到矛盾的各个方面。群众运动有主流,有支流。到下面去看,对运动的成绩和缺点要有辩证的观点,不要把任何一件事情绝对化。好事情不要全信,坏事情也不要只看到它的消极一面。比方瞒产,我对隐瞒产量是寄予同情的。当然,不说实话,是不好的。但是为什么瞒产?有很多原因,最主要的原因是想多吃一点,值得同情。瞒产,除了不老实这一点以外,没有什么不好。隐瞒了产量,粮食依然还在。瞒产的思想要批判,但是对发展生产没有大不了的坏处。

虚报不好,比瞒产有危险性。报多了,拿不出来。如果根据多报的数字作生产计划,有危险性,作供应计划,更危险。

记者到下面去,不能人家说什么,你就反映什么,要有冷静的头脑,要作比较。

强迫命令,不好。一定的命令还是需要的。如果什么事情都命令,就不好了。有些事情也并非强迫命令,例如党委的决议,一定要照办。总要有集中。集中的过程要有民主。要提倡民主作风,反对强迫命令。

记者要善于比较。唐朝有一个太守,他问官司,先去了解原告被告周围的人和周围的情况,然后再审原告被告。这叫作“勾推法”。这就是比较,同周围的环境比较。记者要善于运用这种方法。不要看到好的就认为全好,看到坏的就认为全坏。如果别人说全好,那你就问一问:是不是全好?如果别人说全坏,那你就问一问:一点好处没有吗?

现在全国到处乱哄哄的,大跃进。成绩很大,头脑热了些。

记者的头脑要冷静,要独立思考,不要人云亦云。这种思想方法,首先是各新华分社和《人民日报》的记者、北京的编辑部要有。不要人家讲什么,就宣传什么,要经过考虑。

记者,特别是记者头子,头脑要清楚,要冷静。
