
\title{纠正土地改革宣传中的“左”倾错误}
\date{一九四八年二月十一日}
\thanks{这是毛泽东为中共中央起草的对党内的指示。}
\maketitle


最近几个月中,许多地方的通讯社和报纸,不加选择地没有分析地传播了许多包含“左”倾错误偏向的不健全的通讯或文章。例如:

一、不是宣传依靠贫雇农,巩固地联合中农,消灭封建制度的路线,而是孤立地宣传贫雇农路线。不是宣传无产阶级联合一切劳动人民、受压迫的民族资产阶级、知识分子和其它爱国分子(其中包括不反对土地改革的开明绅士),推翻帝国主义、封建主义和官僚资本主义的统治,建立中华人民共和国和人民民主政府,而是孤立地宣传所谓贫雇农打江山坐江山,或者说民主政府只是农民的政府,或者说民主政府只应该听工人和贫雇农的意见,而对中农,对独立劳动者,对民族资产阶级,对知识分子等,则一概不提。这是严重的原则性的错误。而我们的通讯社、报纸或广播电台竟将这类通讯发表。各地党委宣传部,对于此类错误竟没有任何的反映。此类宣传,在过去几个月中虽然不是普遍的,但是相当多,以致造成了一种空气,使人们误认为似乎这是正确的领导思想。甚至因为陕北广播电台播发了某些不正确的新闻,人们竟误认为这是被中央认可的意见。

二、在整党问题上,关于既反对忽视成分、又反对唯成分论的宣传,有些地区不够有力,甚至有唯成分论的错误宣传。

三、在土地改革问题上,关于既反对观望不前、又反对急性病的宣传,有些地区是抓紧了;但在许多地区却助长急性病,甚至发表赞扬急性病的东西。在领导者和群众的关系问题上,关于既反对命令主义、又反对尾巴主义的宣传,有些地区是注意了;但在许多地区却错误地强调所谓“群众要怎样办就怎样办”,迁就群众中的错误意见。甚至对于并非群众的、而只是少数人的错误意见,也无批判地接受。否定了党的领导作用,助长了尾巴主义。

四、在工商业和工人运动的方针上,对于某些解放区存在着的严重的“左”的倾向,或者加以赞扬,或者熟视无睹。

总之,过去几个月的宣传工作,正确地反映和指导了战争、土地改革、整党、生产、支援前线这些伟大斗争,帮助了这些斗争取得了伟大成绩,并且在宣传工作中占着主要成分,这是必须首先承认的。但是也必须看到一些错误缺点。其特点就是过左。其中有些是完全违背马克思列宁主义原则立场和完全脱离中央路线的。望各中央局、中央分局及其宣传部,新华总社和各地总分社,以及各地报纸的工作同志们,根据马克思列宁主义原则和中央路线,对过去几个月的宣传工作,加以检查,发扬成绩,纠正错误,务使对于战争、土地改革、整党、工人运动这些伟大的斗争,对于这一整个反帝反封建的革命,保障其获得胜利。这种检查,责成各地党委宣传部和新华总社负主要责任,并将结果用自己的名义写一个政策性的报告给中央宣传部。
