
\title{辩证法唯物论}
\date{讲授提纲}
\thanks{一九三七年年四月至七月间,毛泽东在延安抗日军政大学讲授辩证法唯物论。《辩证法唯物论(讲授提纲)》写于一九三七年,曾在《抗战大学》第七期至第八期(一九三八年年四月至六月)连载。据以录入的中国人民解放军政治学院训练部翻印本,年代不详。第二章第十一节和第三章中的部分章节分别为\mxart{实践论}与\mxart{矛盾论}的最初版本,与后来的毛选版本有较多出入。 文中所有注释均为录入者所加,文中不再说明。}
\maketitle


\section{第一章 唯心论与唯物论}

本章讨论下列各问题:(一)哲学中的两军对战;(二)唯心论与唯物论的区别;(三)唯心论发生与发展的根源;(四)唯物论发生与发展的根源。

\subsection{(一)哲学中的两军对战}

全部哲学史,都是唯心论和唯物论这两个互相对抗的哲学派别的斗争和发展的历史,一切哲学思潮和派别,都是这两个基本派别的变相。

各种哲学学说,都是隶属于一定社会阶级的人们所创造的。这些人们的意识,又是历史地被一定的社会生活所决定。所有的哲学学说,表现着一定社会阶级的需要,反映着社会生产力发展的水平和人类认识自然的历史阶段。哲学的命运,看哲学满足社会阶级的需要之程度如何而定。

唯心论和唯物论的社会根源,存在于阶级的矛盾的社会结构中。最初唯心论之发生是原始野蛮人类迷妄无知的产物。此后生产力发展,促使科学知识也随之发展,唯心论理应衰退,唯物论理应起而代之。然而从古至今,唯心论不但不曾衰退,反而发展起来,同唯物论竟长争高,互不相下,原因就在于社会有阶级的划分。一方面压迫阶级为着自己的利益,不得不发展与巩固其唯心论学说;一方面被压迫阶级同样为着自己的利益,不得不发展与巩固其唯物论学说。唯心论和唯物论学说都是作为阶级斗争的工具而存在,在阶级没有消灭以前,唯心论和唯物论的对战是不会消灭的。唯心论在自己的历史发展过程中,代表剥削阶级的意识形态,起着反动的作用。唯物论则是革命阶级的宇宙观,他在阶级社会内,从对反动哲学的唯心论之不断的战斗中生长与发展起来。由此,哲学中唯心论与唯物论的斗争,始终反映着反动阶级与革命阶级在利害上的斗争。哲学中的某一倾向,不管哲学者自身意识到与否,结局总是被他们所属阶级的政治方向所左右的。哲学上的任何倾向,总是直接间接助长着他们所属阶级的根本的政治利害。在这个意义下,哲学中的一定倾向的贯彻,便是他们所属阶级的政策之特殊形态。

马克思主义的哲学——辩证法唯物论的特征在于,在于要明确地理解一切社会意识(哲学也在内)的阶级性,公然声明它那无产阶级的性质,向有产阶级的唯心论哲学作坚决的斗争,并且把自己的特殊任务,从属于推翻资本主义组织、建立无产阶级专政,与建设社会主义社会的一般任务之下。在中国目前阶段上,哲学的任务,是从属于推翻帝国主义与半封建制度、彻底实现资产阶级的民主主义,并准备转变到社会主义与共产主义社会去的一般任务之下,哲学的理论与政治的实践是应该密切联系着的。

\subsection{(二)唯心论与唯物论的区别}

唯心论与唯物论的根本区别在那里呢?在对于哲学的根本问题,即精神与物质的关系问题(意识与存在的关系问题)之相反的回答。唯心论认精神(意识,观念,主体)为世界一切的根源,物质(自然界及社会客体)不过为其附属物。唯物论认物质离精神而独立存在,精神不过为其附属物。从这个根本问题的相反的回答出发,就生出一切问题上的纷歧意见来。

在唯心论看来,世界或者是我们各种知觉的综合,或者是我们的或世界的理性所创造的精神过程。对外面的物质世界或者完全把它看成虚构的幻象,或者把它看成精神元素之物质的外壳。人类的认识,是主体的自动,是精神的自己产物。

唯物论相反,认宇宙的统一就在它的物质性。精神(意识)是物质的本性之一,是物质发展到一定阶段时才发生的。自然,物质,客观世界存在于精神之外,离精神而独立。人的认识,是客观外界的反映。

\subsection{(三)唯心论发生与发展的根源}

唯心论认物质为精神的产物,颠倒着实在世界的姿态,这种哲学的发生与发展的根源何在?

前面说过,最初唯心论之发生是原始野蛮人类迷妄无知的产物。但在生产发展之后促使唯心论形成哲学思潮之首先的条件,乃是体力劳动与精神劳动的分裂。社会生产力发展的结果,社会发生分工,分工再发展,分出了专门从事精神劳动的人们。但在生产力贫弱时期,两者的分裂还没有达到完全分离的程度。到了阶级出现、私产发生,剥削成为支配阶级存在的基础之时,就起了大变化了,精神劳动成为支配阶级的特权,体力劳动成为被压迫阶级的命运。支配阶级开始颠倒地去考察自己与被压迫阶级之间的相互关系,不是劳动者给他们以生活资料,反而是他们以生活资料给与劳动者,因此他们鄙视体力劳动,发生了唯心论的见解。消灭体力劳动与精神劳动的区别是消灭唯心论哲学的条件之一。

使唯心论哲学能够发展的社会根源,主要的还在于这种哲学意识它表现剥削阶级的利害。唯心论哲学在一切文化领域的优越,应该拿这个去说明。假如没有剥削阶级的存在,唯心论就会失掉它的社会根据。唯心论哲学之最后消灭,必须在阶级消灭与共产主义社会成立之后。

使唯心论能够发达、深化,并有能力同唯物论斗争,还须在人类的认识过程中找寻其本源。人类在使用概念来思考的时候,存在着溜到唯心论去的可能性。人类在思考时不能不使用概念,这就容易使我们的认识分裂为二方面:一方面,是个别的与特殊性质的事物,一方面是一般性质的概念(例如“延安是城”这个判断)。特殊和一般本来是互相联系不可分裂的,分裂就脱离了客观真理。客观真理是表现于一般与特殊之一致的。没有特殊,一般就不存在;没有一般,也不会有特殊。把一般同特殊脱离开来,即把一般当作客观实体看待,把特殊只当作一般之存在的形式,这就是一切唯心论所采用的方法。一切唯心论者都是拿意识、精神或观念来代替离开人的意识而独立存在的客观实体的。从这里出发,唯心论者便强调人类意识在社会实践中的能动性,他们不能指出意识受物质限制的这种唯物论的真理,却主张只有意识是能够动的,物质不过是不动的集合体。加上被阶级的本性所驱策,唯心论者便用一切方法把意识的能动性夸张起来,片面地发展了它,使这一方面在心智之中无限制地胀大,成为支配的东西,掩蔽着别一方面并使之服从,而把这一人工地胀大的东西确定为一般的宇宙观,以至化为物神或偶象。经济学上的唯心论,过分夸大交换中非本质的一方面,把供求法则提高到资本主义的根本法则。许多人看到科学在社会生活上发生了能动的作用,不知道这种作用受一定的社会生产关系所规定与限制,而作出科学是社会发动力的结论。唯心论历史家把英雄看成历史的创造者,唯心论政治家把政治看成万能的东西,唯心论军事家实行拼命主义的作战,唯心论革命家主张白朗基主义,×××\mnote{1}说要复兴民族惟有恢复旧道德,都是过分夸张主观能动性的结果。我们的思维不能以一次反映来当作全体看的对象,而是构成一个具有接近于现实的,一切种类的无数色调的,生动的,认识之辩证法的过程。唯心论依据于思维的这种特性,夸大其个别方面,不能给过程以正确的反映,反把过程弄弯曲了。列宁说:“人类的认识不是直线的,而是曲线的。这一曲线之任何一段,都可以变为一段单独的完整的直线,这段直线就有引你陷入迷阵的可能。直线性和片面性是见树不见林和呆板固执性,主观主义和主观盲动性——这些就是唯心论的认识论的根源”\mnote{2}。哲学的唯心论是将认识的一个片段或一个方面,片面地夸张成为一种脱离物质脱离自然的神化的绝对体。唯心论就是宗教的教义,这是很对的。

马克思以前的唯物论(机械唯物论)没有强调思维在认识上的能动性,仅给以被动作用,把它当作反映自然的镜子看。机械唯物论对唯心论采取横暴的态度,不注意其认识论的根源,因此不能克服唯心论。只有辩证法唯物论,才正确地指出思维的能动性,同时又指出思维受物质的限制;指出思维从社会实践中发生,同时又能动地指导实践。只有这种辩证法的“知行合一”论,才能彻底地克服唯心论。

\subsection{(四)唯物论发生与发展的根源}

承认离意识而独立存在于外界的物质是唯物论的基础。这一基础是人类从实践中得到的。劳动生产的实践,阶级斗争的实践,科学实验的实践,使人类逐渐脱离迷信与妄想(唯心论),逐渐认识世界之本质,而到达于唯物论。

屈服于自然力之前,而只能使用简单工具的原始人类,不能说明周围的事变,因而求助于神灵,这就是宗教同唯心论的起源。

然而人类在长期生产过程中同周围的自然界接触,作用于自然界,变化着自然界,创造着衣食住用的东西,使之适合于人类的利益,使人类深信物质是客观地存在着。

人类在社会生活中,人同人之间互相发生关系与影响,在阶级社会中并且实行着阶级斗争。被压迫阶级考虑形势,估计力量,建立计划,在他们的斗争成功时,使他们确信自己的见解并不是幻想的产物,而是客观上存在着的物质世界的反映。被压迫阶级因为采取错误的计划而失败,又因为改正其计划而成功,使他们懂得只有主观的计划依靠于对客观世界的物质性与规律性的正确的认识,才能达到目的。

科学的历史给人类证明了世界的物质性及其规律性,使人类觉悟到宗教与唯心论的幻想之无用,而到达于唯物论的结论。

总之,人类的实践史——自然斗争史、阶级斗争史、科学史,在长久年月中,为了生活与斗争的必要,考虑物质的现实及其法则,证明了唯物论哲学的正确性,找到了自己斗争的思想工具——唯物论哲学。社会的生产发展越发进到高度,阶级斗争越发发展,科学认识越发暴露了自然的“秘密”,唯物论哲学就越发发展与巩固,人类便能逐渐从自然与社会的双重压迫下解放出来。

资产阶级在为了向封建阶级斗争的必要及无产阶级还没有威胁他们的时候,也曾经找到了并使用了唯物论作为自己斗争的工具,也曾经确信周围的事物是物质的产物,而不是精神的产物。直至他们自己变成了统治者,无产阶级的斗争又威胁着他们时,才放弃这个“无用”的工具,重新拿起另一个工具——哲学的唯心论。中国资产阶级的代言人戴季陶、吴稚晖,在1927年以前及其以后思想的变化——从唯物论到唯心论的变化,就是眼前的活证据。

资本主义的掘墓人——无产阶级,他们本质上是唯物论的。但由于无产阶级是历史上最进步的阶级,这就使得无产阶级的唯物论不同于资产阶级的唯物论,是更彻底更深刻的,只有辩证法的性质,没有机械论的性质。无产阶级吸收了人类全历史中的一切实践,经过他们的代言人与领导者——马克思、恩格斯之手,造成了辩证法唯物论,不但主张物质离人的意识而独立存在,而且主张物质是变化的,成为整个完整系统的崭新的世界观与方法论,这就是马克思主义的哲学。

\section{第二章 辩证法唯物论}

这个题目中准备讨论下列各问题:(一)无产阶级革命的武器——辩证法唯物论;(二)过去哲学遗产同辩证法唯物论的关系;(三)在辩证法唯物论中宇宙观和方法论的一致;(四)哲学对象问题;(五)物质论;(六)运动论;(七)时空论;(八)意识论;(九)反映论;(十)真理论;(十一)实践论。下面简述这些问题的观点。

\subsection{(一)辩证法唯物论是无产阶级革命的武器}

这个问题在第一章中已经说过,这里再简单地说一点。

辩证法唯物论,是无产阶级的宇宙观。历史给予无产阶级以消灭阶级的任务,无产阶阶级就用辩证法唯物论作为他们斗争的精神上的武器,作为他们各种见解之哲学基础。辩证法唯物论这种宇宙观,只有当我们站在无产阶阶级的立场去认识世界的时候,才能够被我们正确地和完整地把握住;只有从这种立场出发,现实世界才能真正客观地被认识。这是因为一方面只有无产阶级才是最先进与最革命的阶级;又一方面,只有辩证法唯物论才是高度的和严密的科学性同彻底的和不妥协的革命性密切地结合着的一种最正确的和最革命的宇宙观和方法论。

中国无产阶级担负了经过资产阶级民主革命到达社会主义与共产主义的历史任务,必须采取辩证法唯物论作为自己精神的武器。如果辩证法唯物论——一种最正确最革命的宇宙观和方法论被中国共产党、及一切愿意站在无产阶级立场的广大革命份子所掌握,他们就能够正确地了解革命运动的发展变化,提出革命的任务,团结自己和同盟者的队伍,战胜反动的理论,采取正确的行动,避免工作的错误,达到解放中国与改造中国的目的。辩证法唯物论对于指导革命运动的干部人员尤属必修的科目,因为主观主义与机械观这两种错误的理论与工作方法,常常在干部人员中间存在着,因此常常引导干部人员违犯马克思主义,在革命运动中走入歧途。要避免与纠正这种缺点,只有自觉地研究与了解辩证法唯物论,把自己的头脑重新武装起来。

\subsection{(二)旧的哲学遗产同辩证法唯物论的关系}

现代的唯物论,不是过去各种哲学学说的简单的继承者,它是从反对过去统治哲学的斗争中,从科学解除其唯心论和神秘性的斗争中产生和成长起来的。马克思主义的哲学——辩证法唯物论,不但继承了唯心论的最高产物——黑格尔学说的成果,同时还克服了这一学说的唯心论,唯物地改造了他的辩证法。马克思主义又不但是一切过去唯物论发展的继续和完成,同时还是一切过去唯物论的狭隘性之反对者,即机械的直觉的唯物论(主要的是法国唯物论与费尔巴哈唯物论)之反对者。马克思主义的哲学——辩证法唯物论,继承了过去文化之科学的遗产,同时又给此种遗产以革命的改造,形成了一种历史上从来没有过的最正确最革命和最完备的哲理的科学。

中国在1919年五四运动以后,随着中国无产阶级自觉地走上政治舞台及科学水平之提高,发生了与发展着马克思主义的哲学运动。然而在它的第一时期,中国的唯物论思潮中唯物辩证法的了解还很微弱,受资产阶级影响的机械唯物论,和德波林派的主观主义风气占着主要的成分。1927年革命失败以后,马克思列宁主义的了解进了一步,唯物辩证法的思想逐渐发展起来。到了最近,由于民族危机与社会危机的严重性,也由于苏联哲学清算运动的影响,在中国思想界发展了一个广大的唯物辩证法运动。这个运动,目前虽还在青年的阶段上,然从其广大的姿态来看,它将随着中国与世界无产阶级同革命人民的革命斗争之发展,以横扫的阵势树立自己的权威,指导中国革命运动,勇往迈进,定下中国无产阶级领导中国革命进入胜利之途的基础。

由于中国社会进化的落后,中国今日发展着的辩证法唯物论哲学思潮,不是从继承与改造自己哲学的遗产而来的,而是从马克思列宁主义的学习而来的。然而要使辩证法唯物论思潮在中国深入与发展下去,并确定地指导中国革命向着彻底胜利之途,便必须同各种现存的反动哲学作斗争,在全国思想战线上树立批判的旗帜,并因而清算中国古代的哲学遗产,才能达到目的。

\subsection{(三)辩证法唯物论中宇宙观和方法论的一致性}

辩证法唯物论是无产阶级的宇宙观,同时又是无产阶级认识周围世界的方法和革命行动的方法,它是宇宙观和方法论的一致体。唯心论的马克思主义修正派认为辩证法唯物论的全部实质只在于它的“方法”。他们把方法从一般哲学的宇宙观割裂开来,把辩证法从唯物论割裂开来。他们不了解马克思主义的方法论——辩证法,不是如同黑格尔一样的唯心的辩证法,而是唯物的辩证法,马克思主义的方法论是丝毫也不能离开它的宇宙观的,另一方面机械唯物论者却又把马克思主义的哲学看作一般哲学的宇宙观,割去了它的辩证法,而且认为这种宇宙观就是机械的自然科学之各种结论。他们不了解马克思主义的唯物论不是简单的唯物论,而是辩证法的唯物论。对于马克思主义哲学之这两种割裂的看法都是错误的,辩证法唯物论是宇宙观和方法论的一致体。

\subsection{(四)唯物辩证法的对象问题——唯物辩证法是研究什么的?}

列宁把(作为马克思主义的哲理科学看的)唯物辩证法看做关于客观世界的发展法则及(在辩证法各范畴中反映这客观世界的)认识的发展法则的学问。他说:论理学不是关于思维的外在形式的学问,而是关于一切物质的,自然的,及精神的事物之发展法则的学问,即关于世界的一切具体内容及其认识之发展法则的学问。换言之,论理学是关于世界认识之历史的总计、总和、结论。列宁虽然把作为一般的科学方法论看的唯物辩证法的意义强调起来,然而这是因为辩证法系由世界认识的历史中得出来的结论。因此他说:“辩证法就是认识的历史”\mnote{3}。

上述列宁对于当作科学看的唯物辩证法及其对象所给与的定义,他的意思是说:第一、唯物辩证法与其他任何科学一样,有它的研究对象,这个对象便是自然、历史和人类思维之最一般的发展法则。并且研究的时候,唯物辩证法的任务,不是从头脑里想出存在于各现象间的关联,而是要在各现象本身中观察出它们之间的关系来。列宁的这种见解同少数派唯心论者把(事实上离开了具体科学及具体知识的)范畴的研究当做唯物辩证法的对象之间,存在着根本的区别,因为少数派唯心论者企图建立一个从认识历史社会科学和自然科学的现实发展中游离了的各范畴的哲学体系,这样他们就事实上放弃了唯物辩证法。第二、各个科学分科(数学、力学、化学、物理学、生物学、经济学及其他自然科学、社会科学),是研究物质世界及其认识之发展的各个方面。因此各个科学的法则是狭隘的,片面的,被各个具体研究领域所限制了的。唯物辩证法则不然,它是一切具体科学中的一切有价值的一般内容,及人类的其它一切科学认识之总计、结论、加工和普遍化。这样,唯物辩证法的概念、判断和法则,是极其广泛的(包含着一切科学的最一般的法则,因此也包含着物质世界的本质的)各种规律性和规定,这是一方面。在这方面,它是宇宙观。另一方面,唯物辩证法是从一切空想、僧侣、主义、和形而上学解放出来的真正科学认识上的论理学和认识论的基础,因此它同时又是研究具体科学的唯一确实的、有客观正确性的方法论。我们说唯物论辩证法或辩证法唯物论是宇宙观和方法论的一致体,在这里更加明白了。这样对于否认哲学存在权的马克思主义哲学的歪曲者和庸俗化者的错误也可以懂得了。

关于哲学对象问题,马克思、恩格斯和列宁,都反对使哲学脱离实在的现实,使哲学变为某种独立实质的东西。指出了那根据实在生活和实在关系的分析而生长出来的哲学之必然性,反对单单以论理观念和论理观念的自然做研究的对象,如同形式论理学及少数派唯心论的那种干法。所谓根据实在生活和实在关系的分析生长出来的哲学就是唯物辩证法这种论发展的学说。马克思,恩格斯和列宁,都解说唯物辩证法为论发展的学说。恩格斯称唯物辩证法为“论自然社会及思维之一般的发展法则”\mnote{4}的学说。列宁把唯物辩证法看作“最多方面的,内容最丰富的,和最深刻的发展学说。”\mnote{5}他们都认为在这种学说以外的其他一切哲学学说所述一切发展原则的公式,概属狭隘的无内容的“截去了自然和社会之实际发展过程的东西”\mnote{6}(列宁)。至于唯物辩证法之所以被称为最多方面的,内容最丰富的和最深刻的发展学说的原故,乃是因为唯物辩证法是最多方面地和最丰富地、最深刻地反映了自然和社会变化过程中的矛盾性和飞跃性,而不是因为别的东西。

在哲学对象问题中还要解决一个问题,就是辩证法、论理学及认识论的一致性的问题。

列宁着重指出辩证法、论理学及认识论的同一性,说这是“极其重要的问题”,说“三个名词是多余的,它们只是一个东西”\mnote{7},根本反对那些马克思主义修正派把三者当做完全各别独立的学说去处理的那种干法。

唯物辩证法是唯一科学的认识论,也是唯一科学的论理学。唯物辩证法研究吾人对外界认识的发生及发展,研究由不知到知,由不完全的知到更完全的知的转移,研究自然及社会的发展法则在人类头脑中日益深刻和日益增多的反映,这就是唯物辩证法与认识论的一致。唯物辩证法研究客观世界最一般的发展法则,研究客观世界最发展的姿态在思维中的反映形态。这就是唯物辩证法研究现实事物的各过程及各现象的发生、发展,消灭及相互转化的法则,同时又研究反映客观世界发展法则的人类思维的形态,这就是唯物辩证法与论理学的一致。

要彻底了解辩证法、论理学、认识论三者为什么是一个东西,我们看下面唯物辩证法怎样解决关于论理的东西与历史的东西之相互关系这个问题,就可以明白了。

恩格斯说:“对于一切哲学家的思维方法来说,黑格尔思维方法的长处就在于横亘在根底面的极其丰富的历史感,他的形式虽说是抽象的唯心论的,然而他的思想的发展却常常是与世界历史的发展平行着的。并且历史原来就是思想的检证。”\mnote{8}“历史常常在飞跃地错杂地进行着。因为有这种情形,所以假若常常要依从历史的话,不但要注意许多不重要的材料,而且会不得不使思想行程中断。这时唯一适当的方法,就是论理的方法。然而这—论理的方法根本仍然是历史的方法,不过舍去了它那历史的形态与偶然性罢了”\mnote{9}。这种“论理发展与历史发展一致”的思想,是被马克思、恩格斯、列宁充分注意了的。“论理学的范畴,是外在的与活动之无数个别性的简约”\mnote{10}。“范畴就是分离的阶段,帮助我们去认识这一个网和网的结节点的”\mnote{11}。“人的实践活动,把人类的意识几十亿次反复不息地应用到各种各样的论理学式子里面,这样,这些式子就得到了所谓公理的意义了”\mnote{12}。“人类的实践,反复了几十亿次,才当做论理的式子固定在人类意识中。这些式子,都有着成见的永续性,因为是反复了几十亿次的结果,才有着公理的性质”\mnote{13}。上述列宁的那些话,指明唯物辩证法的论理学的特点,不象形式论理学那样,把它的法则和范畴看成空虚的,脱离内容而独立的,对于内容无关心的形式,也不象黑格尔那样,把他看成脱离物质世界而独立发展的观念要素,而是把它当做反映到和移植到我们头脑里,并且经由头脑加工制造过的,物质运动的表现去处理。黑格尔立脚在存在和思维的同一性上,把辩证法、论理学和认识论的同一性当做唯心论的同一性去处理。反之,马克思主义的哲学里,辩证法、论理学和认识论的同一性,是建立在唯物论基础上的。只有用唯物论解决存在与思维的关系问题,只有站在反映论的立场上,才能使辩证法、论理学和认识论的问题得到彻底的解决。

用辩证法唯物论去解决论理的东西和历史的东西的相互关系的最好的模范,首先要算马克思的《资本论》。《资本论》中包含了资本主义社会的历史发展,同时又包含了这一社会的论理发展。《资本论》所分析的,是那把资本主义社会的发生、发展及消灭反映出来的各经济范畴的发展的辩证法。这问题之解决的唯物论性质,在于他以物质的客观历史做基础,在于把概念和范畴当做这一现实历史的反映。资本主义的理论\mnote{14}和历史的一致,资本主义的社会的论理学和认识论的一致,模范地表现在《资本论》里面,我们可以从它懂得一点辩证法、论理学和认识论一致的门径。

以上是辩证法唯物论的对象问题。

\subsection{(五)物质论}

马克思主义继续和发展哲学中的唯物论路线,正确地解决了思维与存在的关系问题,即彻底唯物地指出世界的物质性,物质的客观实在性,和物质对于意识的根源性(或意识对于存在的依赖关系)。

承认物质对于意识的根源性是以世界的物质性及其客观存在为前提的。隶属于唯物论营垒的第一个条件就承认物质世界离人的意识而独立存在——人类出现以前它就存在,人类出现以后也是离开人的意识而独立存在的。承认这一点是一切科学研究的根本前提。

拿什么来证明这一点呢?证据是多得很的。人类时刻同外界接触;还须用残酷的手段去对付外界(自然界同社会)的压迫和反抗;还不但应该而且能够克服这些压迫和反抗——所有这些在人类社会的历史发展中表现出来的人类社会实践的实在情形,就是最好的证据。经过了万里长征的红军,不怀疑经过地区连同长江大河雪山草地以及和它作战的敌军等等的客观存在,也不怀疑红军自己的客观存在,中国人不怀疑侵略中国的日本帝国主义同中国人自己的客观存在,抗日军政大学的学生也不怀疑这个大学和学生自己的客观存在。这些东西都是客观地离开我们意识而独立存在的物质的东西,这是一切唯物论的基本观点,也就是哲学的物质观。

哲学的物质观同自然科学的物质观是不相同的。如果说哲学的物质观在于指出物质的客观存在,所谓物质就是说的离开人的意识而独立存在的整个世界(这个世界作用于人的感官,引起人的感觉,并在感觉中得到反映)。那末这种说法是永远不起变化的,是绝对的。自然科学的物质观则在于研究物质的构造,例如从前的原子论,后来的电子论等等,这些说法是随着自然科学的进步而变化的,是相对的。

根据辩证法唯物论的见地去区别哲学的物质观与自然科学的物质观,是彻底贯彻哲学的唯物论方向之必须条件。在向唯心论和机械唯物论作斗争方面,有着重要的意义。

唯心论者根据电子论的发见轰传物质消灭的谬说,他们不知道关于物质构造之科学知识的进步,正是证明辩证法唯物论的物质论之正确性。因为表现在旧的物质概念中的某些物质属性(重量硬度,不可入性,惰性等等),经过现代自然科学的发现,即电子论的发现,证明这些属性仅存在于某几种物质形态中,而在其它物质形态中则不存在,这种事实,破除了旧唯物论对于物质观念的片面性与狭隘性,而对于承认世界的物质性及其客观存在之辩证法唯物论的物质观,却恰好证明其正确。原来辩证法唯物论的物质观,正是以多样性去看物质的世界的统一,就是物质多样性的统一。这种物质观,对于物质由一形态转化到另一形态之永久普遍的运动变化这一种事实,丝毫也没有矛盾。以太、电子、原子、分子、结晶体、细胞、社会现象、思维现象——这些都是物质发展的种种阶段,是物质发展史中的种种暂时形态。科学研究的深入,各种物质形态的发现(物质多样性的发现),只是丰富了辩证法唯物论的物质观的内容,那里还会有什么矛盾?区别哲学的物质观同自然科学的物质观是必要的,因为二者有广狭之别然而是不相矛盾的,因为广义的物质包括了狭义的物质。

辩证法唯物论的物质观,不承认世界有所谓非物质的东西(独立的精神的东西)。物质是永久与普遍存在的,不论在时间与空间上都是无阻的,如果说世界上有一种“从来如此”与“到处如此”的东西(就其统一性而言),那就是哲学上的所谓客观存在的物质。用彻底的唯物论见地(即唯物论辩证法见地)来看意识这种东西,那末所谓意识不是别的,它是物质运动的一种形态,是人类物质头脑的一种特殊性质,是使意识以外的物质过程反映到意识之中来的那种物质头脑的特殊性质。由此可知,我们区别物质同意识并把二者对立起来是有条件的。就是说:只在认识论的见地有意义。因为意识或思维只是物质(头脑)的属性,所以认识与存在的对立就是认识的物质同被认识的物质的对立,不会多一点。这种主体同客体的对立,离开认识论领域就毫无意义。假如在认识论以外还把意识同物质对立起来,就无异于背叛唯物论。世界上只有物质同它的各种表现,主体自身也是物质的,所谓世界的物质性(物质是永久与普遍的),物质的客观实在性与物质对于意识的根源性,就是这个意思。一句话,物质是世界的一切。“一统归于司马懿”,我们说“一统归于物质”。这就是世界的统一原理。

以上是辩证法唯物论的物质论。

\subsection{(六)运动论(发展论)}

辩证法唯物论的第一个基本原则在于它的物质论,即承认世界的物质性、物质客观实在性和物质对于意识的根源性,这种世界的统一原理,在前面物质论中已经解决了。

辩证法唯物论的第二个基本原则在于它的运动论(或发展论),即承认运动是物质存在的形式,是物质内在的属性,是物质多样性的表现,这就是世界的发展原理。世界的发展原理同上述世界的统一原理相结合,就成为辩证法唯物论的整个的宇宙观。世界不是别的,就是无限发展的物质世界(或物质世界是无限发展的)。

辩证法唯物论的运动观,对于(一)离开物质而思考运动,(二)离开运动而思考物质,(三)物质运动的简单化,都是不能容许的,辩证法唯物论的运动论,就是同这些唯心的、形而上学的、及机械的观点作明确而坚决的斗争建立起来的。

辩证法唯物论的运动论,首先是同哲学的唯心论及宗教的神道主义相对立的。一切哲学的唯心论及宗教的神道主义的本质,在于它们从否认世界的物质统一性出发,设想世界的运动及发展是没有物质的、或在最初是没有物质的、而是精神作用或上帝神力的结果。德国唯心论哲学家黑格尔认为现在的世界是从所谓“世界理念”发展而来的,中国的周易哲学及宋、明理学都作出唯心论的宇宙发展观。基督教说上帝创造世界,佛教及中国一切拜物教都把宇宙万物的运动发展归之于神力。所有这些离物质而思考运动的说法都和辩证法唯物论根本不相容。不但唯心论与宗教,就是马克思以前的一切唯物论及现在一切反马克思主义的机械唯物论,当他们说到自然现象时,是唯物论的运动论者,但一说到社会现象时,就无不离开物质的原因而归着于精神的原因了。

辩证法唯物论坚决驳斥所有这些错误的运动观,指出他们的历史限制性——阶级地位的限制与科学发展程度的限制,而把自己的运动观建设在以无产阶级立场及最发达的科学水准为基础的、彻底的唯物论上面。辩证法唯物论首先指出运动是物质存在的形式、是物质内在的属性(不是由外力推动的),设想没有物质的运动,同设想没有运动的物质是一样不可思议的事。把唯物的运动观同唯心的及唯神的运动观尖锐地对立着。

离开运动而思考物质,则有形而上学的宇宙不动论或绝对均衡论,他们认为物质是永远不变的,在物质中没有发展这回事,认为绝对的静止是物质的一般状态或原始状态。辩证法唯物论坚决反对这种意见,认为运动是物质存在的最普遍的形式,是物质内在的不可分离的属性。一切的静止与均衡仅有相对的意义,而运动则是绝对的。辩证法唯物论承认一切物质形态均有相对的静止或均衡之可能,并认为这是辨别物质,因而亦即辨别生命的最重要条件(恩格斯)。但认为静止或均衡只是运动的要素之一,是运动的一种特殊情况。离开运动而考察物质的错误,就在于把这种静止要素或均衡要素夸张起来,把它掩蔽了并代替了全体,把运动的特殊情况一般化、绝对化起来。中国古代形而上学思想家爱说的一句话:“天不变,道亦不变”,就是这样的宇宙不动论。他们也承认宇宙及社会现象的变动,但否认其本质的变动,在他们看来,宇宙及社会的本质是永远不变动的。他们之所以如此,主要的原因在于他们的阶级限制性,封建地主阶级如果也承认宇宙及社会的本质是运动与发展的,就无异在理论上宣布他们自己阶级的死刑。一切的反动势力,他们的哲学都是不动论。革命的阶级同民众,却眼睛看到了世界的发展原理,因而主张改造这个社会及世界,他们的哲学是辩证法唯物论。

此外辩证法唯物论也不承认简单化的运动观,就是说把一切的运动都归结到一种形式上去,即归结到机械的运动,这是旧唯物论宇宙观的特点。旧唯物论(十七八世纪的法国唯物论,十九世纪的德国费尔巴哈唯物论)也承认物质的永久存在和永久运动(承认运动的无限性),但仍然没有跳出形而上学的宇宙观,不去说他们在社会论上的见解依然是唯心论的发展观。就在自然论上,也把物质世界的统一,归结到某种片面的属性,即归结到运动的一个形态——机械的运动,这种运动的原因在外力,象机械一样,由外力推之而运动。他们不从本质上,也不从内部原因上去说明物质或运动、本质或关系的一切多样性,而从单纯的外面的发现形式上从外力原因上去说明它,这样在实际上就失掉了世界的多样性。他们把世界一切的运动,都解作场所的移动与数量的增减。物质某一瞬间在某一场所,另一瞬间则在另一场所,这样就叫做运动。如果有变化,也只是数量增减的变化,没有性质的变化,变化是循环的,是反复产生同一结果的。辩证法唯物论与此相反,不把运动看作单纯的场所移动及循环运动,而把它看作无限的质的多样性,看作由一形态向他一形态的转化,世界物质的统一和物质的运动,便是世界物质无限多样性的统一与运动。恩格斯说:“运动的一切高级形态必然同力学的(外的或分子的)运动形态结合着,例如:如果没有热和电气的变化,化学的作用就不可能,如果没有力学的(分子的)热量的、电气的、化学的变化等等,有机的生命也不可能,这当然是不能否认的。然而如果只有某些低级运动形态的存在,是决不能包括各种状态中主要形态的物体的”。这话是千真万确地合于事实。即使就单纯机械运动而论,也不能从形而上学的观点去解释它。须知一切运动形态都是辩证法的,虽然它们之间的辩证法内容的深度与多面性有着很大的差异。机械运动仍然是辩证法的运动,所谓物体某一瞬间“在”某处,其实是同时“在”某处,同时又不在某处,所谓“在”某处,所谓“不动”,仅是运动的一种特殊情况,它根本上依然是在运动,物体在被限制着的时间内和被限制着的空间内运动着。物体总是不绝地克服这种限制性跑出这种一定的有限的时间及空间的界限以外去成为不绝的运动之流。而且机械运动只是物质的运动形态之一,在实在的现实世界中,没有它的绝对独立的存在,它总是联系于别种运动形态的。热、化学的反应,光、电气,一直到有机现象与社会现象,都是质地上特殊的物质运动形态。十九世纪与二十世纪交界时期的自然科学的划时代的大功劳,就在于发现了运动转化法则,指出物质的运动总是由一形态转化成为另一形态,这样的转化的新形态是与旧形态本质上不同的。物质所以转化的原因不在外部而在内部,不是由于外部机械力的推动,而是由于内部存在着性质不同的互相矛盾的两种因素相争相斗推动着物质的运动与发展。由于这个运动转化法则的发现,辩证法唯物论就能够把世界的物质统一原理扩大到自然与社会的历史上去,不但把世界当作永远运动的物质去考察,而且把世界当做由低级形态到高级形态的无限前进运动的物质去考察,即把世界当作发展,当作过程去考察,做一句话来说:“统一的物质世界是一个发展的过程”。这样就把旧唯物论的循环论击破了。辩证法唯物论深刻地多方面地观察了自然及社会的运动形态,认为当作全体看的世界之发展过程是永久的(无始无终的)。但同时各个历史地进行的具体的运动形态又是暂时的(有始有终的),就是说它是在一定的条件下发生,并在一定的条件下消灭的。认为世界的发展过程由低级的运动形态生出高级的运动形态,表示了它的历史性与暂时性,但同时任何一个运动形态无不是处在永久的长流中(无始无终的长流中)。依据着对立斗争的法则(自己运动的原因),使每一运动形态总是较之先行形态进到了高一级的阶段,它是向前直进的,但同时就各个运动形态来说(就各个具体的发展来说),却也会发生转向运动或后退运动,前进运动同后退运动相结合,在全体上就成为复杂的螺旋运动,认为新的运动形态是作为旧的运动形态的对立物(反对物)而发生的,但同时新的运动形态又必然保存着旧的运动形态中的许多要素,新东西是从旧东西里面生长出来的。认为事物的新形态、新性质、新属性的出现,是由连续性的中断即经过冲突和破局而飞跃地产生的,但同时事物的连结和相互关系又决不会绝对破坏。最后辩证法唯物论认为世界无穷尽(无限),不但就其全体来看是这样的,同时就其局部来看也是这样的,电子不是同原子分子一样表现着一个复杂而无穷尽的世界么?

物质运动的根本形态,又规定根本的自然科学与社会科学各科目。辩证法唯物论把世界的发展当作无机界经过有机界而达到最高物质运动形态(社会)的一个前进运动去考察,这一运动形态的从属关系就成了和它相应的科学(无机界科学,有机界科学,社会科学的从属关系的基础)。恩格斯说:“各种分类的科学是把特定的运动形态或相互关联相互推移的一联的运动形态拿来分析,因此科学的分类就在于要依从着运动的固有顺序去把各个运动分类排列起来,仅在这一点来说,分类才有意义。”\mnote{15}

整个世界包括人类社会在内,是采取质地不相同的各种形式的物质的运动,因此也就不能忘记物质运动的各种具体形式这个问题。所谓“物质一般”与“运动一般”是没有的,世界上只有各种不同形式的具体的物质或运动。“物质和运动这些字眼只是一些简写的名词,在这些名词中,我们依照它们的共同特性是把各种不同的被感觉的事物一概包括在内的。”\mnote{16}(恩格斯)

以上就是辩证法唯物论的世界运动论或世界发展原理。这个学说是马克思主义哲学的精髓,是无产阶级的宇宙观与方法论,无产阶级及一切革命的人们如果拿着这个彻底科学的武器,他们就能够理解这个世界并改造这个世界。

\subsection{(七)时空论}

运动是物质存在的形式,空间和时间也是物质存在的形式,运动的物质存在于空间和时间中,并且物质的运动本身是以空间和时间这两种物质存在的形式为前提的。空间和时间不能与物质相分离。“物质存在于空间”这句话,是从物质本身具有伸张性,物质世界是内部存在着伸张性的世界,不是说物质被放在一种非物质的空虚的空间中。空间和时间都不是独立的非物质的东西,也不是我们感觉性的主观形式。它们是客观物质世界存在的形式。它们是客观的,不存在物质以外,物质也不存在于它们以外。

把空间和时间看作物质存在的形式的这种见解,是彻底的唯物论的见解。这种时空观,同下列几种唯心论的时空观是根本相反的:(一)康德主义的时空观,认时间和空间不是客观的实在,而是人类的直觉形式;(二)黑格尔主义的时空观,认发展着的时间和空间的概念,日益接近于绝对观念;(三)马赫主义的时空观,认时间和空间是“感觉的种类”,“使经验和谐化的工具”。所有这些唯心论观点,都不承认时间和空间的客观实在性,都不承认时间和空间的概念在自身发展中反映着物质存在的形式。这些错误理论,都被辩证法唯物论一个一个地驳翻了。

辩证法唯物论在时空问题上,不但要同上述那些唯心论观点作斗争,而且要同机械唯物论作斗争。特别显著的是牛顿的(机械论),他把空间看做同时间无关系的不动的空架子,物质被安置到这种空架子里面去。辩证法唯物论反对这种机械论,指出我们的时空观念是在发展的。“世界上除了运动的物质以外便没有别的东西,而运动的物质若不在空间和时间中便无运动之可能。人类关于空间和时间的概念是相对的,但是这些相对的概念积集起来就成为绝对的真理。这些相对的概念不断发展着,循着绝对真理的路线而前进,日益走近于绝对真理。人类关于空间时间概念的变动性,始终不能推翻二者的客观实在性,这正和关于物质的运动形式及其组织之科学知识的变动性,不能推翻外界的客观实在性,是一样的。”\mnote{17}(列宁)

以上是辩证唯物的时空论。

\subsection{(八)意识论}

辩证唯物论认意识是物质的产物,是物质发展之一形式,是一定物质形态的特性。这种唯物主义同历史主义的意识论是和一切唯心论及机械唯物论对于这个问题的观点根本相反的。

依照马克思主义的见解,意识的来源,是由无意识的无机界发展到具有低级意识形态的动物界,再发展到具有高级意识形态的人类。高级意识形态不但同生理发展中的高级神经系统不可分离,而且同社会发展中的劳动生产不可分离。马克思、恩格斯曾经着重指出意识对物质生产发展的依赖关系,和意识同人类言语发展的关系。

所谓意识是一定物质形态的特性,这种物质形态就是组织复杂的神经系统,这样的神经系统只能发生于自然界进化的高级阶段上。整个无机界、植物界和低级的动物界,都没有认识在他们内面或外面发生着的那些过程的能力,它们是没有意识的。仅在有高级神经系统的动物体,才具有认识过程的能力,即具有自内反映或领悟这些过程的能力。吾人神经系统中的客观生理过程,是同它之内部取意识形式的主观表现相随而行的。凡就本身论是客观的东西,是某种物质过程,它对于具有头脑的实体却同时又是主观的心理的行为。

特殊思想实质的精神是没有的,有的只是思想的物质——脑子。这种思想的物质是有特别质地的物质,这种物质随着人类社会生活中言语的发展而达到高度的发展。这种物质具有思想这一种特殊性质,这是任何别的物质所不具备的。

然而庸俗唯物论者却认思想是脑子分泌出来的物质,这种见解歪曲了我们关于这个问题的观念。须知思想感情和意志的行为,不是具有重量和伸张性的东西,意识同重量伸张性等是同一物质之不同的性质。意识是运动的物质之内部状态,是反映着在运动的物质中所发生的生理过程的特殊性质。这种特殊性,同客观的神经作用过程不可分离,但又不与这过程相同,把这二者混同起来,推翻意识的特殊性,这就是庸俗唯物论的观点。

和这同样冒牌的马克思主义的机械论,附和心理学中某些资产阶级的左翼学派的见解,实质上也完全推翻了意识。他们把意识解作理化的生理的过程,认为高级实体的行为之研究,可以由客观生理学和生物学的研究去执行。他们不了解意识的本质之质的特殊性,看不到意识是人类社会实践的产物。他们把客体和主体之具体历史的一致,代之以主客的等同,代之以片面的机械的客观的世界。这种把意识混同于生理过程的观点,无异取消了思维与存在关系这个哲学中的根本的问题。

孟塞维克的唯心论企图用一种妥协理论去代替马克思主义的意识论,把唯物论同唯心论调和起来,他们拿客观主义同主观主义的原则,而这种原则既非机械的客观主义,也非唯心的主观主义,而是客观和主观之具体历史的一致。

可是还有怀疑论,这就是普列汉诺夫关于意识问题的物活论的见解。在他的“石子也是有意识的”一句名言中充分表现着。照他的意见,意识不是发生于物质发展过程中的,而是最初就存在于一切物质的。石子的、低级有机体的和人的意识之间,仅仅在于程度上的区别。这种反历史的见解,对于辩证唯物论认为意识是最后发生的具备着质的特殊性的见解,也是根本相反的。

只有辩证唯物论的意识论才是意识问题上的正确的理论。

\subsection{(九)反映论}

做一个彻底的唯物论者,单承认物质对于意识的根源性是不够的,还须承认意识对于物质的可认识性。

关于物质能否被认识的问题,是一个复杂的问题,是一切过去哲学都觉得无力对付的问题,只有辩证法唯物论能够给予正确的解决。在这个问题上,辩证法唯物论的立场既同不可知论相反,又同直率的实在论不同。

休谟同康德的不可知论,把认识的主体隔离开来,认为越出本体的界限是不可能的,“自在之物”和它的形象之间存在着不可跳过的深沟。

马赫主义的直率实在论,则把客体同感觉等同起来,认为真理在感觉中就已经成就了完成的形态。同时,他们不但不了解感觉是外界作用的结果,而且不了解主体在认识过程中的积极作用,即外界作用在主体的感觉机关和思想的脑子中所做的改造工夫(取印象和概念的形式表现出来)。

只有辩证法唯物论的反映论,肯定地答复了可认识性问题,成为马克思主义认识论的“灵魂”。根据这一理论,指明我们的印象和概念不但被客观事物所引起,而且还反映客观事物。指明印象和概念,既不象唯心论者所说的那样,是主体自动发展的产物,也不是不可知论者所说的那样,是客观事物的标符,而是客观事物的反映、照象和样本。

客观的真理是不依靠主体而独立存在的,它虽然反映在我们的感觉和概念中,但不是一下子就取完成的形态,而是一步一步完成的,认为客观真理在感觉中就已经取着完成形态,而被我们获得的那种直率实在论的见解是一种错误的见解。

客观真理在我们感觉和概念中虽不是一次就取完成的形态,然而不是不能认识的。辩证唯物论的反映论,反对不可知论的见解,认为意识是能够在认识过程中反映客观真理的。认识过程是一个复杂的过程,在这个过程中,当未被认识的“自在之物”,反映到我们的感觉印象、概念上来时,就变成“为我之物”了。感觉和思维,并不是如同康德所说的那样,把我们同外界隔离开来,而是把我们同外界联系起来的。感觉和思维就是客观外界的反映。思想的东西(印象和概念)并非别的,不过是“人类头脑中所转现出来和改造过来的物质的东西”(马克思)。在认识过程中,物质世界是愈走而愈接近地愈精确地愈多方面地和愈深刻地反映在我们的认识中。向着马赫主义和康德主义作两条战线的斗争,揭破直率实在论和不可知论的错误,是马克思主义认识论的任务。

唯物辩证法的反映论认为我们认识客观世界的能力是无限度的,这和不可知论者认为人的认识能力是有限度的那种见解根本相反。但我们之接近绝对真理,却每一次有其历史上的确定界限。列宁这样说:吾人知识之接近客观的绝对真理,是历史地有限度的。但是这一真理的存在是绝对的,我们不断地向真理接近也是绝对的。图画的外形是历史地有条件的,但这张图画描绘着客观上存在的模型则是绝对的,我们承认人的认识受历史条件的限制,真理是不能一次获得的。但我们不是不可知论者,我们又承认真理能够完成于人类认识的历史运动中。列宁还说:对于自然人类思想中的反映,不要死板板地或绝对地去了解他,认识不是无运动与无矛盾的,认识是处于永久的运动过程中,“即矛盾之发生和解决的永久的运动过程中”\mnote{18}。认识运动时一个复杂的充满着矛盾与斗争的运动,这就是辩证唯物论的认识论之见解。

一切哲学在认识论上的反历史的观点,都不把认识当作过程看待,因此都带着狭隘性。感觉主义的经验论之狭隘性,在感觉和概念之间挖开了深沟。理性主义学派的狭隘性,则使概念脱离了感觉。只有把认识当作过程看待的辩证唯物论的认识论(反映论)才彻底除去了这样狭隘性,把认识放在唯物的与辩证的地位。

反映论指出:反映过程不限于感觉和印象,也存在于思维中(抽象的概念中),认识是一个由感觉到思维的运动过程。列宁曾说:“反映自然的认识,不是简单的,直接的整体的反映,而是许多抽象的思考、概念、法则等等之形成过程”\mnote{19}。

同时列宁还指出:由感觉到思维的认识过程,是飞跃式地进行的,在这一点上,列宁精确地阐明了:认识中的经验元素和理性元素相互关系之辩证唯物论的见解。许多哲学家都不了解认识的运动过程中,即从感觉到思维(从印象到概念)的运动过程中所发生的突变。因此理解这一由矛盾而产生的飞跃式的转变,即理解感觉和思维的一致为辩证的一致,便是理解了列宁反映论的本质之最重要的元素。

\subsection{(十)真理论}

真理是客观的,相对的,又是绝对的。这就是唯物辩证法的真理观。

真理首先是客观的。在承认了物质的客观实在性及物质对于意识的根源性之后,就等于承认了真理的客观性。所谓客观真理,就是说:客观存在的物质世界,是我们的知识或概念的内容之唯一来源,再也没有别的来源;只有唯心论者否认物质世界离人的意识而独立存在——这一唯物论的基本原则,才主张知识或概念是主观自主的,不要任何客观的内容,因而承认主观真理,否认客观真理。然而这是不合事实的。任何一种知识或一个概念,如果它不是反映客观世界的规律性,它就不是科学的知识,不是客观真理,而是主观地自欺欺人的迷信或妄想。人类以改变环境为目的之一切实际行动,不管是生产行动也罢,阶级斗争或民族斗争的行动也罢,其他任何一种行动也罢,都是受着思想(知识)的指挥的。这种思想如果不适合于客观的规律性,即客观规律性没有反映到行动的人的脑子里去,没有构成他的思想或知识的内容,那末这种行动是一定不能达到目的的。革命运动中所谓主观指导犯错误,就是指的这种情形。马克思主义所以成为革命的科学知识,就是因为它正确地反映了客观世界的实际规律,它是客观的真理。一切反马克思主义的思想所以都是错的东西,就是因为它们不根据于正确的客观规律,完全是主观的妄想。有人说,一般公认的就是客观真理(主观唯心论者波格达诺夫\mnote{20}这样说)。照这种意见,那末,宗教和偏见也是客观真理了,因为宗教和偏见虽然实质上是谬见,可是却常常为多数人所公认;有时正确的科学思想反不及这些谬见的普及。唯物辩证法根本反对这种意见,认为只有正确地反映客观规律性的科学知识,才能被称为真理,一切真理必须是客观的。真理与谬说是绝对对立的,判断一切知识是否为真理,唯一的看他们是否反映客观的规律。如果不合乎客观规律,尽管是一般人都承认的,或革命运动中某些说得天花乱坠的理论,都只能把它当作谬说看待。

唯物辩证法真理论的第一个问题,是主观真理和客观真理的问题,它的答复是否认前者而承认后者。唯物辩证法真理论的第二个问题,是绝对真理和相对真理的问题,它的答复不是片面地承认或否认某一方面,而是同时承认它们,并指出它们正确的相互关系,即指出它们的辩证性。

唯物辩证法在承认客观真理时,就是承认了绝对真理的。因为当我们说知识的内容是客观世界的反映时,这就等于承认了我们知识的对象是那个永久的绝对的世界。“关于自然之一切真理的认识,就是永久的无穷的认识,因此它实质上是绝对的”\mnote{21}(恩格斯)。然而客观的绝对的真理不是一下子全部成为我们的知识,而是在我们认识之无穷的发展过程中,经过无数相对真理的介绍,而到达于绝对的真理。这无数相对真理之总和,就是绝对真理的表现。人类的思维,就它的本性说,能给我们以绝对真理,绝对真理乃由许多相对真理积集而成,科学发展的每一阶段,增加新的种子到这个绝对真理的总和中去。但是每一科学原理的真理界限却总是相对的。绝对真理仅能表现在无数相对真理之上,如果不经过相对真理的表现,绝对真理就无从认识。唯物辩证法不否认一切知识之相对性,但这只是指吾人知识接近于客观绝对真理的限度之历史条件性而言,而不是说知识本身只是相对的。一切科学上的发明,都是历史地有限度的和相对的,但是科学知识跟谬说不同,它显示着描画着客观的绝对的真理,这就是绝对真理与相对真理相互关系之辩证法的见解。

有两种见解:一种是形而上学的唯物论;另一种是唯心论的相对论。对于绝对真理与相对真理之相互关系问题都是不正确的。

形而上学的唯物论者,根据于他们的“物质世界无变化”的形而上学的基本原则,认为人类思维也是不变化的,即认为在人的意识中这一不变的客观世界,是一下子整个被摄取了。这就是说他们承认绝对真理,而这个绝对真理是一次被人获得的,他们把真理看成不动的,死的,不发展的东西。他们的错误不在于他们承认有绝对真理——承认这一点是正确的,而在于他们不了解真理的历史性,不把真理的获得看作一个认识的过程。不了解所谓绝对真理者,只能在人类认识的发展过程中一步一步地开发出来,而每一步向前的认识,都表现着绝对真理的内容,但对于全部真理说来,它具有相对的意义,并不能一下子获得绝对真理的全部。形而上学的唯物论关于真理的见解,表现了认识论一个极端。

认识论中关于真理问题的再一个极端,就是唯心论的相对论。他们否认知识之绝对真理,只承认它的相对意义。他们认为一切科学的发明,都不包含绝对真理,因而也不是客观真理,真理只是主观的与相对的。既然这样,那末一切谬说就都有存在的权利了,帝国主义侵略弱小民族,统治阶级剥削劳动群众,这些侵略主义与剥削制度也就是真理,因为真理横直只是主观的与相对的。否认客观真理与绝对真理的结果,必然到达这样的结论。并且唯心论的相对论,他们的目的本来就是要替统治阶级作辩护的,例如相对论的实用主义(或实验主义)之目的就在于此。

这样看来,不论是形而上学的唯物论,或是唯心论的相对论,都不能正确解决绝对真理和相对真理的相互关系的问题。只有唯物论辩证法,既给思维与存在相互关系问题以正确的解答,并且随之而来又确定了科学知识的客观性,再则,还同时给了绝对与相对真理以正确的理解。这就是唯物辩证法的真理论。

\subsection{(十一)实践论(认识与实践的关系,理论与实际的关系,知与行的关系。)}

马克思以前的唯物论,离开人的社会性,离开人的历史发展,去观察认识问题,因此不能了解认识对社会实践的依赖关系,即认识对生产与阶级斗争的依赖关系。

首先,马克思主义者认为人类的生产活动是最基本的实践活动,是决定其他一切活动的东西。人的认识,主要的依赖物质的生产活动,逐渐了解自然的现象、自然的性质(自然的规律性)、人与自然的关系;而且经过生产活动,同时也认识了人与人的相互关系。一切这些知识,离开生产活动是不能得到的。每个人以社会一员的资格,与其他社会成员协力从事生产活动,以解决人类物质生活问题,这是人的认识发展的基本来源。

人的社会实践,不限于生产活动一种形式,还有多种其他的形式,阶级斗争,政治生活,科学活动,总之,社会实际生活的一切领域都是社会的人所参加的。因此,人的认识,在物质生活以外,还从政治文化生活中(与物质生活密切联系)了解了人与人的各种复杂的关系。其中尤以各种形式的阶级斗争,给予人的认识发展以深刻的影响。在阶级社会中,各种思想无不打上阶级的烙印,就是这个原故。

因此,马克思主义者认为只有人们的社会实践,提给人们对于外界认识之真理性的标准。实际的情形是这样的,只有在社会实践过程中(物质生产过程中、阶级斗争过程中、科学实验过程中),人们达到了思想中所预想的结果时,人们的认识才会发生力量。农民如果得不到收获,工人如果做不成器物,罢工斗争,军队作战,民族革命,如果也都得不到胜利,那末这是为什么呢?这是因为人们的认识没有外界的过程的实况去反映这些过程的规律性。因而在他们的实践活动中不能达到预想的结果。人们要想得到胜利(即得到预想的结果),一定要自己的思想合于客观外界的规律性。如果不合,就会在实践中失败,人们经过失败之后,也就从失败取得教训,改正自己的思想使之适合于外界的规律性,人们就能变失败为胜利,所谓“失败者成功之母”,“吃一堑长一智”,就是这个道理。辩证唯物论的认识论把实践提到第一的地位,认为人的认识一点也不能离开实践,排斥一切否认实践重要性、使认识离开实践的错误理论。列宁这样说过:“实践高于(理论的)认识,因为它不但有一般性的价值,而且还有直接现实性的价值”\mnote{22}。马克思主义的哲学辩证唯物论的最显著的特点有两个:一个是它的阶级性,公然申明辩证唯物论是为无产阶级服务的;再一个是它的实践性,强调理论对于实践的依赖关系,理论来源于实践,又转过来为实践服务。判定认识或理论之是否真理,不是依主观上觉得如何而定,而是依客观上社会实践的结果如何而定。真理的标准只能是社会的实践。实践的观点是辩证唯物论的认识论之第一的与基本的观点。

然而人的认识究竟怎样从实践发生,而又服务于实践呢?这只要看一看认识的发展过程就会明了的。

原来人在实践过程中,开始只是看到过程中各个事物的现象方面,看到各个事物的片面,看到各个事物之间的外部联系。例如国民党考察团到延安的头一二天,看到了延安的地形、街道、屋宇,接触了许多的人,参加了宴会、晚会与群众大会,听到了各种说话,看到了各种文件,这些就是事物的现象,事物的各个片面以及这些事物的外部联系。这叫做认识的感性阶段,就是感觉与印象的阶段。也就是延安这些各别的事物作用于考察团先生们的感官,引起了他们的感觉,在他们的脑子中生起了许多的印象,以及这些印象间的大概的外部的联系,这是认识的第一个阶段。在这个阶段中人们还不能造成深刻的概念,作出理论的结论。

社会实践的继续,使人们在实践中引起感觉与印象的东西反复了多次,于是在人们的脑子里生起了一个认识过程中的突变,产生了概念。概念这种东西已经不是事物的现象,不是事物的各个片面,不是它们外部的联系,而是抓着了事物的本质,事物的全体,事物的内部联系了。概念同感觉,不但是数量上的差别,而且有了性质上的差别。循此继进,使用判断与推理的方法,就可生出理论的结论来。《三国演义》上所谓“眉头一皱计上心来”,我们普通说话所谓“让我想一想”,就是人在脑子中运用概念以作判断与推理的工夫。这是认识的第二个阶段,或叫论理阶段,是认识的第二个阶段。考察团先生们在他们集合了各种材料,加上他们“想了一想”之后,他们就能够作出“共产党抗日民族抗一战线与国共合作的政策是彻底的、诚恳的与真实的”这样一个判断了。在他们作出这个判断之后,如果他们对于团结救国也是真实的话,那末他们就能够进一步作出这样的结论:“国共合作是能够成功的”。这个概念、判断与推理的阶段,在人对于一个事物的整个认识过程中是最重要的一个阶段。认识之真正任务不在感性的认识,而在理性的认识。认识之真正任务在于经过感觉而达到于思维,到达于了解客观事物的内部矛盾,了解它的规律性,了解这一过程与那一过程间的内部联系,即到达于理论的认识。再重复地说,理性的认识所以和感性的认识不同,是因为感性的认识是属于事物之片面的、现象的、外部联系的东西,理性的认识则推进了一大步,到达了事物之全体的、本质的、内部联系的东西,到达了暴露周围世界之内的矛盾,因而能在周围世界之总体上,在周围世界一切方面之内部联系上,去把握周围世界的发展。

这种基于实践之由浅入深的唯物辩证法的认识发展过程的理论,在马克思主义以前,是没有一个人这样解决过的。马克思主义的辩证唯物论,第一次正确地解决了这个问题,唯物地而且辩证地指出了认识之深化的运动,指出了社会的人在他们的生产与阶级斗争之复杂的、经常反复的实践中,由感性认识到理性认识之推移的运动。列宁说过:“物质的抽象,自然的法则,价值的抽象及其他等等,即一切科学的(正确的、重要的、非瞎说的)抽象,都比较深刻、比较正确、比较完全地反映自然。”\mnote{23}列宁又曾这样指出:认识过程中两个阶段的特性,在低级阶段,认识表现为感性的,在高级阶段,认识表现为理性的,但任何阶段,都是统一的认识过程中的阶段。感性与理性二者的性质不同,但又不是互相分离的,它们在实践的基础上统一起来了。我们的实践证明:感觉到了的东西,我们不能立刻理解它,只有理解了的东西才更深刻地感觉它。感觉只解决现象问题,理解才解决本质问题。这些问题的解决,一点也不能离开实践。无论何人要认识什么事物,除了同那个事物接触,即生活于(实践于)那个事物的环境中,是没有法子解决的。不能在封建社会就预先认识资本主义社会的规律,因为资本主义还未出现,还无这种实践。马克思主义只能是资本主义社会的产物。不能在自由资本主义时代就预先认识帝国主义时代的某些特异的规律,因为帝国主义还未出现,还无这种实践,只有列宁和斯大林才能担当此项任务。马克思与列宁也不能在经济落后的殖民地产生,这是因为虽然同时但不同地。马克思、恩格斯、列宁之所以能够作出他们的理论,除了他们的天才条件之外,主要地是他们亲身参加了当时的阶级斗争与科学实验的实践,没有这后一个条件,任何天才也是不能成功的。“秀者不出门,全知天下事”,在技术不发达的古代只是一句空话,在技术发达的现代虽然可以实现这句话,然而真正亲知的是天下实践的人,那些人在他们实践中间取得了“知”,经过文字与技术的传达而到达于“秀才”之手,秀才乃能间接地“知天下事”。如果要直接地认识某种或某些事物,便只有亲身参加于变革现实、变革某种或某些事物的实践中,才能触到那种或那些事物的现象,也只有在亲身参加变革现实的实践中,才能暴露那种或那些事物的本质而理解它。这是任何人实际上走着的认识路程,不过有些人故意歪曲地说些反对的话罢了。世上最可笑的是那些“知识份子”,有了道听途说的一知半解,便自封为“天下第一”,多见其不自量而已。知识的问题是一个科学问题,来不得半点虚伪与骄傲,决定地需要的到是他的反面——诚实与谦逊的态度。你要有知识,你就得参加变革现实的实践。你要知道梨子的滋味,你就得变革梨子,亲口吃一吃。你要知道原子的组织同性质,你就得实行化学家的实验,变革原子的情况。你要知道革命的具体理论与方法,你就得参加革命。一切真知都是从直接经验发源来的。但人不能事事直接经验,事实上多数的知识都是间接经验的东西,这就是一切古代的与外域的知识。这些知识在古人在外人是直接经验的东西,如果在古人外人直接经验时是附合于列宁所说的条件:“科学的(正确的、重要的、非瞎说的)抽象”,那末它们是可靠的,否则便是不可靠。所以一个人的知识,不外直接经验与间接经验的两部分。而且在我为间接经验者,在人则仍属直接经验。因此,就知识的总体说来,无论何种知识都是不能离开直接经验的。任何知识的来源,在于人的肉体感官对客观外界的感觉,否认了这个感觉,否认了直接经验,否认了亲身参加变革现实的实践,他就不是唯物论者。“知识份子”之所以可笑,原因就在这个地方。中国商人有一句话:“要赚畜生钱,要跟畜生眠”。这句话对于商人赚钱是真理,对于认识论也是真理,离开实践的认识是不可能的。

为明了基于变革现实的实践而产生的唯物辩证法的认识运动——认识之逐渐深化的运动,下面再举出几个具体的例子。

无产阶级对于资本主义过程的认识,在其实践的初期——破坏机器与自发斗争时期,他们还只在感性认识的阶段,只认识资本主义个别现象的片面及其外部的联系。这时,他们还是一个所谓“自在的阶级”。但到了他们实践的后期——有意识有组织的阶级斗争与政治斗争的时期,由于实践,由于长期斗争的经验,教训了他们,他们就理解了资本主义社会的本质,理解了社会阶级的剥削关系,产生了马克思主义的理论,这时他们就造成了一个“自为的阶级”。

中国人民对于帝国主义的认识也是这样。第一阶段是表面的感性的认识,表现在太平天国运动与义和团运动等笼统的排外主义的斗争上。第二阶段才进到理性的认识,看出了帝国主义内部与外部的各种矛盾,并看出了帝国主义联合中国封建阶级以压榨中国人民大众的实质,这种认识是从五四运动前后才开始的。

我们再来看战争。战争的领导者,如果他们是一些没有战争经验的人,对于一个具体的战争(例如我们过去十年的苏稚埃战争)的深刻的指导规律,在开始阶段是不了解的。他们在开始阶段只是身历了许多作战的经验,而且败仗是很多的。然而由于这些经验(胜仗,特别是败仗的经验),使他们能够理解贯串整个战争的内部的东西,即那个具体的战争之规律性,懂得了战略与战术,因而能够有把握地去指导战争。此时,如果改换一个无经验的人去指导,又会要在吃了一些败仗之后(有了经验之后),才能理会战争的正确的规律。

常常听到一些同志在不能勇敢接受工作任务时说出来的一句话,就是说:他没有把握。为什么没有把握呢?因为他对这项工作的内容与环境没有规律性的了解,或者他从来就没有接触过这类工作,或者接触得不多,因而无从说到了解这类工作的规律性。及至把工作的情况同环境给以详细分析之后,他就觉得比较有了把握,愿意去做这项工作。如果这个人在这项工作中经过了一个时期(他有了这项工作的经验),而他又是一个肯虚心体察客观情况的人,不是一个主观地、片面地、表面地看问题的人,他就能够自己做出应该怎样进行工作的结论,他的工作勇气也就可以大大地提高。只有那些主观地、片面地与表面地看问题的人,跑到一个地方,不问环境的情况,不看事情的全体(事情的历史与全部现状),也不触到事情的本质(事情的性质及此一事情与其他事情的内部联系),就“自以为是”地发号施令起来,这样的人是没有不跌交子的。

由此看来,认识的过程,第一步是开始接触外界事情,属于感觉的阶段。第二步是综合感觉的材料加以改造和整顿,属于概念、判断、与推理的阶段。只有感觉的材料十分丰富(不是零碎不全)与合于实际(不是错觉),才能根据这样的材料造出正确的概念与理论来。

这里有两个要点须着重指明:第一个,在前面已经说过的,这里再重复说一说,就是理性认识依赖于感性认识的问题。如果以为理性认识可以不从感性认识得来,他就是一个唯心论者。哲学史上有所谓“唯理论”一派,就是只承认理性的实在性,不承认经验的实在性,以为只有理性靠得住,而感觉的经验是靠不住的。这一派的错误在于颠倒了事实。理性的东西所以靠得住,正由于它来源于感性,否则理性的东西就成了无源之水,无本之木,而只是主观自生的靠不住的东西了,从认识过程的秩序说来,感觉经验是第一的东西,我们强调社会实践在认识过程中的意义,就在于只有社会实践才能使人的认识开始发生,开始从客观外界得到感觉经验。一个闭目塞听、同客观外界根本绝缘的人,是无所谓认识的。认识发源于经验——这就是认识论的唯物论。

第二是认识有待于深化,有待于发展到理性阶段——这就是认识论的辩证法。如果以为认识可以停顿在低级的感性阶段,以为只有感性认识可靠,而理性认识是靠不住的,这便重复了历史上“经验论”的理论。这种理论的错误,在于不知道感觉材料固然是客观外界某些真实性的反映(不去说“经验只是内省体验的那种唯心的经验论”),但它们仅是片面的与表面的东西,这种反映是不完全的,是没有反映事物本质的。要完全地反映整个的事物,反映事物的本质,反映其内部联系规律性,就非经过思考作用,将丰富的感觉材料加以去粗取精、去伪存真、由此及彼、由表及里的改造制作工夫,造成概念及理论的系统不可,非从感性认识,改变到理性认识不可。这种改造过的认识,不是更空虚更不可靠了的认识,相反地,只要是在认识过程中根据于实践基础而科学地改造过的东西,正如列宁所说:它是更深刻、更正确、更完全地反映客观事物的东西。\mnote{24}庸俗的事物主义家不是这样,他们尊重经验而看轻理论,因而不能通观客观过程的全体,缺乏明确的方针,没有远大的前途,沾沾自喜于一得之功与一孔之见。这种人如果指导革命,就会引导革命走上碰壁的地步。

理性认识依赖于感性认识,感性认识有待于发展到理性认识,这就是唯物辩证法的认识论。哲学上的认识论与经验论,都不懂得认识的历史性或辩证性,虽然各有片面的真理(对于唯物的唯理论与经验论而言,非指唯心的唯理论与经验论),但在认识论的全体上则都是错误的。由感性到理性之唯物辩证法的认识运动,对于一个小的认识过程(例如一个事物或一件工作)是如此,对于一个大的认识过程(例如一个社会或一个革命)也是如此。

然而认识运动至此还没有完结。唯物辩证法的认识运动,如果只到理性认识为止,那么还只说到问题的一半。而且对于马克思主义的哲学说来,还只说到非十分重要的那一半。马克思主义哲学认为十分重要的问题,不在于懂得了客观世界的规律性,因而能够解释宇宙,而在于拿了这种对于客观规律性的认识去改造宇宙。在马克思主义看来,理论是重要的,它的重要性充分地表现在列宁说过的一句话:“没有革命的理论,就没有革命的运动”\mnote{25}。人的一切行动(实践)都是受人的思想指导的,没有思想,当然就没有任何的行动。然而马克思主义看重理论,正是,也仅仅是,因为它能够指导行动。如果有了正确的理论,只在把它空谈一会,束之高阁,并不实行,那么这种理论再好也是没有用的。认识从实践始,经过实践得到了理论的认识,还须再回到实践去。认识的能动作用,不但表现于从感性的认识到理性的认识之能动的飞跃,更重要的还须表现于从理性的认识到革命的实践这一个飞跃。抓住了世界现实规律性的认识,必须把它再用到改造世界的实践中去,再用到生产的实践、革命的阶级斗争与民族斗争的实践以及科学实验的实践中去。这就是检验理论与发展理论的过程,是整个认识过程的继续。理论的东西或理性的认识之是否符合于客观真理性这个问题,在前面说的由感性到理性之认识运动中是没有完全解决的,也不能完全解决的。要完全地解决此问题,只有把理性的认识再回到社会实践中去,应用理论于实际,看它是否能够达到预想的目的。许多自然科学理论之所以被认为真理,不但在于发现此学说时,而且在于为尔后的科学实践所证实。马克思主义之所以被称为真理,也不但在于马克思等人科学地构成此学说时,而且在于为尔后革命的阶级斗争与民族斗争的实践所证实。辩证唯物论之是否为真理,在于经过无论什么人的实践都不能逃出它的范围。认识史的实践告诉我们,许多理论的真理性是不完全的,经过实践的检验而纠正了它们的不完全性。许多理论是错误的,经过实践的检验而纠正其错误。所谓“实践是真理的标准”,所谓“实践是认识论第一与基本的观点”,理由就在这个地方。斯大林说的好:“离开实践的理论,是空洞的理论,离开理论的实践,是盲目的实践”\mnote{26}。

说到这里,认识运动就完成了吗?我们的答复是完成了,又没有完成。社会的人投身于变革在某一一定发展阶段内之某一一定客观过程的实践中(不论是关于变革某一自然过程的实践,或变革某一社会过程的实践),由于客观过程的反映与主现能动性的作用,使得人的认识由感性的推移到了理性的,造成了大体上相应于该客观过程之法则性的理论、思想、计划、或方案,然后再应用这种理论、思想、计划或方案于该同一客观过程的实践,如果能够实现预想的目的,即将预定的理论、思想、计划、方案于该同一过程的实践中变为事实,或大体上变为事实,那末,对于这一具体过程的认识运动算是完成了。例如,在变革自然的过程中,某一工程计划的实现,某一科学假想的证实,某一器物的制成,某一农产的收获,在变革社会过程中,某一罢工的胜利,某一战争的胜利,某一教育计划的实现,某一救国团体的成立,都算实现了预想的目的。然而一般说来,不论在变革自然或变革社会的实践中,人们原定的理论、思想、计划、方案,毫无改变地实现出来之事,是很少的。这是因为从事变革现实的人们,常常受着许多的限制,不但常常受着科学条件与技术条件的限制,而且也受着客观过程表现程度的限制(客观过程的方面及本质尚未充分暴露)。在这种情形之下,由于实践中发现前所未料的情况,因而部分地改变理论、思想、计划、方案的事是常有的,全部地改变的事也是有的。即是说原定的理论、思想、计划、方案,部分或全部不合于实际,部分错了或全部错了的事,都是有的。许多时候须反复失败过多次,才能纠正错误的认识,才能到达于同客观过程的规律性相符合,因而才能够变主观的东西为客观的东西(即在实践中得与预想结果之正确的认识)。但不管怎样,到了这种时候,人们对于在某一一定发展阶段内之某一一定客观过程的认识运动,算是完成了。

然而对于过程之推移而言,人的认识运动是没有完成的。任何过程,不论是属于自然界的与属于社会的,由于内部的矛盾与斗争,都是向前推移向前发展的,人的认识运动也应跟着推移与发展。依社会运动来说,所贵于革命的指导者,不但在于当自己的理论、思想、计划、方案有错误时须得善于加以改正,如同上面已经说到的,而且在于当某一一定的客观过程已经从某一一定的发展阶段向另一一定的发展阶段推移转变的时候,须得善于使自己及参加革命的人员在主观认识上也跟着推移转变,即是要使新的革命任务与新的工作方案的提出,适合于新的情况的变化。革命时期情况的变化是很急速的,如果革命党人的认识不能随之而急速变化,就不能引导革命走向胜利。然而思想落后于实际的事是常有的,这是因为人的认识受了许多限制的原故。许多人受了阶级条件的限制(反动的剥削阶级,他们已无认识任何真理的能力,因而也没有改造宇宙的能力,相反地,他们变成了阻碍认识真理与改造世界的敌人),有些人受了劳动分工的限制(劳心、劳力的分工,各业之间的分工),有些人受了原来的错误思想的限制(唯心论与机械论等多属于剥削份子;但也有被剥削份子,由于剥削份子的教育而来),而一般的原因则在受限制于技术水平与科学水平的历史条件。无产阶级及其政党,应该利用自己天然优胜的阶级条件(这是任何别的阶级所没有的),利用新的技术与科学,利用马克思主义的世界观与方法论,紧密地依靠革命实践的基础,使自己的认识跟着客观情况的变化而变化,使理论的东西随历史的东西,平行并进,达到完满地改造世界的目的。

我们反对革命队伍中的顽固派,他们的思想不能随变化了的客观情况而前进,在历史上表现为右倾机会主义。中国1927年的陈独秀主义,苏联的布哈林主义,都属于这一类。这些人看不出矛盾的斗争已将客观过程推向前进了,而他们的认识仍然停止在旧阶段。一切顽固派的思想都有这样的特征。他们的思想离开了社会的实践,他们不能站在社会车轮的前头充任向导的工作,他们只知跟在车轮后面怨恨车轮走的太快了,企图把它向后拉,开倒车。

我们也反对“左”翼清谈主义。中国1930年的李立三主义,苏联在尚可作为一个共产主义派别看待时的托洛斯基主义(现在则已成最反动的派别),以及世界各国的超左思想,都属于这一类。他们的思想超过客观过程的一定发展阶段,有些把幻想看作真理,有些则把仅在将来有现实可能性的理想,强迫放在现时来做,离开了当前大多数人的实践,离开了当前的现实性,行动上表现为冒险主义。

唯心论与机械论,机会主义与冒险主义,都没有唯物辩证的认识论的根据,他们都是以主观同客观相分裂,以认识与实践相舍离为特征的。以科学的社会实践为特征的马克思主义的认识论,不能不坚决反对这些错误思想。马克思主义者承认,在绝对的总的宇宙发展过程中,各个具体过程的发展都是相对的,因而人的认识也在绝对的真理中对于在各个一定发展阶段上的具体过程之认识只有相对的真理。客观过程的发展是充满着矛盾与斗争的发展,人的认识运动也是充满着矛盾与斗争的发展。一切客观世界的辩证法的运动,都或先或后地能够反映到认识中来。实践中之发展与消灭的过程是无穷的,人的认识之发生、发展与消灭的过程也是无穷。根据于一定的理论、思想、计划、方案以从事于变革客观现实的实践,一次又一次地向前,人对客观现实的认识也就一次又一次地深化。客观现实世界的变化运动永远没有完结,人在实践中对真理的认识也永远没有完结。马克思主义没有结束真理,而是在实践中不断地开辟认识真理的道路。我们的结论是主观与客观、理论与实践、知与行的具体历史的统一,反对一切离开具体历史的“左”的或“右”的错误思想。

大宇宙中自然发展与社会发展到了今日的时代,正确地认识宇宙与改造宇宙的责任,已经历史地落在无产阶级及其政党的肩上。这种根据科学认识而定下来的改造世界的实践过程,在世界、在中国均已到达了一个历史的时节——自有历史以来未曾有过的重大时节,这就是整个儿地推翻世界与中国的黑暗面,把它转变过来成为前所未有的光明世界。无产阶级及革命人民改造世界的斗争,包括实现下述的任务:改造客观世界,也改造自己的主观世界——改造自己的认识能力,改造主观世界同客观世界的关系。地球上已经有一部分实行了这种改造,这就是苏联。他们还正在为自己为世界推进这种改造过程。中国人民与世界人民也都正开始或将要通过这样的改造过程。所谓被改造的客观世界,其中包括了一切反对改造的人们,他们的被改造,须通过强迫的阶段,然后才能进入自觉的阶段。世界到了全人类都自觉地改进自己与改造世界的时候,那就是世界的共产主义时代。

通过实践而发现真理,又通过实践而证实真理与发展真理。从感性认识而能动地发展到理性认识,又从理性认识而能动地指导革命实践,改造主观世界与客观世界。实践、认识、再实践、再认识的形式,循环发展以至无穷,而实践与认识之每一循环的内容,都比较地进到高一级的程度——这就是唯物辩证法的全部认识论,这就是唯物辩证法的知行统一观。

\section{第三章 唯物辩证法}

前面简述了“唯心论与唯物论”及“辩证法唯物论”两个问题。关于辩证法问题,仅有概略的提到,现在来系统地讲这个问题。

马克思主义的世界观(或叫宇宙观),是辩证法唯物论,不是形而上学的唯物论(或叫机械的唯物论)。这一点区别,是一个天翻地复\mnote{27}的大问题。世界是一个什么样子的?从古至今有三种主要的答案:第一种是唯心论(不管是形而上学的唯心论,或辩证法的唯心论),说世界是心造的,引申起来又可说是神造的。第二种是机械唯物论,否认世界是心的世界,说世界是物质的世界,但物质是不发展的,不变化的。第三种是马克思主义的答案,推翻了前面两种,说世界不是心造的,也不是不发展的物质,而是发展的物质世界,这就是辩证法唯物论。马克思主义这样地看世界,把世界在从来人们眼睛中的样子翻转了过来,这不是天翻地复的大议论吗?世界是发展的物质世界,这种议论,在西洋古代的希腊就有人说过了,不过因为时代的限制,还只简单地笼统地说了一说,叫做朴素的唯物论。没有(也不可能有)科学的基础,然而议论是基本上正确的。黑格尔创造了辩证的唯心论,说世界是发展的,但是心造的,他是唯心发展论,其正确是发展论(即辩证论),其错误是唯心发展论。西洋十七、十八、十九三个世纪,法德等国的资产阶级唯物论,则是机械观的唯物论。他们说世界是物质世界,这是对的,说是象机械一样的运动,只有增减或位置的变化,没有性质上的变化,这是不对的。马克思继承了希腊朴素的辩证唯物论,改造了机械唯物论与辩证唯心论,造成了从古以来没有过的、放在科学基础之上的辩证唯物论,成为全世界无产阶级及一切被压迫人民的革命的武器。

唯物辩证法是马克思主义的科学方法论,是认识的方法,是论理的方法,然而它就是世界观。世界本来是发展的物质世界,这是世界观。拿了这样的世界观转过来去看世界,去研究世界上的问题,去想世界上的问题,去解决世界上的问题,去指导革命,去做工作,去从事生产,去指挥作战,去议论人家长短,这就是方法论。此外并没有别的什么单独的方法论。所以在马克思主义者手里,世界观同方法论是一个东西,辩证法、认识论、论理学,也是一个东西。

我们要系统地来讲唯物辩证法,就要讲到唯物辩证法的许多问题,这就是它的许多范畴、许多规律、许多法则(这几个名词是一个意思)。

唯物辩证法究竟有些什么法则呢?这些法则中那些是根本法则,那些是附从于根本法则而又为唯物辩证法学说中不可缺少不可不解决的方面、侧面或问题呢?所有这些法则,为什么不是主观自造的,而是客观世界本来的法则呢?对于这些法则的学习、了解,是为了什么呢?

这个完整的革命的唯物辩证法学说,创造于马克思与恩格斯,列宁发展了这个学说。到了现在苏联社会主义胜利与世界革命时期,这个学说又走上了新的发展阶段,更加丰富了它的内容。这个学说中包含的范畴首先是如下各项:

矛盾统一法则;

质量互变法则;

否定之否定法则。

以上是唯物辩证法的根本法则。除古代希腊的朴素唯物论曾经简单地无系统地指出了这些法则的某些意义,及黑格尔唯心地发展了这些法则外,都是被一切形而上学(所谓形而上学,就是反发展论的学说)所否定了的。直到马克思、恩格斯,才唯物地改造了黑格尔的这些法则,成为马克思主义世界观与方法论之最基本的部份。

唯物辩证法所包含的范畴,除了上述根本法则外,同这些根本法则联系着,还有如下各范畴:

1.本质与现象;

2.形式与内容;

3.原因与结果;

4.根据与条件;

5.可能与现实;

6.偶然与必然;

7.必然与自由;

8.链与环、等等。

这些范畴,有些是从来形而上学及唯心辩证法所着重研究过的,有些是从来哲学片面地研究过的,有些则是马克思主义新提出的。这些范畴,在马克思主义的革命理论家与实践家手里,揭去了从来哲学唯心的及形而上学的外衣,克服其片面性,发现了它们的真实形态,并且随着时代的进步,极大地丰富了它们的内容,成为革命的科学方法论中重要的成份。拿这些范畴同上述根本的范畴合在一起,就形成一个完整的深刻的唯物辩证法的系统。

所有这些法则或范畴,都不是人的思想自己造出来的,而是客观世界本来的法则。一切唯心论都说精神造出物质,那末,在他们看来,哲学的法则、原则、规律或范畴,自然更是心造的了。发挥了辩证法系统的黑格尔,就是这样的去看辩证法的。在他看来,辩证法不是从自然和社会的历史中抽取出来的法则,而是纯粹思想上的论理系统。人的思想造出了这一套系统之后,再把它们套到自然和社会上去。马克思、恩格斯揭去黑格尔的神秘的外衣,丢弃了它们的唯心论,把辩证法放在唯物论的地位。恩格斯说;“辩证法的法则,是从自然和人类历史抽取出来的,但他们并非别的,就是这两个历史发展领域的最普遍的发展法则,就实质论,可以归纳为质量互变,矛盾统一,否定之否定这三个根本法则”\mnote{28}。辩证法法则是客观世界的法则,同时也是主观思想里头的法则,因为人的思想里头的法则不是别的,就是客观世界的法则通过实践在人类头脑中的反映。辩证法、认识论、论理学是一个东西,前面已经讲过了。

我们学习辩证法是为了什么呢?不为别的,单单为了要改造这个世界,要改造这个世界上面人与人、人与物的老关系。这个世界上面的人类,大多数过着苦难的日子,受着少数人所控制的各种政治经济制度的压迫。在我们中国这个地方生活着的人类,受着惨无人道的双重性制度的压迫——民族压迫与社会压迫,我们必须改变这些老关系,争取民族解放与社会解放。

要达到改造中国同世界的目的,为什么要学习辩证法呢?因为辩证法是自然同社会的最普遍的发展发展,我们明了它,就得到了一种科学的武器。在改造自然同社会的革命实践中,就有了同这种实践相适应的理论同方法。唯物辩证法本身是一种科学(一种哲理的科学),它是一切科学的出发点,又是方法论。我们的革命实践本身也是一种科学,叫做社会科学或政治科学。如果不懂得辩证法,则我们的事情是办不好的。革命中间的错误,无一不违反辩证法。但如懂得了它,那就能生出绝大的效果。一切做对了的事,考究起来,都是合乎辩证法的。因此,一切革命的同志们,首先是干部,都应用心地研究辩证法。

有人说:许多人懂得实际的辩证法,而且也是实际的唯物论者,他们虽没有读过辩证法书,可是做起事来是做得对的,实际上合乎唯物辩证法,他们就没有特别研究辩证法的必要了。这种话是不对的。唯物辩证法是一种完备的深刻的科学,实际上具有唯物的与辩证的头脑之革命者,他们虽从实践中学得了许多辩证法,但是没有系统化,没有如同已经成就的唯物辩证法那样的完备性与深刻性,因此还不能洞察运动的远大前途,不能分析复杂的发展进程,不能捉住重要的政治关节,不能处理各方面的革命工作,因此仍有学习辩证法的必要。

又有人说,辩证法是深奥难懂的,一般人没有学会的可能。这话也是不对的。辩证法是自然、社会与思想的法则,任何有了一些社会经验(生产与阶级斗争的经验)的人,他就本来了解了一些辩证法。社会经验更多的人,他本来了解的辩证法就更多些,不过还处在零乱的常识状态,没有完备的深刻的了解。拿着这种常识辩证法加以整理与深造,是并不困难的。辩证法之所以使人觉得困难,是因为没有善于讲解的辩证法书,中国许多辩证法书,不是错了,就是写的不好或不大好,使人望而生畏。所谓善于讲解的书,在于以通俗的言语,讲亲切的经验,这种书将来总是要弄出来的。我这个讲义也不是好的,因为我自己还在开始研究辩证法,还没有可能写出一本好书,也许将来有此可能,我也有这个志愿,但要依研究的情形才能决定。

以下分述辩证法的各个法则。

\subsection{矛盾统一法则}

这个法则,是辩证法最根本的法则。列宁说:“就根本意义上来讲,辩证法就是研究客体本质中的矛盾”\mnote{29}。所以列宁常称这个法则为辩证法的实质,又称之为辩证法的核心。因此,我们的辩证法,就从这个问题讲起,并且把这个问题讲得比其他问题详细一些。

这个问题中,包含着许多问题,这些问题就是:

1.两种发展观;

2.形式论理学的同一律,与辩证法的矛盾律;

3.矛盾的普遍性;

4.矛盾的特殊性;

5.主要的矛盾与主要的矛盾方面;

6.矛盾的同一性与斗争性;

7.对抗在矛盾中的地位。

下面逐一说明这些问题。

\subsection{(一)两种发展观}

人类思想史中,从来就有关于世界发展的两种见解,一种是形而上学的发展观;一种是辩证法的发展观。这两种发展观的区别何在呢?

\subsubsection{形而上学的发展观}

形而上学,亦称玄学,在历来的思想中,占着统治的地位,这种哲学的内容,是说明他们所谓处于经验以外的事物,即论绝对体、论实质等等的学说。在近代哲学中,所谓形而上学,是用静的观点去观察事物的一种思想方法,把世界一切事物的形态和种类看成是永远不变化的。这种思想,统治于十七十八世纪的欧洲。由于阶级斗争和科学发展的结果,到了现代,即十九、二十世纪,辩证法的思想就一日千里地走上了世界舞台,但形而上学却又以庸俗的进化论(庸俗的、即谓鄙陋的、简单的)的形态,顽固地对抗着辩证法。

所谓形而上学的与庸俗进化论的发展观,概括说来,是说发展就是数量的增减,外力的推动,场所的变化。一切事物及这些事物在人的思想上的反映,都是永远如此的。事物的特性,是事物原来就有的,不过开头取萌芽状态,后来进到显著的地步而已。说到社会的发展,他们就认为是某些永远不变其性质的特点之增长和反复。这些特点,例如资本主义的剥削、竞争、个人主义等等,就是在古代奴隶社会,甚至原始野蛮社会,都可以找得出来。说到社会发展的原因,就用社会外部的地理、气候条件去说明它。这种发展观,从事物外部去找发展的原因,反对事物因内部矛盾引起发展的学说,它就不能解释事物之质的多样性,不能解释一种质变化到他种质的现象。这种思想,在十七、十八世纪是自然绝对不变论(机械唯物论),在二十世纪是庸俗进化论(布哈林的均衡论)等。

\subsubsection{辩证法的发展观}

主张从事物自己里头,从一事物对他事物的关系里头,去研究事物的发展,即把事物的发展看做是事物内部必然的、独立的、自己的运动,即事物的自动。事物发展的根本原因,不在外面而在内面,在于事物内部的矛盾性,任何事物内部都有这种矛盾性,因此引起了事物的运动与发展。

这样看起来,辩证法的发展观,反对了形而上学的与庸俗进化论的外因论,或被动论。这是清楚的,单纯的外部原因只能引起事物之机械的运动,即范围之大或小,数量之增或减,不能说明世界上事物何以有性质上的千差万别。事实上,即便是外力推动的机械运动,也要通过事物内部的矛盾性。植物动物之单纯的增长,也不只是数量的增加,同时就发生性质的变化,单纯增长也是矛盾引起的发展。至于社会的发展,同样主要地不是外因而是内因。许多国家在差不多一样的地理气候条件下,各个国家发展的差异性和不平衡性,却非常之大。设同一个国家罢,在地理气候并没有变化的情形下,社会变化却是很大的。地球各国都有此种情形。旧俄帝国变为社会主义的苏联,单纯封建的闭关锁国的日本变为帝国主义的日本,封建的西班牙正在变化到人民民主的西班牙,这些国家的地理气候并没有变。几千年封建制度的中国是变化最少的,然而近来却起了大变动,正在变化到自由解放的新中国去,难道中国今天的地理气候同数十年前的有什么两样?很明显的,不是外因而是内因。自然界的变化,由于自然界事物内部矛盾的发展。社会变化,由于社会内部矛盾的发展。生产力与生产关系的矛盾,阶级之间的矛盾,推动了社会的前进。辩证法排除外因吗?并不排除的。外因是变化的条件,内因是变化的根据,外因通过内因而起作用。鸡蛋因得适当温度而变化为鸡子,但温度不能使石头变为鸡子,因为内的根据不同。帝国主义的压力加速了中国社会的变化,也是通过中国内部自己的规律性而起变化的。两军相争,一胜一败,所以胜败,皆决于内因,胜者或因其强,或因其指挥无误,败者或因其弱,或因其指挥失宜,外因通过内因而引起变化。1927年资产阶级战败了无产阶级,是通过了无产阶级内部的(共产党内部的)机会主义而起作用的。一个阶级或一个政党要引导革命归于胜利,依靠自己没有政治路线的错误,依靠自己政治上组织上的巩固。中国东北沦亡,华北危急,主要由于中国之弱(1927年革命失败,政权不在人民手里,造成了内战与独裁制度),日本帝国主义乃得乘机而入。驱逐日寇,主要依靠民族统一战线执行坚决的革命战争。“物必先腐也,而后虫生之,人必先疑也,而后谗入之”,这是苏东坡的名言。“内省不疚,夫何忧何惧”,这也是孔夫子的实话。一个人少年充实,他就不容易感受风寒;苏联至今没有受日本的侵袭,全是因为他的强固;雷公打豆腐,拣着软的欺了,全在自强,怨天尤人,都没有用,人定胜天,困难可以克服,外界的条件可以改变,这就是我们的哲学。

我们反对形而上学的发展观,主张辩证法的发展观。我们是变化论者,反对不变论,我们是内因论者,反对外因论。

\subsection{(二)形式论理学的同一律与辩证法的矛盾律}

上面说了形而上学的发展观与辩证法的发展观,这两种对于世界观上面的斗争,就形成了思想方法上面形式论理与辩证论理的斗争。

资产阶级的形式论理学上有三条根本规律,第一条叫做同一律,第二条叫做矛盾律,第三条叫做排中律。什么是同一律呢?同一律说:在思想过程中概念是始终不变化的,它永远等于自己。例如原素永远等于原素,中国永远等于中国,某人永远等于某人。它的公式是:甲等于甲。这一规律是形而上学的。恩格斯说它是旧宇宙观的根本规律。它的错误,在于不承认事物的矛盾与变化,因而从概念中除去了暂时性相对性,给与了永久性、绝对性。不知事物同反映事物的概念都是相对的变化的,某一原素并不永远等于某一原素,各种原素都在变化着。中国也不永远等于中国,中国在变化着,过去古老封建的中国同今后自由解放的中国是两个东西。某人也不永远等于某人,人的体格思想都在变化着。1925—1927年的蒋介石不等于1927年以后的蒋介石,现在以后的蒋介石又将不等于以前。思想中的概念是客观事物的反映,客观事物在变化,概念的内容也在变化。事实上永久等于自身的概念,世界上一个也没有。

什么是矛盾律呢?矛盾律说:概念自身不能同时包含二个或二个以上互相矛盾的意义,假如某一个概念中包含了二个矛盾的意义,就算是论理的错误。矛盾的概念,不能同时两边都对,或两边都不对,对的只能是其中的一边,它的公式是:甲不等于非甲。康德曾举出如下四种矛盾思想:世界在时间上是有始终的,在空间上是有限度的;世界在时间上没有始终,在空间上亦无限度。这是第一种。世上一切都是单纯的(不可再分的)物性组成的,世上没有单纯的东西,一切都是复杂的(可以再分的)。这是第二种。世上存在着自由的原因,世上没有任何的自由,一切都是必然的。这是第三种。世上存在着某种必然的实质,世上没有必然的东西,一切都是偶然的。这是第四种。康德把这些不可调和的,互相反对的原理,名之曰“二律矛盾”。但是他说这些只是人的思想上的矛盾,实际世界里是并不存在的。依照形式论理学的矛盾律,这些矛盾乃是一种错误,必须加以排除。但是实际上思想是事物的反映,事物无一不包含着矛盾,因之概念也无一不包含矛盾。这不是思想的错误,正是思想的正确。辩证论理的矛盾统一律,就在这个基础上面建立起来。只有形式论理排除矛盾的矛盾律,乃是真正的错误思想。矛盾律在形式论理学中只是同一律之消极的表现,作为同一律的一种补充,目的在于巩固所谓概念等于自身,甲等于甲的同一律。

排中律是什么呢?排中律说:在概念之两相反的意义中,正确的不是这个就是那个,决不会两个都不正确,而跑出第三个倒是正确的东西来。它的公式是:甲等于乙,或不等于乙,但不会等于丙。他们不知道事物同概念是发展着的,在事物同概念的发展过程中,不但表现其内部的矛盾因素,而且可以看见这些矛盾因素的移去、否定、解决,而转变成为非甲非乙的第三者,转变成为较高一级的新事物或新概念。正确的思想,不应排除第三者,不应排除否定之否定律。无产阶级同资产阶级矛盾着,照排中律说来,正确的不是前者,就是后者,不会是没有阶级的社会;然而恰好社会进化的过程不是停止于阶级斗争,而要走到无阶级的社会中去。中国同日本帝国主义矛盾着,但我们不但反对日本帝国主义的侵略,也不赞同中国独立后同日本处于永久敌对的地位,而主张经过民族革命及日本国内的革命,把两个民族进到自由联合的阶段去。资产阶级的民主主义同无产阶级的民主主义的对立也是一样,它们的更高一级是无国家无政府的时代,经过无产阶级民主去达到它。形式论理的排中律,也是它的同一律的补充,只承认概念的固定状态,反对它的发展,反对革命的飞跃,反对否定之否定的法则。

由此看来,整个形式论理学的规律,都是反对矛盾性,主张同一性,反对概念及事物的发展变化,主张概念及事物的凝固静止,是同辩证法正相反对的东西。

形式论理家为什么这样做?因为他们在事物的联系以外,在事物不间断的相互作用以外去看事物,即在静止中看事物,不在运动中看事物;在割断中看事物,不在联系中看事物。所以他们以为承认事物及概念中的矛盾性及否定之否定的关系,是不可能的,而主张了死板凝固的同一律。

辩证法则不然,在运动中联系中看事物,和形式论理学的同一律针锋相对,主张了革命的矛盾律。

辩证法认为思想上的矛盾不是别的,乃客观外界矛盾的反映。辩证法不拘泥于两条原则外表上似乎相冲突的情形(例如康德所举的四条矛盾原理及上面我所举的许多矛盾思想),而透视到事物内部的本质中。辩证法家的任务,在于做那些形式论理家所不做的工作,向着研究的对象,集中注意于找出它的矛盾的力量、矛盾的倾向、矛盾的方面、矛盾的定性之内部的联系来。客观世界与人的思想都是动的、辩证的,不是静的、形而上学的。革命的矛盾律(即矛盾统一法则)在辩证法中所以占据着最主要的位置,理由就在这个地方。

全部形式论理学只有一个中心,就是反动的同一律。全部辩证法也只有一个中心,就是革命的矛盾律。辩证法是否反对事物或概念的同一性呢?不反对的,辩证法承认事物或概念之相对的同一性。那末,辩证法为什么要反对形式论理学的同一律呢?因为形式论理学的同一律,是排除矛盾的绝对的同一律。辩证法承认事物或概念的同一性,说的是同时包含矛盾,同时又互相联结;这种同一性就是指矛盾之互相联结,它是相对的、暂时的。形式论理的同一律既然是排除矛盾的绝对的同一律,它就不得不提出反对一概念转变到它概念,一事物转变到它事物的排中律。而辩证法却把事物或概念的同一性看作暂时的、相对的、有条件的,而因矛盾的斗争引导事物或概念变化发展的这种规律,则是永久的、绝对的、无条件的。因为形式论理不反映事物的真相,因此辩证法不能容许其存在。科学的真理只有一个,这真理就是辩证法。

\subsection{(三)矛盾的普遍性}

这个问题,有两方面的意义:其一是说,矛盾存在于一切过程中;其二是说,每一过程中存在着自始至终的矛盾运动。这就叫做矛盾的普遍性或绝对性。

恩格斯说:“矛盾就是运动”\mnote{30}。列宁对于矛盾统一法则所下的定义,说它就是“承认(发见)一切自然(社会和精神也在内)现象和过程中的相互排除的对立倾向。”\mnote{31}这些意见是对的吗?是对的。一切事物中包含的矛盾方面之相互依赖和相互斗争,决定一切事物的生命,推动一切事物的发展。没有矛盾,就没有世界。因此,这一法则,是最普遍的法则,适用于客观世界的一切现象,也适用于思想现象。它在辩证法中,是一个最根本的、最具有决定意义的法则。

为什么说矛盾就是运动?恩格斯的说法,不是有人反驳过了的吗?这是因为马克思、恩格斯、列宁论矛盾的学说,变成无产阶级革命之最重要的理论基础,因此,引起了资产阶级理论家之拼命的攻击,总想推翻恩格斯这个“运动即矛盾”的定律,举起了他们的反驳,并且搬出了下述的理由。他们说:实在世界中运动的事物,是在各个不同的瞬间,经过各个不同的空间点,当事物处于某一点时,它就占据那一点,到另一点时,又占据另一点。这样,事物的运动是在空间和时间上分成许多段落的,这里没有任何的矛盾;如有矛盾就不能运动。

列宁指出这种说法的全部荒谬性。指出这种说法,事实上把不断的运动,看成在空间和时间上的许多段落,许多静止状态,结果是否定了运动。他们不知事物处于某一个新位置,是因为事物从空间的某一点走到另一点的结果,即运动的结果。所谓运动,就是处于一点,同时又不处于一点。没有这一个矛盾,没有这个连续和中断的统一,动和静,止和行的统一,运动就根本不可能。否定矛盾,就是否定运动。一切自然、社会和思想的运动,都是这样一种矛盾统一的运动。

矛盾,不只是简单的运动形式(例如上述的机械性的运动的基础),而且也是世界一切复杂的运动形式的基础。

生的过程,同它相反的死的过程,不可分的联系着,这不仅在各种有机体的生命中,或有机体内细胞们的生命中,都是如此。新与陈之代谢、生与死之更迭,这一矛盾统一的运动,是一切有机体的生活和发展的必要条件。如果没有这种矛盾,生命现象是不能想象的。

机械学中,任何一种“动作”,都带着内部的矛盾性,引起“反动作”,没有反动作,动作就无从说起。

数学中,任何一个数量都带有内部的矛盾性,都可能成为正数与负数,整数与零数。正数与负数,整数与零数,组成了数学的矛盾运动。

化学中分化化合的矛盾统一律,组成了化学变化的无量的运动,没有这一矛盾,化学现象就不能存在。

社会生活中,任何一种现象,都带有阶级的矛盾性,劳动的买卖如此,国家的组织如此,哲学的内容也是如此。阶级斗争,是阶级社会的根本规律。

战争中的攻守、进退、胜败,都是矛盾现象。失去一方,他方就不存在,双方斗争而又联结,组成了战争的总体,推动了战争的发展。

人的概念之每一差异,都应把它看作客观矛盾的反映。客观矛盾反映人主观的思想,组成了概念的矛盾运动,推动了思想的发展。

党内不同思想之对立与斗争是经常发生的,这是社会阶层的矛盾在党内的反映。党内没有矛盾和解决矛盾的思想斗争,党的生命也就停止了。

不论是简单的运动形式,或复杂的运动形式,不论是客观现象,或思想现象,矛盾是普遍地存在着,矛盾存在于一切过程中。

说到这里,有人要说:可以承认恩格斯同列宁的原则,矛盾即是运动,矛盾存在于一切过程中。但是所谓每一过程中自始至终的矛盾运动,那就未必然罢?不是德波林等人明明说过,在每一过程中并无所谓自始至终的矛盾运动吗?按照德波林的说法,矛盾是存在的,但只存在于过程发展之一定阶段上,不是一开始就在过程中发现。根据德波林,过程的发展循着如次的阶段:开始是简单的差异,随后发生对立,最后成为矛盾。这种公式究竟是对的,还是错的呢?

这是错的。所谓矛盾的普遍性,不但存在于一切过程中,而且存在于每一过程之一切发展阶段中,这才是马克思主义的革命的矛盾律。根据德波林一派,矛盾不是一开始就在过程中出现,须待过程发展到一定阶段才出现,那末,在那一瞬间以前,过程发展的原因不是由于内在矛盾即过程的分裂,而是由于外在原因了。这样,德波林回到形而上学的外因论、机械论去了。拿这种见解去分析具体问题,他们就看见在苏联的条件下工农之间只有差异,并无矛盾,完全同意于布哈林的意见。在分析法国革命时,他们就认为在革命前,工、农资产阶级合组的第三等级中,也只有差异,并无矛盾(郭列夫\mnote{32}的说法)。他们不知道世上的每一差异中就已经包含着矛盾,差异就是矛盾。劳资之间,从两阶级发生的瞬间起,就是互相矛盾的,仅仅没有激化而已。工农之间,即使在苏联条件下,他们的差异就是矛盾,仅仅不会激化成为对抗,不取阶级斗争的形态,不同于劳资间的矛盾,这是矛盾的差别性,而不是有无矛盾的问题。矛盾是普遍的、绝对的,存在于一切过程中,又贯串于一切过程的始终。新过程的发生是什么呢?乃是旧的统一和组成此统一的对立体,让位于新的统一和组成此统一的对立体,新过程就代替旧过程而发生,新过程包含着新矛盾,开始它自己的矛盾发展史。

过程之自始至终的矛盾运动,列宁指出马克思在资本论中模范地应用了这个原则。他指出,这是研究任何过程所必须应用的方法,列宁自己也正确地应用了它,贯彻于他的全部著作中。

“马克思在《资本论》中,首先,分析资产阶级社会(商品社会)之最单纯的、最普遍的、最根本的、最经常的、最日常的、数十亿万回被人亲眼看见的关系——商品交换。在这最单纯的现象之中(资产阶级社会的细胞之中),暴露了现代社会之一切矛盾(或一切矛盾的胚芽)。从那里开始的叙述,把这个矛盾的发展(成长及运动),这个社会的发展,在其个别部分的总和上,自始至终地指示于我们。”

列宁说了上面的话之后,接着说道:“这正是辩证法的一般的叙述方法或研究方法。”\mnote{33}

好,我们不用读桐城派的古文义法了,列宁告诉了我们更好的义法,这就是马克思主义的科学研究法。

\subsection{(四)矛盾的特殊性}

矛盾存在于一切过程中,矛盾贯串于每一过程之始终,这是矛盾的普遍性与绝对性,前面已经说过了。现在说的,是关于矛盾的特殊性与相对性。这个问题,应从几种情形中研究它。

首先是各种物质运动形式中的矛盾,都带特殊性。人的认识物质,就是认识物质的运动形式,因为除了运动的物质以外,世界上什么也没有。对于每一种运动形式,应当注意它和其它各种运动形式的共同点。但尤其重要的,成为我们认识事物的基础的东西,乃是注意它的特殊点,就是说,注意它同其它运动形式之质的区别。只有注意这点,才有可能区别事物。唯物辩证法指明:任何运动形式,其内部都包含着本身特殊的矛盾。这种特殊矛盾,就构成一事物区别于他事物之特殊的质。自然界存在着许多运动形式,机械运动、发声、发光、发热、电流、化分、化合等等都是。所有这些物质的运动形式,都是互相依存的,又是本质上互相区别的。每一运动形式所具有的特殊的本质,为它自己的特殊矛盾所规定。这种情形,不但自然界,社会现象和思想现象也是一样。每一社会形式和思想形式,都有它的特殊矛盾和特殊本质。

科学研究的区分,就是根据科学对象所具有的特殊矛盾性。因此,对于某一现象领域所特有的某种矛盾之研究,就构成某一门科学的对象。例如数学中的正数与负数,机械学中的作用与反作用,物理学中的阴电与阳电,化学中的化分与化合,社会科学中的生产力与生产关系,阶级斗争,军事学中的攻击与防御,哲学中的唯心与唯物、形而上学观与辩证观等等,都是因为具有特殊矛盾与特殊本质,才构成了不同的科学研究的对象。固然,如果不研究矛盾的普遍性,就无从发现事物运动发展的普遍原因;但如果不研究矛盾的特殊性,就无从确定一事物与他事物的特殊的本质,就无从发现事物运动发展的特殊原因.也就无从辨别事物,无从区分科学研究的领域。

不但要研究每一大系统的物质运动形式之特殊的矛盾性及其所规定的本质;而且要研究每一物质运动形式在其发展的长途中,每一过程的特殊矛盾及其本质。一切运动形式之每一发展过程内,都是不同质的,天下没有同型的矛盾,研究要着重一点。

不同质的矛盾,只有用不同质的方法才能解决。例如无产阶级与资产阶级的矛盾,用社会主义革命的方法去解决;人民大众与封建制度的矛盾.用民主革命的方法去解决;殖民地与帝国主义的矛盾,用民族战争去解决;无产阶级与农民的矛盾,用农业社会化去解决;共产党内的矛盾,用思想斗争去解决;社会与自然的矛盾,用发展生产力去解决,过程变化,旧过程与旧矛盾消灭,新过程与新矛盾发生,解决矛盾的方法也因之而不同。俄国二月革命与十月革命,所用以解决矛盾的方法是根本不同的。用不同的方法去对付不同的矛盾,这是原则。

为要暴露过程中的矛盾在其总体上、在其相互联结上的特殊性,就是说暴露过程的本质,必须暴露过程中矛盾各方面的特殊性,否则暴露过程本质为不可能,这是研究问题要十分注意的。

一个大过程中包含着许多矛盾。例如在中国资产阶级民主革命过程中,有整个中国社会对帝国主义的矛盾,在中国社会内部有封建制度同人民大众的矛盾,有无产阶级同资产阶级的矛盾,有农民小资产阶级同资产阶级的矛盾,有各个统治集团间的矛盾等等,情形是非常复杂的。这些矛盾不但各个有其特殊性,不能一律看待,而且每一矛盾的两方两,又各有其特点,也是不能一律看待的。我们从事中国革命的人,不但要对各个矛盾总体即矛盾之相互联结,了解其特殊性,而且只有从矛盾的各个方面着手研究,才能了解其总体。所谓了解矛盾之各个方面,就是了解它们每一方面各占何等特定的地位,各用何种具体形式同对方发生依存关系,在依存中及依存破裂后又各用何种具体方法同对方作斗争。研究这些问题,乃是十分重要的事情。列宁主义的主要特点,就是研究无产阶级同资产阶级作斗争之各种具体形式的科学。

研究问题,忌带主观性、片面性与表面性。所谓主观性,就是不知道客观地看问题,也就是不知道用唯物的观点去看问题。这一点,第二章中已经说过,本节末尾也还要说。现在来说片面性与表面性。所谓片面性,就是不知道全面地看问题。例如只了解中国一方、不了解日本一方,只了解共产党一方、不了解国民党一方,只了解无产阶级一方.不了解资产阶级一方,只了解农民一方、不了解地主一方,只了解顺利情形一方、不了解困难情形一方,只了解正人君子一方、不了解奸巧狡诈一方,只了解现在一方。不了解将来一方,只了解自己一方、不了解他人一方,只了解骄傲一方、不了解谦逊一方,只了解缺点一方、不了解成绩一方,只了解原告一方、不了解被告一方,只了解秘密工作一方、不了解公开工作一方,如此等等。一句话,不了解矛盾各方的特点。这就叫做片面地看问题,或叫做只看见局部,不看见全体,是不能找出解决矛盾的方法的(是不能完成革命任务的,是不能做好所任工作的,是不能正确发展党内思想斗争的)。孙子论军事说:“知己知彼,百战百胜。”他说的是矛盾的双方。唐太宗也说:“兼听则明,偏听则暗。”也懂得片面性不对。可是我们的同志看问题,往往带片面性,这样的人就往往碰钉子。乡下两家或两族相争,做和事老的,须熟识双方争论的原因、争点、现状、要求等等,才能思出和解的办法来。乡下有那种善于和事的人,遇有纠纷,总请他到,这种人实在懂得我们说的要了解矛盾各方面特点这一条辩证法。水浒传上宋公明三打祝家庄,两次都因情况不明,方法不对,打了败仗。后来改变方法,从调查情形入手,于是熟悉了盘陀路,拆散了李家庄、扈家庄与祝家庄的联盟,并且布置了藏在敌人营盘里的伏兵,第三次就打了胜仗。水浒传上有很多唯物辩证的范例,这个三打祝家庄,算是最好的一例,列宁屡次说到对问题要全面去看,坚决反对片面性,我们应该记得他的话。表面性,是说对矛盾总体与矛盾各方的特点,都不去看,否认深入事物里面精细研究矛盾特点的必要,仅仅远远地望一望,粗枝大叶地看到一点矛盾的形相,就想动手去解决矛盾(答复问题,解决纠纷,处理工作,指挥战争)。这样的干法,没有不出乱子的。不但全过程中矛盾运动在其相互联结上,在其各方面情况上,应该注意其特点;过程发展的各阶段,也有特点,也应该注意。过程的根本矛盾及为此根本矛盾所规定的过程之本质,非到过程完结之日是不会消灭的;但是过程的各个发展阶段,情形又往往互相区别。这是因为过程之根本矛盾的性质及过程的本质虽没有变化,但根本矛盾在各个发展阶段上采取逐渐激化的形式,并且,为根本矛盾所规定的许多大小矛盾中.有些是激化了,有些是暂时地局部地解决了,或缓和了,又有些是发生了,因此过程就显出阶段性来。

例如帝国主义之别于自由资本主义,无产阶级与资产阶级这个根本矛盾的性质及这个社会之资本主义的本质,并没有变;但是两阶级的矛盾激化了,独占资本与自由资本之间的矛盾发生了,各独占集团之间的矛盾发生了,资本输出与商品输出的矛盾发生了,宗主国与殖民地的矛盾激化了,各资本主义国家间的矛盾,即各国不平衡发展状态激化了,因此形成了帝国主义的特殊阶段。

拿从辛亥革命开始的中国民主革命过程的情形来看,也表现了若干特殊阶段。直到这一革命完成为止,也许还要经过若干阶段,虽然整个过程中根本矛盾的性质及过程之反帝反封建的民主革命的本质(其反面是半殖民地半封建的本质),并没有变;但中间经过辛亥失败,北洋军阀统治,第一次民族统一战线建立与大革命,统一战线的破裂与资产阶级的转入反革命,军阀战争,苏维埃战争,东四省丧失,苏维埃战争停止,国民党政策转变,第二次统一战线建立等等大事变,过去二十多年间已经通过了四五个发展阶段。这些阶段中,包含着有些矛盾激化(例如中日矛盾),有些矛盾部份地暂时地解决(例如北洋军阀的消灭,苏区没收地主土地),有些矛盾又重新发生(例如新军阀之间的斗争,苏区丧失后地主又重新收回土地)等等特殊的情形。

研究过程各阶段上矛盾的特性,不但在其联结上、在其总体上去看,也同样要从各个方面去看。

例如国共两党。国民党方面,在第一次统一战线时是革命的、有朝气的,它是各阶级的民主革命联盟。1927年以后,变到相反的方面,成为地主资产阶级的反动集团。西安事变后,又开始向新的方面转变。这就是国民党在三个阶段上的特点。形成这些特点,当然有种种的原因。共产党方面,第一次统一战线时期,它是幼年的党,对于革命的性质、任务、方法等等的认识,均表现了它的幼年性,因此发生了陈独秀主义;但是它领导了第一次大革命。1927年以后,领导了苏维埃战争,在同国际国内敌人斗争中锻炼了自己,创造了苏区与红军,但它也犯过一些政治上军事上的错误。1935年以后,它又领导了统一战线,提出了抗日民族战争与民主共和国的口号。这就是共产党在三个阶段上特点。形成这些特点,也有种种的原因。不研究这些特点,就不能了解两党在各个发展阶段上的特殊的相互关系(统一战线,统一战线破裂,再一个统一战线)。不但两党间,而且更根本的,还有这两个党向其他方面形成矛盾的对立。例如国民党同国外帝国主义的矛盾(有时取妥协形态),同国内人民大众的矛盾。共产党同国外帝国主义的矛盾,同国内剥削阶级的矛盾。由于这些矛盾,所以造成了两党的斗争,又造成了两党的合作。不了解这些矛盾方面的特点,不但不能了解这两个党各个同其他方面的关系,也不能了解两党之间的关系。国民党为什么有与共产党重新合作之可能?就是因为国民党受了日本压迫与人民不满而发生了自己内部变动的原故。

由此看来,不论研究何种矛盾的特性——各个物质运动形式的矛盾,各运动形式在各个发展过程的矛盾,各个发展过程的矛盾之各方面,各个发展过程在其各个发展阶段上的矛盾,以及各个发展阶段上矛盾之各方面,研究所有这些矛盾的特性,都不能带主观随意性,必须以对它们的具体分析为前提。离开具体分析,就决不能认识矛盾的特性。

这种具体分析,马克思、恩格斯给了我们以很好的模范。

当马克思.恩格斯把这一矛盾统一法则应用到社会历史过程的研究时,他们看出社会发展的根本原因,在于生产力和生产关系之间的矛盾,阶级斗争的矛盾,以及由这些矛盾所产生的经济基础同政治及思想的上层建筑之间的矛盾。

马克思把这—法则应用到资本主义社会经济结构的研究时,他看出这一社会的基本矛盾在于生产的社会性和占有的私人性之间的矛盾。这个矛盾,表现于在个别企业中生产的有组织性和在全社会中生产的无组织性之间的矛盾。这个矛盾的阶级表现则是资产阶级与无产阶级之间的矛盾。

马克思、恩格斯对于应用辩证法到客观现象的研究时,是这样不带任何主观随意性,而从客观现象的实际运动所包含的具体条件,去看出这些现象中的具体矛盾,矛盾各方面之具体的地位,矛盾之具体的相互关系等等。这种研究态度,是我们应当学习的,舍此便没有第二种研究法。

矛盾的普遍性与矛盾的特殊性之关系,就是矛盾的共性与个性之关系。其共性是矛盾存在于一切过程中,贯串于一切过程的始终,矛盾即是运动,即是事物,即是过程,即是世界,也即是思想。否认矛盾就是否认了一切。这是共通的道理,古今中外,概莫能例外。所以它是共性,是绝对性。然而这种共性,即包含于个性之中,共性表现于一切个性之中,无个性之存在,也就不能有共性之存在。假如除去了一切个性,还有什么共性呢?因矛盾之各各特殊,大宇长宙,无一同者,变化无穷,其存在也暂,所以是相对的。苏东坡说,“自其变者而观之,则天地曾不能以一瞬”。照现在的意思来说,可以说他说的是矛盾的特殊性,相对性。“自其不变者而观之,则物与我皆无尽。”说的是矛盾的普遍性,绝对性。这一共性个性、绝对相对的道理,是矛盾学说的精髓,懂得了它,就可以一通百通。古人所谓闻道,以今观之,就是闻这个矛盾之道。

\subsection{(五)主要的矛盾与主要的矛盾方面}

在矛盾特殊性问题中,有两种情形应该特别提出研究的,这就是主要的矛盾与主要的矛盾方面。

在复杂的过程中.有许多矛盾存在,其中一个是主要矛盾,由于它的存在与发展,规定或影响其他矛盾的存在与发展。

例如资本主义社会中,无产阶级与资产阶级的矛盾是主要的矛盾;其他如残存的封建势力与资产阶级的矛盾,农民小资产者与资产阶级的矛盾,无产阶级与农民小资产者的矛盾,自由资产阶级与金融资产阶级的矛盾,资产阶级民主主义与法西斯主义的矛盾,资本主义国家相互间的矛盾,帝国主义与殖民地的矛盾,以及其他矛盾等等,都为这个主要矛盾所规定、所影响。

半殖民地的社会如中国,其主要矛盾与非主要矛盾的关系呈现着复杂的情况。当半殖民地没有遭受帝国主义压迫时,其主要矛盾是封建或半封建制度与人民大众的矛盾,一切其他矛盾都受这个主要矛盾所规定。但当这种社会遭受帝国主义压迫时,内部的主要矛盾能够暂时地转化到非主要地位,而帝国主义与整个或差不多整个半殖民地社会之间的矛盾,能够占据主要的地位,规定一切其他矛盾的发展。这种时候,依帝国主义压迫及半殖民地人民革命的程度,而变化着或主要或非主要矛盾的地位。

例如当帝国主义向这种国家举行侵略战争,这种国家的内部各阶级,能够暂时地团结起来进行民族战争去反对帝国主义。这时,帝国主义与这种国家之间的矛盾成为主要矛盾,而这种国家内部各阶层的一切矛盾(包括封建制度与人民大众之间这个主要矛盾在内),便都暂时地降到次要与服从的地位。中国的鸦片战争,义和团战争,甲午中日战争,目前的中日战争,在外国,有美国的独立战争,南非洲同英国的战争,菲律宾同西班牙的战争等等,都是如此。

然而在另一种情形,则矛盾的地位起了变化。当着帝国主义不用战争压迫而用政治、经济、文化的形式进行比较温和的压迫,半殖民地国家的统治阶级就向帝国主义投降,二者之间结成同盟,由二者的对抗变成二者的统一,共同压迫人民大众。这时,人民大众往往采取用国内战争的形式,去反对帝国主义与封建阶级的联盟,而帝国主义则往往采取秘密援助国内的统治阶级压迫国内的革命战争,而不直接行动,显出了内部矛盾的特别尖锐性。例如中国的太平军战争,辛亥革命,1925—1927年的大革命,1927年以后的苏维埃战争。在外国,则有俄国的二月革命与十月革命(俄国也带了若干半殖民地性),中美洲南美洲若干带革命性的内战等等的情形,都是如此.还有半殖民地各统治集团之间的内战,也表现了内部矛盾尖锐的情况。在中国,在中南美,也是很多的,也属这一类。

当着国内战争发展到根本威胁帝国主义及其走狗国内统治者的存在时,帝国主义就往往采取上述方法以外的方法,企图维持其统治;或者分化革命阵线的内部,例如1927年中国资产阶级的叛变,或者直接出兵援助国内统治者,例如苏联内战的末期,今日的西班牙战争。这时,帝国主义与国内封建阶级乃至资产阶级完全站在一个极端,人民大众则站在另一极端,这时帝国主义与全殖民地之间这个外部的主要矛盾,封建势力与人民大众之间这个内部的主要矛盾,就几乎合并起来,成为一个主要矛盾,而规定其他矛盾的发展地位,情形是非常明显的。

然而不管怎样,过程发展之各个阶段中,只有一个主要矛盾起着领导的作用,是完全没有疑义的。

由此可知,任何过程如果有多数矛盾的话,其中必定有一个是主要的,起着领导的、决定的作用,其他则处于次要与服从的地位。因此,研究任何过程,首先要弄清它是单纯的过程,还是复杂的过程。如果是存在着二个以上矛盾的复杂过程的话,就要用全力找出它的主要矛盾。捉住了这个主要矛盾,一切问题就迎刃而解了。这是马克思研究资本主义社会告诉我们的方法。列宁研究帝国主义时,列宁和斯大林研究苏联过渡期经济时,也同样告诉了我们这种方法。万千的学问家、实行家,不懂得这种方法,结果如堕烟海.找不到中心,也就找不到解决矛盾的方法。

不能把过程中的矛盾平均看待,应把它们区别为主要的与次要的两类,着重于捉住主要矛盾,既如上述。但是矛盾之中,不论主要的或次要的,矛盾着的两个方面或侧面又是否可平均看待呢?也是不可以的。无论什么矛盾,也无论在什么时候。矛盾的方面或侧面,其发展是不平衡的。有时候似乎势均力敌,然而这是暂时的与相对的情形,基本形态则是不平衡,就是在似乎平衡之时,实际上也没有绝对的平衡。矛盾着的两方面中,必有一方是主要的,他方是次要的。其主要方面,即所谓矛盾起主导作用的方面。

然而这种情形不是固定的,矛盾的主要与非主要的方面互相转化着。在矛盾发展的一定过程或一定阶段上,主导方面属于甲方,非主导方面属于乙方;及到另一发展阶段或另一发展过程时,就互易其位置,这是依靠双方斗争的力量来决定的。

例如资本主义社会,在长时期中,资产阶级处于主要地位,起着主导的作用,无产阶级则服从之;但到革命前夜及革命之后,无产阶级就转化到主要地位,起着主导的作用,而资产阶级作了相反的转化。十月革命前后的苏联就是这种情形。

在资本主义社会中,资本主义已从过去封建社会时的附庸地位,转化成了主要力量,封建势力则由主要化为附庸。但何以解释日本及革命前的俄国呢?他们依然是封建势力占着优势,资本主义尚不起决定一切的作用。这是因为他们的矛盾方面尚未完成其决定的转化的原故。这种转化,因为时代的关系,已经不能走历史的老路,而为另一种情形的转化所代替,即是把地主阶级与资产阶级整个儿地转到被统治的地位,而由无产阶级与农民起来占据主导的方面。目前一切尚未完成资本主义转化的国家(中国也在内)都将走向这条新路,虽然并不跳过民主革命的阶段,可是这种革命是由无产阶级领导执行的。

帝国主义与整个中国社会的矛盾中,主导的方面属于前者,它在双方斗争中占着优势。然而事情也正在变化,在彼此对立的局面中,中国一方正由被压迫地位向自由独立的地位转化,而帝国主义则将转化到被打倒的地位。

中国国内封建势力同人民大众对抗的情况也正在变化,人民将依靠革命斗争把自己转化为主要与统治的力量。过去已有过例证,这就是南方革命势力由次要地位转化到主要地位,而北洋军阀则作了相反的转化。苏区中也有此种情形,农民由被统治者转化为统治者,地主则作了相反的转化。

以中国无产阶级与资产阶级的关系而言,资产阶级因握有生产手段与统治权,至今还居于主导地位,然在反帝反封建的革命领导上说来,由于无产阶级觉悟程度与革命的彻底性,却较之动摇的资产阶级反居于主导地位,这一点将影响到中国革命之前途。无产阶级要在政治上物质上都居于主导地位,只有联合农民与小资产阶级。果能如此,革命之决定的主导的作用就属于无产阶级了。

在农民与工人的矛盾中,目前工人的主导地位,曾经是由附庸地位转化而来,而农民作了相反的转化。在产业工人与手工工人的矛盾中,在熟练工人与非熟练工人的矛盾中,在城市与乡村的矛盾中,在劳心与劳力的矛盾中,在唯物论与唯心论的矛盾中,都作了同样的转化。

革命斗争中某些时候,困难条件超过顺利条件,这时,困难是矛盾的主要方面,顺利是其次要方面。然而由于革命党人的努力,利用已有的若干顺利条件作基础,能够逐渐克服困难,开展顺利的新局面,困难的主导地位转化到以顺利为主导。1927年革命失败后的情形,红军长征中的情形,都是如此。今日的中日战争,中国又处在十分困难的地位,但我们应该也能够努力于它的转变。在相反的情形,顺利也能转化为困难,如果是革命党人犯了错误的话。1925至27年的大革命的胜利转化为失败,中央苏区一、二、三、四次战争粉碎围剿的胜利转化为五次围剿的失败,等等皆是。

研究学问时,由不知到知的矛盾也是如此。没有研究马克思主义的人,不知或知之不多是矛盾的主要方面,精深博大的马克思主义则是矛盾的另一方面,然而由于学习的努力,可以由不知转化到知,由知之不多转化到知之甚多,我们的许多同志正是走的这条路。在相反的情形也一样,如果中途拒绝前进,或甚至想入非非走了邪路,已有的知可以化为不知,正确可以化为错误。考茨基、普列哈诺夫\mnote{34}、陈独秀等人就是走了这条路。我们队伍中的若干自大主义者,如果他不改变,也有这种危险。

据我看来,一切矛盾方面之主导与非主导的地位,都是这样互相转化的。

有人觉得有些矛盾并不是这样。例如生产力与生产关系的矛盾,生产力是主导;理论与实践的矛盾,实践是主导;经济基础与上层建筑的矛盾,经济基础是主导。如此等等,它们并不互相转化。须知这是就一般情形而言,站在唯物论的基点上,它们确是不转化的绝对的东西。然而就历史上许多特殊情形而言,它们仍在转化着。生产力、实践、经济基础,一般表现主导的决定的作用,谁不承认这一点,谁就不是唯物论者。然而,生产关系、理论、上层建筑这些方面,有时亦表现其主导的决定的作用,这也是应该承认的。当着不变更生产关系,生产力就不能发展之时,生产关系的变更就起了主导的决定的作用。当着如同列宁所说的“没有革命理论,就没有革命运动”\mnote{35}之时,革命理论的提倡就起了主导的决定的作用。当着某一件事情(任何事情都是一样)要做,但还没有方针、方法、计划或政策之时,确定方针、方法、计划或政策,也就是主导的决定的东西。当着政治文化等等上层建筑阻碍着经济基础的发展时,对于政治文化上面的革新就成为主导的决定的东西了。这样来说,是否违反唯物论呢?不违反的。因为我们承认总的历史发展中是物质的东西决定精神的东西;但同时又承认而且应该承认,精神的东西之反作用。这不是违反唯物论,而正是避免机械唯物论,坚持了辩证唯物论。

在研究矛盾特殊性问题中,如果不研究过程中主要矛盾与非主要矛盾,及矛盾之主要方面与非主要方面这两种情形,也就是说,研究这两种的差别性,那就仍将陷入于抽象的研究,不能具体地懂得矛盾,因而也不能找出解决矛盾的正确方法来。这两种差别性或特殊性,都是矛盾的不平衡性。世界没有绝对地平衡发展的东西,所以成其为世界,我们应该反对平衡论(或均衡论)。矛盾之各种不平衡中,对于主要与非主要的矛盾、主要与非主要的矛盾方面之研究,成为革命政党正确决定其政治上战略战术的基本方法之一(军事上也是一样)。所以不能不充分注意这个问题。

\subsection{(六)矛盾的同一性与斗争性}

在解决了矛盾的普遍性与特殊性的问题之后,必须进而研究矛盾的同一性与斗争性的问题,矛盾统一律的研究才算全部地解决了。

同一性、统一性、一致性、互相渗透、互相贯通、互相依赖(或依存)、互相联结或互相合作,这些不同的名词都是一个意思,说的是如下两种情形:第一、过程每一矛盾的两方面,各以它方面为自己存在的前提,共处于一个不可分的统一体中;第二、矛盾的双方依据一定条件,各向着其相反的方面转化。这些就是所谓同一性。

列宁说:“辩证法是关于矛盾怎样能够是同一性,又怎样是同一性(怎样变成同一性),在怎样的条件之下矛盾变成同一性而互相转化。为什么人的思想不把这些矛盾当作死的、凝固了的东西去看,却当作生动的、附条件的、可变动的、互相转化的东西去看等等问题的学说。”\mnote{36}

列宁这句话,说的是什么意思呢?

一切过程中矛盾着的各方面,本是互相对立的,是彼此不融洽、不对头、不相好、不和气的,都是些充满怨气的冤家。世上一切过程、现象、事物、思想里面,都包含着选样带冤家性的方面,没有一个例外。单纯的过程只有一对冤家,复杂的过程却有二对以上的冤家。各对冤家之间,又互相成为冤家。这样组成过程、现象、事物,并推使发生运动。

如此说来,只是极不同一,极不统一,怎样又说是同一或统一呢?世事之怪就怪在这里,妙也就妙在这里。

原来矛盾的各方面,不能孤立存在。假如没有冤家一方,它自己这方就失掉了存在的条件。试想一切矛盾的事物,或人的心中矛盾的概念,矛盾的任何一方面能够独立存在吗?不能够的。没有生,死就不见;没有死,生也不见。没有上,就无所谓下;没有下,也无所谓上。没有祸,就无所谓福;没有福,就无所谓祸。没有顺利,就无所谓困难;没有困难,也无所谓顺利。没有资产阶级,就没有无产阶级;没有无产阶级,也没有资产阶级。没有殖民地,就不能有帝国主义的压迫;没有帝国主义的压迫,也就不能有殖民地。一切过程、现象、事物之内的对立,对立的双方都是这样,因一定的条件,一面互相对立,一面又互相联结、互相贯通、互相渗透、互相依赖、互相勾搭、又是冤家又聚头,这种性质,叫做同一性。一切矛盾都因一定条件具备着不同一性,所以称为矛盾。然而又具备着同一性,所以互相联结。列宁所谓辩证法研究怎样能够是同一性,就是说的这种情形。这是同一性的第一个意义。

然而单说了矛盾双方互为存在条件,双方之间有同一性,因而能够共处于一个统一体中,这样就够了吗?那是不够的。事情不是矛盾互相依存就完了,还没有完,重要的事情,还在矛盾之互相转化。事物内部矛盾的方面,因一定的条件而向着相反的方面转化了去,这就是矛盾的同一性之第二个意义。

为什么这里也有同一性呢?你看,生死关系中生向死转化,不论是有机体中或有机体内细胞的生命中,生总不能长久,而在一定条件下走向它的反对方面,变为死。死呢?也不是一死完事,又必在一定条件下产出新生命来,死变为生。试问如果没有联系、没有瓜葛、没有亲属关系,就是说没有同一性,为什么生死这样相反的东西之间能够互相转化呢?

被压迫被剥夺的无产阶级向着无产阶级专政,即不再被压迫、不再被剥夺的方面转化,而资产阶级却经过阶级崩溃转到受无产阶级国家统治方面。苏联已经这样做了,全世界也都将要这样做。试问其间没有在一定条件之下的联系与同一性,如何能够发生这样的变化?

帝国主义压迫殖民地与殖民地受帝国主义压迫的命运都不能长久,帝国主义者将要被殖民地人民与他本国人民的革命势力所推翻而站在人民的统治之下。殖民地和帝国主义内部的人民呢?却有解除压迫走到自由解放(被压迫的反面)之一日,二者之间由于一定条件有共同性、同一性。

1927年大革命的正规战争,转化为苏维埃的游击战争;开始时期的苏维埃游击战,又转化为后来的正规战争;今后又正在由苏维埃战争向着抗日的民族战争转化了去。这其间都因在一定条件下而发生同一性,相反的东西中间互相渗透、贯通、勾搭着。

国民党的带革命性的三民主义,因为它的阶级性及帝国主义的引诱(这就是条件),在1927年以后转化成为反动政策。又由于中日矛盾的尖锐化及共产党的统一战线政策(这也是条件),而被迫着转向抗日救亡的方面去。矛盾的东西这一个变到那一个,其间包含了这样的同一性。

苏区的土地革命,已经是并将要是这样的过程:拥有土地的地主阶级转化成为失掉土地的阶级,而曾经是失掉土地的农民却转化到取得士地的小私有者。有无、得失之间,因一定条件而互相联结,变为同一性。在社会主义条件之下,农民的私有制又将转化到社会主义农业的公有制,苏联已经这样做了,我们将来也会要这样做。私产与公产之间有一条由此达彼的桥梁,哲学上名之曰同一性,或互相渗透。

资产阶级民主主义与无产阶级民主主义是相反的,然而前者必会转化为后者。相反的东西中间,在一定条件下,就产生了相成的因素。

提高民族文化,正是准备转化到国际文化的条件。争取民主共和国,正是准备取消民主共和国转向新的国家制度的条件。巩固无产阶级专政,正是准备取消这种专政走到消灭任何国家制度的条件。建立与发展共产党,正是准备消灭共产党及一切党派的条件。建立革命军进行革命战争,正是准备了永远消灭战争的条件。这许多相反的东西,却同时又是相成的东西。

有些人说:共产党是国际主义者,决不会也不能同时又是爱国主义者。我们却宣称:我们是国际主义者,但同时因为我们是殖民地的党(条件),所以必须为着保卫祖国反对帝国主义的压迫而斗争,因为必须首先脱离帝国主义的压迫,才能参加世界的共产主义社会,这就使二者构成了同一性。爱国主义与国际主义,在一定条件下,相反而又相成。为什么帝国主义国家的共产党坚决反对爱国主义,因为那里的爱国主义只同资产阶级的利益有同一性,它同无产阶级的利益则是根本相反的。

有些人说:共产党不会也不能,同时又相信三民主义。我们却宣称:我们是坚持共产主义的党纲的,但是当前阶段的共产主义运动不是别的,正是坚决领导反帝反封建的民族民主运动(这就是条件),因此我们不但不反对,而且早已执行了真正的三民主义纲领(反帝的民族主义,工农苏维埃的民权主义,土地革命的民生主义);并且十年来真正的三民主义传统也仅仅在于共产党一方面。国民党除若干分子如宋庆龄、何香凝、李锡九等人而外,抛弃了这个传统。共产党的民主革命政纲不与真正的三民主义冲突,而且就是彻底的急进的三民主义,我们将经过民主阶段转变到共产主义。三民主义与共产主义不是一个东西,二者矛盾着,现在阶段与将来阶段不是一个东西,二者矛盾着,但是相反而又相成,因一定的条件造成了同一性。

还可以说一些最眼前的事情。战争与和平是矛盾的,但又是联结的。战争转化为和平(例如第一次大战转化为凡尔赛条约,中国的国内战争在西安事变后转化为国内和平),和平转化为战争(目前的世界和平是暂时的,即将转化为第二次大战;日本侵略东四省后几年的和平是暂时的,现已开始转化为大陆战争)。为什么?因为在一定条件下具备了同一性。中国无产阶级同资产阶级订立抗日的统一战线,这是矛盾的一方面;无产阶级须得提高政治的警觉性,密切注视资产阶级的政治动摇及其对于共产党的腐化作用与破坏作用,以保证党与阶级的独立性,这是矛盾的又一方面。各党的统一战线与各党的独立性,这样矛盾着的两方面,组成了当前的政治运动,两方面中去掉一方面,就没有党的政策,就没有统一战线了。我们给人民以自由,这是一方面,我们又给汉奸卖国贼破坏者以压制,这是又一方面。自由与不自由二者因一定条件而联系着,缺一就不行,这是矛盾的统一或同一性。共产党、苏维埃,以及我们主张的抗日政府之组织形式,是民主集中制的。它们是民主的,但又是集中的,二者矛盾而又统一着,因为在一定条件之下有同一性。苏联的无产阶级民主专政,我们过去十年的工农民主专政,它们是民主的,对于革命阶级而设;它们又是专制的(或叫独裁的),对于反革命阶级而设,极端相反的东西之间有同一性。

军队的休息、训练,同时就是作战胜利的条件。“养兵千日”,正是为了“用在一朝”。分开前进,同时就是到达协同攻击的条件(分进合击)。退却与防御,同时就是为着反攻与进攻(以退为进,以守为攻)。迂回不是别的,就是最有效地消灭敌人的方法(以迂为直)。向东方打一打,为的要在西方得手(声东击西)。分兵以争取群众,为了便于集中以消灭敌人;集中以消灭敌人,为了便于分兵以争取群众。要坚决执行命令,又容许在统一意图下有机动的自由。要严格执行纪律,又要发扬自觉自动性。允许陈述个人志趣,但最后还是要服从团体的决定。前方工作要紧,但后方工作不能抛弃不顾。身体不好需要调养,但紧张时候又要讲牺牲。谁不赞成生活优裕,但经济困难却要准备吃苦。军事操练是重要的,非此不能破敌;但政治工作又重要,非此也就要打败仗。老兵、老干部经验丰富,是值得宝贵的;但如果没有新兵、新干部,战争与工作就不能继续。勇猛要紧,也还要智谋;张飞虽不错,到底不如赵子龙。自己领导的局部工作是重要的;但他人领导的局部及全体工作也重要或更重要,小团体主义是不正确的。自己的意见与团体的或上级机关的意见相矛盾时,可以而且应该陈述自己的意见,可是绝不容许在自己意见未被团体或上级批准时,向任何其他人员自由发表;或甚至煽动下级人员反对上级的意见。这种少数服从多数,下级服从上级的纪律,是共产党与红军的起码的纪律。“良药苦口利于病,忠言逆耳利于行”;“祸兮福所倚,福兮祸所伏”;“爱而知其恶,恶而知其美”。顾前不顾后,叫做莽夫。知一不知二,未为贤者。

一切矛盾的东西,互相联系着,不但在一定条件之下共处于一个统一体中,而且在一定条件之下互相转化,这就是矛盾的同一性之全部意义。列宁所谓怎样是同一性,在怎样条件之下变成同一性而互相转化,就是这个意思。

“为什么人的思想不把这些矛盾当作死的、凝固了的东西去看,却当作生动的、附条件的、可变动的、互相转化的东西去看”\mnote{37}呢?因为客观事物本来是如此的。客观事物中矛盾的统一或同一性,本来不是死的、凝固的,而是生动的、附条件的、可变动的、暂时的、相对的东西,一切矛盾都依一定条件向它们的反面转化着。

为什么鸡蛋转化为鸡子,而石头不能转化为鸡子呢?为什么战争与和平有同一性,而战争与石头却没有同一性呢?为什么人能生人却不能生狗呢?没有别的,就是因为矛盾的同一性要在一定条件之下,缺乏一定的必要的条件,就没有任何的同一性。

为什么俄国的民主革命与社会主义革命直接地联系着,而法国的民主革命没有直接联系社会主义革命,巴黎公社到底失败了呢?为什么外蒙古与中亚细亚的游牧制度又直接与社会主义联系了呢?为什么中国的革命可以避免资本主义前途,可以同社会主义直接联系起来,而避免再走英美法等的历史老路呢?为什么俄国1905年的革命同中国1911及1927年的革命都不与革命的胜利联系,却与失败联系了呢?为什么拿破仑一生的战争大都与胜利联系着,而滑铁炉\mnote{38}一战却军败身俘一蹶不振呢?为什么“可以修一条铁路往新疆,却不能修一条铁路往月球”呢?为什么德苏亲交变为敌视,而法苏敌视却又变为暂时的亲交呢?所有这些问题,没有别的,都是当前的具体条件的问题。一定的必要的条件具备,过程就发生矛盾,而且矛盾互相依存,又互相变化,否则一切都不可能。唐吉诃德的奋力同风车作战、孙悟空的十万八千里的筋斗云、阿丽斯的漫游奇境、鲁滨孙的漂流孤岛、阿Q的精神胜利、希特勒的世界统治、黑格尔的绝对精神、布哈林的均衡论、托洛茨基的不断革命、御用学者的思想统一、陈独秀的机会主义、亲日派的唯武器论以及中国古代传说中的杞人忧天、夸父追日等等,都不能成为矛盾的同一性,不能成为具体的矛盾,反在人间添些麻烦与笑话的资料,也就是这个道理。

同一性的问题如此。那末,什么是斗争性呢?同一性同斗争性的关系怎样呢?

列宁说:“矛盾的统一(合致,同一,均势),是有条件的、一时的、暂存的、相对的。互相排除的斗争则是绝对的,发展运动是绝对的”\mnote{39}。这话怎讲?

一切过程都有始有终,一切过程都转化为它们的对立物。一切过程的常住性是相对的,但是一种过程转化为他种过程则是绝对的。矛盾的统一、同一、一致、常住性、联合性,被包含于矛盾的斗争之中,成为矛盾斗争之一因素.这就是列宁这句话的意思。

这就是说,单只承认矛盾引起运动是不够的,还须明白矛盾在那些状态引起运动。

矛盾在第一种统一(同一)状态引起运动,那是运动的特殊状态,日常生活中叫做静止、有常不变、不动、死、停顿、僵局、相持、和平、平衡、均势、调和、妥协、联合,等等,这些都是相对的、暂时的、有条件的。还须承认矛盾在第二种统一状态引起运动,即运动之一般状态。这就是统一物的分裂、斗争、生动、无常、活跃、变化、不和平、不平衡、不调和、不妥协、甚至冲突、对抗、或战争,这是绝对的。同一、统一、静、死等等相对的矛盾状态,包含于绝对的斗争的矛盾状态中,因为斗争贯彻于过程的始终,贯彻于一切过程之中,所以成其为绝对的东西。不懂这个道理,就是形而上学、机械论,实质上拒绝了辩证法。

国际间的和平条约是相对的,国际间的斗争是绝对的。阶级间的统一战线是相对的,阶级间的斗争是绝对的。党内思想上的一致是相对的,党内思想上的斗争是绝对的。自然现象中的平衡、凝聚、吸引、化合等等是相对的,而不平衡、不凝聚、排斥、分解等等是绝对的。当着过程在和平条约、统一战线、团结一致、平衡、凝聚、吸引、化合等等状态之时,矛盾与斗争也仍然存在着,不过没有取激化的形式,并不是没有了矛盾,停止了斗争。由于斗争,不绝地破坏一个相对状态而转到另一个相对状态去,破坏一种过程而转到另一种过程去,这种无所不在的斗争性,就是矛盾的绝对性。

前面我们说,两个相反的东西中间有同一性,所以二者能够共处于一个统一体中,又能够互相转化,这是说的条件性,即谓在一定条件之下,矛盾的东西能够统一起来,又能够互相转化;无此一定条件,就不能成为矛盾,不能共居,也不能转化。由于一定的条件才构成了矛盾的同一性,所以说同一性是有条件的、相对的。这里我们又说,斗争贯彻于过程的始终,并使一过程向着他过程转化,斗争无所不在,所以说斗争性是无条件的、绝对的。

有条件的相对的同一性与无条件的绝对的斗争性相结合,就构成了一切事物的矛盾运动。

为明了这一论点,下面再举出生死关系及劳资关系以为例。

有机体中旧细胞的死亡,是新细胞产生的前提,也是生活过程的前提。这里生死两矛盾方面互相统一于有机体中,而又互相转变着。生细胞变为死细胞,死细胞变为生细胞(生细胞从死细胞中脱胎出来)。但这种生死的统一,生死的共处于有机体中,是有条件的、暂时的、相对的。而生死的不并存,互相排斥、斗争、否定、转化、则始终如此,它是无条件的、永久的、绝对的。有机体中生的原素总是不绝地战胜死的原素,并且统治着死的原素,表示了斗争的绝对性。生,在一定条件之下转化为死;死,又在一定条件之下转化为生。这种条件使生死有同一性,能够互相转化。由于生死二矛盾物互相斗争,使得生必然转化为死,死必然转化为生。这种必然性,是无条件的、绝对的。由此看来,必须在一定的发展阶段上,必须有一定的温度环境等等条件,生死才能互相转变,互相有同一性。这是一个问题。所谓使生或死都带暂时性、相对性、就是条件不变也不能长生或长死,原因在于两者的斗争、否定、互相排除,这种情形是永久的、绝对的。这又是一个问题。

无产阶级替资产阶级创造剩余价值,资产阶级剥削无产阶级的劳动力,这是一个决定资本主义生存的统一的过程,劳资双方互为存在的条件。然而这种条件有一定的限度,就是要资本主义发展还在一定限度之内,过此限度,统一的过程发生破裂,出现了社会主义的革命。这种破裂是突然发生的,但又不是突然发生的,是从两阶级存在的一天起就开始准备着,双方的斗争是不断的,由此准备了突变。由此看来,两阶级的共存,由于一定的条件而保存着,这种一定条件下的共存,造成了两阶级的统一或同一性。两阶级又在一定的条件之下互相转化,使得剥削者被剥削、被剥削者变为剥削剥削者,资本主义社会变为社会主义社会,二矛盾物有一定条件下的同一性。这是一个问题。双方总是斗争,不但在统一体中是斗争的,尤其是革命的斗争,这种不可避免的状态是无条件的、绝对的、必然的。这又是一个问题。

在同一性中存在着斗争性。拿着列宁的话来说,叫做在“相对中存在着绝对”\mnote{40}。因此,矛盾的统一本身,也就是矛盾的斗争之一种表现或一种因素.这就是我们对于这个问题的结论。

根据这种结论,所谓阶级调和论与思想统一论是否还有立足之余地,也就不言而喻了。国际的阶级调和论,形成了各国工人运动中的机会主义派。他们没有别的作用,单单充当了资产阶级的走狗。中国也有所谓阶级调和论,却从资产阶级改良主义的口中唱出来,他们的目的也不为别的,在于专们欺骗无产阶级,使之永为资产阶级的奴隶。所谓思想统一的滥调,则由若干直接间接依靠官场吃饭、“文人学者”们吹打出来,目的无非在抹煞真理,阻碍革命的前进。真的科学的理论不是这些调儿,而是唯物辩证法的矛盾统一律。

\subsection{(七)对抗在矛盾中的地位}

在矛盾斗争性问题中,包含着对抗是什么的问题。我们回答道:一切过程是自始至终存在着矛盾的,矛盾的双方之间也是自始至终存在着斗争的。对抗是斗争的一种形式,不是一切矛盾都有,而是某些矛盾在其发展过程中到达了采取外部物体力量的形式而互相冲突时,矛盾的斗争便表现为对抗,对抗是矛盾斗争的特殊表现。

例如剥削阶级同被剥削阶级之间的矛盾,无论在奴隶社会也好,封建社会也好,资本主义社会也好,互相矛盾的两阶级长期并存于一个社会中,它们互相斗争着,但要待两阶级的矛盾发展到一定阶段时,双方才取外部对抗的形式,此时社会破裂,革命的战争就出现了。

炸弹的爆炸,小鸡的出卵,动物的脱胎,都是矛盾物共居于一个统一体中,待至一定时机,才取冲突、破局、决裂的形式。

各国之间的和平共居,乃至社会主义国家同资本主义国家也是一样,矛盾与斗争无日不存在,但战争却要在一定发展阶段上才能出现。

苏联的新经济政策容许资本主义成分的相当发展,列宁认为那时有在无产阶级专政下利用国家资本主义的可能。就是说,利用某些资产阶级成分发展生产力,同时使之受苏维埃法律的支配,并随时限制和排斥他们,这时社会主义与资本主义两矛盾体共处于社会主义社会之内,互相斗争,又互相联系。待到消灭富农及消灭资本主义残余的任务提出之后,两种成分的并存就成为不可能,而生死斗争的外部对抗形式就发生了。

国共两党第一次统一战线的情况也是如此。

然而许多过程、现象、事物中的矛盾是不发展成为对抗的。

例如共产党内正确思想与错误思想的矛盾,文化上先进与落后的矛盾,经济上城市与乡村的矛盾,生产力与生产关系的矛盾,生产与消费的矛盾,交换价值与使用价值的矛盾,各种技术分工的矛盾,阶级关系中工农的矛盾,自然界中的生与死、遗传与变异、寒与暑、昼与夜等类的矛盾,者没有对抗形态的存在。

布哈林把矛盾和对抗同一看待,因此,认为在完成了的社会主义社会中,对抗没有了,矛盾也没有了。列宁回答道:“这是极端不正确的,对抗和矛盾断然不同。在社会主义下,对抗消灭了,矛盾存在着。”\mnote{41}布哈林是否认事物发展由于内部矛盾的推动之均衡论者,认为社会主义下没有矛盾,社会也可发展。

托洛茨基从另一极端出发,也把矛盾和对抗同一看待。因此,认为在社会主义下,工农之间不但存在着矛盾,而且将发展到对抗,如同劳资间的矛盾一样,只有用革命的方法才能解决。然而苏联却用农业社会化的方法解决了,并且是在一国社会主义的情况下解决了,无须如托派所谓要待至国际革命之时。

布哈林把矛盾降低到消灭,托派把矛盾提升到对抗,右倾与左倾的两极端,都不了解矛盾的问题。

解决一般矛盾的方法与解决对抗的方法是根本不同的,这是矛盾特殊性与解决矛盾的方法之特殊性应该有具体认识的问题。凡对抗都包含矛盾性,但凡矛盾不一定都取对抗的形态,总的区别就在这里。

矛盾统一律是宇宙的根本法则,也是思想方法的根本法则。列宁称之为辩证法的核心。它是与形而上学的发展观相反的,它是与形式论理学的绝对的同一律相反的。矛盾存在于一切客观与主观事物的过程中,矛盾贯彻于一切过程的始终,这是矛盾的普遍性、绝对性。矛盾及矛盾的侧面各有其特点,人心之不同如其面,矛盾之不同如其形,这是矛盾之特殊性、相对性。矛盾着的东西依一定的条件有同一性,因此能够共居于一个统一体中,又能够互相转化到相反方面、这又是矛盾的特殊性、相对性。然而矛盾的斗争则是不绝的,不管在其共居时或其转化时,都有斗争的存在,尤其是表现在矛盾的转化时,这又是矛盾的普遍性、绝对性。研究矛盾的特殊性相对性时,要注意矛盾及矛盾方面之主要与非主要的区别;研究矛盾的斗争性时要注意矛盾的一般斗争形式与特殊斗争形式——即矛盾发展为对抗这种区别。这就是我们对于矛盾统一律的总结论。


\begin{maonote}
\mnitem{1}此处省略的为“蒋介石”。
\mnitem{2}出自《谈谈辩证法问题》。参见《列宁全集》中文第2版,第55卷,第311页:“人的认识不是直线(也就是说,不是沿着直线进行的),而是无限地近似于一串圆圈、近似于螺旋的曲线。这一曲线的任何一个片断、碎片、小段都能被变成(被片面地变成)独立的完整的直线,而这条直线能把人们(如果只见树木不见森林的话)引到泥坑里去,引到僧侣主义那里去(在那里统治阶级的阶级利益就会把它巩固起来)。直线性和片面性,死板和僵化,主观主义和主观盲目性就是唯心主义的认识论根源。”
\mnitem{3}参见《黑格尔〈逻辑学〉一书摘要》中《第2版序言》:“逻辑不是关于思维的外在形式的学说,而是关于“一切物质的、自然的和精神的事物”的发展规律的学说,即关于世界的全部具体内容的以及对它的认识的发展规律的学说,即对世界的认识的历史的总计、总和、结论。”(《列宁全集》中文第2版,第55卷,第77页)
\mnitem{4}出自《自然辩证法》中《辩证法作为科学·辩证法》。参见《马克思恩格斯选集》中文第3版,第3卷,第907页:“自然界、社会和思维的发展的一个一般规律”。
\mnitem{5}出自《卡尔·马克思(传略和马克思主义概述)》中《辩证法》。参见《列宁全集》中文第2版,第26卷,第55页:“最全面、最富有内容、最深刻的发展学说。”
\mnitem{6}出自《卡尔·马克思(传略和马克思主义概述)》中《辩证法》。参见《列宁全集》中文第2版,第26卷,第55页:“把自然界和社会的实际发展过程(往往伴有飞跃、剧变、革命)弄得残缺不全。”
\mnitem{7}出自《黑格尔辩证法(逻辑学)的纲要》。参见《列宁全集》中文第2版,第55卷,第290页:“不必要三个词:它们是同一个东西。”
\mnitem{8}出自《卡尔·马克思〈政治经济学批判。第一分册〉》第2部分。参见《马克思恩格斯选集》中文第3版,第2卷,第12页:“黑格尔的思维方式不同于所有其他哲学家的地方,就是他的思维方式有巨大的历史感作基础。形式尽管是那么抽象和唯心,他的思想发展却总是与世界历史的发展平行着,而后者按他的本意只是前者的验证。”
\mnitem{9}出自《卡尔·马克思〈政治经济学批判。第一分册〉》第2部分。参见《马克思恩格斯选集》中文第3版,第2卷,第13页:“历史常常是跳跃式地和曲折地前进的,如果必须处处跟随着它,那就势必不仅会注意许多无关紧要的材料,而且也会常常打断思想进程;并且,写经济学史又不能撇开资产阶级社会的历史,这就会使工作漫无止境,因为一切准备工作都还没有做。因此,逻辑的方式是唯一适用的方式。但是,实际上这种方式无非是历史的方式,不过摆脱了历史的形式以及起扰乱作用的偶然性而已。”
\mnitem{10}出自《黑格尔〈逻辑学〉一书摘要》中《第2版序言》。参见《列宁全集》中文第2版,第55卷,第75页:“逻辑的范畴是‘外部存在和活动的’‘无数’‘细节’的简化”。
\mnitem{11}出自《黑格尔〈逻辑学〉一书摘要》中《第2版序言》。参见《列宁全集》中文第2版,第55卷,第78页:“范畴是区分过程中的梯级,是帮助我们认识和掌握自然现象之网的网上纽结。”
\mnitem{12}出自《黑格尔〈逻辑学〉一书摘要》中《主观逻辑或概念论·第2篇 客观性》。参见《列宁全集》中文第2版,第55卷,第160页:“人的实践活动必须亿万次地使人的意识去重复不同的逻辑的式,以便这些式能够获得公理的意义。”
\mnitem{13}出自《黑格尔〈逻辑学〉一书摘要》中《主观逻辑或概念论·第3篇 观念》。参见《列宁全集》中文第2版,第55卷,第186页:“人的实践经过亿万次的重复,在人的意识中以逻辑的式固定下来。这些式正是(而且只是)由于亿万次的重复才有着先入之见的巩固性和公理的性质。”
\mnitem{14}原文如此,恐为“论理”(即“逻辑”)误。
\mnitem{15}出自《自然辩证法》中《物质的运动形式以及各门科学的联系》。参见《马克思恩格斯选集》中文第3版,第3卷,第943页:“每一门科学都是分析某一个别的运动形式或一系列互相关联和互相转化的运动形式的,因此,科学分类就是这些运动形式本身依其内在序列所进行的分类、排序,科学分类的重要性也正在于此。”
\mnitem{16}出自《自然辩证法》中《辩证法作为科学·认识》。参见《马克思恩格斯选集》中文第3版,第3卷,第939页:“‘物质’和‘运动’这样的词无非是简称,我们就用这种简称把可感知的许多不同的事物依照其共同的属性概括起来。”
\mnitem{17}出自《唯物主义和经验批判主义》第3章第5节。参见《列宁全集》中文第2版,第18卷,第180页:“世界上除了运动着的物质,什么也没有,而运动着的物质只能在空间和时间中运动。人类的时空观念是相对的,但绝对真理是由这些相对的观念构成的;这些相对的观念在发展中走向绝对真理,接近绝对真理。正如关于物质的构造和运动形式的科学知识的可变性并没有推翻外部世界的客观实在性一样,人类的时空观念的可变性也没有推翻空间和时间的客观实在性。”
\mnitem{18}出自《黑格尔〈逻辑学〉一书摘要》中《主观逻辑或概念论·第3篇 观念》。参见《列宁全集》中文第2版,第55卷,第165页:“矛盾的发生和解决的永恒过程中”。
\mnitem{19}出自《黑格尔〈逻辑学〉一书摘要》中《主观逻辑或概念论·第1篇 主观性》。参见《列宁全集》中文第2版,第55卷,第152页:“认识是人对自然界的反映。但是,这并不是简单的、直接的、完整的反映,而是一系列的抽象过程,即概念、规律等等的构成、形成过程”。
\mnitem{20}即波格丹诺夫。
\mnitem{21}出自《自然辩证法》中《辩证法作为科学·认识》。参见《马克思恩格斯选集》中文第3版,第3卷,第938页:“对自然界的一切真实的认识,都是对永恒的东西、对无限的东西的认识,因而本质上是绝对的。”
\mnitem{22}出自《黑格尔〈逻辑学〉一书摘要》中《主观逻辑或概念论·第3篇 观念》。参见《列宁全集》中文第2版,第55卷,第183页:“实践高于(理论的)认识,因为它不仅具有普遍性的品格,而且还具有直接现实性的品格。”
\mnitem{23}出自《黑格尔〈逻辑学〉一书摘要》中《主观逻辑或概念论·概念总论》。参见《列宁全集》中文第2版,第55卷,第142页:“物质的抽象,自然规律的抽象,价值的抽象以及其它等等,一句话,一切科学的(正确的、郑重的、非瞎说的)抽象,都更深刻、更正确、更完全地反映着自然。”
\mnitem{24}参见前注。
\mnitem{25}出自《俄国社会民主党人的任务》以及《怎么办?》第1章第4节。分别参见《列宁全集》中文第2版,第2卷,第443页;第6卷,第23页:“没有革命的理论,就不会有革命的运动。”
\mnitem{26}出自《论列宁主义基础》第3部分《理论》。参见《斯大林选集》上卷,人民出版社,1979年,第199—200页:“离开革命实践的理论是空洞的理论,而不以革命理论为指南的实践是盲目的实践。”
\mnitem{27}原文如此。
\mnitem{28}出自《自然辩证法》中《辩证法作为科学·辩证法》。参见《马克思恩格斯选集》中文第3版,第3卷,第901页:“辩证法的规律是从自然界的历史和人类社会的历史中抽象出来的。辩证法的规律无非是历史发展的这两个方面和思维本身的最一般的规律。它们实质上可归结为下面三个规律:量转化为质和质转化为量的规律;对立的相互渗透的规律;否定的否定的规律。”
\mnitem{29}出自《黑格尔(哲学史讲演录)一书摘要》。参见《列宁全集》中文第2版,第55卷,第213页:“就本来的意义讲,辩证法是研究对象的本质自身中的矛盾。”
\mnitem{30}出自《反杜林论》第1编第12节《辩证法。量和质》。参见《马克思恩格斯选集》中文第3版,第3卷,第498页:“运动本身就是矛盾。”
\mnitem{31}出自《谈谈辩证法问题》。参见《列宁全集》中文第2版,第55卷,第306页:“承认(发现)自然界的(也包括精神的和社会的一切现象和过程具有矛盾着的、相互排斥的、对立的倾向。”
\mnitem{32}即波里斯·伊萨科维奇·哥列夫(戈尔德曼)(Борис Исаакович Горев, 1874—1937),原为孟什维克,苏联历史学家和哲学家。
\mnitem{33}出自《谈谈辩证法问题》。参见《列宁全集》中文第2版,第55卷,第307页:“这应该是一般辩证法的……叙述(以及研究)方法。”
\mnitem{34}即普列汉诺夫。——录入者注
\mnitem{35}出自《俄国社会民主党人的任务》以及《怎么办?》第1章第4节。分别参见《列宁全集》中文第2版,第2卷,第443页;第6卷,第23页:“没有革命的理论,就不会有革命的运动。”
\mnitem{36}出自列宁《黑格尔〈逻辑学〉一书摘要》。参见《列宁全集》中文第2版,第55卷,第90页:“辩证法是一种学说,它研究对立面怎样才能够同一,是怎样(怎样成为)同一的——在什么条件下它们是相互转化而同一的,——为什么人的头脑不应该把这些对立面看作僵死的、凝固的东西,而应该看作活生生的、有条件的、活动的、彼此转化的东西。”
\mnitem{37}参见前注。
\mnitem{38}原文如此,即滑铁卢。
\mnitem{39}出自列宁《谈谈辩证法问题》。参见《列宁全集》中文第2版,第55卷,第306页:“对立面的统一(一致、同一、均势)是有条件的、暂时的、易逝的、相对的。相互排斥的对立面的斗争是绝对的,正如发展、运动是绝对的一样。”
\mnitem{40}出自《谈谈辩证法问题》。参见《列宁全集》中文第2版,第55卷,第307页:“相对中有绝对”。
\mnitem{41}出自《在尼·布哈林〈过渡时期经济学〉一书上作的批注和评论》。参见《列宁全集》中文第2版,第60卷,第281—282页:“极不确切。对抗和矛盾完全不是一回事。在社会主义下,对抗将会消失,矛盾仍将存在。”。
\end{maonote}
