
\title{学校教育革命的方向——要和工农兵结合}
\date{一九六八年七、八、九月}
\maketitle


\date{一九六八年七月二十二日}
\section*{(一)\mnote{1}}

大学还是要办的,我这里主要说的是理工科大学还要办,但学制要缩短,教育要革命,要无产阶级政治挂帅,走上海机床厂从工人中培养技术人员的道路。要从有实践经验的工人农民中间选拔学生,到学校学几年以后,又回到生产实践中去。

\date{一九六八年八月二十二日}
\section*{(二)\mnote{2}}

工人宣传队进入教育阵地,这是一件惊天动地的大事。

自古以来,学校这个地方,就是为剥削阶级及其子女所垄断。解放以后,好了一些,但基本上还是被资产阶级知识分子所垄断。从这些学校出来的学生,有些人由于各种原因(这些原因大概是:或本人比较好,或教师比较好,或受了家庭、亲戚、朋友的影响,而主要的是受社会的影响)能同工农兵结合,为工农兵服务,有一些人则不能。在无产阶级专政的国家内,存在着资产阶级与无产阶级争夺领导权的严重现象。在这场无产阶级文化大革命中,红卫兵小将奋起造党内一小撮走资派的反,学校中的资产阶级反动势力暂时地遭到了一次沉重的打击。但随后不久,有些人又暗地活动起来,挑动群众斗群众,破坏文化大革命,破坏斗、批、改,破坏大联合和革命的三结合,破坏清理阶级队伍的工作和整党的工作。这种情况引起了广大群众的不满。现实的状况告诉我们:在这种情况下,单靠学生、知识分子不能完成教育战线的斗、批、改及其他一系列任务,必须有工人、解放军参加,必须有工人阶级的坚强领导。

实现无产阶级教育革命,必须有工人阶级领导,必须有工人群众参加,配合解放军战士,同学校的学生、教员、工人中决心把无产阶级教育革命进行到底的积极分子实行革命的三结合。工人宣传队要在学校中长期留下去,参加学校中全部斗、批、改任务,并且永远领导学校。在农村,则应由工人阶级的最可靠的同盟者——贫下中农管理学校。

\date{一九六八年八月三十一日}
\section*{(三)\mnote{3}}

此件是上海市革命委员会八月二十九日送给中共中央、中央文革的调查报告,现在本刊发表。我们请全国各大、中、小工业城市的革命委员会予以注意,对你们那里的工程技术人员的情况给以调查,陆续送给中央,我们将择要予以发表。从根本上说来,走从工、农、兵及其后代中选拔工程技术人员及其他意识形态工作人员(教授、教员、科学家、新闻记者、文学家、艺术家和马克思主义理论家)的路,是已经确定的了。同时,对过去和现在的大专院校毕业生和在学学生,党和工人阶级、贫下中农、人民解放军有责任,热情地、严肃地帮助他们中间一些至今还不懂得或还未下决心同工农兵结合、为工农兵服务的人们,逐步地改变过来。对于已经结合或者愿意结合的人们要加以鼓励,这方面的动人事迹及其变化过程也要加以调查和择要发表。顽固的走资派或其他被广大群众认为不好的人,只是我国人口中的一小撮。对于这种人,也要从教育入手,以期使他们虽顽固而能转化,虽不好而能变好。总之是要给出路。不给出路的政策,不是无产阶级的政策。这些是我党长期以来一贯的传统政策,应向一切人讲清楚。文化大革命是一场严重的阶级斗争,也是一次伟大的教育运动。

\date{一九六八年九月八日}
\section*{(四)\mnote{4}}

这是上海市的又一个调查报告,现发表,供参考。全国各大、中、小工业城市所属各工厂的工程技术干部情况如何,各理工科高等、中等学校教育革命的情况如何,希望各地革命委员组织一些人做些典型的调查,报告中央,本刊将择要予以发表。这里提出一个问题,就是对过去大量的高等及中等学校毕业生早已从事工作及现正从事工作的人们,要注意对他们进行再教育,使他们与工农结合起来。其中必有结合得好的并有所发明创造的,应予以报导,以资鼓励。实在不行的,即所谓顽固不化的走资派及资产阶级技术权威,民愤很大需要打倒的,只是极少数。就是对于这些人,也要给出路,不给出路的政策,不是无产阶级的政策。上述各项政策,无论对于文科、理科新旧知识分子,都应是如此。

\date{一九六八年九月二十二日}
\section*{(五)\mnote{5}}

从旧学校培养的学生,多数或大多数是能够同工农兵结合的,有些人并有所发明、创造,不过要在正确的路线之下,由工农兵给他以再教育,彻底改变旧思想。这样的知识分子,工农兵是欢迎的。

\begin{maonote}
\mnitem{1}这是毛泽东同志对《从上海机床厂看培养工程技术人员的道路》报告的按语,“七·二一指示”,一九六八年七月二十二日,《人民日报》发布了这个报告,全文如下:

一、无产阶级文化大革命带来的深刻变化

上海机床厂是著名的生产精密磨床的大厂。全厂拥有工程技术人员六百多人。这支技术队伍由三方面的人员组成:从工人中涌现的技术人员约占百分之四十五;解放后大专院校输送进厂的约占百分之五十;其余是解放前留下来的老技术人员。无产阶级文化大革命的急风暴雨,使这个厂的技术人员队伍,发生了深刻的变化。

这个伟大的革命变化主要表现在:

第一,无产阶级革命派真正掌握了全厂的领导权,包括技术大权。过去控制工厂技术领导权的资产阶级反动技术“权威”,被赶下了台。许多工人出身的技术人员、革命的青年技术人员和革命干部,成了科学研究和技术设计的主人。他们是无产阶级革命派战士,对毛主席、对共产党有深厚的阶级感情。这支过去受压抑的革命的技术队伍,不断显示出他们的创造智慧和技术才能。他们怀着对毛主席的无产阶级革命路线的无限忠诚,在技术上不断攀登新的高峰。仅今年上半年,他们就试制成功了十种新型精密磨床,其中四种达到国际先进水平。速度之快,质量之高,是这个工厂历史上前所未有的。

第二,中国赫鲁晓夫的反革命修正主义技术路线和反动的资产阶级世界观,受到了深刻的批判。资产阶级的反动技术“权威”,政治上臭了,技术上的纸老虎的原形,也暴露无遗。过去走资派竭力把反动“权威”捧为青年技术人员的学习“偶像”,要他们“向专家看齐”,“为当工程师而奋斗”。现在,许多青年技术人员思想面貌发生了显著变化,认识到名利思想是“修”字的根,资产阶级的头衔不能争。磨床研究所的不少研究人员,过去把技术资料记在本本上,作为个人的“小仓库”;现在都自觉拿出来,汇集成册,分发给大家参考。全厂技术人员纷纷主动下车间,和工人一起劳动,一起研究和改进设计。老年技术人员下车间,也注意放下架子,虚心向工人学习。

第三,工人和技术人员的关系变了。过去,这个厂的几个走资派和反动“权威”曾提出所谓“一对一”(即一个工人为一个技术人员服务),他们的所谓“结合”是“工程师动嘴,工人动手”,“工程师出主意,工人照着干”,还是多少年来“劳心者治人,劳力者治于人”的那一套。他们还制造了“工人和技术人员必须互相制约”、“造成对立面”的反动理论,建立了管、卡、压工人的一套规章制度。一本《生产工人手册》,就规定了一百七十多条,要工人背熟牢记,条条照办。凡此种种,进一步加深了工人和技术人员之间的鸿沟。文化大革命中,这个厂实行了工人、革命技术人员、革命干部的“三结合”,普通的工人参加了设计,技术人员也在第一线参加了实际操作,把理论和实践紧密结合起来,双方的关系有很大改善。

二、培养工程技术人员的道路

上海机床厂青年技术员(包括三十五岁左右的人)来源于两方面:大专院校历届毕业生(约三百五十人,其中百分之十是研究生和留学生)和工人中提拔的技术人员(约二百五十人,其中少数中间曾派往中等专科学校进修几年)。实践证明,后一部分比前一部分强。一般说,前者落后思想较多,实际工作能力较差;后者思想较前进,实际工作能力较强。目前,工人出身的技术人员,绝大多数是工艺方面的技术骨干,约有十分之一的人能独立设计高、精、尖的新产品。在今年上半年试制成功的十种新型精密磨床中,有六种就是由工人出身的技术人员担任主任设计师的。

从工人中选拔技术人员,这是一条培养无产阶级工程技术人员的道路。

从两个年龄相仿而经历不同的技术员身上,可以看到一个鲜明的对比:

一个是上海某大学的毕业生,毕业后又专门学了一年外语,再去外国留学四年,得了一个“副博士”的学位,一九六二年开始到磨床研究所试验室担任技术员。象这样一个在学校里读了二十多年书的人,过去由于理论脱离实际,又没有很好地同工人结合,所以长时期在科学研究方面没有作出突出的成就。

一个是工人,他十四岁当学徒,十八岁被保送到上海机器制造学校学习四年,一九五七年开始在磨床研究所担任技术员。今年四月,由他担任主任设计师,试制成功了一台具有国际先进水平、为我国工业技术的发展所急需的大型平面磨床,填补了我国精密磨床方面的一个空白点。

文化大革命以前,党内一小撮走资派和反动技术“权威”,对工人走上设计舞台进行疯狂的压制。一九五八年前后,有一批工人担任了技术员,但是,厂里的反动“权威”以各种借口,陆续地把不少人调离设计部门。尽管如此,工人出身的技术人员还是冲破层层阻力,表现了自己惊人的创造智慧。据统计:一九五八年以来,这个厂自行设计试制成功的新产品中,由工人出身的技术人员以及青年技术人员和工人相结合试制成功的,在一九五八年约占百分之六十,一九五九年约占百分之七十,一九六〇年约占百分之八十,一九六〇年以后,特别是无产阶级文化大革命以来,几乎所有新产品都是他们设计试制成功的。其中不少是具有国际先进水平的新产品,如镜面磨削万能外圆磨床,高中心外圆磨床等重点产品,都是工人出身的技术人员设计试制成功的。

有些大专院校毕业的青年技术人员,逐步摆脱修正主义教育路线的影响,放下架子和工人结合,经过一段时期的实践,在设计和新产品试制上也作出了比较显著的贡献。比如一个一九六四年的大学毕业生,刚进厂时,成天捧着一本外国螺纹磨床的书(不是说不要读外国书),从理论到理论,几年来,在工作上没有什么创造。文化大革命中,他提高了阶级觉悟和两条路线斗争觉悟,坚决走和工人结合的道路,今年初,就和两个工人技术人员和一个老师傅一起,试制成功了一种磨床的重要电器设备。

三、为什么工人出身的技术人员成长快、贡献大呢?

最重要的一条是工人出身的技术人员对毛主席、对党有着深厚的无产阶级感情,他们在向科学技术进军的道路上,不为名,不为利,不畏艰险,不怕困难,不达目的誓不罢休。他们牢记毛主席的教导,时刻想到与帝修反争速度,争质量,并且处处考虑为国家节约,为工人操作方便。可是,有些受了修正主义教育路线毒害的青年知识分子,长期脱离劳动,脱离工人,追求资产阶级名利,结果一事无成。有一个技术人员,想成名成家,一鸣惊人,十多年来先后提出六十多个课题,搞一个丢一个,浪费了国家大量资金,一个也没有搞成。一个一九五六年的大学毕业生,起初为了自己出名,关门搞磨头试验,先后报废了三十多个“磨头”,后来向有经验的老工人请教,在老工人的帮助下终于搞成功了。他深有体会地说:“闭门搞磨头,吃尽了苦头;同工人结合搞磨头,尝到了甜头;磨头磨头,归根结底,先要磨炼自己的脑头。”

工人出身的技术人员同名利思想极为严重的旧资产阶级知识分子对比,就更加鲜明了。有一个资产阶级“专家”设计一台磨床,前后搞了八年,浪费了国家大量资金,始终没有搞成功,而他自己却捞到了不少所谓“数据”,作为他争夺名利的资本。工人们说:这种人对我们的新社会那里有一点感情呢?

毛主席说:“最聪明、最有才能的,是最有实践经验的战士。”工人出身的技术人员,在长期的劳动中,积累了丰富的实践经验,他们经过几年业余学校的学习,或者读了几年业余技术专科学校,理论和实践紧密结合起来,出现了一个飞跃,很快就能胜任科学研究和独立设计的任务。这是他们能够迅速成长的一个十分重要的原因。他们参加学习,叫作“带着问题读书”,因此学得进,懂得快,用得上。有一位工人出身的技术人员,运用丰富的实践经验,解决了一个产品的复杂的加工工艺问题,他一边实践,一边学习金属切削的原理,很快地把实践经验上升为理论,使他在金属切削加工工艺方面,有独到的见解。

从大专院校出来的技术人员,当他们未和工人结合之前,缺乏实践经验,书本知识往往和实际脱节,因此很难作出什么成就来。有一次,有几个从大专院校出来的技术人员,因为缺乏实践经验,设计了一台内螺纹磨床,工人们按照图纸的零件加工,结果根本无法装配。后来还是有丰富实践经验的工人,把某些零件进行了再加工,才装配了起来。

敢想、敢做、敢闯的革命精神和严格的科学态度相结合,是工程技术人员攀登科学技术高峰的一个极其重要的条件。而要做到这一点,又是和人的世界观和实践经验息息相关的。许多工人出身的技术人员,由于没有个人名利的精神枷锁的束缚,加上有丰富的实践经验,因此敢于破除迷信,破除不适用的框框,最少保守思想。以最近试制成功的具有国际水平的一台精密磨床为例,由于工人出身的技术人员大胆突破了沿袭已久的框框,试制周期从一年半缩短为半年,精度提高四级,零件和重量分别减少三分之一,造价相当于进口的百分之十五点五。而有些学校出来的技术人员,不注意思想改造,往往容易考虑个人得失,怕丢面子,失架子,同时,他们条条框框也比较多,因而不容易破除迷信,创造出新技术来。他们自己也说,“书读得越多,束缚得越紧,结果就没有闯劲了。”

在上海机床厂,如果以大学生和中等技术学校的毕业生相比较,工人们还比较地欢迎中等技术学校的学生。因为中专生书本知识虽然少一些,但他们的架子小些,实践经验多些,洋框框少些。不少中专生进步比大学生快得多。比如正在进行的两条高级的自动流水线的设计,就是由两个一九五六年毕业的中专生负责的。

四、从工厂看教育革命的方向

分析一下上海机床厂不同类型的工程技术人员的状况和他们所走过的道路,我们也可以看出一个教育革命的方向问题。

这个工厂的老工人和许多青年技术人员从实践中更加深刻地体会到毛主席关于“资产阶级知识分子统治我们学校的现象,再也不能继续下去了”这一教导无比英明正确。他们感到按照毛主席的教育思想进行无产阶级教育革命,已经是刻不容缓的一件大事。毛主席关于教育革命的一系列指示,已经给我们指明前进的方向。现在,是坚定不移地、老老实实地按照毛主席的教导去做的问题。

这个厂的工人和技术人员根据毛主席的教育思想和工厂的实际情况,对教育革命提出了一些看法和设想:

第一,学校培养的必须是毛主席所指出的“有社会主义觉悟的有文化的劳动者”,决不能象修正主义教育路线那样,培养那种脱离无产阶级政治、脱离工农群众、脱离生产实践的“三脱离”的“精神贵族”。这是关系到出不出修正主义的根本问题。上海机床厂的同志认为,过去大学毕业生分配到工厂、农村,就当干部,是不合理的。青年学生和工农结合,参加生产劳动,是改造世界观和学到实际技术知识的重要途径。因此,他们建议,大学毕业生应当先到工厂、农村,参加劳动,当一个普通劳动者,在工人、农民那里取得“合格证书”,然后根据实际斗争的需要,有些可以参加技术工作,但也还要有一定时间参加劳动。有的则继续当工人、农民。

第二,学校教育一定要与生产劳动相结合。毛主席教导说:“从战争学习战争——这是我们的主要方法。”从上海机床厂有些技术人员的情况看,旧教育制度的一个严重弊病,就是理论脱离实际,大搞烦琐哲学,学生钻到书堆里,越读越蠢。只有接触实践,对于理论才能掌握得快,理解得深,运用得活。这个厂的工人和技术人员提出:学校要由有经验的工人去当教师,让工人登上讲台。有些课程就可以在车间里由工人讲授。有一个青年技术人员,大学一毕业就进了研究所的门,整天啃书本,钻理论,学外文,脱离实际,自己也感到越来越空虚。文化大革命初期,他到机床厂拜了一些有丰富经验的工人师傅,自己参加了技术操作实践,情况就变了。最近他和工人一起,在镜面磨削方面,作出了一个很有意义的创造。他对于必须有工人做自己的教师这一点,感受特别深切。

第三,关于工程技术人员的来源问题。他们认为,除了继续从工人队伍中提拔技术人员外,应该由基层选拔政治思想好的,具有两三年或四、五年劳动实践经验的初、高中毕业生进入大专院校学习。这样做,现在是完全有条件的。以上海机床厂为例,大部分工人都具有初中以上的文化水平。挑选这样的青年进入大专院校的好处是:第一,他们政治思想基础比较好。第二,是有一定的实际工作能力,有生产劳动的经验。第三,一个初、高中毕业生,经过几年劳动,大约在二十岁左右,再经过几年学习,二十三四岁毕业就能独立工作,而目前一个大学毕业生到工作岗位后,一般要经过二三年实践,才逐步能独立工作,因此选拔有实践经验的知识青年到大学培养,这是符合多快好省的原则的。

第四,关于改造和提高工厂现有的技术队伍的问题。他们指出:从学校出来的大量技术人员,长期受过修正主义教育路线和修正主义办企业路线的毒害。还有一批过去留下来的老技术人员,虽然他们当中有些人是爱国的,是积极工作的,是不反党反社会主义的,是不里通外国的,但他们在世界观和作风上,存在很多问题。工厂应该高举毛泽东思想的革命批判大旗,根据《中共中央关于无产阶级文化大革命的决定》中规定的政策,组织他们积极参加革命大批判,彻底批判中国赫鲁晓夫鼓吹的“专家治厂”、“技术第一”等谬论和“爬行哲学”、“洋奴哲学”,彻底批判资产阶级名利思想。同时,还应该组织他们分期分批去当工人,或者让他们有更多的时间到车间去劳动,帮助他们走上和工人相结合的道路,走上理论和实践相结合的道路。
\mnitem{2}一九六八年八月二十五日出版的《红旗》杂志第二期发表《工人阶级必须领导一切》,这两段是毛泽东在审阅时修改的文字。
\mnitem{3}这是毛泽东为准备发表调查报告《上海工人技术人员在斗争中成长》写的《红旗》杂志编者按语。
\mnitem{4}这是毛泽东在一九六八年九月八日的《红旗》杂志上一份调研报告上的按语,一九六八年九月三日,上海市革命委员会继八月二十九日上送调查报告《上海工人技术人员在斗争中成长》之后,又上送一个调查报告,说上海机械学院的前身,上海机器制造学校创办时,第一届招收的二千余名学生全是从工人、农民和一部分农村基层干部中选拔的。这批有实践经验的学生,学习目的性明确,阶级觉悟高,学以致用,冲击了旧的教育制度、教学内容和教学方法。最近部分当年从这个学校毕业的上海机床厂工程技术人员以及这个学校附属工厂的工人和广大师生总结建校以来正反两方面的经验以后,对理工科大学的教育革命提出了一些看法和设想:

一、理工科大学要走上海机床厂的道路,必须解决由哪个阶级掌握领导权的问题。

二、理工科大学的学制以两至三年为宜。

三、理工科大学还要担负起办好业余技术教育的任务。

四、建立一支无产阶级的教师队伍。

今后的教师队伍,应当采取有高度无产阶级政治觉悟的有实践经验的工人、工农学生和革命知识分子三结合的形式。社会上一大批在实践有发明创造的工人、技术人员,要有计划地定期地深入学校讲课。学生们也都有实践经验,可以走上讲台,互相交流。现有的教师应该分期分批地到工农中去,走与工农相结合的道路。
\mnitem{5}一九六八年九月二十二日,《人民日报》公布了毛泽东的这个指示。
\end{maonote}
