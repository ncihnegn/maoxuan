
\title{驻外机关也要革命化}
\date{一九六六年九月九日}
\thanks{这是毛泽东同志在《文化大革命中涉外问题情况简报》第九号上的批语。}
\maketitle


\mxname{退陈毅\mnote{1}同志:}

这个批评文件\mnote{2}写得很好,值得一切驻外机关注意,来一个革命化,否则很危险。可以先从维也纳做起。请酌定。

\begin{maonote}
\mnitem{1}陈毅,时任中共中央政治局委员、国务院副总理兼外交部部长、外事办公室主任。
\mnitem{2}指国务院外事办公室秘书组一九六六年九月八日编印的《文化大革命中涉外问题情况简报》第九号上登载的一位奥地利人写给中共中央马恩列斯著作编译局的信。与这期简报一同报送毛泽东的,还有共青团中央机关文革筹委会、临时书记处九月八日印发的《一坦桑尼亚群众来信对我外事活动中的资产阶级思想作风提出尖锐批评》的材料,批评我驻坦使馆在外交活动中讲排场、摆阔气。毛泽东的批语写在第九号简报上。奥地利来信全文:

亲爱的同志们:

读到关于红卫兵支持你们伟大的无产阶级文化大革命的英雄行为的报导等,我们非常赞赏。以你们的伟大领袖毛泽东的智慧为基础的这一历史革命,对于我们这些致力消灭资产阶级生活方式和资产阶级社会的人来说是一个鼓舞。但是我们认为有些更必要提醒你们注意,你们国内的革命斗争同你们在维也纳的商务代表的突出的资产阶级举止和资本主义生活方式是极不相称的。从他们衣着看来,很难(即使不说是不可能的)把他们同蒋介石走狗区别开来。精制的白绸衬衫和高价西服同先进工人阶级代表的身份是很不相称的。这些代表不仅占有一辆,而且是两辆“列尔来得——奔驰”牌汽车(这种汽车可以说是资本主义剥削者的标志)难道具有必要吗?由于这一明显对比而引起了维也纳人的窃窃私语和嘲讽,使我们听了很痛苦。这样的资产阶级行为不仅损害我们的共产主义事业,而且对于伟大的无产阶级文化大革命也起了不好的作用。我们尊敬地并且迫切地要求你们把这种事到有关当局报告,并且立即采取措施,加以纠正。

致以同志的敬礼

奥地利《红旗》派的同志

一九六六年八月三十日维也纳
\end{maonote}
