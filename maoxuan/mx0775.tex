
\title{关于理论问题的谈话要点}
\date{一九七四年十二月}
\thanks{这是毛泽东同志在听取四届人大筹备工作的汇报\mnote{1}后的谈话要点。}
\maketitle


关于理论问题,列宁为什么说对资产阶级专政,要写文章。要告诉春桥、文元\mnote{2}把列宁著作中好几处提到这个问题的找出来,印大字本送我。大家先读,然后写文章。要春桥写这类文章\mnote{3}。这个问题不搞清楚,就会变修正主义。要使全国知道。

我同丹麦首相谈过社会主义制度\mnote{4}。我国现在实行的是商品制度,工资制度也不平等,有八级工资制,等等。这只能在无产阶级专政下加以限制。

所以,林彪一类如上台,搞资本主义制度很容易。因此,要多看点马列主义的书。\mnote{5}

列宁说,“小生产是经常地、每日每时地、自发地和大批地产生着资本主义和资产阶级的”。工人阶级一部分,党员一部分,也有这种情况。

无产阶级中,机关工作人员中,都有发生资产阶级生活作风的。

\begin{maonote}
\mnitem{1}一九七四年十二月二十三日至二十七日,毛泽东在长沙听取周恩来、王洪文(时任中共中央副主席)关于四届人大筹备工作情况的汇报。十二月二十六日,他与周恩来单独长谈,其中讲到理论问题。一九七五年一月七日,周恩来将这次关于理论问题谈话要点整理稿送毛泽东审阅,毛泽东作了个别文字的修改。一月八日,周恩来将这个谈话要点送全体政治局委员、候补委员传阅。二月十八日,中共中央发出通知,将毛泽东的这个谈话要点发给各省、市、自治区党委,各大军区、省军区、野战军党委,中央和国家机关各部委领导小组或党的核心小组,军委各总部、各军兵种党委,要求“认真组织广大党员、干部和党外群众学习”。根据毛泽东的指示摘编的《马克思、恩格斯、列宁论无产阶级专政》语录三十三条,在一九七五年二月二十二日《人民日报》和三月一日出版的《红旗》杂志第三期发表,《人民日报》、《红旗》杂志的编者按中公布了毛泽东这个谈话的主要内容。
\mnitem{2}春桥,即张春桥,时任中共中央政治局常委、国务院副总理、中国人民解放军总政治部主任。文元,即姚文元,时任中共中央政治局委员。
\mnitem{3}《红旗》杂志一九七五年第四期发表了张春桥的文章《论对资产阶级的全面专政》。全文如下:

论对资产阶级的全面专政(张春桥)

无产阶级专政问题,是长期以来马克思主义同修正主义斗争的焦点。列宁说:“只有承认阶级斗争、同时也承认无产阶级专政的人,才是马克思主义者。”毛主席号召全国搞清楚无产阶级专政问题,也正是为了使我们在理论和实践上都搞马克思主义,不搞修正主义。

我们的国家正处在一个重要的历史发展时期。经过二十多年的社会主义革命和社会主义建设,特别是经过无产阶级文化大革命,摧毁了刘少奇、林彪两个资产阶级司令部,我们的无产阶级专政空前巩固,社会主义事业欣欣向荣。当前,全国人民斗志昂扬,下定决心,要在本世纪内把我国建设成为社会主义强国。在这个过程中,以及在整个社会主义历史阶段中,能不能始终坚持无产阶级专政,是关系我国发展前途的头等大事。现实的阶级斗争也要求我们搞清楚无产阶级专政问题。毛主席说:“这个问题不搞清楚,就会变修正主义。”少数人搞清楚不行,一定“要使全国知道”。搞好这次学习的现实的和长远的意义,怎样估计也不会过高。

早在一九二〇年,列宁根据领导伟大十月社会主义革命和第一个无产阶级专政国家的实践经验,尖锐地指出,“无产阶级专政是新阶级对更强大的敌人,对资产阶级进行的最奋勇和最无情的战争,资产阶级的反抗,因为自己被推翻(哪怕是在一个国家内)而凶猛十倍。它的强大不仅在于国际资本的力量,不仅在于它的各种国际联系牢固有力,而且还在于习惯的力量,小生产的力量。因为,现在世界上还有很多很多小生产,而小生产是经常地、每日每时地、自发地和大批地产生着资本主义和资产阶级的。由于这一切原因,无产阶级专政是必要的”。列宁指出,这个专政是对旧社会的势力和传统进行的顽强斗争,流血的和不流血的,暴力的和和平的,军事的和经济的,教育的和行政的斗争,是对资产阶级的全面专政。列宁反复地强调说,不对资产阶级实行长期的全面的专政,便不能战胜资产阶级。列宁的这些话,特别是列宁自己加了着重号的那些话,已经为后来的实践所证实。新的资产阶级果然一批又一批地产生出来了。他们的代表人物就是赫鲁晓夫、勃列日涅夫叛徒集团。这些人的出身一般都很好,几乎都是在红旗下长大的,在组织上加入了共产党,又经过大学培养,成了所谓红色专家。但是,他们是资本主义旧土壤产生出来的新毒草,他们背叛了自己的阶级,篡夺了党和国家的权力,复辟了资本主义,成了资产阶级对无产阶级专政的头目,做了希特勒想做而没有做到的事。这个“卫星上天、红旗落地”的历史经验,我们任何时候都不要忘记,在决心建设强大国家的时候特别不能忘记。

应当清醒地看到,中国仍然存在变修的危险。因为不但帝国主义、社会帝国主义念念不忘侵略和颠覆我们,不但老的地主资产阶级人还在,心不死,而且新的资产阶级分子正象列宁讲的那样每日每时地在产生着。有些同志说:列宁讲的是合作化以前的情况。这显然是不对的。列宁的话并没有过时。这些同志可以读一读毛主席一九五七年发表的《关于正确处理人民内部矛盾的问题》。毛主席在这部著作中,具体地分析了我国包括合作化在内的社会主义改造在所有制方面取得基本胜利以后,仍然存在着阶级、阶级矛盾和阶级斗争,仍然存在着生产关系和生产力之间、上层建筑和经济基础之间又相适应又相矛盾的情况。毛主席总结了列宁以后无产阶级专政的新经验,系统地回答了所有制改变以后出现的各种问题,规定了无产阶级专政的任务和政策,奠定了党的基本路线和无产阶级专政下继续革命的理论基础。十八年来的实践,特别是无产阶级文化大革命的实践,证明了毛主席提出的理论、路线和政策是完全正确的。

毛主席最近指出:“总而言之,中国属于社会主义国家。解放前跟资本主义差不多。现在还实行八级工资制,按劳分配,货币交换,这些跟旧社会没有多少差别。所不同的是所有制变更了。”为了加深对毛主席指示的理解,让我们看一看我国所有制变更的情况,看一看一九七三年各种经济成份在我国工、农、商业中的比重。

先说工业。全民所有制工业占全部工业固定资产的百分之九十七,工业人数的百分之六十三,工业总产值的百分之八十六。集体所有制工业占固定资产的百分之三;人数的百分之三十六点二,总产值的百分之十四。此外,还有人数占百分之零点八的个体手工工业。

再说农业。在农业生产资料中,耕地、排灌机械的百分之九十左右,拖拉机、大牲畜的百分之八十左右是集体所有的。全民所有制的比重很小。因此,全国的粮食和各种经济作物,百分之九十以上是集体经济生产的。国营农场所占比重很小。此外,还保留着少量的社员自留地和家庭副业。

再说商业。国营商业占商品零售总额的百分之九十二点五,集体所有制商业占百分之七点三,个体商贩占百分之零点二。此外,在农村还保留着相当数量的集市贸易。

以上数字可以说明,社会主义的全民所有制和劳动群众的集体所有制,在我国确实已经取得了伟大胜利。不但全民所有制的优势有很大的增长,而且在人民公社经济中,公社、大队、生产队三级所有的比重也有一些变化。以上海市郊区为例,一九七四年公社一级收入占总收入的比重,由上一年的百分之二十八点一,上升为三十点五,大队由百分之十五点二,上升为十七点二,生产队由百分之五十六点七,下降为五十二点三,人民公社一大二公的优越性越来越明显。由于这二十五年来,我们逐步地消灭了帝国主义所有制、官僚资本主义所有制和封建主义所有制,逐步地改造了民族资本主义所有制和个体劳动者所有制,社会主义的两种公有制逐步地代替了这五种私有制,可以自豪地说,我国的所有制已经变更,我国无产阶级和劳动人民已经基本上挣脱了私有制的锁链,我国社会主义的经济基础已经逐步地巩固和发展起来。四届人大通过的宪法,已经明确地记载了我们取得的这些伟大胜利。

但是,我们必须看到,在所有制方面,问题还没有完全解决。我们常说所有制“基本解决”,也就是说还没有完全解决,资产阶级法权在所有制范围内,也没有完全取消。从以上数字就可以看出,在工、农、商业中都还有部分的私有制,社会主义的公有制并不都是全民所有制,而是两种所有制;全民所有制在作为国民经济基础的农业方面还很薄弱。马克思、列宁所设想的在社会主义社会资产阶级法权在所有制范围内已经不存在了,是指的全部生产资料已经归整个社会所有。我们显然还没有走到这一步。我们在理论上和实践上都不要忽视无产阶级专政在这方面还有很艰难的任务。

我们还必须看到,不论是全民所有制,还是集体所有制,都有一个领导权问题,就是说,不是名义上而是实际上归哪个阶级所有的问题。

毛主席一九六九年四月二十八日在党的九届一中全会上说过:“看来,无产阶级文化大革命不搞是不行的,我们这个基础不稳固。据我观察,不讲全体,也不讲绝大多数,恐怕是相当大的一个多数的工厂里头,领导权不在真正的马克思主义者、不在工人群众手里。过去领导工厂的,不是没有好人。有好人,党委书记、副书记、委员,都有好人,支部书记有好人。但是,他是跟着过去刘少奇那种路线走,无非是搞什么物质刺激,利润挂帅,不提倡无产阶级政治,搞什么奖金,等等。”“但是,工厂里确有坏人。”“就是说明革命没有完”。毛主席的这段话;不仅说明了无产阶级文化大革命的必要性,而且使我们比较清醒地认识到,所有制问题,如同其他问题一样,不能只看它的形式,还要看它的实际内容。人们重视所有制在生产关系中起决定作用,这是完全对的。但是,如果不重视所有制是形式上还是实际上解决了,不重视生产关系的另外两个方面,即人们的相互关系和分配形式又反作用于所有制,上层建筑也反作用于经济基础,而且它们在一定条件下起决定作用,则是不对的。政治是经济的集中表现。思想上政治上的路线是否正确,领导权掌握在哪个阶级手里,决定了这些工厂实际上归哪个阶级所有。同志们可以回想一下,一个官僚资本或者民族资本的企业,怎样变成社会主义企业的呢?还不是我们派了一个军管代表或者公方代表到那里,按照党的路线和政策加以改造?历史上任何一种所有制的大变更,不论是封建制代替奴隶制,还是资本主义代替封建主义,都是先夺取政权,再运用政权的力量大规模地改变所有制,巩固和发展新的所有制。社会主义公有制不可能在资产阶级专政下产生,更是只能如此。占旧中国工业百分之八十的官僚资本,只有在人民解放军打败了蒋介石以后,才可能加以改造,归全民所有。同样,资本主义的复辟,也必然是先夺取领导权,改变党的路线和政策。赫鲁晓夫、勃列日涅夫不就是这样改变了苏联的所有制吗?刘少奇、林彪不就是这样程度不同地改变了我们一批工厂企业的性质吗?

还必须看到,我们现在实行的是商品制度。毛主席说:“我国现在实行的是商品制度,工资制度也不平等,有八级工资制,等等。这只能在无产阶级专政下加以限制。所以,林彪一类如上台,搞资本主义制度很容易。”毛主席指出的这种情况,短期内还改变不了。以公社、大队两级经济发展较快的上海郊区人民公社为例,就三级所有的固定资产来看,公社占百分之三十四点二,大队只占百分之十五点一,生产队仍占百分之五十点七。因此,由生产队为基本核算单位过渡到以大队为核算单位,再过渡到以公社为核算单位,单就公社本身的经济条件来说,还需要相当长的时间。就是过渡到以公社为核算单位,也仍然是集体所有制。因此,在短时间内,全民所有制和集体所有制这两种所有制并存的局面不会有根本改变。而只要有这两种所有制,商品生产,货币交换,按劳分配就是不可避免的。由于“这只能在无产阶级专政下加以限制”,城乡资本主义因素的发展,新资产阶级分子的出现,也就是不可避免的。如果不加限制,资本主义和资产阶级就会更快地发展起来。因此,我们决不能因为我们在所有制改造方面取得了伟大胜利,决不能因为进行了一次无产阶级文化大革命,而放松警惕。必须看到,我们的经济基础还不稳固,资产阶级法权在所有制方面还没有完全取消,在人们的相互关系方面还严重存在,在分配方面还占统治地位。在上层建筑的各个领域,有些方面实际上仍然被资产阶级把持着,资产阶级还占着优势,有些正在改革,改革的成果也并不巩固,旧思想、旧习惯势力还顽强地阻碍着社会主义新生事物的生长。随着城乡资本主义因素的发展,新资产阶级分子一批又一批地产生,无产阶级和资产阶级之间的阶级斗争,各派政治力量之间的阶级斗争,无产阶级和资产阶级之间在意识形态方面的阶级斗争还是长期的,曲折的,有时甚至还是很激烈的。就是老一代的地主资产阶级都死光了,这种阶级斗争也决不会停止,林彪一类人物上台,资产阶级的复辟,仍然可能发生。毛主席在《抗日战争胜利后的时局和我们的方针》这篇讲话中说过,一九三六年,党中央所在地保安附近,有一个土围子,里面住着一小股反革命武装,就是死不投降,直到红军打进去才解决了问题。这个故事具有普遍意义,它告诉我们,“凡是反动的东西,你不打,他就不倒。这也和扫地一样,扫帚不到,灰尘照例不会自己跑掉。”现在,资产阶级的土围子还很多,打掉一个还会长出一个,就是将来被消灭得只剩一个了,无产阶级专政的铁扫帚不到,它也不会自己跑掉。列宁说得完全对:“由于这一切原因,无产阶级专政是必要的”。

历史经验告诉我们,无产阶级能不能战胜资产阶级,中国会不会变修正主义,关键在于我们能不能在一切领域、在革命发展的一切阶段始终坚持对资产阶级的全面专政。什么是对资产阶级的全面专政?最简单的概括,就是我们大家正在学习的马克思一八五二年给魏德迈信中的那段话。马克思说:“无论是发现现代社会中有阶级存在或发现各阶级间的斗争,都不是我的功劳。在我以前很久,资产阶级的历史学家就已叙述过阶级斗争的历史发展,资产阶级的经济学家也已对各个阶级作过经济上的分析。我的新贡献就是证明了下列几点:\mnote{1}阶级的存在仅仅同生产发展的一定历史阶段相联系;\mnote{2}阶级斗争必然要导致无产阶级专政;\mnote{3}这个专政不过是达到消灭一切阶级和进入无阶级社会的过渡。”列宁说,马克思的这一段精彩论述,极其鲜明地表达了马克思的国家学说同资产阶级的国家学说之间的主要的和根本的区别,表达了马克思国家学说的实质。这里,应当注意,马克思把关于无产阶级专政的那句话分了三点,这三点是互相联系的,不能割裂的。不能只要其中的一点,不要其他两点。因为这句话完整地表达了无产阶级专政发生、发展和消亡的全过程,包括了无产阶级专政的全部任务和实际内容。在《一八四八年至一八五〇年的法兰西阶级斗争》一书中,马克思更具体地说,这种专政是达到消灭一切阶级差别,达到消灭这些差别所产生的一切生产关系,达到消灭和这些生产关系相适应的一切社会关系,达到改变由这些社会关系产生出来的一切观念的必然的过渡阶段。在这里,马克思讲的是一切,四个都是一切!不是一部分,不是大部分,也不是绝大部分,而是全部!这也没有什么奇怪,无产阶级只有解放全人类才能最后解放自己。要做到达一点,就只有对资产阶级全面专政,把无产阶级专政下的继续革命进行到底,直到在地球上消灭这四个一切,使资产阶级和一切剥削阶级既不能存在,也不能再产生,决不能在过渡的路上停下来。我们认为,只有这样理解,才算领会了马克思国家学说的实质;请同志们想一想,如果不是这样理解,如果在理论和实践上限制、割裂、歪曲马克思主义,把无产阶级专政变成一句空话,把对资产阶级的全面专政变成残缺不全,只在某些领域专政,不在一切领域专政,只在某个阶段(比如所有制改造以前)专政,不在一切阶段专政,也就是说,不是全部地打掉资产阶级的一切土围子,而是留下一些,让它再扩大队伍,那岂不是为资产阶级复辟准备条件吗?那岂不是把无产阶级专政变成保护资产阶级特别是保护新产生的资产阶级的东西了吗?一切不愿吃两遍苦、受二茬罪的工人、贫农、下中农和其他劳动人民,一切决心为实现共产主义奋斗终身的共产党员,一切不愿中国变修的同志们,都要牢记马克思主义的这条基本原理:必须对资产阶级实行全面专政,决不能半途而废。不能否认,我们有些同志组织上加入了共产党,思想上并没有入党。他们的世界观,还没有跳出小生产的圈子,还没有跳出资产阶级的圈子。他们对于无产阶级在某个阶段、某个领域的专政是赞成的,对于无产阶级的某些胜利是高兴的,因为这可以给他带来某种利益,而只要这种利益到手,他就觉得可以安营扎寨,经营经营他的安乐窝了。什么对资产阶级全面专政,什么万里长征第一步,对不起,让别人去干吧,我已经到站了,该下车了。我们劝这些同志:半路上停下来,危险!资产阶级在向你招手,还是跟上大队,继续前进吧!

历史经验又告诉我们,随着无产阶级专政取得一个又一个的胜利,资产阶级表面上也会装作承认无产阶级专政,而实际上干的仍然是复辟资产阶级专政。赫鲁晓夫、勃列日涅夫就是这样干的。他们一不改变苏维埃的名字,二不改变列宁党的名字,三不改变社会主义共和国的名字,而是用承认这些名字作掩护,把无产阶级专政的实际内容改掉,使它变成反苏维埃的、反列宁党的、反社会主义共和国的垄断资产阶级专政。他们提出了全民国家、全民党这样的公开地背叛马克思主义的修正主义纲领,但是,当着苏联人民起来反抗他们的法西斯专政的时候,他们又打起无产阶级专政的旗号来镇压群众。在我们中国,也有类似的情况。刘少奇、林彪不只是宣传阶级斗争熄灭论,当他们镇压革命的时候,也是打着无产阶级专政的旗号。林彪不是有四个“念念不忘”吗?其中之一就是“念念不忘无产阶级专政”。他确实念念不忘,只是要加“推翻”两个字,叫作“念念不忘推翻无产阶级专政”,用他们自己的供词,就是“打着毛主席的旗号打击毛主席的力量”。他们有时候“顺”着无产阶级,甚至装得比谁都革命,提一些“左”的口号,制造混乱,进行破坏,经常地则是针锋相对地同无产阶级斗。你要搞社会主义改造吗?他说要巩固新民主主义秩序。你要搞合作化、公社化吗?他说太早了。你说文艺要革命,他说演点鬼戏也无害。你要限制资产阶级法权吗?他说这可是好东西,应当扩大。他们是一批维护旧事物的专家,象一群苍蝇,一天围着马克思说的那个旧社会的“痕迹”和“弊病”嗡嗡叫。他们特别热心于利用我们的青少年没有经验,向孩子们鼓吹什么物质刺激象臭豆腐,闻闻很臭,吃起来很香。而他们干这些丑事的时候,又总是打着社会主义旗号。有些搞投机倒把、贪污盗窃的坏蛋,不是说他在搞社会主义协作吗?有些毒害青少年的教唆犯不是打着关心爱护共产主义接班人的旗号吗?我们必须研究他们的策略,总结我们的经验,以便更有效地对资产阶级实行全面专政。

“你们要刮‘共产’风吗?”用提出这种问题的方式制造谣言,是某些人最近使用的一种策略。我们可以明确回答:刘少奇、陈伯达刮的那种“共产”风,决不允许再刮。我们从来认为,我们国家的商品不是多了,而是不够丰富。只要公社还没有多少东西可以拿出来同生产大队、生产队“共产”,全民所有制也拿不出极为丰富的产品来对八亿人口实行按需分配,就只能继续搞商品生产、货币交换、按劳分配。对它带来的危害,我们已经采取了并将继续采取适当办法加以限制。无产阶级专政是群众的专政。我们相信,广大群众在党的领导下是有力量、有本领同资产阶级进行斗争,并且最后地战胜他们的。旧中国是一个小生产象汪洋大海一样的国家。对几亿农民进行社会主义教育始终是一个严重问题,需要几代人的努力。但是,这几亿农民中,贫下中农占多数,他们从实践中知道,只有跟着共产党,走社会主义道路,才是他们的光明大道。我们党依靠他们团结中农,一步一步地从互助组、初级社、高级社走到人民公社,我们也一定能够引导他们继续前进。

我们倒是请同志们注意,现在刮的是另一种风,叫“资产”风。就是毛主席指出的资产阶级生活作风,就是那几个“一部分”变成资产阶级分子的妖风。在这几个“一部分”中,共产党员特别是领导干部中刮的“资产”风,对我们的危害最大。受这种妖风的毒害,有的人满脑子资产阶级思想,争名于朝,争利于市,不以为耻,反以为荣。有的人已经发展到把一切都当作商品,包括他们自己在内。他们加入共产党,为无产阶级办事,不过是为了抬高自己这个商品的等级,不过是为了向无产阶级卖高价。那种名曰共产党员,实际上是新资产阶级分子的人,表现了整个资产阶级处于腐朽垂死状态的特点。在历史上,当奴隶主阶级、地主阶级、资产阶级处于上升时期的时候,他们还为人类作些好事。现在这种新资产阶级分子,完全走向他们祖宗的反面,对人类只有破坏作用,完全是一堆“新”垃圾。那种造谣要刮“共产”风的人,其中就有一些是把公共财产占为私有,怕人民再“共”这些“产”的新资产阶级分子或者想乘机捞一把的人。这种人比我们许多同志敏感。我们有的同志说学习是软任务,他们却本能地感觉到了这次学习对无产阶级和资产阶级两个阶级都是硬任务。他们也可能真的刮点“共产”风,或者接过我们的某一个口号,故意地混淆两类不同性质的矛盾,搞点什么名堂,这是值得我们注意的。

在以毛主席为首的党中央领导下,我国亿万群众组成的无产阶级革命大军正在迈动着前进的步伐。我们有了二十五年无产阶级专政的实践经验,又有巴黎公社以来的国际经验,只要我们几百个中央委员、几千个高级干部带头,同广大干部群众一起认真读书学习,调查研究,总结经验,我们一定能够实现毛主席的号召,搞清楚无产阶级专政问题,保证我们的国家沿着马克思主义、列宁主义、毛泽东思想指引的道路胜利前进。“无产者在这个革命中失去的只是锁链。他们获得的将是整个世界。”这个无限光明的远景必将继续鼓舞越来越多的觉悟的工人、劳动人民和他们的先锋队共产党人,坚持党的基本路线,坚持对资产阶级的全面专政,把无产阶级专政下的继续革命进行到底!资产阶级和一切剥削阶级的灭亡,共产主义的胜利,是不可避免的,必然的,不以人们的意志为转移的。
\mnitem{4}毛泽东在一九七四年十月二十日会见丹麦首相保罗·哈特林时说过,总而言之,中国属于社会主义国家。解放前跟资本主义差不多。现在还实行八级工资制,按劳分配,货币交换,这些跟旧社会没有多少差别。所不同的是所有制变更了。
\mnitem{5}《红旗》杂志一九七五年第三期发表了姚文元的文章《论林彪反党集团的社会基础》,全文如下:

论林彪反党集团的社会基础(姚文元)

毛主席在讲到必须搞清楚无产阶级对资产阶级专政的问题时明确指出:“林彪一类如上台,搞资本主义很容易。因此,要多看点马列主义的书。”这就提出了一个极其重要的问题:即“林彪一类”的阶级本质是什么?林彪反党集团产生的社会基础是什么?把这个问题弄清楚,对于巩固无产阶级专政、防止资本主义复辟,对于坚定地执行党在社会主义历史阶段的基本路线,一步一步地造成资产阶级既不能存在也不能再产生的条件,无疑是十分必要的。

同一切修正主义者和修正主义思潮一样,林彪及其修正主义路线不是一种偶然的现象。林彪及其死党在全党、全军和全国人民中是极其孤立的,但产生出这一伙极端孤立的“天马行空”、“独往独来”的人物,却有它深刻的社会基础。

林彪反党集团代表了被打倒的地主资产阶级的利益,代表了被打倒的反动派推翻无产阶级专政、复辟资产阶级专政的愿望,这一点,是比较清楚的。林彪反党集团反对无产阶级文化大革命,对我国无产阶级专政的社会主义制度怀着刻骨的仇恨,诬蔑为“封建专制”,咒骂为“当代的秦始皇”。他们要使地、富、反、坏、右“政治上、经济上得到真正解放”,即在政治上经济上变无产阶级专政为地主买办资产阶级专政,变社会主义制度为资本主义制度。作为力图复辟的资产阶级在党内的代理人,林彪反党集团向党和无产阶级专政进攻达到了很疯狂的程度,直到搞特务组织和策划反革命武装政变。这种疯狂性,反映了丧失政权和生产资料的反动派,为了夺回他们失去的剥削阶级的阵地,必然要用尽一切他们所能采取的手段。我们看到了林彪在政治上、思想上破产以后,怎样象一个亡命的赌徒一样想把无产阶级“吃掉”,孤注一掷,直到叛国投敌,毛主席、党中央非常耐心的教育、等待、挽救也丝毫不能改变他的反革命本性。这都反映了无产阶级专政下无产阶级同资产阶级两大对抗阶级的生死斗争,这种斗争会继续一个很长的时期。只要还存在被打倒的反动阶级,党内(以及社会上)就有可能出现把复辟愿望变为复辟行动的资产阶级代表人物。因此,要提高警惕,要警觉和粉碎国内外反动派的种种阴谋,切不可麻痹大意。但是,这样认识还不是事物的全部。林彪反党集团不但代表了被打倒的地主资产阶级复辟的愿望,而且代表了社会主义社会中新产生的资产阶级分子篡权的愿望,他们身上具有新产生的资产阶级分子的某些特点,他们当中若干人本身就是新产生的资产阶级分子,他们的某些口号适应和反映了资产阶级分子和想走资本主义道路的人发展资本主义的需要。正是这后一个方面,需要我们进一步加以分析。

毛主席指出:“列宁说,‘小生产是经常地、每日每时地、自发地和大批地产生着资本主义和资产阶级的。’工人阶级一部分,党员一部分,也有这种情况。无产阶级中,机关工作人员中,都有发生资产阶级生活作风的。”林彪反党集团中的某些人物就是这种新产生出来的资产阶级和资本主义的代表。其中如林立果及其小“舰队”,就完全是在社会主义社会中产生出来的反社会主义的资产阶级分子、反革命分子。

资产阶级影响的存在,国际帝国主义、修正主义影响的存在,是产生新的资产阶级分子的政治思想根源。而资产阶级法权的存在,则是产生新的资产阶级分子的重要的经济基础。

列宁指出:“在共产主义社会的第一阶段(通常称为社会主义),‘资产阶级法权’没有完全取消,而只是部分地取消,只是在已经实现的经济变革的范围内,也就是在对生产资料的关系上取消。”“但是它在另一方面却依然存在,依然是社会各个成员间分配产品和分配劳动的调节者(决定者)。‘不劳动者不得食’这个社会主义原则已经实现了‘按等量劳动领取等量产品’这个社会主义原则也已经实现了。但是,这还不是共产主义,还没有消除对不同等的人按不等量的(事实上是不等量的)劳动给予等量产品的‘资产阶级法权’。”

毛主席指出:“中国属于社会主义国家。解放前跟资本主义差不多。现在还实行八级工资制,按劳分配,货币交换,这些跟旧社会没有多少差别。所不同的是所有制变更了。”“我国现在实行的是商品制度,工资制度也不平等,有八级工资制,等等。这只能在无产阶级专政下加以限制。”

社会主义社会中,还存在全民所有制和集体所有制这两种社会主义所有制,这就决定了我国现在实行的是商品制度。列宁和毛主席的分析都告诉我们,对于社会主义制度下在分配和交换方面不可避免还存在的资产阶级法权,应当在无产阶级专政下加以限制,以便在长期的社会主义革命过程中,逐步创造消灭这种差别的物质的和精神的条件。如果不是这样,相反地,要求巩固、扩大、强化资产阶级法权及其所带来的那一部分不平等,那就必然会产生两极分化的现象,即少数人在分配方面通过某种合法及大量非法的途径占有越来越多的商品和货币,被这种“物质刺激”刺激起来的资本主义发财致富、争名夺利的思想就会泛滥起来,化公为私、投机倒把、贪污腐化、盗窃行贿等现象也会发展起来,资本主义的商品交换原则就会侵入到政治生活以及党内生活,瓦解社会主义计划经济,就会产生把商品和货币转化为资本和把劳动力当作商品的资本主义剥削行为,就会在某些执行修正主义路线的部门和单位改变所有制的性质,压迫和剥削劳动人民的情况就会重新发生。其结果,在党员、工人、富裕农民、机关工作人员中都会产生少数完全背叛无产阶级和劳动人民的新的资产阶级分子、暴发户。工人同志说得好:“你不限制资产阶级法权,资产阶级法权就要限制社会主义的发展,助长资本主义的发展。”而当资产阶级在经济上的力量发展到一定程度时,它的代理人就会要求政治上的统治,要求推翻无产阶级专政和社会主义制度,要求全盘改变社会主义所有制,公开地复辟和发展资本主义制度。新的资产阶级一上台,首先要血腥地镇压人民,并在上层建筑包括各个思想文化领域中复辟资本主义,接着,他们就会按资本和权力的大小进行分配,“按劳分配”只剩下一个外壳,一小撮垄断了生产资料的新生资产阶级分子同时垄断了消费品和其他产品的分配大权。——这就是今天在苏联已经发生的复辟过程。

林彪反党集团如何不择手段地聚敛财富,如何穷奢极欲地追求资产阶级的生活方式,如何利用资产阶级法权为自己干种种见不得人的阴险丑恶的勾当,人们已揭发批判了很多。但更能说明问题的是反革命政变计划《“571工程”纪要》,这个计划中,林彪反党集团用以煽动或挑拨各个阶级中某些人反对无产阶级专政的,不是别的,正是资产阶级法权思想。或者说,这个计划中所代表的阶级利益,除了老的资产阶级之外,正是一部分新的资产阶级分子,以及少数想得用资产阶级法权发展资本主义的人。因而它攻击的矛头便对准了毛主席的无产阶级革命路线,因而它特别仇恨我国无产阶级专政下通过社会主义革命对资产阶级法权进行的某些限制。

对于机关干部参加五.七干校,林彪反党集团诬蔑为“变相失业”;精简机构,接近群众,被他们攻击为打击干部。他们认为干部应当是骑在人民头上的老爷,所以一参加集体生产劳动就变成“失业”。这是挑动机关工作人员中一部分想扩大资产阶级法权,做官当老爷、有严重资产阶级生活作风的人,反对党的路线,反对社会主义制度。

对于知识分子同工农相结合,上山下乡,林彪反党集团诬蔑为“等于变相劳改”。一批又一批有共产主义觉悟的青年生气勃勃地奔赴农村,这是对缩小三大差别、限资产阶级法权有深远意义的伟大事业,一切革命的人们都热情赞扬它,而受了资产阶级思想侵蚀特别是受资产阶级法权思想束缚的人则反对它。能不能支持知识青年同工农结合,直接联系到大学教育革命能不能坚持走上海机床厂道路,学生不但从工农中来,而且回到工农中去。林彪反党集团对此特别仇恨,不但表现了他们同劳动人民的对立,而且也暴露了他们利用资产阶级法权向党进攻,妄图煽动一部分受资产阶级法权思想影响较深的人,反对社会主义革命。他们的纲领是扩大城市同农村之间、脑力劳动同体力劳动之间的差别,把知识青年变成新的贵族阶层,想以此来争取某些受资产阶级法权思想影响较深的人对他们反革命政变的支持。

对于工人阶级发扬共产主义精神,批判修正主义的“物质刺激”,林彪反党集团诬蔑为“变相受剥削”。林彪是“物质刺激”狂热的鼓吹者,他在黑笔记中就亲笔写下了“物质刺激还是必要的”、“唯物主义=物质刺激”、“诱:以官,禄,德”之类的修正主义黑话。林彪反党集团的一个主要成员也在黑笔记中写道:“按劳分配和物质利益原则”是发展生产的“决定动力”。他们表面上是主张用钞票去“刺激”工人,实际上是想无限止地扩大工人的等级差别,在工人阶级中培养和收买一小部分背叛无产阶级专政、也背叛无产阶级利益的特殊阶层,分裂工人阶级的团结。他们用资产阶级世界观腐蚀工人,又妄图把工人阶级中一小部分受资产阶级法权思想影响较深的人,作为支持他们反无产阶级专政的力量之一。林彪一伙“特别”注意用“工资”来引诱“青年工人”,所谓“诱:以官,禄,德”就是他们的阴谋诡计,这从反面告诉我们:青年工人特别是当了的干部的青年工人,必须自觉地抵制资产阶级的物质引诱和各种资产阶级法权思想的捧场,要保持和发扬共产主义的为无产阶级和全人类彻底解放而英勇奋斗的革命精神,要努力用马列主义世界观武装自己,切不可被商品、货币交换、庸俗的捧场、阿谀奉承、宗派主义之类的花花世界弄昏了头脑,以致上了林彪一类政治骗子或社会上地主资产阶级分子的当。他们以“关心”为名,实则“刺激”青年工人走资本主义道路,可以说是一种政治上的“教唆犯”。缺少经验的新产生的资产阶级分子在前面违法乱纪,老奸巨猾的老资产阶级分子躲在后面出谋划策,这是今天社会阶级斗争中经常见到的一种现象。我们在处理被腐蚀的青少年罪犯时特别着重打击幕后教唆犯,这个方针要坚持下去。在现实斗争中已经涌现出了一批同资产阶级腐蚀进行旗帜鲜明斗争的青年工人,应当支持他们,总结他们的斗争经验。

林彪反党集团还诬蔑农民“缺吃少穿”,诬蔑部队干部“生活水平下降”,诬蔑红卫兵在文化大革命中批判资产阶级那种敢想、敢说、敢闯、敢做、敢革命的精神是“被利用”……这一切,无不是想从根本上否定社会主义制度和党的群众路线,否定无产阶级对资产阶级的专政,扩大资产阶级法权,复辟资本主义。他们诬蔑农民“缺吃少穿”,其目的是煽动农民搞“吃光分光”,瓦解和取消社会主义集体经济。如果照这条路线去做,其结果,是少数人上升为新资产阶级,绝大多数人受资本主义剥削,这是地主、富农和农村中一部分走资本主义道路的富裕中农所盼望的那样一种局面。

现在我们可以看到林彪所谓“建设真正的社会主义”是什么东西了。这就是在社会主义招牌下扩大资产阶级法权,使新的资产阶级分子和某些想走资本主义道路的派别和集团,同被打倒的地主资产阶级相勾结,“指挥一切、调动一切”,推翻无产阶级专政,复辟资本主义。林彪一类人物则是他们的政治代表。林彪反党集团在《“571工程”纪要》中提出的这些纲领,既不是从天上掉下来的,也不是他们自封为“超天才”的头脑中所固有的,而是社会存在的反映。确切地说,从他们的资产阶级反动立场出发,他们反映了只占人口百分之几的没有改造好的地、富、反、坏、右的要求,反映了少数新的资产阶级分子和想利用资产阶级法权上升为新资产阶级分子的人的要求,而反对占人口百分之九十以上的革命人民坚持社会主义道路的要求。他们用唯心论的先验论反对唯物论的反映论,然而他们本身反革命思想的形成却必须用唯物论的反映论来说明。

为什么林彪一类上台搞资本主义制度很容易呢?就因为我们社会主义社会中还存在阶级和阶级斗争,还存在产生资本主义的土壤和条件。为了逐步减少这种土壤和条件,直到最后消灭它,就必须坚持无产阶级专政下的继续革命。这是在毛主席革命路线指引下的无产阶级先锋队,经过好几代人坚韧不拔的努力才能完成的任务。这就必须坚持党的基本路线,提高工人阶级的政治觉悟,巩固工农联盟,团结一切可以团结的力量,并团结和领导广大革命群众在同阶级敌人的斗争和三大革命运动的实践中自觉地改造自己的世界观。这就必须巩固和发展社会主义的全民所有制和劳动群众集体所有制,防止在所有制方面已被取消的资产阶级法权复辟,继续在较长时间内逐步完成所有制改造方面尚未完成的那一部分任务;并在生产关系的其他两个方面,即人与人相互关系和分配关系方面,限制林彪反党集团,批判资产阶级法权思想,不断削弱产生资本主义的基础。这就必须坚持上层建筑领域中的革命,深入批判修正主义,批判资产阶级,实现无产阶级对资产阶级的全面专政。

毛主席在一九七一年八月至九月巡视各地的谈话中说过:“我们唱了五十年国际歌了,我们党有人搞了十次分裂。我看还可能搞十次、二十次、三十次,你们信不信?你们不信,反正我信。到了共产主义就没有斗争了?我就不信。到了共产主义也还是有斗争的,只是新与旧,正确与错误的斗争就是了。几万年以后,错误的也不行,也是站不住的。”列宁说过:“是的,我们推翻了地主和资产阶级,扫清了道路,但是我们还没有建成社会主义大厦。旧的一代被清除了,而在这块土壤上还会不断产生新的一代,因为这块土壤过去产生过、现在还在产生许许多多资产者。有些人象小私有者一样看待对资本家的胜利,他们说:‘资本家已经捞了一把,现在该轮到我了。’可见他们每一个人都是产生新的一代资产者的根源。”列宁说的是社会主义阶级斗争的长期性,毛主席说的是这种斗争反映在党内而形成两条路线斗争的长期性。我们必须经过这种阶级斗争和路线斗争,不断战胜资产阶级及其代表人物搞修正主义、搞分裂、搞阴谋诡计的行动,才能逐步造成资产阶级既不能存在也不能再产生的条件,最后消灭阶级,而这正是无产阶级专政的整个历史时代要完成的伟大事业。

由于资产阶级思想腐蚀和资产阶级法权存在而产生出来的新资产阶级分子,一般都具有两面派和暴发户的政治特点。为了在无产阶级专政下进行资本主义活动,他们总是要打着某种社会主义的招牌;由于他们的复辟活动不是夺回自己丧失的生产资料而是要夺取他们未曾占有过的生产资料,因而表现特别贪婪,恨不得一下子把属于全国人民所有或集体所有的财富吞下肚子里去,化为私有制。林彪反党集团即具有这种政治特点。“子系中山狼,得志便猖狂”,是《红楼梦》刻画“应酬权变”而又野蛮毒辣的孙绍祖的两句诗,用来移赠林彪反党集团,是颇为适合的。当林彪在“得志”即掌握一部分政治经济大权之前,他用反革命两面派的手段欺骗党、欺骗群众,并利用群众运动的力量为自己的目的服务,为此他可以打出革命的招牌或喊出革命的口号,同时又加以歪曲。毛主席在文化大革命初期写的一封信中分析林彪一伙的内心世界时指出:“我猜他们的本意,为了打鬼,借助钟馗。”是很能说明这种现象的。“借助”,就是敲门砖,等到他们的目的达到之后,便不要这个“借助”,而要反过头来恶狠狠地搞掉这个“借助”了。反革命两面派也好,或者用林彪反党集团自己招供的话,“打着毛主席的旗号打击毛主席的力量”也好,都是同一类做法的不同说法。等到林彪反党集团如他们自己刻画的那样,自以为“经过几年准备,在思想上、组织上、军事上的水平都有相当提高。具有一定的思想和物质基础”时,他们就要猖狂起来了。他们在自己把持、控制的单位、部门,变社会主义公有制为林彪反党集团私有制,他们暴露出越来越露骨的政治野心,这种野心会随着他们“得志”的程度而膨胀,正同资产阶级的贪欲会随着资本积累的增长而发展一样,永不会有止境。马克思分析资产阶级时说过:“当作资本家,他只是人格化的资本。他的灵魂,便是资本的灵魂。”林彪作为资产阶级在党内的代理人,他的灵魂也只是已被打倒而梦想复辟以及正在产生而妄想统治的老的和新的资产阶级的灵魂。从阶级分析出发,林彪一伙那些倒行逆施的反革命政治活动的根源便很清楚了:他们鼓吹孔孟之道,他们背叛党、背叛中国人民而投靠社会帝国主义,正是尊孔卖国的中国买办资产阶级干过的勾当,而他们狂热地策划反革命政变,也不过重复世界上许多国家的资产阶级使用过无数次并至今还在使用的手段罢了。

我们的任务,就是一方而要逐步地削弱资产阶级和资本主义产生的土壤;另一方面,当林彪一类新的资产阶级产生出来或正在产生的时候能及时地识别他们。学习马克思主义、列宁主义、毛泽东思想的积极性就在这里。离开马克思主义的指导,我们不可能完成上述两个方面的任务;而且当修正主义思潮出来时还会由于自己有资产阶级法权思想或分辨不清而上当受骗,甚至糊里糊涂地上了贼船。不然,为什么一条修正主义出来会有人跟着走呢?为什么林彪一伙在九届二中全会上可以用唯心论加起哄来骗人呢?为什么林彪反党集团那些赤裸裸的分裂党、推翻无产阶级专政的话会在少数干部中找到市场呢?为什么大、小“舰队”可以明目张胆地把请客送礼、封官许愿之类作为拉山头、搞宗派、耍阴谋的手段呢?为什么他们在黑笔记中要把“用技术掩盖政治”之类作为自己反革命活动的策略呢?这当中有深刻的教训。一九五九年反对彭德怀反党集团时,毛主席曾经指出,“现在,主要危险是经验主义”,因此要认真读书,这十几年来,毛主席多次重复了这个意见,毛主席强调党的高中级干部首先是中央委员,“都应程度不同地认真看书学习,弄通马克思主义”,并强调“这几年应当特别注意宣传马、列”,林彪反党集团垮台以后又再次说“我正式劝同志们读一点书”,最近讲无产阶级专政时又再一次强调了这一条。这些语重心长的谆谆教导,我们感到多么亲切啊!全党同志特别是高级干部,一定要把这件事当作关系到巩固无产阶级专政的大事来抓,对马、列和毛主席关于无产阶级专政的有关论述和主要著作,首先要自己学好,搞清楚,力求从理论和实践的结合上说明问题,力求从思想和行动上打掉那些脱离群众的资产阶级思想作风,和群众打成一片,真正做社会主义新生事物的促进派,善于分辨和敢于抵制资本主义的腐蚀。我们党几十年来形成的艰苦奋斗的光荣传统,一定要发扬和继承下去。要了解情况,研究政策,包括经济政策。抓革命,促生产,促工作,促战备,行之有效,必须坚持。要注意区别两类不同性质的矛盾,准确而有力地打击极少数坏人,对群众中的资本主义影响,要根据“团结——批评——团结”的公式,主要采取学习和提高觉悟的方法,支持坚决抵制资本主义的先进事物的方法,回忆对比的方法,说服教育、批评和自我批评的方法去解决,做到团结两个百分之九十五。批判资本主义倾向要形成舆论,争取多数,启发自觉,积极引导。对个别陷在资本主义泥坑里很深的人,要向他猛喝一声:“同志,赶快回头!”

我们在文章开头时曾指出:林彪反党集团在全国人民中是很孤立的。为了分析它产生的阶级根源,我们指出了林彪反党集团得以产生的土壤和条件。在讲了这一面之后,我们还必须指出:林彪反党集团在本质上是很虚弱的,同一切反动派一样,不过是纸老虎。林彪反党集团的一切反革命活动,只不过记录了它的失败和困境,而不是记录它的胜利。社会主义制度一定要代替资本主义制度,共产主义一定会在全世界取得胜利,这是不以人们意志为转移的客观规律。社会主义社会是从旧社会脱胎而来的,“因此它在各方面,在经济、道德和精神方面都还带着它脱胎出来的那个旧社会的痕迹”。这并没有什么奇怪。二十五年以来的历史告诉我们:只要我们坚持无产阶级专政,坚持毛主席无产阶级专政下继续革命的学说,坚持毛主席给我们规定的社会主义革命的路线、方针和政策,我们就能够粉碎阶级敌人的反抗,一步一步地减少这些痕迹,不断夺取新的胜利。我们今天社会主义事业蒸蒸日上、欣欣向荣的大好形势,同帝国主义、社会帝国主义内部四分五裂、内外交困,形成鲜明的对照。这一次毛主席提出的理论问题,必将从理论实践上使我们进一步认识无产阶级专政的历史任务和完成这些任务的方法,大大促进无产阶级专政的巩固,促地社会主义革命的深入和社会主义建设的发展,促进全国的安定团结。中国的共产党人是有信心的,中国的无产阶级和革命人民是有信心的,他们正在党的领导下团结一致意气风发地投入反修防修的斗争。中国革命的历史是革命人民经过曲折斗争走向胜利的历史,也是反动派经过反复较量走向灭亡的历史。正如毛主席总结的那样:“中国自从一九一一年皇帝被打倒以后,反动派当权总是不能长久的。最长的不过二十年(蒋介石),人民一造反,他也倒了。蒋介石利用了孙中山对他的信任,又开了一个黄埔学校,收罗了一大批反动派,由此起家。他一反共,几乎整个地主资产阶级都拥护他,那时共产党又没有经验,所以他高兴地暂时地得势了。但这二十年中,他从来没有统一过,国共两党的战争,国民党和各派军阀之间的战争,中日战争,最后是四年大内战,他就滚到一群海岛上去了。中国如发生反共的右派政变,我断定他们也是不得安宁的,很可能是短命的,因为代表百分之九十以上人民利益的一切革命者是不会容忍的。”“结论:前途是光明的,道路是曲折的,还是这两句老话。”让我们沿着毛主席指引的方向和道路奋勇前进吧!
\end{maonote}
