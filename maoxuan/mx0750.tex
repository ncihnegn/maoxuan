
\title{关于战备疏散的指示}
\date{一九六九年十月十四日}
\thanks{这是毛泽东同志针对苏联对我核战争威胁\mnote{1}下关于战备疏散的指示。}
\maketitle


中央领导同志都集中在北京不好,一颗原子弹就会炸死很多人,应该分散些。一些老同志可以疏散到外地去。

\begin{maonote}
\mnitem{1}“文革”初期,中苏边界的紧张局势加剧,一九六九年三月十五日凌晨,苏军边防军六十余人在六辆装甲车的掩护下,从珍宝岛北端侵入,中国边防部队自卫还击,击毙了苏军边防部队总指挥列昂诺夫上校和杨辛中校,胜利地保卫了珍宝岛。苏联方面亡五十八人,伤九十四人。中国边防部队亡二十九人,伤六十二人,失踪一人。八月十三日,苏联军方经密谋在新疆铁克列提袭击我军三十余人的巡逻队,三十八名官兵几乎全部壮烈牺牲,仅一人生还。大规模流血冲突使中苏双方走到了战争边缘。

八月二十日,苏联驻美大使多勃雷宁奉命在华盛顿紧急约见了美国总统国家安全事务助理基辛格博士,试探美国:如果苏联打击中国核设施,美国将如何反应?苏联的意图非常明显:在中美关系当时也很尖锐的情况下,如果苏联动手,让美国至少保持中立。尼克松在同他的高级官员紧急磋商后认为西方国家的最大威胁来自苏联,一个强大中国的存在符合西方的战略利益,因此准备把此消息透露出去,八月二十八日,《华盛顿明星报》在醒目位置刊登一则消息,标题是《苏联欲对中国做外科手术式核打击》。文中说:“据可靠消息,苏联欲动用中程弹道导弹,携带几百万吨当量的核弹头,对中国的重要军事基地——酒泉、西昌发射基地、罗布泊核试验基地,以及北京、长春、鞍山等重要工业城市进行外科手术式的核打击。”

同时,苏联秘密通知东欧各国:苏将先发制人打击中国核设施。下旬,东欧国家泄露了苏联的意图。八月二十七日,中共中央转发了军委办事组《关于加强全国人民防空工作的报告》,在“备战、备荒、为人民”的口号下,成立了以周恩来为组长的中国人民防空工作领导小组,各省、市、自治区也成立了人民防空领导小组。全国很快进入“要准备打仗”的临战态势,许多企业转向军工生产,国民经济开始转向临战状态,大批工厂转向交通闭塞的山区,实行“山、散、洞”配置,北京等大城市开挖地下工事。

八月二十八日,毛泽东主席签发《中国共产党中央委员会命令》(“八·二八命令”),指出“苏修社会帝国主义越来越疯狂地在我边境进行武装挑衅”,要求全国人民防止苏联的“突然袭击”。

八月下旬,美国还侦察到苏联空军在远东的一次停飞待命(在整个九月份都继续保持着),这意味着所有飞机同时进入高度准备状态,往往是一次进攻的信号,至少也是对敌人的严酷警告。

九月十日,苏联塔斯社指责中国从一九六九年六月到八月中旬,蓄意侵犯苏联边境四百八十八次并且挑起武装冲突。

九月十六日,伦敦《星期六邮报》登载了苏联自由撰稿记者维克多·路易斯的文章,称苏联“为了自身的利益或者那些受到威胁的国家的利益”,有权单方面干涉其他社会主义国家的事务。文章还声称“世界只会在战争爆发之后才得知它”,并提到苏联对设在新疆罗布泊的中国核试验基地进行空袭的可能性。维克托·路易斯的真实身份令人怀疑。据熊向晖回忆,中方注意到此人经常向外界透露苏联重大决策。基辛格则认为“他很可能是苏联政府的一个代言人”。《剑桥中华人民共和国史》甚至称他是间谍。维克多的文章是对美国的一个试探,更是对中国的示警。受毛泽东委托研究国际形势的陈毅等四位老帅经过紧急讨论,次日即向中央提交了《对目前局势的看法》。苏联方面蠢蠢欲动,而打击手段和打击对象都与核武器有关,核战争的阴影顿时笼罩在中国的上空。

出于美国全球战略利益和发生大规模核战争的严重后果的考虑,美国尼克松总统决定,故意用已被苏联破译的密码,向苏联本土一百三十四个城市、军事要点、交通枢纽、重工业基地发出进行准备核打击的指令,以牵制苏联。

为了展示抵抗决心,九月二十三日和二十九日,在毛泽东指示下,我国先后进行了当量为两万多吨当量的地下原子弹裂变爆炸和轰炸机空投的当量约三百多万吨的氢弹热核爆炸,美联社播发的一篇评论颇具代表性:“中国最近进行的两次核试验,不是为了获取某项成果,而是临战前的一种检测手段。”

九月十一日,在北京机场,苏联会议部长主席柯西金与周恩来会谈,双方会谈似乎异常顺利,但柯西金并没有否认“苏联对中国进行核打击”的传闻,九月二十六日,柯西金在给周恩来的信中称,他建议自十月十日起在北京开始中苏边界问题谈判,九月二十九日,中国建议将边界谈判改到十月二十日开始,十月十四日,柯西金再次致信周恩来,告知苏联谈判代表团将于十月二十日前抵京,针对苏联突然的这种缓和姿态,中央认为,有一种极大的可能是,柯西金来华会谈,与日本特使来栖“珍珠港事变”前赴美一样,只是战争爆发前放的烟雾。且柯并未承诺不对中国发动核打击,苏联战略火箭部队也已做好随时打击中国的部署。由于苏军各师都配备有战术核武器,且受过在核战场上作战的训练,确有能力先发制人,摧毁中国绝大部分核武器和导弹基地、海空军基地和地面部队,占领大片中国领土,数以亿计的人将遭到灭顶之灾。

为保证中国领导人不被一次消灭掉,毛泽东主席做了这段指示。十月十四日,中共中央发出通知,要求在京的中央党政军主要领导人及一些老同志,于十月二十日以前全部战备疏散。通知说:为了适应反侵略战争的需要,应付苏修社会帝国主义的突然袭击,经中央讨论决定:中央机关集中到北京郊区战备地下指挥部办公,由周恩来同志留在北京主持工作;毛泽东主席到武汉主持全国的大政方针,林彪副主席到苏州负责战备。

其他人员疏散为:董必武、朱德、李富春、滕代远、张鼎丞、张云逸去广州;张闻天去肇庆;陈云、王震及邓小平去南昌;陈毅去石家庄;徐向前及刘少奇去开封;聂荣臻去邯郸;刘伯承去武汉(后转上海);叶剑英、曾山去长沙;邓子恢去南宁(后转桂林);谭震林去桂林;陶铸去合肥;王稼祥去信阳。何长工去江西峡江,宋任穷去盘锦地区,陈再道、钟汉华去江西,秦基伟、李成芳等六人去湖南汉寿。
\end{maonote}
