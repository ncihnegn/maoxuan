
\title{国民党进攻的真相}
\date{一九四五年十一月五日}
\thanks{这是毛泽东以中共发言人的名义发表的谈话。这时蒋介石已经撕毁《双十协定》,进攻解放区的内战规模已经日趋扩大。}
\maketitle


合众社重庆三日电报道,国民党中央宣传部长吴国桢宣称,“政府在此次战争中全居守势”,并提出所谓恢复交通的办法\mnote{1}。新华社记者为此询问中共方面发言人。

中共发言人告记者称:吴氏所说“守势”云云,全系撒谎。除我军已撤退的浙东、苏南、皖中、皖南、湖南五个解放区全被国民党军队进占、大肆蹂躏人民外,其它大多数解放区,例如广东、湖北、河南、苏北、皖北、山东、河北等省,国民党正规军已有七十余师开到我解放区及其附近,压迫人民,进攻我军,或准备进攻。正在向我解放区开进者,尚有数十师。这难道是取守势吗?其中由彰德\mnote{2}北进一路,攻至邯郸地区之八个师,两个师反对内战,主张和平,六个师(其中有三个美械师)在我解放区军民举行自卫的反击之后,始被迫放下武器。这一路国民党军的许多军官,其中有副长官、军长、副军长多人,现在都在解放区\mnote{3},他们都可以证明他们是从何处开来、如何奉命进攻的全部真情。这难道也是取守势吗?我豫鄂两省解放区军队,现被国民党第一、第五、第六等三个战区的军队共二十几个师四面包围,刘峙任该区“剿共”总指挥。我豫西、豫中、鄂南、鄂东、鄂中等处解放区都被国民党军队侵占,大肆烧杀,迫得我李先念、王树声等部无处存身,不得不向豫鄂交界地区觅一驻地,以求生存,但又被国民党军队紧紧追击\mnote{4}。这难道也是取守势吗?在晋绥察三省,也是如此。十月上旬,阎锡山指挥十三个师,攻入我上党解放区襄垣、屯留区域,被当地军民在自卫战斗中全部缴械,被俘人员中亦有军长师长多人。他们现在我太行解放区,一个个活着,足以证明他们是从何处开来、如何奉命进攻的全部真情。最近阎锡山在重庆报道他如何被攻,而他则仅取“守势”,说了种种谎话。他大概忘记了他的十九军军长史泽波,暂四十六师师长郭溶,暂四十九师师长张宏,六十六师师长李佩膺,六十八师师长郭天兴,暂三十七师师长杨文彩等位将军\mnote{5},现正住在我解放区,足以驳斥吴国桢氏、阎锡山氏和一切挑拨内战的反动派的任何谎话。傅作义将军奉命进攻我绥远、察哈尔、热河三省解放区,已两个多月,曾打到张家口的门口,占领我整个绥远解放区和察哈尔西部。难道这也是取守势和未放“第一枪”吗?我察绥两省军民起而自卫,在反攻战斗中亦俘虏大批官兵,他们都可以证明他们是从何处开来、如何进攻等等\mnote{6}。在各次自卫战斗中,我方缴获大批“剿匪”和反共文件,其中有国民党最高当局所颁发而被吴国桢氏称为不过是“笑话”的《剿匪手本》、“剿匪”命令\mnote{7}和其它反共文件,正在向延安解送中。这些反共文件,都是国民党军队进攻解放区的铁证。

记者又问中共发言人,吴国桢氏所提恢复交通办法,你的意见如何?该发言人答道:这不过是缓兵之计而已。国民党当局正在大举调兵,像洪水一样,想要淹没我整个解放区。他们在九、十两月几个进攻失败之后,正在布置新的更大规模的进攻。而阻碍这种进攻,亦即有效地制止内战的武器之一,就是不许他们在铁路上运兵。我们和旁人一样,主张交通线迅速恢复,但是必须在受降、处置伪军和实行解放区自治三项问题获得解决之后,才能恢复。先解决交通问题,后解决三项问题呢,还是先解决三项问题,后解决交通问题呢?解放区军队艰苦抗日八年,为什么没有受降资格,而劳其它军队从远远的地方开去受降呢?伪军人人得而诛之,为什么一律编为“国军”,并且指挥他们进攻解放区呢?地方自治,《双十协定》\mnote{8}上已有明文规定,孙中山先生早主省长民选,为什么还要政府派遣官吏呢?交通问题应该迅速解决,这三大问题尤其应该迅速解决。三大问题不解决而言恢复交通,只是使内战扩大延长,达到内战发动者们淹没解放区的目的。为着迅速制止已经普及全国的反人民反民主的内战,我们主张:(一)已经进入华北、苏北、皖北、华中各解放区及其附近的政府受降军队和进攻军队,立即撤返原防,由解放区军队去接受敌人投降和驻防各城市与交通线,恢复被侵占的解放区;(二)全部伪军立即缴械遣散,在华北、苏北、皖北者,由解放区负责缴械遣散;(三)承认一切解放区的人民民主自治,中央政府不得委派官吏,实现《双十协定》的规定。发言人说:只有如此,才能制止内战;否则是完全没有保障的。绥远、上党、邯郸三次自卫战斗中所缴获的文件以及大举调兵和大举进攻的实际行动,已充分证明国民党当局所谓恢复交通是为着人民,不是为着内战,乃是毫不足信的。中国人民被欺骗得已经够了,现在再不能被欺骗。现在的中心问题,是全国人民动员起来,用一切方法制止内战。


\begin{maonote}
\mnitem{1}抗日战争结束时,中国的大部分铁路交通线都在解放区军民的控制或包围之中。国民党反动派在所谓“恢复交通”的借口下,企图利用这些交通线分割解放区,并且把它的几百万军队运往东北、华北、华东、华中,进攻解放区,抢占大城市。
\mnitem{2}彰德,今河南省安阳市。
\mnitem{3}一九四五年九月,国民党军队自郑州、新乡一带沿平汉路进攻晋冀鲁豫解放区。十月下旬,其先头三个军,侵入磁县、邯郸地区。解放区军民奋起自卫,经一周激战,国民党第十一战区副司令长官兼新八军军长高树勋率新编第八军等部一万余人,在邯郸地区起义,其余的两个军,在溃退中被人民解放军围歼,放下武器。当时被迫放下武器的高级军官有:国民党第十一战区副司令长官兼第四十军军长马法五,第四十军副军长刘世荣、军参谋长李旭东、副师长刘树森等多人。
\mnitem{4}日本投降以后,国民党调集了三个战区的二十多个师的兵力大举进犯豫、鄂两省解放区。国民党第一战区司令长官胡宗南分兵一部,自西北方向,沿陇海路两侧东犯豫西解放区;第五战区司令长官刘峙所部,沿平汉路两侧,自北向南进犯豫中、鄂中、鄂东解放区;第六战区的部队自鄂南北犯加以配合。以上的国民党军队大都归刘峙指挥。豫鄂解放区的人民军队,对进犯军作了坚决的斗争,保存了实力,于一九四五年十月下旬,转移至豫、鄂交界之大洪山、桐柏山、枣阳地区,后因国民党军队继续追逼,又转移至平汉路东之宣化店地区。
\mnitem{5}这里所列举的一些军官,都是上党战役中被人民解放军俘虏的阎锡山部队的高级将领。关于上党战役,见本卷\mxnote{关于重庆谈判}{2}。
\mnitem{6}绥远,一九五四年撤销,原辖地区划归内蒙古自治区。察哈尔,一九五二年撤销,原辖地区划归河北、山西两省。热河,一九五五年撤销,原辖地区划归河北、辽宁两省及内蒙古自治区。傅作义当时担任国民党第十二战区的司令长官。他的部队,抗日战争时期驻扎在绥远西部的五原、临河一带。日本投降后,他奉命进攻绥远、热河、察哈尔三省解放区。一九四五年八月攻占归绥(今呼和浩特市)、集宁、丰镇。九月初攻占兴和、尚义、武川、陶林、新堂、凉城,大举向察哈尔解放区进攻,迫近张家口。人民解放军起而自卫,将其击退,并俘虏其大批官兵。
\mnitem{7}《剿匪手本》,是一九三三年蒋介石编的专门讲述进攻中国人民军队和革命根据地的方法的反革命小册子。一九四五年抗日战争结束以后,蒋介石重印这个小册子发给国民党军官,并下达密令称:“此次剿匪为人民幸福之所系,务本以往抗战之精神,遵照中正所订《剿匪手本》,督励所属,努力进剿,迅速完成任务。其功于国家者必得膺赏,其迟滞贻误者当必执法以罪。希转饬所属剿匪部队官兵一体悉遵为要。”
\mnitem{8}见本卷\mxnote{关于重庆谈判}{1}。
\end{maonote}
