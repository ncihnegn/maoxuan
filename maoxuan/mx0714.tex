
\title{不许镇压学生运动}
\date{一九六六年八月四日}
\thanks{这是毛泽东同志在中央常委扩大会议上的讲话节选。}
\maketitle


在前清时代,以后是北洋军阀,后来是国民党,都是镇压学生运动的。现在到共产党也镇压学生运动\mnote{1}。这与陆平、蒋南翔\mnote{2}有什么分别?!

中央自己违背自己命令。中央下令停课半年,专门搞文化大革命,等到学生起来了,又镇压他们。不是没有人提过不同意见,就是听不进;另一种意见却是津津有味。说得轻一些,是方向性的问题,实际上是方向问题,是路线问题,是路线错误,违反马克思主义的。这次会议要解决问题,否则很危险。

自己下命令要学生起来革命,人家起来了又加以镇压。所谓走群众路线,所谓相信群众,所谓马列主义等等都是假的。已经是多年如此,凡碰上这类的事情,就爆发出来。明明白白站在资产阶级方面反对无产阶级。说反对新市委就是反党,新市委镇压学生群众,为什么不能反对?!

我是没有下去蹲点的。有人越蹲越站在资产阶级方面反对无产阶级。规定班与班、系与系、校与校之间一概不许来往,这是镇压,是恐怖,这个恐怖来自中央。有人对中央对“六·一八”的批语\mnote{3}有意见,说不好讲。

北大聂元梓等七人的大字报,是二十世纪六十年代的巴黎公社宣言——北京公社。贴大字报是很好的事,应该给全世界人民知道嘛!而雪峰\mnote{4}报告中却说“党有党纪,国有国法,要内外有别,大字报不要贴在大门外给外国人看见。”其实除了机密的地方,例如国防部、公安部等不让外人去看以外,其他地方有什么要紧?在无产阶级专政条件下,也允许群众请愿、示威游行和告状。而且言论、集会、结社、出版自由,是写在宪法上的。从这次镇压学生群众文化大革命的行动看来,我不相信(现在的中央)有真正民主、真正马列主义,而是站在资产阶级方面反对无产阶级文化大革命。

团中央,不仅不支持青年学生运动,反而镇压学生运动,应严格处理。

\begin{maonote}
\mnitem{1}一九六六年六月初到七月底的五十多天中,刘少奇、邓小平主持中央工作,以历来的官办政治运动的方式搞文革运动,方法包括:派工作组、给群众进行“左、中、右”排队、把敢于给领导提意见的人打成“右派”、“反革命”、整他们的黑材料等等,在全国各地发生了如西安交大六·六事件、清华大学六·七事件、广西六·八事件、北京地院六·二〇事件、北师大六·二〇事件、林院《谈话纪要》事件等学生与工作组对抗、工作组和党委镇压学生的普遍现象。仅清华大学就有800多学生被打成“反革命”。
\mnitem{2}陆平,原任北京大学校长、党委书记。蒋南翔,原任清华大学校长、党委书记。两人都镇压本校的文化革命运动。
\mnitem{3}“六·一八”,一九六六年六月十八日,上午九点到十一点,北大工作组正在开会,群众离开工作,揪斗了四十多名校领导人和教授,有些粗暴行动。工作组制止了他们,“明确指出避开工作组乱打乱斗的做法是有害于革命运动的行为,并指出这种做法会被而且已经被坏人利用。”并规定今后“斗争人要经过工作组批准”,见《北京大学文化革命简报(第九号)》。

一九六六年六月二十日,主持中央工作的刘少奇加批语后转发各地:“现将北京大学文化革命简报(第九号)发给你们,中央认为北大工作组处理乱斗的办法是正确的,及时的。各单位如果发生这种现象,都可参照北大的办法处理。”
\mnitem{4}雪峰,指李雪峰,时任北京市委第一书记。聂元梓等七人大字报贴出来后,李雪峰当晚赶到北大,随后召开会议指出:“北大出现了故意泄密的大字报,我很不满意。文化大革命不是你想怎么胡来就怎么胡来,而要在各级党委的领导下,有步骤地进行。贴大字报也要经过批准才行,不要把内部和外部问题的大字报都贴在一起,党内问题贴大字报,涉及到党和国家机密的,不要在外面张贴,要内外有别嘛。”“党有党纪,国有国法,我们搞文化革命不是乌合之众,不能乱七八糟,北大要组织好,炮火要猛,要打中要害,但要有组织,北大的党委要把运动领导好。”并把讲话稿上报中央。
\end{maonote}
