
\title{团结一切抗日力量,反对反共顽固派}
\date{一九四〇年二月一日}
\thanks{这是毛泽东在延安民众讨汪大会上所作的讲演。}
\maketitle


我们延安的各界人民今天在这里开会,为了什么呢?为了声讨卖国贼汪精卫\mnote{1},又是为了团结一切抗日力量,反对反共顽固派。

我们共产党人屡次指出,日本帝国主义的灭华方针是坚决的。不管日本掉换什么内阁,它的灭亡中国把中国变为殖民地的基本方针是不会变更的。中国亲日派大资产阶级的政治代表汪精卫,看了这种情形,吓得发疯,跪倒在日本面前,订立了日汪卖国条约,把中国出卖给日本帝国主义。他还要成立傀儡政府,和抗日政府相对立;还要成立傀儡军队,和抗日军队相对立。他对于反蒋近来不大提了,据说已经改为“联蒋”。反共是日汪的主要目的。他们知道共产党抗日最彻底,国共合作则力量更大,他们就用全力分裂国共合作,使两党各自孤立,最好是两党打起来。这样,他们就利用国民党内部的顽固派,到处放火。在湖南就闹平江惨案\mnote{2},在河南就闹确山惨案\mnote{3},在山西就闹旧军打新军\mnote{4},在河北就闹张荫梧打八路军\mnote{5},在山东就闹秦启荣打游击队\mnote{6},在鄂东就闹程汝怀杀死五六百个共产党员\mnote{7},在陕甘宁边区就闹内部的“点线工作”\mnote{8}、外部的“封锁工作”,并且还准备着军事进攻\mnote{9}。此外,又逮捕了一大批进步青年送进集中营\mnote{10};又雇请玄学鬼张君劢提出取消共产党\mnote{11}、取消陕甘宁边区、取消八路军新四军的反动主张,雇请托洛茨基分子叶青等人做文章骂共产党。所有这些,无非是要破坏抗日的局面,使全国人民都当亡国奴。

这样,汪精卫派和国民党的反共顽固派两家里应外合,把时局闹得乌烟瘴气了。

许多人看了这种情形,非常气愤,就以为抗日没有希望了,国民党都是坏人,都应当反对。我们必须指出,气愤是完全正当的,哪有看了这些严重情形而不气愤的呢?但是抗日仍然是有希望的,国民党里面也不都是坏人。对于各部分的国民党人,应当采取不同的政策。对于那些丧尽天良的坏蛋,对于那些敢于向八路军新四军阵地后面打枪的人,对于那些敢于闹平江惨案、确山惨案的人,对于那些敢于破坏边区的人,对于那些敢于攻打进步军队、进步团体、进步人员的人,我们是决不能容忍的,是必定要还击的,是决不能让步的。因为这类坏蛋,已经丧尽天良,当一个民族敌人深入国土的时候,他们还要闹磨擦,闹惨案,闹分裂。不管他们心里怎样想,他们是在实际上帮助了日本和汪精卫,或者有些人本来就是暗藏的汉奸。对于这些人,如果不加以惩罚,我们就是犯错误,就是纵容汉奸国贼,就是不忠实于民族抗战,就是不忠实于祖国,就是纵容坏蛋来破裂统一战线,就是违背了党的政策。但是这种给投降派和反共顽固派以打击的政策,全是为了坚持抗日,全是为了保护抗日统一战线。因此,我们对于那些忠心抗日的人,对于一切非投降派、非反共顽固派的人们,对于这样的国民党员,是表示好意的,是团结他们的,是尊重他们的,是愿意和他们长期合作以便把国家弄好的。谁如果不这样做,谁也就是违背了党的政策。

这就是我们党的两条政策:一方面,团结一切进步势力,团结一切忠心抗日的人,这是一条政策;一方面,反对一切丧尽天良的坏蛋,反对那些投降派和反共顽固派,这是又一条政策。我们党的这些政策,为了达到一个目的,这就是力争时局好转,战胜日本。我们共产党和全国人民的任务,就是团结一切抗日的进步的势力,抵抗一切投降的倒退的势力,力争时局的好转,挽救时局的逆转。这就是我们的根本方针。我们决不悲观失望,我们是乐观的。我们不怕任何投降派和反共顽固派的进攻,我们一定要粉碎他们,我们也一定能够粉碎他们。中华民族的解放是一定的,中国决不会亡国。中国的进步是一定的,倒退只是暂时的现象。

我们今天开会还要向全国人民表明一种态度,这就是为了抗日,全国人民的团结和进步是必要的。有些人单单强调了抗日,但不愿意强调团结和进步,甚至完全不提团结和进步,这是不对的。没有真正的、坚强的团结,没有迅速的、切实的进步,怎能坚持抗日?国民党的反共顽固派强调统一,但是他们的所谓“统一”,乃是假统一,不是真统一;乃是不合理的统一,不是合理的统一;乃是形式主义的统一,不是实际的统一。他们高唤统一,却原来是要取消共产党、八路军、新四军和陕甘宁边区,说有共产党、八路军、新四军和边区存在,中国就不统一,他们要把全国一切都化为国民党;不但继续他们的一党专政,而且还要扩大他们的一党专政。如果是这样,那还有什么统一呢?老实说,过去如果没有共产党、八路军、新四军和陕甘宁边区真心实意地出来主张停止内战一致抗日,那就无人发起抗日民族统一战线,无人领导和平解决西安事变,那就无从实行抗日。今天如果没有共产党、八路军、新四军、陕甘宁边区和各抗日民主根据地真心实意地出来维持抗日的大局,反对投降、分裂、倒退的危险倾向,那就会弄得一团糟。八路军、新四军几十万人挡住了五分之二的敌人,和四十个日本师团中的十七个师团打\mnote{12},为什么要取消他们呢?陕甘宁边区是全国最进步的地方,这里是民主的抗日根据地。这里一没有贪官污吏,二没有土豪劣绅,三没有赌博,四没有娼妓,五没有小老婆,六没有叫化子,七没有结党营私之徒,八没有萎靡不振之气,九没有人吃磨擦饭\mnote{13},十没有人发国难财,为什么要取消它呢?只有不要脸的人们才说得出不要脸的话,顽固派有什么资格站在我们面前哼一声呢?同志们,当然不能是这样的。不是取消边区,而是全国要学习边区;不是取消八路军、新四军,而是全国要学习八路军、新四军;不是取消共产党,而是全国要学习共产党;不是要进步的人们向落后的人们看齐,而是要落后的人们向进步的人们看齐。我们共产党是最主张统一的人,我们发起了统一战线,我们坚持了统一战线,我们提出了统一的民主共和国的口号。谁人能够提出这些呢?谁人能够实行这些呢?谁人能够只要每月五块钱薪水\mnote{14}呢?谁人能够创造这样的廉洁政治呢?统一,统一,投降派有一套统一论,要我们统一于投降;反共顽固派有一套统一论,要我们统一于分裂,统一于倒退。我们能够信这些道理吗?不以抗战、团结、进步三件事做基础的统一,算得真统一吗?算得合理的统一吗?算得实际的统一吗?真是做梦!我们今天开大会,就是要提出我们的统一论。我们的统一论,就是全国人民的统一论,就是一切有良心的人的统一论。这种统一论是以抗战、团结、进步三件事做基础的。只有进步才能团结,只有团结才能抗日,只有进步、团结、抗日才能统一。这就是我们的统一论,这就是真统一论,这就是合理的统一论,这就是实际的统一论。那种假统一论,不合理的统一论,形式主义的统一论,乃是亡国的统一论,乃是丧尽天良的统一论。他们要把共产党、八路军、新四军和民主的抗日根据地消灭,要把一切地方的抗日力量消灭,以便统一于国民党。这是阴谋,这是借统一之名,行专制之实,挂了统一这个羊头,卖他们的一党专制的狗肉,死皮赖脸,乱吹一顿,不识人间有羞耻事。我们今天开大会,就要戳破他们的纸老虎,我们要坚决地反对反共顽固派。


\begin{maonote}
\mnitem{1}见本书第一卷\mxnote{论反对日本帝国主义的策略}{31}。
\mnitem{2}见本卷\mxnote{必须制裁反动派}{1}。
\mnitem{3}确山惨案,也称竹沟惨案、竹沟事变。一九三九年十一月十一日,河南省确山、信阳、泌阳等县的国民党反动武装一千八百余人,围攻确山县竹沟镇新四军留守处,惨杀因抗日受伤的新四军干部、战士和他们的家属以及当地群众共二百余人。
\mnitem{4}旧军,指国民党山西地方实力派阎锡山指挥的晋绥军。新军,指抗日战争初期,由中国共产党人在与阎锡山建立统一战线的过程中组建和领导的、以山西青年抗敌决死队为主力的山西人民的抗日武装。一九三九年十二月,阎锡山在蒋介石掀起的第一次反共高潮中,与日本侵略军相勾结,集中晋绥军六个军的兵派力,向驻在山西西部的新军进攻。新军奋起反击,在八路军的支援下,粉碎了旧军的进攻。同时,在山西的东南部,旧军又同国民党中央军中的顽固派相配合进攻新军,摧毁沁水、阳城、晋城、高平、长治等县的抗日民主政权和人民团体,屠杀了大批的共产党员和进步分子。
\mnitem{5}张荫梧,当时任国民党河北省民军总指挥。从一九三八年以来,他就在国民党当局的指令下,不断制造磨擦,进攻八路军。一九三九年六月,他率部袭击河北深县八路军的后方机关,惨杀八路军干部和战士四百余人。
\mnitem{6}秦启荣,当时任国民党军事委员会别动总队第五纵队司令。在国民党当局的指使下,他不断地制造同八路军的磨擦。一九三九年三月三十日,他的部队在山东博山太河镇伏击八路军山东纵队第三支队南下受训干部及护送部队,逮捕和杀害团级以下干部二百余人。到一九四〇年春,他已经残杀了八路军在山东的游击队和地方工作人员七百余人。
\mnitem{7}程汝怀,当时任国民党湖北省政府鄂东行署主任、第五战区鄂东游击总指挥。从一九三九年六月至九月,他先后多次调集部队,围攻新四军在鄂东的游击部队和后方机关,惨杀共产党员五六百人。
\mnitem{8}一九三九年后,国民党派遣特务间谍进入陕甘宁边区,发展秘密党员、特务,建立据点,设置情报网,并通过若干秘密交通线进行联系。这种反革命的特务活动,他们自称为“点线工作”。
\mnitem{9}在一九三九年初至一九四〇年六月,国民党军队不断向陕甘宁边区发动军事进攻,占领了边区所属的淳化、旬邑、正宁、宁县、镇原五座县城。
\mnitem{10}在抗日战争期间,国民党反动派模仿德意法西斯的办法,从中国西北的兰州、西安至东南的赣州、上饶等地,设立了很多集中营,用以囚禁大批的共产党员、爱国人士和进步青年。
\mnitem{11}见本卷\mxnote{新民主主义论}{19}。
\mnitem{12}中国共产党领导的军队所抗击的日本侵略军的数目,在后来有了变动。到一九四三年,八路军、新四军以及其它人民武装抗击了侵华日军总数的百分之六十四和全部伪军的百分之九十五。参见本书第三卷\mxart{论联合政府}一文的《两个战场》一节。
\mnitem{13}“吃磨擦饭”,即是说有些国民党人以反共为专门职业。
\mnitem{14}当时共产党领导下的抗日军队和抗日政府的工作人员,每人每月的伙食费和津贴费平均为银币五元。
\end{maonote}
