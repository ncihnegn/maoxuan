
\title{在听了氢弹会议汇报后的讲话}
\date{一九六七年六月、七月}
\thanks{这是毛泽东同志听取氢弹工作汇报后的两次谈话。}
\maketitle


\date{一九六七午六月十八日}
\section*{(一)}

有些外国人对我们《北京周报》、新华社的对外宣传有意见,宣传毛泽东思想发展了马克思主义。过去不搞,现在文化大革命以后大搞特搞,吹得太厉害,人家也接受不了。有些话何必要自己来说,我们要谦虚,特别是对外,出去要谦虚一点,当然就不要失掉原则。昨天氢弹公报\mnote{1},我就把“伟大的导师、伟大的领袖、伟大的统帅、伟大的舵手”统统勾掉了。“万分喜悦和激动的心情”我把“万分”也勾掉了,不是十分,也不是百分,也不是千分,而是万分,我一分也不要,统统勾掉了。

\date{一九六七年七月七日}
\section*{二}

新武器,导弹、原子弹搞得很快,两年零八个月出氢弹,我们的发展速度超过了美国、英国、苏联、法国,现在在世界上是第四位。导弹、原子弹有很大成绩,这是赫鲁晓夫“帮忙”的结果,撤走了专家,逼着我们走自己的路,要给他一吨重的勋章。

\begin{maonote}
\mnitem{1}一九六七年六月十七日,新华社向全世界发布了《新闻公报》,全文如下:

毛泽东主席早在一九五八年六月就指出:搞一点原子弹、氢弹,我看有十年功夫完全可能。

在我国无产阶级文化大革命取得决定性胜利的凯歌声中,我们向全国人民和全世界人民庄严宣布:毛主席的这一英明预言和伟大号召已经实现了,在两年八个月的时间内进行了五次核试验之后,今天,一九六七年六月十七日,中国的第一颗氢弹在中国的西部地区上空爆炸成功了。

这次氢弹试验的成功,是中国核武器发展的又一个飞跃,标志着我国核武器的发展进入了一个崭新的阶段。中国人民为此而自豪,全世界革命人民也将为此而感到骄傲。我们怀着喜悦和激动的心情,欢呼毛泽东思想的又一伟大胜利,欢呼无产阶级文化大革命的又一辉煌成果。

中国共产党中央委员会、国务院、中央军事委员会、中央文化革命小组,向从事核武器研制和试验的中国人民解放军全体指战员、工人、工程技术人员、科学工作者和一切有关人员,致以最热烈的祝贺。他们在党中央、毛主席和他的亲密战友林彪同志的正确领导下,高举毛泽东思想伟大红旗,突出无产阶级政治,活学活用毛主席著作,坚决捍卫以毛主席为代表的无产阶级革命路线,坚决反对党内最大的一小撮走资本主义道路当权派的修正主义路线,抓革命,促生产,群策群力,大力协同,以“只争朝夕”的革命精神,克服了各种困难,闯出一条自己的道路,保证了这次氢弹试验的圆满成功。

毛主席说:“在生产斗争和科学实验范围内,人类总是不断发展的,自然界也总是不断发展的,永远不会停止在一个水平上。因此,人类总得不断地总结经验,有所发现,有所发明,有所创造,有所前进。”希望中国人民解放军、广大的革命职工和科学技术人员,遵循毛主席的这一教导,响应林彪同志关于“加强革命性、科学性、组织纪律性”的号召,戒骄戒躁,再接再厉,为加速发展我国国防科学技术和实现我国国防现代化,建立新的更加伟大的功勋。

中国有了原子弹,有了导弹,现在又有了氢弹,这就大长世界各国革命人民的志气,大灭帝国主义、现代修正主义和各国反动派的威风。中国氢弹试验的成功,进一步打破了美帝国主义和苏联修正主义的核垄断地位,沉重地打击了它们的核讹诈政策。中国氢弹试验的成功,对于正在英勇地进行抗美救国战争的越南人民,对于正在反抗美英帝国主义及其工具以色列侵略的阿拉伯人民,对于全世界一切革命的人民,都是极大的鼓舞和支持。

人是战争胜负的决定因素。中国进行必要而有限制的核试验,发展核武器,完全是为了防御,其最终目的就是为了消灭核武器。我们再一次郑重宣布,在任何时候、任何情况下,中国都不会首先使用核武器。我们说的话,从来是算数的。中国人民和中国政府,将一如既往地继续同全世界一切爱好和平的人民和国家一道,共同努力,坚持斗争,为全面禁止和彻底销毁核武器的崇高目标而奋斗。
\end{maonote}
