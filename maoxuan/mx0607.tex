
\title{介绍一封信\mnote{1}}
\date{一九五八年七月三日、三十日}
\thanks{本篇是毛泽东对中共广东省委书记处书记赵紫阳一九五八年六月八日给广东省委的信写的三个批语。标题是毛泽东在第一个批语上拟的。}
\maketitle


\section*{一}

广东省委书记赵紫阳同志最近率领北路检查团到从化县,经四天工作,给省委写了一封信,提出了三个问题:一、对早造\mnote{2}生产的看法问题;二、群众路线问题;三、大字报问题。这些都是全国带普遍性的重要问题,值得一切从中央到基层的领导同志们认真一阅。红旗半月刊应当多登这样的通信。这封信的风格脱去了知识分子腔,使人高兴看下去。近来的文章和新闻报导,知识分子腔还是不少,需要改造。这封信在广东党内刊物上发表,由新华通讯社当作一份党内文件发到北京的。其实,这类通讯或文章,完全可以公开发表,无论对当地同志和全党同志都有极大好处。我同意赵紫阳同志的意见,早造每亩么能收300斤已经很好,比去年的200斤增长50\%,何况还有350—400斤的希望。原先的800斤指标是高了,肥料和深耕两个条件跟不上去。这是由于缺乏经验,下半年他们就有经验了。对于这件事,从化的同志们感到难受,这种难受将促使他们取得经验,他们一定会大进一步。群众路线问题,仍然是一个值得全党注意的问题。其办法是从全省各县、全县各乡中,经过鉴定,划分为对于群众路线执行得很好的,执行得不很好也不很坏处于中间状态的和执行得很坏的这样三大类,加以比较,引导第二、第三两类都向第一类看齐,到第一类县乡去开现场会议,可以逐步地解决这个问题。这个问题,不但农村有,城市也有,故是全党性的问题,仍然需要采取大鸣大放大辩论大字报的方法去解决。

毛泽东

一九五八年七月三日

\section*{二}

\mxname{小平、彭真、震林、伯达\mnote{3}同志:}

你们看这封信是否可以发表?我看发表毫无害处。请伯达打电话给广东省委,问一下这封信是否已在党内刊物上发表,或者是用单个文件发表(到)各县,或者并没有发去?再则告诉他们,我们拟在红旗上发表,他们意见如何?以其结果告我为盼!

毛泽东

六(七)月三日上午七时

请在七日下午退给陈伯达。

\section*{三}

\mxname{陈伯达同志:}

此事请你处理,我来不及了。

毛泽东

七月卅日

\begin{maonote}
\mnitem{1}赵紫阳的信后来发表在一九五八年八月十六日出版的《红旗》第六期上,题为《从化四日——给广东省委的一封信》。
\mnitem{2}早造,即早稻,又叫上造。
\mnitem{3}小平,即邓小平,时任中共中央总书记。彭真,时任中共中央书记处书记。震林,即谭震林,时任中共中央书记处书记。伯达,即陈伯达,时任《红旗》杂志总编辑。
\end{maonote}
