
\title{中共中央关于九月会议的通知}
\date{一九四八年十月十日}
\thanks{这是毛泽东为中共中央起草的对党内的通知。一九四八年九月会议是在河北省平山县西柏坡村召集的。它是日本投降以来到会人数最多的一次中央会议,因为在这以前,绝大多数中央委员都分散在各个解放区从事紧张的解放战争,交通十分困难,不可能举行这样大的会议。}
\maketitle


(一)一九四八年九月,中央召开了一次政治局会议,到政治局委员七人,有中央委员和候补中央委员十四人、重要工作人员十人参加,其中有华北、华东、中原、西北的党和军队的主要负责同志。这是从日本投降以来到会人数最多的一次中央会议。会议检查了过去时期的工作,规定了今后时期的工作任务。

(二)一九四五年四月党的第七次全国代表大会以后,中央委员会和全党领导骨干,表现了比较抗日时期更为良好的团结。这种团结,使得我党能够应付日本投降以后整三年内国际国内所发生的许多重大事变,并在这些事变中使中国革命向前推进了一大步,摧毁了美帝国主义在中国广大人民中的政治影响,抵抗了国民党的再一次叛变\mnote{1},打退了它的军事进攻,使人民解放军由防御转到了进攻。

在一九四六年七月至一九四八年六月的两年作战中,人民解放军歼敌二百六十四万人,其中俘敌一百六十三万人。两年主要缴获,计有步枪近九十万枝,重轻机枪六万四千余挺,小炮八千余门,步兵炮五千余门,山野重炮一千一百余门。两年中人民解放军由一百二十余万人增加到了二百八十万人。其中正规军由一百一十八个旅增加到了一百七十六个旅,正规军人数由六十一万增加到了一百四十九万。解放区现有面积二百三十五万平方公里,占全国面积九百五十九万七千平方公里的百分之二十四点五;现已有人口一亿六千八百万,占全国人口四亿七千五百万的百分之三十五点三;现有县城以上大中小城市五百八十六座,占全国城市二千零九座的百分之二十九。

由于我党坚决领导农民实现了土地制度的改革,现已在大约一万万人口的区域彻底解决了土地问题,地主阶级和旧式富农的土地大致平均地分配给了农村人民,首先是贫雇农。

我党党员由一九四五年五月的一百二十一万,增加到了现在的三百万(我党党员一九二七年国民党叛变以前为五万人,一九二七年国民党叛变以后降为大约一万人左右,一九三四年因土地革命顺利发展升至三十万人,一九三七年因南方革命失败\mnote{2}降为大约四万人左右,一九四五年因抗日战争顺利发展增至一百二十一万人,现在因反蒋战争和土地革命顺利发展又增至三百万人)。党在最近一年内,一方面基本上克服了并正在继续克服着党内在某种程度上存在着的成分不纯(地主富农分子)、思想不纯(地主富农思想)和作风不纯(官僚主义和命令主义)的不良现象,另一方面又克服了和正在继续克服着跟着大规模发动农民群众解决土地问题的斗争而产生的,部分地但是相当多地侵犯了中农,破坏了某些私人工商业,以及某些地方越出了镇压反革命的某些政策界限等项“左”的错误。经过过去三年、特别是最近一年的伟大的激烈的革命斗争,和对于自己错误的认真的纠正,全党的政治成熟程度是大进一步了。

党在国民党区域的工作,有了很大的成绩,这表现在各大城市中争取了广大的工人、学生、教员、教授、文化人、市民和民族资本家站在我党方面,争取了一切民主党派、人民团体站在我党方面,抗拒了国民党的压迫,使国民党完全陷于孤立。在南方几个大区域内(闽粤赣边区,湘粤赣边区,粤桂边区,桂滇边区,云南南部,皖浙赣边区和浙江东部南部)建立了游击战争根据地,使这些地区的游击部队发展到了三万余人。

两年内,特别是最近一年内,在人民解放军中,实行了有秩序的、有领导的、由全体战斗员和指挥员一起参加的民主运动,开展了自我批评,克服了和正在继续克服着军队中的官僚主义,恢复了在一九二七年至一九三二年期间曾经实行有效、而在后来被取消了的军队中的各级党委制和连队中的战士委员会制,这样就使军队指战员的政治积极性和自觉性大为提高,战斗力和纪律性大为增强,溶化了大约八十万左右从国民党军队来的俘虏兵,使他们变为解放战士\mnote{3},掉转枪口打国民党。两年内,从解放区动员了大约一百六十万左右分得了土地的农民参加人民解放军。

我们现在已经有了相当多的铁路、矿山和工业,我党正在大规模地学习管理工业和做生意。两年内,我们的军事工业,有了相当大的增长。但是还不足以应付战争的需要。我们缺乏若干重要的原料和机器,我们基本上还不能炼钢。

我们已在华北四千四百万人口的区域建立了统一的党和党外民主人士合作的人民政府,并决定由这个政府将华北、华东(有人口四千三百万)和西北(有人口七百万)三区的经济、财政、贸易、金融、交通和军事工业的领导和管理工作统一起来,以利支持前线,并且准备在不久的将来,将东北和中原两区的上述工作也统一起来。

(三)中央会议,根据过去两年作战的成绩和整个敌我形势,认为建设五百万人民解放军,在大约五年左右的时间内(从一九四六年七月算起)歼敌正规军共五百个旅(师)左右(平均每年一百个旅左右),歼敌正规军、非正规军和特种部队共七百五十万人左右(平均每年一百五十万人左右),从根本上打倒国民党的反动统治,是有充分可能性的。

国民党的军事力量,在一九四六年七月为四百三十万人,两年被歼和逃亡三百零九万人,补充二百四十四万人,现有三百六十五万人。估计今后三年尚能补充三百万人,今后三年被歼和逃亡可能达到四百五十万人左右。这样,五年作战结果,国民党的军事力量可能只剩下二百万人左右了。我军现有二百八十万人,今后三年准备收容俘虏参加我军一百七十万人(以占俘虏全数百分之六十计算),动员农民参军二百万人,除去消耗,五年作战结果,我军可能接近五百万人。如果五年作战出现了这样的结果,就可以说国民党的反动统治已经从根本上被我们打倒了。

为了实现这一任务,必须每年歼敌正规军一百个旅(师)左右,五年共歼敌正规军五百个旅(师)左右。这是解决一切问题的关键。我们第一年歼敌正规军折合成九十七个旅(师),第二年歼敌正规军折合成九十四个旅(师),根据这一情形看来,这样的目标是可能达到并且可能超过的。国民党现有全部军事力量三百六十五万人中的百分之七十是在第一线(长江和巴山山脉之线以北,兰州和贺兰山脉之线以东,承德和长春之线以南),在其后方者(包括长江和巴山山脉之线以南,兰州和贺兰山脉之线以西)仅有大约百分之三十。国民党现有全部正规军二百八十五个旅,一百九十八万人,其中在第一线者二百四十九个旅,一百七十四万二千人(北线九十九个旅,六十九万四千人,南线一百五十个旅,一百零四万八千人),在其后方者,仅有三十六个旅,二十三万八千人,并且大部分是新建立的部队,缺乏战斗力。因此中央决定人民解放军第三年仍然全部在长江以北和华北、东北作战。为着执行歼敌任务,除有计划地谨慎地从解放区动员人民参军外,必须大量利用俘虏。

(四)由于我党我军在过去长时期内是处于被敌人分割的、游击战争的并且是农村的环境之下,我们曾经允许各地方党的和军事的领导机关保持着很大的自治权,这一种情况,曾经使得各地方的党组织和军队发挥了他们的自动性和积极性,渡过了长期的严重的困难局面,但在同时,也产生了某些无纪律状态和无政府状态,地方主义和游击主义,损害了革命事业。目前的形势,要求我党用最大的努力克服这些无纪律状态和无政府状态,克服地方主义和游击主义,将一切可能和必须集中的权力集中于中央和中央代表机关\mnote{4}手里,使战争由游击战争的形式过渡到正规战争的形式。过去两年中,军队和作战的正规性是增长了一步,但是还不够,必须在第三年内再进一大步。为此目的,必须尽一切可能修理和掌握铁路、公路、轮船等近代交通工具,加强城市和工业的管理工作,使党的工作的重心逐步地由乡村转到城市。

(五)夺取全国政权的任务,要求我党迅速地有计划地训练大批的能够管理军事、政治、经济、党务、文化教育等项工作的干部。战争的第三年内,必须准备好三万至四万下级、中级和高级干部,以便第四年内军队前进的时候,这些干部能够随军前进,能够有秩序地管理大约五千万至一万万人口的新开辟的解放区。中国地方甚大,人口甚多,革命战争发展甚快,而我们的干部供应甚感不足,这是一个很大的困难。第三年内干部的准备,虽然大部分应当依靠老的解放区,但是必须同时注意从国民党统治的大城市中去吸收。国民党区大城市中有许多任务人和知识分子能够参加我们的工作,他们的文化水准较之老解放区的工农分子的文化水准一般要高些。国民党经济、财政、文化、教育机构中的工作人员,除去反动分子外,我们应当大批地利用。解放区的学校教育工作,必须恢复和发展。

(六)召集政治协商会议的口号\mnote{5},团结了国民党区域一切民主党派、人民团体和无党派民主人士于我党周围。现在,我们正在组织国民党区域的这些党派和团体的代表人物来解放区,准备在一九四九年召集中国一切民主党派、人民团体和无党派民主人士的代表们开会,成立中华人民共和国临时中央政府。

(七)恢复和发展解放区的工业生产和农业生产,是支援战争、战胜国民党反动派的重要环节。中央会议认为,必须一方面使人民解放军向国民党区域发展胜利的进攻,将战争所需要的人力资源和物力资源大量地从国民党方面和国民党区域去取给;另一方面,必须用一切努力恢复和发展老解放区的工业生产和农业生产,使之较现有的水平有若干的增长。只有这两方面的任务都完成了,才能够保证打倒国民党反动统治,否则是不可能的。

执行这两方面的任务,我们有很多的困难。大军进入国民党区域执行无后方的或半有后方的作战,一切军事需要必须全部地或大部地就地自己解决。而恢复和发展工业生产和农业生产则需要有较好的组织工作,很好地领导解放区内部的市场和管制对外贸易,解决某些机器和原料缺乏的问题,首先是解决交通运输和修理铁路、公路、河道的问题。目前解放区的经济状况和财政状况,存在着很大的困难,虽然我们的困难比较国民党的困难要小得多,但是确实有困难。这主要是物资和兵员不足供应战争的需要,通货膨胀已到了相当大的程度,而我们的组织工作特别是财经方面的组织工作不够,则是形成这种困难的原因之一。我们相信这些困难是能够克服的,并且必须克服这些困难。在克服困难的斗争中,必须反对浪费,厉行节约:在前线注意缴获归公,爱护自己的有生力量,爱护武器,节省弹药,保护俘虏;在后方,减少国家机构的开支,减少不急需的人力和畜力的动员,减少开会时间,注意农业的季节,不违农时,节省工业生产的成本,提高劳动生产率,全党动员学习管理工业生产、农业生产和做生意,尽可能地将各解放区的经济加以适当的组织,克服市场上的盲目性,并同一切投机操纵的分子进行必要的斗争。从这一切着手,我们就必能克服自己面前的困难。

(八)提高干部的理论水平,扩大党内的民主生活,成为完成上述任务的重要环节。中央会议已通过关于扩大党内民主生活的专门决议\mnote{6}。关于提高干部理论水平的问题也进行了讨论,并引起了到会同志的注意。

(九)全国第六次劳动大会已经胜利地召开,并成立了中华全国总工会\mnote{7}。明年上半年,将召开全国妇女代表大会,成立全国民主妇女联合会\mnote{8};将召开全国青年代表大会,成立全国青年联合会\mnote{9};并将建立新民主主义青年团\mnote{10}。


\begin{maonote}
\mnitem{1}国民党的第一次叛变是在一九二七年。这里所说的“再一次叛变”,是指抗日战争结束后国民党发动反革命的全面内战。
\mnitem{2}南方革命失败,指以陈绍禹为代表的“左”倾冒险主义错误造成的一九三四年红军第五次反“围剿”战争的失败,和红军主力退出南方各革命根据地。
\mnitem{3}指被人民解放军俘虏而从国民党反动军队中解放出来、经过教育、参加人民解放军的原国民党军士兵。
\mnitem{4}这里所说的中央代表机关,是指中央局和中央分局。
\mnitem{5}召开政治协商会议的口号是毛泽东提出的。在一九四八年四月三十日中共中央发布的《纪念“五一”劳动节口号》中,根据毛泽东的提议,提出:“各民主党派、各人民团体、各社会贤达迅速召开政治协商会议,讨论并实现召集人民代表大会,成立民主联合政府。”这个口号立即得到各民主党派、人民团体和无党派民主人士的热烈响应。政治协商会议,后来称为新政治协商会议,以后又改称中国人民政治协商会议,参见本卷\mxnote{在新政治协商会议筹备会上的讲话}{1}。
\mnitem{6}指《中共中央关于召开党的各级代表大会和代表会议的决议》。这个决议,对于建立和扩大党内的正常民主生活问题作了以下的规定:各级党委要按照党章定期召开党的各级代表大会和代表会议。对于这种会议要赋予党章所规定的一切权力,不许侵犯。开会要有充分的准备。党内有不同意见的争论要及时地、真实地向上级报告,其中重要的争论并须报告中央。此外,还规定要健全党委制,各级党委必须实行重要问题经党委集体讨论和作出决定的制度,不应当由个人决定重要问题,但是集体领导和个人负责不可偏废。
\mnitem{7}全国第六次劳动大会,是一九四八年八月在哈尔滨召开的。在这次大会上,恢复了中国工人阶级统一的全国组织中华全国总工会。前五次全国劳动大会,先后举行于一九二二年、一九二五年、一九二六年、一九二七年、一九二九年。
\mnitem{8}中国妇女第一次全国代表大会,是一九四九年三月在北平召开的。在这次大会上,成立了全国妇女群众组织的领导机构中华全国民主妇女联合会。一九五七年九月中国妇女第三次全国代表大会决定,中华全国民主妇女联合会改称中华全国妇女联合会。
\mnitem{9}中华全国青年第一次代表大会,是一九四九年五月在北平召开的。在这次大会上,成立了中华全国民主青年联合总会。一九五八年四月中华全国青年第三次代表大会决定,中华全国民主青年联合总会改称中华全国青年联合会。
\mnitem{10}中国新民主主义青年团,是一九四九年一月由中共中央正式决定建立的。它的第一次全国代表大会,于一九四九年四月在北平召开。一九五七年五月团的第三次全国代表大会决定,中国新民主主义青年团改名为中国共产主义青年团。
\end{maonote}
