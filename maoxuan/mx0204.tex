
\title{国共合作成立后的迫切任务}
\date{一九三七年九月二十九日}
\maketitle


还在一九三三年,中国共产党就发表了在停止进攻红军、给民众以自由和武装民众三个条件之下,准备同任何国民党部队订立抗日协定的宣言。那是因为在一九三一年九一八事变发生后,中国人民的首要任务已经是反对日本帝国主义进攻中国了。但是我们的目的没有达到。

一九三五年八月,中国共产党和中国红军号召各党各派和全国同胞组织抗日联军和国防政府,共同反对日本帝国主义\mnote{1}。同年十二月,中国共产党通过了同民族资产阶级组织抗日民族统一战线的决议\mnote{2}。一九三六年五月,红军又发表了要求南京政府停止内战一致抗日的通电\mnote{3}。同年八月,中国共产党中央委员会又对国民党中央委员会送了一封信,要求国民党实行停战,并组织两党的统一战线,共同反对日本帝国主义\mnote{4}。同年九月,共产党又作了在中国建立统一的民主共和国的决议\mnote{5}。不但发了这些宣言、通电、书信和决议,而且派遣了自己的代表,多次和国民党方面进行谈判,然而还是没有结果。直至西安事变\mnote{6}发生,在一九三六年年底,中国共产党的全权代表才同国民党的主要负责人取得了在当时政治上的一个重要的共同点,即是两党停止内战,并实现了西安事变的和平解决。这是中国历史上的一件大事,从此建立了两党重新合作的一个必要的前提。

今年二月十日,当国民党三中全会的前夜,中国共产党中央为了具体地建立两党合作,乃以一个系统的建议电告该会\mnote{7}。在这个电报内,要求国民党向共产党保证停止内战,实行民主自由,召开国民大会,迅速准备抗日和改良人民生活等五项;共产党也向国民党保证取消两个政权敌对,红军改变名称,在革命根据地实行新民主制度和停止没收地主的土地等四项。这也是一个重要的政治步骤,因为如果没有这一步骤,则两党合作的建立势将推迟,而这对于迅速准备抗日是完全不利的。

自此以后,两党的谈判接近了一步。关于两党共同的政治纲领问题,要求开放民众运动和释放政治犯问题,红军改名问题等,共产党方面都提出了更具体的建议。虽然共同纲领的颁布,民众运动的开放,革命根据地的新制度的承认等事,至今还没有实现;然而红军改名为国民革命军第八路军(按抗日战线的战斗序列,又称第十八集团军)的命令,已在平津失守约一个月之后颁布了。还在七月十五日就已交付了国民党的中国共产党中央为宣布两党合作成立的宣言,以及当时约定随之发表的蒋介石氏承认中国共产党的合法地位的谈话,虽延搁太久,未免可惜,也于九月二十二日和二十三日,正当前线紧张之际,经过国民党的中央通讯社,先后发表了。共产党的这个宣言和蒋介石氏的这个谈话,宣布了两党合作的成立,对于两党联合救国的伟大事业,建立了必要的基础。共产党的宣言,不但将成为两党团结的方针,而且将成为全国人民大团结的根本方针。蒋氏的谈话,承认了共产党在全国的合法地位,指出了团结救国的必要,这是很好的;但是还没有抛弃国民党的自大精神,还没有必要的自我批评,这是我们所不能满意的。但是不论如何,两党的统一战线是宣告成立了。这在中国革命史上开辟了一个新纪元。这将给予中国革命以广大的深刻的影响,将对于打倒日本帝国主义发生决定的作用。

中国的革命,自从一九二四年开始,就由国共两党的情况起着决定的作用。由于两党在一定纲领上的合作,发动了一九二四年至一九二七年的革命。孙中山先生致力国民革命凡四十年还未能完成的革命事业,在仅仅两三年之内,获得了巨大的成就,这就是广东革命根据地的创立和北伐战争的胜利。这是两党结成了统一战线的结果。然而由于一部分人对于革命主义未能坚持,正当革命走到将次完成之际,破裂了两党的统一战线,招致了革命的失败,外患乃得乘机而入。这是两党统一战线破裂了的结果。现在两党重新结成的统一战线,形成了中国革命的一个新时期。尽管还有某些人还不明了这个统一战线的历史任务及其伟大的前途,还在认为结成这个统一战线不过是一个不得已的敷衍的临时的办法,然而历史的车轮将经过这个统一战线,把中国革命带到一个崭新的阶段上去。中国是否能由如此深重的民族危机和社会危机中解放出来,将决定于这个统一战线的发展状况。新的有利的证据已经表现出来了。第一个证据,是还在中国共产党开始提出统一战线政策的时候,就立即得到了全国人民的赞同。人心的向背,于此可见。第二个证据,是西安事变和平解决,两党实行停战以后,立即引起了国内各党各派各界各军进入了前所未有的团结状况。虽然这个团结对于抗日的需要说来还是异常不够的,特别是政府和人民之间的团结问题至今在基本上还没有解决。第三个证据,这是最为显着的,就是全国性抗日战争的发动。这个抗战,就目前的情况说来,我们是不能满意的,因为它虽然是全国性的,却还限制于政府和军队的抗战。我们早已指出,这样的抗战是不能战胜日本帝国主义的。虽然如此,但是确实已经发动了百年以来未曾有过的全国范围的对外抗战,没有国内和平和两党合作这是做不到的。如果说当两党统一战线破裂的时候,日寇可以不费一弹而得东北四省\mnote{8},那末,当两党统一战线重新建立了的今日,日寇就非经过血战的代价不能得到中国的土地。第四个证据,就是对国际的影响。全世界工农民众和共产党,都拥护中国共产党提出的抗日统一战线的主张。国共合作成立后,各国人民,特别是苏联,将更积极地援助中国。中苏已签订了互不侵犯条约\mnote{9},今后两国关系有更进一步的希望。根据上述的这些证据,我们可以判断,统一战线的发展,将使中国走向一个光明的伟大的前途,就是日本帝国主义的打倒和中国统一的民主共和国的建立。

然而这样伟大的任务,不是停止在现在状况的统一战线所能完成的。两党的统一战线还需要发展。因为现在成立的统一战线,还不是一个充实的坚固的统一战线。

抗日民族统一战线是否只限于国共两个党的呢?不是的,它是全民族的统一战线,两个党仅是这个统一战线中的一部分。抗日民族统一战线是各党各派各界各军的统一战线,是工农兵学商一切爱国同胞的统一战线。现在的统一战线事实上还停止在两个党的范围之内,广大的工人、农民、兵士、城市小资产阶级及其它许多爱国同胞还没有被唤起,还没有被发动,还没有组织起来和武装起来。这是目前的最严重的情形。它的严重性,就是影响到前线不能打胜仗。华北以至江浙前线的严重危机,现在已经不能掩饰,也无须掩饰了,问题是怎样挽救这个危机。挽救危机的唯一道路,就是实行孙中山先生的遗嘱,即“唤起民众”四个字。孙先生临终时的这个遗嘱,说他是积四十年的经验,深知必须这样做,才能达到革命的目的。究竟根据什么理由一定不肯实行这个遗嘱?究竟根据什么理由在如此危急存亡的关头还不下决心实行这个遗嘱?谁也明白,统制、镇压,是和“唤起民众”的原则相违背的。单纯的政府和军队的抗战,是决然不能战胜日本帝国主义的。我们还在今年五月间,就对于这个问题大声疾呼地警告过当权的国民党,指出了没有民众起来抗战,就会蹈袭阿比西尼亚的覆辙。不但中国共产党人,各地的许多先进同胞以及国民党的许多贤明的党员,都曾指出了这一点。可是统制政策依然没有改变。其结果就是政府和人民隔离,军队和人民隔离,军队中指挥员和战斗员隔离。统一战线没有民众充实起来,前线危机就无可避免地只会增大,不会缩小。

今天的抗日统一战线,还没有一个为两党所共同承认和正式公布的政治纲领,去代替国民党的统制政策。现在国民党对待民众的一套,还是十年来的一套,从政府机构,军队制度,民众政策,到财政、经济、教育等项政策,大体上都还是十年来的一套,没有起变化。起了变化的东西是有的,并且是很大的,这就是停止内战,一致抗日。两党的内战停止了,全国的抗日战争起来了,这是从西安事变以来中国政局的极大的变化。然而上述的一套则至今没有变化,这叫做没有变化的东西和变化了的东西不相适应。过去的一套仅适用于对外妥协和对内镇压革命,现在还是用了这一套去对付日本帝国主义的进攻,所以处处不适合,各种弱点都暴露出来。不干抗日战争则已,既然要干了,并且已经干起来了,又已经暴露出严重的危机了,还不肯改换一套新的干法,前途的危险是不堪设想的。抗日需要一个充实的统一战线,这就要把全国人民都动员起来加入到统一战线中去。抗日需要一个坚固的统一战线,这就需要一个共同纲领。共同纲领是这个统一战线的行动方针,同时也就是这个统一战线的一种约束,它像一条绳索,把各党各派各界各军一切加入统一战线的团体和个人都紧紧地约束起来。这才能说得上坚固的团结。我们反对旧的那一套约束,因为它不适应于民族革命战争。我们欢迎建立一套新的约束代替旧的,这就是颁布共同纲领,建立革命秩序。必须如此,才能适应抗日战争。

共同纲领是什么呢?这就是孙中山先生的三民主义和共产党在八月二十五日提出的抗日救国十大纲领\mnote{10}。

中国共产党在公布国共合作的宣言上说:“孙中山先生的三民主义为中国今日之必需,本党愿为其彻底实现而奋斗。”若干人们对于共产党愿意实行国民党的三民主义觉得奇怪,如像上海的诸青来\mnote{11},就是在上海的刊物上提出这种疑问的一个。他们以为共产主义和三民主义是不能并存的。这是一种形式主义的观察。共产主义是在革命发展的将来阶段实行的,共产主义者在现在阶段并不梦想实行共产主义,而是要实行历史规定的民族革命主义和民主革命主义,这是共产党提出抗日民族统一战线和统一的民主共和国的根本理由。说到三民主义,还在十年前两党的第一次统一战线时,共产党和国民党就已经经过国民党第一次全国代表大会而共同决定加以实行,并且已经在一九二四年至一九二七年,经过每一个忠实的共产党人和每一个忠实的国民党人的手,在全国很大的地区中实行过了。不幸在一九二七年统一战线破裂,从此产生了国民党方面十年来反对实行三民主义的局面。然而在共产党方面,十年来所实行的一切政策,根本上仍然是符合于孙中山先生的三民主义和三大政策的革命精神的。共产党没有一天不在反对帝国主义,这就是彻底的民族主义;工农民主专政制度也不是别的,就是彻底的民权主义;土地革命则是彻底的民生主义。为什么共产党现在又申明取消工农民主专政和停止没收地主的土地呢?这个理由我们也早已说明了,不是这种制度和办法根本要不得,而是日本帝国主义的武装侵略引起了国内阶级关系的变化,使联合全民族各阶层反对日本帝国主义成了必需,而且有了可能。不但在中国,而且在世界范围内,为了共同反对法西斯,建立反法西斯的统一战线也有了必需和可能。所以,我们主张在中国建立民族的和民主的统一战线。我们用以代替工农民主专政的各阶层联合的民主共和国的主张,是在这种基础之上提出的。实行“耕者有其田”的土地革命,正是孙中山先生曾经提出过的政策;我们今天停止实行这个政策,是为了团结更多的人去反对日本帝国主义,而不是说中国不要解决土地问题。关于这种政策改变的客观原因和时间性,我们曾经毫不含糊地说明了自己的观点。正是因为中国共产党根据马克思主义的原则,一贯地坚持了并发展了第一次国共统一战线的共同纲领即革命的三民主义,所以共产党能于强寇压境民族危急之际,及时地提出民族民主的统一战线这种唯一能够挽救危亡的政策,并且不疲倦地实行之。现在的问题,不是共产党信仰不信仰实行不实行革命的三民主义的问题,反而是国民党信仰不信仰实行不实行革命的三民主义的问题。现在的任务,是在全国范围内恢复孙中山先生的三民主义的革命精神,据以定出一定的政纲和政策,并真正而不二心地、切实而不敷衍地、迅速而不推延地实行起来,这在中国共产党方面真是日夜馨香祷祝之的。为此,共产党在卢沟桥事变之后,提出了抗日救国的十大纲领。这个十大纲领,符合于马克思主义,也符合于真正革命的三民主义。这是现阶段中国革命即抗日民族革命战争中的初步的纲领,只有实行了它,才能挽救中国。一切和这个纲领相抵触的东西,如果还要继续下去,就会要受到历史的惩罚。

这个纲领之在全国范围内实行,不得到国民党同意是不可能的,因为国民党现在还是中国的最大的握有统治权的政党。我们相信,那些贤明的国民党人会有一天同意这个纲领的。因为如果不同意的话,三民主义就始终是一句空话,孙中山先生的革命精神就不能恢复,日本帝国主义就不能战胜,中国人民的亡国奴境遇就无可避免。真正贤明的国民党人是决不愿意这样的,全国人民也决不会眼看着尽当亡国奴。而况蒋介石先生在其九月二十三日的谈话中已经指出:“余以为吾人革命所争者,不在个人之意气与私见,而为三民主义之实现。在存亡危急之秋,更不应计较过去之一切,而当与全国国民彻底更始,力谋团结,以保国家之生命与生存。”这是很对的。现在的急务在谋三民主义的实现,放弃个人和小集团的私见,改变过去的老一套,立即实行符合于三民主义的革命纲领,彻底地与民更始。这是今天的唯一的道路。再要推延,就会悔之无及了。

然而要实行三民主义和十大纲领,需要实行的工具,这就提出了改造政府和改造军队的问题。现在的政府还是国民党一党专政的政府,不是民族民主的统一战线的政府。三民主义和十大纲领的实行,没有一个民族民主的统一战线的政府是不可能的。现在国民党军队的制度还是老制度,要用这种制度的军队去战胜日本帝国主义是不可能的。现在的军队都在执行抗战的任务,我们对于所有这样的军队,特别是在前线抗战的军队,都是具有钦敬之忱的。然而国民党军队的制度不适宜于执行彻底战胜日寇的任务,不适宜于顺利地执行三民主义和革命纲领,必须加以改变,这在三个月来的抗战教训中已经证明了。改变的原则就是实行官兵一致、军民一致。现在国民党军队的制度是基本上违反这两个原则的。广大的将士虽有忠勇之心,但束缚于旧制度,无法发挥其积极性,因此旧制度应该迅速地开始改造。不是说把仗停下来改造了制度再打,一面打仗一面就可以改变制度。中心任务是改变军队的政治精神和政治工作。模范的前例,就是在北伐战争时代的国民革命军,那是大体上官兵一致、军民一致的军队,恢复那时的精神是完全必要的。中国应学习西班牙战争的教训,西班牙共和国的军队是从极困难的境遇中创造出来的。中国的条件优于西班牙,但是缺乏一个充实的坚固的统一战线,缺乏一个能执行全部革命纲领的统一战线的政府,又缺乏大量的新制度的军队。中国应该补救这些缺点。中国共产党领导的红军,在今天,对于整个抗日战争,还只能起先锋队的作用,还不能在全国范围内起决定的作用,但是它的一些政治上、军事上、组织上的优点是足供全国友军采择的。这个军队也不是一开始就像现在的情形,它也曾经过许多的改造工作,主要地是肃清了军队内部的封建主义,实行了官兵一致和军民一致的原则。这个经验,可以供全国友军的借鉴。

当权的国民党的抗日同志们,我们和你们在今天一道负着救亡图存的责任。你们已经和我们建立起抗日统一战线了,这是很好的。你们实行了对日抗战,这也是很好的。但是我们不同意你们继续其它的老政策。我们应该把统一战线发展充实起来,把民众加进去。应该把它巩固起来,实行一个共同纲领。应该决心改变政治的制度和军队的制度。一个新政府的出现是完全必要的,有了这样一个政府,才能执行革命的纲领,也才能在全国范围内着手改造军队。我们的这个建议是时代的要求。这个要求,你们党中也有许多人感觉到,现在是实行的时候了。孙中山先生曾经下决心改造政治制度和军事制度,因而奠定了一九二四年到一九二七年的革命的基础。实行同样改造的责任,今天是落在你们的肩上了。一切忠诚爱国的国民党人当不以我们的建议为不切需要。我们坚决相信,这个建议是符合于客观的需要的。

我们民族已处在存亡绝续的关头,国共两党亲密地团结起来啊!全国一切不愿当亡国奴的同胞在国共两党团结的基础之上亲密地团结起来啊!实行一切必要的改革来战胜一切困难,这是今日中国革命的迫切任务。完成了这个任务,就一定能够打倒日本帝国主义。只要我们努力,我们的前途是光明的。


\begin{maonote}
\mnitem{1}参见本书第一卷\mxnote{中国共产党在抗日时期的任务}{3}。
\mnitem{2}参见本书第一卷\mxnote{中国共产党在抗日时期的任务}{4}。
\mnitem{3}见本书第一卷\mxnote{中国共产党在抗日时期的任务}{5}。
\mnitem{4}参见本书第一卷\mxnote{关于蒋介石声明的声明}{9}。
\mnitem{5}参见本书第一卷\mxnote{中国共产党在抗日时期的任务}{7}。
\mnitem{6}参见本书第一卷\mxnote{关于蒋介石声明的声明}{1}。
\mnitem{7}见本书第一卷\mxnote{中国共产党在抗日时期的任务}{9}。
\mnitem{8}见本书第一卷\mxnote{论反对日本帝国主义的策略}{5}。
\mnitem{9}《中苏互不侵犯条约》订立于一九三七年八月二十一日。
\mnitem{10}见本卷\mxart{为动员一切力量争取抗战胜利而斗争}。
\mnitem{11}诸青来,上海人。一九三四年参加中国国家社会党,曾任上海大夏大学、光华大学等校教授。一九三七年七月抗日战争爆发后,他在上海《新学识》杂志上发表文章,反对中国共产党关于建立抗日民族统一战线的政策,反对国共合作。后来,他公开投降日本帝国主义,成为汪精卫汉奸政府中的一员。
\end{maonote}
