
\title{评赫尔利政策的危险}
\date{一九四五年七月十二日}
\thanks{这是毛泽东为新华社写的评论。}
\maketitle


以美国驻华大使赫尔利\mnote{1}为代表的美国对华政策,越来越明显地造成了中国内战的危机。坚持反动政策的国民党政府,从它在十八年前成立之日起,就是以内战为生活的;仅在一九三六年西安事变\mnote{2}和一九三七年日本侵入中国本部这样的时机,才被迫暂时地放弃全国规模的内战。但从一九三九年起,局部的内战又在发动,并且没有停止过。国民党政府在其内部的动员口号是“反共第一”,抗日被放在次要的地位。目前国民党政府一切军事布置的重心,并不是放在反对日本侵略者方面,而是放在向着中国解放区“收复失地”和消灭中国共产党方面。不论是为着抗日战争的胜利,或是战后的和平建设,这种情况均须严重地估计到。罗斯福总统在世时,他是估计到了这一点的,为了美国的利益,他没有采取帮助国民党以武力进攻中国共产党的政策。一九四四年十一月,赫尔利以罗斯福私人代表的资格来到延安的时候,他曾经赞同中共方面提出的废止国民党一党专政、成立民主的联合政府的计划。但是他后来变卦了,赫尔利背叛了他在延安所说的话。这样一种变卦,露骨地表现于四月二日赫尔利在华盛顿所发表的声明。这时候,在同一个赫尔利的嘴里,以蒋介石为代表的国民党政府变成了美人,而中共则变成了魔怪;并且他率直地宣称:美国只同蒋介石合作,不同中共合作。当然这不只是赫尔利个人的意见,而是美国政府中的一群人的意见,但这是错误的而且危险的意见。就在这个时候,罗斯福去世了,赫尔利得意忘形地回到重庆的美国大使馆。这个以赫尔利为代表的美国对华政策的危险性,就在于它助长了国民党政府的反动,增大了中国内战的危机。假如赫尔利政策继续下去,美国政府便将陷在中国反动派的又臭又深的粪坑里拔不出脚来,把它自己放在已经觉醒和正在继续觉醒的几万万中国人民的敌对方面,在目前,妨碍抗日战争,在将来,妨碍世界和平。这一种必然的趋势,难道还看不清楚吗?在中国的前途这个问题上,看清楚了中国人民要求独立、自由、统一的不可阻止的势力必然要代替民族压迫和封建压迫而勃兴的美国一部分舆论界,对于赫尔利式的危险的对华政策,是感到焦急的,他们要求改变这个政策。但是,美国的政策究竟是否改变和哪一天才改变,今天我们还不能说什么。可以确定地说的,就是赞助中国反人民势力和以如此广大的中国人民为敌的这个赫尔利式的政策,如果继续不变的话,那就将给美国政府和美国人民以千钧重负和无穷祸害,这一点,必须使美国人民认识清楚。


\begin{maonote}
\mnitem{1}见本卷\mxnote{愚公移山}{3}。
\mnitem{2}参见本书第一卷\mxnote{关于蒋介石声明的声明}{1}。
\end{maonote}
