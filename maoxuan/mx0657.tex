
\title{中法之间有共同点}
\date{一九六四年一月三十日}
\thanks{这是毛泽东同志同法国议员代表团的谈话。}
\maketitle


欢迎你们。我们做个朋友,做个好朋友。你们不是共产党,我也不是你们的党;我们反对资本主义,你们也许反对共产主义。但是,还是可以合作。在我们之间有两个根本的共同点:第一,反对大国欺侮我们。就是说,不许世界上有哪个大国在我们头上拉屎拉尿。我讲得很粗。不管资本主义大国也好,社会主义大国也好,谁要控制我们,反对我们,我们是不允许的。你们国家的本钱比我们的大,你们的原子弹都已经制造出来了,可能已经成批地生产了吧?我不反对你们生产原子弹。大批大批的原子弹在美国、在苏联,它们经常拿在手上晃着吓唬人。第二,使两国间在商业上、在文化上互相往来。希望你们把什么禁运战略物资也反掉。现在卖给我们的只是些民用物资,战略物资还不卖,美国不让卖。我说,总有一天会突破这个缺口。如石油,因为是战略物资,现在还不许你们拿此做生意。粮食,生意我们已经做成了,因为它不是战略物资。英国卖给我们一些飞机,你们也可以做这生意。有些普通军火为什么不可以做点生意呢?

美国吓唬一些国家,不让它们跟我们做生意。美国是只纸老虎,你们不要相信它,一戳就穿了的。苏联也是纸老虎。我们不信它们那一套,我不迷信。也许你们是有神论者,我是无神论者,啥也不怕。大国来控制我们国家,那不行。法国是小国,中国是小国,只有美国和苏联才是大国,难道一切事都要照它们办,要到它们那里“朝圣”?从前我们也照办过,那是在斯大林的时候。一九五七年,我还去过一次莫斯科。那时,苏联还不是公开反对我们。现在不去了,因为它撕毁了大批合同,不讲信用,公开反对我们,同美国配合起来搞。这很好,我很赞同。美国、苏联这些大国来反对我们,总有个什么道理,我们也一定有一点东西值得它们反。现在,西哈努克\mnote{1}不吃美国这一套了。柬埔寨这个国家只有五百多万人,但敢于跟美国斗争。

你们可以在亚洲和我们合作,同美国顶一顶。美国到处不得人心。本月二十六日,日本一百多万人示威游行反美。我曾经同你们前总理富尔先生谈过,希望你们把欧洲的工作做好,例如,使英国、西德、比利时、意大利等等国家同美国隔开一些,同你们靠拢一些。你们不是说要建立“第三世界”吗?“第三世界”只有一个法国,那不行,太少了,要把整个欧洲团结起来。英国,我看总有一天要起变化。美国人对英国人也不那么客气。在东方,你们可以做日本的工作。如果把英国拉过来,从欧洲的伦敦、巴黎到中国、日本,就可以把“第三世界”扩大起来。

你们不要学英国在台湾问题上的态度。英国同我们只有一个分歧,就是它对台湾的地位不肯定。第一,英国承认中华人民共和国,不承认台湾,这是好的;第二,英国现在在联合国投我们的票,这也是好的;第三,英国同美国都搞“两个中国”,这点上表明它是美国的代理人。我们同英国已有十来年的外交关系,它也像你们那样,要我们派个大使去,它派个大使来。我们说不行,再搞十五年,甚至几十年也可以,我们不派大使去。联合国进不了,那也不要紧。十五年没有进联合国,我们也活下来了,再让蒋介石“大元帅”在联合国呆上十五年、三十年、一百年,我们照样活下去。要我们承认“两个中国”或者是“一个半中国”,那都不行。你们要派就派个大使来,不要学英国那样,搞了十几年,还是个代办,不要钻进美国的圈套。这一点不搞清楚,我们不接纳你们的大使,我们也不派大使到你们那里去,事先讲个清楚。我见富尔先生时,也同他讲清楚了这个问题。我们外交部发表过声明,也在瑞士和你们打过招呼,取得了协议。你们同国内有什么密码通讯吗?在外国跑,没有个密码通讯可不方便。

你们要同英国区别开来,要痛痛快快地把话讲个清楚。我是个军人,打过二十二年仗,戴高乐\mnote{2}将军也是个军人,讲话不要弯弯曲曲,不要搞外交手腕。

法国已经不是希特勒\mnote{3}的法国,我们中国也不是日本的中国了。过去,从北京到南京,大半个中国都被日本强占。日本人被赶走后,美国人又来了,我们把美国人、蒋介石都赶走了。那时,我们啥东西也没有,也没有飞机,也没有坦克,更没有原子弹。我们就是有些步枪、手榴弹、轻炮。感谢美国人给我们运来一批重炮,当运输大队长的是蒋介石。我们没有兵工厂,也没有任何外国援助。你们没有到我们那个小地方——延安去过吗?那里很落后,只有农业,一点点手工业。那时,我们说美国和蒋介石是纸老虎。我们也说,希特勒是纸老虎,他最后倒了嘛,死了嘛。现在我们说有两个大纸老虎,就是美国和苏联。我说得灵不灵将来瞧。请你们记住,我同法国议员代表团说过,它们是大纸老虎,但是不包括广大的苏联人民、广大的苏联党员和干部,他们对我们是友好的,美国人民有一部分人受了欺骗,总有一天他们要同我们友好的。所谓纸老虎,就是说美国、苏联脱离了群众。当年希特勒占领了几乎整个欧洲,多大的势力!这你们都经历过。

什么全面、彻底裁军,你们相信不相信?没有那回事,现在是全面彻底扩军。减少一些步兵是可能的,把省下来的钱用来制造原子弹。你们法国已经能爆炸原子弹了。我们比你们落后了一步,现在原子弹还没有爆炸,但是总有一天要爆炸的。

还有一条我们跟你们是共同的,什么三国条约\mnote{4},我们不参加。那是一种欺骗、讹诈,是压我们的,只许它们有,不许我们有。事先我们两国并没有交换过意见,你们不参加,我们也没有参加。

有些亚洲国家的人,反对你们到亚洲来支持西哈努克;对南越,只许美国占领它,不让你们来帮助。美国国务卿腊斯克在东京说,戴高乐将军想拿着橄榄枝打进亚洲,但是没有打进来。美国一手拿橄揽枝,一手拿剑,在越南南方打了几年,越打人民的斗争就越发展。它的剑在那里杀死了两个人,一个叫吴庭艳\mnote{5},一个叫吴庭儒,做法很恶劣。我看你们也不高兴吧?做这样的事干什么!现在扶植起来的所谓新政府照样不行,美国的政策太错了。我们中国四川省有一句俗话,叫做十个手指按十个跳蚤,一个也捉不到。

我们双方还可以对日本做工作。日本总有一天要把美国赶跑的。我说的不光是指日本共产党,还指日本的大资本家,现在日本有些大资本家对美国很不舒服。英国问题麻烦一点,哪一天它不当美国的代理人就好了。

我们不反对你们同美国好,对你们说来,也是又团结又斗争。我们同美国在台湾问题解决了以后,要恢复外交关系。即使恢复了外交关系,美国如果还像今天这样到处干涉、控制,我们还是要反对。我们要求美帝国主义从亚洲滚出去,从非洲滚出去,从拉丁美洲滚出去,从欧洲滚出去。欧洲是欧洲人的欧洲,美国人去干什么?英国有个上议院议员,就是蒙哥马\mnote{6}元帅,他就反对北大西洋条约\mnote{7}中美国人来称霸。他反对加拿大同美国关系太密切。我那次同他说你去找戴高乐将军。那是在一九六一年他第二次访华时,大概他没有去,他是保守党。我问他,持你这种意见的只有你一个人吗?他说,不,还有人。他坚决反对美国在欧洲称霸,他并不是共产党员。

\begin{maonote}
\mnitem{1}西哈努克,即诺罗敦·西哈努克,一九二二年生,当时是柬埔寨国家元首。
\mnitem{2}戴高乐,时任法国总统。在第二次世界大战期间领导法国的抵抗运动,反抗法西斯德国的武装占领。
\mnitem{3}希特勒(一八八九——一九四五),德国法西斯首领、纳粹党党魁。一九三三年在德国垄断资产阶级支持下出任总理,次年总统兴登堡死后,自称国家元首,实行法西斯统治,积极扩军备战。一九三九年九月派德军入侵波兰,挑起第二次世界大战;一九四一年六月大举进攻苏联。一九四五年四月在苏军解放柏林时自杀。
\mnitem{4}指一九六三年八月五日美、苏、英三个发起国在莫斯科签订的《禁止在大气层、外层空间和水下进行核武器试验条约》。
\mnitem{5}吴庭艳(一九〇一——一九六三),原“越南共和国”总统兼总理和国防部长。一九六三年十一月一日在美国策划的军事政变中,同其弟吴庭儒一起被击毙。
\mnitem{6}蒙哥马利。原英国陆军元帅,当时已退休。
\mnitem{7}指北大西洋公约组织。一九四九年四月,美国、英国、法国、荷兰、比利时、卢森堡、挪威、葡萄牙、意大利、丹麦、冰岛和加拿大在华盛顿签署《北大西洋公约》。同年八月二十四日公约生效,北大西洋公约军事集团建立。希腊和土耳其于一九五二年,联邦德国于一九五五年,西班牙于一九八二年,波兰、捷克和匈牙利于一九九九年,正式加入该组织。
\end{maonote}
