
\title{关于中华人民共和国宪法草案}
\date{一九五四年六月十四日}
\thanks{这是毛泽东同志在中央人民政府委员会第三十次会议上的讲话。}
\maketitle


这个宪法草案,看样子是得人心的。宪法草案的初稿,在北京五百多人的讨论中,在各省市各方面积极分子的讨论中,也就是在全国有代表性的八千多人的广泛讨论中,可以看出是比较好的,是得到大家同意和拥护的。今天很多人讲了话,也都是这样讲的。

为什么要组织这样广泛的讨论呢?有几个好处。首先,少数人议出来的东西是不是为广大人们所赞成呢?经过讨论,证实了宪法草案初稿的基本条文、基本原则,是大家赞成的。草案初稿中一切正确的东西,都保留下来了。少数领导人的意见,得到几千人的赞成,可见是有道理的,是合用的,是可以实行的。这样,我们就有信心了。其次,在讨论中搜集了五千九百多条意见(不包括疑问)。这些意见,可以分作三部分。其中有一部分是不正确的。还有一部分虽然不见得很不正确,但是不适当,以不采用为好。既然不采用为什么又搜集呢?搜集这些意见有什么好处呢?有好处,可以了解在这八千多人的思想中对宪法有这样一些看法,可以有个比较。第三部分就是采用的。这当然是很好的,很需要的。如果没有这些意见,宪法草案初稿虽然基本上正确,但还是不完全的,有缺点的,不周密的。现在草案也许还有缺点,还不完全,这要征求全国人民的意见了。但是在今天看来,这个草案是比较完全的,这是采纳了合理的意见的结果。

这个宪法草案所以得人心,是什么理由呢?我看理由之一,就是起草宪法采取了领导机关的意见和广大群众的意见相结合的方法。这个宪法草案,结合了少数领导者的意见和八千多人的意见,公布以后,还要由全国人民讨论,使中央的意见和全国人民的意见相结合。这就是领导和群众相结合,领导和广大积极分子相结合的方法。过去我们采用了这个方法,今后也要如此。一切重要的立法都要采用这个方法。这次我们采用了这个方法,就得到了比较好的、比较完全的宪法草案。

在座的各位和广大积极分子为什么拥护这个宪法草案呢?为什么觉得它是好的呢?主要有两条:一条是总结了经验,一条是结合了原则性和灵活性。

第一、这个宪法草案,总结了历史经验,特别是最近五年的革命和建设的经验。它总结了无产阶级领导的反对帝国主义、反对封建主义、反对官僚资本主义的人民革命的经验,总结了最近几年来社会改革、经济建设、文化建设和政府工作的经验。这个宪法草案也总结了从清朝末年以来关于宪法问题的经验,从清末的“十九信条”\mnote{1}起,到民国元年的《中华民国临时约法》\mnote{2},到北洋军阀政府的几个宪法和宪法草案\mnote{3},到蒋介石反动政府的《中华民国训政时期约法》,一直到蒋介石的伪宪法。这里面有积极的,也有消极的。比如民国元年的《中华民国临时约法》,在那个时期是一个比较好的东西;当然,是不完全的,有缺点的,是资产阶级性的,但它带有革命性、民主性。这个约法很简单,据说起草时也很仓卒,从起草到通过只有一个月。其余的几个宪法和宪法草案,整个说来都是反动的。我们这个宪法草案,主要是总结了我国的革命经验和建设经验,同时它也是本国经验和国际经验的结合。我们的宪法是属于社会主义宪法类型的。我们是以自己的经验为主,也参考了苏联和各人民民主国家宪法中好的东西。讲到宪法,资产阶级是先行的。英国也好,法国也好,美国也好,资产阶级都有过革命时期,宪法就是他们在那个时候开始搞起的。我们对资产阶级民主不能一笔抹杀,说他们的宪法在历史上没有地位。但是,现在资产阶级的宪法完全是不好的,是坏的,帝国主义国家的宪法尤其是欺骗和压迫多数人的。我们的宪法是新的社会主义类型,不同于资产阶级类型。我们的宪法,就是比他们革命时期的宪法也进步得多。我们优越于他们。

第二、我们的宪法草案,结合了原则性和灵活性。原则基本上是两个:民主原则和社会主义原则。我们的民主不是资产阶级的民主,而是人民民主,这就是无产阶级领导的、以工农联盟为基础的人民民主专政。人民民主的原则贯串在我们整个宪法中。另一个是社会主义原则。我国现在就有社会主义。宪法中规定,一定要完成社会主义改造,实现国家的社会主义工业化。这是原则性。要实行社会主义原则,是不是在全国范围内一天早晨一切都实行社会主义呢?这样形式上很革命,但是缺乏灵活性,就行不通,就会遭到反对,就会失败。因此,一时办不到的事,必须允许逐步去办。比如国家资本主义,是讲逐步实行。国家资本主义不是只有公私合营一种形式,而是有各种形式。一个是“逐步”,一个是“各种”。这就是逐步实行各种形式的国家资本主义,以达到社会主义全民所有制。社会主义全民所有制是原则,要达到这个原则就要结合灵活性。灵活性是国家资本主义,并且形式不是一种,而是“各种”,实现不是一天,而是“逐步”。这就灵活了。现在能实行的我们就写,不能实行的就不写。比如公民权利的物质保证,将来生产发展了,比现在一定扩大,但我们现在写的还是“逐步扩大”。这也是灵活性。又如统一战线,共同纲领中写了,现在宪法草案的序言中也写了。要有这么一个“各民主阶级、各民主党派、各人民团体的广泛的人民民主统一战线”,可以安定各阶层,安定民族资产阶级和各民主党派,安定农民和城市小资产阶级。还有少数民族问题,它有共同性,也有特殊性。共同的就适用共同的条文,特殊的就适用特殊的条文。少数民族在政治、经济、文化上都有自己的特点。少数民族经济特点是什么?比如第五条讲中华人民共和国的生产资料所有制现在有四种,实际上我们少数民族地区现在还有别种的所有制。现在是不是还有原始公社所有制呢?在有些少数民族中恐怕是有的。我国也还有奴隶主所有制,也还有封建主所有制。现在看来,奴隶制度、封建制度、资本主义制度都不好,其实他们在历史上都曾经比原始公社制度要进步。这些制度开始时是进步的,但到后来就不行了,所以就有别的制度来代替了。宪法草案第七十条规定,少数民族地区,“可以按照当地民族的政治、经济和文化的特点,制定自治条例和单行条例”。所有这些,都是原则性和灵活性的结合。

这个宪法草案所以得到大家拥护,大家所以说它好,就是因为有这两条:一条是正确地恰当地总结了经验,一条是正确地恰当地结合了原则性和灵活性。如果不是这样,我看大家就不会赞成,不会说它好。

这个宪法草案是完全可以实行的,是必须实行的。当然,今天它还只是草案,过几个月,由全国人民代表大会通过,就是正式的宪法了。今天我们就要准备实行,通过以后,全国人民每一个人都要实行,特别是国家机关工作人员要带头实行,首先在座的各位要实行。不实行就是违反宪法。

我们的宪法草案公布以后,将会得到全国人民的一致拥护,提高全国人民的积极性。一个团体要有一个章程,一个国家也要有一个章程,宪法就是一个总章程,是根本大法。用宪法这样一个根本大法的形式,把人民民主和社会主义原则固定下来,使全国人民有一条清楚的轨道,使全国人民感到有一条清楚的明确的和正确的道路可走,就可以提高全国人民的积极性。

这个宪法草案公布以后,在国际上会不会发生影响?在民主阵营中,在资本主义国家中,都会发生影响。在民主阵营中,看到我们有一条清楚的明确的和正确的道路,他们会高兴的。中国人高兴,他们也高兴。资本主义国家中被压迫被剥削的人民如果看到了,他们也会高兴的。当然也有人不高兴,帝国主义、蒋介石都不会高兴的。你说蒋介石会不会高兴?我看不需要征求他的意见就知道他是不高兴的。我们对蒋介石很熟悉,他决不会赞成的。艾森豪威尔威尔总统也不高兴,也要说它不好。他们会说我们这个宪法是一条清楚的明确的但是很坏的道路,是一条错路,什么社会主义、人民民主,是犯了错误。他们也不赞成灵活性。他们最喜欢我们在一天早晨搞出个社会主义,搞得天下大乱,他们就高兴了。

中国搞统一战线,他们也不赞成,他们希望我们搞“清一色”。我们的宪法有我们的民族特色,但也带有国际性,是民族现象,也是国际现象的一种。跟我们同样受帝国主义、封建主义压迫的国家很多,人口在世界上占多数,我们有了一个革命的宪法,人民民主的宪法,有了一条清楚的明确的和正确的道路,对这些国家的人民会有帮助的。

我们的总目标,是为建设一个伟大的社会主义国家而奋斗。我们是一个六亿人口的大国,要实现社会主义工业化,要实现农业的社会主义化、机械化,要建成一个伟大的社会主义国家,究竟需要多少时间?现在不讲死,大概是三个五年计划,即十五年左右,可以打下一个基础。到那时,是不是就很伟大了呢?不一定。我看,我们要建成一个伟大的社会主义国家,大概经过五十年即十个五年计划,就差不多了,就象个样子了,就同现在大不一样了。现在我们能造什么?能造桌子椅子,能造茶碗茶壶,能种粮食,还能磨成面粉,还能造纸,但是,一辆汽车、一架飞机、一辆坦克、一辆拖拉机都不能造。牛皮不要吹得太大,尾巴不要翘起来。当然,我不是讲能造一辆,尾巴就可以翘一点,能造十辆,尾巴就可以翘得高一点,随着辆数的增加,尾巴就翘得更高一些。那是不行的。就是到五十年后象个样子了,也要和现在一样谦虚。如果到那时候骄傲了,看人家不起了,那就不好。一百年也不要骄傲。永远也不要翘尾巴。

我们的这个宪法,是社会主义类型的宪法,但还不是完全社会主义的宪法,它是一个过渡时期的宪法。我们现在要团结全国人民,要团结一切可以团结和应当团结的力量,为建设一个伟大的社会主义国家而奋斗。这个宪法就是为这个目的而写的。

最后,解释一个问题。有人说,宪法草案中删掉个别条文是由于有些人特别谦虚。不能这样解释。这不是谦虚,而是因为那样写不适当,不合理,不科学。在我们这样的人民民主国家里,不应当写那样不适当的条文。不是本来应当写而因为谦虚才不写。科学没有什么谦虚不谦虚的问题。搞宪法是搞科学。我们除了科学以外,什么都不要相信,就是说,不要迷信。中国人也好,外国人也好,死人也好,活人也好,对的就是对的,不对的就是不对的,不然就叫做迷信。要破除迷信。不论古代的也好,现代的也好,正确的就信,不正确的就不信,不仅不信而且还要批评。这才是科学的态度。


\begin{maonote}
\mnitem{1}指清政府一九一一年十一月发布的《重大信条十九条》。
\mnitem{2}《中华民国临时约法》时孙中山在辛亥革命后担任中华民国临时大总统时颁布的。
\mnitem{3}指袁世凯政府一九一三年的天坛宪法草案。一九一四年的约法,曹锟政府一九二三年的宪法和段祺瑞执政府一九二五年的宪法草案。
\end{maonote}
