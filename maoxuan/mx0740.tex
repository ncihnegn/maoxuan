
\title{关于国家机关的改革和革命委员会的基本经验}
\date{一九六八年二月、三月、八月}
\maketitle


\date{一九六八年二月}
\section*{(一)\mnote{1}}

大办学习斑。用学习班的方法斗私批修,提高认识,解决问题,狠抓思想革命化,组织革命化。已经成立革命委员会的应该巩固和发展,革命委员会就是好。应该总结经验,应该解放大批革命干部,干部只要不是三反分子、走资派、投敌叛变分子、特务分子,而是在运动中犯了路线错误认真检查,认识错误就可以三结合。在三结和中应该注意成份,但不能唯成份论,不要把坏人也结合进来。三结合要体现老、中、少,光小娃娃不行。

一般大学的革命委员会,解放军不结合进去,特殊情况下要结合,须经过市革命委员会批准。要警惕坏人,防止破坏。已经成立了革命委员会的应该爱护她,尊重她,帮助她,保卫她,维护她的无产阶级权威,严防阶级敌人破坏。革命委员会的成立,不是“三支两军”\mnote{2}工作的完成而是进入一个新的阶段,巩固和发展革命委员会的无产阶级权威。

\date{一九六八年二月}
\section*{(二)\mnote{3}}

革命委员会的基本经验有三条:一条是有革命干部的代表,一条是有军队的代表,一条是有革命群众的代表,实现了革命的三结合。革命委员会要实行一元化的领导,打破重迭的行政机构,精兵简政,组织起一个革命化的联系群众的领导班子。

\date{一九六八年八月二十日}
\section*{(三)\mnote{4}}

在革命委员会的领导下,以工人为主体的、有解放军指战员参加的毛泽东思想宣传队有步骤地进入学校及其他一切还没有搞好斗、批、改的单位,促进革命的大联合,促进清理阶级队伍,促进教育革命,认真搞好斗、批、改,就是根据毛主席伟大战略部署而来的一个革命的创举。

\begin{maonote}
\mnitem{1}这是毛泽东同志在一九六八年二月对党内的指示。
\mnitem{2}“三支两军”,一九六七年一月二十三日,中共中央、国务院、中央军委、中央文革小组联合发布《关于中国人民解放军坚决支持革命左派群众的决定》,中国人民解放军介入地方“文化大革命”,进行“支左”。

一九六七年三月十九日,中央军委做出《关于集中力量执行支左、支农、支工、军管、军训任务的决定》,即“三支两军”。首次将“三支两军”作为一个整体,向全军各部队提出。“三支两军”是指:军队支左(支持当时被称为左派群众的人们)、支工(支援工业)、支农(支援农业)、军管(对一些地区、部门和单位实行军事管制)、军训(对学生进行军事训练)。此后,以一月二十三日发布“支左”决议为起点,解放军“三支两军”便逐步在全国各地展开。
\mnitem{3}《人民日报》一九六八年三月三十日发表的《人民日报》、《红旗》杂志、《解放军报》社论《革命委员会好》中,用黑体字引用了这两段话。
\mnitem{4}这是毛泽东对一九六八年八月二十日送审稿《人民日报》、《解放军报》社论稿《团结起来,共同对敌》的修改。
\end{maonote}
