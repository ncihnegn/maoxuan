
\title{对晋绥日报编辑人员的谈话}
\date{一九四八年四月二日}
\maketitle


我们的政策,不光要使领导者知道,干部知道,还要使广大的群众知道。有关政策的问题,一般地都应当在党的报纸上或者刊物上进行宣传。我们正在进行土地制度的改革。有关土地改革的各项政策,都应当在报上发表,在电台广播,使广大群众都能知道。群众知道了真理,有了共同的目的,就会齐心来做。这和打仗一样,要打好仗,不光要干部齐心,还要战士齐心。陕北的部队经过整训诉苦以后,战士们的觉悟提高了,明了了为什么打仗,怎样打法,个个磨拳擦掌,士气很高,一出马就打了胜仗。群众齐心了,一切事情就好办了。马克思列宁主义的基本原则,就是要使群众认识自己的利益,并且团结起来,为自己的利益而奋斗。报纸的作用和力量,就在它能使党的纲领路线,方针政策,工作任务和工作方法,最迅速最广泛地同群众见面。

在我们一些地方的领导机关中,有的人认为,党的政策只要领导人知道就行,不需要让群众知道。这是我们的有些工作不能做好的基本原因之一。我党二十几年来,天天做群众工作,近十几年来,天天讲群众路线。我们历来主张革命要依靠人民群众,大家动手,反对只依靠少数人发号施令。但是在有些同志的工作中间,群众路线仍然不能贯彻,他们还是只靠少数人冷冷清清地做工作。其原因之一,就是他们做一件事情,总不愿意向被领导的人讲清楚,不懂得发挥被领导者的积极性和创造力。他们主观上也要大家动手动脚去做,但是不让大家知道要做的是怎么一回事,应当怎样做法,这样,大家怎么能动起来,事情怎么能够办好?要解决这个问题,根本上当然要从思想上进行群众路线的教育,同时也要教给同志们许多具体办法。办法之一,就是要充分地利用报纸。办好报纸,把报纸办得引人入胜,在报纸上正确地宣传党的方针政策,通过报纸加强党和群众的联系,这是党的工作中的一项不可小看的、有重大原则意义的问题。

同志们是办报的。你们的工作,就是教育群众,让群众知道自己的利益,自己的任务,和党的方针政策。办报和办别的事一样,都要认真地办,才能办好,才能有生气。我们的报纸也要靠大家来办,靠全体人民群众来办,靠全党来办,而不能只靠少数人关起门来办。我们的报上天天讲群众路线,可是报社自己的工作却往往没有实行群众路线。例如,报上常有错字,就是因为没有把消灭错字认真地当做一件事情来办。如果采取群众路线的方法,报上有了错字,就把全报社的人员集合起来,不讲别的,专讲这件事,讲清楚错误的情况,发生错误的原因,消灭错误的办法,要大家认真注意。这样讲上三次五次,一定能使错误得到纠正。小事如此,大事也是如此。

善于把党的政策变为群众的行动,善于使我们的每一个运动,每一个斗争,不但领导干部懂得,而且广大的群众都能懂得,都能掌握,这是一项马克思列宁主义的领导艺术。我们的工作犯不犯错误,其界限也在这里。当着群众还不觉悟的时候,我们要进攻,那是冒险主义。群众不愿干的事,我们硬要领导他们去干,其结果必然失败。当着群众要求前进的时候,我们不前进,那是右倾机会主义。陈独秀机会主义\mnote{1}的错误,就是落后于群众的觉悟程度,不能领导群众前进,而且反对群众前进。这些问题有许多同志还不懂得。我们的报纸要好好地宣传这些观点,使大家都能明白。

报纸工作人员为了教育群众,首先要向群众学习。同志们都是知识分子。知识分子往往不懂事,对于实际事物往往没有经历,或者经历很少。你们对于一九三三年制订的《怎样分析农村阶级》的小册子,就看不大懂;这一点,农民比你们强,只要给他们一说就都懂得了。崞县\mnote{2}两个区的农民一百八十多人,开了五天会,解决了分配土地中的许多问题。假如你们的编辑部来讨论那些问题,恐怕两个星期也解决不了。原因很简单,那些问题你们不懂得。要使不懂得变成懂得,就要去做去看,这就是学习。报社的同志应当轮流出去参加一个时期的群众工作,参加一个时期的土地改革工作,这是很必要的。在没有出去参加群众工作的时候,也应当多听多看关于群众运动的材料,并且下工夫研究这些材料。我们练兵的口号是:“官教兵,兵教官,兵教兵。”战士们有很多打仗的实际经验。当官的要向战士学习,把别人的经验变成自己的,他的本领就大了。报社的同志也要经常向下边反映上来的材料学习,慢慢地使自己的实际知识丰富起来,使自己成为有经验的人。这样,你们的工作才能够做好,你们才能担负起教育群众的任务。

《晋绥日报》在去年六月的地委书记会议以后,有很大进步。内容丰富,尖锐泼辣,有朝气,反映了伟大的群众斗争,为群众讲了话。我很愿意看它。但是从今年一月开始纠正“左”的偏向以后的这一时期,你们的报纸却有点泄气的样子,不够明确,不够泼辣,材料也少了,使人不大想看。你们现在正在检查工作,总结经验,这样很好。总结了反右反“左”的经验,使头脑清醒起来,你们的工作就会有改进。

《晋绥日报》在去年六月以后进行的反对右倾的斗争,是完全正确的。在反右倾的斗争中,你们作得很认真,充分地反映了群众运动的实际情况。对于你们认为错误的观点和材料,你们采用编者按语的形式加以批注。你们的批注后来也有缺点,但是那种认真的精神是好的。你们的缺点主要是把弓弦拉得太紧了。拉得太紧,弓弦就会断。古人说:“文武之道,一张一弛。”\mnote{3}现在“弛”一下,同志们会清醒起来。过去的工作有成绩,但也有缺点,主要是“左”的偏向。现在作一次全面的总结,纠正了“左”的偏向,就会做出更大的成绩来。

在我们纠正偏差的时候,有的人把过去的工作看得毫无成绩,认为完全错了。这是不对的。这些人没有看到,党领导了那么多的农民得到土地,打倒了封建主义,整顿了党的组织,改进了干部的作风,现在又纠正了“左”的偏向,教育了干部和群众。这不是很大的成绩吗?对于我们的工作,对于群众的事业,应当采取分析的态度,不应当否定一切。过去发生“左”的偏向,是因为大家没有经验。没有经验,就难免要犯错误。从没有经验到有经验,要有一个过程。去年六月到现在的短短时期内,经过反右和反“左”的斗争,使大家都知道了反右、反“左”是怎么一回事。没有这样一个过程,大家是不会知道的。

经过检查工作、总结经验以后,我相信,你们的报纸会办得更好。应当保持你们报纸的过去的优点,要尖锐、泼辣、鲜明,要认真地办。我们必须坚持真理,而真理必须旗帜鲜明。我们共产党人从来认为隐瞒自己的观点是可耻的。我们党所办的报纸,我们党所进行的一切宣传工作,都应当是生动的,鲜明的,尖锐的,毫不吞吞吐吐。这是我们革命无产阶级应有的战斗风格。我们要教育人民认识真理,要动员人民起来为解放自己而斗争,就需要这种战斗的风格。用钝刀子割肉,是半天也割不出血来的。


\begin{maonote}
\mnitem{1}见本书第一卷\mxnote{中国革命战争的战略问题}{4}。
\mnitem{2}崞县,今山西省原平县。
\mnitem{3}参见《礼记·杂记下》。原文是:“张而不弛,文武弗能也。弛而不张,文武弗为也。一张一弛,文武之道也。”文武,指周文王、周武王。
\end{maonote}
