
\title{关于中国人民志愿军撤出朝鲜问题}
\date{一九五八年一月二十四日}
\thanks{这是毛泽东同志关于志愿军撤出朝鲜问题给金日成的电报。}
\maketitle


\mxname{金日成\mnote{1}首相同志:}

一九五七年十二月十六日和二十五日两次来信都已经收到了。

来信中关于中国人民志愿军撤出朝鲜的问题所提出的两个方案,我们已经仔细地研究过。我们觉得,由朝鲜民主主义人民共和国主动提出外国军队撤出朝鲜的要求,然后由中国政府响应朝鲜政府的要求,是比较适宜的。因此,我们认为采用十二月十六日来信中所提出的方案较好。对于这个方案,我们提出一些具体意见。这些意见,我们已经同苏联政府商量过,他们表示完全同意。现将这些意见函告如下,请你和朝鲜劳动党中央考虑是否妥当。

1.由朝鲜民主主义人民共和国最高人民会议写信给联合国,要求联合国军撤出朝鲜,的确是有一定的好处的,因为这样可以便于苏联作为联合国的一个成员国在联合国内提出主张,推动联合国采取行动。但是,这个方式也有缺点,那就是把整个联合国作为同我们敌对的一方,而实际上派出侵略军队组成联合国军的,只是少数联合国的成员国。因此,我们建议由朝鲜民主主义人民共和国政府发表一个公开声明。声明中根据朝、中方面在一九五四年日内瓦会议\mnote{2}上关于朝鲜问题的基本主张,提出以下的建议:

(1)联合国军和中国人民志愿军同时撤出朝鲜;

(2)由朝鲜南北双方在对等的基础上进行协商,以建立和发展南北朝鲜之间的经济和文化关系,并且筹备全朝鲜的自由选举;

(3)在外国军队完全撤出南北朝鲜以后的一定时期内,在中立国机构监督之下举行全朝鲜的自由选举。

2.在朝鲜民主主义人民共和国政府发表了公开声明以后,中国政府接着发表声明,支持朝鲜政府的主张,并且正式表示准备同朝鲜民主主义人民共和国政府协商分批定期撤走中国人民志愿军的问题,同时要求联合国军方面有关各国政府采取同样的步骤。

3.苏联政府接着也发表声明,支持朝、中政府的声明,强调联合国军方面各国政府应该像中国政府那样响应朝鲜民主主义人民共和国政府的要求,并且建议召开有关国家的会议讨论和平解决朝鲜问题。

4.今年二月中,周恩来同志代表中国政府访问朝鲜期间,朝中两国政府可以在联合公报中宣布,中国政府已经商得中国人民志愿军的同意,中国人民志愿军决定在一九五八年年底以前分批撤出朝鲜。在联合公报中,朝、中两国政府可以声明,中国人民志愿军在联合国军之前撤出朝鲜,是为了和缓紧张局势,便于朝鲜南北双方在对等的基础上协商朝鲜的和平统一。因此,联合国军应该采取同样的行动。同时,中国人民志愿军发表声明表示:中国人民和朝鲜人民是唇齿相依、患难与共的,中国人民志愿军撤出朝鲜决不是对朝、中人民休戚相关的利益置之不理。如果李承晚\mnote{3}和美国重新进行挑衅,越过停战线,那末,中国人民志愿军在朝鲜政府提出要求的情况下,将毫不犹豫地再一次同朝鲜人民军并肩击退侵略。

5.中国人民志愿军撤出朝鲜的时间表,我们初步拟定如下:

(1)一九五八年三月至四月,在朝、中两国政府发表联合公报以后,撤回三分之一,其余的三分之二均放在第二道防线,由朝鲜人民军全部接防第一线;

(2)一九五八年七月至九月,撤回第二个三分之一;

(3)一九五八年年底以前撤回最后的三分之一。

6.在朝中两国政府联合公报发表后,中立国监察委员会\mnote{4}的两瑞方面很可能再次提出撤销监察委员会的要求,届时,我们可以根据联合国军尚未撤走的理由,请他们留一最小限度的人数在板门店执行监察任务,但是也要准备他们会不顾而去。

以上各点意见,请你们研究后答复。

此致敬礼

毛泽东

一九五八年一月二十四日

\begin{maonote}
\mnitem{1}金日成,时任朝鲜劳动党中央委员会委员长、朝鲜民主主义人民共和国内阁首相。
\mnitem{2}日内瓦会议,指一九五四年四月二十六日至七月二十一日在瑞士日内瓦召开的讨论和平解决朝鲜问题和恢复印度支那和平问题的国际会议。中、苏、美、英、法五国参加所有两项议题的讨论。朝鲜北南双方及美、英、法以外的其他十二个侵略朝鲜北方的国家参加了朝鲜问题的讨论,越南民主共和国、老挝、柬埔寨和南越政权参加了印度支那问题的讨论。关于朝鲜问题没有达成任何协议;关于恢复印度支那和平问题,分别达成关于在印度支那三国停止敌对行动的协定和《日内瓦会议最后宣言》(总称日内瓦协议),实现了印度支那的停战。
\mnitem{3}李承晚(一八七五——一九六五),时任南朝鲜即韩国总统。
\mnitem{4}中立国监察委员会,是根据《朝鲜停战协定》于一九五四年八月一日成立的,由波兰、捷克斯洛伐克、瑞典和瑞士四国成员组成,执行《朝鲜停战协定》有关条款所规定的监督、观察、视察与调查的职能。
\end{maonote}
