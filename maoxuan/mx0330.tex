
\title{给福斯特同志的电报}
\date{一九四五年七月二十九日}
\maketitle


\mxname{福斯特同志和美国共产党中央委员会:}

欣悉美国共产主义政治协会特别会议决定抛弃白劳德的修正主义的即投降主义的路线\mnote{1},重新确立马克思主义的领导,并已恢复了美国共产党。我们对于美国工人阶级和马克思主义运动的这个伟大的胜利,谨致热烈的祝贺。白劳德的整个修正主义——投降主义路线(这条路线充分表现于白劳德所著《德黑兰》一书中),本质上是反映了美国反动资本集团在美国工人运动中的影响。这个反动资本集团现在也正在力图扩大其影响于中国,赞助中国国民党内反动集团的反民族反人民的错误政策,使中国人民面临着严重的内战危机,危害中美两大国人民的利益。美国工人阶级及其先锋队美国共产党反对白劳德修正主义——投降主义的胜利,对于中美两国人民目前所进行的反日战争和战后建设和平民主世界的伟大事业,无疑地将有重大的贡献。


\begin{maonote}
\mnitem{1}白劳德(一八九一——一九七三),曾任美国共产党总书记。在第二次世界大战期间,美国共产党内以白劳德为代表的右倾思想,曾经形成反马克思主义的修正主义——投降主义路线,并于一九四四年四月出版了作为他的纲领性的著作《德黑兰:我们在战争与和平中的道路》一书。白劳德“修正”了列宁主义关于帝国主义是垄断的、腐朽的和垂死的资本主义的基本理论,否认美国资本主义的帝国主义性质,认为它还“保持着青年的资本主义制度的一些特点”,认为美国无产阶级和大资产阶级之间有“共同利害”,主张保护托拉斯制度,经过“阶级调和”来避免美国资本主义所不可避免的危机。白劳德于一九四四年五月,主持解散了美国无产阶级的政党——美国共产党,而另行组织非党的美国共产主义政治协会。白劳德的这一错误路线一开始就遭到以福斯特为首的许多美国共产党员的反对。一九四五年六月,在福斯特领导下,美国共产主义政治协会通过了批判白劳德路线的决议。同年七月,又举行美国共产主义政治协会特别代表大会,决定重建美国共产党。白劳德后来仍然坚持其错误主张,公开拥护杜鲁门政府的帝国主义政策,并进行反党的派别活动,因此在一九四六年二月被开除出党。
\end{maonote}
