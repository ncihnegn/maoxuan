
\title{解决“五多”问题}
\date{一九五三年三月十九日}
\thanks{这是毛泽东同志为中共中央起草的党内指示。}
\maketitle


(一)我们党政组织在农村工作中存在一些严重地脱离农民群众、损害农民及其积极分子的利益的问题,即所谓“五多”问题。“五多”,就是任务多,会议集训多,公文报告表册多,组织多,积极分子兼职多。这些问题,很久就存在了,中央曾对其中有些问题有过指示,要求各级党委予以重视和解决,但是不但没有解决,反而越来越严重。其原因,是没有将整个问题系统地提出来,尤其重要的是没有在中央、大区、省(市)、专区和县这五级党政领导机关中展开反对分散主义和官僚主义的斗争。因为区、乡的“五多”,基本上不是从区、乡产生的,而是从上面产生的,是因为在县以上各级党政领导机关中存在着严重的分散主义和官僚主义所引起的,有些则是过去革命战争和土地改革时期的产物,未加改变,遗留至今的。因此,必须在一九五三年内,在执行中央一九五三年一月五日关于反对官僚主义、反对命令主义、反对违法乱纪的指示中,着重地克服领导机关中的官僚主义和分散主义,并将那些过去需要而现在已不需要的制度和办法加以改变,方能解决这个问题。今后各级领导机关在规定任务的问题上,在召集会议和调人集训的问题上,在发出公文表册和向下级要报告的问题上,在规定区、乡组织形式的问题上以及在使用乡村积极分子的问题上,都要由县以上党委和政府的主要负责同志,按照实际可行的情况,加以适当的规定,有些则要由中央作出统一的规定。过去由各级党、政、民组织的许多工作部门,各自独立地向下级分派任务,随便召集下级人员和农村积极分子开会或训练,滥发公文表册和向下级或农村随便要报告等项不良制度和不良办法,必须坚决废止,而代之以有领导的、统一的和适合情况的制度和办法。至于在农村中每个乡存在着几十种委员会以及积极分子兼职太多,均属妨碍生产,脱离群众,也应坚决地但是有步骤地加以改变。

(二)中央一级党、政、民组织有关各部门,中央分别责成中央组织部,中央人民政府政务院及其所属财经、文教、政法三个委员会的主管同志负责,对于过去引起“五多”问题的各事项迅速加以清理,并规定适当的制度和办法,向中央作报告。

(三)各大区和省市,由各中央局、分局、省市委及各该级行政机关主管同志负责,对于“五多”问题加以清理,规定自己的解决办法,并报告中央。为达此项目的,请各中央局、分局、省市委仿照西北局的办法,派出一个专为了解“五多”问题的检查组,检查所属的一二个区、乡(在城市是检查一二个区、街)的情况,以为解决问题的参考材料。

(四)专区级和县级的“五多”问题,由省委负责指导解决之。

(五)农业生产是农村中压倒一切的工作,农村中的其它工作都是围绕着农业生产而为它服务的。凡足以妨碍农民进行生产的所谓工作任务和工作方法,都必须避免。目前我国的农业,基本上还是使用旧式工具的分散的小农经济,这和苏联使用机器的集体化的农业,大不相同。因此,我国在目前过渡时期,在农业方面,除国营农场外,还不可能施行统一的有计划的生产,不能对农民施以过多的干涉,还只能用价格政策以及必要和可行的经济工作和政治工作去指导农业生产,并使之和工业相协调而纳入国家经济计划之中。超过这种限度的所谓农业“计划”、所谓农村中的“任务”是必然行不通的,而且必然要引起农民的反对,使我党脱离占全国人口百分之八十以上的农民群众,这是非常危险的。所谓区、乡工作中的“五多”问题,其中有很大的成分就是这种过多地干涉农民的表现(另一部分成分是因为革命战争和土地改革的需要而产生和遗留下来的),已经引起农民的不满,必须加以改变。
