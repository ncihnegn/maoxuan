
\title{游击区也能够进行生产}
\date{一九四五年一月三十一日}
\thanks{这是毛泽东为延安《解放日报》写的社论。}
\maketitle


我们在敌后解放区中那些比较巩固的根据地内,能够和必须发动军民的生产运动的问题,早已解决了,不成问题了。但是在游击区中,在敌后之敌后,是否也能够这样,在过去,在许多人的思想中,还是没有解决的,这是因为还缺少证明的缘故。

可是现在有了证据了。根据一月二十八日《解放日报》所载张平凯同志关于晋察冀游击队的生产运动的报道,晋察冀边区的许多游击区内,已于一九四四年进行了大规模的生产,并且收到了极好的成绩。张同志报道中所提到的区域和部队,有冀中的第六分区,第二分区的第四区队,第四分区的第八区队,徐定支队,保满支队,云彪支队,有山西的代县和崞县\mnote{1}的部队。那些区域的环境是很恶劣的:“敌伪据点碉堡林立,沟墙公路如网,敌人利用它的军事上的优势和便利的交通条件,时常对我袭击,包围,‘清剿’;游击队为了应付环境,往往一日数处地转移。”然而他们仍然能够于战争的间隙,进行了生产。其结果:“使得大家的给养有了改善,每人每日增加到五钱油和盐,一斤菜,每月斤半肉。而且几年没有用过的牙刷、牙粉和识字本,现在也都齐全了。”大家看,谁说游击区不能生产呢!

许多人说:人稠的地方没有土地。果真没有土地吗?请看晋察冀:“首先在农业为主的方针下,解决了土地问题。他们共有九种办法:第一,平毁封锁墙沟;第二,平毁可被敌人利用的汽车路,在其两旁种上庄稼;第三,利用小块荒地;第四,协助民兵,用武装掩护,月夜强种敌人堡垒底下的土地;第五,与缺乏劳动力的农民伙耕;第六,部队化装,用半公开的形式,耕种敌人据点碉堡旁边的土地;第七,利用河沿,筑堤修滩,起沙成地;第八,协助农民改旱地为水地;第九,利用自己活动的村庄,到处伴种。”

农业生产是可以的,手工业及其它生产大概不能吧?果真不能吗?请看晋察冀:“沟线外部队的生产,不限于农业,而且也和巩固区一样,开展了手工业和运输业。第四区队开设了一个毡帽坊,一个油坊,一个面坊,七个月中盈利五十万元本币。不仅解决了本身困难,而且游击区群众的需要也解决了。毛衣毛袜等,战士们已能全部自给。”

游击区战斗那样频繁,军队从事生产,恐怕要影响作战吧?果真如此吗?请看晋察冀:“实现了劳力和武力相结合的原则,把战斗任务和生产任务同样看重。”“以第二分区第四区队为例。当春耕开始时,就派有专门的部队去打击敌人,并进行强有力的政治攻势。正因为这样,军事动作也积极了,部队战斗力也提高了。这个小部队从二月至九月初,作了七十一次战斗,攻下了朱东社、上庄、野庄、凤家寨、崖头等据点,毙伤敌伪一百六十五名,俘伪军九十一名,缴了三挺轻机枪,一百零一枝长短枪。”“把军事动作和大生产运动的宣传配合起来,马上进行政治攻势:‘谁要破坏大生产运动就打击谁。’代、崞等县城内敌人问老百姓:‘为什么八路军近来这么厉害?’老百姓说:‘因为你们破坏边区的大生产运动。’伪军在下面纷纷议论:‘人家搞大生产运动,可不要出去。’”

游击区人民群众是否也可以发动生产运动呢?那些地方,也许是还没有减租,或减租不彻底的,农民是否也有兴趣去增加生产呢?这一点,晋察冀那边也肯定地答复了。“沟线外部队生产运动的开展,还给了当地群众以直接的帮助。一方面,用武力掩护了群众的生产;另一方面,又用劳力进行了普遍的帮助。有的部队,规定了农忙时期以百分之五十的力量,无代价地帮助群众生产。群众生产情绪因此大大提高,军民关系更为融洽,群众都有了饭吃。游击区群众对共产党、八路军的同情和拥护,从此更增高一步。”

游击区能够和必须进行军民的大规模的生产运动,一切问题都解决了。我们要求一切解放区党政军工作人员,特别是游击区工作人员,从思想上完全认识这一点,认识这个“能够”和“必须”,事情就可以普遍地办起来。晋察冀边区也正是从这里开始的:“在沟线外部队的生产运动中,由于干部的思想转变,重视生产,重视劳力和武力相结合,培养了群众中的英雄模范(初步总结中,有六十六个英雄模范),仅仅五个月中,我们沟线外的部队,不仅在生产任务上按时完成了计划,而且特别有了许多实事求是的新创造。”

一九四五年,整个解放区,必须全体一致地从事一个比过去规模更大的军民生产运动,到今年冬季,我们来比较各区的成绩。

战争不但是军事的和政治的竞赛,还是经济的竞赛。我们要战胜日本侵略者,除其它一切外,还必须努力于经济工作,必须于两三年内完全学会这一门;而在今年——一九四五年,必须收到较前更大的成绩。这是中共中央所殷殷盼望于整个解放区全体工作人员和全体人民的,我们希望这一计划能够完成。


\begin{maonote}
\mnitem{1}崞县,今山西省原平县。
\end{maonote}
