
\title{中国将修建青藏铁路}
\date{一九七三年十二月九日}
\thanks{这是毛泽东同志在会见尼泊尔国王比兰德拉时的谈话节选。}
\maketitle


中国将修建青藏铁路\mnote{1},不仅要修到拉萨,而且还要与尼泊尔接轨,通到加德满都\mnote{2}去。青藏铁路修不通,我睡不着觉。

\begin{maonote}
\mnitem{1}青藏铁路,一九五八年,在毛泽东指示下,青藏铁路一期工程(西宁—格尔木段)动工修建。从一九六一年三月开始,因连续三年困难时期,青藏铁路全线停建。一九七四年,为落实毛泽东在会见尼泊尔国王比兰德拉时的谈话精神,一九七四年三月和一九七五年三月,铁道兵第十师和第七师共六点二万人,奉中央军委命令先后开进青海,开始了青藏铁路西宁至格尔木段的建设。一九七九年铺轨到格尔木,一期工程完成,一九八四年通车。由于当时没有攻克高原冻土和高寒缺氧两大难题的把握,青藏铁路格尔木—拉萨段在一九七八年七月被叫停。直到一九九八年,高原冻土地区修筑铁路的技术问题获得了重大进展,青藏铁路再次提到日程上来,二〇〇〇年十一月十一日,江泽民批示修建青藏铁路十分必要,二〇〇一年六月二十九日,青藏铁路二期格拉段开工,二〇〇六年七月一日全线通车。目前青藏铁路三期工程拉萨—加德满都段还没有立项,与毛主席的战略设想还有距离。
\mnitem{2}加德满都,尼泊尔首都。
\end{maonote}
