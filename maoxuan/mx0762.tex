
\title{恢复联合国席位后的谈话}
\date{一九七一年十月二十六日、十一月八日}
\thanks{这是毛泽东同志在我国恢复联合国席位后对有关同志的谈话。}
\maketitle


\date{一九七一年十月二十六日}
\section*{(一)}

\mxsay{毛泽东:}小唐\mnote{1}呀,密斯南希·唐,你的国家失败了呀,看你怎么办哪?

\mxsay{周恩来:}主席本来指示……\mnote{2}

\mxsay{毛泽东:}那是老皇历喽,不做数喽。

\mxsay{周恩来:}我们刚才开过会,都认为这次联大解决得干脆、彻底,没有留下后遗症。只是我们毫无准备,特别是安理会比较麻烦,现在就参加,不符合主席“不打无准备之仗”的教导。我临时想了个主意,让熊向晖\mnote{3}带几个人先去联合国,作为先遣人员,就地了解情况,进行准备。

\mxsay{毛泽东:}那倒不必喽。联合国秘书长不是来了电报吗?我们就派代表团去。让乔老爷\mnote{4}当团长,熊向晖当代表,开完会就回来,还要接待尼克松\mnote{5}嘛。派谁参加安理会,你们再研究。

\mxsay{周恩来:}就让黄华\mnote{6}作副团长,留在联合国当常驻安理会的代表。

\mxsay{毛泽东:}黄华到加拿大当大使不到四个月,现在就调走,人家可能不高兴咧。

\mxsay{周恩来:}做做工作,加拿大政府会理解的。

\mxsay{毛泽东:}好,那就这么办。

\mxsay{毛泽东:}今年有两大胜利,一个是林彪\mnote{7},一个是联合国。这两大胜利,我都没有想到。林彪搞鬼,我有觉察,就是没有想到他跑外国,更没有想到他坐的那架“三叉戟”飞机,摔在外蒙古,“折戟沉沙”。对联合国,我的护士长\mnote{8}是专家。她对阿尔巴尼亚那些国家的提案有研究。这些日子她常常对我说:联合国能通过;我说:通不过;她说:能;我说:不能。你们看,还是她说对了。我对美国的那根指挥棒,还有那么多的迷信呢。

英国、法国、荷兰、比利时、加拿大、意大利,都当了“红卫兵”,造美国的反,在联合国投我们的票。葡萄牙也当了“红卫兵”。欧洲国家当中,只有马耳他投反对票,希腊、卢森堡和佛朗哥\mnote{9}的西班牙投弃权票。除了这四国,统统投赞成票。投赞成票的,亚洲国家十九个,非洲国家二十六个,拉丁美洲是美国的“后院”,只有古巴和智利同我们建交,这次居然有七个国家投我们的票。美国的“后院”起火,这可是一件大事。一百三十一个会员国,赞成票一共七十六,十七票弃权,反对票只有三十五。表决结果一宣布,唱歌呀,欢呼呀,还有人拍桌子。拍桌子是什么意思?(周恩来解释说:在会场拍桌子,表示极为高兴。)那么多国家欢迎我们,再不派代表团,那就没有道理了。不高兴的人也有,“蒋委员长”就是头一个。美国国务院说要发表声明,还没有看到,不过是一篇“吊丧文”。

毫无准备怎么办?我讲过,不打无准备之仗。我也讲过,在战争中学习战争。现在请总理挂帅,抓紧准备。最重要的是准备在联合国大会的第一篇发言。\mnote{10}

一九五〇年,我们还是“花果山时代”,你\mnote{11}跟伍修权\mnote{12}去了趟联合国。伍修权在安理会讲话,题目叫做《控诉美国武装侵略中国领土台湾》。控诉就是告状,告“玉皇大帝”的状。那个时候“玉皇大帝”神气十足,不把我们放在眼里。现在不同了,“玉皇大帝”也要光临花果山了。这次你们去,不是去告状,是去伸张正义,长世界人民的志气,灭超级大国的威风。给反对外来干涉、侵略、控制的国家呐喊声援。

第一篇发言就要讲出这个气概。

第一要算账,这么多年不让我们进联合国,中国人民和世界人民都有一股子气。主要是美国,其次是日本,要点他们的名,不点不行。对提案国要一一列举。

第二,要讲讲联合国成立以来世界形势的变化。就是这次同基辛格\mnote{13}谈公报\mnote{14}讲的,“国家要独立,民族要解放,人民要革命,已成为不可抗拒的历史潮流”。要讲点历史,一七七六年美国独立战争,一七八九年法国大革命,一九一七年俄国十月革命,都是伟大的,但是都没有一九四五年以来这样大的规模。要讲讲中国,自力更生,艰苦奋斗,推翻三座大山,取得国家独立、民族解放、新民主主义革命胜利。这不是吹牛,是事实。目的是给世界人民鼓劲。美国必须从台湾撤走它的武装力量,不论是谁,要把台湾从中国分割出去,都是痴心妄想。

第三,要讲讲我们对国际问题的基本态度。这次同基辛格谈公报的许多话可以用。我们反对帝国主义的战争政策和侵略政策,反对超级大国的霸权主义,支持一切被压迫人民和被压迫民族的正义斗争。各国人民的斗争都是互相支持的。要宣传五项原则,大小国家一律平等,中国属于第三世界,永远不做超级大国,反对大国欺侮小国,强国欺侮弱国,不许任何国家操纵联合国。

还要讲些什么,请总理考虑。总而言之,要旗帜鲜明,“高屋建瓴”,“势如破竹”。“势如破竹”是晋主司马炎的“三军总司令”杜预讲的,此人号称“左传癖”。他带兵占领武昌,准备进攻东吴的首都建业。一个“二杆子”参谋向他建议,现在长江涨水,等明年再打。杜预说:“今兵威大振,如破竹之势,数节之后,皆迎刃而解,无复有着手处也。”果然一举成功,“三分天下归一统”。做文章就要“势如破竹”,才能说服人。

曹操是大军事家。诸葛亮在《后出师表》里称赞他:“曹操智计,殊绝于人,其用兵也,仿佛孙吴”,同时也批评他打过败仗。怎么批评的?请“参座”讲讲。

\mxsay{叶剑英:}“困于南阳,险于乌巢,逼于黎阳,几败北山,殆死潼关。”

“几败北山”,说的是夏侯渊战死以后,曹操争夺汉中的事。《后出师表》三处提到夏侯渊,另外两处是“夏侯败亡”,“夏侯授首”。夏侯渊是曹操的一员大将,曹操封他为征西将军,担任汉中的“警备司令”。刘备攻打汉中,夏侯渊把主力部队部署在定军山,命令张郃守住东围。刘备“引蛇出洞”,先打张郃,夏侯渊分兵一半亲自援助张郃,被黄忠砍了头。有一出京剧就叫《定军山》,是谭鑫培、谭富英\mnote{15}的拿手戏。你们看看《魏书》的夏侯渊传。当初夏侯渊打了几次胜仗,曹操写信提醒他:“为将当有怯弱时,不可但恃勇也。将当以勇为本,行之以智计;但知任勇,一匹夫敌耳。”“当有怯弱时”,就是要想到自己的弱点和不足,有打败仗的可能。夏侯渊把曹操的告诫不当一回事,结果全军覆没。你们去联合国,困难很多,要“以勇为本”,更要注意“为将当有怯弱时”。代表团团长就是“将”,不要被胜利冲昏头脑。送你们两句话,一句是:“没有调查就没有发言权”;一句是:“虚心使人进步,骄傲使人落后。”

我们在联合国的方针是“团结大多数,孤立极少数”。二十三个提案国是我们的患难之交,要同他们讲团结。其他投票赞成我们的五十四个国家也要团结。对投弃权票的十七个国家要正确对待。在美国那样大的压力下,他们不支持美国,用弃权的办法对我们表示同情,应当感谢他们。投反对票的三十五个国家不是铁板一块,也要做工作。团结是有原则的团结,原则就是我们对国际问题的基本立场。我们当前的口号是:维护各国的独立和主权,维护国际和平,促进人类进步。用这个口号团结大多数。

\date{一九七一年十一月八日}
\section*{(二)\mnote{16}}

\mxsay{毛泽东:}“没有调查就没有发言权”,这是针对教条主义者讲的,至今我认为这句话还是对的。对这句话的理解不要偏。客观事物不断发展变化,人的认识总是赶不上这种变化,认识总是落后于实际。要求把一切都调查清楚再说话,再办事,那就永远不能说话,永远不能办事。了解了主要情况、本质情况,就可以作出判断,就应该下决心。我一向反对下车伊始,哇哩哇啦的人,那样的人成事不足,败事有余。他们自以为了不起,光想当先生,不愿当学生。有的人打过仗,有点功劳,或者自以为有点功劳,吃饭、拉屎、睡觉、做梦,都念念不忘他那点功劳。说他没有什么功劳,他就说,没有功劳,也有苦劳;没有苦劳,也有疲劳。这是低级趣味。这几年,部队有些人的思想被林彪搞乱了。济南军区提出“反骄破满”,提得好,我就让全军学习。我最近常讲,军队要谨慎,这是有的放矢。今年在联合国打了一个大胜仗,这个胜仗主要是我们的外国朋友帮我们打的,我们没有理由翘尾巴。现在是“盛名之下,其实难副”。所以我讲“为将当有怯弱时”。还是“三个臭皮匠,胜过一个诸葛亮”。遇事要商量,要多谋善断,不要像袁绍那样“多谋寡断”,更不能“不谋专断”。谨慎不是谨小慎微。看准了的,该说就说,该做就做。

在联合国要搞统一战线。这是国际统一战线,和国内统一战线有同、有不同。根本区别是,国内统一战线是不同阶级的统一战线,无产阶级必须掌握领导权;国际统一战线是不同国家的统一战线,没有谁领导谁的问题。大小国家一律平等,谁也不应该领导谁,谁也不应该听谁的领导。过去我们说以苏联为首,因为它是老大哥,为了对付帝国主义,必要的时候让它牵个头,开会的时候让它当主席。但是它要掌握领导权,搞父子党,父子国,这就完全错误了。美国总是要别的国家听它的,这就是搞霸权主义。霸权主义应该被打倒。所以,搞国际统一战线就要平等协商,绝对不能以大国自居,颐指气使,绝对不能干涉人家内政,绝对不能有领导人家的想法。

你们这次去联合国可以放心了,我的那个“亲密战友”不在了,在座的同志知道吗?

\mxsay{周恩来:}还没有告诉他们,主席谈完后,我们就到大会堂把文件\mnote{17}读给他们听,并介绍有关情况。“五七一”是“武装起义”的谐音。这是林彪反革命集团阴谋暗害主席、发动反革命政变的纲领。

\mxsay{毛泽东:}等一会把这件东西念给他们听。要尽快全文印发到全国各个党支部。

\mxsay{周恩来:}这里面尽是恶毒诽谤主席的谰言,怎么能印发?

\mxsay{毛泽东:}怎么不能?一个字都不改,原原本本发下去,让所有的党员所有的群众都知道。

\mxsay{毛泽东:}安全问题很重要,去了上上下下要住在一起。

\mxsay{毛泽东:}(对周恩来)马上打电报给黄镇\mnote{18}的助手,让他转告基辛格,我们的代表团在美国期间,美国政府必须保证安全。如果出了问题,唯美国政府是问。

\begin{maonote}
\mnitem{1}小唐,唐闻生,在美国纽约布鲁克林出生,唐闻生和王海容是毛泽东晚年的重要翻译。参加本次会见的还有周恩来、叶剑英、姬鹏飞、乔冠华、熊向晖、章文晋、王海荣。
\mnitem{2}一九七一年七月九日至十一日,秘密访华的美国国务卿基辛格告诉周恩来:尼克松总统已经决定,美国今年将支持中华人民共和国取得联合国和安全理事会(简称安理会)的席位,但不同意从联合国驱逐台湾的行动。

在向毛泽东汇报此事时,毛泽东说:我们绝不上“两个中国”的“贼船”,不进联合国,中国照样生存,照样发展。我们下定决心,不管是喜鹊叫还是乌鸦叫,今年不进联合国。
\mnitem{3}熊向晖,时任周恩来总理的助理。我党历史上著名的王牌间谍,原名熊汇荃,祖籍安徽凤阳,生于山东省掖县(今莱州市),清华大学中文系毕业,一九三七年,奉周恩来指示打入国民党胡宗南部,一九三九年三月至一九四七年五月,任胡宗南的侍从副官、机要秘书,成了胡宗南的亲信,负责处理机密文电和日常事务,起草讲话稿。毛泽东称赞熊向晖,说他一人可以顶几个师。建国后从事外交工作。
\mnitem{4}乔老爷,指乔冠华,时任外交部副部长。“乔老爷”一号其源于六十年代喜剧影片《乔老爷上轿》。
\mnitem{5}尼克松,时任美国总统。
\mnitem{6}黄华,时任中国驻加拿大大使。
\mnitem{7}林彪(一九〇七——一九七一),湖北黄冈人。一九二五年加入中国共产党。一九五八年五月在中共八届五中全会上被增选为中共中央副主席、政治局常务委员。一九五九年任中央军委副主席、国防部长,主持中央军委工作。在九届二中全会上主张设国家主席(毛泽东主席明确表示要改变国家体制不设国家主席),并组织人企图压服中央,犯了错误,被毛泽东主席识破,对其进行了警告和批评,并等待其认错达一年之久(从一九七〇年九月到一九七一年九月),不料,其子林立果狂妄自大,趁毛泽东南巡之时,妄图谋杀毛泽东主席,事情败露后,九月十三日夜,林立果挟制林彪和叶群驾机逃往苏联,最后坠毁于蒙古温都尔汗,史称“九一三”事件。后,林立果制定的《“五七一”工程纪要》被发现,因此,中央认定,林彪叛国。一九七三年八月中共中央决定,开除他的党籍。静火有言:林彪教子无方,最后身败名裂,家破人亡,说冤也不冤。
\mnitem{8}指吴旭君,毛泽东的护士长,自一九五三年至一九七四年在毛泽东身边工作二十一年,恪尽职守,兢兢业业,日夜坚守在护理毛泽东的第一线,兼做部分国际问题秘书的工作。
\mnitem{9}弗朗西斯科·佛朗哥(一八九二年十二月四日——一九七五年十一月二十日),西班牙政治家,军事家,法西斯主义独裁者,西班牙大元帅,西班牙长枪党党魁。一九三九年三月二十一日,佛朗哥军队占领马德里,推翻共和政府,对西班牙进行了长达三十六年的独裁统治。
\mnitem{10}指后来中华人民共和国代表团团长乔冠华十一月十五日在联合国大会第二十六届会议全体会议上的发言。
\mnitem{11}指乔冠华。
\mnitem{12}伍修权,一九五〇年十一月,联合国安理会审议中国提出的“美国武装侵略台湾案”,伍修权作为中国政府特派代表赴会,在联合国讲台上慷慨陈词,严厉驳斥美国及其同伙对我国的种种诬蔑和诽谤,痛斥了美国对我国领土台湾的入侵和战争威胁,维护了我国的主权和尊严。
\mnitem{13}基辛格,时任美国尼克松政府国家安全事务助理。一九七一年七月九日至十一日,基辛格作为尼克松总统秘密特使访华,七月十六日中美双方公布了基辛格访华的公告:“周恩来总理和尼克松总统的国家安全事务助理基辛格博士,于一九七一年七月九日至十一日在北京进行了密谈。获悉,尼克松总统曾表示希望访问中华人民共和国,周恩来总理代表中华人民共和国政府邀请尼克松总统于一九七二年五月以前的适当时间访问中国。尼克松总统愉快地接受了这一邀请。中美两国领导人的会晤,是为了谋求两国关系的正常化,并就双方关心的问题交换意见。”公告发表后,在世界上引起了震动。一九七一年十月二十日至二十六日,基辛格公开访华,为尼克松访华做准备工作,与中方共同草拟联合公报草案。
\mnitem{14}公报,指后来尼克松访华时一九七二年二月二十七日在上海发表的《中美联合公报》,简称《上海公报》。
\mnitem{15}谭鑫培、谭富英,谭家是著名京剧世家,谭鑫培是谭富英的祖父,京剧史上第一个老生流派谭派创始人。谭富英是二十世纪三、四十年代的四大须生之一。
\mnitem{16}这是一九七一年十一月八日晚八时,毛泽东主席约见周恩来总理、姬鹏飞、乔冠华、符浩、熊向晖、陈楚、唐明照、安致远、王海容、唐闻生、章文晋及回国述职的驻法大使黄镇、驻苏大使刘新权时的谈话。
\mnitem{17}这个文件指《“五七一工程”纪要》,一九七一年三月二十一日,林彪之子林立果连同周宇驰、于新野、李伟信在上海秘密所站密谋。他们分析了形势,认为在全国范围内,林彪的权力势力,目前是占绝对优势,但是可能逐渐削弱。“文人力量”(张春桥、姚文元等)正在发展,发展趋势是用张春桥代替林彪的可能性最大。他们研究了林彪的“接班”问题,认为有三种可能:一是林彪“和平接班”,二是林彪“被人抢班”,三是林彪“提前抢班”。他们提出两个办法:把张春桥一伙搞掉,保持“首长”地位不变,再和平过渡;或直接干掉毛泽东,但毛泽东影响大、威信高,以后政治上不好收拾,尽可能不这样干。他们商定:争取“和平过渡”,做好“武装起义”的准备。先做两件事:写个计划和让空四军组建一个“教导队”。林立果确定计划名称为“五七一工程”计划。(“五七一”为“武装起义”的谐音。)

一九七一年三月二十三日至二十四日,于新野执笔起草了《“五七一工程”纪要》。原稿后来被缴获。分九个部分:(一)可能性(二)必要性(三)基本条件(四)时机(五)力量(六)口号和纲领(七)实施要点(八)政策和策略(九)保密和纪律。

根据“五七一工程纪要”,林立果等人共策划了八种手法杀死毛泽东,“逼宫形式:利用特种手段如毒气、细菌武器、轰炸、543(静火注:国防科委统一规定的地空导弹兵器代号)、车祸、暗杀、绑架、城市游击小分队”。

该纪要还恶毒地辱骂毛泽东是“暴君”“独裁者”,并提出“打着B-52旗号打击B-52力量”(静火注:越战中臭名昭著的B-52轰炸机,借此暗示毛泽东亲美,并借指毛泽东),并规定“此工程属特级绝密,不经批准不得准向任何人透露。坚决做到一切行动听指挥,发扬“江田岛”(静火注:日本海军学校所在地,专门以武士道精神训练日本死士)精神。不成功便成仁,泄密者、失责者、动摇者、背叛者严厉制裁”。

该纪要后来被原原本本地印发到县团级以上干部手中。
\mnitem{18}黄镇,历任印度尼西亚大使,外交部副部长,驻法国首任大使。自一九七一年起,黄镇奉命秘密同美国展开外交沟通,负责与美国代表基辛格秘密访华的联络工作。
\end{maonote}
