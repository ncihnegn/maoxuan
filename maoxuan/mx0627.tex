
\title{庐山会议\mnote{1}讨论的十八个问题}
\date{一九五九年六月二十九日、七月二日}
\thanks{这是根据毛泽东一九五九年六月二十九日去庐山出席中共中央政治局扩大会议途中在船上同协作区主任委员的谈话和七月二日在庐山中央政治局常委扩大会上的讲话整理的。}
\maketitle


一、读书。有鉴于去年许多领导同志,县、社干部,对于社会主义经济问题还不大了解,不懂得经济发展规律,有鉴于现在工作中还有事务主义,所以应当好好读书。八月份用一个月的时间来读书,或者实行干部轮训。不规定范围,大家不会读。中央、省、市、地委一级委员,包括县委书记,要读苏联《政治经济学教科书》(第三版)\mnote{2}。时间三至六个月,或者一年。去年郑州会议提出读三本书\mnote{3},问读了没有,说是读了一点,读得不多,有的自己也没有读。对县、社干部,山东、河北的想法,是给他们编三本书:一本是好人好事的书,收集去年大跃进中敢于坚持真理、不随风倒,工作有前进的,不说谎、不浮夸、实事求是的例子,例如河北王国藩\mnote{4},山东菏泽一个生产队。一本是坏人坏事的书,收集专门说假话的、违法乱纪的或工作中犯了严重错误的例子。每省要找几个,各省编各省的,每本不超过二万五千字。第三本是中央从去年到现在的各种指示文件(加上各省的),有系统地编一本书。三本书大体十万字左右,每天读一万多字,一星期可以读完。读完后讨论,不仅读,还要考试。县、社党委成员能读政治经济学的也可以读。设法给县、社党委每年有一个系统思考问题的时间。我们提倡读书,使这些同志不要像热锅上的蚂蚁,整年整月陷入事务主义,搞得很忙乱,要使他们有时间想想问题。现在这些人都是热锅上的蚂蚁,要把他们拿出来冷一下。去年有了一年的实践,再读书会更好些。学习苏联,要读《政治经济学教科书》,教科书有缺点,但比较完整。缺点如第一章讲和平过渡,通过议会夺取政权,哪有那回事?在西藏都不可能,一定要有武装。他们的缺点我们不要去学,但在去年,把苏联一些好的经验也丢了。

二、形势。国内形势是好是坏?大形势还好,有点坏,但还不至于坏到“报老爷,大事不好”的程度。八大二次会议\mnote{5}的方针对不对?我看要坚持。总的说来,像湖南省一个同志所说的,是两句话:“有伟大的成绩,有丰富的经验。”“有丰富的经验”,说得很巧妙,实际上是:有伟大的成绩,有不少的问题,前途是光明的。基本问题是:(一)综合平衡;(二)群众路线;(三)统一领导;(四)注意质量。四个问题中最基本的是综合平衡和群众路线。要注意质量,宁肯少些,但要好些、全些,各种各样都要有。农业中,粮、棉、油、麻、丝、烟、茶、糖、菜、果、药、杂都要有。工业中,要有轻工业、重工业,其中又要各样都有。去年“两小无猜”(小高炉、小转炉)的搞法不行,把精力集中搞这“两小”,其他都丢了。去年大跃进、大丰收,今年是大春荒。现在形势在好转,我看了四个省,河北、河南、湖南、湖北,大体可以代表全国。今年夏收估产普遍偏低,这是一个好现象。

今年这时的形势和去年这时的形势比较,哪个形势好?去年这时很快地刮起了“共产风”,今年不会刮,比去年好。明年“五一”可以完全好转。去年人们的热情是宝贵的,只是工作中有些盲目性。有人说,你大跃进,为什么粮食紧张?为什么买不到头发夹子?现在讲不清楚,促进派也讲不清楚。说得清楚就说,说不清楚就硬着头皮顶住,去干。明年东西多了,就说清楚了。

去年许多事情是一条腿走路,不是两条腿走路。我们批评斯大林一条腿走路,可是在我们提出两条腿走路以后,反而搞一条腿了。在大跃进形势中,包含着某些错误,某些消极因素。现在虽然存在一些问题,但是包含着有益的积极因素。去年形势本来很好,但是带有一些盲目性,只想好的方面,没有想到困难。现在形势又好转了,盲目性少了,大家认识了。

三、今年任务。

四、明年任务。

五、四年任务。工、农、轻、重、商、交方面,过去是两条腿,后来丢掉了一条腿,重工业挤掉了农业和轻工业,挤掉了商业。如果当时重视一下农业、轻工业就好了。这几方面到底如何搞法?建设如何安排?

今年钢的产量是否定一千三百万吨?能超过就超过,不能超过就算了。今后应由中央确定方针,再交业务部门算账。粮食有多少?去年增产有无三成?今后是否每年增加三成?每年增加一千亿斤,搞到一万亿斤,要好几年。明年钢增加多少?增加四百万吨,是一千七百万吨。后年再增加四百万吨。十五年内主要工业产品的数量赶上和超过英国的口号还要坚持。总之,要量力而行,留有余地,让下面超过。人的脑子是逐渐变实际的,主观主义减少了。去年做了一件蠢事,就是要把好几年的指标在一年内达到,像粮食的指标一万零五百亿斤,恐怕要到一九六四年才能达到。

过去安排是重、轻、农,这个次序要反一下,现在是否提农、轻、重?要把农、轻、重的关系研究一下。过去搞过十大关系,就是两条腿走路,多快好省也是两条腿,现在可以说是没有执行,或者说是没有很好地执行。过去是重、轻、农、商、交,现在强调把农业搞好,次序改为农、轻、重、交、商。这样提还是优先发展生产资料,并不违反马克思主义。重工业我们是不会放松的,农业中也有生产资料。如果真正重视了优先发展生产资料,安排好了轻、农,也不一定要改为农、轻、重。重工业要为轻工业、农业服务。过去陈云\mnote{6}同志提过:先市场,后基建,先安排好市场,再安排基建。有同志不赞成。现在看来,陈云同志的意见是对的。要把衣、食、住、用、行五个字安排好,这是六亿五千万人民安定不安定的问题。安排好了之后,就不会造反了。怎么才会不造反?就是要使他们过得舒服,少说闲话,不骂我们。这样有利于建设,同时国家也可以多积累。赞成成立第三机械工业部,来管农业机械,搞农业机械设计院。现在这些事谁也不管,这么大个国家,没有人管不行。过去在土地革命战争时期反“左”倾时我曾说过,“炮是要打死人的,人是要吃饭的,路是要脚走的”。现在炮没有了,第二条、第三条还有,如果忘记了这些,不好办事。现在讲挂帅,第一应该是农业,第二是工业。

饲料要有单独的政策。现在是人吃一斤,牲口吃半斤;过一段,要人吃一斤,牲口吃一斤;再过一段,要人吃一斤,牲口吃两斤,牲口吃的要逐渐比人多。

农业问题:一曰机械,二曰化肥,三曰饲料。农、轻、重问题,把重放到第三位,放四年,不提口号,不作宣传。工业要支援农业,明年需要多少化肥、多少钢材支援农业,这次会议要定一下。粮食去年说增产三成,是否达到四千八百亿斤,我还有怀疑。今年说不增加了,我看增点还是可能的。以后每年增一千亿斤,一九六二年达到八千亿斤。

明年钢的指标是一千七百万吨,形成一个马鞍形。今年是一千三百万吨,比去年多四百多万吨。后年二千万吨,大后年二千一百万吨到二千三百万吨,可以赶上英国。一九六二年二千五百万吨,可能少点,也可能多点,多了到二千八百万吨,少了到二千三百万吨也好。赶上英国,说的是主要产品,钢太多了不一定好。

积极性有两种:一种是实事求是的积极性,一种是盲目的积极性。红军的三大纪律,现在有两条还有用:“一切行动听指挥”,即统一领导,反对无政府主义;“不拿群众一针一线”,即不搞一平二调\mnote{7}。总的说来,群众生活提高了,文化水平也提高了。共产主义风格有两种:一种是真要搞共产主义;另一种,这种占多数,是事情归他办,权力都归他,他就说是“共产主义”,归人家就是“资本主义”。山东曹县出现抢粮现象,这很好,抢得还少了,抢多了可以引起我们的注意。对那些摧残人民积极性的官僚主义就是要整一下。我们的国家是不会亡的,社会主义是亡不了的,蒋介石是回不来的。美国打来,我们最多退到延安,将来还会回来的。

六、宣传问题。去年有些虚夸,四大指标定高了,弄得今年不好宣传,现在有些被动。如何转为主动?上海会议\mnote{8}时,有人提出,利用开人民代表大会的机会,把指标改了,后来没有这么做。现在看来失掉了点时机,但不要紧。指标改不改?看来改一下好。但改成多少,还拿不准。是否人大常委会开个会,把指标改过来。粮食是否以后不公布绝对数字,可以学习苏联,不宣传粮食指标。今后钢不算小转炉的,铁不算土铁。

七、综合平衡问题。大跃进的重要教训之一、主要缺点是没有搞平衡。说了两条腿走路、并举,实际上还是没有兼顾。在整个经济中,平衡是个根本问题,有了综合平衡,才能有群众路线。

有三种平衡:农业内部农、林、牧、副、渔的平衡;工业内部各个部门、各个环节的平衡;工业和农业的平衡。整个国民经济的比例关系是在这些基础上的综合平衡。

八、群众路线问题。群众路线有没有?有多少?

九、工业管理问题。特别要强调质量问题,能否在很短时间内解决?应该争取在一二年内解决。

十、体制问题。“四权”\mnote{9}下放多了一些,快了一些,造成混乱,有些半无政府主义。要强调一下统一领导、集权问题。下放的权力,要适当收回,收回来归中央、省市两级。对下放要适当控制。反对无政府主义,不是说现在是完全无政府主义,而是说有些半无政府主义。说得过死不好,过活也不好。现在看来,不可过活。

十一、协作关系。划区协作,倒把原来的协作关系打乱了,搞了大的,挤了小的。搞体系,工厂要综合发展,公社要工业化。

十二、公共食堂。要积极办好。按人定量,分粮到户,自愿参加,节余归己。吃饭基本上要钱。在这几项原则下,把食堂办好,不要一轰而散,都搞垮了,保持百分之二十也好。

办食堂全国有两种办法:一为河南的积极维持,一为湖北的提倡自愿。湖北的基本解散了,有的未散,暂时回去了。湖北拟从少到多,开始百分之三十至百分之五十,将来达到百分之八十。食堂要小,形式要多种,供给部分要少些,三七开或四六开,可以灵活些。食堂和供给制是两回事。

十三、学会过日子。包括农村、城市,要留有余地,富日子当穷日子过,增产节约。湖北是穷日子当富日子过了,农民批评有些干部,一不会生产,二不会过日子。应当把富日子当穷日子过。有些地方生产不见得比别处多多少,但只要安排得好,日子好过。今年不管增产多少,估计增产一点,还是按去年四千八百亿斤或者再少一些的标准安排过日子。口号是:富日子当穷日子过。

十四、三定政策。定产、定购、定销,群众要求恢复,看来是非恢复不可。政策三年不变,定多少,这次会议要定一下。增产部分四六开,征四留六,有灾照减。自留地不征税。

十五、恢复农村初级市场。

十六、使生产小队成为半核算单位。四川省的同志说,生产、分配在一个核算单位较好,现在改,影响生产,如何办?

十七、农村党的基层组织领导作用问题。基层党的活动削弱了,党不管党,只管行政。

十八、团结问题。要统一思想,对去年的估计是:有伟大成绩,有不少问题,前途是光明的。缺点只是一、二、三个指头的问题。许多问题是要经过较长的时间才看得出来的。过去一段时间的积极性中带有一定的盲目性。这样看问题,就能鼓起积极性来。

\begin{maonote}
\mnitem{1}一九五九年七月二日至八月一日在庐山召开的中共中央政治局扩大会议
\mnitem{2}指苏联科学院经济研究所编写的《政治经济学教科书》修订第三版下册(人民出版社一九五九年版)。
\mnitem{3}一九五八年十一月郑州会议期间,毛泽东写信给中央、省市自治区、地、县四级党委委员,建议读斯大林《苏联社会主义经济问题》、《马克思恩格斯列宁斯大林论共产主义社会》,同时提出也可以读苏联《政治经济学教科书》。
\mnitem{4}王国藩,一九一九年生,河北遵化人。当时是河北省遵化县建明人民公社管理委员会主任。
\mnitem{5}八大二次会议,即中国共产党第八次全国代表大会第二次会议,一九五八年五月五日至二十三日在北京召开。
\mnitem{6}陈云(一九〇五——一九九五),江苏青浦(今属上海市)人。时任中共中央副主席、国务院副总理。
\mnitem{7}一平二调是人民公社化运动中“共产风”的主要表现,即:在公社范围内实行贫富拉平平均分配;县、社两级无偿调走生产队(包括社员个人)的某些财物。三收款,指银行将过去发放给农村的贷款统统收回。
\mnitem{8}指一九五九年三月二十五日至四月一日在上海召开的中共中央政治局扩大会议。
\mnitem{9}四权,指人权、财权、商权、工权。
\end{maonote}
