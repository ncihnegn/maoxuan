
\title{对《文汇报》、《解放日报》夺权事件的谈话}
\date{一九六七年一月八日}
\thanks{这是毛泽东同志对中央文革小组就这两个报社夺权\mnote{1}问题的谈话。}
\maketitle


《文汇报》现在左派夺了权,四号造了反,《解放日报》六号也造了反,这个方向是好的。《文汇报》夺权后,三期报都看了,选登了红卫兵的文章,有些好文章可以选登。《文汇报》五日《告全市人民书》\mnote{2},《人民日报》可转载,电台可广播。内部造反很好!过几天可以综合报导,这是一个阶级推翻一个阶级,这是一场大革命。许多报纸,依我说封了好,但报还是要出的,问题是由什么人出。《文汇报》、《解放日报》造反好。这两张报一出来,一定会影响华东、全国各省、市。这件大事对于整个华东,对于全国各省市的无产阶级文化大革命运动的发展,必将起着巨大的推动作用。

搞一场革命,定要先造舆论。“六一”《人民日报》夺了权,中央派了工作组,发表了《横扫一切牛鬼蛇种》的社论。我不同意另起炉灶,但要夺权,唐平铸换了吴冷西\mnote{3},开始群众不相信,因为《人民日报》过去骗人\mnote{4},又未发表声明。两个报纸夺权是全国性的问题,要支持他们造反。

我们报纸要转载红卫兵文章,他们写得很好,我们的文章死得很。中宣部可以不要,让那些人住那里吃饭,许多事宣传部、文化部都管不了,你\mnote{5}我管不了,红卫兵一来就管住了。

上海革命力量起来,全国就有希望,它不能不影响华东,以及全国各省市,《告全市人民书》是少有的好文章,讲的是上海市,问题是全国性的。

现在搞革命有些人要这要那,我们搞革命,白一九二〇年起,先搞青年团,后搞共产党,哪有经费、印刷厂、自行车?我们搞报纸同工人很熟,一边聊天一边改稿子。我们要与各种人,左、中、右都发生联系。一个单位统统搞得那样干净,我向来不赞成。(有人反映吴冷西很舒服,胖了。)太让吴冷西他们舒服了,不主张让他们都罢官,留在岗位上让群众监督。

我们开始搞革命,接触的是机会主义,不是马列主义。青年时《共产党宣言》也未看过。要抓革命,促生产,不能脱离生产搞革命。

保守派不搞生产,这是一场阶级斗争。你们不要相信“死了张屠夫,就吃混毛猪”。以为是没有他们不行,不要相信那一套!

\begin{maonote}
\mnitem{1}一九六七年一月四日,文汇报社“星火燎原革命造反总部”在该报发表《告读者书》,宣布接管《文汇报》;一月六日,解放日报社“革命造反联合司令部”在该报发表《告读者书》,宣布接管《解放日报》。
\mnitem{2}指一九六七年一月五日,“上海工人革命造反总司令部”等十一个造反组织在《文汇报》发表《抓革命,促生产,彻底粉碎资产阶级反动路线的新反扑——告上海全市人民书》。一月九日,《人民日报》全文转载,告上海全市人民书全文如下:

伟大的无产阶级文化大革命在以毛主席为代表的无产阶级革命路线的指引下,几个月来所开展的批判资产阶级反动路线的群众运动,已经取得了极大的胜利。我们在胜利的战歌声中,跨进了一九六七年。《人民日报》《红旗》杂志元旦社论指出:“一九六七年,将是全国全面展开阶级斗争的一年。一九六七年,将是无产阶级联合其他革命群众,向党内一小撮走资本主义道路的当权派和社会上的牛鬼蛇神,展开总攻击的一年。一九六七年,将是更加深入地批判资产阶级反动路线,清除它的影响的一年。一九六七年,将是一斗、二批、三改取得决定性的胜利的一年。”也就是说,一九六七年将是资产阶级反动路线全线崩溃彻底瓦解的一年,将是无产阶级文化大革命取得决定性的伟大胜利的一年。

上海市广大革命群众,在批判上海地区党内一小撮人所执行的资产阶级反动路线的斗争中,也取得了初步的胜利,并进入了一个更深入、更广阔的新阶段。

我们上海市工厂的无产阶级文化大革命群众运动,正以排山倒海之势,雷霆万钧之力,冲破一切阻力,汹涌澎湃地向前发展。我们革命造反派的工人,最听毛主席的话,坚决执行毛主席亲自提出的“抓革命,促生产”的方针。毛主席教导我们:“政治工作是一切经济工作的生命线。”我们革命造反派深深地懂得:不搞好无产阶级文化大革命,我们的生产就会迷失方向,就会滑到资本主义的方向上去。我们在无产阶级文化大革命中所亲身经历的事实越来越多地证明:只有无产阶级文化大革命搞好了,生产才会有更大的发展。把文化大革命同发展生产对立起来的看法,是错误的。

可是,党内一小撮走资本主义道路的当权派和顽固地坚持资产阶级反动路线的人十分仇视无产阶级文化大革命,他们千方百计地对抗毛主席提出的“抓革命,促生产”的方针。他们的阴谋手段大致表现如下:

运动初期,他们以“抓生产”为名,来压制革命,反对抓革命。我们革命造反派的工人要起来革命,要批判资产阶级反动路线,他们就以生产任务压工人,给我们扣上“破坏生产”的大帽子。他们是真的要“抓生产”吗?不是的,他们是为了保他们自己的乌纱帽,企图阻挠我们革命。我们戳穿了他们的阴谋诡计,勇敢地起来造反了。

于是,他们又抛出了另一种花招,以极“左”的面目,以漂亮的革命词句,煽动大批被他们蒙蔽的工人赤卫队队员借口北上“告状”为名,破坏生产,破坏交通运输,以达到他们破坏无产阶级文化大革命,破坏无产阶级专政的目的。最近,更有一小撮反动的家伙在阴谋策划停水、停电、停交通。对这样一些反动的家伙,我们一定要把他们揪出来,实行无产阶级专政,严加惩办,决不能让他们的罪恶阴谋得逞。

革命的工人同志们!紧急行动起来!坚决执行毛主席提出的“抓革命,促生产”的方针!我们革命造反派的工人要成为“抓革命,促生产”的模范。我们不但要做抓革命的先锋和骨干,而且也要做促生产的先锋和骨干。我们上海是一个全国最大的工业生产城市,它在国家的整个经济生活中起着极其重大的作用。但是,最近在很多工厂中出现部分或者大部分的赤卫队员停止生产、离开生产岗位的现象,这就直接违反了中央关于抓革命、促生产的规定,直接影响了人民的生活和国民经济建设的发展。我们的革命造反派工人牢记着毛主席的教导,顶着这股逆流,发挥了高度的革命负责精神,在极其困难的条件下,顶起了全厂的生产,有力地打击了党内一小撮走资本主义道路的当权派,粉碎了他们企图用破坏生产来打击革命的大阴谋。这样做得对!做得好!我们全体革命造反派的同志,都要向他们学习。毛主席教导我们:“凡是敌人反对的,我们就要拥护;凡是敌人拥护的,我们就要反对。”我们革命造反派工人有志气,有决心,有力量,一定能把革命和生产搞得更好,实现毛主席提出的“抓革命,促生产”的伟大号召。

工人赤卫队的广大的要革命的阶级兄弟们!“抓革命,促生产”是毛主席提出的方针,是党中央一再强调的方针,是保证无产阶级文化大革命进行到底的重要方针。拥护不拥护、执行不执行这个方针是一个原则问题,大是大非问题。你们受他们煽动而离开生产岗位,究竟是对谁有利呢?你们这样做到底是使谁高兴、使谁心痛呢?我们希望你们要听毛主席的话,在这个重大原则问题上,一定要擦亮眼睛,明辨是非,不要再受骗了,赶快觉悟过来,回到生产岗位上来,回到无产阶级革命路线上来。我们革命造反派的同志们一定会热情地欢迎你们回来,和我们共同革命,共同搞好生产,我们一定不会责怪你们,因为我们都是阶级亲兄弟,因为你们中绝大部分是受资产阶级反动路线毒害的人,是受党内走资本主义道路当权派和顽固地执行资产阶级反动路线的人蒙蔽的革命群众。

全市一切革命学生和革命的机关干部们!让我们和广大革命工人紧紧结合在一起,为了坚决贯彻执行毛主席提出的“抓革命,促生产”的方针,广泛地开展宣传和斗争,更坚决地向资产阶级反动路线猛烈开火,打垮资产阶级反动路线的一切新反扑,将工厂的无产阶级文化大革命推向一个新高潮!

在伟大的毛泽东思想无限光芒照耀下,展望未来,革命前程灿烂辉煌。我们工人阶级、贫下中农、一切劳动者同革命学生、革命知识分子、革命干部联合起来,共同努力,并肩战斗,乘胜前进,把无产阶级文化大革命进行到底!

伟大的无产阶级文化大革命万岁!

我们心中的红太阳、最最伟大的领袖毛主席万岁!万岁!万万岁!

上海工人革命造反总司令部

红卫兵上海市大专院校革命委员会(红革会)

上海市反到底联络总部

上海新闻界革命造反委员会

上海市炮打司令部联合兵团

同济大学东方红兵团

上海交通大学反到底兵团

首都第三司令部驻沪联络站

北京航空学院红旗战斗队驻沪联络站

哈尔滨军事工程学院红色造反团驻沪联络站

西安军事电讯工程学院文革临委会驻沪联络站

一九六七年一月四日
\mnitem{3}吴冷西,一九五七年六月至一九六六年五月任《人民日报》总编辑,《人民日报》被夺权后,由唐平铸接任。
\mnitem{4}《人民日报》在大跃进期间发表了一系列浮夸和错误的文章,如《人有多大胆,地有多大产》,吴冷西在《忆毛主席》中说:“到了(一九五八)五月间的八大二次会议,解放思想、敢想敢做的呼声压倒一切。我主持人民日报和新华社的宣传也随大流,但因有毛主席的再三叮咛,开始还是比较谨慎,但到了六月份,农业上的生产“卫星”开始放了,接着是钢铁“卫星”、煤炭“卫星”也陆续出现了,大跃进形成高潮,浮夸风到处泛滥。对人民公社,开始还只限于典型报道,后来从河南全省公社化起,就刮起一股共产风。虽然不能说人民日报和新华社应对一九五八年的浮夸风和共产风负有主要责任,但我主持这两个单位的宣传工作在这期间所造成的恶劣影响,至今仍深感内疚。”
\mnitem{5}指陈伯达,时任中央文革组长。
\end{maonote}
