
\title{目前形势和我们的任务}
\date{一九四七年十二月二十五日}
\thanks{这是毛泽东在中共中央一九四七年十二月二十五日至二十八日在陕北米脂县杨家沟召集的会议上的报告。这次会议除有当时能够到会的中央委员和候补中央委员以外,还有陕甘宁边区和晋绥边区负责同志参加。会议讨论了和通过了毛泽东的这个报告和他所写的\mxart{关于目前国际形势的几点估计}(见本卷第一一八四页)。关于毛泽东的报告,会议的决定指出:“这个报告是整个打倒蒋介石反动统治集团,建立新民主主义中国的时期内,在政治、军事、经济各方面带纲领性的文件。全党全军应将这个文件联系一九四七年双十节各项文件(按指一九四七年十月十日公布的\mxart{中国人民解放军宣言}、《中国人民解放军口号》、\mxart{中国人民解放军总部关于重行颁布三大纪律八项注意的训令}、《中国土地法大纲》和《中共中央关于公布中国土地法大纲的决议》),进行深入教育,并在实践中严格地遵照实施。各地实施政策中如果有和报告所指出的原则不相符合的地方,应即加以改正。”这次会议的其它重要决定有:(1)中国人民革命战争应该力争不间断地发展到完全胜利,应该不让敌人用缓兵之计(和谈)获得休整时间,然后再来打人民。(2)组织革命的中央政府的时机目前尚未成熟,须待我军取得更大胜利,然后考虑此项问题,颁布宪法更是将来的问题。会议还详细讨论了当时党内的倾向问题以及土地改革和群众运动中的几个具体政策问题。讨论的结果后来由毛泽东写在\mxart{关于目前党的政策中的几个重要问题}(见本卷第一二六七页)一文中。从本篇起,直到一九四八年三月二十日\mxart{关于情况的通报}止,都是在陕北米脂县杨家沟写的。}
\maketitle


\section*{一}

中国人民的革命战争,现在已经达到了一个转折点。这即是中国人民解放军已经打退了美国走狗蒋介石的数百万反动军队的进攻,并使自己转入了进攻。还在一九四六年七月至一九四七年六月此次战争的第一个年头内,人民解放军即已在几个战场上打退了蒋介石的进攻,迫使蒋介石转入防御地位。而从战争第二年的第一季,即一九四七年七月至九月间,人民解放军即已转入了全国规模的进攻,破坏了蒋介石将战争继续引向解放区、企图彻底破坏解放区的反革命计划。现在,战争主要地已经不是在解放区内进行,而是在国民党统治区内进行了,人民解放军的主力已经打到国民党统治区域里去了\mnote{1}。中国人民解放军已经在中国这一块土地上扭转了美国帝国主义及其走狗蒋介石匪帮的反革命车轮,使之走向覆灭的道路,推进了自己的革命车轮,使之走向胜利的道路。这是一个历史的转折点。这是蒋介石的二十年反革命统治由发展到消灭的转折点。这是一百多年以来帝国主义在中国的统治由发展到消灭的转折点。这是一个伟大的事变。这个事变所以带着伟大性,是因为这个事变发生在一个拥有四亿七千五百万人口的国家内,这个事变一经发生,它就将必然地走向全国的胜利。这个事变所以带着伟大性,还因为这个事变发生在世界的东方,在这里,共有十万万以上人口(占人类的一半)遭受帝国主义的压迫。中国人民的解放战争由防御转到进攻,不能不引起这些被压迫民族的欢欣鼓舞。同时,对于正在斗争的欧洲和美洲各国的被压迫人民,也是一种援助。

\section*{二}

从蒋介石发动反革命战争的一天起,我们就说,我们不但必须打败蒋介石,而且能够打败他。我们必须打败蒋介石,是因为蒋介石发动的战争,是一个在美帝国主义指挥之下的反对中国民族独立和中国人民解放的反革命的战争。中国人民的任务,是要在第二次世界大战结束、日本帝国主义被打倒以后,在政治上、经济上、文化上完成新民主主义的改革,实现国家的统一和独立,由农业国变成工业国。然而恰在这时,在反法西斯的第二次世界大战胜利地结束以后,美国帝国主义及其在各国的走狗代替德国和日本帝国主义及其走狗的地位,组成反动阵营,反对苏联,反对欧洲各人民民主国家,反对各资本主义国家的工人运动,反对各殖民地半殖民地的民族运动,反对中国人民的解放。在这种时候,以蒋介石为首的中国反动派,和日本帝国主义的走狗汪精卫一模一样,充当美国帝国主义的走狗,将中国出卖给美国,发动战争,反对中国人民,阻止中国人民解放事业的前进。在这种时候,如果我们表示软弱,表示退让,不敢坚决地起来用革命战争反对反革命战争,中国就将变成黑暗世界,我们民族的前途就将被断送。中国共产党领导中国人民解放军坚决地进行了爱国的正义的革命的战争,反对蒋介石的进攻。中国共产党依据马克思列宁主义的科学,清醒地估计了国际和国内的形势,知道一切内外反动派的进攻,不但是必须打败的,而且是能够打败的。当着天空中出现乌云的时候,我们就指出:这不过是暂时的现象,黑暗即将过去,曙光即在前头。当着一九四六年七月,蒋介石匪帮发动全国规模的反革命战争的时候,蒋介石匪帮认为,只须三个月至六个月,就可以打败人民解放军。他们认为他们有正规军二百万,非正规军一百余万,后方军事机关和部队一百余万,共有军事力量四百余万人;他们已经利用时间完成了进攻的准备;他们重新控制了大城市;他们拥有三万万以上的人口;他们接收了日本侵华军队一百万人的全部装备;他们取得了美国政府在军事上和财政上的巨大援助。他们又认为,中国人民解放军在八年抗日战争中已经打得很疲倦,而且在数量上和装备上远不及国民党军队;中国解放区还只有一万万多一点的人口,其中大部分地区的反动封建势力还没有被肃清,土地改革还不普遍和不彻底,就是说,人民解放军的后方还不是巩固的。从这种估计出发,蒋介石匪帮就不顾中国人民的和平愿望,最后地撕毁在一九四六年一月间签订的国共两党的停战协定\mnote{2}和各党派政治协商会议的决议\mnote{3},发动了冒险的战争。那时,我们说,蒋介石军事力量的优势,只是暂时的现象,只是临时起作用的因素;美国帝国主义的援助,也只是临时起作用的因素;蒋介石战争的反人民的性质,人心的向背,则是经常起作用的因素;而在这方面,人民解放军则占着优势。人民解放军的战争所具有的爱国的正义的革命的性质,必然要获得全国人民的拥护。这就是战胜蒋介石的政治基础。十八个月战争的经验,充分地证明了我们的论断。

\section*{三}

十七个月(一九四六年七月至一九四七年十一月为止,十二月的尚未计入)作战,共打死、打伤、俘虏了蒋介石正规军和非正规军一百六十九万人,其中打死和打伤的有六十四万人,俘虏的有一百零五万人。这样,就使我军打退了蒋介石的进攻,保存了解放区的基本区域,并使自己转入了进攻。我们所以能够如此,在军事方面来说,是因为执行了正确的战略方针。我们的军事原则是:(1)先打分散和孤立之敌,后打集中和强大之敌。(2)先取小城市、中等城市和广大乡村,后取大城市。(3)以歼灭敌人有生力量为主要目标,不以保守或夺取城市和地方为主要目标。保守或夺取城市和地方,是歼灭敌人有生力量的结果,往往需要反复多次才能最后地保守或夺取之。(4)每战集中绝对优势兵力(两倍、三倍、四倍、有时甚至是五倍或六倍于敌之兵力),四面包围敌人,力求全歼,不使漏网。在特殊情况下,则采用给敌以歼灭性打击的方法,即集中全力打敌正面及其一翼或两翼,求达歼灭其一部、击溃其另一部的目的,以便我军能够迅速转移兵力歼击他部敌军。力求避免打那种得不偿失的、或得失相当的消耗战。这样,在全体上,我们是劣势(就数量来说),但在每一个局部上,在每一个具体战役上,我们是绝对的优势,这就保证了战役的胜利。随着时间的推移,我们就将在全体上转变为优势,直到歼灭一切敌人。(5)不打无准备之仗,不打无把握之仗,每战都应力求有准备,力求在敌我条件对比下有胜利的把握。(6)发扬勇敢战斗、不怕牺牲、不怕疲劳和连续作战(即在短期内不休息地接连打几仗)的作风。(7)力求在运动中歼灭敌人。同时,注重阵地攻击战术,夺取敌人的据点和城市。(8)在攻城问题上,一切敌人守备薄弱的据点和城市,坚决夺取之。一切敌人有中等程度的守备、而环境又许可加以夺取的据点和城市,相机夺取之。一切敌人守备强固的据点和城市,则等候条件成熟时然后夺取之。(9)以俘获敌人的全部武器和大部人员,补充自己。我军人力物力的来源,主要在前线。(10)善于利用两个战役之间的间隙,休息和整训部队。休整的时间,一般地不要过长,尽可能不使敌人获得喘息的时间。以上这些,就是人民解放军打败蒋介石的主要的方法。这些方法,是人民解放军在和国内外敌人长期作战的锻炼中产生出来,并完全适合我们目前的情况的\mnote{4}。蒋介石匪帮和美国帝国主义的在华军事人员,熟知我们的这些军事方法。蒋介石曾多次集训他的将校,将我们的军事书籍和从战争中获得的文件发给他们研究,企图寻找对付的方法。美国军事人员曾向蒋介石建议这样那样的消灭人民解放军的战略战术;并替蒋介石训练军队,接济军事装备。但是所有这些努力,都不能挽救蒋介石匪帮的失败。这是因为我们的战略战术是建立在人民战争这个基础上的,任何反人民的军队都不能利用我们的战略战术。在人民战争的基础上,在军队和人民团结一致、指挥员和战斗员团结一致以及瓦解敌军等项原则的基础上,人民解放军建立了自己的强有力的革命的政治工作,这是我们战胜敌人的重大因素。当着我们避开优势敌人的致命打击,并转移军力求得在运动中歼灭敌人,而主动地放弃许多城市的时候,我们的敌人是兴高采烈了。他们认为这就是他们的胜利和我们的失败。他们被一时的所谓胜利冲昏了头脑。张家口被占领的当天下午,蒋介石即下令召集他的反动的国民大会\mnote{5},似乎他的反动统治从此可以安如泰山了。美国帝国主义分子也手舞足蹈,似乎他们将中国变为美国殖民地的狂妄计划,从此可以毫无阻碍地实现了。但是,随着时间的推移,蒋介石及其美国主子的腔调也发生了变化。现在是一切国内外敌人都被他们的悲观情绪所统治的时候。他们唉声叹气,大叫危机,一点欢乐的影子也看不见了。十八个月中,蒋介石的前线高级指挥官,大部分因为战败被撤换。这里有郑州的刘峙,徐州的薛岳,苏北的吴奇伟,鲁南的汤恩伯,豫北的王仲廉,沈阳的杜聿明、熊式辉,北平的孙连仲等人。负指挥全部作战责任的蒋介石的参谋总长陈诚,亦被取消此种指挥职权,降为东北一个战场的指挥官\mnote{6}。而在蒋介石自己代替陈诚担任全局指挥的期间,却发生了蒋军由进攻转入防御,人民解放军由防御转入进攻这样一个局面。蒋介石反动集团及其美国主子,现在应当感觉到他们自己的错误了。他们将日本投降以后一个长时间内,中国共产党代表中国人民的愿望,力争和平反对内战的一切努力,看作是胆怯和力量薄弱的表现。他们过高地估计了自己力量,过低地估计了革命力量,冒险地发动战争,因而落在他们自己布置的陷阱里。我们敌人的战略打算是彻底地输了。

\section*{四}

现在,比较十八个月以前,人民解放军的后方也巩固得多了。这是由于我党坚决地站在农民方面实行土地改革的结果。在抗日战争时期,为着同国民党建立抗日统一战线和团结当时尚能反对日本帝国主义的人们起见,我党主动地把抗日以前的没收地主土地分配给农民的政策,改变为减租减息的政策,这是完全必需的。日本投降以后,农民迫切地要求土地,我们就及时地作出决定,改变土地政策,由减租减息改为没收地主阶级的土地分配给农民。我党中央一九四六年五月四日发出的指示,就是表现这种改变。一九四七年九月,我党召集了全国土地会议,制定了中国土地法大纲\mnote{7},并立即在各地普遍实行。这个步骤,不但肯定了去年《五四指示》的方针,而且对于去年《五四指示》中的某些不彻底性作了明确的改正。中国土地法大纲规定,在消灭封建性和半封建性剥削的土地制度、实行耕者有其田的土地制度的原则下,按人口平均分配土地\mnote{8}。这是最彻底地消灭封建制度的一种方法,这是完全适合于中国广大农民群众的要求的。为着坚决地彻底地进行土地改革,乡村中不但必须组织包括雇农贫农中农在内的最广泛群众性的农会及其选出的委员会,而且必须首先组织包括贫农雇农群众的贫农团及其选出的委员会,以为执行土地改革的合法机关,而贫农团则应当成为一切农村斗争的领导骨干。我们的方针是依靠贫农,巩固地联合中农,消灭地主阶级和旧式富农的封建的和半封建的剥削制度。地主富农应得的土地和财产,不能超过农民群众。但是,曾经在一九三一年至一九三四年期间实行过的所谓“地主不分田,富农分坏田”的过左的错误的政策,也不应重复。地主富农在乡村人口中所占的比例,虽然各地有多有少,但按一般情况来说,大约只占百分之八左右(以户为单位计算),而他们占有的土地,按照一般情况,则达全部土地的百分之七十至八十。因此,我们的土地改革所反对的对象,人数甚少,而乡村中能够参加和应当参加土地改革统一战线的人数(户数),则有大约百分之九十以上这样多。这里必须注意两条基本原则:第一,必须满足贫农和雇农的要求,这是土地改革的最基本的任务;第二,必须坚决地团结中农,不要损害中农的利益。只要我们掌握了这两条基本原则,我们的土地改革任务就一定能够胜利地完成。旧式富农按照平分原则所多余的土地及其一部分财产之所以应当拿出来分配,是因为中国的富农一般地带着很重的封建和半封建剥削的性质,富农大都兼出租土地和放高利贷,其雇佣劳动的条件亦是半封建的\mnote{9}。还因为他们所占的土地数量较多,质量较好\mnote{10},如不平分则不能满足贫雇农的要求。但是按照土地法大纲的规定,对待富农和对待地主一般地应当有所区别。土地改革中,中农表现赞成平分,这是因为平分并不损害中农利益。在平分时,中农中一部分土地不变动,一部分增加了土地,只有一部分富裕中农有少数多余的土地,他们也愿意拿出来平分,这是因为在平分后他们的土地税的负担也减轻了。虽然如此,各地在平分土地时,仍须注意中农的意见,如果中农不同意,则应向中农让步。在没收分配封建阶级的土地财产时应当注意某些中农的需要。在划分阶级成分时,必须注意不要把本来是中农成分的人,错误地划到富农圈子里去。在农会委员会中,在政府中,必须吸收中农积极分子参加工作。在土地税和支援战争的负担上,必须采取公平合理的原则。这些,就是我党在执行巩固地联合中农这一战略任务时所必须采取的具体政策。全党必须明白,土地制度的彻底改革,是现阶段中国革命的一项基本任务。如果我们能够普遍地彻底地解决土地问题,我们就获得了足以战胜一切敌人的最基本的条件。

\section*{五}

为了坚决地彻底地实行土地改革,巩固人民解放军的后方,必须整编党的队伍。抗日战争时期我党内部的整风运动\mnote{11},是一般地收到了成效的。这种成效,主要地是在于使我们的领导机关和广大的干部和党员,进一步地掌握了马克思列宁主义的普遍真理和中国革命的具体实践的统一这样一个基本的方向。在这点上我们党是比抗日以前的几个历史时期,大进一步了。但是,在党的地方组织方面,特别是在党的农村基层组织方面所存在的成分不纯和作风不纯的问题,则没有获得解决。一九三七年至一九四七年,十一年时间内,我们党的组织,由几万党员,发展到了二百七十万党员,这是一个极大的跃进。这使我们的党成了一个在中国历史上空前强大的党。这使我们有可能打败日本帝国主义,并打退蒋介石的进攻,领导一万万以上人口的解放区和二百万人民解放军。但是缺点也就跟着来了。这即是有许多地主分子、富农分子和流氓分子乘机混进了我们的党。他们在农村中把持许多党的、政府的和民众团体的组织,作威作福,欺压人民,歪曲党的政策,使这些组织脱离群众,使土地改革不能彻底。这种严重情况,就在我们面前提出了整编党的队伍的任务。这个任务如果不解决,我们在农村中就不能前进。党的全国土地会议彻底地讨论了这个问题,并规定了适当的步骤和方法。这些步骤和方法,现在正和平分土地的决定一道在各地坚决地实施。其中首先重要的,是在党内展开批评和自我批评,彻底地揭发各地组织内的离开党的路线的错误思想和严重现象。全党同志必须明白,解决这个党内不纯的问题,整编党的队伍,使党能够和最广大的劳动群众完全站在一个方向,并领导他们前进,是解决土地问题和支援长期战争的一个决定性的环节。

\section*{六}

没收封建阶级的土地归农民所有,没收蒋介石、宋子文、孔祥熙、陈立夫为首的垄断资本归新民主主义的国家所有,保护民族工商业。这就是新民主主义革命的三大经济纲领。蒋宋孔陈四大家族,在他们当权的二十年中,已经集中了价值达一百万万至二百万万美元的巨大财产,垄断了全国的经济命脉。这个垄断资本,和国家政权结合在一起,成为国家垄断资本主义。这个垄断资本主义,同外国帝国主义、本国地主阶级和旧式富农密切地结合着,成为买办的封建的国家垄断资本主义。这就是蒋介石反动政权的经济基础。这个国家垄断资本主义,不但压迫工人农民,而且压迫城市小资产阶级,损害中等资产阶级。这个国家垄断资本主义,在抗日战争期间和日本投降以后,达到了最高峰,它替新民主主义革命准备了充分的物质条件。这个资本,在中国的通俗名称,叫做官僚资本。这个资产阶级,叫做官僚资产阶级,即是中国的大资产阶级。新民主主义的革命任务,除了取消帝国主义在中国的特权以外,在国内,就是要消灭地主阶级和官僚资产阶级(大资产阶级)的剥削和压迫,改变买办的封建的生产关系,解放被束缚的生产力。被这些阶级及其国家政权所压迫和损害的上层小资产阶级和中等资产阶级,虽然也是资产阶级,却是可以参加新民主主义革命,或者保守中立的。他们和帝国主义没有联系,或者联系较少,他们是真正的民族资产阶级。在新民主主义的国家权力到达的地方,对于这些阶级,必须坚决地毫不犹豫地给以保护。蒋介石统治区域的上层小资产阶级和中等资产阶级,其中有为数不多的一部分人,即这些阶级的右翼分子,存在着反动的政治倾向,他们替美国帝国主义和蒋介石反动集团散布幻想,他们反对人民民主革命。当着他们的反动倾向尚能影响群众时,我们应当向着接受他们影响的群众进行揭露的工作,打击他们在群众中的政治影响,使群众从他们的影响之下解放出来。但是,政治上的打击和经济上的消灭是两件事,如果混同这两件事,我们就要犯错误。新民主主义革命所要消灭的对象,只是封建主义和垄断资本主义,只是地主阶级和官僚资产阶级(大资产阶级),而不是一般地消灭资本主义,不是消灭上层小资产阶级和中等资产阶级。由于中国经济的落后性,广大的上层小资产阶级和中等资产阶级所代表的资本主义经济,即使革命在全国胜利以后,在一个长时期内,还是必须允许它们存在;并且按照国民经济的分工,还需要它们中一切有益于国民经济的部分有一个发展;它们在整个国民经济中,还是不可缺少的一部分。这里所说的上层小资产阶级,是指雇佣工人或店员的小规模的工商业者。此外,还有不雇佣工人或店员的广大的独立的小工商业者,对于这些小工商业者,不待说,是应当坚决地保护的。革命在全国胜利以后,由于新民主主义国家手里有着从官僚资产阶级接收过来的控制全国经济命脉的巨大的国家企业,又有从封建制度解放出来、虽则在一个颇长时间内在基本上仍然是分散的个体的、但是在将来可以逐步地引向合作社方向发展的农业经济,在这些条件下,这种小的和中等的资本主义成分,其存在和发展,并没有什么危险。土地改革后,在农村中必然发生的新的富农经济,也是如此。对于上层小资产阶级和中等资产阶级经济成分采取过左的错误的政策,如像我们党在一九三一年至一九三四年期间所犯过的那样(过高的劳动条件,过高的所得税率,在土地改革中侵犯工商业者,不以发展生产、繁荣经济、公私兼顾、劳资两利为目标,而以近视的片面的所谓劳动者福利为目标),是绝对不许重复的。这些错误如果重犯,必然要损害劳动群众的利益和新民主主义国家的利益。中国土地法大纲上有一条规定:“保护工商业者的财产及其合法的营业,不受侵犯。”这里所说的工商业者,就是指的一切独立的小工商业者和一切小的和中等的资本主义成分。总起来说,新中国的经济构成是:(1)国营经济,这是领导的成分;(2)由个体逐步地向着集体方向发展的农业经济;(3)独立小工商业者的经济和小的、中等的私人资本经济。这些,就是新民主主义的全部国民经济。而新民主主义国民经济的指导方针,必须紧紧地追随着发展生产、繁荣经济、公私兼顾、劳资两利这个总目标。一切离开这个总目标的方针、政策、办法,都是错误的。

\section*{七}

一九四七年十月,人民解放军发表宣言,其中说:“联合工农兵学商各被压迫阶级、各人民团体、各民主党派、各少数民族、各地华侨和其它爱国分子,组成民族统一战线,打倒蒋介石独裁政府,成立民主联合政府。”这就是人民解放军的、也是中国共产党的最基本的政治纲领。从表面上看来,现在时期,比较抗日时期,我们的革命的民族统一战线,似乎是缩小了。但是在实际上,只是在现在时期,只是在蒋介石出卖民族利益给美国帝国主义,发动反人民的全国规模的国内战争之后,只是在美国帝国主义和蒋介石反动统治集团的罪恶已经在中国人民面前暴露无遗之后,我们的民族统一战线才是真正地扩大了。在抗日时期,蒋介石和国民党在中国人民中还没有完全丧失威信,他们还有许多的欺骗作用。现在不同了,他们的一切欺骗都已被他们自己的行为所揭穿,他们已经没有什么群众,他们已经完全孤立了。和国民党相反,中国共产党不但在解放区得到最广大人民群众的信任;在国民党统治区,在国民党控制的大城市,也得到了广大人民群众的拥护。如果说,在一九四六年,在蒋介石统治下的上层小资产阶级和中等资产阶级的知识分子中,还有一部分人怀着所谓第三条道路\mnote{12}的想法,那末,在现在,这种想法已经破产了。由于我党采取了彻底的土地政策,使我党获得了比较抗日时期广大得多的农民群众的衷心拥护。由于美国帝国主义的侵略、蒋介石的压迫和我党坚决保护群众利益的正确方针,我党获得了蒋介石统治区域工人阶级、农民阶级、城市小资产阶级和中等资产阶级的广大群众的同情。这些群众,因为挨饿,因为政治上受压迫,因为蒋介石的反人民的内战夺去了人民的一切活路,他们就不断地掀起了反对美国帝国主义和蒋介石反动政府的斗争,他们的基本口号是反饥饿,反迫害,反内战和反对美国干涉中国内政。而在抗日以前,在抗日时期,乃至在日本投降后的一个时期,他们的觉悟都没有达到这样的程度。因此我们说,我们的新民主主义的革命的统一战线,现在比过去任何时期都要广大,也比过去任何时期都要巩固。这件事,不但同我们的土地政策和城市政策相联系,而且同人民解放军的胜利,同蒋介石由进攻转入防御,人民解放军由防御转入进攻,中国革命已经进入新的高潮时期,这一总的政治形势,密切地联系着。现在,人们看到了蒋介石统治的灭亡已经不可避免,因而将希望寄托在中国共产党和人民解放军身上,这是很自然的道理。中国新民主主义的革命要胜利,没有一个包括全民族绝大多数人口的最广泛的统一战线,是不可能的。不但如此,这个统一战线还必须是在中国共产党的坚强的领导之下。没有中国共产党的坚强的领导,任何革命统一战线也是不能胜利的。在一九二七年北伐战争达到高潮的时期,我党领导机关的投降主义分子,自愿地放弃对于农民群众、城市小资产阶级和中等资产阶级的领导权,尤其是放弃对于武装力量的领导权,使那次革命遭到了失败。抗日战争时期,我党反对了和这种投降主义思想相类似的思想,即是对于国民党的反人民政策让步,信任国民党超过信任人民群众,不敢放手发动群众斗争,不敢在日本占领地区扩大解放区和扩大人民的军队,将抗日战争的领导权送给国民党。我党对于这样一种软弱无能的腐朽的违背马克思列宁主义原则的思想,进行了坚决的斗争,坚决地执行了“发展进步势力,争取中间势力,孤立顽固势力”的政治路线,坚决地扩大了解放区和人民解放军。这样,就不但保证了我党在日本帝国主义侵略时期能够战胜日本帝国主义,而且保证了我党在日本投降以后蒋介石举行反革命战争时期,能够顺利地不受损失地转变到用人民革命战争反对蒋介石反革命战争的轨道上,并在短时期内取得了伟大的胜利。这些历史教训,全党同志都要牢记。

\section*{八}

蒋介石反动集团在一九四六年发动全国规模的反人民的国内战争的时候,他们之所以敢于冒险,不但依靠他们自己的优势的军事力量,而且主要地依靠他们认为是“异常强大”的、“举世无敌”的、手里拿着原子弹的美国帝国主义。一方面,以为它能够像流水一样地供给他们以军事上和财政上的需要;另一方面,狂妄地设想所谓“美苏必战”,所谓“第三次世界大战必然爆发”。这种对于美国帝国主义的依赖,是第二次世界大战结束以后全世界各国反动势力的共同特点。这件事,反映了第二次世界大战给予世界资本主义的打击的严重性,反映了各国反动派力量的薄弱及其心理的恐慌和丧失信心,反映了全世界革命力量的强大,使得各国反动派除了依靠美国帝国主义的援助,就感到毫无出路。但是,在实际上,在第二次世界大战以后的美国帝国主义,是否真如蒋介石和各国反动派所设想的那么强大呢?是否真能像流水一样地接济蒋介石和各国反动派呢?并不如此。美国帝国主义在第二次世界大战期间所增强起来的经济力量,遇着了不稳定的日趋缩小的国内市场和国际市场。这种市场的进一步缩小,就要引起经济危机的爆发。美国的战争景气,仅仅是一时的现象。它的强大,只是表面的和暂时的。国内国外的各种不可调和的矛盾,就像一座火山,每天都在威胁美国帝国主义,美国帝国主义就是坐在这座火山上。这种情况,迫使美国帝国主义分子建立了奴役世界的计划,像野兽一样,向欧亚两洲和其它地方乱窜,集合各国的反动势力,那些被人民唾弃的渣滓,组成帝国主义和反民主的阵营,反对以苏联为首的一切民主势力,准备战争,企图在将来,在遥远的时间内,有一天发动第三次世界大战打败民主力量。这是一个狂妄的计划。全世界民主势力必须打败这个计划,也完全能够打败它。全世界反帝国主义阵营的力量超过了帝国主义阵营的力量。优势是在我们方面,不是在敌人方面。以苏联为首的反帝国主义阵营,已经形成。没有危机的、向上发展的、受到全世界广大人民群众爱护的社会主义的苏联,它的力量,现在就已经超过了被危机严重威胁着的、向下衰落的、受到全世界广大人民群众反对的帝国主义的美国。欧洲各人民民主国家,正在巩固其内部,并互相团结起来。以法意为首的欧洲各资本主义国家内的人民的反帝国主义力量,正在发展。美国内部,存在着日趋强大的人民民主势力。拉丁美洲的人民,并不是顺从美国帝国主义的奴隶。整个亚洲,兴起了伟大的民族解放运动。反帝国主义阵营的一切力量,正在团结起来,并正在向前发展。欧洲九个国家的共产党和工人党,业已组成了情报局,发表了号召全世界人民起来反对帝国主义奴役计划的檄文\mnote{13}。这篇檄文,振奋了全世界被压迫人民的精神,指示了他们的斗争方向,巩固了他们的胜利信心。全世界的反动派,在这篇檄文面前惊惶失措。东方各国一切反帝国主义的力量,也应当团结起来,反对帝国主义和各国内部反动派的压迫,以东方十万万以上被压迫人民获得解放为奋斗的目标。我们自己的命运完全应当由我们自己来掌握。我们应当在自己内部肃清一切软弱无能的思想。一切过高地估计敌人力量和过低地估计人民力量的观点,都是错误的。我们和全世界民主力量一道,只要大家努力,一定能够打败帝国主义的奴役计划,阻止第三次世界大战使之不能发生,推翻一切反动派的统治,争取人类永久和平的胜利。我们清醒地知道,在我们的前进道路上,还会有种种障碍,种种困难,我们应当准备对付一切内外敌人的最大限度的抵抗和挣扎。但是,只要我们能够掌握马克思列宁主义的科学,信任群众,紧紧地和群众一道,并领导他们前进,我们是完全能够超越任何障碍和战胜任何困难的,我们的力量是无敌的。现在是全世界资本主义和帝国主义走向灭亡,全世界社会主义和人民民主主义走向胜利的历史时代,曙光就在前面,我们应当努力。


\begin{maonote}
\mnitem{1}关于人民解放军在各个战场陆续转入进攻,打到国民党统治区的情况,见本卷\mxnote{评西北大捷兼论解放军的新式整军运动}{2}。
\mnitem{2}见本卷\mxnote{以自卫战争粉碎蒋介石的进攻}{1}。
\mnitem{3}见本卷\mxnote{以自卫战争粉碎蒋介石的进攻}{2}。
\mnitem{4}一九五八年六月二十三日,毛泽东在中央军委扩大会议小组长座谈会上的讲话中指出:十大军事原则,是根据十年内战、抗日战争、解放战争前期的经验,在解放战争进入反攻时期提出来的,是马克思列宁主义普遍真理同中国革命战争实践相结合的产物。运用十大原则,取得了解放战争、抗美援朝战争的胜利(当然还有其它原因)。十大原则目前还可以用,今后有许多地方还可以用。但马克思列宁主义不是停止的,是向前发展的,十大原则也要根据今后战争的实际情况,加以补充和发展,有的可能要修正。
\mnitem{5}见本卷\mxnote{美国“调解”真相和中国内战前途}{4}。
\mnitem{6}刘峙原任国民党郑州绥靖公署主任,因为一九四六年九月在定陶战役中失败,当月被撤职。薛岳原任国民党徐州绥靖公署主任,因为他所指挥的国民党军队在一九四六年十二月宿北战役、一九四七年一月鲁南战役、同年二月莱芜战役中,接连遭受严重失败,于三月被撤职。吴奇伟原任国民党徐州绥靖公署副主任,因为一九四六年十二月宿北战役失败,于一九四七年三月被撤职。汤恩伯原任国民党第一兵团司令官,因为一九四七年五月孟良崮战役中国民党整编第七十四师被歼,于六月被撤职。王仲廉原任国民党第四兵团司令官,因为一九四七年七月在鲁西南战役中失败,于八月被撤职。杜聿明原任国民党东北保安司令长官,熊式辉原任国民党东北行辕主任,均因在东北民主联军一九四七年夏季攻势中遭到失败而于八月被撤职。孙连仲原任国民党保定绥靖公署主任,因为一九四七年十月至十一月在清风店、石家庄战役中遭到失败而被撤职。陈诚原任国民党军参谋总长,因指挥山东历次战役屡遭失败,于一九四七年八月兼任东北行辕主任,被取消了参谋总长的实际权力。
\mnitem{7}中国共产党全国土地会议,一九四七年九月举行于河北省平山县西柏坡村。这个会议在九月十三日通过的《中国土地法大纲》,于同年十月十日由中共中央公布。土地法大纲规定:“废除封建性及半封建性剥削的土地制度,实行耕者有其田的土地制度”;“乡村中一切地主的土地及公地,由乡村农会接收,连同乡村中其它一切土地,按乡村全部人口,不分男女老幼,统一平均分配”;“乡村农会接收地主的牲畜、农具、房屋、粮食及其它财产,并征收富农的上述财产的多余部分,分给缺乏这些财产的农民及其它贫民,并分给地主同样的一份”。这样,土地法大纲就不但肯定了一九四六年《五四指示》所提出的“没收地主土地分配给农民”的原则,而且改正了《五四指示》中对某些地主照顾过多的不彻底性。
\mnitem{8}《中国土地法大纲》所规定的平分土地的办法,在以后的执行过程中有了一些改变。一九四八年二月,中共中央在《关于在老区半老区进行土地改革工作与整党工作的指示》中规定:在一切封建制度已被推翻的老区半老区,不再平分土地,而只在必要时采取抽多补少、抽肥补瘦的办法,调剂一部分土地和其它生产资料给尚未彻底翻身的贫雇农,并容许中农保有比较一般贫农所得土地的平均水平为高的土地数量。在封建制度还存在的地方,平分的重点,也限于地主的土地财产和旧式富农的多余的土地财产方面。无论在哪一种地方,对于中农和新式富农的多余土地,只有在确有调剂必要和本人确实同意的条件下,才允许抽出调剂。在新解放区的土地改革中,对一切中农的土地都不再抽动。
\mnitem{9}中国土地改革中的富农问题,是在中国具体的历史经济条件下形成的一个特殊问题。中国的富农,一般具有很重的封建和半封建剥削的性质,而这种富农经济在全国农业经济中又不占重要地位,这两点都不同于许多资本主义国家中的富农。在中国的反对地主阶级封建剥削的斗争中,广大的贫雇农要求同时废除富农的封建和半封建剥削。在解放战争时期,中国共产党采取了征收富农多余的土地和财产分给农民的政策,从而满足了广大贫雇农的要求,保证了人民解放战争的胜利。随着战争的胜利发展,在一九四八年二月,中共中央规定在新解放区实行土地改革的新政策,即将新区土地改革分为两个阶段,在第一阶段,中立富农,专门打击地主,首先是打击大地主;第二阶段,在平分地主土地的时候,也分配富农出租和多余的土地,但是对待富农和对待地主仍然有所区别(见本卷\mxart{新解放区土地改革要点})。在中华人民共和国成立以后,中央人民政府在一九五〇年六月发布土地改革法,规定在土地改革中对富农只征收其出租的土地的一部或全部,对富农的其它土地和财产则予以保护。在以后的社会主义改造进程中,随着农业合作化运动的深入和农村经济的发展,富农经济就不再存在了。
\mnitem{10}这里所说的富农所占的土地数量较多,质量较好,是就每户富农同每户贫农所占的土地的比较而言。从全国说来,中国的富农所占的生产资料和所生产的农产品的数量,都是很小的。富农经济在全国农村经济中并不占有重要的地位。
\mnitem{11}见本书第三卷\mxnote{学习和时局}{11}。
\mnitem{12}在人民解放战争初期,有一些民主人士幻想在国民党的大地主、大资产阶级专政和中国共产党领导的人民民主专政之外,另找所谓第三条道路。这条道路,实际上就是英美式的资产阶级专政的道路。
\mnitem{13}共产党和工人党情报局,是一九四七年九月在波兰华沙举行的保加利亚、罗马尼亚、匈牙利、波兰、苏联、法国、捷克斯洛伐克、意大利、南斯拉夫等九国共产党和工人党代表会议上通过成立的。一九五六年四月宣布停止活动。毛泽东在这里所说情报局发表的号召全世界人民起来反对帝国主义奴役计划的檄文,是指情报局一九四七年九月的会议通过的《关于国际形势的宣言》。
\end{maonote}
