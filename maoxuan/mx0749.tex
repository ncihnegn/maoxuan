
\title{对中国共产党第九次全国代表大会的讲话}
\date{一九六九年四月一日、十一日、二十八日}
\thanks{这是毛泽东同志在第九次全国代表大会上的主要讲话。}
\maketitle


\date{一九六九年四月一日开幕式上的讲话}
\section*{(一)}

同志们!

中国共产党第九次全国代表大会现在开幕。

我希望,我们的大会,能够开得好,能够开成一个团结的大会,胜利的大会。

我们党,从一九二一年成立,到今年已经有四十八年这么长的时间了。第一次代表大会,只有十二个代表。现在在座的还有两个,一个就是董老\mnote{1},再一个就是我。有好几个代表牺牲了,山东的代表王尽美、邓恩铭\mnote{2},湖北的代表陈潭秋\mnote{3},湖南的代表何叔衡\mnote{4},上海的代表李汉俊\mnote{5},都是牺牲了。叛变的,当了汉奸、反革命的,有陈公博、周佛海、张国焘、刘仁静\mnote{6}四个,后头这两个还活着。还有一个叫李达\mnote{7},在早两年去世了。那个时候,全国的党员只有几十个人,大多数是知识分子。后头就发展了。第一次、第二次、第三次\mnote{8}、第四次代表大会,每次到会的代表都很少,十几个人,二十几个人。第五次代表大会在武汉开,人数比较多一些,有几百人\mnote{9}。第六次代表大会在莫斯科开,几十名代表\mnote{10},恩来同志、伯承同志\mnote{11}参加了。

第七次代表大会在延安开的,开了一个团结大会。那个时候,也是党内分歧,因为有瞿秋白、李立三、王明\mnote{12}的错误,特别是王明路线。那个时候,有人建议不要选王明路线那些同志到中央,我们不赞成,说服他们选举。结果呢?结果,就有几个不好了,王明跑到国外反对我们,李立三也是不好的,张闻天、王稼祥\mnote{13}犯了错误,就这么几个。其他一些,比如刘少奇\mnote{14}呀,什么彭真、薄一波\mnote{15}这些人,我们不知道他们不好,他们的政治历史我们不清楚,也选进来了。经过“八大”到现在,搞得比较清楚了。在政治路线上,组织路线上,思想方面,都搞得比较清楚。因此,我们希望这一次大会,能够开成一个团结的大会。在这个团结的基础上,我们能不能取得胜利?就这个大会来说,能不能开成一个胜利的大会?大会以后,能不能在全国取得更大的胜利?我以为,可以的。可以开成一个团结的大会,胜利的大会,大会以后,可以在全国取得更大的胜利!

\date{一九六九年四月十一日}
\section*{(二)}

共和国建立了,社会主义就开始了,但我们没有宣布,土改以后才逐步宣布。土地改革彻底?就不那么彻底。有的是和平土改。对民族资产阶级是利用、限制、改造。

在几年内,国内主要矛盾,是无产阶级和资产阶级的矛盾,已经指出了,但是没有经常讲,所以广大干部不知道,所以这一次又来了二次革命,叫文化大革命。是从文化开始的,叫不叫大革命?以后历史学家去做。叫文化大革命也可以,名称叫什么都可以。主要是针对上层建筑、机关、学校、工厂,这个工作还没有完,恐怕还要一个时期要认真抓一下。

(三)(一九六九年四月二十八日)闭幕式上的讲话

我的话是些老话,就是大家知道的,没有什么新话。无非是讲团结,团结起来的目的,是要争取更大的胜利。现在苏修攻击我们,什么塔斯社的广播,王明的材料,以及《共产党人》的长篇大论,说我们现在不是无产阶级的党,叫做“小资产阶级的党”。说我们搞一元化,回到了过去根据地那个时代,就是讲倒退了。什么叫一元化呢?他们说就是军事、官僚体制。按照日本人的话叫体制;按照苏联的话叫做“军事官僚专政”。他们一看我们名单里头,军人不少,这就叫“军事”;还有什么“官僚”,大概就是我呀,恩来呀,康生呀,陈伯达\mnote{16}呀这批“官僚”。总而言之,你们凡是不是军人,都属于“官僚”系统就是了!所以叫做“军事官僚专政”。这些话嘛,我看让人家去讲!人家怎么讲,就怎么讲。但他有个特点,他就是不骂我们叫资产阶级的政党,而叫“小资产阶级的政党”。而我们呢,就说他是资产阶级的专政,恢复资产阶级专政。

我们讲胜利,就要保证在无产阶级领导之下,团结全国广大人民群众,去争取胜利。社会主义革命还要继续。这个革命,还有些事没有做完,现在还要继续做,比如讲斗、批、改。过若干年,也许又要进行革命。

我们几个老同志,在工厂里头看了一个时期,希望你们以后有机会,还得下去看,还得去研究有关各个工厂里的问题。看来,无产阶级文化大革命不搞是不行的,我们这个基础不稳固。据我观察,不讲全体,也不讲绝大多数,恐怕是相当大的一个多数的工厂里头,领导权不在真正的马克思主义者、不在工人群众手里。过去领导工厂的,不是没有好人。

有好人,党委书记、副书记、委员,都有好人,支部书记有好人。但是,他是跟着过去刘少奇那种路线走,无非是搞什么物质刺激,利润挂帅,不提倡无产阶级政治,搞什么奖金,等等。现在有些工厂已经把他们解放出来了,参加了三结合的领导;有些工厂还没有。但是,工厂里确有坏人。比如二七工厂,就是长辛店铁路机车车辆修理厂,是一个大工厂,八千工人,连家属几万人,过去国民党有九个区分部,三青团有三个机构,另有八个什么特务机构。这个里头当然就要分析了,因为那个时候不加入国民党那个东西是不行呀!有些是老工人了,老工人都不要了吗?那也不行。应该分别轻重,有些人是国民党的挂名党员,不得不加入,只要谈一谈就行了;有些比较负些责任的;有少数,就是钻得很深的,做了坏事的。要分别几种情况。做坏事的,也要分别,坦白从宽,抗拒从严。如果他现在检讨得好,那还应该让他工作,当然不是给领导工作。不让这些人工作,他在家怎么办呢?他的子女怎么办呢?并且老工人一般都是有技术的,虽然技术有些并不那么高明。

我举这么一个例子,就是说明革命没有完。所以整个中央的同志,包括候补中央委员,请你们注意,要过细地做工作。对于这种事情要过细,粗枝大叶不行,粗枝大叶往往搞错。有些地方抓多了人,这个不好。你抓多了人,抓起来干什么呢?他也没有杀人,也没有放火,又没有放毒,我说只要没有这几条,就不要抓。至于犯走资派错误,那更不要抓。工厂里头,要让他工作,要他参加群众运动。人家犯了错误,无非是过去犯的,或者加入国民党,或者做了些坏事,或者是犯了最近一个时期的错误,就是所谓走资派,要他们跟群众一道,如果不让他们跟群众一道,那就不好了。有些人关了两年,关在“牛棚”里头,世界上的事情不知道了,出来一听呀,讲的话不对头了,他还是讲两年前的话。他脱离了两年的生活。对这些人就要帮助了,要办学习班,还要跟他讲历史,讲两年的文化大革命过程的历史,使他逐步清醒。

团结起来,为了一个目标,就是巩固无产阶级专政,要落实到每个工厂、农村、机关、学校。开头不要全面铺开,可以铺开,但不要一铺开不管了。不要搞了半年或者更久,就是没有人去管它。要一个一个工厂,一个一个学校,一个一个机关地去总结经验。所以,在林彪同志报告\mnote{17}里头讲,要一个一个工厂,一个一个学校,一个一个公社,一个一个党支部,一个一个单位地搞。还有一个一个团支部,整团的问题,也提出来了。

此外,过去讲过的了,就是要准备打仗。无论哪一年,我们要准备打仗。人家就问了:他不来怎么办呢?不管他来不来,我们应该准备。不要造手榴弹都要中央配发材料。手榴弹,到处可以造,各省都可以造。什么步枪,轻武器,每省都可以造。这是讲物质上的准备。而主要的,是要有精神上的准备。精神上的准备,就是要有准备打仗的精神。不仅是我们中央委员会,要使全体人民中间的大多数有这个精神准备。我这里讲的不包括专政对象,什么地富反坏那套人。因为那套人是很高兴帝国主义、修正主义打来的,他以为打来了,这个世界就翻身了,他就可以翻身了。还要准备这一条。

社会主义革命过程还要革这个命。

人家打来,我们不打出去。我们是不打出去的。我说不要受挑拨,你请我去我也不去。但是你打来呢,那我就要对付了。看你是小打还是大打。小打就在边界上打。大打,我主张让出点地方来。中国这个地方不小。他不得点好处,我看他也不会进来。要使全世界看到我们打仗是有理的,有利的。他进来了,我看比较有利,不仅有理,而且有利,好打,使他陷在人民包围中间。至于什么飞机、坦克、装甲车之类,现在到处经验证明,可以对付。

为了胜利,就是要人多一点了,是不是呀?各方面的人,不管你是哪个山头或者哪一个省的,或者是北方、南方的,还是多团结一点人好,还是少团结一点人好呢?总是多团结一点人好。有些人的意见不一定跟我们一致,但是,不是敌我关系。我就不相信,比如具体来讲,说是什么杨得志\mnote{18}跟王效禹\mnote{19}是敌我关系。你们两个人的关系是敌我关系,还是人民内部的关系呀?据我看是人民内部吵吵架。中央也是有点官僚主义,没有大管你们,你们也没有提向中央来讨论。山东这么一个大省,是人民内部矛盾,乘此机会,你们谈一谈嘛,好不好呀?我看华东也有这个人民内部问题。还有山西,也是人民内部,你支一派,我支一派,何必那么尽吵干什么!还有云、贵、川的问题。各地方多多少少都有一些问题就是了,那比较去年跟前年好多了。你这个同志不是叫许世友\mnote{20}吗?前年我们在上海那个时候可不得了,七、八、九三个月。现在日子总好过一点嘛!我说的是整个局面。你那个南京跑出一个什么“红总”,做工作之结果,还是合作了嘛,一个“八·二七”,一个“红总”\mnote{21},还不是合作了嘛。

我说主要问题还是我们的工作。过去不是讲过两句话吗?地方的问题在军队,军队的问题在工作。不是生死冤仇,那何必呢?要讲个人恩怨呀,那个账算不了那么多。总而言之,我看都是前世无冤,今世无仇,碰在一块,有些意见不对头。人家或者是搞什么批评了自己,反对了自己,自己又反攻一下,结果就发生矛盾了。反对自己的人不一定是坏人。

北京经常要打倒的人物之一就叫谢富治\mnote{22}。后头他采取的方针是这样:凡见要打倒他的那些团体,他都说你们无事。

而拥护他的,不一定都是好的。

所以,我讲的还是那几句老话,无非是团结起来,争取更大的胜利。这个里头有具体内容的,干什么事,什么具体的胜利,怎么个团结法。

我相信过去犯错误的一些老同志。原先有个大名单三十几个,我们认为都要选举到政治局委员就好。后头有人提出个小名单,十几个,又觉得它太小了。大多数人是个中间派,反对这个大名单,也反对小名单,主张二十几个人的中等名单,这就只好选代表啰。并不是说候补中央委员就比正式中央委员在政治水平上、工作能力上、德才资各方面差,并不是这个问题。在这个里头有个不公平。你说那么公平哪,我看就不那么公平,不那么公道。

大家要谨慎小心,无论是候补中央委员、中央委员、政治局委员,都要谨慎小心。不要心血来潮的时候,就忘乎所以。从马克思以来,从来不讲什么计较功劳大小。你是共产党员,是整个人民群众中间比较更觉悟的一部分人,是无产阶级里面比较更觉悟的一部分人。所以,我赞成这样的口号,叫做“一不怕苦,二不怕死”,而不赞成那样的口号:“没有功劳也有苦劳,没有苦劳也有疲劳”。这个口号同“一不怕苦,二不怕死”是对立的。你看我们过去死了多少人,我们现在存在的这些老同志,是幸存者,偶然存在下来的。皮定钧\mnote{23}同志,你那个鄂豫皖那个时候多少人?后头剩了多少人?那个时候人可多啦,现在存在的就不那么多了。那个时候,江西苏区,井冈山苏区,赣东北,闽西,湘鄂西,陕北,经过战争有很大的牺牲,老人存下的就不多了,那叫做一不怕苦,二不怕死。多少年我们都是没有啥薪水的,没有定八级工资制,就是吃饭有个定量,叫三钱油、五钱盐、一斤半米就了不起了。至于菜呢?大军所过,哪里能够到处搞到菜吃呀?

现在进了城。这个进城,是好事,不进城,蒋介石霸住这些地方了;进城又是坏事,使得我们这个党不那么好了。所以,有些外国人、新闻记者说,我们这个党在重建。现在我们自己也提出这个口号,叫整党建党。事实是需要重建。每一个支部,都是要重新在群众里头进行整顿。要经过群众,不仅是几个党员,要有党外的群众参加会议,参加评论。个别实在不行的,劝他退出。极少数的人,可能要采取纪律的处分了,党章规定了的,是不是呀?还要经过支部大会,上级批准。总而言之,要采取谨慎的方法。要做,一定要做,但是要采取谨慎的方法。

这次全国代表大会,看起来开得不错。据我看,是开成了一个团结的大会,胜利的大会。我们采取发公报的办法,现在外国人捞不到我们的新闻,说我们开秘密会议。我们是又公开又秘密。北京这些记者,我看也不大行,大概我们把他们混到我们里头的什么叛徒、特务搞得差不多了。过去每开一次会,马上透露出去,红卫兵小报就是起来。自从王、关、戚\mnote{24},杨、余、傅\mnote{25}下台之后,中央的消息他们就不知道了。

\begin{maonote}
\mnitem{1}董老,指董必武,一九二一年七月以武汉共产主义小组代表身分,出席中国共产党第一次全国代表大会。出席中共九大时是中央政治局委员。
\mnitem{2}王尽美、邓恩铭,一九二一年七月以济南共产主义小组代表身分,出席中国共产党第一次全国代表大会,是山东的中共组织最早的组织者和领导者。王尽美一九二五年八日因病逝世。邓恩铭一九二八年底被捕,一九三一年四月被国民党反动派杀害。
\mnitem{3}陈潭秋,一九二一年七月以武汉共产主义小组代表身分,出席中国共产党第一次全国代表大会。一九三九年任中共驻新疆代表和八路军驻新疆办事处主任。一九四二年九月被新疆军阀盛世才秘密逮捕,一九四三年九月被秘密杀害。
\mnitem{4}何叔衡,一九二一年七月以长沙共产主义小组代表身分,出席中国共产党第一次全国代表大会。一九三四年十月中央红军长征后,留在南方坚持斗争。一九三五年二月在转赴上海途中被敌人包围,突围时壮烈牺牲。
\mnitem{5}李汉俊,一九二一年七月以上海共产主义小组代表身分,出席中国共产党第一次全国代表大会。一九二二年自动脱党。一九二七年四一二反革命政变后,对蒋介石的背叛行为进行了愤怒声讨。七月在汪精卫叛变革命后,对国民党右派的反共活动进行了抵制和斗争,掩护了大批共产党员。十二月被桂系军阀杀害于汉口。
\mnitem{6}陈公博,一九二一年七月以广州共产主义小组代表身分,出席中国共产党第一次全国代表大会。一九二五年脱离共产党,加入国民党。抗日战争时期,随汪精卫投降日本侵略者,历任汪伪政府立法院院长、行政院院长、代主席等职。一九四五年抗战胜利后逃往日本,后被押解回国。一九四六年四月被国民党高等法院判处死刑,六月被处决。

周佛海,一九二一年七月以旅日共产主义小组代表身分,出席中国共产党第一次全国代表大会。一九二四年脱离共产党,加入国民党。抗日战争时期,随汪精卫投降日本侵略者,任汪伪政府行政院副院长等职。一九四五年九月被国民党军统局软禁,一九四六年被国民党高等法院判处死刑,一九四七年三月蒋介石下特赦令,改为无期徒刑。一九四八年二月病死狱中。

张国焘,一九二一年七月以北京共产主义小组代表身分,出席中国共产党第一次全国代表大会。曾任中共中央委员、中央政治局委员、中央政治局常委,一九三五年六月红军第一、第四方面军在四川懋功(今小金)地区会师后,任红军总政治委员。他反对中央关于红军北上的决定,进行分裂、危害党和红军的活动,另立中央。一九三六年六月被迫取消第二中央,随后与红军第二、第四方面军一起北上,十二月到达陕北。一九三七年九月起,任陕甘宁边区政府代主席。一九三八年四月,乘祭黄帝陵之机逃出陕甘宁边区,投入国民党特务集团,随即被开除出党。一九七九年死于加拿大。

刘仁静,一九二一年七月以北京共产主义小组代表身分,出席中国共产党第一次全国代表大会。一九二六年赴苏联莫斯科国际列宁主义学院学习,后参加托派。一九二九年被开除出党。建国后长期担任人民出版社特约编辑,从事翻译工作。一九八七年任国务院参事室参事。同年八月因车祸去世。
\mnitem{7}李达,一九二一年七月以上海共产主义小组代表身分,出席中国共产党第一次全国代表大会。一九二三年与陈独秀在国共合作问题上发生激烈争论,随后脱离党组织。此后长期从事理论研究和教育工作,坚持宣传马列主义,并积极参加党所领导的一些革命活动。一九四九年参加中国人民政治协商会议第一届全体会议,当选为全国治协委员。同年十二月,经中共中央批准,重新加入中国共产党。一九六六年八月逝世。
\mnitem{8}中国共产党第三次全国代表大会到会的代表有三十多人。
\mnitem{9}中国共产党第五次全国代表大会一九二七年四月二十七日至五月十日在武汉举行,出席代表八十多人。
\mnitem{10}中国共产党第六次全国代表大会一九二八年六月十八日至七月十一日在莫斯科举行,出席这次大会的中央主要领导人和各地代表共一百四十二人,其中有选举权的代表八十四人。
\mnitem{11}恩来,即周恩来。伯承,即刘伯承,出席中共九大时是中央政治局委员。
\mnitem{12}瞿秋白,一九二七年在中共五大上当选为中央委员、中央政治局委员。第一次大革命失败后,主持召开八七会议,会后任中央政治局常委,主持中央工作。一九二七年十一月至一九二八年四月犯过“左”倾盲动主义错误。一九三〇年九月主持召开中共六届三中全会,纠正李立三“左”倾冒险主义错误。在一九三一年一月中共六届四中全会上,受到王明“左”倾教条主义宗派的打击,被排斥于中央领导机关之外。一九三四年任中华苏维埃共和国中央政府教育人民委员。中央红军长征后留在南方坚持斗争。一九三五年二月在转移途中被国民党逮捕,六月十八日在福建长汀就义。

李立三,一九三〇年六月至九月,在担任中共中央政治局常委兼秘书长、中央宣传部部长,并实际主持中央工作期间,犯了“左”倾冒险主义错误。一九六七年六月逝世。

王明,一九三一年一月在共产国际代表的支持下,在中共六届四中全会被补选为中央委员、中央政治局委员。此后,以他为代表的“左”倾冒险主义路线在党中央统治长达四年之久,给党和革命事业造成重大损失。在中共七大上,经毛泽东等中央领导人做工作,他继续当选为中央委员。一九五六年一月去苏联后,长期进行反对中共中央的活动。一九七四年病死在莫斯科。
\mnitem{13}张闻天,一九二五年加入中国共产党。一九三四年一月在中共六届五中全会上当选为中央政治局委员,二月当选为中华苏维埃共和国中央政府人民委员会主席,十月参加长征。一九三五年一月参加遵义会议,支持和拥护以毛泽东为代表的正确路线,作批判“左”倾军事路线的报告。会后根据中央政治局常委分工代替博古负总责。一九四五年在中共七大上当选中央委员,在中共七届一中全会上当选中央政治局委员。一九五六年在中共八届一中全会上当选中央政治局候补委员。一九五九年在庐山会议上受到批判。一九七六年七月逝世。

王稼祥,一九二五年加入中国共产主义青年团,一九二八年转入中国共产党。一九三四年一月在中共六届五中全会上被增选为中央委员、中央政治局候补委员。一九三五年一月在遵义会议上支持和拥护以毛泽东为代表的正确路线。会后被增选为中央政治局委员,同毛泽东、周恩来组成中央三人军事领导小组。一九四五年在中共七大上当选为候补中央委员。一九五六年中共八大后任中央委员、中央书记处书记。一九七三年八月在中共十大上当选中央委员。一九七四年一月逝世。
\mnitem{14}刘少奇,原任中共中央副主席、中华人民共和国主席。一九六八年被诊断为“肺炎杆菌性肺炎”,在七月中旬的一次发病后,虽经尽力抢救,从此丧失意识,一九六八年十月中共八届十二中全会通过《关于叛徒、内奸、工贼刘少奇罪行的审查报告》。这次全会公报,宣布了中央“把刘少奇永远开除出党,撤销其党内外的一切职务”的决议。一九六九年十月,在战备大疏散中被疏散到开封,同年十一月十二日逝世。
\mnitem{15}彭真,一九二三年加入中国共产党,曾任中共天津市委书记、顺直省委代理书记。一九二九年被捕。一九三五年刑满出狱。一九四五年,在中共七大上当选为中央委员,在七届一中全会上当选为中央政治局委员,八月任中央书记处候补书记。建国后曾任中央政治局委员、中央书记处书记、北京市委书记、北京市市长等职。曾拒绝刊登《评〈海瑞罢官〉》一文,在“五·一六通知”中被点名批判。

薄一波,一九二五年加入中国共产党。一九二九年后在顺直省委指导兵运工作。后被捕入狱,在狱中任中共支部书记。一九三六年经中共中央批准履行监狱方手续后集体出狱,后到山西主持牺牲救国同盟会的工作。一九四五在中共七大上当选为中央委员。建国后曾任中央政治局候补委员、国务院副总理。一九六七年三月十六日,中共中央发布了《关于薄一波、刘澜涛、安子文、杨献珍等人自首叛变问题的初步调查》,认定一九三六年在北平草岚子监狱履行登报《反共启事》手续出狱的薄一波等人为“六十一人叛徒集团案”。

薄一波曾追随刘邓路线,但其晚年深刻反思,发出了“我总算看明白了,他(邓小平)所做的一切只为一个目的,就是专门和毛主席对着干。凡是毛主席生前所肯定的,他统统都要否定;凡是毛主席生前所否定的,他统统都要肯定”的感叹。二〇〇七年,薄一波去世后,右派文人曾托其名以《十九次谈话》为名散布反毛言论,其子女薄熙来等公开郑重辟谣:“他从一九二五年入党,已有八十多年党龄,始终热爱党,崇敬毛主席。”“此文无中生有,刻意编造,显然别有用心,扰乱试听。本着对大众和传媒负责的态度,我们特此澄清。”其子薄熙成接受电视采访时说:“毛主席在我父亲心中是最伟大的领袖和他的导师”。
\mnitem{16}恩来,即周恩来。康生,陈伯达,在中共九届一中全会上均当选为中央政治局常委。
\mnitem{17}指林彪一九六九年四月一日代表中央委员会向中国共产党第九次全国代表大会所作的政治报告。
\mnitem{18}杨得志,时任山东省革委会第一副主任、中国人民解放军济南军区司令员。
\mnitem{19}王效禹,时任山东省革委会主任、中国人民解放军济南军区第一政委。
\mnitem{20}许世友,时任江苏省革委会主任、中国人民解放军南京军区司令员。
\mnitem{21}“八·二七”、“红总”,是当时南京的两个群众造反组织。
\mnitem{22}谢富治,时任中共中央政治局委员、国务院副总理、北京市革委会主任。
\mnitem{23}皮定钧,时任福建省革委会第一副主任、中国人民解放军福州军区副司令员。
\mnitem{24}指的是王力、关锋和戚本禹三人,一九六六年后,王力、关锋和戚本禹相继成为《红旗》副总编辑、“中央文革小组”的成员,曾为中央文革成员,王力被任命为中央宣传组组长,关锋成为总政副主任、“军委文革小组”副组长,并受林彪委托兼管《解放军报》,而戚本禹则是中央办公厅秘书局副局长、中共中央办公厅代主任。

一九六七年八月中共中央决定对王力、关锋隔离审查。一九六八年一月中共中央决定对戚本禹隔离审查。
\mnitem{25}杨,指杨成武,原任中共中央军委常委兼副秘书长、全军文革小组副组长、中国人民解放军代总参谋长。余,指余立金,原任中国人民解放军空军政委。傅,指傅崇碧,原任中国人民解放军北京卫戍区司令员。一九六八年的三月二十四日,中共中央撤销了他们三人的职务,隔离审查。
\end{maonote}
