
\title{批评“二月逆流”\mnote{1}}
\date{一九六七年二月十八日}
\thanks{这是毛泽东同志召集部分政治局委员的谈话。}
\maketitle


我听说二月十六日下午有人在怀仁堂闹事,反对中央文革小组。反对中央文革小组就是反对我,反对我们这个党嘛!这半年多来,中央文革做了大量工作,伯达、江青、康生等同志都做了大量工作,你们看见了没有?

谭震林、徐向前、陈毅向党发难,满口放屁。

中央文革小组坚决执行了八届十一中全会精神,成绩是第一位的,是主要的。如果打分的话,我看可以打九十七分。你们谁反对中央文革,我就坚决反对谁!你们想否定文化大革命,我告诉你们:这绝对办不到!

你们都想闹事,那就闹嘛!无非是文化大革命失败,我马上走,林彪同志也走,我们重上井冈山,重新闹革命。你们说江青、陈伯达不行,那就把文革小组改组,让谭震林当组长,陈毅、徐向前当副组长,余秋里\mnote{2}、薄一波\mnote{3}当组员,再不行,把王明、张国焘都请回来。力量还不够,那就请美国、苏联一块来。你们把江青、陈伯达枪毙,康生充军,其他人你们爱怎么办就怎么办。这下总行了吧!这下就达到你们的目的了吧!

谭震林算什么老革命,还有我呢!你陈毅要翻延安整风的案,全党答应么?谭震林、陈毅都是老党员,为什么要站在资产阶级立场上说话呢!

我提议:这件事政治局要开会讨论,一次不行就开两次,一个月不行就开两个月,政治局解决不了,就发动全体党员解决!

\begin{maonote}
\mnitem{1}二月逆流,起因是陶铸被打倒,陶铸是八届十一中全会上被选为中央政治局常委,后又任副总理、中央文革顾问。一九六六年十月九日至二十八日召开中央工作会议,批判刘邓的错误。中组部组织三百多群众,要求去中南海向毛主席、党中央送交决心书,表示彻底揭发批判刘邓,陶铸却说:“组织部几百人去中南海喊打倒刘少奇,贴他的大字报,这种做法我不赞成。”十一月初,看到批判刘邓的大字报,陶铸对群众说:“不能把刘少奇同志叫作敌人,不能喊打倒。”其实,陶铸死保刘邓的立场是他到中央宣传部担任领导工作以后就表现出来了。他开始以极右的面目出现,反对毛主席《炮打司令部》这张伟大的大字报;群众起来了,他又跳到极左,在群众面前大叫大喊:“在文化大革命中,怀疑一切是正确的”,“每个司令部都不知道是什么司令部……我是主张普遍轰!”更为恶劣的是,他利用掌管宣传大权,在报道八届十一中全会决议时,还大登刘邓的照片,并授意伪造毛泽东和刘少奇在一起的照片。最突出的是他的“割头术”,即把陈毅的头割去,换上邓小平的头,以示邓小平站在天安门城楼上,仍然是党和国家的领导人。这张照片发出以后,被许多省市报纸刊用,在全国起了极恶劣的影响。

针对陶铸这种保皇立场,一九六七年一月八日,中央召开紧急会议。会上,毛泽东讲了如下一段话:“陶铸问题很严重,陶铸这个人是邓小平介绍到中央来的。我起初说,陶铸这个人不老实,邓小平说,陶铸还可以。陶铸在十一中全会前坚决执行了刘邓路线,在红卫兵接见时,在报纸上和电视里有刘邓的照片镜头,是陶铸安排的。陶铸领导下的八个部都垮了。那些部可以不要,搞革命不一定都要部。教育部管不了,我们也管不了,红卫兵一来就能管了。陶铸的问题我们没有解决了,红卫兵起来就解决了。”

陶铸在被打倒以后,曾和他的老婆曾志谈起他和刘少奇的关系:“一九四二年我在军委工作期间,刘从华中回到延安,彼此才认识,我对他印象不错。后来,在对待柯老(指柯庆施)的问题上,我觉得刘比较偏,评论不太公正。因此,一直到一九五三年,我对刘都是敬而远之的。在中央财经会议上,我向刘开了一炮,差点被高岗利用。高饶事件后,刘非但没有批评我,相反向我做了耐心的解释,并且承认他看人可能有些偏。尤其一九五九年三年困难时期,在社会主义建设的领导方针政策上,我都是赞成刘的观点的。”他还讲:庐山会议上,他曾以女人嫁丈夫的比喻劝说黄克诚:“你我都读过一点所谓古圣贤之书,一个人过身于世,不讲究操守是可悲的。尤其我们作为一个党员,对党的忠诚等于旧社会一个女人嫁了人一样,一定要‘从一而终’,决不可‘移情别恋’,否则便不能称为‘贞节’之妇。”从这些自白里可以看出,陶铸的保皇立场是有很深的历史根源和思想根源的。姚文元在《评陶铸的两本书》里揭露批判陶铸是一个赫鲁晓夫式的野心家,“从政治上、文化上到生活上,他的‘理想’都是在中国搞资本主义复辟。他头脑中装满了从叛徒哲学到‘士为知己者死’之类剥削阶级反动的世界观。”陶铸确实是刘邓的“贞洁之妇”,在刘邓垮台之后,他仍然“从一而终”,作为刘邓的代言人,在无产阶级司令部里兴风作浪。但是,他没有逃过毛泽东的火眼金睛。

打倒陶铸是完全必要的。但是,作为中央文革小组的负责人江青、陈伯达,却采取了不讲策略的方式,受到毛主席的严厉批评。一九六七年一月七日,他俩专门接见新华社工作人员,把陶铸的问题捅向社会。二月六日的会议上,毛主席严厉批评说:“你们文化革命小组,毫无政治经验,毫无军事经验,老干部统统打倒,你们掌权掌的起来?江青眼睛向天,天下没有几个她看得起的人。犯了错误就打倒,就要打到自己头上来了。你们就不犯错误?陶铸是犯了错误,可是一下子就捅出去,不同我打招呼。上脱离,下没有同干部群众商量。”二月十日,毛主席继续召集有林彪、周恩来、康生、李富春、叶剑英参加的会议,当面批评陈伯达、江青。他气愤地对陈伯达说:“你是一个常委打倒一个常委。”又对江青说:“你眼里只有一个人,眼高手低,志大才疏。”并决定:立即举行会议,批评陈伯达、江青。

左派犯错误,右派利用,历来如此,这次也不例外。一些对文化大革命不理解的老帅们和老领导,就利用这次批评左派,向着毛主席,向着文革大闹起来。

一九六七年二月十四日下午三时,周恩来在怀仁堂召开会议。一些老帅和老领导借机大发心中的牢骚和不满。叶剑英对未经政治局讨论就让上海市改为上海人民公社以及各单位没有党的领导表示出极大的愤懑。并说什么“各地都有一帮右派在造反,他们哄抢档案,查抄文件,冲击军事机关……”叫嚷:“对那些敢于向无产阶级专政宣战的反革命分子,要坚决镇压,决不手软……”徐向前说:“这回我们派上用场了。大夺权以来,全国到处混乱得一塌糊涂。连大军区、小军区都受到冲击,军队不表态的确不行了。”谭震林对陈毅说:“陈老总,咱们可不能只是发发牢骚就算了,底下的群众斗不赢他们,我们上头得斗垮他们呀,不然,他们更是无法无天了。”李先念说:“中央文革小组不伦不类,本身就是一个奇奇怪怪的组织……再不给他们点措施,中国就彻底乱套了。”陈毅说:“怀仁堂这边我和谭老板(谭震林)打冲锋,军委会议那边有叶帅、徐帅。你们放心吧,大家等待的就是时机。现在冲锋号已经吹响了,我们不上阵冲杀还行吗?”

在二月十六日碰头会议上,这些人同中央文革小组发生了激烈的争论。谭震林拍着桌子大骂:“你们(指中央文革小组)的目的,就是专整老干部。你们把老干部一个一个打光……革命到头来落得家破人亡,妻离子散。”“你们所谓的群众是什么?就是蒯大富之流。蒯大富之流是什么东西?就是一个反革命!”“我跟毛主席跟了四十年,到四十一年我不跟了,我一不该参加革命,二不该加入共产党,三不该跟着毛主席干革命。让你们这些人跟着他干去吧!我不干了!砍脑袋,坐监牢,开除党籍,我也要斗争到底!”“我就是和你们斗,我还有三千御林军。”陈毅气得脸色发青,哆嗦着嘴唇说:“这一次(指文革)是党的历史上斗争最残酷的一次,超过历史上任何一次。我看不仅这次是错误的,而且延安整风也是错误的。”“林彪同志国庆讲话(指一九六六年经过毛主席亲自批阅的国庆讲话)也有问题,什么叫‘反对革命的路线’,这就把矛头指向了广大干部……”李先念说:“现在可以说全国范围内都在大搞逼供信。不但老干部们挨整,连他们的子女也挨整。把‘红卫兵联合行动委员会’打成反革命就是证明。‘联动’怎么是反动组织……”“全国处处搞路线斗争,把许多老干部都伤害了。”叶剑英说:“我搞了这么多年革命,从来没见过什么大串连……我根本不赞成!”

徐向前说:“中央文革关于军队院校文化大革命的五条指示也不正确。”“连国民党都没达到的目的,他们(指中央文革小组代表的革命造反派)达到了。”

这就形成一股对抗文化大革命的逆流,当时被称为“二月逆流”。
\mnitem{2}余秋里是二月逆流中的两帮凶之一,二月逆流中的“三老四帅两帮凶”,三老指李富春、谭震林、李先念,四帅指陈毅、徐向前、叶剑英、聂荣臻,两个帮凶是指余秋里和谷牧。
\mnitem{3}一九六七年三月十六日,中共中央发布了《关于薄一波、刘澜涛、安子文、杨献珍等人自首叛变问题的初步调查》,认定一九三六年在北平草岚子监狱出狱的薄一波等人为“六十一人叛徒集团案”。调查报告指出:

“薄一波等人自首叛变出狱,是刘少奇策划和决定,张闻天同意,背着毛主席干的。这批人的出狱,决不是象他们自己事后向中央所说的那样,只是履行了一个什么‘简单手续’。他们是签字画押,公开发表《反共启事》,举行‘自新仪式’后才出来的。”

“当时在狱中的人,对刘少奇、张闻天这个叛卖的决定,有两种截然相反的态度。刘格平(静火注:刘格平一直到一九四四年才出狱,文革时任山西省革命委员会主任)、张良云(静火注:张良云后来出狱后在抗日战争中牺牲了)同志坚决反对,拒绝执行,表现了共产党人崇高的革命气节。”

《反共启事》内容是:

“余等幸蒙政府宽大为怀不咎既往,准予反省自新,现已诚心悔悟,愿在政府领导之下坚决反共,做一忠实国民。以后决不参加共党组织及其它任何反动行为,并望有为青年俟后莫再受其煽惑,特此登报声明”。

一九六七年二月十八日,张闻天写给南开大学抓叛徒战斗队的交待材料:

“2、事实真相:……刘少奇初去北方局(一九三六年春)不久,就给我一封关于如何解决白区工作干部问题的信……。他说,现在北平监狱中有一批干部,过去表现好,据监狱内部传出消息,管理监狱的人自知日子不长,准备逃走,也想及早处理这批犯人,所以只要履行一个反共不发表的简单手续,犯人即可出狱……。此外他还附带着寄来狱中干部提出有三个条件的请求书,要我签字,好使狱中干部相信,中央是同意那样办的。我当时很相信刘少奇的意见,并也在请求书上签了字,退回去了……。我现在记得,我当时没有把此事报告毛主席,或提到中央特别讨论。

3、责任问题:根据以上的立场和具体情况,即可看出,关于此案的直接主谋者,组织者和执行者是刘少奇。他利用他的资产阶级的招降纳叛的干部政策,以实现他篡党,篡军,篡政的政治野心。但是我在这方面,也负有严重的政治责任……。没有请示我们的伟大领袖毛主席,没有提到中央会议上正式讨论,而轻率地以我个人的名义,同意了刘少奇的建议,并在请求书上签了字。这样,我不但违反了党章,党纪的规定,损害了党的组织的纯洁性,玷污了我们共产党人永不变节,忠贞不屈的光荣传统,而且也给刘少奇招降纳叛的干部政策打开了方便之门。我在这个问题上犯了严重的反党罪行,而且也成了刘少奇的帮凶。”

曾在白区工作过的中共中央政治局候补委员李雪峰一九八九年受访时曾表示:“比如六十一人的《反共启事》,和叛徒出卖了组织和同志,带领敌人去抓我们的同志这样的叛徒还是有区别的吧。我在战争年代里处理叛变革命的人,是要区别两种情况的:一是经不起敌人的严刑拷打而写了自首书,但是没有出卖组织,没有杀人;二是经不起敌人的利诱和拷打,出卖泄露了党的机密后,又用同志的鲜血来换得了自己的生命或者也被敌人处死,这样的人是坚决要镇压的。总之,要有区别,没有区别就没有政策,就没有党的规矩。关于薄一波和刘少奇他们被捕以后的情况,我没有发言权。所以康生和戚本禹他们或他们派来的人都找过我,我都是这么回答的,他们也没有办法说什么。但是,让薄一波他们写这种《反共启事》,我保留我自己的看法。”

静火有言:通过这样一种特殊手续把大批干部尽快救出去似乎无可厚非,但另一方面,多少同志宁肯死在监狱也绝不背叛自己的信仰发表什么《反共启事》,如江姐等,即使是在同一个监狱,刘格平张良云等同志也拒绝发表《反共启事》,又坐了八年牢,从这点上说,指示一部分同志发表《反共启事》对不起牺牲的同志,也对不起自己的信仰。
\end{maonote}
