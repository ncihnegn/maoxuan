
\title{必须强调团结和进步}
\date{一九四〇年二月七日}
\thanks{这是毛泽东为延安《新中华报》改版一周年纪念写的文章。}
\maketitle


抗战、团结、进步,这是共产党在去年“七七”纪念时提出的三大方针。这是三位一体的方针,三者不可缺一。如果单单强调抗战而不强调团结和进步,那末,所谓“抗战”是靠不住的,是不能持久的。缺乏团结和进步纲领的抗战,终久会有一天要改为投降,或者归于失败。我们共产党认为一定要三者合一。为了抗战就要反对投降,反对汪精卫的卖国协定\mnote{1},反对汪精卫的伪政府,反对一切暗藏在抗日阵线中的汉奸和投降派。为了团结,就要反对分裂运动,反对内部磨擦,反对从抗日阵线后面进攻八路军、新四军和一切进步势力,反对破坏敌后的抗日根据地,反对破坏八路军的后方陕甘宁边区,反对不承认共产党的合法地位,反对雪片一样的“限制异党活动”的文件。为了进步,就要反对倒退,反对把三民主义和《抗战建国纲领》\mnote{2}束之高阁,反对不实行《总理遗嘱》上“唤起民众”的指示,反对把进步青年送进集中营,反对把抗战初期仅有的一点言论出版自由取消干净,反对把宪政运动变为少数人包办的官僚事业,反对在山西进攻新军、摧残牺盟和残杀进步人员\mnote{3},反对三民主义青年团在咸榆公路、陇海铁路一带拦路劫人\mnote{4},反对讨九个小老婆和发一万万元国难财的无耻勾当,反对贪官污吏的横行和土豪劣绅的猖獗。不这样做,没有团结和进步,所谓抗战只是空唤,抗日胜利是没有希望的。《新中华报》\mnote{5}第二年的政治方向是什么?就是强调团结和进步,以反对一切危害抗战的乌烟瘴气,以期抗日事业有进一步的胜利。


\begin{maonote}
\mnitem{1}见本卷\mxnote{克服投降危险,力争时局好转}{1}。
\mnitem{2}见本卷\mxnote{陕甘宁边区政府、第八路军后方留守处布告}{3}。
\mnitem{3}在山西进攻新军的事件,见本卷\mxnote{团结一切抗日力量,反对反共顽固派}{4}。“牺盟”即“山西牺牲救国同盟会”,一九三六年九月成立。它是中国共产党倡议创建、并始终受共产党领导的群众抗日团体,在山西的抗日斗争中曾起了重大的作用。一九三九年十二月,阎锡山发动“晋西事变”,后并在晋东南等地的国民党中央军配合下,公开摧残牺盟会,许多共产党员、牺盟会的干部和群众中的进步分子,遭到残酷的杀害。
\mnitem{4}从一九三九年起,国民党用三民主义青年团“招待所”的名义,派遣特务,配合军队,在咸阳榆林公路和陇海铁路上设立许多封锁站口,截留出入陕甘宁边区的进步青年和知识分子,把他们送往集中营监禁残杀,或者强迫他们充当特务。
\mnitem{5}见本卷\mxnote{和中央社、扫荡报、新民报三记者的谈话}{2}。
\end{maonote}
