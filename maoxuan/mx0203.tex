
\title{反对自由主义}
\date{一九三七年九月七日}
\maketitle


我们主张积极的思想斗争,因为它是达到党内和革命团体内的团结使之利于战斗的武器。每个共产党员和革命分子,应该拿起这个武器。

但是自由主义取消思想斗争,主张无原则的和平,结果是腐朽庸俗的作风发生,使党和革命团体的某些组织和某些个人在政治上腐化起来。

自由主义有各种表现。

因为是熟人、同乡、同学、知心朋友、亲爱者、老同事、老部下,明知不对,也不同他们作原则上的争论,任其下去,求得和平和亲热。或者轻描淡写地说一顿,不作彻底解决,保持一团和气。结果是有害于团体,也有害于个人。这是第一种。

不负责任的背后批评,不是积极地向组织建议。当面不说,背后乱说;开会不说,会后乱说。心目中没有集体生活的原则,只有自由放任。这是第二种。

事不关己,高高挂起;明知不对,少说为佳;明哲保身,但求无过。这是第三种。

命令不服从,个人意见第一。只要组织照顾,不要组织纪律。这是第四种。

不是为了团结,为了进步,为了把事情弄好,向不正确的意见斗争和争论,而是个人攻击,闹意气,泄私愤,图报复。这是第五种。

听了不正确的议论也不争辩,甚至听了反革命分子的话也不报告,泰然处之,行若无事。这是第六种。

见群众不宣传,不鼓动,不演说,不调查,不询问,不关心其痛痒,漠然置之,忘记了自己是一个共产党员,把一个共产党员混同于一个普通的老百姓。这是第七种。

见损害群众利益的行为不愤恨,不劝告,不制止,不解释,听之任之。这是第八种。

办事不认真,无一定计划,无一定方向,敷衍了事,得过且过,做一天和尚撞一天钟。这是第九种。

自以为对革命有功,摆老资格,大事做不来,小事又不做,工作随便,学习松懈。这是第十种。

自己错了,也已经懂得,又不想改正,自己对自己采取自由主义。这是第十一种。

还可以举出一些。主要的有这十一种。

所有这些,都是自由主义的表现。

革命的集体组织中的自由主义是十分有害的。它是一种腐蚀剂,使团结涣散,关系松懈,工作消极,意见分歧。它使革命队伍失掉严密的组织和纪律,政策不能贯彻到底,党的组织和党所领导的群众发生隔离。这是一种严重的恶劣倾向。

自由主义的来源,在于小资产阶级的自私自利性,以个人利益放在第一位,革命利益放在第二位,因此产生思想上、政治上、组织上的自由主义。

自由主义者以抽象的教条看待马克思主义的原则。他们赞成马克思主义,但是不准备实行之,或不准备完全实行之,不准备拿马克思主义代替自己的自由主义。这些人,马克思主义是有的,自由主义也是有的:说的是马克思主义,行的是自由主义;对人是马克思主义,对己是自由主义。两样货色齐备,各有各的用处。这是一部分人的思想方法。

自由主义是机会主义的一种表现,是和马克思主义根本冲突的。它是消极的东西,客观上起着援助敌人的作用,因此敌人是欢迎我们内部保存自由主义的。自由主义的性质如此,革命队伍中不应该保留它的地位。

我们要用马克思主义的积极精神,克服消极的自由主义。一个共产党员,应该是襟怀坦白,忠实,积极,以革命利益为第一生命,以个人利益服从革命利益;无论何时何地,坚持正确的原则,同一切不正确的思想和行为作不疲倦的斗争,用以巩固党的集体生活,巩固党和群众的联系;关心党和群众比关心个人为重,关心他人比关心自己为重。这样才算得一个共产党员。

一切忠诚、坦白、积极、正直的共产党员团结起来,反对一部分人的自由主义的倾向,使他们改变到正确的方面来。这是思想战线的任务之一。
