
\title{解放战争第二年的战略方针}
\date{一九四七年九月一日}
\thanks{这是毛泽东为中共中央起草的对党内的指示,当时他和中共中央住在陕北佳县朱官寨。这个指示规定,解放战争第二年的基本任务,是以主力打到国民党区域,由内线作战转入外线作战,也就是由战略防御阶段转入战略进攻阶段。人民解放军按照毛泽东所规定的战略计划,从一九四七年七月至九月,转入了全国规模的进攻。晋冀鲁豫野战军于六月三十日在鲁西南地区强渡黄河,八月上旬越过陇海线,挺进大别山。晋冀鲁豫野战军的太岳兵团,八月下旬由晋南强渡黄河,挺进豫西地区。华东野战军在打破敌人的重点进攻以后,其主力部队于八月初挺进鲁西南,九月下旬进入豫皖苏地区。华东野战军的内线兵团(一九四八年三月改称山东兵团),从十月初起向胶东地区之敌发起攻势作战。西北野战军八月下旬转入反攻。晋察冀野战军九月初对平汉线北段之敌发起攻势作战。东北野战军紧接着全东北范围的夏季攻势之后,从九月起,在长春、吉林、四平地区和北宁线锦西至义县地区发起大规模的秋季攻势。所有这些战场上的攻势,组成了人民解放军全面进攻的总形势。人民解放军的大举进攻,使解放战争达到了一个转折点,标志着战争形势的根本改变,参看本卷\mxart{目前形势和我们的任务}一文。}
\maketitle


(一)第一年作战(去年七月至今年六月),歼灭敌正规军九十七个半旅,七十八万人,伪军、保安队等杂部三十四万人,共计一百十二万人。这是一个伟大的胜利。这一胜利,给了敌人以严重打击,在整个敌人营垒中引起了极端深刻的失败情绪,兴奋了全国人民,奠定了我军歼灭全部敌军、争取最后胜利的基础。

(二)第一年作战,敌人以二百四十八个正规旅中的二百十八个旅一百六十多万人,近百万的特种兵(海军、空军、炮兵、工兵、装甲兵),以及伪军、交通警察部队、保安部队等,向我解放区大举进攻。我军正确地采取战略上的内线作战方针,不惜付出三十余万人的伤亡,大块土地的被敌占领,使自己随时随地立于主动地位,因而争取了歼敌一百十二万人,分散了敌军,锻炼和壮大了我军,并且在东北、热河、冀东、晋南、豫北举行了战略性的反攻,收复和新解放了广大的土地\mnote{1}。

(三)我军第二年作战的基本任务是:举行全国性的反攻,即以主力打到外线去,将战争引向国民党区域,在外线大量歼敌,彻底破坏国民党将战争继续引向解放区、进一步破坏和消耗解放区的人力物力、使我不能持久的反革命战略方针。我军第二年作战的部分任务是:以一部分主力和广大地方部队继续在内线作战,歼灭内线敌人,收复失地。

(四)我军执行外线作战、将战争引向国民党区域的方针,当然要遇到许多困难。因为到国民党区域创立新根据地需要时间,需要在多次往返机动的作战中大量歼灭敌人、发动群众、分配土地、建立政权、建立人民武装之后,方能创立巩固的根据地。在这以前,困难将是不少的。但是,这种困难能够克服和必须克服。因为敌人将被迫更加分散,有广大地区作为我军机动作战的战场,可以求得运动战;那里的广大民众是痛恨国民党拥护我军的;虽然部分敌军仍然有较强的战斗力,但一般地敌军士气比一年前低落得多,其战斗力比一年前削弱得多了。

(五)到国民党区域作战争取胜利的关键:第一是在善于捕捉战机,勇敢坚决,多打胜仗;第二是在坚决执行争取群众的政策,使广大群众获得利益,站在我军方面。只要这两点做到了,我们就胜利了。

(六)敌军分布,到今年八月底止,连被歼灭和受歼灭性打击者都算在内,南线一百五十七个旅,北线七十个旅,国民党后方二十一个旅,全国总数仍是二百四十八个旅,实际人数约一百五十万人;特种部队、伪军、交通警察、保安部队等约一百二十万人;敌后方军事机关非战斗人员约一百万人。敌全军共约三百七十万人。南线各军中,顾祝同系统一百十七个旅,程潜系统和其它七个旅,胡宗南系统三十三个旅。顾军一百十七个旅中,被我歼灭和受歼灭性打击者有六十三个旅。其中一部尚未补充;一部虽已补充,但人数很少,战斗力很弱;另一部虽有较多人员武器补充,战斗力也恢复到某种程度,但仍然远不如前。尚未被歼和尚未受歼灭性打击者只有五十四个旅。全部顾军,使用于守备和仅能作地方性机动之用者占了八十二到八十五个旅,能用于战略性机动者只有三十二到三十五个旅。程潜系统和其它的七个旅大体均只能任守备,其中一个旅曾受歼灭性打击。胡宗南系统(包括兰州以东,宁夏榆林以南,临汾洛阳以西)之三十三个旅中,被歼灭和受歼灭性打击者有十二个旅,能用于战略性机动者只有七个旅,其余均任守备。北线敌军共有七十个旅。其中,东北系统二十六个旅,内有十六个旅被歼灭和受歼灭性打击;孙连仲系统十九个旅,内有八个旅被歼灭和受歼灭性打击;傅作义十个旅,内有二个旅受歼灭性打击;阎锡山十五个旅,内有九个旅被歼灭和受歼灭性打击。这些敌军现在大体均取守势,能机动作战的兵力只有一小部分。国民党后方任守备的兵力仅有二十一个旅。其中,新疆和甘西八个旅,川、康七个旅,云南两个旅,广东两个旅(即被歼灭的第六十九师),台湾两个旅,湖南、广西、贵州、福建、浙江、江西六省全无正规军。国民党在美国援助下,今年计划征兵一百万补充前线并训练若干新旅和若干补充团。但是,只要我军能如第一年作战平均每月歼敌八个旅,在第二年再歼敌九十六至一百个旅(七、八两月已歼敌十六个半旅),则敌军将进一步大受削弱,其战略性机动兵力将减少至极度,势将被迫在全国一切地方处于防御地位,到处受我攻击。国民党虽有征兵百万训练新旅和补充团之计划,也将无济于事。其征兵纯用捕捉和购买方法,必难达到百万,而且逃亡甚多。我军执行外线作战方针,又可缩小其人力资源和物质资源。

(七)我军作战方针,仍如过去所确立者,先打分散孤立之敌(包括一次打几个旅的大规模歼灭性战役在内,例如今年二月莱芜战役\mnote{2},七月鲁西南战役\mnote{3}),后打集中强大之敌。先取中、小城市和广大乡村,后取大城市。以歼灭敌人有生力量为主要目标,不以保守和夺取地方为主要目标;保守或夺取地方是歼敌有生力量的结果,往往须反复多次才能最后地保守或夺取之。每战集中绝对优势兵力,四面包围敌人,力求全歼,不使漏网。在特殊情况下,则采用给敌以歼灭性打击之方法,即集中全力打敌正面及其一翼或两翼,以求达到歼灭其一部、击溃其另一部之目的,以便我军能够迅速转移兵力,歼击他部敌军。一方面,必须注意不打无准备之仗,不打无把握之仗,每战都应力求有准备,力求在敌我条件对比上有胜利之把握;另方面,必须发扬勇敢战斗、不惜牺牲、不怕疲劳和连续作战(即短期内接连打几仗)的优良作风。必须力求调动敌人打运动战,但同时必须极大地注重学习阵地攻击战术,加强炮兵、工兵建设,以便广泛地夺取敌人据点和城市。一切守备薄弱之据点和城市则坚决攻取之,一切有中等程度的守备而又环境许可之据点和城市则相机攻取之,一切守备强固之据点和城市则暂时弃置之。以俘获敌人的全部武器和大部兵员(十分之八九的士兵和少数下级官佐)补充自己。主要向敌军和国民党区域求补充,只有一部分向老解放区求补充,特别是南线各军应当如此。在一切新老解放区必须坚决实行土地改革(这是支持长期战争取得全国胜利的最基本条件),发展生产,厉行节约,加强军事工业的建设,一切为了前线的胜利。只有这样做,才能支持长期战争,取得全国胜利。果然这样做了,就一定可以支持长期战争,取得全国胜利。

(八)以上是一年战争的总结和今后战争的方针。望各地领导同志传达给军队团级以上、地方地委和专署以上的各级干部,使大家明白自己的任务而坚决地毫不动摇地执行之。


\begin{maonote}
\mnitem{1}本文所说我军在东北、热河(现分属河北、辽宁两省和内蒙古自治区)、冀东的战略反攻,是指东北民主联军一九四七年的夏季攻势。自五月十三日开始,东北民主联军在东北和热河、冀东各个战场上同时展开进攻,到七月一日止,歼灭国民党军八万二千余人,收复和一度攻克城镇四十二座,打通了南满、北满的联系,迫使敌军收缩于长春至大石桥、沈阳至山海关铁路沿线,转为防御,从而改变了东北整个局势。本文所说我军在晋南、豫北的战略性反攻,是指晋冀鲁豫野战军一九四七年三、四、五月间在豫北和晋南同蒲路两侧所展开的攻势。豫北攻势于三月二十三日发起,在接连攻克延津、阳武(今并入原阳)、濮阳、封丘后向北扩张战果,四月三日挥师北上,至五月二十八日止又攻克淇县、浚县、滑县、汤阴等城,歼敌四万余人。晋南攻势于四月四日发起,至五月十一日,连克曲沃、新绛、永济等二十二座县城和黄河渡口的禹门口、风陵渡要点,歼敌一万四千余人。
\mnitem{2}莱芜战役,是华东野战军在山东莱芜(位于济南东南)地区所进行的运动战。一九四七年一月底,国民党军队分南北两线进攻山东解放区。南线国民党军以八个整编师,分三路沿沂河、沭河北犯临沂,北线国民党军李仙洲集团三个军由明水(今章丘)、淄川、博山等地南下莱芜、新泰策应,同时从冀南、豫北抽调一个军及三个整编师集结在鲁西南地区,阻止华东野战军西撤和晋冀鲁豫野战军东援,企图同华东野战军主力在临沂地区决战。华东野战军以一部阻击南线之敌,佯作决战模样,主力则隐蔽兼程北上莱芜歼击李仙洲集团。战斗自二月二十日开始至二十三日下午结束,歼灭该敌五万六千余人,生俘国民党第二绥靖区副司令官李仙洲。
\mnitem{3}鲁西南战役,是晋冀鲁豫野战军在山东省西南部菏泽、郓城、巨野、定陶、金乡、曹县地区所进行的一次战役。这次战役于一九四七年六月三十日夜发起,至七月二十八日结束,先后歼灭国民党军四个整编师师部和九个半旅,共六万余人。
\end{maonote}
