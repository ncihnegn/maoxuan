
\title{一九四八年的土地改革工作和整党工作}
\date{一九四八年五月二十五日}
\thanks{这是毛泽东为中共中央起草的对党内的指示。}
\maketitle


\section*{一}

必须注意季节。必须利用今年整个秋季和冬季,即自今年九月至明年三月,共七个月时间,在各中央局和分局所划定的地区内,依次完成下列各项工作:(甲)乡村情况调查。(乙)按照正确政策实行初步整党。上级派到乡村的工作团或工作组,必须首先团结当地党的支部组织内的一切积极分子和较好分子,共同领导当地的土地改革工作。(丙)组织或改组或充实贫农团和农会,发动土地改革斗争。(丁)按照正确标准,划分阶级成分。(戊)按照正确政策,实行分配封建土地和封建财产。实行分配的最后结果,必须使一切主要阶层都感觉公道和合乎情理,地主阶级分子亦感觉生活有出路,有保障。(己)建立乡(村)、区、县三级人民代表会议,并选举三级政府委员会。(庚)发给土地证,确定地权。(辛)调整或改订农业税(公粮)负担的标准。这种标准,必须遵守公私兼顾的原则,这即是一方面利于支持战争,一方面使农民有恢复和发展生产的兴趣,利于改善农民的生活。(壬)按照正确政策,完成党的支部组织的整理工作。(癸)将工作方向由土地改革方面,转移到团结农村中一切劳动人民并组织地主富农的劳动力为共同恢复和发展农业生产而奋斗的方面去。开始组织在自愿和等价交换两项原则上的小规模的变工组织和其它合作团体;准备好种子、肥料和燃料;做好生产计划;发放必要的和可能的农业贷款(以贷给生产资料为主,必须有借有还,严格区别于救济性质的赈款);在可能的地点,做好兴修水利的计划。以上是由土地改革到生产的全部工作过程,必须使一切直接从事土地改革工作的同志了解这样的工作过程,避免工作的片面性,并不失时机地于秋冬两季全部完成上述工作。

\section*{二}

为达上述目的,今年六月至八月的三个月内,必须完成下列工作:(甲)划定土地改革工作范围。这种范围,必须是在下列三项条件下划定之:第一,当地一切敌人武装力量已经全部消灭,环境已经安定,而非动荡不定的游击区域。第二,当地基本群众(雇农、贫农、中农)的绝对大多数已经有了分配土地的要求,而不只是少数人有此要求。第三,党的工作干部在数量上和质量上,确能掌握当地的土地改革工作,而非听任群众的自发活动。如果某一地区,在上述三个条件中,有任何一个条件不具备,即不应当将该地区列入一九四八年进行土地改革的范围。例如,在华北、华东、东北、西北各解放区的接敌区域和中原局所属江淮河汉区域的绝大部分地区,因为尚不具备第一个条件,即不应当列入今年的土地改革计划内。明年是否列入,还要看情况才能决定。在这类地区,应当充分利用抗日时期的经验,实行减租减息和酌量调剂种子食粮的社会政策和合理负担的财政政策,以便联合或中立一切可能联合或中立的社会力量,帮助人民解放军消灭一切国民党武装力量和打击政治上最反动的恶霸分子。在这类地区,既不要分土地,也不要分浮财,因为这些都是在新区和接敌区的条件之下,不利于联合或中立一切可能联合或中立的社会力量、完成消灭国民党反动力量这一基本任务的。(乙)开好干部会议。在为着土地改革和整党工作召集的干部会议中,必须充分讲明关于这两项工作的全部正确政策,将许可做的事和不许可做的事,分清界限。必须将中央颁布的各项重要文件,责成一切从事土地改革工作和整党工作的干部,认真学习,完全了解,并责成他们全部遵守,不许擅自修改。如有不适合当地情况的部分,可以和应当提出修改的意见,但必须取得中央同意,方能实行修改。今年的各级干部会议,必须由各地高级领导机关,在开会之前,作充分而恰当的准备,这即是事前由少数人商量(由一个人负主责),提出问题和分析问题,写好成文的纲要,精心斟酌这个纲要的内容和文字(注意简明扼要,反对不着边际的长篇大论),然后向干部会议作报告,开展讨论,吸收讨论中的意见,加以补充和修改,作为定论;并将此项文件通知全党和尽可能地在报纸上公开发表。必须反对经验主义的方法,这即是事前毫无准备,不提出问题,不分析问题,不向干部会议作精心准备的、内容文字都有斟酌的报告,而听凭到会人员无目的地杂乱无章地议论,致使会议时间延长,得不到明确而周密的结论。各中央局、中央分局、区党委、省委和地委的领导工作中,如果存在着这种有害的经验主义方法,必须注意克服。讨论政策的会议,人数不可太多,只要事先有良好准备,会议的时间亦可缩短。按情况,大约以十几个人,或二三十人,或四五十人,开会一星期左右为适宜。传达政策的会议,人数可以多些,时间亦不可过长。只有整党性质的高级和中级的干部会议,人数可以多些,时间亦可以长些。(丙)九月上半月,至迟九月下半月,全部直接从事土地改革工作的干部必须到达乡村,并开始工作,否则就不能利用秋冬两季的全部时间,完成全部土地改革、整党建政和准备春耕的工作。

\section*{三}

在干部会议中和在工作中,必须教育干部善于分析具体情况,从不同地区、不同历史条件的具体情况出发,决定当地当时的工作任务和工作方法。必须区别城市和农村的不同,必须区别老区、半老区、接敌区和新区的不同,否则就要犯错误。

\section*{四}

凡属封建制度已经根本消灭,贫雇农已经得到大体上相当于平均数的土地,他们同中农所有的土地虽有差别(这种差别是许可的),但是相差不多者,即应认为土地问题已经解决,不要再提土地改革问题。在这类地区的中心任务,是恢复和发展生产,完成整党建政工作和支持前线的工作。在这类地区的部分乡村中,如果尚有土地须待分配或调剂,阶级成分须待改订,土地证须待发给者,自然应当按照实际情形完成这些工作。

\section*{五}

在一切解放区,不论是已经完成土地改革的地区,或者尚未完成土地改革的地区,都必须在今年秋季指导农民耕种麦地,并进行一部分土地的秋耕。在冬季,要号召农民积肥。所有这些,都对一九四九年解放区农业的生产和收成有极大重要性,必须用行政力量,配合群众工作,加以实现。

\section*{六}

必须坚决地克服许多地方存在着的某些无纪律状态或无政府状态,即擅自修改中央的或上级党委的政策和策略,执行他们自以为是的违背统一意志和统一纪律的极端有害的政策和策略;在工作繁忙的借口之下,采取事前不请示事后不报告的错误态度,将自己管理的地方,看成好像一个独立国。这种状态,给予革命利益的损害,极为巨大。各级党委必须对这一点进行反复讨论,认真克服这种无纪律状态或无政府状态,将一切可能和必须集中的权力,集中于中央和中央代表机关\mnote{1}。

\section*{七}

中央、中央局(分局)、区党委(省委)、地委、县委、区委、直到支部,必须充分利用无线电、有线电、电话、邮递、专人送信等项通讯方法,小型会议(例如四五个人的),区域会议(例如几个县的),和个别谈话等项会谈方法,小型巡视团(例如三至五个人的)和个别有威信的委员的巡视方法,同时充分利用通讯社和报纸,密切地互相联系起来,以便掌握运动的动态,随时互通情报,交流经验,及时纠正错误,发扬成绩。不要等候几个月,或半年,甚至更长时间,下面才向上面作总结性的报告,上面才向下面作一般性的指示。这种报告和指示,往往过时,失去作用,或者减少了作用。犯错误的已经犯过,来不及纠正,损失太大。全党迫切需要的,是不失时机的生动的具体的报告和指示。

\section*{八}

必须将城市工作和农村工作,将工业生产任务和农业生产任务,放在各中央局、分局、区党委、省委、地委和市委的领导工作的适当位置。即是说,不要因为领导土地改革工作和农业生产工作,而忽视或放松对于城市工作和工业生产工作的领导。我们现在已经有了许多大中小城市和广大的工矿交通企业,如果各有关领导机关忽视或放松这一方面的工作,我们就要犯错误。


\begin{maonote}
\mnitem{1}这里所说的中央代表机关,是指中央局和中央分局。
\end{maonote}
