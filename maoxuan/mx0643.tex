
\title{中国在联合国只能有一个代表}
\date{一九六一年六月十三日}
\thanks{这是毛泽东同志同印度尼西亚总统阿哈默德·苏加诺谈话的一部分。}
\maketitle


\mxsay{毛泽东主席(以下简称毛):}上一次总统来的时候,我曾经说过,因为联合国里有蒋介石的代表,所以我们不进联合国,这是同台湾问题有关系的。只要蒋介石的代表还在联合国,我们就不进联合国。我们已经等了十一年了,再等十一年或者更久也没有关系。我们不忙于进联合国。你们没有联合国问题,只有西伊里安\mnote{1}问题,这是与我们不同的。

\mxsay{阿哈默德·苏加诺总统(以下简称苏):}关于中国进入联合国的问题,目前外界有两种主张:一种主张是中国大陆同台湾成为整体,作为一个国家进入联合国;另一种主张认为中国可以先进入联合国,然后在联合国里同朋友们一道进行斗争,使得在联合国中只有中国,把蒋介石的代表驱逐出联合国,台湾归还中国。不久前我同陈毅\mnote{2}元帅谈话时,曾把这两种主张转告给他,并说明这不是印尼的观点。不过陈毅元帅已经明确表明只接受前者,而不接受后者;只愿意一步走,而不愿意分两步走。

\mxsay{毛:}只能一步走。

\mxsay{苏:}我愿意很好协助,为实现一步走而奋斗。

\mxsay{毛:}如果台湾归还中国,中国就可以进联合国。如果台湾不作为一个国家,没有中央政府,它归还中国,那末台湾的社会制度问题也可以留待以后谈。我们容许台湾保持原来的社会制度,等台湾人民自己来解决这个问题。

\mxsay{苏:}这是否就像苏联同乌克兰在联合国的情形一样呢?

\mxsay{毛:}不一样。

\mxsay{苏:}我不是说社会制度方面,而是指能不能像乌克兰同苏联那样在联合国有两个代表。

\mxsay{毛:}不行,中国在联合国只能有一个代表。乌克兰,还有白俄罗斯在联合国有代表,这是有历史原因的。第二次世界大战以后,乌克兰同白俄罗斯都参加联合国,苏联也在联合国里,因而苏联在联合国拥有三票。苏联当时碰到许多困难,不能不这样做。但是这样并没有两个苏联的问题。

\begin{maonote}
\mnitem{1}西伊里安指印度尼西亚新几内亚岛西半部及其近海岛屿,今伊里安查亚省。一九四九年印度尼西亚独立时,荷兰政府在美国的支持下继续霸占这一地区。印尼政府曾多次希望通过谈判解决西伊里安问题,却屡遭阻挠和破坏。为维护国家的独立和主权,印尼人民展开声势浩大的收复西伊里安运动和反殖民主义的武装斗争,迫使荷兰政府同意谈判。一九六三年五月一日荷兰政府将西伊里安主权交还印度尼西亚。
\mnitem{2}陈毅(一九〇六——一九七二),四川乐至人。时任国务院副总理兼外交部长。
\end{maonote}
