
\title{一九五七年夏季的形势}
\date{一九五七年七月}
\thanks{这是毛泽东同志一九五七年七月在青岛召开的省市委书记会议期间写的一篇文章,在这个会议上印发过,同年八月,又印发给党内领导干部阅读。}
\maketitle


在我国社会主义革命时期,反共反人民反社会主义的资产阶级右派和人民的矛盾是敌我矛盾,是对抗性的不可调和的你死我活的矛盾。

向工人阶级和共产党举行猖狂进攻的资产阶级右派是反动派、反革命派。不这样叫,而叫右派,一是为了便于争取中间派;二是为了便于分化右派,使一部分右派分子有可能转变过来。

最后不能转变的那一部分资产阶级右派分子是死硬派,只要他们不当特务,不再进行破坏活动,也给他们一点事做,也不剥夺他们的公民权。这是鉴于许多历史事件采取了极端政策的后果,并不良好。我们应当看得远一些,在几十年后看这个事件,将会看到我们这样对待资产阶级右派分子,对于无产阶级革命事业,会有深远影响和巨大利益的。

我们的目标,是想造成一个又有集中又有民主,又有纪律又有自由,又有统一意志、又有个人心清舒畅、生动活泼,那样一种政治局面,以利于社会主义革命和社会主义建设,较易于克服困难,较快地建设我国的现代工业和现代农业,党和国家较为巩固,较为能够经受风险。总题目是正确地处理人民内部的矛盾和正确地处理敌我矛盾。方法是实事求是,群众路线。派生的方法是党内党外在一起开一些有关大政方针的会议,公开整风,党和政府的许多错误缺点登报批评。民主党派,教育界,新闻界,科技界,文艺界,卫生界,工商界,工人阶级、农民阶级各阶层、手工业工人和其它城乡劳动者,都应当进行整风和社会主义教育,分期分批逐步推行。其中,资产阶级和资产阶级知识分子是使他们接受社会主义改造的问题;小资产阶级(农民和城乡独立劳动者),特别是富裕中农,也是使他们接受社会主义改造的问题;工人阶级和共产党的基本队伍,则是整顿作风的问题。这是性质不同的两个社会范畴的问题。既然有这样的不同,为什么通用整风这个口号呢?这是因为整风的口号较易为多数人所接受。我们向人们说:共产党和工人阶级尚且进行整风,难道你们不应当整风吗?这就十分主动了。整风的方法是批评和自我批评,摆事实,讲道理。整风的目的是把斗争方向引导到端正政治方向,提高思想水平,改正工作缺点,团结广大群众,孤立和分化资产阶级右派和一切反对社会主义的分子。这里所说的资产阶级右派,包括混入共产党内和青年团内的一些同党外团外右派分子政治面貌完全相同的人,他们背叛无产阶级革命事业,向党猖狂进攻,因此必须充分揭露,并把他们开除出去,借以纯洁党团组织。

必须坚定地信任群众的多数,首先是工农基本群众的多数,这是我们的基本出发点。即就商、学两界而论,在右派猖狂进攻时期,多数人可以一时被蒙蔽,发生动摇。几个星期以后,反击右派展开,多数人也都清醒了,都过来了。所以,这两界的多数人,最后也是可以信任的,他们是可以接受改造的。对无产阶级力量估计过低,对资产阶级右派力量估计过高,不少同志曾经犯过这种错误。现在地县区乡及工厂干部中还有许多人是这样,应当好好说服他们,不要低估了自己方面的力量,不要夸大敌人方面的力量。农村中,地主、富农正在被改造;其中,一部分人还在捣乱,必须对他们提高警惕。富裕中农,多数愿意留在合作社,少数闹退社,想走资本主义道路。我们应分别对待。在农村中,必须注意阶级路线,必须使原来的贫雇农在领导机关占优势,同时注意联合中农。我赞成迅即由中央发一个指示,向全体农村人口进行一次大规模的社会主义教育,批判党内的右倾机会主义思想,批判某些干部的本位主义思想,批判富裕中农的资本主义思想和个人主义思想,打击地富的反革命行为。其中的主要锋芒是向着动摇的富裕中农,对他们的资本主义思想进行一次说理斗争。以后一年一次,进行坚定的说理斗争,配合区乡干部的整风,配合第三类社整社,使合作社逐步巩固起来。农村中也要先让农民“鸣放”,即提意见,发议论。然后择其善者而从之,其不善者批判之。这应当在上级派有工作组协助当地干部主持农村整风的条件下,逐步推行。和城市一样,在农村中,仍然有或者是社会主义或者是资本主义,这样两条道路的斗争。这个斗争,需要很长时间,才能取得彻底胜利。这是整个过渡时期的任务。农村中,勤俭持家应当和勤俭办社并提,爱国、爱社应当和爱家并提。为了解决勤俭持家问题,特别要依靠妇女团体去做工作。最近几年,三百五十亿斤征粮和五百亿斤购粮,必须坚决收到,不能短少。按照年成丰歉,可以有所调节。农村中,因为逐年增产,缺粮户逐年减少,因此销粮应当逐年减少。城市销粮过多之处,也应酌量减少。这样,国家的粮食库存才有可能逐年增加,以备可能发生的紧急之需。如果八百五十亿斤左右的粮食收不到手,则将牵动市场物价,牵动整个国民经济计划的顺利进行,并且无法应付紧急情况,这是很危险的。今年秋收以前,必须在农村中进行一次反对不顾国家利益和集体利益的个人主义和本位主义的斗争。

有反必肃。杀人要少,但是决不废除死刑,决不大赦。对刑满释放再犯罪者,再提再判。社会上流氓、阿飞、盗窃、凶杀、强奸犯、贪污犯、破坏公共秩序、严重违法乱纪等严重罪犯以及公众公认为坏人的人,必须惩办。现在政法部门有些工作人员,对于本来应当捕处的人,也放弃职守,不予捕处,这是不对的。轻罪重判不对,重罪轻判也不对,目前时期的危险是在后者。禁止赌博。认真贯彻取缔会道门。右派学生首领应予彻底批判,但一般宜留在原地管教,并当“教员”。以上各点,适用于过渡时期,都由省市委、自治区党委负责。在不违背中央政策法令的条件下,地方政法文教部门受命于省市委、自治区党委和省、市、自治区人民委员会,不得违反。

正确处理人民内部矛盾是一个总题目。大谈特谈,习以为常,也就见怪不怪了。把人民内部矛盾想通,说开,正确处理一批问题,收了效果,得了经验,再也不怕了。

再说一遍,所谓正确处理人民内部矛盾问题,就是我党从来经常说的走群众路线的问题。共产党员要善于同群众商量办事,任何时候也不要离开群众。党群关系好比鱼水关系。如果党群关系搞不好,社会主义制度就不可能建成;社会主义制度建成了,也不可能巩固。

军队多次整风,实行三大纪律八项注意,实行军事、政治、经济三大民主,战时班上建立互助组,实行官兵、军民打成一片,禁止打人骂人,禁止枪毙逃兵。因此士气高涨,战无不胜。手执武器的军队能够这样做,为什么工厂、农村、机关、学校不能够发扬民主,用说服的方法而不是用压服的方法去解决自己的问题(矛盾)呢?

帝国主义都不怕,为什么反而怕老百姓呢?怕老百姓,认为人民群众不讲道理,只能压服,不能说服,这样的人不是真正的共产主义者。

除了叛徒和严重违法乱纪分子这两种人,在整风中保护一切党团员,用大力用诚心帮助他们改正错误缺点,改善工作方法,提高工作能力,提高政治水平,提高思想水平。共产党员一定要有朝气,一定要有坚强的革命意志,一定要有不怕困难和用百折不挠的意志去克服任何困难的精神,一定要克服个人主义、本位主义、绝对平均主义和自由主义,否则就不是一个名副其实的共产党员。有一些丧失朝气、丧失革命意志和坚持错误的人,在累戒不改的情况下,党委应当予以正当处理,重者绳之以纪律。

省市委、自治区党委的第一书记(其它书记也是一样),在半年到一年内,要求亲身研究一个合作社,一个工厂,一个商店,一个学校,取得知识,取得发言权,以利指导全般工作。地县区的党委书记也应这样做。

这一次批判资产阶级右派的意义,不要估计小了。这是一个在政治战线上和思想战线上的伟大的社会主义革命。单有一九五六年在经济战线上(在生产资料所有制上)的社会主义革命,是不够的,并且是不巩固的。匈牙利事件就是证明。必须还有一个政治战线上和一个思想战线上的彻底的社会主义革命。共产党在民主党派、知识界和工商界的一部分人(右派)中当然不可能有领导权,因为他们是敌人;在多数人(中间派)中的领导权不巩固;有些文教单位还根本没有建立党的领导。必须建立对中间派的巩固的领导权,并且尽可能早日巩固起来。资产阶级和资产阶级知识分子对共产党不心服,他们中的右派分子决心要同我们较量一下。较量了,他们失败了,他们才懂得他们的大势已去,没有希望了。只有在这时,他们中的多数人(中间派及一部分右派)才会逐渐老实起来,把自己的资产阶级立场逐渐抛弃,站到无产阶级方面来,下决心依靠无产阶级吃饭。少数人至死不改,只好让他们把他们的反动观点带到棺材里去。但是我们应当提高警惕。要知道,他们一遇机会,又会要兴风作浪的。这个斗争,从现在起,可能还要延长十年至十五年之久。做得好,可能缩短时间。当然不是说,十年至十五年之后,阶级斗争就熄灭了。只要世界上还存在着帝国主义和资产阶级,我国的反革命分子和资产阶级右派分子的活动,不但总是带着阶级斗争的性质,并且总是同国际上的反动派互相呼应的。目前的斗争,在一段必要时间之后,应当由急风暴雨的形式转变为和风细雨的形式,以便从思想上搞得更深更透。第一个决定性的战斗,在过去几个月,主要是最近两个月内,我们已经胜利了。但是还需要几个月深入挖掘的时间,取得全胜,决不可以草率收兵。要知道,如果这一仗不打胜,社会主义是没有希望的。

大辩论,全民性的,解决了和正在解决着革命和建设工行是否正确(革命和建设的成绩是不是主要的),是否应走社会主义道路,要不要共产党领导,要不要无产阶级专政,要不要民主集中制,以及我国的外交政策是否正确等项重大问题。祁自然地要发生这样一次全民性的大辩论。苏联在二十年代曾经发生过(同托洛茨基等人辩论一国能否建成社会主义\mnote{1}),我国在五十年代的第七年发生了。我们如果不能在这次辩论中取得完全胜利,我们就不能继续前进。只要我们在辩论中胜利了,就将大大促进我国的社会主义改造与社会主义建设。这是一个伟大的带有世界意义的事件。

必须懂得,在我国建立一个现代化的工业基础和现代化的农业基础,从现在起,还要十年至十五年。只有经过十年至十五年的社会生产力的比较充分的发展,我们的社会主义的经济制度和政治制度,才算获得了自己的比较充分的物质基础(现在,这个物质基础还很不充分),我们的国家(上层建筑)才算充分巩固,社会主义社会才算从根本上建成了。现在还未建成,还差十年至十五年时间。为了建成社会主义,工人阶级必须有自己的技术干部的队伍,必须有自己的教授、教员、科学家、新闻记者、文学家、艺术家和马克思主义理论家的队伍。这是一个宏大的队伍,人少了是不成的。这个任务,应当在今后十年至十五年内基本上解决。十年至十五年以后的任务,则是进一步发展生产力,进一步扩大工人阶级知识分子的队伍,准备着逐步地由社会主义过渡到共产主义的必要条件,准备以八个至十个五年计划在经济上赶上并超过美国。共产党员、青年团员和全体人民,人人都要懂得这个任务,人人都要努力学习。有条件的,要努力学技术,学业务,学理论,造成工人阶级知识分子的新部队(这个新部队,包含从旧社会过来的真正经过改造站稳了工人阶级立场的一切知识分子)。这是历史向我们提出的伟大任务。在这个工人阶级知识分子宏大新部队没有造成以前,工人阶级的革命事业是不会充分巩固的。

中央、省市两级,在整风、批判右派和争取中间群众这三个任务方面,取得了经验,是一件大事。有了这个经验,事情就好办了。今后几个月内的任务是教会地县两级取得经验。从现在起,到今冬明春,是逐步教会区乡两级取得经验。城市是教会区级、工矿基层和街道居民委员会取得经验。这样一来,豁然开朗,群众路线对于许多人说来就不是一句假话了,人民内部矛盾就比较容易解决了。

省市委、自治区党委的第一书记和整个党委,必须把这个伟大斗争完全掌握起来。必须把民主党派(政治界),教育界,新闻界(包括一切报纸和刊物),科技界,文艺界,卫生界,工商界的政治改造工作和思想改造工作完全掌握在自己手中。各省、市、自治区要有自己的马克思主义理论家,自己的科学家和技术人才,自己的文学家、艺术家和文艺理论家,要有自己的出色的报纸和刊物的编辑和记者。第一书记(其它书记也是一样)要特别注意报纸和刊物,不要躲懒,每人至少要看五份报纸,五份刊物,以资比较,才好改进自己的报纸和刊物。

批判右派这件事,整个民主党派,知识界,工商界,震动极大。应当看到他们中的多数人(中间派)是倾向于接受社会主义道路和无产阶级领导的。这种倾向,各类人程度深浅不同。应当看到,现在他们对于真正接受社会主义道路和真正接受无产阶级领导这些基本点虽然还只是一种倾向,但是,只要有了这种倾向,他们就从资产阶级立场到工人阶级立场的长距离路程中开动了第一步。如果有一年整风时间(从今年五月到明年五月)就可以跨进一大步。这些人在过去,并没有参加社会主义革命的精神准备。这个革命,对于他们,是突然发生的。共产党员中,也有一些人是这样。批判右派和整风,对于这些人,对于广大人群,将是一个深刻的社会主义教育。

大字报,除商店的门市部、农村(区乡)、小学、军队的营和连队以外,都可以用。在我国条件下,这是一个有利于无产阶级而不利于资产阶级的斗争形式。怕大字报,是没有根据的。在高等学校,在中央、省市、地、县的机关和城市的大企业,大字报、座谈会和辩论会,是揭露和克服矛盾、推动人们进步的三种很好的形式。

在整风中,任何时候都不应当耽误生产和工作。各地整风,不应当在所属一切单位同时并举,而应当分期分批地推行。

不要怕惊涛骇浪,硬着头皮顶住。就一个单位来讲,大约两三个星期,洪峰就过去了,就可以转到反击右派的新阶段。在两三个星期内,各单位的领导者对于右派的猖狂进攻,硬着头皮,只听不驳,聚精会神,分析研究,聚集力量,准备反攻,团结左派,争取中间派,孤立右派,这是一套很好的马克思主义的策略。

大鸣大放阶段(边整边改),反击右派阶段(边整边改),着重整改阶段(继续鸣放),每人研究文件、批评反省、提高自己阶段,这是中央、省市、地、县四级整风的四个必经阶段。还有城乡基层整风。这样整一次,全党和全国人民的面目必将焕然一新。

八月,请省市委、自治区党委一级和地委一级的第一书记,抽出一段时间;摸一下农村中整社、生产、粮食等项问题,以备九月中央全会之用。四十条农业纲要,请你们逐条研究一下是否需要修改。


\begin{maonote}
\mnitem{1}参看《苏联共产党(布)历史简明教程》第九章第五节。
\end{maonote}
