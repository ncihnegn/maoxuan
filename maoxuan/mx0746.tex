
\title{在中共八届扩大的十二中全会上的讲话}
\date{一九六八年十月十三日、三十一日}
\thanks{这是毛泽东同志在八届十二中全会上的主要讲话。}
\maketitle

\date{一九六八年十月一十三日开幕式上的讲话}
\section*{(一)}

会议大概开七天到十天,想一想有些什么问题要提出来。形势问题:一个国内,一个国际。国内要总结上次全会到现在两年多的工作。两年文化大革命,无非是两种意见,一是不错,基本上是正确的;二是也有人说,不见得。十一中全会时我就说过,我在十月工作会议上,也打过招呼:不要认为会议通过了都能执行,在会上举手是一回事,真正闹意见是另外一回事。后来证明还是不理解,有的人出了问题。

有一位同志叫张鼎丞\mnote{1}也没有发现有叛徒、特务、反革命问题,为什么没有来开会?是什么原因?邓老\mnote{2}你是很熟悉的。(周恩来:大连会议他参加了。高岗、饶漱石、徐海东他们四个人,徐海东同高岗抱头痛哭,说中央亏待了他,没有他在陕北打胜仗,中央就站不住脚,这正是公布高岗当副主席的时候,他同高岗联盟有勾搭,值得审查。)

山东谭启龙\mnote{3},从小当红军,没有别的问题,就是工作错误。山东不谅解他,现在还不能解决问题。山东要揪他回去斗,我们不让。(周恩来:揪回去了,现在在济南。)山东的同志要作说服工作。

湖北张体学\mnote{4},是好同志犯了错误。回湖北检讨得好,群众就谅解了。群众都欢迎他。他还要检讨,群众说不要再检讨了。王任重\mnote{5}是内奸,国民党。陶铸也是历史上有问题。

湖南王延春\mnote{6},这个人不行了。邓小平还没有发现他历史上有什么问题,就是发现他在七军开小差那回事。(张云逸:他在红七军当政委时,情况紧张了,他藉口到中央去汇报工作,在崇义地方开了小差,叶季壮同志死前也揭发过此事。)

主要问题还是到北京后搞独立王国,他不服气,他说他不揽权,实际上他是刘少奇那个司令部里摇鹅毛扇子的。但是有时我还找他说几句话的,他在抗日战争、解放战争,他是打敌人的。又没有查出来他历史上投敌叛变自首这一类问题。这个人一个是错误不小,一个是自己写了个自传检讨\mnote{7},可以给大家看看。他要求不要开除党籍,最好还是找点工作。一说工作,许多同志摇头。很大的工作也很难作了,我说给点室内整理材料之类的工作还可以吧!

同志们经验很多。可以设想,无产阶级文化大革命究竟要搞,还是不要搞?成绩是主要的,还是缺点是主要的?文化大革命究竟能不能搞到底?大家议议。比如浙江问题,“红暴”到现在还没有解决好,也是一个工作问题。

福建问题很大。韩先楚\mnote{8}你那里不是天下大乱吗?(韩先楚:现在形势很好。)看到了。你们的报告很好。工作一做,还不是解决了嘛!问题是可以解决的,怎么不能解决呢?

要总结经验。过去南征北战,解放战争好打。秋风扫落叶,一扫三年半扫完了。那时候敌人是比较清楚的。现在搞文化大革命,困难多,仗不好打。文化革命总比过去快一点。过去打了二十二年,从一九二七年到一九四九年,文化革命只打了两年半。

问题就是有思想错误同敌我矛盾混合在一起,一下搞不清楚。只好一个省、一个省的解决。比如辽宁省三大派\mnote{9},打了八个月,天下大乱。不是解决了吗?还是能够清楚的。

自己要多负担些责任,使下面轻一点。主要是思想问题,人民内部矛盾。内部也夹一些敌人,是少数,是做好工作的问题。

肖克\mnote{10}还是打过仗的,国庆节上了天安门,这次没有去,他是什么问题?我也不清楚(周恩来:他有段历史不清楚。他在潮安县一个村庄打了一仗,向敌人交了枪,还成立了一个党。这次被造反派查出来了。)

哪个是李德生\mnote{11}?你们安徽的事件办得不错,你们整芜湖整的不错嘛!

(李德生:是主席批的“七·三”布告起了作用,是舆论造得好。)

就是要造舆论。好像一个发明,搞了几十年,不懂得造舆论。搞了几十年,就是造舆论嘛。不然哪里能搞起八路军、新四军?没有群众,哪有无产阶级专政?没有群众,没有军队,谁收你这个党?

十一中全会讲,我们要文斗,不要武斗,结果他们就要武斗,不要文斗。一个学校两派斗,各有武器,一个工厂两派斗,两个中心,一个部也是两派斗。

基本群众还是工农兵,兵也是工农。然后才是知识分子。对知识分子的大多数要争取、团结、教育。知识分子的缺点是容易动摇,主要是没有和工农兵结合。

不能一讲就是臭知识分子,但是臭一点也可以。知识分子不可不要,也不可把尾巴翘到天上去了。

大学两年不办。叫他们统统毕业,下去同工农兵去混。两年不办,天塌不下来。以后重新搞,从农村、工厂中工作好的中学生选来上学。

军事院校,谁知道办了一百一十个,“幺幺幺”。其中确实有“妖”。只怪我们自己,办那么多干什么呀?又没人管,叶剑英你不是管吗?搞四清不行,还是统统开进工人,开进解放军去。

这场无产阶级文化大革命究竟能不能搞到底?我们讲要搞到底,什么到底呢?这是一个问题。估计三年差不多,到明年夏季差不多了。到底就是包括大批判、清理阶级队伍、整党、精兵简政、改革不合理的规章制度。

工厂不整你们看,不搞文化大革命怎么得了?比如北京一个木材厂,有那么千把人,是两个资本家合起来的。有的厂三朝元老,有的四朝元老。还有新华印刷厂,北洋军阀时期它给印票子,日本来了给日本人印报纸传单。国民党接管的,共产党来了也吃得开,它都吃得开,有一部分三开、四开干部,其中有的是隐藏的反革命。我们的人也不得力,思想状态、精神状况也不好。隐藏的反革命不清理一下,工厂究竟是谁的呀?

按说农村要好一些,也有问题。没有搞好的公社、大队,要好好抓一下。靠人民解放军、省地县革命委员会。有一个省组织了一百万人的宣传队,解放军只有几万,百分之九十是贫下中农,就是广东。江西也不错。江西是个中等省,两千二百万人,有的地方人太多,省级机关就七千人,至少要减六千人,减到哪里去?还不是中国境内?一个工厂,一个农村,还是下放农村为主,工厂不能容纳好些人。

\date{一九六八年十月三十一日闭幕式上的讲话}
\section*{(二)}

这次会议我看开得可以,开得很好嘛。是否有些同志对“二月逆流”这件事不了解,经过十八天比以前了解了。过去高级干部也不清楚,我们没有透风。过去各省来解决问题,也没有工夫来讲这些事情。在这次会上有犯错误的同志说明问题,同志们质问他答复。质问再答复,比较清楚了。这件事要说小嘛也不算小,要说大嘛,依我看也不是十分了不起,是一种自然现象。他们有意见嘛,总是要讲嘛。几个人在一起,都是政治局委员、副总理、军委副主席,讲一讲也是允许的。党的生活也是允许的,是公开讲的。两个大闹就是公开讲的嘛。有些细节,过去我也不太清楚。细节如来往多少次,有过什么交往,不要过分注意这些,不要把党内生活引到细微末节。不注意大问题就不好了。这不是一件小事,你说天就塌下来?!地球就不转了?!照样转,还能不转吗?!我倒是佩服邓老,一直顶住。要是我,就不理,结果你还是搞出一篇自我批评。他早就有自我批评了,比如北戴河会议的时候就有自我批评了,有些老同志将来是否可以要工作,打倒的谭启龙、江渭清\mnote{12},过几年以后,大家气消了,也可以做点工作嘛。不作主要工作,还有其他一些人,身强力壮,将来总得给点事做吧!邓华\mnote{13}来了吧?(邓站起来)好久不见了。在四川几年没有人说他不好,不晓得红卫兵又把他关起来了,你这个人跟彭德怀犯了错误,改正错误就好了。在北京的有黄克诚、谭政\mnote{14}。犯错误的人,允许人家改正错误,要给他时间,直至多少年。因为群众还要看他的表现嘛。

现在正搞大批判,清理阶级队伍。这件事一是要抓紧,二是要注意政策。不是要稳、准、狠嘛?稳,有右的,稳就不稳了。狠就可以搞得很“左”,就搞过了火。重点就是“准”字。否则,不准,稳也稳不了,狠也狠不起来。要调查研究,要注重证据,不要搞挂黑牌子、喷气式,不要打人。我在一九二七年写了湖南一篇文章,对土豪劣绅,戴高帽子,游乡,其实几十年都不搞了。红卫兵就按那个办法逐步升级,挂黑牌子,搞喷气式,还有各种各样的。对特务、汉奸、死不悔改的走资派,要调查研究,注重证据,不要重口供,不要打人、戴高帽子、搞喷气式那一套。这样结果并不好。在北京有杜聿明、王耀武\mnote{15}嘛。过去对敌人俘虏也不搞这一套。

有些死不悔改的人,你整他,他也不悔改。冯友兰\mnote{16}你不叫他搞唯心主义一套,我看也难。还有一个翦伯赞,北大教授,历史学家,资产阶级历史权威嘛。你不要他搞帝王将相也难。对这些人不要搞不尊重他人格的办法。如薪水每月只给二十四元,最多的给四十元,不要扣得太苦了。这些人用处不多了。还有吴晗,可能还有某些用处。要问唯心主义,要问帝王将相,还得问一问他们。在坐的范老\mnote{17},也是搞帝王将相,郭老\mnote{18}也算一个吧。那时没有别的书看嘛,都是二十四史之类的。要反帝王将相,还要知道什么是帝王将相。不然人家问你答不出来。我可不是劝你们在坐的人去搞帝王将相。你们还是按总理讲的,按那几个文件去搞。我也不赞成青年学生去搞。少数人去搞。比如范老,你不搞,帝王将相不就绝种了?不是要再出帝王将相,而是历史要人去研究。

清理阶级队伍,对这些人你们要注意,有好的,中间的,极右的三种。我说的对右的,应当如何对待他们。世界上总有左中右。没有右,你左从哪里来呀?没有那么绝。统统是左派,我不赞成。哪有的事?那样纯也不见得。我们的党通过这场文化大革命,我看比较纯了。从来没有这样搞过。但太纯了也不太好。比如九大代表,二月逆流的同志不参加就是缺点。所以我们还是推荐各地把他们选作代表。陈毅同志说他没有资格,我看你有资格,代表左中右的那个右派嘛。你对九大代表三个条件中的第二条不那么符合,可以协商嘛。有少数人参加有好处。

现在情况同两年前十一中全会时不同了,很大的不同了。运动还没有完,就是九大开过了,运动也不见得马上就完。因为这涉及到每个工厂、每个学校、机关。

\date{引自八届扩大的十二中全会公报}
\section*{(三)}

这次无产阶级文化大革命,对于巩固无产阶级专政,防止资本主义复辟,建设社会主义,是完全必要的,是非常及时的。

\begin{maonote}
\mnitem{1}张鼎丞,时任最高人民检察院检察长。
\mnitem{2}邓老,邓子恢,与张鼎丞曾长期在华中、华野共事,相互熟悉。时任全国政协副主席,兼任国家计委副主任,分管银行工作。
\mnitem{3}谭启龙,时任中共山东省委第一书记。
\mnitem{4}张体学,时任湖北省省长。
\mnitem{5}王任重,原任湖北省委第一书记。
\mnitem{6}王延春,曾任湖南省委常务书记及第二书记、代理第一书记。
\mnitem{7}邓小平的自传检讨,一九六八年七八月间,邓小平写了一封检讨书《我的自述》,转给中央及毛主席,全文分为“红七军工作时期”、“在中央苏区的三年多”、“在太行工作时期”、“在北京工作时期”,“在北京工作时期”内容如下:

一九五二年我到北京工作以后,特别是被“八大”选为中央总书记的十年中,我的头脑中,无产阶级的东西越来越少,资产阶级的东西越来越多,由量变到质变,一直发展到推行了一条资产阶级反动路线,变成了党内最大的走资本主义道路当权派之一。

准备党的“八大”时,指定我主持修改党章。在修改的党章中,删去了“七大”党章中以毛泽东思想为党的指导思想的内容,这个重大原则问题虽然不是由我提出的,但我是赞成的。我的这个罪过,对于党和人民,对于社会主义事业,带来了极大的损害。“八大”会上,我代表中央作的关于修改党章的报告中,错误地评价了苏共二十大的作用,错误地提到反对个人崇拜问题。这个报告是几个人集体起草的,这一段也不是由我写的,似乎记得还是参照一论无产阶级专政的历史经验写的,但作为主持起草的我应负不可推卸的主要责任。这是一个丧失原则立场的错误。

在考虑“八大”中央委员人选时,对过去曾有叛变行为,以后又在长期工作中有所表现的人,是否可以当中委的问题,我当时认为,对某些人可以作特殊情况处理,提为中委候选人。随即由安子文等人起草了一个文件,这个文件是完全违反党的组织原则的,是极端错误的,它给一些人混入党的各级领导机关,大开方便之门。我是筹备“八大”的一个重要负责人,我是赞成这个文件的,应负严重的责任。回想日本投降后,我和薄一波违反党的组织原则,介绍叛徒刘岱峰入党,虽然此事在组织上是经过上级批准的,回想起来,也是犯了与上述问题同一性质的政治错误。这直接违反了主席一九四零年十二月在《论政策》这个指示中规定的“对于叛徒,……如能回头革命,还可予以接待,但不准重新入党”这样明确的原则的。

我在担任总书记的十年中,最根本、最严重的罪行,是不突出无产阶级政治,不传播毛泽东思想,长期不认识毛泽东思想在国内和国际革命中的伟大意义。没有认真学习,认真宣传,还讲过在宣传毛泽东思想中不要简单化这类的话。

一九五八年实现人民公社化,我确实高兴,但在我的思想中,从此滋长了阶级斗争减弱的观点,所以在后来的长时期中,我在处理阶级斗争的问题上,总是比较右的,无论在两条路线和两条道路的斗争方面,或者在党内斗争(阶级斗争在党内的反映)方面,都是如此。

一九六一年我参与制定了工业企业管理条例(草案)七十条,这个文件不是强调政治挂帅、即毛泽东思想挂帅的,是包含许多严重错误的东西,我对此要负主要责任。

一九六二年刮单干风的时候,我赞成安徽搞“包产到户”这种破坏社会主义集体经济,其实就是搞单干的罪恶主张,说过“不管黄猫黑猫,抓得住老鼠就是好猫”等极其错误的话。这几年,还存在着高估产、高征购的错误,每年征购任务的确定,我都是参与了的。基本建设项目,有些不该退的也退了。我作为总书记,对这些错误负有更多的责任。

一九六三年开始的社会主义教育运动,有了主席亲自主持制定的前十条,明确地以阶级斗争、两条道路的斗争为纲,规定了一套完整的、正确的理论、方针、政策和方法,完全没有必要再搞一个第二个十条。第二个十条是完全错误的。在杭州搞这个文件的时候,我是参加了的,我完全应该对这个文件的错误,负重大的责任。

我主持书记处工作十年之久,没有系统地总结问题和提出问题,向毛主席报告和请示,这在组织上也是绝不允许的,犯了搞独立王国的错误。一九六五年初,伟大领袖毛主席批评我是一个独立王国,我当时还以自己不是一个擅权的人来宽解,这是极其错误的。近来才认识到,独立王国不可能没有政治和思想内容的,不可能只是工作方法的问题。既是独立王国,就只能是资产阶级司令部的王国。书记处成员前后就有彭真、黄克诚、罗瑞卿、陆定一、杨尚昆等多人出了问题,这是与我长期不突出无产阶级政治,不突出毛泽东思想的错误密切关连的,结果我自己最后也堕落到这个修正主义份子的队伍中了。在书记处里,我过份地信任彭真,许多事情都交给他去处理,对杨尚昆安窃听器,我处理得既不及时,又不认真,对此我应负严重的政治责任。在处理对罗瑞卿斗争的问题上,我同样犯了不能容忍的严重错误。

大量事实表明,在每个重要关节,在两个阶级、两条道路、两条路线的斗争中,我不是站在无产阶级方面,而是站在资产阶级方面;不是站在毛主席的无产阶级革命路线和社会主义道路方面,而是站在资产阶级路线和资本主义道路方面。

文化大革命一开始,我就同刘少奇提出了一条打击革命群众、打击革命左派、扼杀群众运动、扼杀文化大革命的资产阶级反动路线。毛主席《炮打司令部》的大字报出来后,我才开始感到自己问题的严重。接着,革命群众大量揭发了我多年来在各方面的错误和罪行,才使我一步一步地清醒起来。我诚恳地、无保留地接受党和革命群众对我的批判和指责。当我想到自己的错误和罪行给革命带来的损害时,真是愧悔交集,无地自容。我完全拥护把我这样的人作为反面教员,进行持久深入的批判,以肃清我多年来散布的流毒和影响。对于我本人来说,文化大革命也挽救了我,使我不致陷入更加罪恶的深渊。

我入党四十多年,由于资产阶级世界观没有得到改造,结果堕落成为党内最大的走资派。革命群众揭发的大量事实,使我能够重新拿着一面镜子来认识我自己的真正面貌。我完全辜负了党和毛主席长期以来对我的信任和期望。我以沉痛的心情回顾我的过去。我愿在我的余年中,悔过自新,重新做人,努力用毛泽东思想改造我的资产阶级世界观。对我这样的人,怎样处理都不过分。我保证永不翻案,绝不愿做一个死不悔改的走资派。我的最大希望是能够留在党内,请求党在可能的时候分配我一个小小的工作,给我以补过从新的机会。我热烈地欢呼无产阶级文化大革命的伟大胜利。
\mnitem{8}韩先楚,时任福建省革命委员会主任。一九六八年八月十九日福建省成立革命委员会。
\mnitem{9}辽宁三大派,当时辽宁分为三派,“辽宁省革命造反派大联合委员会”(简称“辽联”)、“辽宁无产阶级革命派联络站”(简称“辽革站”)、“八三一革命造反总司令部”(简称“八三一”,后经过联合,于一九六八年五月十日辽宁成功成立革命委员会,主要负责人是陈锡联。
\mnitem{10}肖克,即萧克,上将,长征中萧克支持张国焘的分裂主张,萧克在二十世纪九十年代,支持创办了反毛非毛反社会主义的极右杂志《炎黄春秋》。
\mnitem{11}李德生,时任安徽省革命委员会主任,一九六八年四月十八日安徽成立革命委员会。
\mnitem{12}谭启龙,一九七〇年复出后历任福建省革委会副主任、省委书记,一九七二年四月起历任浙江省委书记(主持工作)、省革委会副主任,浙江省委第一书记、省革委会主任、省军区第一政委。江渭清,一九七四年底复出,中央派江渭清到江西工作,历任江西省委第一书记、省革委会主任、福州军区政委、江西省军区第一政委。
\mnitem{13}邓华,一九五九年八月,庐山会议上,邓华因彭德怀案被撤销党内外一切职务,一九六零年转业到地方,任四川省副省长,主管农业机械工作。
\mnitem{14}黄克诚,一九五九年八月,庐山会议上,黄克诚与彭德怀等人被定为“反党集团”成员,被解除军委总参谋长职务,一九六五年,黄克诚任山西省任副省长,分管农业。

谭政,原任中国人民解放军总政治部主任,一九六〇年九、十月间召军委扩大会议通过《中共中央军事委员会扩大会议关于谭政同志错误的决议》,定性为“在总政结成反党宗派集团”,一九六一年一月后,被撤销了中央军委常务委员、中央书记处书记等职务,一九六五年十一月任福建省副省长,一九七五年八月复出任中共中央军委顾问。
\mnitem{15}杜聿明,著名国民党高级将领,一九四九年一月在淮海战役中被俘,一九五九年被特赦,其女婿为美籍华裔诺贝尔得主杨振宁博士。王耀武,著名国民党高级将领,军事才能突出,在济南战役中被俘,一九五九年被特赦。两人都被粟裕将军所俘。
\mnitem{16}冯友兰,字芝生,河南唐河人。一九一二年入上海中国公学大学预科班,一九一五年入北京大学文科中国哲学门,一九一九年赴美留学,一九二四年获哥伦比亚大学博士学位。回国后历任中州大学、广东大学、燕京大学教授、清华大学文学院院长兼哲学系主任。抗战期间,任西南联大哲学系教授兼文学院院长。一九四六年赴美任客座教授。一九四八年末至一九四九年初,任清华大学校务会议主席。曾获美国普林斯顿大学、印度德里大学、美国哥伦比亚大学名誉文学博士。一九五二年后一直为北京大学哲学系教授。

在燕京大学任教期间,冯友兰讲授中国哲学史,分别于一九三一年、一九三四年完成《中国哲学史》上、下册,后作为大学教材,为中国哲学史的学科建设做出了重大贡献。

从一九三九年到一九四六年七年冯友兰连续出版了六本书,称为“贞元之际所著书”:《新理学》(1937)、《新世训》(1940)、《新事论》(1940)、《新原人》(1942)、《新原道》(1945)、《新知言》(1946)。通过“贞元六书”,冯友兰创立了新理学思想体系,使他成为中国当时影响最大的哲学家。

二十世纪五六十年代是冯友兰学术思想的转型期。新中国成立后,冯友兰放弃其新理学体系,接受马克思主义,开始以马克思主义为指导研究中国哲学史。著有《中国哲学史新编》第一、二册、《中国哲学史论文集》、《中国哲学史论文二集》、《中国哲学史史料学初稿》、《四十年的回顾》和七卷本的《中国哲学史新编》等书。
\mnitem{17}范老指范文澜。
\mnitem{18}郭老指郭沫若。
\end{maonote}
