
\title{美帝国主义是纸老虎}
\date{一九五六年七月十四日}
\thanks{这是毛泽东同志同两位拉丁美洲人士谈话的一部分。}
\maketitle


美国到处打着反共的招牌,为着达到侵略别人的目的。

美国到处欠账。欠中南美国家、亚非国家的账,还欠欧洲、大洋洲国家的账,全世界,包括英国在内,都不喜欢美国。广大人民都不喜欢美国。日本不喜欢美国,因为美国压迫日本。东方各国,没有一国不受到美国的侵略。美国侵略中国的台湾省。日本、朝鲜、菲律宾、越南、巴基斯坦,都受到美国的侵略,其中有些还是美国的盟国。人民不高兴,有些国家的当局也不高兴。

一切受压迫的民族都要独立。

一切会有变化。腐朽的大的力量要让位给新生的小的力量。力量小的要变成大的,因为大多数人要求变。美帝国主义力量大要变小,因为美国人民也不高兴本国的政府。

我这一辈子就经历了这种变化。我们这里在座的有清朝出生的人,有民国出生的人。

清朝,早被推翻了。什么人推?孙中山领导的党和人民一起推。孙中山力量很小,清朝的官员看不起他。他多次起义总是失败。最后,还是孙中山推翻了清朝。大,不可怕。大的要被小的推翻。小的要变大。推翻清朝以后,孙中山失败了。因为他没有满足人民的要求,比如没有满足人民对土地的要求,对反帝的要求。他也不晓得镇压反革命,当时反革命到处跑。后来,他就失败于北洋军阀首领袁世凯之手。袁世凯的力量比孙中山的大。但还是照这个规律:力量小的,同人民联系的,强;力量大的,反人民的,弱。尔后,孙中山的资产阶级民主革命派同我们共产党合作,把袁世凯留下来的军阀系统打败了。

蒋介石统治中国,得到全世界各国政府的承认,统治了二十二年,力量最大。我们力量小,原先有五万多党员,经过反革命的镇压,只剩下一万多党员。敌人到处捣乱。但还是照这个规律:强大的失败,因为它脱离人民;弱小的胜利,因为它同人民联系在一起,为人民工作。结果,也就是这样。

抗日战争的时候,日本很强大,国民党的军队被赶到了偏僻的地区,共产党领导的武装力量,也只能在敌后农村开展游击战争。日本占领了中国的大城市北京、天津、上海、南京、武汉、广州。但是,日本军国主义,还有德国希特勒,也是照这个规律,没几年就倒了台。

我们经过了很多困难,从南方被赶到北方,从几十万人到只剩下几万人。长征二万五千里,剩下二万五千人。

我们党的历史上有过多次“左”倾和右倾的路线错误。其中最严重的是陈独秀的右倾和王明的“左”倾。此外,还有张国焘、高岗等人的右倾错误。

犯错误也有好处,可以教育人民,教育党。我们有很多反面教员,如日本、美国、蒋介石、陈独秀、李立三、王明、张国焘、高岗。向这些反面教员学习,付出了很大的代价。在历史上英国同我们打过很多仗。英国、美国、日本、法国、德国、意大利、沙俄、荷兰,都很喜欢我们这块地方。他们都是我们的反面教员,我们是他们的学生。

经过抗战时期,打日本,我们的军队发展到了九十万。然后是解放战争。我们的枪炮不如国民党。国民党军队四百万,打了三年,累计起来,被我们消灭了八百万。在美帝国主义帮助下的国民党打不赢我们。强大的打不赢,弱小的总是胜利。

现在美帝国主义很强,不是真的强。它政治上很弱,因为它脱离广大人民,大家都不喜欢它,美国人民也不喜欢它。外表很强,实际上不可怕,纸老虎。外表是个老虎,但是,是纸的,经不起风吹雨打。我看美国就是个纸老虎。

整个历史证明这一点,人类阶级社会的几千年的历史证明这一点:强的要让位给弱的。美洲也是这样。

只有帝国主义被消灭了,才会有太平。总有一天,纸老虎会被消灭的。但是它不会自己消灭掉,需要风吹雨打。

我们说美帝国主义是纸老虎,是从战略上来说的。从整体上来说,要轻视它。从每一局部来说,要重视它。它有爪有牙。要解决它,就要一个一个地来。比如它有十个牙齿,第一次敲掉一个,它还有九个,再敲掉一个,它还有八个。牙齿敲完了,它还有爪子。一步一步地认真做,最后总能成功。

从战略上说,完全轻视它。从战术上说,重视它。跟它作斗争,一仗一仗的,一件一件的,要重视。现在美国强大,但从广大范围、从全体、从长远考虑,它不得人心,它的政策人家不喜欢,它压迫剥削人民。由于这一点,老虎一定要死。因此不可怕,可以轻视它。但是,美国现在还有力量,每年产一亿多吨钢,到处打人。因此还要跟它作斗争,要用力斗,一个阵地一个阵地地争夺。这就需要时间。

看样子,美洲国家、亚洲非洲国家只有一直同美国吵下去,吵到底,直到风吹雨打把纸老虎打破。

为了反对美帝国主义,中南美国家的欧洲移民要同本地印第安人团结起来。从欧洲移入的白种人,是不是可以分为两部分,一部分人是统治者,另外一部分人是被统治者。这样,这一部分被压迫的白种人就容易同本地人接近了,因为所处的地位相同。

我们和拉丁美洲的朋友,和亚洲非洲的朋友,是处在同一种地位,做同样的工作,为人民办点事,减少帝国主义对人民的压迫。搞得好了,可以根本取消帝国主义的压迫。在这一点上,我们是同志。

在反对帝国主义的压迫上,我们同你们性质相同,只是所在地区、民族、语言不同。我们同帝国主义却有性质上的分别,我们看到帝国主义就不舒服。

要帝国主义干什么?中国人民不要帝国主义,全世界人民也不要帝国主义。帝国主义无存在之必要。
