
\title{关于划分三个世界的谈话}
\date{一九七四年二月、三月}
\thanks{这是毛泽东同志会见卡翁达、布迈丁、尼雷尔\mnote{1}时的谈话内容纪要。}
\maketitle


\date{一九七四年二月二十二日}
\section{(一)毛泽东会见赞比亚总统卡翁达}

希望第三世界团结起来。第三世界人口多啊。

我看美国、苏联是第一世界。中间派,日本、欧洲、澳大利亚、加拿大是第二世界。咱们是第三世界。美国、苏联原子弹多,也比较富。第二世界,欧洲、日本、澳大利亚、加拿大,原子弹没有那么多,也没有那么富,但是比较第三世界要富。第三世界人口很多。

亚洲除了日本,都是第三世界。整个非洲都是第三世界。拉丁美洲也是第三世界。

我们是共产党,是要帮助人民的。如果不帮助人民,就是背叛马克思主义。你呢,我们希望也是要帮助人民。我劝你对人民要好啊!没有人民就会垮台。

(当卡翁达谈到赞比亚支持世界的反帝反殖斗争和在国内反对剥削阶级的斗争时),你现在不能当共产党,你当共产党,人家就都反对你,但是你可以看一点马克思的书。

(当卡翁达赞扬我援赞工程人员时),我们是共产党啊,应该好一点!我们的人也犯了一些错误呢。要教育。共产党内也有大国沙文主义。有一些人看不起第三世界一些国家的人民,所以应该教育。我委托你教育中国的工程人员,还有尼雷尔总统,也应该这样做。在你们那里工作的,世界上的人作了坏事,不管哪个国家的都应该教育、处分或者把他们赶回来。不然那些人就把尾巴翘到天上:帮助你们修了铁路,了不起呀!

非洲最好统一起来。南部非洲难啊!中间和北方的慢慢地统一起来。你们应该发展人口。中国人口太多。非洲人还不够。

\date{一九七四年二月二十五日}
\section{(二)毛泽东会见阿尔及利亚革命委员会主席布迈丁}

中国属于第三世界。因为政治、经济、各方面,中国不能跟富国、大国比,只能跟一些比较穷的国家在一起。

(当布迈丁问到美苏是否达成某种协议和进一步问到战争问题时)

协议可能有,但是我看不那么巩固。一是暂时的,同时是骗人的。骨子里头还是争夺为主。我看会打仗。总而言之,将来总有一天会要打的。争夺的结果最后可能会武力解决,暂时还不会。现在都在讲和平。现在世界上的舆论我看要研究一下,就是不要真正相信所谓的永久和平。这个社会制度不改变,战争不可避免,不是相互之间的战争,就是人民起来革命。

这个世界上是有帝国主义存在。俄国也叫社会帝国主义,这种制度也就酝酿着战争。不是你们要打世界战争,我们要打,第三世界要打世界战争,也不是这些富国的人民想要打世界大战,这种东西是不以人们的意志为转移的。谁想到希特勒几乎统一了欧洲,又失败了。谁想到第一次世界大战中间又产生了十月革命。中国在第二次世界大战结尾把日本、后来又把蒋介石赶走了。

至于打不打原子武器,可能打,也可能不打。第二次世界大战,美国对日本打过,但是后头在朝鲜战争,就没有打。在越南战争,也没有打。在中东,以色列也好,埃及也好,美国人跟俄国人也没有援助他们原子武器。

(布迈丁说,我们国家太小了,没有办法对付大国。)

你们不小,你们把法国人赶走了,越南人也把美国问题解决了,朝鲜民主主义人民共和国也把美国军队赶到南边去了。现在美国到处霸了地方,它要保护这些地方,力量就分散。苏联要对付的地方也多,欧洲、地中海、阿拉伯世界、中东、南亚、中国、日本、美国,太平洋的美国。它的事情也不好办呢。

(布迈丁问,面对这种形势,中国的态度如何?)

准备打仗!准备它们(美苏)在世界上闹事。绝不相信持久和平,或者说所谓一代人的和平。

(谈到中国的成就和反对大国沙文主义时)

中国成就有一点,但是不大。我们犯过许多错误,犯了错误就改正。有时候工作方法比较好,有时候不大好。如果片面地介绍中国,说怎么好,那是不妥的。当然,说中国是一片黑暗,也是不对的。光明面是主要的,但是有时候有黑暗这一面。我们下面的工作人员就爱吹他的成绩,而不爱把自己的错误讲出来,所以你们要注意。别国大体也是如此,总是有光明的一面,也有缺点。

地中海是密切关系欧洲的。欧洲安全,阿拉伯世界不安全,地中海问题不解决,那怎么行呢?几十个国家怎么能取得一致?单是欧洲就有三十几个国家。如果是听两个大国美国跟苏联,这也不行吧。

\date{一九七四年三月二十五日}
\section{(三)毛泽东会见坦桑尼亚总统尼雷尔}

(当尼雷尔提到美苏不是为和平而努力时),和平是暂时的,将来就难说了。

总而言之,所谓裁军,第一次世界大战以后就说要裁军,结果谈出了一个第二次世界大战。第二次大战后又说要裁军,又是几年了,没有一个裁的。但是他们双方都说他们要搞和平,而且是长久的和平,或者是一代人的和平。一代人嘛,大概是半个世纪,五十年吧。何不讲两代人呢?因为这一讲,他的武器就没有销路了。大国都靠出卖武器赚钱。总而言之,我们对“持久和平”这样的口号要看一看呢。大概一段时间可以,太长了不行,因为社会制度没改变。总而言之,这个全世界是不安定的。他们之所以需要讲和平,就是因为这样讲对他们比较有利。他们又利用各国人民怕打核战争的心理状况,所以有许多人接受和平的口号。特别是在第一次世界大战以后,大家都讲和平,结果讲出一个第二次世界大战来了。现在又讲和平,可能也讲出一个什么战争来吧。

他们(指美苏两个超级大国)现在有点怕第三世界。

(尼雷尔说,如果第三世界没有中国,他们就不会怕)也怕呢。

(尼雷尔说,第三世界没有中国,就成了纸老虎)那不能这么讲!第三世界团结起来,使得工业国家,比如日本、欧洲和两个超级大国,都得要注意一点。

整个非洲的事情怎么样了?比过去都要好些。就是讲北部和中部非洲。

(当尼雷尔谈及他们在南部非洲正在努力赢得独立时)我跟你们意见一致,就是不喜欢南非白人政权控制非洲人民。并说,(南部非洲)将来总是要起变化的。非洲的变化已经够快的了。

(尼雷尔说,中国现在对非洲的帮助是很多的),帮助很小。

(当尼雷尔称赞中国医疗队时),应该主要是帮忙教会你们的医生。搞铁路的也应该主要是教会你们勘察、各种建设、修路、桥梁。各种技术人员都这样,我们将来一走,你们就完全可以自己管理了。如果不教,那就不好哩。

听说我们的人在你们那里做了一些坏事。给了赔偿没有啊?有些犯错误的也撤回了吧。人多了,我们教育又不严,势必将来也还要出一些问题。你们发现有什么错误,就告诉我们的大使。

(尼雷尔称赞中国人员在坦桑人对他们不好时和在极端困难的条件下工作从不发牢骚),不能怪你们。这个不能发牢骚,发牢骚是错误的。

(尼雷尔谈及他的母亲向我人员送活羊和鸡蛋时),收人家礼物不大好吧。以后成为风气不大好。

\begin{maonote}
\mnitem{1}一九七四年二月二十二日,毛泽东会见了赞比亚总统卡翁达;二月二十五日,会见了阿尔及利亚革命委员会主席布迈丁;三月二十五日,毛泽东会见了坦桑尼亚总统尼雷尔等外宾。
\end{maonote}
