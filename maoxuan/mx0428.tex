
\title{在不同地区实施土地法的不同策略}
\date{一九四八年二月三日}
\thanks{这是毛泽东给刘少奇的一封电报。}
\maketitle


关于土地法\mnote{1}的实施,应当分三种地区,采取不同策略。

一、日本投降以前的老解放区。这种地区大体上早已分配土地,只须调整一部分土地。这种地区的工作中心,应当是按照平山经验\mnote{2},用党内党外结合的方法整理党的队伍,解决党同群众间的矛盾。在这种老区,不是照土地法再来分配一次土地,也不是人为地、勉强地组织贫农团去领导农会,而是在农会中组织贫农小组。这种小组的积极分子,可以担负农会和农村政权中的领导工作,但是不应当排斥中农,规定一定由贫农做领导工作。这种地区的农会和农村政权的领导工作,应当由贫农和中农中思想正确、办事公道的积极分子去做。在这种地区,过去的贫农大多数已升为中农,中农已占乡村人口的大多数,所以必须吸收中农中的积极分子参加农村的领导工作。

二、日本投降至大反攻,即一九四五年九月至一九四七年八月两年内所解放的地区。这种地区,占现在解放区的绝大部分,可称为半老区。在这种地区,经过两年清算斗争,经过执行《五四指示》\mnote{3},群众的觉悟程度和组织程度已经相当提高,土地问题已经初步解决。但群众觉悟程度和组织程度尚不是很高,土地问题尚未彻底解决。这种地区,完全适用土地法,普遍地彻底地分配土地,并且应当准备一次分不好再分第二次,还要复查一、二次。这种地区,中农占少数,并且是观望的。贫农占大多数,积极要求土地。因此,必须组织贫农团,必须确定贫农团在农会中、在农村政权中的领导地位。

三、大反攻后新解放的地区。这种地区,群众尚未发动,国民党和地主、富农的势力还很大,我们一切尚无基础。因此,不应当企图一下实行土地法,而应当分两个阶段实行土地法。第一阶段,中立富农,专门打击地主。在这个阶段中,又要分为宣传,做初步组织工作,分大地主浮财\mnote{4},分大、中地主土地和照顾小地主等项步骤,然后进到分配地主阶级的土地。在这个阶段中,应当组织贫农团,作为领导骨干,还可组织以贫农为主体的农会(可称为农民协会)。第二阶段,将富农出租和多余的土地及其一部分财产拿来分配,并对前一阶段中分配地主土地尚不彻底的部分进行分配。第一阶段,大约须有两年时间;第二阶段,须有一年时间。太急了,必办不好。老区和半老区的土地改革和整党,也须有三年时间(从今年一月算起),太急了,也办不好。


\begin{maonote}
\mnitem{1}土地法,指《中国土地法大纲》。见本卷\mxnote{目前形势和我们的任务}{7}。
\mnitem{2}平山县在河北省西部,当时属于晋察冀解放区。这里说的平山经验,是指该县在土地改革运动中,采取邀请党外群众列席党支部会议的方式,整顿党在农村的基层组织的经验。
\mnitem{3}即中共中央一九四六年五月四日发出的《关于土地问题的指示》,见本卷\mxnote{三个月总结}{5}。
\mnitem{4}浮财,指粮食、金钱、衣服、什物等动产。
\end{maonote}
