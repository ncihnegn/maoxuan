
\title{彻底纠正“五风”}
\date{一九六〇年十一月十五日}
\thanks{这是毛泽东同志为纠正“五风”写的党内指示。}
\maketitle


\mxname{各中央局,各省委、市委、自治区党委:}

发去湖北省委王任重同志报告\mnote{1}一件,湖北省沔阳县总结\mnote{2}一件,湖北省沔阳县通海口公社纠正错误后新情况报告\mnote{3}一件,供你们参考。必须在几个月内下决心彻底纠正十分错误的共产风、浮夸风、命令风、干部特殊风和对生产瞎指挥风,而以纠正共产风为重点,带动其余四项歪风的纠正。省委自己全面彻底调查一个公社(错误严重的)使自己心中有数的方法是一个好方法。经过试点然后分批推广的方法,也是好方法。省委不明了情况是很危险的。只要情况明了,事情就好办了。一定要走群众路线,充分发动群众自己起来纠正干部的五风不正,反对恩赐观点。下决心的问题,要地、县、社三级下决心(坚强的贯彻到底的决心),首先要省委一级下决心,现在是下决心纠正错误的时候了。只要情况明,决心大,方法对,根据中央十二条指示\mnote{4},让干部真正学懂政策(即十二条),又把政策交给群众,几个月时间就可把局面转过来,湖北的经验就是明证。十二月上旬或中旬,中央将召集你们开会,听取你们的汇报,请你们对自己的工作预作安排。

\begin{maonote}
\mnitem{1}指中共湖北省委第一书记王任重一九六〇年十一月十二日关于纠正“五风”给中南局第一书记陶铸并毛泽东的报告。其内容如下:(一)送上《沔阳县贯彻政策第一阶段的总结》和《通海口公社贯彻政策后的变化》两个材料。看看这两个材料,可以使人早下决心,更坚定地去纠正“共产风”。(二)省委宣传通海口公社的经验后,全省大部分县都已先后进行了试点,少数县已向面上铺开。(三)为贯彻执行中央的“十二条指示”,全省三级干部会议已于十一日开始。会议围绕“苦战三年,总结经验”这一主题,采取群众路线、整风和批评与自我批评的方法,充分发扬民主,让大家畅所欲言,先集中揭发错误,然后再作全面评价,目的是为了弄清真实情况,接受三年来的教训,使全党思想一致,团结一致,去战胜当前的困难,夺取农业丰收。这次会议中,讲对讲错,一律不记帐,不戴帽子;要求地、县委首先批评省委,然后县委批评地委,最后是省、地、县三级作自我批评。总之,要建立正常的党内民主生活制度。会议打算开十天左右,开长开短,以解决问题为原则。
\mnitem{2}指中共沔阳县委一九六〇年十一月三日关于贯彻政策第一阶段的总结报告。其主要内容是:(一)我们以贯彻省委“十项政策”为中心,开展了群众性整风整社运动。到目前为止,第一阶段已基本结束。这个阶段,主要地解决“共产风”和瞎指挥生产的问题。从揭发情况看,全县所有公社,问题都极为严重,在经济、政治上都带来了极为严重的后果。(二)对过去所犯的政策错误和作风问题进行了纠正,有的还在继续纠正。对“共产风”中的损失,坚决兑现,物在还物,物不在赔钱。其他政策问题,根据省委“十项政策”一一作了检查处理。干部强迫命令、瞎指挥,都向群众作了深刻检讨。(三)经过整风整社,群众生产积极性大大提高,干部作风有了很大转变,干群关系在新的基础上密切起来了。生产形势变了,群众的生活也发生了变化。(四)这次整风整社的经验,就是通海口公社总结的五条:领导下决心,干部、群众“两头挤”,把兑现抓到底,明确地划界线,与生产扭在一起。此外,我们也有些新的体会:运动必须坚持分批展开,要不断壮大干部队伍;强调大搞群众运动,反对政策兑现中的“恩赐观点”和少数干部包办代替的做法;加强对运动的领导,一方面正确掌握政策,一方面紧紧地依靠群众;必须把政策、作风、生产、生活紧紧地扭在一起,以处理“共产风”为重点全面贯彻“十项政策”。经验证明,纠正“共产风”、强迫命令风、浮夸风、特殊化风,哪里搞得彻底,哪里工作就能前进。
\mnitem{3}指中共沔阳县委第一书记马杰一九六〇年十一月九日关于通海口公社贯彻政策后变化情况的报告。报告中说,通海口公社经过这次贯彻省委“十项政策”、坚决纠正“五风”之后,面貌发生了根本的变化。群众心情舒畅,对社会主义的误解消除了,对党的政策信任了,普遍树立了兴家立业、当家作主的思想,人人关心生产,爱护公物,生产队则大搞农田基本建设,改善经营条件。干部作风有了显著改变,参加劳动已开始形成制度,通过生产了解了实际情况,克服了工作上的主观主义;通过处理“共产风”,干部的政策水平有了提高,纠正了强迫命令、瞎指挥,使党群关系密切起来,干部工作也好做了。这些变化,大大推动了生产和生活,使得生产出现高潮,生活面貌发生了改观,鼓舞了社员群众夺取明年大跃进的信心。今年冬季,将根据中央“十二条指示”,继续开展整风整社。
\mnitem{4}周恩来一九六〇年十月三十一日送毛泽东审阅的这个中共中央紧急指示信稿指出,“共产风”必须坚决反对,彻底纠正。必须把当前农村中迫切需要解决的一系列政策问题,特别是关于人民公社所有制方面的一系列政策问题,向各级党组织讲清楚,做到家喻户晓。把政策交给群众,发动群众监督党员干部认真地、不折不扣地贯彻执行。指示信的主要内容有十二条:(一)三级所有,队为基础,是现阶段人民公社的根本制度。(二)坚决反对和彻底纠正“一平二调”的错误。(三)加强生产队的基本所有制。(四)坚持生产小队的小部分所有制。(五)允许社员经营少量的自留地和小规模的家庭副业。(六)少扣多分,尽力做到百分之九十的社员增加收入。(七)坚持各尽所能、按劳分配的原则,供给部分和工资部分三七开。(八)从各方面节约劳动力,加强农业生产第一线。(九)安排好粮食,办好公共食堂。(十)有领导有计划地恢复农村集市,活跃农村经济。(十一)认真实行劳逸结合。(十二)放手发动群众,整风整社。
\end{maonote}
