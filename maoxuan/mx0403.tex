
\title{第十八集团军总司令给蒋介石的两个电报}
\date{一九四五年八月}
\thanks{这两个电报是毛泽东为第十八集团军总司令写的。当时蒋介石政府,在日本侵略者宣布投降但尚未实行投降之际,在美国帝国主义的武力援助下,垄断接受日本投降的权利,并且借口受降调运大军向解放区进逼,积极准备反革命内战。毛泽东写第一个电报的目的,就在于揭露蒋介石的反革命面目,教育全国人民警惕蒋介石的内战阴谋。在第二个电报里,进一步揭穿了蒋介石集团准备内战的阴谋,并提出了中国共产党关于制止内战的六项主张。为着同样的目的,毛泽东还为新华社写了两篇评论,即本卷\mxart{蒋介石在挑动内战}和\mxart{评蒋介石发言人谈话}。由于中国共产党采取了这种决不被蒋介石的反动气焰所吓倒的坚定的果断的立场,就使解放区和解放军得到了迅速的扩大,并且使蒋介石在国内外反对中国内战的强大政治压力之下,不得不改变策略,装出和平姿态,邀请毛泽东到重庆举行和平谈判。}
\maketitle


\section{一 八月十三日的电报}

我们从重庆广播电台收到中央社两个消息,一个是你给我们的命令,一个是你给各战区将士的命令。在你给我们的命令上说:“所有该集团军所属部队,应就原地驻防待命。”此外,还有不许向敌人收缴枪械一类的话。你给各战区将士的命令,据中央社重庆十一日电是这样说的:“最高统帅部今日电令各战区将士加紧作战努力,一切依照既定军事计划与命令积极推进,勿稍松懈。”我们认为这两个命令是互相矛盾的。照前一个命令,“驻防待命”,不进攻了,不打仗了。现在日本侵略者尚未实行投降,而且每时每刻都在杀中国人,都在同中国军队作战,都在同苏联、美国、英国的军队作战,苏美英的军队也在每时每刻同日本侵略者作战,为什么你叫我们不要打了呢?照后一个命令,我们认为是很好的。“加紧作战,积极推进,勿稍松懈”,这才像个样子。可惜你只把这个命令发给你的嫡系军队,不是发给我们,而发给我们的另是一套。朱德在八月十日下了一个命令给中国各解放区的一切抗日军队\mnote{1},正是“加紧作战”的意思。再有一点,叫他们在“加紧作战”时,必须命令日本侵略者投降过来,将敌、伪军的武装等件收缴过来。难道这样不是很好的吗?无疑这是很好的,无疑这是符合于中华民族的利益的。可是“驻防待命”一说,确与民族利益不符合。我们认为这个命令你是下错了,并且错得很厉害,使我们不得不向你表示:坚决地拒绝这个命令。因为你给我们的这个命令,不但不公道,而且违背中华民族的民族利益,仅仅有利于日本侵略者和背叛祖国的汉奸们。

\section{二 八月十六日的电报}

在我们共同敌人——日本政府已接受波茨坦公告\mnote{2}宣布投降,但尚未实行投降之际,我代表中国解放区、中国沦陷区一切抗日武装力量和二亿六千万人民,特向你提出下列的声明和要求。

在抗日战争将要胜利结束的时候,我提起你注意目前中国战场上的这样的事实,即在敌伪侵占而为你所放弃的广大沦陷地区中,违背你的意志,经过我们八年的苦战,夺回了近百万平方公里的土地,解放了过一万万的人民,组织了过一百万的正规部队和二百二十多万的民兵,在辽宁、热河、察哈尔、绥远\mnote{3}、河北、山西、陕西、甘肃、宁夏、河南、山东、江苏、安徽、湖北、湖南、江西、浙江、福建、广东十九个省区内建立了十九个大块的解放区\mnote{4},除少数地区外,大部包围了自一九三七年七七事变以来敌伪所侵占的中国城镇、交通要道和沿海海岸。此外,我们还在中国沦陷区(在这里,有一亿六千万人口)中组织了广大的地下军,打击敌伪。在作战中,我们至今还抗击和包围着侵华(东北不在内)日军的百分之六十九和伪军的百分之九十五。而你的政府和军队,却一向采取袖手旁观、坐待胜利、保存实力、准备内战的方针,对于我们解放区及其军队,不仅不予承认,不予接济,而且以九十四万大军包围和进攻它们。中国解放区全体军民虽受尽了敌伪和你的军队两方面夹击之苦,但丝毫未减弱他们坚持抗战、团结和民主的意志。中国解放区人民和中国共产党,曾经多次向你和你的政府提议召开各党派会议,成立民主的举国一致的联合政府,以便停止内部纷争,动员和统一全中国人民的抗日力量,领导抗日战争取得胜利,保证战后的和平,但都被你和你的政府所拒绝。凡此一切,我们是非常之不满意的。

现在敌国投降将要签字了,而你和你的政府仍然漠视我们的意见,并且于八月十一日下了一个非常无理的命令给我,又命令你的军队以收缴敌人枪械为借口大举向解放区压迫,内战危机空前严重。凡此种种,使得我们不得不向你和你的政府提出下列的要求:

一、你和你的政府及其统帅部,在接受日伪投降、缔结受降后的一切协定和条约的时候,我要求你事先和我们商量,取得一致意见。因为你和你的政府为人民所不满,不能代表中国解放区、中国沦陷区的广大人民和一切抗日的人民武装力量。如果协定和条约中,有涉及中国解放区、中国沦陷区一切抗日的人民武装力量之处,而未事先取得我们同意的时候,我们将保留自己的发言权。

二、中国解放区、中国沦陷区的一切抗日的人民武装力量,有权根据波茨坦公告和同盟国规定的受降办法\mnote{5},接受我们所包围的日伪军队的投降,收缴其武器资材,并负责实施同盟国在受降后的一切规定。我在八月十日下了一道命令给中国解放区军队,叫他们努力进击敌军,并准备接受敌人投降。八月十五日,我已下令给敌军统帅冈村宁次,叫他率部投降\mnote{6},但这只限于解放区军队作战的范围内,并不干涉其它区域。我的这些命令,我认为是非常合理、非常符合中国和同盟国的共同利益的。

三、中国解放区、中国沦陷区的广大人民和一切抗日武装力量,应有权派遣自己的代表参加同盟国接受敌人的投降,和处理敌国投降后的工作。

四、中国解放区和一切抗日武装力量,应有权选出自己的代表团,参加将来关于处理日本的和平会议和联合国会议。

五、请你制止内战。其办法就是:凡被解放区军队所包围的敌伪军由解放区军队接受其投降,你的军队则接受被你的军队所包围的敌伪军的投降。这不但是一切战争的通例,尤其是为了避免内战,必须如此。如果你不这样做,势将引起不良后果。关于这一点,我现在向你提出严重的警告,请你不要等闲视之。

六、请你立即废止一党专政,召开各党派会议,成立民主的联合政府,罢免贪官污吏和一切反动分子,惩办汉奸,废止特务机关,承认各党派的合法地位(中国共产党和一切民主党派至今被你和你的政府认为是非法的),取消一切镇压人民自由的反动法令,承认中国解放区的民选政府和抗日军队,撤退包围解放区的军队,释放政治犯,实行经济改革和其它各项民主改革。

此外,我在八月十三日发了一个电报给你,回答你在八月十一日给我的命令,谅你已经收到了。我这里重复声言,你那个命令是完全错误的。你在八月十一日叫我的军队“就原地驻防待命”,不打敌人了。但是不但在八月十一日,就是在今天(八月十六日),日本政府还只在口头上宣布投降,并没有在事实上投降,投降协定尚未签字,投降事实尚未发生。我的这个意见,和英美苏各同盟国的意见是完全一致的。就在你下命令给我的那一天(八月十一日),缅甸前线英军当局宣布:“对日战争仍在进行中。”美军统帅尼米兹\mnote{7}宣布:“不仅战争状态是存在的,而且具有一切毁灭结果的战争,必须继续进行。”苏联远东红军宣布:“敌人必须粉碎,不要留情。”八月十五日,红军总参谋长安东诺夫大将还作了下列声明:“八月十四日日皇所发表的日本投降声明,仅仅是无条件投降的一般宣言,给武装部队关于停止敌对行动的命令尚未发布,而且日本军队还在继续进行抵抗。因此,日本实际投降尚未发生。我们只有在日皇命令其军队停止敌对行为和放下武器,而且这个命令被实际执行的时候,才承认日本军队投降了。鉴于上述各点,远东苏军将继续进行对日攻势作战。”由此看来,一切同盟国的统帅中,只有你一个人下了一个绝对错误的命令。我认为你的这个错误,是由于你的私心而产生的,带着非常严重的性质,这就是说,你的命令有利于敌人。因此,我站在中国和同盟国的共同利益的立场上,坚决地彻底地反对你的命令,直至你公开承认错误,并公开收回这个错误命令之时为止。我现在继续命令我所统帅的军队,配合苏联、美国、英国的军队,坚决向敌人进攻,直至敌人在实际上停止敌对行为、缴出武器,一切祖国的国土完全收复之时为止。我向你声明:我是一个爱国军人,我不能不这样做。

以上各项,我请你早日回答。


\begin{maonote}
\mnitem{1}见本卷\mxnote{蒋介石在挑动内战}{1}。
\mnitem{2}见本卷\mxnote{蒋介石在挑动内战}{2}。
\mnitem{3}热河、察哈尔,见本卷\mxnote{抗日战争胜利后的时局和我们的方针}{11}。绥远,原来也是一个省,一九五四年撤销,原辖地区划归内蒙古自治区。
\mnitem{4}十九个大块的解放区,指陕甘宁、晋绥、晋察冀、冀热辽、晋冀豫、冀鲁豫、山东、苏北、苏中、苏南、淮北、淮南、皖中、浙东、广东、琼崖、湘鄂赣、鄂豫皖、河南。
\mnitem{5}一九四五年八月十日,日本政府向苏、中、美、英四国请降。十一日,四国政府复文规定,“日本一切陆海空军当局以及彼等控制下之一切部队(不论其在何处)”,必须“停止积极活动,缴出武器”。
\mnitem{6}冈村宁次当时是日本的中国派遣军总司令官。朱德总司令给冈村宁次的命令如下:“(一)日本政府已正式接受波茨坦宣言条款宣布投降。(二)你应下令你所指挥下的一切部队,停止一切军事行动,听候中国解放区八路军、新四军及华南抗日纵队的命令,向我方投降,除被国民党政府的军队所包围的部分外。(三)关于投降事宜,在华北的日军,应由你命令下村定将军派出代表至八路军阜平地区,接受聂荣臻将军的命令;在华东的日军,应由你直接派出代表至新四军军部所在地天长地区,接受陈毅将军的命令;在鄂豫两省的日军,应由你命令在武汉的代表,至新四军第五师大悟山地区,接受李先念将军的命令;在广东的日军,应由你指定在广州的代表,至华南抗日纵队东莞地区,接受曾生将军的命令。(四)所有在华北、华东、华中及华南的日军(被国民党军队包围的日军在外),应暂时保存一切武器、资材,静候我军受降,不得接受八路军、新四军及华南抗日纵队以外之命令。(五)所有华北、华东之飞机、舰船,应即停留原地;但沿黄海、渤海之中国海岸的舰船,应分别集中于连云港、青岛、威海卫、天津。(六)一切物资设备,不得破坏。(七)你及你所指挥的在华北、华东、华中及华南的日军指挥官,对执行上述命令应负绝对的责任。”
\mnitem{7}尼米兹(一八八五——一九六六),美国海军上将。当时是美国太平洋舰队总司令兼太平洋战区总司令。
\end{maonote}
