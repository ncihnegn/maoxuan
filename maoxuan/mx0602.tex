
\title{工作方法六十条(草案)}
\date{一九五八年一月}
\thanks{这是毛泽东同志为中共中央起草的党内指示。}
\maketitle


我国人民,在共产党领导下,一九五六年在社会主义所有制方面取得了基本的胜利,一九五七年发动整风运动,又在思想战线和政治战线方面取得了基本的胜利,就在这一年,又超额地完成了第一个五年建设计划。这样,我国六亿多人民就在共产党的领导下,认清了自己的前途,自己的责任,打击了从资产阶级右派分子方面刮起来的反党反人民反社会主义的妖风,同时也纠正了和正在继续纠正党和人民自己从旧社会带来的和由于主观主义造成的一些缺点和错误。党是更加团结了,人民的精神状态是更加奋发了。党群关系大为改善,我们现在看见了从来没有看见过的人民群众在生产战线上这样高涨的积极性和创造性。全国人民为在十五年或者更多一点时间内在钢铁及其他主要工业生产品方面赶上或者超过英国这个口号所鼓舞。一个新的生产高潮已经和正在形成。为了适应这种情况,中央和地方党委的工作方法,有作某些改变的需要。这里所说的几十条,并不都是新的。有一些是多年积累下来的,有一些是新提出的。这是中央和地方同志一九五八年一月先后在杭州会议和南宁会议上共同商量的结果。这几十条,大部分是会议上同志们的发言启发了我,由我想了一想写成的;一部分是直接记录同志们的意见;有一个重要条文(关于规章制度)\mnote{1}是由刘少奇同志和地方同志商定而由他起草的;由我直接提出的只占一部分。这里讲的也不完全是工作方法,有一些是工作任务,有一些是理论原则,但是工作方法占了主要地位。我们现在的主要目的,是想在工作方法方面求得一个进步,以适应已经改变了的政治情况的需要。这几十条现在只是建议,还待征求意见。条文或者要减少,或者要增加,都还未定。请同志们加以研究,提出意见,以便修改,然后提交政治局批准,方能成为一个正式的内部文件。

毛泽东

一九五八年一月三十一日

(一)县以上各级党委要抓社会主义建设工作。这里有十四项:1、工业,2、手工业,3、农业,4、农村副业,5、林业,6、渔业,7、畜牧业,8、交通运输业,9、商业,10、财政和金融,11、劳动、工资和人口,12、科学,13、文教,14、卫生。

(二)县以上各级党委要抓社会主义工业工作。这里也有十四项:1、产量指标,2、产品质量,3、新产品试剂,4、新技术,5、先进定额,6、节约原材料,找寻和使用代用品,7、劳动组织、劳动保护和工资福利,8、成本,9、生产准备和流动资金,10、企业的分工和协作,11、供产销平衡,12、地质勘探,13、资源综合利用,14、设计和施工。这是初步拟定的项目,以后应该逐步形成工业发展纲要“四十条”。

(三)各级党委要抓社会主义农业工作。这里也有十四项:1、产量指标,2、水利,3、肥料,4、土壤,5、种子,6、改制(改变耕作制度,如扩大复种面积,晚改早,旱改水等),7、病虫害,8、机械化(新式农具,双轮双铧犁,抽水机、适合中国各个不同区域的拖拉机及用摩托开动的运输工具等),9、精耕细作,10、畜牧,11、副业,12、绿化,13、除四害,14、治疾病讲卫生。这是从农业发展纲要四十条中抽出来的十四个要点。四十条必须全部施行。抽出一些要点,目的在于有所侧重。纲举目张,全网自然提起来了。

(四)全面规划,几次检查,年终评比。这是三个重要方法。这样一来,全局和细节都被掌握了;可以及时总结经验,发扬成绩,纠正错误;又可以激励人心,大家奋进。

(五)五年看三年,三年看头年,每年看前冬。这是一个掌握时机的方法。时机上有所侧重,把握就更大了。

(六)一年至少检查四次。中央和省一级,每季要检查一次;下面各级按情形办理。重要的任务在没有走上轨道之前,要每月检查一次。这也是掌握时机的方法,是就一年内说的。

(七)如何评比?省和省比,市和市比,县和县比,社和社比,厂和厂比,矿和矿比,工地和工地比。可以订评比公约,也可以不订。农业比较易于评比。工业可以根据可比的条件评比,按产业系统评比。

(八)什么时候交计划?省、自治区、直属市、专区、县都要按照三个十四项订出计划。订计划时要有重点,不可在同一时间内百废俱兴。区、乡、社的计划内容主要就是农业十四项。项目可以根据当地情况有所增减。先订五年的计划,可以是粗线条的。一九五八年七月一日以前文卷。计划要逐级审查。为了便于比较,省委要在县、区、乡、社的计划中选一些最好的和少数最坏的送给中央审查。省和专区的计划都要按期交中央,一个也不能少。

(九)生产计划三本帐。中央两本帐,一本是必成的计划,这一本公布;第二本是期成的计划,这一本不公布。地方也有两本帐。地方的第一本就是中央的第二本,这在地方是必成的;第二本在地方是期成的。评比以中央的第二本帐为标准。

(十)从今年起,中央和省、市、自治区党委要着重抓工业,抓财金贸。一年要抓四次,主要是七月(或八月)、十一月、一月(上旬)三次。再不抓,十五年赶上英国的口号可能落空。要把工业部门和财贸部门的若干主要负责干部带到讨论地方工作的会场上去,中央的带到地方去,省、直属市和自治区的带到专区、市属区和县里去,许多在中央工作的同志和在地方工作的同志都有这种要求。

(十一)各地方的工业产值(包括中央下放的厂矿、原来的地方国营工业和手工业的产值,不包括中央直属厂矿的产值),争取在五年内,或者七年内,或者十年内,超过当地的农业产值。各省市对于这件事要立即着手订计划,今年七月一日以前订出来。主要的任务是使工业认真地为农业服务。大家要切实摸一下工业,做到心中有数。

(十二)在今后五年内,或者六年内,或者七年内,或者八年内,完成农业发展纲要四十条的规定。各省委、直属市委、自治区党委对于这个问题应当研究一下。就全国范围来看,五年完成四十条不能普遍作到,六年或者七年可能普遍作到,八年就更加有可能普遍作到。

(十三)十年决于三年,争取在三年内大部分地区的面貌基本改观。其他地区的时间可以略为延长。口号是:苦战三年。方法是:放手发动群众,一切经过试验。

(十四)反对浪费。在整风中,每个单位要以若干天功夫,来一次反浪费的鸣放整改。每个工厂、每个合作社、每个商店、每个机关、每个学校、每个部队都要进行一次认真的反浪费斗争。今后每年都要反一次浪费。

(十五)在我国的国民经济中,积累和消费的比例怎样才算恰当,这是一个关系我国经济发展速度的大问题,希望大家研究。

(十六)关于农业合作社的积累和消费的比例问题也需要研究。湖北同志有这样的意见:以一九五七年生产和分配的数字为基础,以后的增产部分四六分(即以四成分配给社员,六成作为合作社积累)、对半分、倒四六分(即以四成作为合作社积累,六成分配给社员)。如果生产和收入已经达到当地富裕中农的水平的,可以在经过鸣放辩论取得群众同意以后,增产的部分三七分(即以三成分配给社员,七成作为合作社积累),或者一两年内暂时不分,以便增加积累,准备生产大跃进。这个意见是否适当,请各地讨论。

(十七)集体经济和个体经济的矛盾需要解决,需要定出一个适当的比例。现在的情况是,有的地方,有些农家的收入中个体经济和集体经济的比例是倒四六、倒三七(即是家庭副业和经营自留地的收入,占到总收入的百分之六十、七十)。这种情况必然影响农民对于社会主义集体经济的积极性。这种情况应当改变。各省可以经过鸣放辩论,研究出控制的办法,对经济关系作适当调整,在鼓励农民生产积极性和全面发展生产的基础上,使农家的收入中个体经济和集体经济的比例,在几年内逐步达到三比七或者二比八(即是农民从合作社得到的收入,占家庭总收入的百分之七十或者八十)。

(十八)普遍推广试验田。这是一个十分重要的领导方法。这样一来,我党在领导经济方面的工作作风将迅速改观。在乡村是试验田,在城市可以抓先进的厂矿、车间、工区和工段。突破一点就可以推动全面。

(十九)抓两头带中间。这是一个很好的领导方法。任何一种情况都有两头,即是有先进和落后,中间的状态又总是占多数。抓住两头就把中间带动起来了。这是一个辩证的方法,抓两头,抓先进和落后,就是抓住了两个对立面。

(二十)组织干部和群众对先进经验的参观和集中地展览先进的产品和作法,是两项很好的领导方法。用这些方法可以提高技术水平,推广先进经验,鼓励互相竞赛。许多问题到实地一看就解决了。社和社、乡和乡、县和县、省和省之间,都可以组织互相参观。中央、省、市、专区和县都可以举办生产建设展览会。

(二十一)不断革命。我们的革命是一个接一个的。从一九四九年在全国范围内夺取政权开始,接着就是反封建的土地改革,土地改革一完成就开始农业合作化,接着又是私营工商业和手工业的社会主义改造。社会主义三大改造,即生产资料所有制方面的社会主义革命,在一九五六年基本完成,接着又在去年进行政治战线上和思想战线上的社会主义革命。这个革命在今年七月一日以前可以基本上告一段落。但是问题没有完结,今后一个相当长的时期内每年都要用鸣放整改的方法继续解决这一方面的问题。现在要来一个技术革命,以便在十五年或者更多一点的时间内赶上和超过英国。中国经济落后,物质基础薄弱,使我们至今还处在一种被动状态,精神上感到还是受束缚,在这方面我们还没有得到解放。要鼓一把劲。再过五年,就可以比较主动一些了;十年后将会更加主动一些;十五年后,粮食多了,钢铁多了,我们的主动就更多了。我们的革命和打仗一样,在打了一个胜仗之后,马上就要提出新任务。这样就可以使干部和群众经常保持饱满的革命热情,减少骄傲情绪,想骄傲也没有骄傲的时间。新任务压来了,大家的心思都用在如何完成新任务的问题上面去了。提出技术革命,就是要大家学技术,学科学。右派说我们是小知识分子,不能领导大知识分子。还有人说要对老干部实行“赎买”,给点钱,叫他们退休,因为老干部不懂科学,不懂技术,只会打仗、搞土改。我们一定要鼓一把劲,一定要学习并且完成这个历史所赋予我们的伟大的技术革命。这个问题要在干部中议一议,开个干部大会,议一议我们还有什么本领。过去我们有本领,会打仗,会搞土改,现在仅仅有这些本领就不够了,要学新本领,要真正懂得业务,懂得科学和技术,不然就不可能领导好。我在一九四九年所写的《论人民民主专政》里曾经说过:“严重的经济建设的任务摆在我们面前。我们熟习的东西有些快要闲起来了,我们不熟习的东西正在强迫我们去做。这就是困难。”“我们必须克服困难,我们必须学会自己不懂的东西。”时间过去了八年。这八年中,革命一个接着一个,大家的思想都集中在那些问题上,很多人来不及学科学,学技术。从今年起,要在继续完成政治战线上和思想战线上的社会主义革命的同时,把党的工作的着重点放到技术革命上去。这个问题必须引起全党注意。各级党委可以在党内事先酝酿,向干部讲清楚,但是暂时不要在报上宣传,到七月一日以后我们再大讲特讲,因为那时候基层整风已经差不多了,可以把全党的主要注意力移到技术革命上面去了。注意力移到技术方面,又可能忽略政治,因此必须注意把技术和政治结合起来。

(二十二)红与专、政治与业务的关系,是两个对立物的统一。一定要批判不问政治的倾向。一方面要反对空头政治家,另一方面要反对迷失方向的实际家。

政治和经济的统一,政治和技术的统一,这是毫无疑义的,年年如此,永远如此。这就是又红又专。将来政治这个名词还是会有的,但是内容变了。不注意思想和政治,成天忙于事务,那会成为迷失方向的经济家和技术家,很危险。思想工作和政治工作是完成经济工作和技术工作的保证,它们是为经济基础服务的。思想和政治又是统帅,是灵魂。只要我们的思想工作和政治工作稍为一放松,经济工作和技术工作就一定会走到邪路上去。

现在一方面有社会主义世界同帝国主义世界的严重的阶级斗争;另一方面,就我国内部来说,阶级还没有最后消灭,阶级斗争还是存在的。这两点必须充分估计到。同阶级敌人作斗争,这是过去政治的基本内容。但是,在人民有了自己的政权以后,这个政权同人民的关系,就基本上是人民内部的关系了,采用的方法不是压服而是说服。这是一种新的政治关系。这个政权只对人民中破坏正常社会秩序的犯法分子采取暂时的程度不同的压服手段,作为说服的辅助手段。在由资本主义到社会主义的过渡时期,人民中还隐藏一部分反社会主义的敌对分子,例如资产阶级右派分子,对这种人,我们基本上也是采取由群众鸣放辩论的方法去解决问题。只对严重反革命破坏分子采取镇压的手段。过渡时期完结、彻底消灭了阶级以后,单就国内情况来说,政治就完全是人民内部的关系。那时候,人和人之间的思想斗争、政治斗争和革命一定还是会有的,并且不可能没有。对立统一的规律,量变质变的规律,肯定否定的规律,永远地普遍地存在。但是斗争和革命的性质和过去不同,不是阶级斗争,而是人民内部的先进和落后之间的斗争,社会制度的先进和落后之间的斗争,科学技术的先进和落后之间的斗争。由社会主义过渡到共产主义是一场斗争,是一个革命。进到共产主义时代了,又一定会有很多很多的发展阶段,从这个阶段到那个阶段的关系必然是一种从量变到质变的关系。各种突变、“飞跃都是一种革命,都要通过斗争,“无冲突论”是形而上学的。

政治家要懂些业务。懂得太多有困难,懂得太少也不行,一定要懂得一些。不懂得实际的是假红,是空头政治家。要把政治和技术结合起来,农业方面是搞试验田,工业方面是抓先进典型、试用新技术、试制新产品。这些都是用的“比较”法,在相同的条件下,拿先进和落后比,促使落后赶上先进。先进和落后是矛盾的两个极端,“比较”是对立的统一。企业和企业之间,企业内部车间和车间、小组和小组、个人和个人之间,都是不平衡的。不平衡是普遍的客观规律。从不平衡到平衡,又从平衡到不平衡,循环不已,永远如此,但是每一循环都进到高的一级。不平衡是经常的,绝对的;平衡是暂时的,相对的。我国现在经济上的平衡和不平衡的变化,是在总的量变过程中许多部分的质变。若干年后,中国由农业国变成工业国,那时候将完成一个飞跃,然后再继续量变的过程。

评比不仅比经济、比生产、比技术,还要比政治,就是比领导艺术。看谁领导得比较好些。

(二十三)上层建筑一定要适合经济基础和生产力发展的需要。政府各部门所制订的各种规章制度是上层建筑的一部分。八年来积累起来的规章制度许多还是适用的,但是有相当一部分已经成为进一步提高群众积极性和发展生产力的障碍,必须加以修改,或者废除。在修改或者废除这些不合理的规章制度方面,最近一个时期,在群众中间,已经创造了许多先进经验,例如石景山发电厂改进职工福利待遇的办法,湘江机械制造厂改进职工宿舍制度的办法,江苏戚墅堰发电厂改进奖金的办法,广西省一级几个商业机关合并为一个机关,由总数二千四百人缩减为三百五十人、即减少七分之六的人员等。应该作出这样一个总的规定,即是在多快好省地按计划按比例地发展社会主义事业的前提下,在群众觉悟提高的基础上,允许并且鼓励群众的那些打破限制生产力发展的规章制度的创举。

中央各部门,各省、市、自治区党委,应该派遣负责同志到各地的基层单位去,总结群众中的这一类先进经验,发展下层单位和群众的这一类有利于社会主义建设的创举,建议主管机关给以批准,停止原有的规章制度中某些规定在这个单位实行,并且把这个单位的先进经验推广到其他单位试行。

中央各部门,各省、市、自治区党委,还应当派遣负责同志到各地的基层单位去,发现那里有些什么规章制度已经限制了群众积极性的提高和生产力的发展,根据那里的实际情况,通过基层党委和群众的鸣放辩论,保存现有规章制度中的合理部分,修改或者废除其中的不合理部分,并且拟定一些新的适合需要的规章制度,在这个单位实行,也可以推广到其他单位试行。

中央各部门,各省、市、自治区党委,应该系统地总结这方面的典型的成熟的先进经验。重大的和全国性的,经过党中央和国务院批准。地方性的,经过相应的地方党委和政府批准。技术性的和专业性的,经过主管部门批准。然后在全国或者全省的相同的所有单位中普遍推行。经过一段时间实行以后,在必要的时候,再根据新的经验修改或者重新制定各种规章制度。

这是制定和修改各种规章制度的群众路线的方法。

(二十四)一定要把整风坚持到底。全党要鼓起干劲,打掉官风,实事求是,同人民打成一片,尽可能地纠正一切工作上、作风上、制度上的缺点和错误。

(二十五)中央和省、直属市、自治区两级党委的委员,除了生病的和年老的以外,一年一定要有四个月的时间轮流离开办公室,到下面去作调查研究,开会,到处跑。应当采取走马看花、下马看花两种方法。哪怕到一个地方谈三、四小时就走也好。要和工人、农民接触,要增加感性知识。中央的有些会议可以到北京以外的地方去开,省委的有些会议可以到省会以外的地方去开。

(二十六)以真正平等的态度对待干部和群众。必须使人感到人们互相间的关系确实是平等的,使人感到你的心是交给他的。学习鲁迅。鲁迅的思想是和他的读者交流的,是和他的读者共鸣的。人们的工作有所不同,职务有所不同,但是任何人不论官有多大,在人民中间都要以一个普通劳动者的姿态出现。决不许可摆架子。一定要打掉官风。对于下级所提出的不同意见,要能够耐心听完,并且加以考虑,不要一听到和自己不同的意见就生气,认为是不尊重自己。这是以平等态度待人的条件之一。

(二十七)各级党委,特别是坚决站在中央正确路线方面的负责同志,要随时准备挨骂。人们骂得对的,我们应当接受和改正。骂得不对的,特别是歪风,一定要硬着头皮顶住,然后加以考查,进行批判。在这种情况下,决不可以随风倒,要有反潮流的大无畏的精神。这一点,我们已经在一九五七年受到了考验。

(二十八)在省、地、县三级或者在省、地、县、乡四级的干部会议上,讨论一次党的领导原则问题。讨论一下这些原则是否正确:“大权独揽,小权分散。党委决定,各方去办。办也有决,不离原则。工作检查,党委有责。”这几句话里,关于党委的责任,是说大事由它首先作出决定,并且在执行过程中加以检查。“大权独揽”是一句成语,习惯上往往指的是个人独断。我们借用这句话,指的却是主要权力应当集中于中央和地方党委的集体,用以反对分散主义。难道大权可以分揽吗?这八句歌诀,产生于一九五三年,就是为了反对那时的分散主义而想出来的。所谓“各方去办”,不是说由党员径直去办,而是一定要经过党员在国家机关中、在企业中、在合作社中、在人民团体中、在文化教育机关中,同非党员接触、商量、研究,对不妥当的部分加以修改,然后大家通过,方才去办。第三句话里所说的“原则”,指的是:党是无产阶级组织的最高形式,民主集中制,集体领导和个人作用的统一(党委和第一书记的统一),中央和上级的决策。

(二十九)是否事事都要问过第一书记?可以不必。大事一定要问。要有二把手、三把手,第一书记不在家的时候,要另外有人挂帅。

(三十)党委要抓军事。军队必须放在党委的领导和监督之下,现在基本上也正是这样做的,这是我军的优良传统。作军事工作的同志是要求中央和地方党委抓这项工作的。只是因为忙于社会改革和经济建设工作,近几年来我们抓得少了一些。现在应当改善这种情况。办法也是一年抓几次。

(三十一)大型会议、中型会议和小型会议,都是必需的,各地和各部门要好好安排一下。小型会议,参加的几个人,一、二十人,便于发现问题和讨论问题。上千人参加的大型会议,只能采取先作报告后加讨论的方法,这种会不能太多,每年两次左右。小型和中型会议每年至少要开四次。这种会最好到下面去开。省委可以到地委召开一个地区或者相近几个地区的县书会议。中央同志和国务院各部门可以轮番到地方开些小型会议。各个经济协作区有事就开会,每年至少开四次。

(三十二)开会的方法应当是材料和观点的统一。把材料和观点割断,讲材料的时候没有观点,讲观点的时候没有材料,材料和观点互不联系,这是很坏的方法。只提出一大堆材料,不提出自己的观点,不说明赞成什么反对什么,这种方法更坏。要学会用材料说明自己的观点。必须要有材料,但是一定要有明确的观点去统率这些材料。材料不要多,能够说明问题就行,解剖一个或者几个麻雀就够了,不需要很多。自己应当掌握丰富的材料,但是在会上只需要拿出典型性的。必须懂得,开会同写大著作是有区别的。

(三十三)一般说来,不要在几小时内使人接受一大堆材料,一大堆观点,而这些材料和观点又是人们平素不大接触的。一年要找几次机会,让那些平素不大接触本行事务的人们,接触本行事务,给以适合需要的原始材料或者半成品。不要在一个早上突如其来地把完成品摆在别人面前。要下些毛毛雨,不要在几小时内下几百公厘的倾盆大雨。“强迫受训”的制度必须尽可能废除,“强迫签字”的办法必须尽可能减少。要彼此有共同的语言,必须先有必要的共同的情报知识。

(三十四)十个指头的问题。人有十个指头,要使干部学会善于区别九个指头和一个指头,或者多数指头和少数指头。九个指头和一个指头有区别,这件事看来简单,许多人却不懂得。要宣传这种观点。这是大局和小局、一般和个别、主流和支流的区别。我们要注意抓住主流,抓错了一定翻跟斗。这是认识问题,也是逻辑问题。说一个和九个指头,这种说法比较生动,也比较合于我们工作的情况。我们的工作,除非发生了根本路线上的错误,成绩总是主要的。但是这种说法对于某些人却不适用,例如右派分子。许多极右分子,那是几乎十个指头都烂了;学生中的大部分普通右派分子也不只烂了一个指头,但又不是全烂了,所以还可以留在学校里。

(二十五)“攻其一点或几点,尽量夸大,不及其余。”这是一种脱离实际情况的形而上学的方法。一九五七年资产阶级右派分子向社会主义猖狂进攻,他们用的就是这种方法。我党在历史上吃过这种方法的大亏,这就是教条主义占统治地位的时期。立三路线也是如此。修正主义,或者右倾机会主义,也用这种方法。陈独秀路线和抗日时期的王明路线,就是如此。一九三四年,张国焘也用过这种方法。一九五三年高岗饶漱石反党联盟,用的也是这种方法。我们应当总结过去的经验,从认识论和方法论上加以批判,使干部觉醒起来,以免再吃大亏。好人犯个别错误的时候,也会不自觉地采用这种方法,所以好人也要研究方法论。

(三十六)概念的形成过程,判断的形成过程,推理的过程,就是调查和研究的过程,就是思维的过程。人脑是能够反映客观世界的,但是要反映得正确很不容易。要经过反复的考察,才能反映得比较正确,比较接近客观实际。有了正确的观点和正确的思想,还要有比较恰当的表达方式告诉别人。概念、判断的形成过程,推理的过程,就是“从群众中来”的过程;把自己的观点和思想传达给别人的过程,就是“到群众中去”的过程。在我们的干部中,大概还有不少的人,不明白这样一个简单的真理:任何英雄豪杰,他的思想、意见、计划、办法,只能是客观世界的反映,其原料或者半成品只能来自人民群众的实践中,或者自己的科学试验中,他的头脑只能作为一个加工工厂而起制成完成品的作用,否则是一点用处也没有的。人脑制成的这种完成品,究竟合用不合用,正确不正确,还得交由人民群众去考验。如果我们的同志不懂得这一点,那就一定会到处碰钉子。

(三十七)文章和文件都应当具有这样三种性质:准确性、鲜明性、生动性。准确性属于概念、判断和推理问题,这些都是逻辑问题。鲜明性和生动性,除了逻辑问题以外,还有词章问题。现在许多文件的缺点是:第一,概念不明确;第二,判断不恰当;第三,使用概念和判断进行推理的时候又缺乏逻辑性;第四,不讲究词章。看这种文件是一场大灾难,耗费精力又少有所得。一定要改变这种不良的风气。作经济工作的同志在起草文件的时候,不但要注意准确性,还要注意鲜明性和生动性。不要以为这只是语文教师的事情,大老爷用不着去管。重要的文件不要委托二把手、三把手写,要自己动手,或者合作起来做。

(三十八)不可以一切依赖秘书,或者“二排议员”。要以自己动手为主,别人帮助为辅。不要让秘书制度成为一般制度,不应当设秘书的人不许设秘书。一切依赖秘书,这是革命意志衰退的一种表现。

(三十九)学点自然科学和技术科学。

(四十)学点哲学和政治经济学。

(四十一)学点历史和法学。

(四十二)学点文学。

(四十三)学点文法和逻辑。

(四十四)建议在自愿的原则下,中央和省市的负责同志,学一种外国文。争取在五年到十年的时间内达到中等程度。

(四十五)中央和省的主要负责人,可以设置一名学习秘书。

(四十六)外来干部要学本地话,一切干部要学普通话。先订一个五年计划,争取学好,或者大体学好,至少学会一部分。在少数民族地区工作的汉族干部,必须学当地民族的语言。少数民族的干部,也应当学习汉语。

(四十七)中央各部,省、专区、县三级,都要比培养“秀才”。没有知识分子不行,无产阶级一定要有自己的秀才。这些人要较多地懂得马克思主义,又有一定的文化水平、科学知识、词章修养。

(四十八)一切中等技术学校和技工学校,凡是可能的,一律试办工厂或者农场,进行生产,作到自给或者半自给。学生实行半工半读。在条件许可的情况下,这些学校可以多招些学生,但是不要国家增加经费。

一切高等工业学校的可以进行生产的实验室和附属工场,除了保证教学和科学研究的需要以外,都应当尽可能地进行生产。此外,还可以由学生和教师同当地的工厂订立参加劳动的合同。

(四十九)一切农业学校除了在自己的农场进行生产,还可以同当地的农业合作社订立参加劳动的合同,并且派教师住到合作社去,使理论和实际结合。农业学校应当由合作社保送一部分合于条件的人入学。

农村里的中小学,都要同当地的农业合作社订立合同,参加农、副业生产劳动。农村学生还应当利用假期、假日或者课余时间回到本村参加生产。

(五十)大学校和城市里的中等学校,在可能条件下,可以由几个学校联合设立附属工厂或者作坊,也可以同工厂、工地或者服务行业订立参加劳动的合同。

一切有土地的大中小学,应当设立附属农场;没有土地而邻近郊区的学校,可以到农业合作社参加劳动。

(五十一)开展以除四害为中心的爱国卫生运动。今年要每月检查一次,以便打下基础。各地可以根据当地的情况,增加四害以外的其他内容。

(五十二)化肥工厂,中央、省、专区三级都可以设立。中央化工部门要帮助地方搞中小型化肥工厂的设计,中央机械部门要帮助地方搞中小型化肥工厂的设备。

(五十三)省、自治区、直属市,应当设立农具研究所,专门负责研究各种改良农具和中小型机械农具,同农具制造厂密切联系,研究好了就交付制造。

(五十四)湖北孝感县的联盟农业社,一部分土地每年种一造,亩产二千一百三十斤;四川仁寿县的前进农业社,一部分土地一造亩产一千六百八十斤;陕西宜君县的清河农业社,这个社在山区,一部分土地一造亩产一千六百五十四斤;广西百色县的拿坡农业社,一部分土地一造亩产一千六百斤。这些单季高产的经验,各地可以研究试行。

(五十五)种子搭配的问题(即是在一个地域内,一种作物要有几种品种,同时种植),各地可以进行研究。

(五十六)薯类大有用处。人吃,猪吃,牛吃,造酒,造糖,造粉。各地可以试制薯类粉,有控制地适当地推广薯类种植。

(五十七)绿化。凡能四季种树的地方,四季都种。能种三季的种三季。能种两季的种两季。

(五十八)陕西商雒专区每户种一升核桃。这个经验值得各地研究。可以经过鸣放辩论取得群众同意以后,将这个经验推广到种植果木、桑、柞、茶、漆、油料等经济林木方面去。

(五十九)林业要计算覆盖面积,算出各省、各专区、各县的覆盖面积比例,作出森林覆盖面积规划。

(六十)今年九月以前,要酝酿一下我不作中华人民共和国主席的问题。先在各级干部中间,然后在工厂和合作社中间,组织一次鸣放辩论,征求干部和群众的意见,取得多数人的同意。这是因为去掉共和国主席这个职务,专做党中央主席,可以节省许多时间做一些党所要求我做的事情。这样,对于我的身体状况也较为适宜。如果在辩论中群众发生抵触情绪,不赞成这个建议,可以向他们说明,在将来国家有紧急需要的时候,只要党有决定,我还是可以出任这种国家领导职务的。现在和平时期,以去掉一个主席职务较为有利。关于这个请求,已经得到中央政治局以及中央和地方许多同志的同意,认为这是一个好主意。\mnote{2}所有这些,请向干部和群众解释清楚,免除误会。

这次会议的传达方法。把这些观点逐渐和干部讲明,不要采取倾盆大雨的方式。

这次所谈的意见,都是建议性的。请同志们带回去讨论,可以推翻,可以发展,征求干部的意见。大约要有几个月才可能形成正式条文。

\begin{maonote}
\mnitem{1}指《工作方法六十条(草案)》中的第二十三条。
\mnitem{2}一九五八年十一月二十八日至十二月十日在武昌举行的中共八届六中全会通过决议,同意毛泽东提出的关于他不作下届中华人民共和国主席候选人的建议。
\end{maonote}
