
\title{同苏联驻华大使尤金的谈话\mnote{1}}
\date{一九五八年七月二十二日}
\thanks{这是毛泽东同志同苏联驻华大使尤金谈话的主要部分。}
\maketitle


昨天你们走了以后,我一直睡不着,也没有吃饭。今天请你们来谈谈,当个医生,下午就可以吃饭、睡觉了。你们很幸运,能够吃饭、睡觉。

我们言归正传,吹一吹昨天交谈的问题。就在这个房间里吹!我们之间没有紧张局势。我们是十个指头中,九个指头相同,一个指头不同。这个问题,我讲了两三次了你忘了没有?

昨天的问题我又想了一下,可能我有误会,也可能我是正确的,经过辩论可以解决。看来,关于海军提出的核潜艇的请求\mnote{2}可以撤销。这个问题我脑子里没有印象,问了他们才知道,海军司令部里有那么些热心人,就是苏联顾问,他们说苏联已经有了核潜艇,只要打个电报去,就可以给。

海军核潜艇是一门尖端科学,有秘密,中国人是毛手毛脚的,给了我们,可能发生问题。

苏联同志胜利了四十年,有经验。我们胜利才八年,没有经验,你们才提合营问题。所有制问题老早就提过,列宁就提过租让制\mnote{3},但那是对资本家的。

中国还有资本家,但国家是由共产党领导的。你们就是不相信中国人,只相信俄国人。俄国人是上等人,中国人是下等人,毛手毛脚的,所以才产生了合营的问题。要合营,一切都合营,陆海空军、工业、农业、文化、教育都合营,可不可以?或者把一万多公里长的海岸线都交给你们,我们只搞游击队。你们只搞了一点原子能,就要控制,就要租借权。此外,还有什么理由?

你们控制过旅顺、大连,后来走了。为什么控制?因为当时是国民党的中国。后来你们自动走了,因为是共产党领导的中国了。

在斯大林的压力下,搞了东北和新疆两处势力范围、四个合营企业\mnote{4}。后来,赫鲁晓夫\mnote{5}同志提议取消了,我们感谢他。

你们一直不相信中国人,斯大林很不相信。中国人被看作是第二个铁托\mnote{6},是个落后的民族。你们说欧洲人看不起俄国人,我看俄国人有的看不起中国人。

斯大林在最紧要的关头,不让我们革命,反对我们革命。在这一点上,他犯了很大的错误,与季诺维也夫\mnote{7}是一样的。

另外,我们对米高扬\mnote{8}不满意。他摆老资格,把我们看做儿子。他摆架子,可神气了。一九四九年他第一次来西柏坡的时候,架子就很大,后来又来了几次,都是这样。每次来都劝我去莫斯科,我说去干什么?他说,总会有事情做的。后来,还是赫鲁晓夫同志出了题目,去开会,搞个文件。

去庆祝十月革命四十周年,这是我们共同的事业。当时我说过,什么兄弟党,只不过是口头上说说,实际上是父子党,是猫鼠党。这一点,我在小范围内同赫鲁晓夫等同志谈过。他们承认。这种父子关系,不是欧洲式的,是亚洲式的。当时在场的有布尔加宁、米高扬、库西宁、苏斯洛夫\mnote{9}等人,还有你(指尤金)吗?中国方面,有我和邓小平\mnote{10}。

我对米高扬在我们八大\mnote{11}上的祝词不满意,那天我故意未出席,表示抗议。很多代表都不满意,你们不知道。他摆出父亲的样子,讲中国是俄国的儿子。

中国有它自己的革命传统,但中国革命没有十月革命也不能胜利,没有马克思列宁主义也不能胜利。

苏联的经验要学。普遍真理要遵守,这就是《莫斯科宣言》里所写的那九条\mnote{12}。要学习所有的经验,正确的经验要学,错误的经验也要学。错误的经验是:斯大林的形而上学、教条主义。他不完全是形而上学,有一部分辩证法,但大部分是形而上学。你们叫做个人崇拜,是一个东西。斯大林很爱摆架子。

我们支持苏联,但错误的东西不支持。关于和平过渡问题,我们没有公开谈,报上没有讲。我们很谨慎,也未公开批评你们,采取了内部交谈的办法。我去莫斯科以前,和你谈过。在莫斯科期间,由邓小平同志谈了五条\mnote{13}。今后,我们也不准备公开谈,因为这对赫鲁晓夫同志不利,应该巩固他的领导。我们不谈,并不是因为我们这些意见不是真理。

在国家关系上,我们两国是团结一致的。这连我们的敌人都承认,一直到现在都是这样。只要是不利于苏联的,我们都反对。帝国主义、修正主义对苏联的进攻,在大的问题上我们都反对。苏联也是这样做的。

苏联人从什么时候开始相信中国人的呢?从打朝鲜战争开始的。从那个时候起,两国开始合拢了,才有一百五十六项\mnote{14}。斯大林在世时是一百四十一项,后来赫鲁晓夫同志添了好多项。

我们对你们是没有秘密的。我们的军事、政治、经济、文化,你们都知道,你们有一千多个专家在我们这儿工作。我们相信你们,因为你们是社会主义国家,是列宁的后代。

但在我们的关系中,也有过问题,主要与斯大林有关。有三件事:第一,两次王明路线\mnote{15}。王明是斯大林的后代。第二,不要我们革命,反对我们革命。第三国际\mnote{16}已经解散了,还下命令,说你们不与蒋介石讲和、打内战的话,中国民族有灭亡的危险。然而我们并没有灭亡。第三,我第一次去莫斯科时,斯大林、莫洛托夫、贝利亚\mnote{17}就向我进攻。

为什么当时我请斯大林派一个学者来看我的文章?是不是我那样没有信心,连文章都要请你们来看?没有事情干吗?不是的,是请你们来中国看看,中国是真的马克思主义,还是半真半假的马克思主义。

你回去以后说了我们的好话。你对斯大林说的第一句话,就是“中国人是真正的马克思主义者”。但斯大林还是怀疑。只是到朝鲜战争时才改变了他的看法,也改变了东欧和其他各国兄弟党对我们的怀疑。

这种怀疑是必然的:“第一,你们反王明;第二,不要你们革命,你们非革命不可;第三,到莫斯科要斯大林订条约,要收回中长路,那么神气。”在莫斯科,科瓦廖夫\mnote{18}招待我,费德林\mnote{19}当翻译。我发了脾气,拍了桌子。我说,我在这儿有三个任务:一、吃饭;二、睡觉;三、拉屎。

军事学院有个苏联顾问,在讲战例的时候,只准讲苏联的,不准讲中国的,不准讲朝鲜战争的,只准讲苏军的十大打击。

让我们自己讲讲嘛!他连我们自己讲都不让。我们打了二十二年仗嘛!在朝鲜还打了三年嘛!请军委把这个材料搜集一下,交给尤金同志,如果他要的话。

有些事情我们没说,怕影响中苏关系,尤其是在波、匈事件的时候。当时波兰要赶走你们的专家,刘少奇同志在莫斯科建议你们撤走一部分,你们接受了,波兰人就高兴了,说他们有自由了。那时我们不能提专家问题,怕你们怀疑我们利用这个机会赶走专家。我们不赶,即使有十个波兰赶,我们也不赶。我们需要苏联的帮助。

我劝过波兰人,要学习苏联,劝他们在反教条主义以后,提出学习苏联的口号。学习苏联,对谁有利?对苏联有利,还是对波兰有利?这首先对波兰有利。

我们要学习苏联,但首先要考虑到我们自己的经验,以我们自己的经验为主。

有些苏联顾问,任职可以定个期限。如在我国军事、公安两个部门的首席顾问,一直没有个期限,换来换去,也不通知我们,也不征求我们的意见。好比说派大使吧,你尤金走了,派另外的人来,如果不和我们商量,能行吗?这种做法是不对的。你们派到我们公安部门的顾问,坐在那里,如果中国人不告诉他情况,他能知道个什么?

我劝你们去各省跑跑,与人民接触,多了解情况。我同你谈了多次,如果不是一万次,也有一千次了。

苏联专家中大部分人基本上是好的,个别人有些缺点。我们过去也有缺点,没有主动多向苏联同志介绍情况。现在要克服这些缺点,采取积极的态度。这次就向他们介绍中国的总路线,一次不成,两次;两次不成,三次、多次。

这些话,都是由于搞核潜艇“合作社”引起的。现在我们决定不搞核潜艇了,撤回我们的请求。要不然就把全部海岸线交给你们,把过去的旅顺、大连加以扩大。但是不要混在一起搞,你们搞你们的,我们搞我们的。我们总要有自己的舰队。两把手不好办。

打起仗来情况就不同了,你们的军队可以到我们这儿来,我们的军队也可以到你们那儿去。如果在我们这儿打,你们的军队也应该听我们的指挥。如果在你们那儿打,我们的军队比你们少的话,也应该听你们的指挥。

我这些话很不好听,你们可以说我是民族主义,又出现了第二个铁托。如果你们这样说,我就可以说,你们把俄国的民族主义扩大到了中国的海岸。

取消四个合营公司、撤销旅顺基地的是赫鲁晓夫同志。斯大林在世时,要在我们这儿搞罐头工厂。我回答他说,你们给我们设备,帮助我们建设,全部产品都给你们。赫鲁晓夫同志夸奖了我,说我回答得好。但为什么现在又搞海军“合作社”?你们建议搞海军“合作社”,怎么向全世界讲话?怎么向中国人民讲话?你们可以训练中国人,同帝国主义斗争,你们作顾问。否则,旅顺,不仅旅顺,可以租给你们九十九年。搞“合作社”有一个所有权问题,你们提出双方各占百分之五十。你们昨天把我气得一宿没有睡觉。他们(指在座的其他中国领导人)没有气,我一个人有气。如果犯错误,是我一个人。

(周恩来:这是我们政治局的一致意见。)

这次没谈通,可以再谈,可以每天向你谈一次。不行,我可以去莫斯科同赫鲁晓夫同志谈,或者请赫鲁晓夫同志来北京,把一切问题都谈清楚。

(彭德怀:今年苏联国防部长马利诺夫斯基同志给我打来一个电报,要求在中国海岸建设一个长波雷达观测站,用来在太平洋指挥潜艇舰队,需要的费用一亿一千万卢布,苏联负担七千万,中国负担四千万。)

这个问题和搞海军“合作社”一样,无法向人民讲,向国外讲,政治上不利。

(彭:彼得罗舍夫斯基\mnote{20},在作风上也很粗暴。他对我们的建军原则,对我们在个别地方不采用苏军条例,很不满。在一次军委扩大会议上,福建军区的叶飞\mnote{21}同志说,福建到处是山,苏军的练兵条例不完全适用,因为苏军条例主要是按平原的条件制定的。彼得罗舍夫斯基听了很不满意当时就说:“你污辱了伟大斯大林所创造的伟大的军事科学。”他这样一说,会场的气氛很紧张。)

上面这些事,有的过去讲了,有的没有讲。你们这样大力地帮助我们,而我们又讲你们的坏话,可能使你们难过。我们的关系,就好像教授与学生的关系。教授可能有缺点,学生是不是要提意见?要提,这不是要把教授赶走,教授还是好教授。

你们就帮助我们建造核潜艇嘛!你们可以作顾问。为什么要提出所有权各半的问题?这是一个政治问题。我们打算搞二三百艘这种潜艇。

要讲政治条件,连半个指头都不行。你可以告诉赫鲁晓夫同志,如果讲条件,我们双方都不必谈。如果他同意,他就来,不同意,就不要来,没有什么好谈的,有半个小指头的条件也不成。

在这个问题上,我们可以一万年不要援助。但其他方面的合作还可以进行,决不会闹翻。我们还是始终一致地支持苏联。我们可以在房子里吵架。

我在莫斯科时同赫鲁晓夫同志谈过,你们不一定满足我们的一切要求。你们不给援助,可以迫使我们自己努力。满足一切要求,反而对我们不利。

政治上的合作很重要。在政治上,我们拆你们的台,你们不好办;你们拆我们的台,我们也不好办。

战时,我们的一切军港、一切机场,你们都可以使用,一切地方你们都可以来。你们的地方,你们的海参崴,我们也可以去。战争结束了,就回来。关于这点,可以先订一个战时协定,不要等到战争开始时才订,要提前订。在协定里也要规定,我们也可以到你们那里去,即使我们不去,也要这样订,因为这是个平等问题。平时,这样做不行。平时你们帮助我们建立基地,建设军队。

搞海军“合作社”,就是在斯大林活着的时候,我们也不干。我在莫斯科也和他吵过嘛!

赫鲁晓夫同志取消了“合作社”\mnote{22},建立了信任。这次提所有权问题,使我想起斯大林的东西又来了。可能是我误会了,但话要讲清楚。

你昨天说,你们的条件不好,核潜艇不能充分发挥力量,没有前途,中国的条件好,海岸线长,等等。你们从海参崴经库页岛、千岛群岛出大洋,条件很好嘛!

你们讲的话,使我感到不愉快。请你照样告诉给赫鲁晓夫同志,我怎么说的,你就怎么讲,不要代我粉饰,好让他听了舒服。他批评了斯大林,现在又在搞斯大林的东西。

分歧还是有的。我们的,有的你们不同意;你们的,有的我们不同意。比如,我们的“人民内部矛盾”、“百花齐放”,你们就那么满意呀!

斯大林支持王明路线,使我们的革命力量损失了百分之九十以上。当革命处在关键的时候,他不让我们革命,反对我们革命。革命胜利后,他又不信任我们。他大吹自己,说什么中国的胜利是在他的理论指导下取得的。一定要彻底打破对他的迷信。斯大林对中国所做的这些事,我在死以前,一定写篇文章,准备一万年以后发表。

(尤金:对于中共的各项政策,我们苏共中央的态度是:中国问题怎样解决,是中国同志自己的事情,因为他们最了解情况。同时,我们认为,议论像中共这样伟大的党的政策是否正确,是轻率的、傲慢的。)

只能说是基本上正确。我自己也犯过错误,由于我的过错,在战争中也打过败仗,比如长沙、土城等四次战役。如果说我基本上是正确的,我就很高兴了。只能说我基本上正确是接近实际的。

建立潜艇舰队的问题,这是个方针问题:是我们搞你们帮助,还是搞“合作社”,这一定要在中国决定。赫鲁晓夫同志也可以来,因为我已经去过他那里了。

对于什么都不能迷信。比如,你们一位专家,根据一个院士的一本书,就说我们山西的煤不能炼焦。这样一来就完了,我们没有炼焦煤了,因为山西的煤最多嘛!

在长江大桥工作过的苏联专家西宁同志,是一个好同志。他的建桥方法,在你们国内一直没有用武之地。大型的不让他搞,让他搞个中型的嘛!中型的也不让他搞,让他搞个小型的嘛!小型的也不让搞。但是,他到我们这儿来一说,蛮有道理。反正我们什么也不懂,就请他搞吧!结果一试验就成功了,成了世界上第一流的科学工作。

我没有见过西宁同志。我和建设长江大桥的很多领导同志谈过话,他们一致反映:西宁是个好同志,一切工作他都亲自参加,工作方法很好,凡事都和中国同志一起做。大桥修好了,中国同志学会了很多东西。你们当中谁认识他,请代我向他问候。

不要在专家中,在两党和两国的关系中造成一种紧张气氛,我没有这个意思。我们的合作是全面的,是很好的。你要向使馆的工作人员和专家们讲清楚,不要说毛泽东同志提了意见,可不得了了。

有些问题早就想讲,但过去情况不好,发生了波、匈事件,你们政治上有困难,不宜于讲。比如专家问题,那时我们不好讲。

斯大林后来也很好了,中苏订了条约\mnote{23},帮助了朝鲜战争\mnote{24},搞了一百四十一项。当然,这不都是他个人的功绩,是整个苏共中央的功绩。因此,我们不强调斯大林的错误。

\begin{maonote}
\mnitem{1}一九五八年七月二十一日,尤金向毛泽东转达了赫鲁晓夫和苏共中央主席团关于苏联同中国建立一支共同核潜艇舰队的建议,并希望周恩来、彭德怀去莫斯科进行具体商量。毛泽东当即表示:“首先要明确方针:是我们办,你们帮助?还是只能合办,不合办你们就不给帮助,就是你们强迫我们合办?”
\mnitem{2}一九五八年六月二十八日,中国方面根据苏联军事顾问的意见,向苏联提出为发展中国海军核潜艇提供技术援助的要求。同年七月二十一日,苏联驻华大使尤金向毛泽东转达了赫鲁晓夫和苏共中央主席团关于苏联同中国建立一支共同核潜艇舰队的建议。由于苏方的这一建议有损中国的主权,中国方面撤销了请苏方就发展核潜艇提供技术援助的要求。
\mnitem{3}租让制,是苏俄政府在从资本主义到社会主义过渡时期采取的国家资本主义的一种形式。列宁在《论粮食税》一文中讲到租让制时说:“苏维埃制度下的租让是什么呢?这就是苏维埃政权、即无产阶级的国家政权为反对小私有者的(宗法式的和小资产阶级的)自发势力而和国家资本主义订立的一种合同、同盟或联盟。承租人就是资本家。他按资本主义方式经营,是为了获得利润,他同意和无产阶级政权订立合同,是为了获得高于一般利润的额外利润,或者是为了获得用别的办法得不到或极难得到的原料。苏维埃政权获得的利益,就是发展生产力,就是立刻或在最短期间增加产品数量。”“租让制这种国家资本主义,和苏维埃体系内其他形式的国家资本主义比较起来,大概是最简单、明显、清楚和一目了然的形式。在这里,我们和最文明先进的西欧资本主义直接订立正式的书面合同。我们确切知道自己的得失、自己的权利和义务,我们确切知道租让的期限,如果合同规定有提前赎回的权利,我们也确切知道提前赎回的条件。我们给世界资本主义一定的‘贡赋’,在某些方面向他们‘赎买’,从而立刻在某种程度上使苏维埃政权的地位得到加强,使我们经营的条件得到改善。”(《列宁选集》第4卷,人民出版社1995年版,第505、506页)
\mnitem{4}一九五〇年三月和一九五一年七月,中苏两国政府分别签订有关创办中苏股份公司的四个协定,在中国境内开办民用航空公司、石油公司、有色及稀有金属公司和造船公司。这四个合营企业的建立对当时中国的经济建设起了积极作用。但由于苏方企图把合营企业变为独立于中国主权之外的经济实体,在一些做法上损害了中国的权益。一九五四年十月十二日,中苏两国政府签署联合公报,苏方承诺于一九五五年一月一日前将四个中苏股份公司中的苏联股份出售给中国。苏联股份移交后,中苏民用航空公司由中国民用航空局接收,其余三个公司分别改名为新疆石油公司、新疆有色金属公司和大连造船厂。
\mnitem{5}赫鲁晓夫,时任苏联共产党中央委员会第一书记、苏联部长会议主席。
\mnitem{6}铁托,(一八九二——一九八〇),前南斯拉夫主要领导人和国际共产主义运动著名活动家,不结盟运动创始人之一。第二次世界大战中,曾领导南斯拉夫各族人民进行反法西斯民族解放战争。一九四五年创建南斯拉夫联邦人民共和国(一九六三年后改称南斯拉夫社会主义联邦共和国)。一九四八年六月二十八日,由保、罗、匈、波、苏、法、捷、意八国共产党和工人党代表参加的情报局会议,通过《关于南斯拉夫共产党状况的决议》,对南共进行公开的指责,并把南共开除出情报局。决议说:“以前用伪装形式存在的民族主义分子,在过去五六个月中,在南斯拉夫共产党的领导机关中取得了统治地位,因此,南斯拉夫共产党的领导机关就背离了南斯拉夫共产党的国际主义传统,走上了民族主义的道路。”。
\mnitem{7}季诺维也夫(一八八三——一九三六),十月革命前夕担任俄国社会民主工党(布尔什维克)中央政治局委员。因反对举行武装起义并泄露起义计划而受到列宁的严厉批评。后任彼得格勒苏维埃主席、共产国际执行委员会主席等职。一九二七年被开除出党。一九三六年在“肃反”中被处决。
\mnitem{8}米高扬(一八九五——一九七八),时任苏联共产党中央主席团委员、苏联部长会议副主席。
\mnitem{9}布尔加宁(一八九五——一九七五),一九五七年时任苏联共产党中央主席团委员、苏联部长会议主席。库四宁(一八八一一——一九六四),一九五七年时住苏联共产党中央主席团委员、中央书记处书记。苏斯洛夫(一九〇二——一九八二),一九五七年时任苏联共产党中央主席团委员、中央书记处书记。
\mnitem{10}一九五七年十一月二日至二十一日,毛泽东率中国党政代表团访问苏联,参加苏联十月革命四十周年庆祝大会,并出席各国共产党工人党莫斯科会议。邓小平当时是代表团成员、中共中央总书记。
\mnitem{11}八大,即中国共产党第八次全国代表大会。
\mnitem{12}《莫斯科宣言》,指一九五七年十一月十四日至十六日在莫斯科召开的社会主义国家共产党和工人党代表会议通过的《社会主义国家共产党和工人党代表会议宣言》。在这个《宣言》中,提出了普遍适用于各个走上社会主义道路的国家的九条共同的规律,即:以马克思列宁主义政党为核心的工人阶级,领导劳动群众进行这种形式或那种形式的无产阶级革命,建立这种形式或那种形式的无产阶级专政;建立无产阶级同农民基本群众和其他劳动阶层的联盟;消灭基本生产资料资本主义所有制和建立基本生产资料的公有制;逐步实现农业的社会主义改造;有计划地发展国民经济,以便建成社会主义和共产主义,提高劳动人民的生活水平;进行思想文化领域的社会主义革命,造成忠于工人阶级、劳动人民和社会主义事业的强大的知识分子队伍;消灭民族压迫,建立各民族间的平等和兄弟友谊;保卫社会主义果实,不让它受到国内外敌人的侵犯;实行无产阶级的国际主义,同各国工人阶级团结一致。
\mnitem{13}一九五七年莫斯科会议期间,十一月十日中共代表团由邓小平向苏共中央系统地说明中国共产党关于从资本主义向社会主义过渡问题的观点,并向苏共中央提出了《关于和平过渡问题的意见提纲》,提纲的主要内容有五条。
\mnitem{14}指我国第一个五年计划期间,由苏联援助中国建设的一百五十六项大中型工业项目。这些建设项目是中苏两国政府一九五〇年至一九五四年间经过反复协商后分批确定的,后调整为一百五十四项。一九六〇年由于苏联单方面废弃协议,实际进行施工的项目为一百五十项。
\mnitem{15}指王明“左”倾冒险主义错误和王明右倾机会主义错误。
\mnitem{16}第三国际,即共产国际。
\mnitem{17}莫洛托夫(一八九〇——一九八六),一九四九年时任苏联共产党中央政治局委员、苏联部长会议副主席。贝利亚,一九四九年时任苏联共产党中央政治局委员、苏联部长会议副主席。
\mnitem{18}科瓦廖夫,一九四九年十二月曾陪同毛泽东访问苏联,当时是苏联驻中国专家总负责人。
\mnitem{19}费德林,一九一二年生,苏联汉学家。长期在苏联外交部门担任中文翻译,曾任苏联驻中国大使馆文化参赞。
\mnitem{20}彼得罗舍夫斯基,曾受苏联政府派遣在中国任军事总顾问。
\mnitem{21}叶飞(一九一四——一九九九),福建南安人。一九五六年八月至一九五七年十月任福州军区司令员兼政治委员。
\mnitem{22}指中华人民共和国建立初期中苏两国合办的民用航空公司、石油公司、有色及稀有金属公司和造船公司。
\mnitem{23}指一九五〇年二月十四日在莫斯科签订的《中苏友好同盟互助条约》。
\mnitem{24}朝鲜战争,指抗美援朝战争。
\end{maonote}
