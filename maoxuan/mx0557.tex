
\title{在省市自治区党委书记会议上的讲话}
\date{一九五七年一月}
\maketitle


\section{一 一月十八日的讲话}

这次会议,要讨论的问题主要是三个:思想动向问题,农村问题,经济问题。今天我讲一讲思想动向问题。

思想动向问题,我们应当抓住。这里当作第一个问题提出来。现在,党内的思想动向,社会上的思想动向,出现了很值得注意的问题。

有一种问题是我们自己家里出的。比如,现在有些干部争名夺利,唯利是图。在评级过程中,有那样的人,升了一级不够,甚至升了两级还躺在床上哭鼻子,大概要升三级才起床。他这么一闹,就解决了一个问题,什么干部评级,根本不评了,工资大体平均、略有差别就是了。以前北洋军阀政府里有个内阁总理,叫唐绍仪,后头当了广东中山县的县长。旧社会的一个内阁总理可以去当县长,为什么我们的部长倒不能去当县长?我看,那些闹级别,升得降不得的人,在这一点上,还不如这个旧官僚。他们不是比艰苦,比多做工作少得享受,而是比阔气,比级别,比地位。这类思想在党内现在有很大的发展,值得我们注意。

农业合作化究竟是有希望,还是没有希望?是合作社好,还是个体经济好?,这个问题也重新提出来了。去年这一年,丰收的地方没有问题,重灾区也没有问题,就是那种灾而不重、收而不丰的合作社发生了问题。这类合作社,工分所值,原先许的愿大了,后头没有那么多,社员收入没有增加,甚至还有减少。于是议论就来了:合作社还好不好,要不要?这种议论也反映到党内的一些干部中间。有些干部说,合作社没有什么优越性。有些部长到乡下去看了一下,回到北京后,放的空气不妙,说是农民无精打采,不积极耕种了,似乎合作社大有崩溃灭亡之势。有些合作社社长抬不起头来,到处挨骂,上面批评,报纸上也批评。有些党委的宣传部长不敢宣传合作社的优越性。农业部的部长廖鲁言,又是党中央农村工作部的副部长,据他讲,他自己泄了气,他下面的负责干部也泄了气,横直是不行了,农业发展纲要四十条也不算数了。泄了气怎么办?这个事情好办,你没有气,给你打点气就是了。现在报纸上的宣传转了一下,大讲合作社的优越性,专讲好话,不讲坏话,搞那么几个月,鼓一点气。

前年反右倾,去年反“冒进”,反“冒进”的结果又出了个右倾。我说的这个右倾,是指在社会主义革命问题上,主要是在农村社会主义改造问题上的右倾。我们的干部中间刮起了这么一股风,象台风一样,特别值得注意。我们的部长、副部长、司局长和省一级的干部中,相当一部分人,出身于地主、富农和富裕中农家庭,有些人的老太爷是地主,现在还没有选举权。这些干部回到家里去,家里人就讲那么一些坏话,无非是合作社不行,长不了。富裕中农是一个动摇的阶层,他们的单干思想现在又在抬头,有些人想退社。我们干部中的这股风,反映了这些阶级和阶层的思想。

合作化一定能化好,但是一两年内不可能完全化好。要向党、政、军、民各界的同志们讲清楚。合作社只有这么一点历史,大多数合作社只有一年到一年半的历史,经验很少。搞了一辈子革命的人还会犯错误,人家只搞了一年到一年半,你怎么能要求他一点错误都不犯呢?一有点风,有点雨,就说合作化不行了,这种思想本身就是个大错误。事实上,多数合作社是办得好的和比较好的。只要拿出一个办得好的合作社,就可以把反对合作化的一切怪论打下去。为什么这个社能办好,别的社就不能办好?为什么这个社有优越性,别的社就没有优越性?你就到处大讲这个社的经验。一个省总可以找出这样一个典型嘛!要找那个条件最差,地势不好,过去产量很低,很穷的社,不要找那个本来条件就好的社。当然,你搞几十个也可以,但是,你只要搞好一个,就算胜利。

在学校里头也出了问题,好些地方学生闹事。石家庄一个学校,有一部分毕业生暂时不能就业,学习要延长一年,引起学生不满。少数反革命分子乘机进行煽动,组织示威游行,说是要夺取石家庄广播电台,宣布来一个“匈牙利”。他们贴了好多标语,其中有这样三个最突出的口号:“打倒法西斯!”“要战争不要和平!”“社会主义没有优越性!”照他们讲来,共产党是法西斯,我们这些人都要打倒。他们提出的口号那样反动,工人不同情,农民不同情,各方面的群众都不同情。北京清华大学,有个学生公开提出:“总有一天老子要杀几千几万人就是了!”百花齐放、百家争鸣一来,这一“家”也出来了。邓小平同志去这个大学讲了一次话,他说,你要杀几千几万人,我们就要专政。

我们高等学校的学生,据北京市的调查,大多数是地主、富农、资产阶级以及富裕中农的子弟,工人阶级、贫下中农出身的还不到百分之二十。全国恐怕也差不多。这种情况应当改变,但是需要时间。在一部分大学生中间,哥穆尔卡很吃得开,铁托、卡德尔也很吃得开。倒是乡下的地主、富农,城市里的资本家、民主党派,多数还比较守规矩,他们在波兰、匈牙利闹风潮的时候还没有闹乱子,没有跳出来说要杀几千几万人。对于他们的这个守规矩,应当有分析。因为他们没有本钱了,工人阶级、贫下中农不听他们的,他们脚底下是空的。如果天下有变,一个原子弹把北京、上海打得稀烂,这些人不起变化呀?那就难说了。那时,地主,富农,资产阶级,民主党派,都要分化。他们老于世故,许多人现在隐藏着。他们的子弟,这些学生娃娃们,没有经验,把什么“要杀几千几万人”、什么“社会主义没有优越性”这些东西都端出来了。

在一些教授中,也有各种怪议论,不要共产党呀,共产党领导不了他呀,社会主义不好呀,如此等等。他们有这么一些思想,过去没有讲,百家争鸣,让他们讲,这些话就出来了。电影《武训传》,你们看了没有?那里头有一枝笔,几丈长,象征“文化人”,那一扫可厉害啦。他们现在要出来,大概是要扫我们了。是不是想复辟?

去年这一年,国际上闹了几次大风潮。苏共二十次代表大会大反斯大林,这以后,帝国主义搞了两次反共大风潮,国际共产主义运动中也有两次大的辩论风潮。在这几次风潮中,欧洲美洲有些党受的影响和损失相当大,东方各国的党受的影响和损失比较小。苏共“二十大”一来,有些从前拥护斯大林非常积极的人,这时候也反得很积极。我看这些人不讲马克思列宁主义,对问题不作分析,也缺乏革命道德。马克思列宁主义也包括无产阶级的革命道德。你从前那么拥护,现在总要讲一点理由,才能转过这个弯来吧!理由一点不讲,忽然转这么一百八十度,好象老子从来就是不拥护斯大林的,其实从前是很拥护的。斯大林问题牵涉整个国际共产主义运动,各国党都牵涉到了。

对苏共“二十大”,我们党内绝大多数干部是不满意的,认为整斯大林整得太过了。这是一种正常的情绪,正常的反映。但是,也有少数人起了波动。每逢台风一来,下雨之前,蚂蚁就要出洞,它们“鼻子”很灵,懂得气象学。苏共“二十大”的台风一刮,中国也有那么一些蚂蚁出洞。这是党内的动摇分子,一有机会他们就要动摇。他们听了把斯大林一棍子打死,舒服得很,就摇过去,喊万岁,说赫鲁晓夫一切都对,老子从前就是这个主张。后头帝国主义几棍子,国际共产主义运动内部几棍子,连赫鲁晓夫的腔调都不得不有所改变,他们又摇过来了。大势所趋,不摇过来不行。墙上一南草,风吹两边倒。摇过来不是本心,摇过去才是本心。党内党外那些捧波、匈事件的人捧得好呀!开口波兹南,闭口匈牙利。这一下就露出头来了,蚂蚁出洞了,乌龟王八都出来了。他们随着哥穆尔卡的棍子转,哥穆尔卡说大民主,他们也说大民主。现在情况起了变化,他们不吭声了。不吭声不是本心,本心还是要吭声的。

台风一刮,动摇分子挡不住,就要摇摆,这是规律。我劝大家注意这个问题。有些人摇那么几次,取得了经验,就不摇了。有那么一种人,是永远要摇下去的,就象稻子那一类作物,因为秆子细,风一吹就要摇。高粱、玉米比较好些,秆子比较粗。只有大树挺立不拔。台风年年都有,国内国际的思想台风、政治台风也是年年都有。这是一种社会的自然现象。政党就是一种社会,是一种政治的社会。政治社会的第一类就是党派。党是阶级的组织。我们中国共产党是无产阶级政党,主要是由工人和半无产阶级的贫农出身的人组成的。但是,也有许多党员是地主、富农、资本家家庭出身,或者是富裕中农和城市小资产阶级出身。他们中间有相当多的人,虽然艰苦奋斗多少年,有所锻炼,但是马克思主义学得不多,在思想方面,精神方面,还是跟稻子一样,风一吹就要摇。

有些党员,过去各种关都过了,就是社会主义这一关难过。有这样典型的人,薛迅就是一个。她原来是河北省的省委副书记、副省长。她是什么时候动摇的呢?就是在开始实行统购统销的时候。统购统销是实行社会主义的一个重要步骤。她却坚决反对,无论如何要反对。还有一个,就是全国供销合作总社副主任孟用潜。他上书言事,有信一封,也坚决反对统购统销。实行农业合作化,党内也有人起来反对。总而言之,党内有这样的高级干部,他们过不了社会主义这一关,是动摇的。这类事情结束没有呢?没有。是不是十年以后这些人就坚定起来,真正相信社会主义呢?那也不一定。十年以后,遇到出什么问题,他们还可能说,我早就料到了的。

发给同志们一个材料,是反映某些军队干部的思想动向的。这些干部的意见中虽然有某些合理的部分,比如说有些干部的工资太高,农民看不惯,但是,他们的意见总的方向不妥,根本路线不对。他们批评我们党的政策是农村“左”了,城市右了。中国虽然有九百六十万平方公里,但是合共只有两块地方,一块叫农村,一块叫城市。照他们这一讲,都错了。

所谓农村政策“左”了,就是说农民收入不多,比工人少。这要有分析,不能光看收入。工人收入一般是比农民多,但是他们生产的价值比农民大,生活必需的支出也比农民多。农民生活的改善,主要依靠农民自己努力发展生产。政府也大力帮助农民,比如兴修水利,发放农贷,等等。我们的农业税,包括副业的税收,约占农民生产总值的百分之八,很多副业没有抽税。我们统购粮食,是按照正常的价格。国家在工业品和农业品交换中间从农民那里得到的利润也很少。我们没有苏联那种义务交售制度。我们对于工农业产品的交换是缩小剪刀差,而不是象苏联那样扩大剪刀差。我们的政策跟苏联大不相同。所以,不能说我们的农村政策“左”了。

在我们军队的高级干部中间,有些人可能是自己回家,或者是接了亲属来,听到富裕中农、富农、地主的那些话,受了触动,于是就替农民叫苦。一九五五年上半年,党内有相当多的人替农民叫苦,跟梁漱溟之流相呼应,好象只有他们这两部分人才代表农民,才知道农民的疾苦。至于我们党中央,在他们”看来,那是不代表农民的,省委也是不代表的,党员的大多数都是不代表的。江苏作了一个调查,有的地区,县区乡三级干部中间,有百分之三十的人替农民叫苦。后头一查,这些替农民叫苦的人,大多数是家里比较富裕,有余粮出卖的人。这些人的所谓“苦”,就是有余粮。所谓“帮助农民”、“关心农民”,就是有余粮不要卖给国家。这些叫苦的人到底代表谁呢?他们不是代表广大农民群众,而是代表少数富裕农民。

至于说城市政策右了,看起来也有点象,因为我们把资本家包了下来,还给他们七年的定息\mnote{1}。七年以后怎么办?到时候还要看。最好留个尾巴,还给点定息。出这么一点钱,就买了这样一个阶级。这个政策,中央是仔细考虑过的。资本家加上跟他们有联系的民主人士和知识分子,文化技术知识一般比较高。我们把这个阶级买过来,剥夺他们的政治资本,使他们无话可讲。剥的办法,一个是出钱买,一个是安排,给他们事做。这样,政治资本就不在他们手里,而在我们手里。我们要把他们的政治资本剥夺干净,没有剥夺干净的还要剥。所以,也不能说我们的城市政策右了。

我们的农村政策是正确的,我们的城市政策也是正确的。所以,象匈牙利事件那样的全国性大乱子闹不起来。无非是少数人这里闹一下,那里闹一下,要搞所谓大民主。大民主也没有什么可怕。在这个问题上,我跟你们不同,你们有些同志好象很怕。我说来一个大民主,第一不怕,第二要加以分析,看他讲什么,做什么。那些坏人在搞所谓大民主的时候,一定要做出错误的行动,讲出错误的话,暴露和孤立他们自己。“杀几千几万人”,是解决人民内部矛盾的方法吗?能得到大多数人同情吗?“打倒法西斯”,“社会主义没有优越性”,这不是公然违反宪法吗?共产党和共产党领导的政权是革命的,社会主义有优越性,这都是宪法里头讲了的,是全国人民公认的。“要战争不要和平”,那好呀!你来号召战争,统共那么几个人,你的兵就不够,军官也没有训练好。这些娃娃们发疯了!石家庄那个学校,把那三个口号一讨论,七十个代表,只有十几个人赞成,有五十几个人反对。然后,又把这几个口号拿到四千学生里头去讨论,结果都不赞成,这十几个人就孤立了。提出和坚持这几个口号的极反动分子,只有几个人。他们不搞什么大民主,不到处贴标语,还不晓得他们想干什么。他们一搞大民主,尾巴就被抓住了。匈牙利事件的一个好处,就是把我们中国的这些蚂蚁引出了洞。

在匈牙利,大民主一来,把党政军都搞垮了。在中国,这一条是不会发生的。几个学生娃娃一冲,党政军就全部瓦解,那除非我们这些人完全是饭桶。所以,不要怕大民主。出了乱子,那个脓包就好解决了,这是好事。帝国主义,我们从前不怕,现在也不怕。我们也从来不怕蒋介石。现在怕大民主?我看不要怕。如果有人用什么大民主来反对社会主义制度,推翻共产党的领导,我们就对他们实行无产阶级专政。

在知识分子问题上,现在有一种偏向,就是重安排不重改造,安排很多,改造很少。百花齐放、百家争鸣一来,不敢去改造知识分子了。我们敢于改造资本家,为什么对知识分子和民主人士不敢改造呢?

百花齐放,我看还是要放。有些同志认为,只能放香花,不能放毒草。这种看法,表明他们对百花齐放、百家争鸣的方针很不理解。一般说来,反革命的言论当然不让放。但是,它不用反革命的面貌出现,而用革命的面貌出现,那就只好让它放,这样才有利于对它进行鉴别和斗争。田里长着两种东西,一种叫粮食,一种叫杂草。杂草年年要锄,一年要锄几次。你说只要放香花,不要放毒草,那就等于要田里只能长粮食,不能长一根草。话尽管那样讲,凡是到田里看过的都知道,只要你不去动手锄,草实际上还是有那么多。杂草有个好处,翻过来就是肥料。你说它没有用?可以化无用为有用。农民需要年年跟田里的杂草作斗争,我们党的作家、艺术家、评论家、教授,也需要年年跟思想领域的杂草作斗争。所谓锻炼出来的,就是奋斗出来的。你草长,我就锄。这个对立面是不断出现的。杂草一万年还会有,所以我们也要准备斗争一万年。

总而言之,去年这一年是多事之秋,国际上是赫鲁晓夫、哥穆尔卡闹风潮的一年,国内是社会主义改造很激烈的一年。现在还是多事之秋,各种思想还要继续暴露出来,希望同志们注意。

\section{二 一月二十七日的讲话}

现在,我讲几点意见。

第一点,要足够地估计成绩。我们的革命和建设,成绩是主要的,缺点错误也有。有那么多成绩,夸大是不行的,但是估低了就要犯错误,可能要犯大错误。这个问题,本来八届二中全会已经解决了,这次会上还多次谈到,可见在一些同志思想上还没有解决。特别在民主人士里头有一种议论:“你们总是讲成绩是基本的,这不解决问题。谁不知道成绩是基本的,还有缺点错误呀!”但是,确实成绩是基本的。不肯定这一点,就泄气。对合作化就有泄气之事嘛!

第二点,统筹兼顾,各得其所。这是我们历来的方针。在延安的时候,就采取这个方针。一九四四年八月,《大公报》作社评一篇,说什么不要“另起炉灶”。重庆谈判期间,我对《大公报》的负责人讲,你那个话我很赞成,但是蒋委员长要管饭,他不管我们的饭,我不另起炉灶怎么办?那个时候,我们向蒋介石提出的一个口号,就是要各得其所。现在是我们管事了。我们的方针就是统筹兼顾,各得其所。包括把国民党留下来的军政人员都包下来,连跑到台湾去的也可以回来。对反革命分子,凡是不杀的,都加以改造,给生活出路。民主党派保留下来,长期共存,对它的成员给予安排。总而言之,全国六亿人口,我们统统管着。比如统购统销,一切城市人口和农村里头的缺粮户,我们都管。又比如城市青年,或者进学校,或者到农村去,或者到工厂去,或者到边疆去,总要有个安排。对那些全家没有人就业的,还要救济,总以不饿死人为原则。所有这些,都是统筹兼顾。这是一个什么方针呢?就是调动一切积极力量,为了建设社会主义。这是一个战略方针。实行这样一个方针比较好,乱子出得比较少。这种统筹兼顾的思想,要向大家说清楚。

柯庆施同志讲,要想尽一切办法。这个话很好,就是要想尽一切办法解决困难。这个口号应当宣传。我们现在遇到的困难不算很大,有什么了不起呀!比起万里长征,爬雪山过草地,总要好一点吧。长征途中,在过了大渡河以后,究竟怎么走呢?北面统是高山,人口又很少,我们那个时候提出要千方百计克服困难。什么叫千方百计呢?千方者,就是九百九十九方加一方,百计者,就是九十九计加一计。现在你们还没有提出几个方几个计来。各省、中央各部究竟有多少方多少计呀?只要想尽一切办法,困难是可以解决的。

第三点,国际问题。在中东,出了一个苏伊士运河事件。一个人,叫纳赛尔,把运河收归国有了;另外一个人,叫艾登,出一支兵去打;接着,第三个人,叫艾森豪威尔威尔,要赶走英国人,把这个地方霸起来。英国资产阶级历来老奸巨猾,是最善于在适当的时候作出妥协的一个阶级。现在它把中东搞到美国人手里去了。这个错误可大啦!这样的错误,在它历史上数得出多少呀?这一回为什么冲昏头脑犯这个错误呢?因为美国压得太凶,它沉不住气,想把中东夺回去,阻止美国。英国的矛头主要是对埃及的吗?不是。英国的文章是对付美国的,美国是对付英国的。

从这个事件可以看出当前世界斗争的重点。当然,帝国主义国家跟社会主义国家的矛盾是很厉害的矛盾,但是,他们现在是假借反共产主义之名来争地盘。争什么地盘呢?争亚洲非洲十亿人口的地盘。目前他们的争夺集中在中东这个具有重大战略意义的地区,特别是埃及苏伊士运河地区。在那里冲突的,有两类矛盾和三种力量。两类矛盾,一类是帝国主义跟帝国主义之间的矛盾,即美国跟英国、美国跟法国之间的矛盾,一类是帝国主义跟被压迫民族之间的矛盾。三种力量,第一种是最大的帝国主义美国,第二种是二等帝国主义英、法,第三种就是被压迫民族。现在帝国主义争夺的主要场所是亚洲非洲。在这些地区都出现了民族独立运动。美国采用的办法,有文的,也有武的,在中东就是这样。

他们闹,对我们有利。我们的方针应当是,把社会主义国家巩固起来,寸土不让。谁要我们让,就一定要跟他斗争。出了这个范围,让他们去闹。那末,我们要不要讲话呢?我们是要讲话的。对亚洲、非洲、拉丁美洲人民的反帝斗争,对各国人民的革命斗争,我们就是要支持。

帝国主义国家和我们之间,是你中有我,我中有你。我们支持他们那里的人民革命,他们在我们这里搞颠覆活动。他们里头有我们的人,就是那里的共产党,革命的工人、农民、知识分子,进步人士。我们里头有他们的人,拿中国来说,就是资产阶级中间和民主党派中间的许多人,还有地主阶级。现在这些人看起来还听话,还没有闹事。但是假使原子弹打到北京来了,他们怎么样?不造反呀?那就大成问题了。至于那些劳改犯,石家庄那个学校闹事的领袖人物,北京那个要杀几千几万人的大学生,就更不用说了。我们一定要把他们消化掉,要把地主、资本家改造成为劳动者,这也是一条战略方针。消灭阶级,要很长的时间。

总之,对于国际问题的观察,我们认为还是这样:帝国主义之间闹,互相争夺殖民地,这个矛盾大些。他们是假借跟我们的矛盾来掩盖他们之间的矛盾。我们可以利用他们的矛盾,这里很有文章可做。这是关系我们对外方针的一件大事。

讲一讲中美关系。我们在会上印发了艾森豪威尔威尔给蒋介石的信。我看这封信主要是给蒋介石泼冷水,然后又打点气。信上说需要冷静,不要冲动,就是说不要打仗,要靠联合国。这是泼冷水。蒋介石就是有那么一点冲动。打气,就是说要对共产党继续采取强硬的政策,还把希望寄托在我们出乱子上。在他看来,乱子已经出了,共产党是没有办法阻止它的。各有各的观察吧!

我还是这样看,迟几年跟美国建立外交关系为好。这比较有利。苏联跟美国建交,是在十月革命之后十七年。一九二九年爆发世界经济危机,持续到一九三三年。这一年,德国是希特勒上台,美国是罗斯福上台,这个时候,苏美才建交。我们跟美国建交,可能要在第三个五年计划完成以后,也就是说,要经过十八年或者更长的时间。我们也不急于进联合国,就同我们不急于跟美国建交一样。我们采取这个方针,是为了尽量剥夺美国的政治资本,使它处于没有道理和孤立的地位。不要我们进联合国,不跟我们建交,那末好吧,你拖的时间越长,欠我们的账就越多。越拖越没有道理,在美国国内,在国际舆论上,你就越孤立。我在延安就跟一个美国人讲过,你美国一百年不承认我们这个政府,一百零一年你还不承认,我就不信。总有一天,美国要跟我们建交。那时美国人跑进中国来一看,就会感到后悔无及。因为中国这个地方变了,房子打扫干净了,“四害”也除了,他们再找不到多少朋友了,散布一点细菌也没有多大作用了。

第二次世界大战以后,资本主义各国很不稳定,乱,人心不安。世界各国都不安,中国也在内。但是,我们总比他们安一点。你们研究一下看,在社会主义国家和帝国主义国家主要是美国之间,究竟谁怕谁?我说都怕。问题是谁怕谁多一点?我有这么一个倾向:帝国主义怕我们多一点。作这样的估计也许有个危险,就是大家都睡觉去了,一睡三天不醒。因此,总要估计到有两种可能性。除了好的可能性,还有一种坏的可能性,就是帝国主义要发疯。帝国主义是不怀好心的,总是要捣鬼的。当然,现在帝国主义要打世界大战也不那么容易,打起来的结果如何,他们要考虑。

再讲一讲中苏关系。我看总是要扯皮的,不要设想共产党之间就没有皮扯。世界上哪有不扯皮的?马克思主义就是个扯皮的主义,就是讲矛盾讲斗争的。矛盾是经常有的,有矛盾就有斗争。现在中苏之间就有那么一些矛盾。他们想问题做事情的方法,他们的历史习惯,跟我们不同。因此,要对他们做工作。我历来说,对同志要做工作。有人说,既然都是共产党员,就应当一样好,为什么还要做工作呀?做工作就是搞统一战线,做民主人士的工作,为什么还要做共产党员的工作呀?这种看法不对。共产党里头还是有各种不同的意见。有些人组织上进了党,思想上还没有通,甚至有些老干部跟我们的语言也不一致。所以,经常要谈心,要个别商谈或者集体商谈,要开多少次会,做打通思想的工作。

据我看,形势比一些人强,甚至比大官强。在形势的压迫下,苏联那些顽固分子还要搞大国沙文主义那一套,行不通了。我们目前的方针,还是帮助他们,办法就是同他们当面直接讲。这次我们的代表团到苏联去,就给他们捅穿了一些问题。我在电话里跟恩来同志说,这些人利令智昏,对他们的办法,最好是臭骂一顿。什么叫利呢?无非是五千万吨钢,四亿吨煤,八千万吨石油。这算什么?这叫不算数。看见这么一点东西,就居然胀满了一脑壳,这叫什么共产党员,什么马克思主义者!我说再加十倍,加一百倍,也不算数。你无非是在地球上挖了那么一点东西,变成钢材,做成汽车飞机之类,这有什么了不起!可是你把它当作那么大的包袱背在背上,什么革命原则都不顾了,这还不叫利令智昏!官做大了也可以利令智昏。当了第一书记,也是一种利,也容易使头脑发昏。昏得厉害的时候,就得用一种什么办法去臭骂他一顿。这回恩来同志在莫斯科就不客气了,跟他们抬杠子了,搞得他们也抬了。这样好,当面扯清楚。他们想影响我们,我们想影响他们。我们也没有一切都捅穿,法宝不一次使用干净,手里还留了一把。矛盾总是有的,目前只要大体过得去,可以求同存异,那些不同的将来再讲。如果他们硬是这样走下去,总有一天要统统捅出来。

在我们自己方面,对外宣传不要夸大。无论什么时候,都要谦虚谨慎,把尾巴夹紧一些。对苏联的东西还是要学习,但要有选择地学,学先进的东西,不是学落后的东西。对落后的东西是另一种学法,就是不学。他错误的东西,我们知道了,就可以避免犯那个错误。他那些对我们有用的东西一定要学。世界上所有国家的有益的东西,我们都要学。找知识要到各方面去找,只到一个地方去找,就单调了。

第四点,百花齐放,百家争鸣。这个方针,是在批判了胡风反革命集团之后提出来的,我看还是对的,是合乎辩证法的。

关于辩证法,列宁说过:“可以把辩证法简要地确定为关于对立统一的学说。这样就会抓住辩证法的核心,可是这需要解释和发展。”\mnote{2}解释和发展,这就是我们的工作。要解释,我们现在解释太少了。还要发展,我们在革命中有丰富的经验,应当发展这个学说。列宁还说:“对立的统一(一致、同一、均势),是有条件的、一时的、暂存的、相对的。互相排斥的对立的斗争则是绝对的,正如发展、运动是绝对的一样。”\mnote{3}从这种观点出发,我们提出了百花齐放、百家争鸣这个方针。

真理是跟谬误相比较,并且同它作斗争发展起来的。美是跟丑相比较,并且同它作斗争发展起来的。善恶也是这样,善事、善人是跟恶事、恶人相比较,并且同它作斗争发展起来的。总之,香花是跟毒草相比较,并且同它作斗争发展起来的。禁止人们跟谬误、丑恶、敌对的东西见面,跟唯心主义、形而上学的东西见面,跟孔子、老子、蒋介石的东西见面,这样的政策是危险的政策。它将引导人们思想衰退,单打一,见不得世面,唱不得对台戏。

在哲学里边,唯物主义和唯心主义是对立统一,这两个东西是相互斗争的。还有两个东西,叫做辩证法和形而上学,也是对立统一、相互斗争的。一讲哲学,就少不了这两个对子。苏联现在不搞对子,只搞“单干户”,说是只放香花,不放毒草,不承认社会主义国家中唯心主义和形而上学的存在。事实上,无论哪个国家,都有唯心主义,都有形而上学,都有毒草。苏联那里的许多毒草,是以香花的名义出现的,那里的许多怪议论,都戴着唯物主义或社会主义现实主义的帽子。我们公开承认唯物主义和唯心主义、辩证法和形而上学、香花和毒草的斗争。这种斗争,要永远斗下去,每一个阶段都要前进一步。

我劝在座的同志,你们如果懂得唯物主义和辩证法,那就还需要补学一点它的对立面唯心主义和形而上学。康德和黑格尔的书,孔子和蒋介石的书,这些反面的东西,需要读一读。不懂得唯心主义和形而上学,没有同这些反面的东西作过斗争,你那个唯物主义和辩证法是不巩固的。我们有些共产党员、共产党的知识分子的缺点,恰恰是对于反面的东西知道得太少。读了几本马克思的书,就那么照着讲,比较单调。讲话,写文章,缺乏说服力。你不研究反面的东西,就驳不倒它。马克思、恩格斯、列宁都不是这样。他们努力学习和研究当代的和历史上的各种东西,并且教人们也这么做。马克思主义的三个组成部分,是在研究资产阶级的东西,研究德国的古典哲学、英国的古典经济学、法国的空想社会主义,并且跟它们作斗争的过程中产生的。斯大林就比较差一些。比如在他那个时期,把德国古典唯心主义哲学说成是德国贵族对于法国革命的一种反动。作这样一个结论,就把德国古典唯心主义哲学全盘否定了。他否定德国的军事学,说德国人打了败仗,那个军事学也用不得了,克劳塞维茨\mnote{4}的书也不应当读了。

斯大林有许多形而上学,并且教会许多人搞形而上学。他在《苏联共产党(布)历史简明教程》中讲,马克思主义辩证法有四个基本特征。他第一条讲事物的联系,好像无缘无故什么东西都是联系的。究竟是什么东西联系呢?就是对立的两个侧面的联系。各种事物都有对立的两个侧面。他第四条讲事物的内在矛盾,又只讲对立面的斗争,不讲对立面的统一。按照对立统一这个辩证法的根本规律,对立面是斗争的,又是统一的,是互相排斥的,又是互相联系的,在一定条件下互相转化的。

苏联编的《简明哲学辞典》第四版关于同一性的一条,就反映了斯大林的观点。辞典里说:“像战争与和平、资产阶级与无产阶级、生与死等等现象不能是同一的,因为它们是根本对立和相互排斥的。”这就是说,这些根本对立的现象,没有马克思主义的同一性,它们只是互相排斥,不互相联结,不能在一定条件下互相转化。这种说法,是根本错误的。

在他们看来,战争就是战争,和平就是和平,两个东西只是互相排斥,毫无联系,战争不能转化到和平,和平不能转化到战争。列宁引用过克劳塞维茨的话:“战争是政治通过另一种手段的继续。”\mnote{5}和平时期的斗争是政治,战争也是政治,但用的是特殊手段。战争与和平既互相排斥,又互相联结,并在一定条件下互相转化。和平时期不酝酿战争,为什么突然来一个战争?战争中间不酝酿和平,为什么突然来一个和平?

生与死不能转化,请问生物从何而来?地球上原来只有无生物,生物是后来才有的,是由无生物即死物转化而来的。生物都有新陈代谢,有生长、繁殖和死亡。在生命活动的过程中,生与死也在不断地互相斗争、互相转化。

资产阶级与无产阶级不能转化,为什么经过革命,无产阶级变为统治者,资产阶级变为被统治者?比如,我们和蒋介石国民党就是根本对立的。对立双方互相斗争、互相排斥的结果,我们和国民党的地位都起了变化,他们由统治者变为被统治者,我们由被统治者变为统治者。逃到台湾去的国民党不过十分之一,留在大陆上的有十分之九。留下来的这一部分,我们正在改造他们,这是在新的情况下的对立统一到台湾去的那十分之一,我们跟他们还是对立统一,也要经过斗争转化他们。

对立面的这种斗争和统一,斯大林就联系不起来。苏联一些人的思想就是形而上学,就是那么硬化,要么这样,要么那样,不承认对立统一。因此,在政治上就犯错误。我们坚持对立统一的观点,采取百花齐放、百家争鸣的方针。在放香花的同时,也必然会有毒草放出来。这并不可怕,在一定条件下还有益。

有些现象在一个时期是不可避免的,等它放出来以后就有办法了。比如,过去把剧目控制得很死,不准演这样演那样。现在一放,什么《乌盆记》、《天雷报》,什么牛鬼蛇神都跑到戏台上来了。这种现象怎么样?我看跑一跑好。许多人没有看过牛鬼蛇神的戏,等看到这些丑恶的形象,才晓得不应当搬上舞台的东西也搬上来了。然后,对那些戏加以批判、改造,或者禁止。有人说,有的地方戏不好,连本地人也反对。我看这种戏演一点也可以。究竟它站得住脚站不住脚,还有多少观众,让实践来判断,不忙去禁止。

现在,我们决定扩大发行《参考消息》,从两千份扩大到四十万份,使党内党外都能看到。这是共产党替帝国主义出版报纸,连那些骂我们的反动言论也登。为什么要这样做呢?目的就是把毒草,把非马克思主义和反马克思主义的东西,摆在我们同志面前,摆在人民群众和民主人士面前,让他们受到锻炼。不要封锁起来,封锁起来反而危险。这一条我们跟苏联的做法不同。为什么要种牛痘?就是人为地把一种病毒放到人体里面去,实行“细菌战”,跟你作斗争,使你的身体里头产生一种免疫力。发行《参考消息》以及出版其它反面教材,就是“种牛痘”,增强干部和群众在政治上的免疫力。

对于一些有害的言论,要及时给予有力的反驳。比如《人民日报》登载的《说“难免”》那篇文章,说我们工作中的错误并不是难免的,我们是用“难免”这句话来宽恕我们工作中的错误。这就是一种有害的言论。这篇文章,似乎可以不登。既然要登,就应当准备及时反驳,唱一个对台戏。我们搞革命和建设,总难免要犯一些错误,这是历史经验证明了的。《再论无产阶级专政的历史经验》那篇文章,就是个大难免论。我们的同志谁愿意犯错误?错误都是后头才认识到的,开头都自以为是百分之百的马克思主义。当然,我们不要因为错误难免就觉得犯一点也不要紧。但是,还要承认工作中不犯错误确实是不可能的。问题是要犯得少一些,犯得小一些。

社会上的歪风一定要打下去。无论党内也好,民主人士中间也好,青年学生中间也好,凡是歪风,就是说,不是个别人的错误,而是形成了一股风的,一定要打下去。打的办法就是说理。只要有说服力,就可以把歪风打下去。没有说服力,只是骂几句,那股歪风就会越刮越大。对于重大问题,要作好充分准备,在有把握的时候,发表有充分说服力的反驳文章。书记要亲自管报纸,亲自写文章。

统一物的两个互相对立互相斗争的侧面,总有个主,有个次。在我们无产阶级专政的国家里,当然不能让毒草到处泛滥。无论在党内,还是在思想界、文艺界,主要的和占统治地位的,必须力争是香花,是马克思主义。毒草,非马克思主义和反马克思主义的东西,只能处在被统治的地位。在一定的意义上,这可以比作原子里面的原子核和电子的关系。一个原子分两部分,一部分叫原子核,一部分叫电子。原子核很小,可是很重。电子很轻,一个电子大约只有最轻的原子核的一千八百分之一。原子核也是可以分割的,不过结合得比较牢固。电子可有些“自由主义”了,可以跑掉几个,又来几个。原子核和电子的关系,也是对立统一,有主有次。从这样的观点看来,百花齐放百家争鸣,就是有益无害的了。

第五点,闹事问题。在社会主义社会里,少数人闹事,是个新问题,很值得研究。

社会上的事情总是对立统一的。社会主义社会也是对立统一的,有人民内部的对立统一,有敌我之间的对立统一。在我们的国家里还有少数人闹事,基本原因就在于社会上仍然有各种对立的方面——正面和反面,仍然有对立的阶级,对立的人们,对立的意见。

我们已经基本上完成了生产资料所有制的社会主义改造,但是还有资产阶级,还有地主、富农,还有恶霸和反革命。他们是被剥夺的阶级,现在我们压迫他们,他们心怀仇恨,很多人一有机会就要发作。在匈牙利事件发生的时候,他们希望把匈牙利搞乱,也希望最好把中国搞乱。这是他们的阶级本性。

有些民主人士和教授放的那些怪议论,跟我们也是对立的。他们讲唯心论,我们讲唯物论。他们说,共产党不能管科学,社会主义没有优越性,合作化坏得很;我们说,共产党能够管科学,社会主义有优越性,合作化好得很。

学生中间跟我们对立的人也不少。现在的大学生大多数是剥削阶级家庭出身的,其中有反对我们的人,毫不奇怪。这样的人北京有,石家庄有,其它地方也有。

社会上还有那样的人,骂我们的省委是“僵尸”。省委是不是僵尸?我看我们的省委根本就没有死,怎么僵呢?骂省委是“僵尸”跟我们说省委不是僵尸,也是对立的。

在我们党内,也有各种对立的意见。比如,对苏共“二十大”一棍子打死斯大林,就有反对和拥护两种对立的意见。党内的不同意见是经常发生的,意见刚刚一致,过一两个月,新的不同意见又出来了。

在人们的思想方法方面,实事求是和主观主义是对立的。我看那一年都会有主观主义。一万年以后,就一点主观主义都没有呀?我不相信。

一个工厂,一个合作社,一个学校,一个团体,一个家庭,总之,无论什么地方,无论什么时候,都有对立的方面。所以,社会上少数人闹事,年年都会有。

对于闹事,究竟是怕,还是不怕?我们共产党历来对帝国主义、蒋介石国民党、地主阶级、资产阶级都不怕,现在倒怕学生闹事,怕农民闹社,这才有点怪哩!对群众闹事,只有段祺瑞怕,蒋介石怕。此外,匈牙利和苏联也有些人怕。我们对于少数人闹事,应当采取积极态度,不应当采取消极态度,就是说不怕,要准备着。怕是没有出路的。越怕,鬼就越来。不怕闹,有精神准备,才不致陷于被动。我看要准备出大事。你准备出大事,就可能不出,你不准备出大事,乱子就出来了。

事情的发展,无非是好坏两种可能。无论对国际问题,对国内问题,都要估计到两种可能。你说今年会太平,也许会太平。但是,你把工作放在这种估计的基础上就不好,要放在最坏的基础上来设想。在国际,无非是打世界大战,甩原子弹。在国内,无非是出全国性的大乱子,出“匈牙利事件”,有几百万人起来反对我们,占领几百个县,而且打到北京来。我们无非再到延安去,我们就是从那个地方来的。我们已经在北京住了七年,第八年又请我们回延安怎么办?大家就呜呼哀哉,痛哭流涕?当然,我们现在并没有打算回延安,来个“虚晃一枪,回马便走”。“七大”的时候,我讲了要估计到十七条困难,其中包括赤地千里,大灾荒,没有饭吃,所有县城都丢掉。我们作了这样充分的估计,所以始终处于主动地位。现在我们得了天下,还是要从最坏的可能来设想。

发生少数人闹事,有些是由于领导上存在着官僚主义和主观主义,在政治的或经济的政策上犯了错误。还有一些不是政策不对,而是工作方法不对,太生硬了。再一个因素,是反革命分子和坏分子的存在。少数人闹事要完全避免是不可能的。这又是个难免论。但是,只要不犯大的路线错误,全国性的大乱子是不会出的。即使犯了大的路线错误,出了全国性的乱子,我看也会很快平息,不至于亡国。当然,如果我们搞得不好,历史走一点回头路,有点回归,这还是很可能的。辛亥革命就走了回头路,革掉了皇帝,又来了皇帝,来了军阀。有问题才革命,革了命又出问题。我相信,假如出一次全国性的大乱子,那时总会有群众和他们的领袖人物来收拾时局,也许是我们,也许是别人。经过那样一次大乱子,脓包破了以后,我们的国家只会更加巩固。中国总是要前进的。

对于少数人闹事,第一条是不提倡,第二条是有人硬要闹就让他闹。我们宪法上规定有游行、示威自由,没有规定罢工自由,但是也没有禁止,所以罢工并不违反宪法。有人要罢工,要请愿,你硬要去阻止,那不好。我看,谁想闹谁就闹,想闹多久就闹多久,一个月不够就两个月,总之没有闹够就不收场。你急于收场,总有一天他还是要闹。凡有学生闹事的学校,不要放假,硬是来它一场赤壁鏖兵。这有什么好处呢?就是把问题充分暴露出来,把是非搞清楚,使大家得到锻炼,使那些没有道理的人、那些坏人闹输。

要学会这么一种领导艺术,不要什么事情总是捂着。人家一发怪议论,一罢工,一请愿,你就把他一棍子打回去,总觉得这是世界上不应有之事。不应有之事为什么又有了呢?可见得是应有之事。你不许罢工,不许请愿,不许讲坏话,横直是压,压到一个时候就要变拉科西。党内、党外都是这样。各种怪议论,怪事,矛盾,以揭露为好。要揭露矛盾,解决矛盾。

对于闹事,要分几种情况处理。一种是闹得对的,我们应当承认错误,并且改正。一种是闹得不对的,要驳回去。闹得有道理,是应当闹的;闹得无道理,是闹不出什么名堂的。再有一种是闹得有对有不对的,对的部分我们接受,不对的部分加以批评,不能步步后退,毫无原则,什么要求都答应。除了大规模的真正的反革命暴乱必须武装镇压以外,不要轻易使用武力,不要开枪。段祺瑞搞的“三一八”惨案,就是用开枪的办法,结果把自己打倒了。我们不能学段祺瑞的办法。

对闹事的人,要做好工作,加以分化,把多数人、少数人区别开来。对多数人,要好好引导、教育,使他们逐步转变,不要挫伤他们。我看什么地方都是两头小中间大。要把中间派一步一步地争取过来,这样,我们就占优势了。对带头闹事的人,要有分析。有些人敢于带头闹,经过教育,可能成为有用之材。对少数坏人,除了最严重犯罪的以外,也不要捉,不要关,不要开除。要留在原单位,剥夺他的一切政治资本,使他孤立起来,利用他当反面教员。清华大学那个要杀几千几万人的大学生,我们邓小平同志去讲话,就请他当教员。这样的人,又没有武装,又没有手枪,你怕他干什么?你一下把他开除,你那里很干净了。但是不得人心。你这个地方开除了,他就要在别的地方就业。所以,急于开除这些人不是好办法。这种人代表反动的阶级,不是个别人的问题,简单处理,爽快是爽快,但是反面教员的作用没有尽量利用。苏联大学生闹事,他们就开除几个领袖人物,他们不懂得坏事可以当作教材,为我们所用。当然,对于搞匈牙利事件那样反革命暴乱的极少数人,就必须实行专政。

对民主人士,我们要让他们唱对台戏,放手让他们批评。如果我们不这样做,就有点象国民党了。国民党很怕批评,每次开参政会就诚惶诚恐。民主人士的批评也无非是两种:一种是错的,一种是不错的。不错的可以补足我们的短处;错的要反驳。至于梁漱溟、彭一湖、章乃器那一类人,他们有屁就让他们放,放出来有利,让大家闻一闻,是香的还是臭的,经过讨论,争取多数,使他们孤立起来。他们要闹,就让他们闹够。多行不义必自毙。他们讲的话越错越好,犯的错误越大越好,这样他们就越孤立,就越能从反面教育人民。我们对待民主人士,要又团结又斗争,分别情况,有一些要主动采取措施,有一些要让他暴露,后发制人,不要先发制人。

对资产阶级思想的斗争,对坏人坏事的斗争,是长期的,要几十年甚至几百年。工人阶级、劳动人民和革命知识分子,将在斗争中取得经验,受到锻炼,这是很有益处的。

坏事有两重性,一重是坏,一重是好。这一点,现在很多同志还不清楚。坏事里头包含着好的因素。把坏人坏事只看成坏,是片面地形而上学地观察问题,不是辩证地观察问题,不是马克思主义的观点。坏人坏事一方面是坏,另一方面有好的作用。比如,象王明这样的坏人,就起着反面教员的好作用。同样,好事里头也包含着坏的因素。比如,解放以后七年来的大胜利,特别是去年这一年的大胜利,使有些同志脑筋膨胀,骄傲起来了,突然来了个少数人闹事,就感到出乎意料之外。

对闹事又怕,又简单处理,根本的原因,就是思想上不承认社会主义社会是对立统一的,是存在着矛盾、阶级和阶级斗争的。

斯大林在一个长时期里不承认社会主义制度下生产关系和生产力之间的矛盾,上层建筑和经济基础之间的矛盾。直到他逝世前一年写的《苏联社会主义经济问题》,才吞吞吐吐地谈到了社会主义制度下生产关系和生产力之间的矛盾,说如果政策不对,调节得不好,是要出问题的。但是,他还是没有把社会主义制度下生产关系和生产力之间的矛盾,上层建筑和经济基础之间的矛盾,当作全面性的问题提出来,他还是没有认识到这些矛盾是推动社会主义社会向前发展的基本矛盾。他以为他那个天下稳固了。我们不要以为天下稳固了,它又稳固又不稳固。

按照辩证法,就象人总有一天要死一样,社会主义制度作为一种历史现象,总有一天要灭亡,要被共产主义制度所否定。如果说,社会主义制度是不会灭亡的,社会主义的生产关系和上层建筑是不会灭亡的,那还是什么马克思主义呢?那不是跟宗教教义一样,跟宣传上帝不灭亡的神学一样?

怎样处理社会主义社会的敌我矛盾和人民内部矛盾,这是一门科学,值得好好研究。就我国的情况来说,现在的阶级斗争,一部分是敌我矛盾,大量表现的是人民内部矛盾。当前的少数人闹事就反映了这种状况。如果一万年以后地球毁灭了,至少在这一万年以内,还有闹事的问题。不过我们管不着一万年那么远的事情,我们要在几个五年计划的时间内,认真取得处理这个问题的经验。

要加强我们的工作,改正我们的错误和缺点。加强什么工作呢?工、农、商、学、兵、政、党,都要加强政治思想工作。现在大家搞业务,搞事务,什么经济事务,文教事务,国防事务,党的事务,不搞政治思想工作,那就很危险。现在我们的总书记邓小平同志,亲自出马到清华大学作报告,也请你们大家都出马。中央和省市自治区党委的领导同志,都要亲自出马做政治思想工作。第二次世界大战以后,苏联共产党,东欧一些国家的党,不讲马克思主义的基本原则了。阶级斗争,无产阶级专政,党的领导,民主集中制,党与群众的联系,这些他们都不讲了,空气不浓厚了。结果出了个匈牙利事件。我们一定要坚持马克思主义的基本理论。每个省市自治区都要把理论工作搞起来,有计划地培养马克思主义的理论家和评论家。

要精简机构。国家是阶级斗争的工具。阶级不等于国家,国家是由占统治地位的阶级出一部分人(少数人)组成的。机关工作是需要一点人,但是越少越好。现在国家机构庞大,部门很多,许多人蹲在机关里头没有事做。这个问题要解决。第一条,必须减人;第二条,对准备减的人,必须作出适当安排,使他们都有切实的归宿。党、政、军都要这样做。

要到下面去研究问题。我希望中央的同志,各省市自治区、各部的主要负责同志都这样做。听说现在许多负责同志不下去了,这不好。中央机关苦得很,在这个地方一点知识也捞不到。你要找什么知识,蹲在机关里是找不到的。真正出知识的地方是工厂、合作社、商店。工厂怎么办,合作社怎么办,商店怎么办,在机关里是搞不清楚的。越是上层越没有东西。要解决问题,一定要自己下去,或者是请下面的人上来。第一不下去,第二不请下面的人上来,就不能解决问题。我建议,省市自治区党委书记兼一个县委书记,或者兼一个工厂或学校的党委书记,地委书记、县委书记也要兼一个下级单位的书记。这样可以取得经验,指导全局。

要密切联系群众。脱离群众,官僚主义,势必要挨打。匈牙利的领导人,没有调查研究,不了解群众情况,等到大乱子出来了,还不晓得原因在什么地方。现在我们有些部和省市自治区党委的领导,不了解群众的思想动态,有人酝酿闹事,酝酿暴动,根本不知道,出了事就措手不及。我们一定要引为鉴戒。中央的同志,各省市自治区、各部的主要负责同志,一年总要有一段时间到工厂、合作社、商店、学校等基层单位去跑一跑,进行调查研究,搞清楚群众的情况怎样,先进的、中间的、落后的各有多少,我们的群众工作做得如何,做到心中有数。要依靠工人阶级,依靠贫农下中农,依靠先进分子,总要有个依靠。这样,才有可能避免出匈牙利那样的事件。

第六点,法制问题。讲三条:一定要守法,一定要肃反,一定要肯定肃反的成绩。

一定要守法,不要破坏革命的法制。法律是上层建筑。我们的法律,是劳动人民自己制定的。它是维护革命秩序,保护劳动人民利益,保护社会主义经济基础,保护生产力的。我们要求所有的人都遵守革命法制,并不是只要你民主人士守法。

一定要肃反。没有完成肃反计划的,今年要完成,如果留下一点尾巴,明年一定要完成。有些单位进行过肃反,但是肃而不清,必须在斗争中逐步肃清。反革命不多了,这一点要肯定。在闹事的地方,广大群众是不会跟反革命跑的,跟反革命跑的只是部分的、暂时的。同时也要肯定,还有反革命,肃反工作没有完。

一定要肯定肃反的成绩。肃反的成绩是伟大的。错误也有,当然要严肃对待。要给做肃反工作的干部撑腰,不能因为一些民主人士一骂就软下来。你天天骂,吃了饭没有别的事做,专做骂人的事,那由你。我看越骂越好,我讲的这三条总是骂不倒的。

共产党不晓得挨了多少骂。国民党骂我们是“共匪”,别人跟我们通,就叫“通匪”。结果,还是“匪”比他们非“匪”好。自古以来,没有先进的东西一开始就受欢迎,它总是要挨骂。马克思主义、共产党从开始就是挨骂的。一万年以后,先进的东西开始也还是要挨骂的。

肃反要坚持,有反必肃。法制要遵守。按照法律办事,不等于束手束脚。有反不肃,束手束脚,是不对的。要按照法律放手放脚。

第七点,农业问题。要争取今年丰收。今年来一个丰收,人心就可以稳定,合作社就可以相当巩固。在苏联,在东欧一些国家,搞合作化,粮食总要减产多少年。我们搞了几年合作化,去年大搞一年,不但没有减产,而且还增产了。如果今年再来一个丰收,那在合作化的历史上,在国际共产主义运动的历史上,就是没有先例的。

全党一定要重视农业。农业关系国计民生极大要注意不抓粮食很危险。不抓粮食,总有一天要天下大乱。

首先,农业关系到五亿农村人口的吃饭问题,吃肉吃油问题,以及其它日用的非商品性农产品问题。这个农民自给的部分,数量极大。比如,去年生产了三千六百多亿斤粮食,商品粮包括公粮在内,大约是八百多亿斤,不到四分之一,四分之三以上归农民。农业搞好了,农民能自给,五亿人口稳定了。

第二,农业也关系到城市和工矿区人口的吃饭问题。商品性的农产品发展了,才能供应工业人口的需要,才能发展工业。要在发展农业生产的基础上,逐步提高农产品特别是粮食的商品率。有了饭吃,学校、工厂少数人闹事也不怕。

第三,农业是轻工业原料的主要来源,农村是轻工业的重要市场。只有农业发展了,轻工业生产才能得到足够的原料,轻工业产品才能得到广阔的市场。

第四,农村又是重工业的重要市场。比如,化学肥料,各种各样的农业机械,部分的电力、煤炭、石油,是供应农村的,铁路、公路和大型水利工程,也都为农业服务。现在,我们建立了社会主义的农业经济,无论是发展轻工业还是发展重工业,农村都是极大的市场。

第五,现在出口物资主要是农产品。农产品变成外汇,就可以进口各种工业设备。

第六,农业是积累的重要来源。农业发展起来了,就可以为发展工业提供更多的资金。

因此,在一定的意义上可以说,农业就是工业。要说服工业部门面向农村,支援农业。要搞好工业化,就应当这样做。

农业本身的积累和国家从农业取得的积累,在合作社收入中究竟各占多大比例为好?请大家研究,议出一个适当的比例来。其目的,就是要使农业能够扩大再生产,使它作为工业的市场更大,作为积累的来源更多。先让农业本身积累多,然后才能为工业积累更多。只为工业积累,农业本身积累得太少或者没有积累,竭泽而渔,对于工业的发展反而不利。

合作社的积累和社员收入的比例,也要注意。合作社要利用价值法则搞经济核算,要勤俭办社,逐步增加一点积累。今年如果丰收,积累要比去年多一点,但是不能太多,还是先让农民吃饱一点。丰收年多积累一点,灾荒年或者半灾荒年就不积累或者少积累一点。就是说,积累是波浪式的,或者叫作螺旋式的。世界上的事物,因为都是矛盾着的,都是对立统一的,所以,它们的运动、发展,都是波浪式的。太阳的光射来叫光波,无线电台发出的叫电波,声音的传播叫声波。水有水波,热有热浪。在一定意义上讲,走路也是起波的,一步一步走就是起波。唱戏也是起波的,唱完一句再唱第二句,没有一口气唱七八句的。写字也起波,写完一个字再写一个字,不能一笔写几百个字。这是事物矛盾运动的曲折性。

总之,要照辩证法办事。这是邓小平同志讲的。我看,全党都要学习辩证法,提倡照辩证法办事。全党都要注意思想理论工作,建立马克思主义的理论队伍,加强马克思主义理论的研究和宣传。要运用马克思主义的对立统一学说,观察和处理社会主义社会阶级矛盾和阶级斗争的新问题,观察和处理国际斗争中的新问题。


\begin{maonote}
\mnitem{1}定息是在我国社会主义改造过程中,国家对民族资产阶级的生产资料实行赎买政策的一种形式。一九五六年资本主义工商业全行业公私合营以后,国家按照资本家的资产,在一定时期内,每年付给他们固定息率的股息,叫做定息。定息仍然属于剥削的性质。
\mnitem{2}见列宁《黑格尔(逻辑学)一书摘要》。
\mnitem{3}见列宁《关于辩证法问题》。
\mnitem{4}克劳塞维茨(一七八〇——一八三一),著名的德国资产阶级军事理论家,主要著作有《战争论》。斯大林对克劳塞维茨的评论,参看斯大林《给拉辛同志的复信》。
\mnitem{5}见列宁《战争与革命》。
\end{maonote}
