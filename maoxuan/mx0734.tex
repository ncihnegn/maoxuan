
\title{论无产阶级专政下继续革命}
\date{一九六七年五月、十一月}
\thanks{这是毛泽东同志审定的有关无产阶级专政下继续革命理论论述。}
\maketitle


\section*{(一)\mnote{1}}

无产阶级专政条件下还要不要革命?

在社会主义社会里,特别是在生产资料所有制的社会主义改造基本上完成以后,还有没有阶级和阶级斗争?

社会上的一切阶级斗争是不是还集中在争夺政权的问题上?

无产阶级专政条件下还要不要革命?革谁的命?怎样进行革命?这一系列重大的理论问题,马克思、恩格斯当时不可能解决这些问题。列宁看到了无产阶级夺取政权以后,被打败的资产阶级甚至比无产阶级还要强大,时时企图复辟,同时小生产者不断生长新的资本主义和资产阶级,威胁无产阶级专政,因此要对付这些反革命威胁,并且战胜它,必须在长时期里强化无产阶级专政,舍此没有第二条路。但因列宁逝世过早,没来得及在实际上解决这些问题。斯大林是个伟大的马克思列宁主义者,他在实际上解决了很大一批钻进党内的反革命资产阶级代表人物,例如托洛茨基,季诺维也夫,加米涅夫,拉狄克,布哈林,李可夫之流。他的缺点是在理论上不承认在无产阶级专政整个历史时代社会上存在阶级和阶级斗争,革命的谁胜谁负没有最后解决,弄得不好,资产阶级就有复辟之可能。在他临死的前一年,他已觉察到了这一点,说是社会主义社会存在矛盾,弄得不好,可能使矛盾变成对抗性的。毛泽东同志充分注意了整个苏联历史的经验,在他的一系列伟大著作和指示中,在这个伟大的历史文件中,在他亲自发动和领导的无产阶级文化大革命的伟大实践中,正确地解决了这一系列问题。

现在的文化大革命,仅仅是第一次,以后还必然要进行多次,毛泽东同志近几年经常说,革命的谁胜谁负,要在一个很长的历史时期内才能解决。如果弄得不好,资本主义复辟将是随时可能的。全体党员,全国人民,不要以为有一二次、三四次文化大革命,就可以太平无事了。千万注意,决不可丧失警惕。

\section*{(二)\mnote{2}}

毛泽东同志关于无产阶级专政下继续革命的理论的要点是:

一、必须用马克思列宁主义的对立统一的规律来观察社会主义社会。毛泽东同志指出:“对立统一规律是宇宙的根本规律。”“矛盾是普遍存在的”,“事物内部的这种矛盾性是事物发展的根本原因”。在社会主义社会中,“有两类社会矛盾,这就是敌我之间的矛盾和人民内部的矛盾”。“敌我之间的矛盾是对抗性的矛盾。人民内部的矛盾,在劳动人民之间说来,是非对抗性的”。毛泽东同志告诉我们:必须“划分敌我和人民内部两类矛盾的界线”,“正确处理人民内部矛盾”,才能使无产阶级专政日益巩固和加强,使社会主义制度日益发展。

二、“社会主义社会是一个相当长的历史阶段。在社会主义这个历史阶段中,还存在着阶级、阶级矛盾和阶级斗争,存在着社会主义同资本主义两条道路的斗争,存在着资本主义复辟的危险性。”在生产资料所有制的社会主义改造基本完成以后,“阶级斗争并没有结束。无产阶级和资产阶级之间的阶级斗争,各派政治力量之间的阶级斗争,无产阶级和资产阶级之间在意识形态方面的阶级斗争,还是长时期的,曲折的,有时甚至是很激烈的。”为了防止资本主义复辟,为了防止“和平演变’,必须把政治战线和思想战线上的社会主义革命进行到底。

三、无产阶级专政下的阶级斗争,在本质上,依然是政权问题,就是资产阶级要推翻无产阶级专政,无产阶级则要大力巩固无产阶级专政。无产阶级必须在上层建筑其中包括各个文化领域中对资产阶级实行全面的专政。“我们对他们的关系绝对不是什么平等的关系,而是一个阶级压迫另一个阶级的关系,即无产阶级对资产阶级实行独裁或专政的关系,而不能是什么别的关系,例如所谓平等关系、被剥削阶级同剥削阶级的和平共处关系、仁义道德关系等等。”

四、社会上两个阶级、两条道路的斗争,必然会反映到党内来。党内一小撮走资本主义道路的当权派,就是资产阶级在党内的代表人物。他们“是一批反革命的修正主义分子,一旦时机成熟,他们就会要夺取政权,由无产阶级专政变为资产阶级专政”。我们要巩固无产阶级专政,就必须充分注意识破“睡在我们的身旁”的“赫鲁晓夫那样的人物”,充分揭露他们,批判他们,整倒他们,使他们不能翻天,把那些被他们篡夺了的权力坚决夺回到无产阶级手中。

五、无产阶级专政下继续进行革命,最重要的,是要开展无产阶级文化大革命。

“无产阶级文化大革命,只能是群众自己解放自己”。“要让群众在这个大革命运动中,自己教育自己”。就是说,这个无产阶级文化大革命,运用无产阶级专政下的大民主的方法,自下而上地放手发动群众,同时,实行无产阶级革命派的大联合,实行革命群众、人民解放军和革命干部的革命三结合。

六、无产阶级文化大革命在思想领域中的根本纲领是“斗私,批修”。“无产阶级要按照自己的世界观改造世界,资产阶级也要按照自己的世界观改造世界。”因此,无产阶级文化大革命是触及人们灵魂的大革命,是要解决人们的世界观问题。要在政治上、思想上、理论上批判修正主义,用无产阶级的思想去战胜资产阶级利己主义和一切非无产阶级思想,改革教育,改革文艺,改革一切不适应于社会主义经济基础的上层建筑,挖掉修正主义的根子。

\begin{maonote}
\mnitem{1}本段是毛泽东同志对《伟大的历史文件》修改的文字。《伟大的历史文件》是《红旗》杂志、《人民日报》为一九六七年五月十七日公开发表一九六六年五月十六日《中国共产党中央委员通知》(又称“五·一六通知”)而写的编辑部文章。
\mnitem{2}本段是毛泽东同志审定的《纪念伟大的十月社会主义革命五十周年》的文字。《沿着十月社会主义革命开辟的道路前进——纪念伟大的十月社会主义革命五十周年》是《人民日报》、《红旗》杂志、《解放军报》编辑部在一九六七年十一月六日发表的文章。
\end{maonote}
