
\title{放手发展抗日力量,抵抗反共顽固派的进攻}
\date{一九四〇年五月四日}
\thanks{这是毛泽东为中共中央起草的给中共中央东南局的指示。在毛泽东为中央起草这个指示的时期,中共中央委员、中共中央东南局书记项英的思想中存在着严重的右倾观点,没有坚决实行中央的方针,不敢放手发动群众,不敢在日本占领地区扩大解放区和人民军队,对国民党的反动进攻的严重性认识不足,因而缺乏对付这个反动进攻的精神上和组织上的准备。中央这个指示到达后,中共中央东南局委员、新四军第一支队司令员陈毅立即执行了;项英却仍然不愿执行。他对于国民党的可能的反动进攻,仍然不作准备,以致在一九四一年一月间蒋介石发动皖南事变时处于软弱无能的地位,使在皖南的新四军九千余人遭受覆灭性的损失,项英亦被反动分子所杀。}
\maketitle


(一)在一切敌后地区和战争区域,应强调同一性,不应强调特殊性,否则就会是绝大的错误。不论在华北、华中或华南,不论在江北或江南,不论在平原地区、山岳地区或湖沼地区,也不论是八路军、新四军或华南游击队\mnote{1},虽然各有特殊性,但均有同一性,即均有敌人,均在抗战。因此,我们均能够发展,均应该发展。这种发展的方针,中央曾多次给你们指出来了。所谓发展,就是不受国民党的限制,超越国民党所能允许的范围,不要别人委任,不靠上级发饷,独立自主地放手地扩大军队,坚决地建立根据地,在这种根据地上独立自主地发动群众,建立共产党领导的抗日统一战线的政权,向一切敌人占领区域发展。例如在江苏境内,应不顾顾祝同、冷欣、韩德勤\mnote{2}等反共分子的批评、限制和压迫,西起南京,东至海边,南至杭州,北至徐州,尽可能迅速地并有步骤有计划地将一切可能控制的区域控制在我们手中,独立自主地扩大军队,建立政权,设立财政机关,征收抗日捐税,设立经济机关,发展农工商业,开办各种学校,大批培养干部。中央前要你们在今年一年内,在江浙两省敌后地区扩大抗日武装至十万人枪和迅速建立政权等项,不知你们具体布置如何?过去已经失去了时机,若再失去今年的时机,将来就会更困难了。

(二)在国民党反共顽固派坚决地执行其防共、限共、反共政策,并以此为投降日本的准备的时候,我们应强调斗争,不应强调统一,否则就会是绝大的错误。因此,对于一切反共顽固派的防共、限共、反共的法律、命令、宣传、批评,不论是理论上的、政治上的、军事上的,原则上均应坚决地反抗之,均应采取坚决斗争的态度。这种斗争,应从有理、有利、有节的原则出发,也就是自卫的原则、胜利的原则和休战的原则,也就是目前每一具体斗争的防御性、局部性和暂时性。对于反共顽固派的一切反动的法律、命令、宣传、批评,我们应提出针锋相对的办法和他们作坚决的斗争。例如,他们要四、五支队\mnote{3}南下,我们则以无论如何不能南下的态度对付之;他们要叶、张两部\mnote{4}南下,我们则以请准征调一部北上对付之;他们说我们破坏兵役,我们就请他们扩大新四军的募兵区域;他们说我们的宣传错误,我们就请他们取消一切反共宣传,取消一切磨擦法令;他们要向我们举行军事进攻,我们就实行军事反攻以打破之。实行这样的针锋相对的政策,我们是有理由的。凡一切有理之事,不但我党中央应该提出,我军的任何部分均应该提出。例如,张云逸对李品仙,李先念对李宗仁\mnote{5},均是下级向上级提出强硬的抗议,就是好例。只有向顽固派采取这种强硬态度和在斗争时采取有理、有利、有节的方针,才能使顽固派有所畏而不敢压迫我们,才能缩小顽固派防共、限共、反共的范围,才能强迫顽固派承认我们的合法地位,也才能使顽固派不敢轻易分裂。所以,斗争是克服投降危险、争取时局好转、巩固国共合作的最主要的方法。在我党我军内部,只有坚持对顽固派的斗争,才能振奋精神,发扬勇气,团结干部,扩大力量,巩固军队和巩固党。在对中间派的关系上,只有坚持对顽固派的斗争,才能争取动摇的中间派,支持同情的中间派,否则都是不可能的。在应付可能的全国性的突然事变的问题上,也只有采取斗争的方针,才能使全党全军在精神上有所准备,在工作上有所布置。否则,就将再犯一九二七年的错误\mnote{6}。

(三)在估计目前时局的时候,应懂得,一方面,投降危险是大大地加重了;另一方面,则仍未丧失克服这种危险的可能性。目前的军事冲突是局部性的,还不是全国性的。是彼方\mnote{7}的战略侦察行动,还不是立即大举“剿共”的行动;是彼方准备投降的步骤,还不是马上投降的步骤。我们的任务,是坚持地猛力地执行中央“发展进步势力”、“争取中间势力”、“孤立顽固势力”这三项唯一正确的方针,用以达到克服投降危险、争取时局好转的目的。如果对时局的估计和任务的提出发生过左过右的意见,而不加以说明和克服,那也是绝大的危险。

(四)四、五支队反对韩德勤、李宗仁向皖东进攻的自卫战争,李先念纵队反对顽固派向鄂中和鄂东进攻的自卫战争,彭雪枫支队在淮北的坚决斗争,叶飞在江北的发展,以及八路军二万余人南下淮北、皖东和苏北\mnote{8},均不但是绝对必要和绝对正确的,而且是使顾祝同不敢轻易地在皖南、苏南向你们进攻的必要步骤。即是说,江北愈胜利、愈发展,则顾祝同在江南愈不敢轻动,你们在皖南、苏南的文章就愈好做。同样,八路军、新四军和华南游击队,在西北、华北、华中、华南愈发展,共产党在全国范围内愈发展,则克服投降危险争取时局好转的可能性愈增加,我党在全国的文章就愈好做。如果采取相反的估计和策略,以为我愈发展,彼愈投降,我愈退让,彼愈抗日,或者以为现在已经是全国分裂的时候,国共合作已经不可能,那就是错误的了。

(五)在抗日战争中,我们在全国的方针是抗日民族统一战线的。在敌后建立民主的抗日根据地,也是抗日民族统一战线的。中央关于政权问题的决定,你们应该坚决执行。

(六)在国民党统治区域的方针,则和战争区域、敌后区域不同。在那里,是荫蔽精干,长期埋伏,积蓄力量,以待时机,反对急性和暴露。其与顽固派斗争的策略,是在有理、有利、有节的原则下,利用国民党一切可以利用的法律、命令和社会习惯所许可的范围,稳扎稳打地进行斗争和积蓄力量。在党员被国民党强迫入党时,即加入之;对于地方保甲团体、教育团体、经济团体、军事团体,应广泛地打入之;在中央军和杂牌军\mnote{9}中,应该广泛地展开统一战线的工作,即交朋友的工作。在一切国民党区域,党的基本方针,同样是发展进步势力(发展党的组织和民众运动),争取中间势力(民族资产阶级、开明绅士、杂牌军队、国民党内的中间派、中央军中的中间派\mnote{10}、上层小资产阶级和各小党派,共七种),孤立顽固势力,用以克服投降危险,争取时局好转。同时,充分地准备应付可能发生的任何地方性和全国性的突然事变。在国民党区域,党的机关应极端秘密。东南局\mnote{11}和各省委、各特委、各县委、各区委的工作人员(从书记至伙夫),应该一个一个地加以严格的和周密的审查,决不容许稍有嫌疑的人留在各级领导机关之内。应十分注意保护干部,凡有被国民党捕杀危险的公开或半公开了的干部,应转移地区荫蔽起来,或调至军队中工作。在日本占领地区(大城市、中小城市和乡村,如上海、南京、芜湖、无锡等地)的方针,和在国民党区域者基本相同。

(七)以上策略指示,经此次中央政治局会议决定,请东南局和军分会诸同志讨论,传达于全党全军的全体干部,并坚决执行之。

(八)此指示,在皖南由项英同志传达,在苏南由陈毅同志传达。并于接电后一个月内讨论和传达完毕。对于全党全军的工作布置,则由项英同志按照中央方针统筹办理,以其结果报告中央。


\begin{maonote}
\mnitem{1}华南游击队,是中国共产党领导的当时广东省几支抗日游击队的总称。后来发展为:东江纵队、琼崖纵队、珠江纵队、韩江纵队、粤中人民抗日解放军、南路人民抗日解放军。
\mnitem{2}顾祝同,当时任第三战区司令长官。辖区包括浙江、福建、苏南、皖南、赣东。冷欣,当时任第三战区第二游击区副总指挥。韩德勤,当时任国民党江苏省政府主席、鲁苏战区副总司令。辖区包括苏北、皖北以及鲁南的小块地方。
\mnitem{3}四、五支队,即新四军第四、第五两个支队,是张云逸任指挥的新四军江北指挥部的主力。当时他们正在淮河以南、长江以北、运河以西、淮南铁路以东地区建立抗日根据地。
\mnitem{4}叶、张两部,这里指叶飞率领的新四军挺进纵队和张道庸(即陶勇)率领的苏皖支队。当时他们在江苏中部一带开展抗日游击战争,建立抗日根据地。
\mnitem{5}一九四〇年春,国民党安徽省政府主席李品仙、第五战区司令长官李宗仁(均属桂系),派军队向在安徽、湖北抗日的新四军发动大规模进攻。当时新四军江北指挥部指挥张云逸、豫鄂挺进纵队司令员李先念,都曾经强硬地抗议他们破坏抗日的行为,并且在军事上进行了坚决的自卫斗争。
\mnitem{6}指陈独秀右倾投降主义的错误。
\mnitem{7}指以蒋介石为首的国民党顽固派。
\mnitem{8}一九四〇年三月,中共中央为增援新四军在淮北、皖东和苏北的抗日斗争,打退国民党军队向新四军的进攻,命令八路军调遣部队南下。同年夏,南下部队二万余人到达豫皖苏边区,与彭雪枫领导的新四军第六支队和八路军陇海南进支队等会合,先后编为八路军第四、第五纵队,开辟了苏北淮海抗日根据地。
\mnitem{9}“中央军”主要指蒋介石集团的部队。“杂牌军”主要指国民党地方军阀的部队。“杂牌军”受蒋介石集团的歧视,他们的待遇与蒋介石集团的部队不同。
\mnitem{10}“国民党内的中间派”和“中央军中的中间派”,指抗日战争时期,在一定时间内对反共不很积极,或者当反共顽固派向中国共产党领导的军队进攻的时候采取中立态度的国民党内的某些派别和某些个人,中央军中的某些军官或个别部队。
\mnitem{11}东南局是当时中共中央领导中国东南地区工作的代表机关。这个地区包括浙江、福建两省的全部和江苏、安徽、江西三省的一部分地方。
\end{maonote}
