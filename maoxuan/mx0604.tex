
\title{干部要以普通劳动者的姿态出现}
\date{一九五八年五月二十日、二十三日}
\thanks{这是毛泽东同志在中国共产党第八次全国代表大会第二次会议上讲话的一部分。}
\maketitle


(一)讲一讲以普通劳动者的姿态出现的问题。这个问题所以要特别提出来,是因为我们有些干部是老子天下第一,看不起人,靠资格吃饭,做了官,特别是做了大官,就不愿意以普通劳动者的姿态出现。这是一种很恶劣的现象。如果大多数干部能够以普通劳动者的姿态出现,那末这少数干部就会被孤立,就可以改变官僚主义的习气。靠做大官吃饭,靠资格吃饭,妨碍了创造性的发挥。因此,要破除官气,要扫掉官气,要在干部当中扫掉这种官气。谁有真理就服从谁,不管是挑大粪的也好,挖煤炭的也好,扫街的也好,贫苦的农民也好,只要真理在他们手里,就要服从他们。如果你的官很大,可是真理不在你手里,也不能服从你。再说一遍,要是大多数干部扫掉了官气,剩下来的人就是有官气,也容易扫掉了,因为他们孤立了。官气是一种低级趣味,摆架子、摆资格、不平等待人、看不起人,这是最低级的趣味,这不是高尚的共产主义精神。以普通劳动者的姿态出现,则是一种高级趣味,是高尚的共产主义精神。能够做到这一点,防止大国沙文主义,就有可能了。如果我们大多数干部特别是领导干部,以科学的态度,以谦虚的态度,是正确的谦虚态度而不是虚伪的谦虚态度待人,以普通劳动者的姿态出现,大国沙文主义就可以防止,即使出现了也不可怕。

(二)我们的大会是有成绩的,开得好,做了认真的工作,制定了我们的总路线。世界上的事情就怕认真,一认真,不管什么困难都可以打开局面。我国在世界上人口最多,国家大,人民群众得到了解放。资产阶级民主革命胜利了,社会主义革命取得基本胜利,建设有很大的发展,这样已经使我们可以看到我们的前途。以前还不清楚不知道什么时候可以摆脱被动状态、落后状态。以前我们在世界上没有地位。使人看不起,杜勒斯把我们看不在眼里。这和我们的情况不相称,其中也有道理,就是因为你虽然人口多,力量还没有表现出来,有一天赶上英国、美国,杜勒斯就得看上眼。确实有这个国家。我们的方针,这个客人暂不请。那时你找上门来。我们只好招待。过去几年,前年还看不清楚,还有人反总路线,多快好省的方针怀疑的人不少.这种情况也是不可避免的,是客现存在。这许多人能多快好省建设社会主义,那时怀疑的人、反对的人不少。有些人看到了,有些人看不到。看到。要经过曲折才能看到,经过一个时期,看到的人就多了。道路总是曲折的。以后还会有曲折。大会制定了多快好省、鼓足干劲、力争上游的总路线,还要在客观实践中证明。过去有些已经证明了。过去三年是马鞍形,两头高,中间低,前年高去年低,今年又高。有了这个变化,这个会就开好了。这次大家反映了人民的情绪、要求、干劲。多快好省建设社会主义,应该说是去年九月三中全会开始反映这一方面,前年十一月二中全会反映得不够。没有能够占上风。

跟什么人走的问题,首先跟什么人?首先是跟人民学习,跟人民走,人民里面这么多干劲,多快好省。许多发明创造,一类社,千斤亩,两千斤亩。工业方面突破定额,发明创造。总之,工业、农业、商业、文教、军事各方面,思想理论各方面,有各种人材,代表人民的。大会讲了这么多经验,要我讲讲不出来,你们讲的比我好,是正确地反映了人民的要求、思想、感情。根据这些正确的反映。制成比较完备的体系,如这次大会决议和报告,过去没有这样。经过这八年,特别是第一个五年计划,一九五七年三中全会的鼓励就给了全党全国人民比较明确的方向,经过全党的努力。最近半年,去冬今春的大跃进,又经过杭州会议、南宁会议、成都会议,给这次大会做了准备。写了总结、决议,又搞了六十条,还没完成,还要改写。大体意思搞出来了,过几个月再改写一下。这就是先跟人民,然后人民跟我们。首先是理论来自实践,然后理论来指导实践,理论与实践统一是马克思主义的原则。这就是理论来自实践.然后又指导实践。开头没什么马克思主义,因为有了阶级斗争的实践,反映到人民脑子里,首先是反映到先觉者马克思、思格斯、列宁、斯大林的脑子里。客观规律反映到主观世界,有了理论性的总结,而他们发展为理论,给我们做模范。如果要政治上不犯错误,就要理论指导实践。但理论又必须从实践中得来的,离开革命实践不可能制造出理论系统来。关着房门不可能制造出理论来。大会的总路线制定不可能是某些人突然想出来的。不曾你地位多高、官多大、多么有名,如果不下去联系人民,或者向与人民有联系的干部同志们接触。不与人民中的积极分子接触,只要半年你不与人民联系,什么也不知道,就贫乏了。所以规定每年四个月下去是很必要的。下去联系人民向与人民有联系的干部、人民中的积极分子接触。了解他们想些什么,做些什么,经过什么艰苦,然后总结上来。

“鼓足干劲,力争上游”的口号很好,反映了人民的干劲。“干劲”用“鼓足”二字比较好,比“鼓起”好。真理有量的问题。因为干劲早鼓起了。问题是足不足。最少有六、七分,最好八、九分,十分才足。所以用“鼓足”二字比较好。干劲各有不同。
