
\title{我国还可能要走一段资本主义复辟的道路}
\date{一九六四年一月}
\thanks{这是毛泽东同志在徐冰《关于中央统战部几年来若干政策理论性问题的检查总结》\mnote{1}上加写的一段话\mnote{2}。}
\maketitle


如果我们和我们的后代不能时刻提高警惕,不能逐步提高人民群众的觉悟,社会主义教育工作做得不深不透,各级领导权不是掌握在真正的马克思主义者手里,而被修正主义者所篡夺,则我国还可能要走一段资本主义复辟的道路。

\begin{maonote}
\mnitem{1}中共中央统战部副部长徐冰一九六三年五月二十七日作的这个检查总结,主要是不点名地批判统战部部长李维汉自一九五六年以来在政策理论方面提出的若干意见。总结共分七个部分:

(一)关于消灭资产阶级的问题。在这个问题上我们发生过的原则性错误,主要是对于阶级斗争的长期性、曲折性和复杂性认识不足,想过早地消灭资产阶级。这种主观上想比较快、比较早地消灭资产阶级,客观上却起了保留它的作用,发展下去,必然会导致阶级斗争熄灭论和削弱无产阶级专政。(二)关于人的改造问题。应当肯定十几年来对资产阶级分子和资产阶级知识分子改造的成绩;另一方面还必须看到这种改造的长期性,即使是他们中的一些人已经改造成为劳动者或劳动人民的知识分子,我们同他们在思想上有时甚至是政治上的两条道路的阶级斗争,也还要继续一个很长的时期。另外,由于资产阶级思想影响的长期存在,小生产者资本主义自发势力还没有消灭,必然会恢复和再生长出新的资产阶级分子。对于新生的资产阶级分子,我们必须坚决予以打击。(三)关于资产阶级“左派”的问题。过去说资产阶级左派在政治立场上是工人阶级的一部分,他们已经从资产阶级中分化出来,这是错误的。(四)关于民主党派的性质问题。过去我们曾提出五年或更长一点时间内,把民主党派由资产阶级性的政党改造成为社会主义性的政党,在民主党派从中央到基层各级组织,基本上建立起巩固的社会主义领导核心,这种提法是错误的。各民主党派怎么也不能成为科学共产主义的政党。一九六二年全国统战会议期间提出的,对民主党派今后一般不再叫它为资产阶级性的政党,模糊了民主党派的阶级性质,应当加以纠正。(五)关于统一战线的性质问题。我们的人民民主统一战线是以工人阶级为领导,工农联盟为基础,包括各民主阶级、各民主党派、少数民族、宗教界、华侨和其他爱国人士的广泛的联盟。它一方面含有社会主义性质的内容,另一方面又是阶级的联盟。不能说,统一战线已经是社会主义统一战线了。(六)关于改造右派分子的工作问题。一九六二年七月我们提出的,如果领导上认为需要和右派分子本人或其家属要求甄别的就进行甄别的意见,是错误的。

那样做客观上必然会形成对右派分子普遍甄别平反,这势必导致否定一九五七年反右斗争的伟大成绩,后果是严重的。(七)关于中央统战部几年来工作的估计。几年来,中央统战部在中央直接领导下,执行了中央和主席的方针、政策和指示,方向是正确的,成绩是主要的。另一方面,我们在若干政策理论性问题的认识上,长期存在着一些原则性的错误,主要是阶级观点不够明确,缺乏阶级斗争的长期观点和全局观点,把资产阶级人们和民主党派的社会主义改造看得过于容易和简单了。这个总结经毛泽东审阅修改后,中共中央于一九六四年一月十三日以中发(64)26号文件批转各中央局、各省市区党委,并中央各部委、各党组。一九八〇年八月十八日,中央统战部在关于为李维汉同志彻底平反、恢复政治名誉的复查报告中,建议撤销这个中央文件。同年十一月二十日,中央书记处讨论同意了这个建议。
\mnitem{2}毛泽东这段话加写在徐冰作的检查总结中关于消灭资产阶级的问题部分以下一段话之后:“历史的发展告诉我们:只要世界上还有帝国主义、资本主义、各国反动派和现代修正主义存在,这些阴风总会不时地吹到我们国内来,不只会影响原来的资产阶级人们,甚至也会使一些共产党员变质,对于资产阶级和它的思想就不能完全彻底干净消灭之。阶级社会在我们国家存在了几千年,资产阶级的存在也有百多年,花几十年甚至几百年的时间,消灭阶级差别、城乡差别和体力劳动者与脑力劳动者的差别来完成从资本主义到共产主义的过渡,是一个伟大的成就,对世界也会发生伟大的影响。这并不是什么遥遥无期,还是把时间放长一点好,我们从来就有欲速则不达的谚语。”这段话中的“资本主义、各国反动派”是毛泽东加写的,“甚至几百年”,是毛泽东改写的,原稿为“甚至上百年”。
\end{maonote}
