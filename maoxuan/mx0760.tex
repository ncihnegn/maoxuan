
\title{对外宣传工作不可强加于人}
\date{一九六七年三月至一九七一年三月}
\thanks{这是毛泽东同志关于对外宣传和外事工作的一组批语。}
\maketitle


(一)一九六八年三月七日,毛主席在一个拟在援外飞机喷刷毛主席语录的请示报告上批示:“不要那样做,做了效果不好。国家不同,做法也不能一样。”

(二)一九六八年三月十日,毛主席对关于开好一九六八年春季出口商品交易会的通知,作了重要批改,在“必须高举毛泽东思想伟大红旗,突出无产阶级政治,把宣传毛泽东思想,宣传我国无产阶级文化大革命和社会主义建设的伟大胜利,当做首要任务”之后,增加了“但应注意,不要强加于人”一句。

(三)一九六八年三月十二日,毛主席删去了我援外某工程移交问题的请示报告中的一段话:“举行移交仪式时,应大力宣传战无不胜的毛泽东思想,说明我援×修建××××工程的成绩,是我们忠实地执行伟大领袖毛主席关于国际主义教导的结果,是伟大的毛泽东思想的胜利”,并批示“这些是强加于人的,不要这样做。”

(四)一九六八年三月十七日,毛主席在关于答复新共威尔科克斯同志对我对外宣传工作的批评的请示报告上批示:“此事我已说了多次,对外(对内也如此)宣传应当坚决地有步骤地予以改革。”

(五)一九六八年三月二十七日,毛主席对中共中央联络部起草的致缅甸共产党武装斗争二十周年的贺电,作了重要修改和批示。毛主席将“马克思列宁主义、毛泽东思想的伟大胜利”,改为“马克思列宁主义与缅甸情况相结合的伟大胜利”,将无产阶级文化大革命“对于全世界被压迫人民和被压迫民族的革命斗争也具有伟大的意义”,改为“对于全世界被压迫人民和被压迫民族的革命斗争在某一方面也将具有一定的意义。”

(六)一九六八年三月二十九日,毛主席在发表缅共武装斗争二十周年的声明的请示报告(涉及在我报刊上发表兄弟党对毛泽东思想的评价问题)上批示:“一般地说,一切外国党(马、列主义)的内政,我们不应干涉。他们怎样宣传,是他们的事。我们应注意自己的宣传,不应吹得太多,不应说得不适当,使人看起来好像有强加于人的印象。”

(七)一九六八年四月六日,毛主席在中央联络部、总参谋部起草的关于帮助外国人员进行训练的文件中,将“主要是宣传全世界革命人民的伟大导师毛主席和战无不胜的马克思主义、列宁主义、毛泽东思想”一句中的“全世界革命人民的伟大导师毛主席和战无不胜的”二十一字删掉,并批示:“这些空话,以后不要再用。”

(八)一九六八年五月十六日,毛主席批评了在一个文件中用了“世界革命的中心——北京”这种提法。毛主席再一次指出:“这种话不应由中国人口中说出,这就是所谓‘以我为核心’的错误思想。”

(九)一九六八年五月二十九日,毛主席对外交部关于加强宣传主席思想和支持西欧、北美革命群众斗争的建议,作了重要批示:“第一,要注意不要强加于人;第二,不要宣传外国的人民运动是由中国影响的,这样的宣传易为反动派所利用,而不利于人民运动。”

(十)一九六八年六月十二日,一个接待外宾的计划中,曾经规定过群众在同外宾接触时可“自发地分别地赠送毛主席像章”。毛主席批示:“不要”。

(十一)一九六八年七、八月间,毛主席在中共中央对外联络部起草的关于处理某兄弟党要求发表它的一篇文章的请示报告上批示:“删去几个字”,报告中两处提到希望该党“在马克思主义、列宁主义、毛泽东思想的原则基础上”解决党内分歧,毛主席都把“毛泽东思想”这几个字删去了。

(十二)一九六八年八月,毛主席在军委办事组《关于更改援外军事专家名称》的报告和电报稿上的批示:“名称问题关系不大,可从缓议。”“资产阶级传下来东西很多,例如共和国、工程师等等不胜枚举,不能都改”。“此件缓发。”

(十三)一九六八年九月,毛主席对中央文革起草的《庆祝中华人民共和国成立十九周年标语口号(送审稿)》的批示:“去掉第11条,不应用自己名义发出的口号称赞自己。”送审稿的第11条是:“向立下丰功伟绩的中央文革致敬!”

(十四)一九六八年九月,外交部《关于巴基斯坦政府友好代表团来访接待计划的请示》原文中有“通过安排参观访问,突出宣传伟大的毛泽东思想和毛主席一系列最新指示,我无产阶级文化大革命全面胜利以及工农业生产的大好形势。”毛主席将“伟大的毛泽东思想和毛主席一系列最新指示,”一句删掉了,并批示:“对这些不应如此做”。

(十五)原文附的《欢迎(送)巴基斯坦政府友好代表团的标语口号》十九条,毛主席批注:“去掉三条”(即:17、毛主席的无产阶级革命路线胜利万岁!18、战无不胜的马克思主义、列宁主义、毛泽东思想万岁!19、毛主席万岁!万岁!万万岁!)

(十六)一九六九年一月,毛主席在《人民日报》、《红旗》杂志评论员文章《走投无路的自供状——评尼克松的“就职演说”和苏修叛徒集团的无耻捧场》上指示:“照发。尼克松的演说也应见报。”

(十七)毛主席在声明稿的“中国政府建议,双方通过外交途径商定举行中苏边界谈判的日期。”一句中的“日期”之后加了“和地点”三字。并在“如果苏联政府认为中国政府的和平解决边界问题的态度是软弱可欺,可以用核讹诈政策吓倒中国人民,用战争实现对中国的领土要求,那就完全打错了算盘。用毛泽东思想武装起来的、经过无产阶级文化大革命锻炼的七亿中国人民,不是好惹的。”一段中的“不是好惹的”改为“是不好欺负的”,还加了“人不犯我,我不犯人;人若犯我,我必犯人”一句。

(十八)毛主席在社论的第二段“二十年来,又取得了社会主义革命和社会主义建设的一系列伟大胜利,把一个贫穷落后的旧中国,变成一个繁荣昌盛的社会主义强国”一句中的“繁荣昌盛”前边加了“有了初步”四个字,将“强国”改为“国家”。毛主席并批示:“请注意:以后不要这种不合实际情况的自己吹擂。”

(十九)一九六九年九月,毛主席对外交部《关于给日中友协(正统)各地组织庆祝我国庆集会发感谢支持电》中的“二十年来。中国人民在伟大领袖毛主席的英明领导下,在社会主义革命和社会主义建设事业中,特别是在三年的无产阶级文化大革命中取得的伟大胜利,使我们的国家发生了翻天覆地的变化。”一句中的“使我们的国家发生了翻天覆地的变化”改为“使我们国家的面貌发生了重大的变化。”

(二十)一九七〇年六月三日,中国红十字会《关于救济秘鲁地震的请示》,拟以该会名义慰问并捐给秘鲁红十字会现款人民币五万元。

(二十一)一九七〇年六月七日,毛主席口头指示,给秘鲁救济五万元人民币太少。给罗马尼亚救济了一百万元,秘鲁比罗马尼亚死人多,可否给秘鲁救济一百万或一百五十万元人民币?请总理酌定。

(二十二)一九七〇年六月,外交部在“关于我救济匈牙利遭水灾地区居民的请示”中,建议我红十字会捐现款十五万元人民币。毛主席批示:“似太少,可赠五十万元,等于赠罗之一半。”

(二十三)一九七〇年九月,毛主席在中联部起草的我党中央致朝鲜劳动党成立二十五周年贺电稿上,将“经过无产阶级文化大革命锻炼的,用马克思主义、列宁主义、毛泽东思想武装起来的中国共产党全体党员和中国人民”里的“毛泽东思想”五个字圈去。

(二十四)一九七〇年十一月,外交部在《关于对越南南方五省受灾的慰问和赠款问题的请示报告》中,建议以中国红十字会名义向越南南方解放红十字会捐赠价值二百万人民币的物资。毛主席批示:“宜增至五百万。”

(二十五)一九七〇年十二月六日,毛主席在中联部《关于邀请“荷兰共产主义统一运动(马列)”派代表团访华的请示》上批示:“对于一切外国人,不要求他们承认中国人的思想,只要求他们承认马、列主义的普遍真理与该国革命的具体实践相结合。这是一个基本原则。我已说了多遍了。至于他们除马、列主义外,还杂有一些别的不良思想,他们自己会觉悟,我们不必当作严重问题和外国同志交谈。只要看我们党的历史经过多少错误路线的教育才逐步走上正轨,并且至今还有问题,即对内对外都有大国沙文主义,必须加以克服,就可知道了。”

(二十六)一九七一年三月十五日,周总理给毛主席的报告中说,我乒乓球队拟仍前往日本名古屋参加三十一届国际乒乓球比赛。到日本后支持柬民族团结政府不承认朗诺集团的球队为合法,并主张驱逐,对南越卖国集团派的球队,亦采取同一态度,对以色列也询问阿联、叙利亚的态度。如不成,就避开与他们比赛。我球队如去,当作好各种警戒准备。毛主席批示:“照办,我队应去,并准备死几个人,不死更好。要一不怕苦,二不怕死。”
