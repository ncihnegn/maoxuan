
\title{给江青的信}
\date{一九六六年七月八日}
\thanks{这是毛泽东同志在武汉致江青的信,写成后在武汉给周恩来、湖北省委第一书记王任重看过。原件为毛泽东销毁。}
\maketitle


\mxname{江青\mnote{1}:}

六月廿九日的信收到。你还是照魏、陈二同志\mnote{2}的意见在那里住一会儿为好。我本月有两次外宾接见,见后行止再告诉你。自从六月十五日离开武林\mnote{3}以后,在西方的一个山洞里\mnote{4}住了十几天,消息不大灵通。廿八日来到白云黄鹤的地方\mnote{5},已有十天了。每天看材料,都是很有兴味的。

天下大乱,达到天下大治。过七、八年又来一次。牛鬼蛇神自己跳出来。他们为自己的阶级本性所决定,非跳出来不可。我的朋友的讲话\mnote{6},中央催着要发,我准备同意发下去,他是专讲政变问题的。这个问题,象他这样讲法过去还没有过。他的一些提法,我总觉得不安。我历来不相信,我那几本小书,有那样大的神通。现在经他一吹,全党全国都吹起来了,真是王婆卖瓜,自卖自夸,我是被他们逼上梁山的,看来不同意他们不行了。

在重大问题上,违心地同意别人,在我一生还是第一次。叫做不以人的意志为转移吧。晋朝人阮籍反对刘帮,他从洛阳走到成皋,叹到:世无英雄,遂使竖子成名。鲁迅也曾对于他的杂文说过同样的话,我跟鲁迅的心是相通的。我喜欢他那样坦率。他说,解剖自己,往往严于解剖别人。在跌了几跤之后,我亦往往如此。可是同志们往往不信,我是自信而又有些不自信。

我少年时曾经说过:自信人生二百年,会当水击三千里。可见神气十足了。但又不很自信,总觉得山中无老虎,猴子称大王,我就变成这样的大王了。但也不是折中主义,在我身上有些虎气,是为主,也有些猴气,是为次。我曾举了后汉人李固写给黄琼信中的几句话:峣峣者易折,皎皎者易污。阳春白雪,和者盖寡。盛名之下,其实难副\mnote{7}。这后两句,正是指我。

我曾在政治局常委会上读过这几句。人贵有自知之明。今年四月杭州会议,我表示了对于朋友们那样提法的不同意见。可是有什么用呢?他到北京五月会议上还是那样讲,报刊上更加讲的很凶,简直吹的神乎其神。这样,我就只好上梁山了。我猜他们的本意,为了打鬼,借助钟馗。我就在二十世纪六十年代当了共产党的钟馗了。

事物总是要走向反面的,吹得越高,跌得越重,我是准备跌得粉碎的。那也没什么要紧,物质不灭,不过粉碎罢了。全世界一百多个党,大多数的党不信马、列主义了,马克思、列宁也被人们打的粉碎了,何况我们呢?我劝你也要注意这个问题,不要被胜利冲昏了头脑,经常想一想自己的弱点、缺点和错误。这个问题我同你将过不知多少次,你还记得吧,四月在上海还讲过。以上写的,颇有点近乎黑话,有些反党分子,不正是这样说的吗?但他们是要整个打倒我们的党和我本人,我则只说对于我所起的作用,觉得一些提法不妥当,这是我跟黑帮们的区别。

此事现在不能公开,整个左派和广大群众都是这样说的,公开就泼了他们的冷水,帮助了右派,而现在的任务是要在全党全国基本上(不可能全部)打倒右派,而且在七、八年以后还要有一次横扫牛鬼蛇神的运动,今后还要多次扫除,所以我的这些近乎黑话的话,现在不能公开,什么时候公开也说不定,因为左派和广大群众是不欢迎我这样说的。也许在我死后的一个什么时机,右派当权之时,由他们来公开吧。他们会利用我的这种讲法去企图永远高举黑旗的,但是这样一做,他们就倒霉了。

中国自从一九一一年皇帝被打倒以后,反动派当权总是不能长久的,最长的不过二十年(蒋介石),人民一造反,他也倒了。蒋介石利用了孙中山对他的信任,又开了一个黄埔学校,收罗了一大批反动派,由此起家。他一反共,几乎整个地主资产阶级都拥护他,那时共产党又没有经验,所以他高兴地暂时地得势了。但这二十年中,他从来没有统一过,国共两党的战争,国民党和各派军阀之间的战争,中日战争,最后是四年大内战,他就滚到一群海岛上去了。

中国如发生反共的右派政变,我断定他们也是不得安宁的,很可能是短命的,因为代表百分之九十以上人民利益的一切革命者是不会容忍的。那时右派可能利用我的话得势于一时,左派则一定会利用我的另一些话组织起来,将右派打倒。这次文化大革命,就是一次认真的演习。有些地区(例如北京市),根深蒂固,一朝覆灭。有些机关(例如北大、清华),盘根错节,倾刻瓦解。凡是右派越嚣张的地方,他们失败就越惨,左派就越起劲。这是一次全国性的演习,左派、右派和动摇不定的中间派,都会得到各自的教训。

结论:前途是光明的,道路是曲折的,还是这两句老话。

\begin{maonote}
\mnitem{1}江青,毛泽东夫人,时任中央文化革命小组副组长。
\mnitem{2}魏、陈二同志,魏,魏文伯,时任中共上海市委书记、华东局书记。陈,陈丕显,时任中共上海市委第一书记、华东局书记、兼上海警备区第一政委。
\mnitem{3}武林,杭州之旧称。
\mnitem{4}指韶山滴水洞。
\mnitem{5}白云黄鹤的地方,指武汉黄鹤楼唐代诗人崔颢所写的七言律诗《黄鹤楼》,其中有一句“黄鹤一去不复返,白云千载空悠悠”。
\mnitem{6}我的朋友的讲话,指林彪于一九六六年五月十八日在中共中央政治局扩大会议上的讲话,其中提出“防止反革命政变”。
\mnitem{7}峣峣者易折,皎皎者易污。阳春白雪,和者盖寡。盛名之下,其实难副,高的东西容易折损,干净的东西容易变污浊。唱高雅古曲《阳春白雪》时,能和唱的人很少,名声常是大于实际才能的。
\end{maonote}
