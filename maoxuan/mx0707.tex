
\title{各行各业均应一业为主兼学别样——五·七指示}
\date{一九六六年五月七日}
\thanks{这是毛泽东通知对总后勤部“关于进一步搞好部队农副业生产”报告的批语。}
\maketitle


\mxname{林彪\mnote{1}同志:}

你在五月六日寄来总后勤部的报告,收到了,我看这个计划是很好的。是否可以将这个报告发到各军区,请他们召集军、师两级干部在一起讨论一下,以其意见上告军委,然后报告中央取得同意,再向全军作出适当的指示。请你酌定。

只要在没有发生世界大战的条件下,军队应该是一个大学校,即使在第三次世界大战的条件下,很可能也成为一个这样的大学校,除打仗以外,还可做各种工作,第二次世界大战的八年中,各个抗日根据地,我们不是这样做了吗?这个大学校,学政治、学军事、学文化。又能从事农副业生产。又能办一些中小工厂,生产自己需要的若干产品和与国家等价交换的产品。又能从事群众工作,参加工厂农村的社教四清运动\mnote{2};四清完了,随时都有群众工作可做,使军民永远打成一片。又要随时参加批判资产阶级的文化革命斗争。这样,军学、军农、军工、军民这几项都可以兼起来。但要调配适当,要有主有从,农、工、民三项,一个部队只能兼一项或两项,不能同时都兼起来。这样,几百万军队所起的作用就是很大的了。

同样,工人也是这样,以工为主,也要兼学军事、政治、文化。也要搞四清,也要参加批判资产阶级。在有条件的地方,也要从事农副业生产,例如大庆油田\mnote{3}那样。

农民以农为主(包括林、牧、副、渔),也要兼学军事、政治、文化,在有条件的时候也要由集体办些小工厂,也要批判资产阶级。

学生也是这样,以学为主,兼学别样,即不但学文,也要学工、学农、学军,也要批判资产阶级。学制要缩短,教育要革命,资产阶级知识分子统治我们学校的现象,再也不能继续下去了。

商业、服务行业、党政机关工作人员,凡有条件的,也要这样做。

以上所说,已经不是什么新鲜意见、创造发明,多年以来,很多人已经是这样做了,不过还没有普及。至于军队,已经这样做了几十年,不过现在更要有所发展罢了。

\begin{maonote}
\mnitem{1}林彪,时任中共中央副主席、中央军委副主席、国防部长,主持军委工作。
\mnitem{2}社教四清,一九六四年底到一九六五年一月,中央政治局召集全国工作会议,在毛泽东的主持下讨论制定了《农村社会主义教育运动中目前提出的一些问题》(共“二十三条”),将“四清”的内容规定为清政治、清经济、清组织、清思想,强调这次运动的性质是解决“社会主义和资本主义的矛盾”,提出这次运动的重点是整“党内那些走资本主义道路的当权派。”
\mnitem{3}大庆油田,大跃进时期,一九五九年九月二十六日,在松嫩平原上一个叫大同的小镇附近,发现了世界级的特大砂岩油田!当时正值国庆十周年之际,时任黑龙江省委书记的欧阳钦提议将大同改为大庆,将大庆油田作为一份特殊的厚礼献给成立十周年的新中国。一九六〇年三月,大庆油田投入开发建设。大庆油区的发现和开发,证实了陆相地层能够生油并能形成大油田,从而丰富和发展了石油地质学理论,改变了中国石油工业落后面貌,对中国工业发展产生了极大的影响。一九六三年十二月四日,新华社播发《第二届全国人民代表大会第四次会议新闻公报》,首次向世界宣告:“我国需要的石油,过去大部分依靠进口,现在已经可以基本自给了。”中国石油工业彻底甩掉了“贫油”的帽子,中国人民使用“洋油”的时代一去不复返。文革时期,大庆人发扬“铁人精神”,石油产量逐年提高,到一九七六年大庆油田原油年产量首次突破五千万吨大关,进入世界特大型油田的行列,此后,年产五千万万吨的纪录,大庆人奇迹般地保持了二十七年。到二〇〇九年,五十年间,大庆油田生产原油超过二十亿吨,占同期全国原油总产量的百分之四十,为建立我国现代石油工业体系做出了重大贡献。

在百年中国科学史上,让中华民族扬眉吐气的有两大事件,一是研制成功了“两弹一星”,一是发现了大庆油田。
\end{maonote}
