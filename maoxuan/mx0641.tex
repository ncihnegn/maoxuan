
\title{要做系统的由历史到现状的调查研究}
\date{一九六一年三月十三日}
\thanks{这是毛泽东同志在广州召开工作会议(称南三区会议)\mnote{1}上讲话的主要部分。}
\maketitle


这次会议要解决两个很重要的问题:一是生产队与生产队之间的平均主义;一是生产队内部人与人之间的平均主义。这两个问题不解决好,就没有可能充分地调动群众的积极性。

看来人民公社需要有一个条例。高级农业合作社的条例已经过时了。几年来没有一个新的完整的条例。这次起草的农村人民公社工作条例草稿,内容太繁杂、太长,逻辑性不强,不能抓住人一气读下去,要压缩到八千字左右。这次会议讨论一下,先听听你们的意见,你们回去再调查,下次会议作决定。

要做系统的由历史到现状的调查研究。省委第一书记要亲自做调查研究,我也是第一书记,我只抓第一书记。其他的书记也要做调查研究,由你们负责去抓。只要省、地、县、社四级党委的第一书记都做调查研究,事情就好办了。

今年一月找出了三十年前我写的一篇文章\mnote{2},我自己看看觉得还有点道理,别人看怎么样不知道。“文章是自己的好”,我对自己的文章有些也并不喜欢,这一篇我是喜欢的。这篇文章是经过一番大斗争以后写出来的,是在红四军党的第九次代表大会\mnote{3}以后,一九三〇年写的。过去到处找,找不到。这篇文章请大家研究一下,提出意见,哪些赞成,哪些不赞成,如果基本赞成,就照办,不用解释了。文章的主题是,做领导工作的人要依靠自己亲身的调查研究去解决问题。书面报告也可以看,但是这跟自己亲身的调查是不相同的。自己到处跑或者住下来做一个星期到十天的调查,主要是应该住下来做一番系统的调查研究。农村情况,只要先调查清楚一个乡就比较好办了,再去调查其他乡那就心中有数了。

过去这几年我们犯错误,首先是因为情况不明。情况不明,政策就不正确,决心就不大,方法也不对头。医生看病是先诊断,中医叫望、闻、问、切,就是先搞清病情,然后处方。我们打仗首先要搞侦察,侦察敌情、地形,判断情况,然后下决心,部署队伍、后勤等等。历来打败仗的原因大都是情况不明。最近几年吃情况不明的亏很大,付出的代价很大。大家做官了,不做调查研究了。我做了一些调查研究,但大多也是浮在上面看报告。现在,我要搞几个点,几个调查的基地,下去交一些朋友。对城市问题我没有发言权,想调查几个工厂,此心早已有了。我现在搞了几个基地,派了几个组住在几个地方。陈伯达、胡乔木、田家英\mnote{4}他们会后还要回去。我和大家相约,搞点副食品基地的调查研究,目的是为了解决问题,不是为了报表。发那么多表格,报上来说粮食增加了,猪也增加了,经济作物也增加了,而实际上没有增加。我看不要看那些表格,报表我是不看的,实在没有味道。河南要求下边报六类干部情况,今天通知明天就要,这只能是假报告。我们要接受教训。报表有一点也可以,统计部门搞统计需要报表,可是我们了解情况主要不靠报表,也不能靠逐级的报告,要亲自了解基层的情况。

人民公社三年没有搞条例,当然也搞了些规定,比如北戴河会议的决议\mnote{5},武昌会议的决议\mnote{6},郑州会议的记录\mnote{7},上海会议十八条\mnote{8},我写给生产队、生产小队信里提的六条\mnote{9}。这些文件和规定在有些地方不灵,在有些地方灵了。在一类县、社、队是灵了;在二类县、社、队基本灵了,一部分不灵,没有执行;在三类县、社、队基本不灵。犯了错误能改就行,只要好好地去干,错误和失败就会走向反面,反面就是正确和胜利,不要抬不起头来。

有些食堂难以为继。广东有个大队总支书记说,办食堂有四大坏处:一是破坏山林,二是浪费劳力,三是没有肉吃(因为家庭不能养猪),四是不利于生产。前三条都是讲的不利于生产,第四条是个总结。这个同志提出的问题值得注意。这些问题不解决,食堂非散伙不可,今年不散伙,明年也得散伙,勉强办下去,办十年也还得散伙。没有柴烧把桥都拆了,还扒房子、砍树,这样的食堂是反社会主义的。看来食堂要有几种形式,一部分人可以吃常年食堂,大部分人吃农忙食堂。北方冬季食堂非散伙让大家回家吃饭不可,因为有个取暖的问题。

\begin{maonote}
\mnitem{1}毛泽东在广州召开的中共中央中南局、西南局、华东局负责人和这三个地区所属省市自治区党委负责人参加的工作会议(称南三区会议)。这次会议于一九六一年三月十日至十三日召开。
\mnitem{2}指《关于调查工作》。是毛泽东一九三〇年五月为了反对当时红军中的教条主义思想而写的一篇文章。一九六一年三月十一日毛泽东在广州召开的中共中央中南局、西南局、华东局负责人和这三个地区所属省市自治区党委负责人参加的工作会议期间批示印发这篇文章时,将题目改为《关于调查工作》。一九六四年编入《毛泽东著作选读(甲种本)》时,毛泽东将题目改为《反对本本主义》。
\mnitem{3}指一九二九年十二月二十八日至二十九日在福建上杭古田召开的中共红四军第九次代表大会。
\mnitem{4}陈伯达,时任中共中央政治局候补委员、中央政治研究室主任、毛泽东的秘书。胡乔木,时任中共中央书记处候补书记、毛泽东的秘书。田家英,时任中共中央政治研究室副主任、毛泽东的秘书。当时他们根据毛泽东的指示,率调查组分别在广东、湖南、浙江做农村调查。
\mnitem{5}指一九五八年八月十七日至三十日在北戴河召开的中共中央政治局扩大会议通过的《中共中央关于在农村建立人民公社问题的决议》。
\mnitem{6}指一九五八年十一月二十八日至十二月十日在武昌召开的中共八届六中全会通过的《关于人民公社若干问题的决议》。
\mnitem{7}这个记录是一九五九年二月二十七日至三月五日在郑州召开的中共中央政治局扩大会议形成并下发的。记录共三部分:(一)《郑州会议纪要》;(二)毛泽东在会议上的讲话;(三)《关于人民公社管理体制的若干规定(草案)》。
\mnitem{8}指一九五九年三月二十五日至四月一日在上海召开的中共中央政治局扩大会议的会议纪要《关于人民公社的十八个问题》。这个纪要于一九五九年四月二日至五日在上海举行的中共八届七中会会上通过。
\mnitem{9}指毛泽东一九五九年四月二十九日关于农业问题写给六级干部的信中讲的六条。
\end{maonote}
