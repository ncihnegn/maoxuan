
\title{把医疗卫生工作的重点放到农村去}
\date{一九六五年六月二十六日}
\thanks{这是毛泽东同志在一九六五年六月二十六日同医务人员的谈话纪要。}
\maketitle


告诉卫生部,卫生部的工作只给全国人口的百分之十五工作,而这百分之十五中主要还是老爷。广大农民得不到医疗。一无医生,二无药。卫生部不是人民的卫生部,改成城市卫生部或城市老爷卫生部好了。

医学教育要改革,根本用不着读那么多书,华陀\mnote{1}读的是几年制?明朝李时珍\mnote{2}读的是几年制?医学教育用不着收什么高中生、初中生,高小毕业生学三年就够了。主要在实践中学习提高,这样的医生放到农村去,就算本事不大,总比骗人的医生与巫医的要好,而且农村也养得起。书读得越多越蠢。现在那套检查治疗方法根本不适合农村,培养医生的方法,也是为了城市,可是中国有五亿多农民。

脱离群众,工作把大量人力、物力放在研究高、深、难的疾病上,所谓尖端,对于一些常见病,多发病,普遍存在的病,怎样预防,怎样改进治疗,不管或放的力量很少。尖端的问题不是不要,只是应该放少量的人力、物力,大量的人力、物力应该放在群众最需要的问题上去。还有一件怪事,医生检查一定要戴口罩,不管什么病都戴。是怕自己有病传染给别人?我看主要是怕别人传染给自己。要分别对待嘛!什么都戴,这首先造成医生与病人的隔阂。

城市里的医院应该留下一些毕业后一年、二年的本事不大的医生,其余的都到农村去。四清到六五年扫尾,基本结束了,可是四清结束,农村的医疗、卫生工作没结束啊!把医疗卫生工作的重点放到农村去嘛!

\begin{maonote}
\mnitem{1}华佗,东汉时期著名医家,医术精湛,成为后世神医代名词。
\mnitem{2}李时珍,李时珍(约一五一八年——一五九三年),字东璧,晚年自号濒湖山人。蕲州(今湖北省黄冈市蕲春县蕲州镇)人,生于蕲州亦卒于蕲州。李时珍是中国明朝也是中国历史上最著名的医学家、药学家和博物学家之一,其所著的《本草纲目》是本草学集大成的著作,对后世的医学和博物学研究影响深远。
\end{maonote}
