
\title{斗争要文明些,坚持文斗,不用武斗}
\date{一九六六年十二月二十七日}
\thanks{这是毛泽东同志给周恩来同志的一封信和指示。}
\maketitle


\section*{(一)一九六六年十二月二十七日的信}

\mxname{恩来\mnote{1}同志:}

最近,不少来京革命师生和革命群众来信问我,给走资本主义道路的当权派和牛鬼蛇神戴高帽子、打花脸、游街是否算武斗?

我认为:这种作法应该算是武斗的一种形式。这种作法不好。这种作法达不到教育人民的目的。

这里我强调一下,在斗争中一定要坚持文斗,不用武斗,因为武斗只能触及人的身体,不能触及人的灵魂。只有坚持文斗,不用武斗,摆事实,讲道理,以理服人,才能斗出水平来,才能真正达到教育人民的目的。

应该分析,武斗绝大多数是少数别有用心的资产阶级反动分子挑动起来的,他们有意破坏党的政策,破坏无产阶级文化大革命,降低党的威信。凡是动手打人的,应该依法处之。

请转告来京革命师生和革命群众。

\section*{(二)一九六七年二月三日的指示}

斗争要文明些,我们是无产阶级专政,要高姿态,要高风格。北京街头上标语水平不高,到处都打倒、砸烂狗头,那有那么多的狗头,都是人头。这样搞群众很难理解。搞喷气式飞机照相片,登报贴在大街上被外国记者搞走了。现在要将斗争水平提高,现在水平太低。

八月初,也没这凶嘛,斗倒斗臭要在政治上斗臭,要对后代进行教育。不然他们\mnote{2}将来掌权了,也这样干,这就太简单化了。

他们认为这样斗臭了,还有把别人生活上的问题摆出来也叫斗臭了,我看不合适。

主要是政治上斗臭。

\begin{maonote}
\mnitem{1}恩来,周恩来,时任国务院总理。
\mnitem{2}他们,指有可能翻案的走资本主义当权派。
\end{maonote}
