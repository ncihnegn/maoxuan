
\title{抗美援朝的伟大胜利和今后的任务}
\date{一九五三年九月十二日}
\thanks{这是毛泽东同志在中央人民政府委员会第二十四次会议上的讲话。}
\maketitle


抗美援朝,经过三年,取得了伟大胜利,现在已经告一个段落。

抗美援朝的胜利是靠什么得来的呢?刚才各位先生说,是由于领导的正确。领导是一个因素,没有正确的领导,事情是做不好的。但主要是因为我们的战争是人民战争,全国人民支援,中朝两国人民并肩战斗。

我们同美帝国主义这样的敌人作战,他们的武器比我们强许多倍,而我们能够打胜,迫使他们不能不和下来。为什么能够和下来呢?

第一,军事方面,美国侵略者处于不利状态,挨打状态。如果不和,它的整个战线就要被打破,汉城就可能落入朝鲜人民之手。这种形势,去年夏季就已经开始看出来了。

作战的双方,都把自己的战线称为铜墙铁壁。在我们这方面,确实是铜墙铁壁。我们的战士和干部机智,勇敢,不怕死。而美国侵略军却怕死,他们的军官也比较呆板,不那么灵活。他们的战线不巩固,并不是铜墙铁壁。

我们方面发生的问题,最初是能不能打,后来是能不能守,再后是能不能保证给养,最后是能不能打破细菌战。这四个问题,一个接着一个,都解决了。我们的军队是越战越强。今年夏天,我们已经能够在一小时内打破敌人正面二十一公里的阵地,能够集中发射几十万发炮弹,能够打进去十八公里。如果照这样打下去,再打它两次、三次、四次,敌人的整个战线就会被打破。

第二,政治方面,敌人内部有许多不能解决的矛盾,全世界人民要求和下来。

第三,经济方面,敌人在侵朝战争中用钱很多,它的预算收支不平衡。

这几个原因合起来,使敌人不得不和。而第一个原因是主要的原因,没有这一条,同他们讲和是不容易的。美帝国主义者很傲慢,凡是可以不讲理的地方就一定不讲理,要是讲一点理的话,那是被逼得不得已了。

在朝鲜战争中,敌人伤亡了一百零九万人。当然,我们也付了代价。但是我们的伤亡比原来预料的要少得多,有了坑道以后,伤亡就更少了。我们越打越强。美国人攻不动我们的阵地,相反,他们总是被我们吃掉。

刚才大家讲到领导这个因素,我说领导是一个因素,而最主要的因素是群众想办法。我们的干部和战士想出了各种打仗的办法。我讲一个例子。战争的头一个月,我们的汽车损失很大。怎么办呢?除了领导想办法以外,主要是靠群众想办法。在汽车路两旁用一万多人站岗,飞机来了就打信号枪,司机听到就躲着走,或者找个地方把汽车藏起来。同时,把汽车路加宽,又修了许多新汽车路,汽车开过来开过去,畅行无阻。这样,汽车的损失就由开始时的百分之四十,减少到百分之零点几。后来,地下仓库修起来了,地下礼堂也修起来了,敌人在上面丢炸弹,我们在下面开大会。我们住在北京的一些人,一想到朝鲜战场,就感到相当危险。当然,危险是有的,但只要大家想办法,并不是那么了不起。

我们的经验是:依靠人民,再加上一个比较正确的领导,就可以用我们的劣势装备战胜优势装备的敌人。

抗美援朝战争的胜利是伟大的,是有很重要意义的。

第一,和朝鲜人民一起,打回到三八线,守住了三八线。这是很重要的。如果不打回三八线,前线仍在鸭绿江和图们江,沈阳、鞍山、抚顺这些地方的人民就不能安心生产。

第二,取得了军事经验。我们中国人民志愿军的陆军、空军、海军,步兵、炮兵、工兵、坦克兵、铁道兵、防空兵、通信兵,还有卫生部队、后勤部队等等,取得了对美国侵略军队实际作战的经验。这一次,我们摸了一下美国军队的底。对美国军队,如果不接触它,就会怕它。我们跟它打了三十三个月,把它的底摸熟了。美帝国主义并不可怕,就是那么一回事。我们取得了这一条经验,这是一条了不起的经验。

第三,提高了全国人民的政治觉悟。

由于以上三条,就产生了第四条:推迟了帝国主义新的侵华战争,推迟了第三次世界大战。

帝国主义侵略者应当懂得:现在中国人民已经组织起来了,是惹不得的。如果惹翻了,是不好办的。

今后,敌人还可能打,就是不打,也一定要用各种办法来捣乱,比如派遣特务进行破坏。他们在台湾、香港和日本这些地方,都设有庞大的特务机构。可是,我们在抗美援朝中得到了经验,只要发动群众,依靠人民,我们是有办法来对付他们的。

我们现在的情况,同一九五〇年冬季的情况不同了。那时候,美国侵略者是不是在三八线那边呢?不是,他们是在鸭绿江、图们江那边。我们有没有对美国侵略者作战的经验呢?没有。对于美国军队熟悉不熟悉呢?不熟悉。现在这些情况都变了。如果美帝国主义不推迟新的侵略战争,他说,我要打!我们就用前三条对付他。如果他说,我不打了!那末我们就有了第四条。这也证明我们人民民主专政的优越性。

我们是不是去侵略别人呢?任何地方我们都不去侵略。但是,人家侵略来了,我们就一定要打,而且要打到底。

中国人民有这么一条:和平是赞成的,战争也不怕,两样都可以干。我们有人民的支持。在抗美援朝战争中,人民踊跃报名参军。对报名参军的人挑得很严,百里挑一,人们说比挑女婿还严。如果美帝国主义要再打,我们就跟它再打下去。

打仗要用钱。可是,抗美援朝战争用的钱也不十分多。打了这几年,用了还不到一年的工商业税。当然,能够不打仗,不用这些钱,那就更好。因为现在建设方面要用钱,农民的生活也还有困难。去年、前年收的农业税重了一点,于是有一部分朋友就说话了。他们要求“施仁政”,好象他们代表农民利益似的。我们赞成不赞成这种意见呢?我们是不赞成的。当时,必须尽一切努力来争取抗美援朝的胜利。对农民说来,对全国人民说来,是生活暂时困难一点,争取胜利对他们有利,还是不抗美援朝,不用这几个钱对他们有利呢?当然,争取抗美援朝的胜利对他们有利。去年和前年,我们多收了一点农业税,就是因为抗美援朝要用钱。今年就不同了,农业税没有增加,我们把税额稳定下来了。

说到“施仁政”,我们是要施仁政的。但是,什么是最大的仁政呢?是抗美援朝。要施这个最大的仁政,就要有牺牲,就要用钱,就要多收些农业税。多收一些农业税,有些人就哇哇叫,还说什么他们是代表农民利益。我就不赞成这种意见。

抗美援朝是施仁政,现在发展工业建设也是施仁政。

所谓仁政有两种:一种是为人民的当前利益,另一种是为人民的长远利益,例如抗美援朝,建设重工业。前一种是小仁政,后一种是大仁政。两者必须兼顾,不兼顾是错误的。那末重点放在什么地方呢?重点应当放在大仁政上。现在,我们施仁政的重点应当放在建设重工业上。要建设,就要资金。所以,人民的生活虽然要改善,但一时又不能改善很多。就是说,人民生活不可不改善,不可多改善;不可不照顾,不可多照顾。照顾小仁政,妨碍大仁政,这是施仁政的偏向。

有的朋友现在片面强调小仁政,其实就是要抗美援朝战争别打了,重工业建设别干了。我们必须批评这种错误思想。这种思想共产党里边也有,在延安就碰到过。一九四一年,陕甘宁边区征了二十万石公粮,一些人就哇哇叫,说共产党不体贴农民。共产党的个别领导干部也提出所谓施仁政问题。那时我就批评了这种思想。当时最大的仁政是什么呢?是打倒日本帝国主义。如果少征公粮,就要缩小八路军、新四军,那是对日本帝国主义有利的。所以,这种意见,实际上是代表日本帝国主义、帮日本帝国主义忙的。

现在,抗美援朝已经告一段落,如果美国还要打,我们还是打。要打就要征粮,就要在农民中做工作,说服农民出点东西。这才是真正代表农民的利益。哇哇叫,实际上是代表美帝国主义。

道理有大道理,有小道理。全国人民的生活水平每年应当提高一步,但是不能提得太高。如果提得过多,抗美援朝战争就不能打了,或者不能那样认真地打。我们是彻底地认真地全力地打,只要我们有,朝鲜前线要什么就给什么。这几年,我们就是这样干的。
