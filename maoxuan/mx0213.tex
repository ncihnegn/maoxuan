
\title{五四运动}
\date{一九三九年五月一日}
\thanks{这是毛泽东为延安出版的中共中央机关报《解放》写的纪念五四运动二十周年的文章。}
\maketitle


二十年前的五四运动\mnote{1},表现中国反帝反封建的资产阶级民主革命已经发展到了一个新阶段。五四运动的成为文化革新运动,不过是中国反帝反封建的资产阶级民主革命的一种表现形式。由于那个时期新的社会力量的生长和发展,使中国反帝反封建的资产阶级民主革命出现一个壮大了的阵营,这就是中国的工人阶级、学生群众和新兴的民族资产阶级所组成的阵营。而在“五四”时期,英勇地出现于运动先头的则有数十万的学生。这是五四运动比较辛亥革命进了一步的地方。

中国资产阶级民主革命的过程,如果要从它的准备时期说起的话,那它就已经过了鸦片战争\mnote{2}、太平天国战争\mnote{3}、甲午中日战争\mnote{4}、戊戌维新\mnote{5}、义和团运动\mnote{6}、辛亥革命\mnote{7}、五四运动、北伐战争、土地革命战争等好几个发展阶段。今天的抗日战争是其发展的又一个新的阶段,也是最伟大、最生动、最活跃的一个阶段。直至国外帝国主义势力和国内封建势力基本上被推翻而建立独立的民主国家之时,才算资产阶级民主革命的成功。从鸦片战争以来,各个革命发展阶段各有若干特点。其中最重要的区别就在于共产党出现以前及其以后。然而就其全体看来,无一不是带了资产阶级民主革命的性质。这种民主革命是为了建立一个在中国历史上所没有过的社会制度,即民主主义的社会制度,这个社会的前身是封建主义的社会(近百年来成为半殖民地半封建的社会),它的后身是社会主义的社会。若问一个共产主义者为什么要首先为了实现资产阶级民主主义的社会制度而斗争,然后再去实现社会主义的社会制度,那答复是:走历史必由之路。

中国民主革命的完成依靠一定的社会势力。这种社会势力是:工人阶级、农民阶级、知识分子和进步的资产阶级,就是革命的工、农、兵、学、商,而其根本的革命力量是工农,革命的领导阶级是工人阶级。如果离开了这种根本的革命力量,离开了工人阶级的领导,要完成反帝反封建的民主革命是不可能的。在今天,革命的根本敌人是日本帝国主义和汉奸,革命的根本政策是抗日民族统一战线,这个统一战线的组织成分是一切抗日的工、农、兵、学、商。抗日战争最后胜利的取得,将是在工、农、兵、学、商的统一战线大大地巩固和发展的时候。

在中国的民主革命运动中,知识分子是首先觉悟的成分。辛亥革命和五四运动都明显地表现了这一点,而五四运动时期的知识分子则比辛亥革命时期的知识分子更广大和更觉悟。然而知识分子如果不和工农民众相结合,则将一事无成。革命的或不革命的或反革命的知识分子的最后的分界,看其是否愿意并且实行和工农民众相结合。他们的最后分界仅仅在这一点,而不在乎口讲什么三民主义或马克思主义。真正的革命者必定是愿意并且实行和工农民众相结合的。

五四运动到现在已有了二十个周年,抗日战争也快到两周年了。全国的青年和文化界对于民主革命和抗日战争负有大的责任。我希望他们认识中国革命的性质和动力,把自己的工作和工农民众结合起来,到工农民众中去,变为工农民众的宣传者和组织者。全国民众奋起之日,就是抗日战争胜利之时。全国青年们,努力啊!


\begin{maonote}
\mnitem{1}见本书第一卷\mxnote{实践论}{6}。
\mnitem{2}见本书第一卷\mxnote{论反对日本帝国主义的策略}{35}。
\mnitem{3}见本书第一卷\mxnote{论反对日本帝国主义的策略}{36}。
\mnitem{4}见本书第一卷\mxnote{矛盾论}{22}。
\mnitem{5}见本卷\mxnote{论持久战}{12}。
\mnitem{6}见本书第一卷\mxnote{论反对日本帝国主义的策略}{37}。
\mnitem{7}见本书第一卷\mxnote{湖南农民运动考察报告}{3}。
\end{maonote}
