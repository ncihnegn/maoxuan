
\title{关于国际形势问题\mnote{1}}
\date{一九五八年九月五日、八日}
\thanks{这是毛泽东同志在第十五次最高国务会议上两次讲话中关于国际形势的部分。}
\maketitle


\section*{一}

国际形势,我们历来有个观点,总是乐观的。后来总结为一个“东风压倒西风”。

美国现在在我们这里来了个“大包干”制度,索性把金门、马祖,还有些什么大担岛、二担岛、东碇岛一切包过去,我看它就舒服了。它上了我们的绞索,美国的颈吊在我们中国的铁的绞索上面。台湾也是个绞索,不过要隔得远一点。它要把金门这一套包括进去,那它的头更接近我们。我们哪一天踢它一脚,它走不掉,因为它被一根索子绞住了。

我现在提出若干观点,提出一些看法供给各位,并不要把它作为一个什么决定,作为一个法律。作为一个法律就死了,作为一个看法就是活的。拿这些观点去观察观察国际形势。

第一条,谁怕谁多一点。我看美国人是怕打仗。我们也怕打仗。问题是究竟哪一个怕得多一点。这也是个观点,也是个看法。请各位拿了这个观点去看一看,观察观察,以后一年、二年、三年、四年,就这样观察下去,究竟是西方怕东方多一点,还是我们东方怕西方多一点?据我的看法,是杜勒斯\mnote{2}怕我们怕得多一点,是英、美、德、法那些西方国家怕我们怕得多一点。为什么它们怕得多一点呢?就是一个力量的问题,人心的问题。人心就是力量,我们这边的人多一点,它们那边的人少一点。共产主义,民族主义,帝国主义,这三个主义中,共产主义和民族主义比较接近。而民族主义占领的地方相当宽,有三个洲:一个亚洲,一个非洲,一个拉丁美洲。即使这些洲里头有许多统治者是亲西方的,比如泰国、巴基斯坦、菲律宾、日本、土耳其、伊朗等国的,可是人民中间亲东方的不少,可能是相当多。就是垄断资本家以及中了他们的毒最深的人是主张战争的。除了垄断资本家,其他的人,大多数(不是全体)是不愿意战争的。比如北欧几个国家,当权的也是资产阶级,他们是不愿意战争的。力量对比是如此。因为真理是抓在大多数人手里,而不抓在杜勒斯手里,他们的心比我们虚,我们的心比较实。我们依靠人民,他们是维持那些反动统治者。现在杜勒斯就干这一套,他就专扶什么蒋委员长、李承晚、吴庭艳\mnote{3}这类人。我看是这样,双方都怕,但是他们怕我们比较多一点,因此战争是打不起来的。

第二条,美帝国主义它们结成军事集团,什么北大西洋\mnote{4},巴格达\mnote{5},马尼拉\mnote{6},这些集团的性质究竟怎么样?我们讲它们是侵略的。它们是侵略的,那是千真万确的。但是它们现在的锋芒向哪一边呢?是向社会主义进攻,还是向民族主义进攻?我看现在是向民族主义进攻,就是向埃及、黎巴嫩和中东那些弱的国家进攻。社会主义国家,除非是比如匈牙利失败了,波兰也崩溃了,捷克、东德也崩溃了,连苏联也发生问题,我们也发生问题,摇摇欲倒,那个时候它们会进攻的。你要倒了,它们为什么不进攻?现在我们不倒,我们巩固,我们这个骨头啃不动,它们就啃那些比较可啃的地方,搞印尼、印度、缅甸、锡兰\mnote{7},想搞垮纳赛尔\mnote{8},想搞垮伊拉克,想征服阿尔及利亚等等。现在拉丁美洲有个很大的进步。尼克松是个副总统,在八个国家不受欢迎\mnote{9},被吐口水,打石头。美国的政治代表在那些人面前被用口水去对付,这就是藐视“尊严”,没有“礼貌”了,在他们心目中间不算数了。你是我们的对头,因此拿口水、石头去对付你。所以,不要把这三个军事集团看得那么严重,要有分析。它们是侵略性的,但是它们并不巩固。

第三条,关于紧张局势。我们每天都是要求缓和紧张局势,紧张局势缓和了对世界人民是有利的。那末,凡是紧张局势就对我们有害,是不是?我看也不尽然。这个紧张局势,对我们并不是纯害无利,也有有利的一面。什么道理呢?因为紧张局势除了有害的一面外,还可以调动人马,调动落后阶层,调动中间派起来奋斗。怕打原子战争的,就要想一想。你看金门、马祖打这样几炮\mnote{10},我就没有料到现在这个世界闹得这样满城风雨,烟雾冲天。这就是因为人们怕战争,怕美国到处闯祸。全世界那么多国家,除了一个李承晚之外,现在还没有第二个国家支持美国。可能还加一个菲律宾,叫做“有条件的支持”。比如伊拉克革命,还不是紧张局势造成的?紧张局势并不取决于我们,是帝国主义自己造成的,但是归根结底对于帝国主义更不利。这个观点列宁说过的,他是讲战争,他说,战争调动人们的精神状态使它紧张起来。现在当然没有战争,但是这种在武装对立的情况下的紧张局势也是能够调动一切积极因素,并且使落后阶层想一想的。

第四条,中东的撤兵问题。美英侵略军必须撤退。帝国主义现在想赖在那里不走,这对人民是不利的,可是同时也有教育人民的作用。你要反对侵略者,如果没有个对象,没有个靶子,没有个对立面,这就不好反。它自己现在跑上来当作对立面,并且赖着不走,就起了动员全世界人民起来反对美国侵略者的作用。所以它迟迟不撤退,总起来看对人民也不见得就那么纯害无利,因为这样人民每天就可以催它走:你为什么不走?

第五条,戴高乐\mnote{11}登台好,还是不登台好?现在法国共产党和人民应该坚决反对戴高乐登台,要投票反对他的宪法,但是同时要准备反对不了时,他登台后的斗争。戴高乐登台要压迫法共和法国人民,但对内对外也有好处。对外,这个人喜欢跟英美闹别扭,他喜欢抬杠子。他从前吃过苦头的,他写过一本回忆录,尽骂英美,而说苏联的好话。现在看起来,他还是要闹别扭的。法国跟英美闹别扭很有益处。对内,为教育法国无产阶级不可少之教员,等于我们中国的蒋委员长一样。没有蒋委员长,六亿人民教不过来的,单是共产党正面教育不行的。戴高乐现在还有威信,你这会把他打败了,他没有死,人们还是想他。让他登台,无非是顶多搞个五年,六年,七年,八年,十年,他得垮的。他一垮了,没有第二个戴高乐了,这个毒放出来了。这个毒必须放,等于我们右派的毒,你得让他放。你不让他放,他总是有毒的,放出来毒就消了。

第六条,禁运,不跟我们做生意。这个东西对于我们的利害究竟怎么样?我看,禁运对我们的利益极大,我们不感觉禁运有什么不利。禁运对于我们的衣食住行以及建设(炼钢炼铁)有极大的好处。一禁运,我们得自己想办法。我历来感谢何应钦\mnote{12}。一九三七年红军改编成国民革命军第八路军,每月有四十万法币,自从他发了法币,我们就依赖这个法币。到一九四〇年反共高潮时就断了,不来了。从此我们得自己想办法,想什么办法呢?我们就下了个命令,说法币没有了,你们以团为单位自己打主意。从此,各根据地搞生产运动,产生的价值不是四十万元,不是四百万元,甚至于不是四千万元,各根据地合起来,可能一亿两亿。从此就靠我们自己动手。现在的“何应钦”是谁呢?就是杜勒斯,改了个名字。现在它们禁运,我们就自己搞,搞大跃进,搞掉了依赖性,破除了迷信,就好了。

第七条,不承认问题。是承认比较有利,还是不承认比较有利?我说,等于禁运一样,帝国主义国家不承认我们比较承认我们是要有利一些。现在还有四十几个国家不承认我们,主要的原因就在美国。比如法国,想承认,但是因为美国反对就不敢。其他还有一些中南美洲、亚洲、非洲、欧洲的国家,以及加拿大,都是因为美国而不敢承认。资本主义国家现在承认我们的,合起来只有十九个,加上社会主义阵营十一个,有三十个\mnote{13},再加上南斯拉夫,有三十一个。我看就是这么一点过日子吧。不承认我们,我看是不坏,比较好,让我们更多搞一点钢,搞个六七亿吨,那个时候它们总要承认。那个时候也可以不承认,它们不承认有什么要紧?

最后一条,就是准备反侵略的战争。头一条讲了双方怕打,仗打不起来,但世界上的事情还是要搞一个保险系数。因为世界上有个垄断资产阶级,恐怕他们冒里冒失乱搞,所以,要准备作战。这一条要在干部里头讲通。第一,我们不要打,而且反对打,苏联也是。要打就是他们先打,逼着我们不能不打。第二,但是我们不怕打,要打就打。我们现在只有手榴弹跟山药蛋。氢弹、原子弹的战争当然是可怕的,是要死人的,因此我们反对打。但是这个决定权不操在我们手中,帝国主义一定要打,那末我们就得准备一切,要打就打。就是说,死了一半人也没有什么可怕。这是极而言之。在整个宇宙史上来说,我就不相信要那么悲观。我跟尼赫鲁\mnote{14}总理辩论过这个问题,他说,那个时候没有政府了,统统打光了,想要讲和也找不到政府了。我说,哪有那个事,你这个政府被原子弹消灭了,老百姓又起一个政府,又可以议和。世界上的事情你不想到那个极点,你就睡不着觉。无非是打死人,无非是一个怕打。但是它一定要打,是它先打,它打原子弹,这个时候,怕,它也打,不怕,它也打。既然是怕也打,不怕也打,二者选哪一个呢?还是怕好,还是不怕好?每天总是怕,在干部和人民里头不鼓起一点劲,这是很危险的。我看,还是横了一条心,要打就打,打了再建设。因此,我们现在搞民兵,人民公社里头都搞民兵,全民皆兵。要发枪,开头发几百万枝,将来要发几千万枝,由各省造轻武器,造步枪、机关枪、手榴弹、小迫击炮、轻迫击炮。人民公社有军事部,到处练习。在座的有文化人,你们也要号召一下,单拿笔杆不行,一手拿笔杆,一手拿枪杆,又是文化,又是武化。

有这么八个观点,当做一种看法,供各位观察国际形势的时候采用。

\section*{二}

还是谈一谈老话。关于绞索,上一次不是谈过吗?现在我们要讲对杜勒斯、艾森豪威尔\mnote{15},对那些战争贩子使用绞索。对美国使用绞索的地方很多。据我看,凡是搞了军事基地的,就被一条绞索绞住了,例如:东方,南朝鲜、日本、菲律宾、台湾;西方,西德、法国、意大利、英国;中东,土耳其、伊朗;非洲,摩洛哥等等。每一个地方美国有许多军事基地,比如土耳其有二十几个基地,日本听说有八百个基地。还有些地方没有基地,但是有军队占领,比如美国在黎巴嫩,英国在约旦。

现在不讲别的,单讲两条绞索:一个黎巴嫩,一个台湾。台湾是老的绞索,美国已经占领几年了。它被什么人绞住了呢?被中华人民共和国绞住了。六亿人民手里拿着一根索子,这根索子是钢绳,把美国的脖子套住了。谁人让它套住的呢?是它自己造的索子,自己套住的,然后把绞索的一头丢到中国大陆上,让我们抓到。黎巴嫩是最近套住的,也是美国自己造的一条绞索,自己套上去的,绞索的一端就丢到阿拉伯民族手里。不但如此,而且是丢到全世界大多数人民手里,大家都骂它,不同情它,大多数国家的人民、政府手里拿着这个绞索。比如中东问题,联合国开了会。但主要是在阿拉伯人民手里套住了,不得脱身。它现在进退两难,早退好,还是迟退好?早退,那末所为何来呢?迟退,越套越紧,可能成为死结,那怎么得了呀?至于台湾,它是订了条约的\mnote{16},比黎巴嫩还不同。黎巴嫩还比较活,没有什么条约,说是一个请,一个就来了,于是乎套上了。至于台湾,就订了个条约,这是个死结。这里不分民主党、共和党,订条约是艾森豪威尔,派第七舰队是杜鲁门\mnote{17}。杜鲁门那个时候可去可来,没有订条约,艾森豪威尔订了个条约。这也是国民党一恐慌、一要求,美国一愿意,就套上了。

金门、马祖套上了没有?金门、马祖据我看也套上了。为什么呢?他们不是讲现在还没有定,要共产党打上去,那个时候看情形再决定吗?问题是十一万国民党军队,金门九万五,马祖一万五,只要有这两堆在这个地方,他们得关心。这是他们的阶级利益,阶级感情。为什么英国人和美国人对一些国家的政府那样好?他们不能见死不救。昨天第七舰队的司令比克利亲自指挥\mnote{18},还有那个斯摩特\mnote{19},不是放大炮吗?引得国务院也不高兴、国防部也不高兴的那位先生,他也在那里跟比克利一道指挥。

总而言之,你是被套住了。要解脱也可以,你得采取主动,慢慢脱身。不是有脱身政策吗?在朝鲜有脱身政策,现在我看形成了金、马的脱身政策。你那一班子实在想脱身,而且舆论上也要求脱身。脱身者,是从绞索里面脱出去。怎么脱法呢?就是这十一万人走路。台湾是我们的,那是无论如何不能让步的,是内政问题;跟你的交涉是国际问题。这是两件事。你美国跟蒋介石搞在一起,这个化合物是可以分解的。比如电解铝、电解铜,用电一解,不就分离了吗?蒋介石这一边是内政问题,你那一边是外交问题,不能混为一谈。

现在五大洲,除了澳洲,四大洲美国都想霸住。首先是北美洲,那主要是它自己的地方,它有军队;然后是中南美洲,虽然没有驻军,但是它要“保护”的。再加上欧洲、非洲、亚洲,主要是欧亚非,主力是在欧亚两洲。这么几个兵,分得这么散,我不晓得它这个仗怎么打法。所以,我总是觉得,它是霸中间地带为主。至于我们这些地方,除非是社会主义阵营出了大乱子,它确有把握,一来,我们苏联、中国就全部崩溃,否则我看它是不敢来的。除了我们这个阵营以外,它都想霸占,一个拉丁美洲,一个欧洲,一个非洲,一个亚洲。还有个澳洲,澳洲也在军事条约上跟它联起来了,听它的命令。它用“反共”的旗帜取得这些地方好些,还是真正反共好些?所谓真正反共,就是拿军队来打我们,打苏联。我说,没有那么蠢的人。它只有几个兵调来调去,黎巴嫩事情发生,从太平洋调去,到了红海地方,形势不对,赶快回头,到马来亚\mnote{20}登陆,名为休息几天,十七天不吭声。后头它一个新闻记者自己宣布是管印度洋的,这一来,印度洋大家都反对。我们这里一打炮,这里兵不够,它又来了。台湾这些地方早一点解脱,对美国比较有利,它赖着不走,就让它套到这里,无损于大局,我们还是搞大跃进。

至于紧张局势,也许还可以讲几句。你搞紧张局势,你以为对你有利呀?不一定,紧张局势调动世界人心,都骂美国人。中东紧张局势大家骂美国人。台湾紧张局势又是大家骂美国人,骂我们的比较少。美国人骂我们,蒋介石骂我们,李承晚骂我们,也许还有一点人骂我们,主要就是这三个。英国是动摇派,军事上不参加,政治上听说它相当同情。因为它有个约旦问题,它不同情一下,美国人如果在黎巴嫩撤退,英国在约旦怎么办呀?尼赫鲁总理发表了声明,基本上跟我们一致的,赞成台湾这些东西归我们,不过希望和平解决。这回中东各国可是欢迎啦,特别是一个阿联\mnote{21},一个伊拉克,每天吹,说我们这个事情好。因为我们这一搞,美国人对它们那里的压力就轻了。

我想可以公开告诉世界人民,紧张局势比较对于西方国家不利,对于美国不利。利在什么地方呢?中东紧张局势对于美国有什么利?对于英国有什么利?还是对于阿拉伯国家有利些,对于亚洲、非洲、拉丁美洲以及其他各洲爱好和平的人民有利些。台湾的紧张局势究竟对谁有利些呢?比如对于我们国家,我们国家现在全体动员,如果说中东事件有三四千万人游行示威、开会,这一次大概搞个三亿人口,使他们得到教育,得到锻炼。这个事情对于各民主党派的团结也好吧,各党派有一个共同奋斗目标,这样一来,过去心里有些疙瘩的,有些气的,受了批评的,也就消散一点吧。就这样慢慢搞下去,七搞八搞,我们大家还不就是工人阶级了。所以,帝国主义自己制造出来的紧张局势,结果反而对于反对帝国主义的我们几亿人口有利,对于全世界爱好和平的人民,各阶级,各阶层,政府,我看都有利。他们得想一想,美国总是不好,张牙舞爪。十三艘航空母舰就来了六艘,其中有大到那么大的,有什么六万五千吨的,说是要凑一百二十条船,第一个最强的舰队。你再强一点也好,把你那四个舰队统统集中到这个地方我都欢迎。你那个东西横直没有用的,统统集中来,你也上来不得。船的特点,就在水里头,不能上岸。你不过是在这个地方摆一摆,你越打,越使全世界的人都知道你无理。

\begin{maonote}
\mnitem{1}第十五次最高国务会议于一九五八年九月五日、六日和八日在北京举行。毛泽东先后在五日和八日的会议上作了讲话,本篇一选自九月五日的讲话,二选自九月八日的讲话。关于毛泽东九月八日的讲话,九月九日《人民日报》发表的经毛泽东修改的新闻稿作了报道,这里将有关国际形势部分摘录如下:

毛泽东主席说,目前的形势对全世界争取和平的人民有利。总的趋势是东风压倒西风。毛主席说,美帝国主义九年来侵占了我国领土台湾,不久以前又派遣它的武装部队侵占了黎巴嫩。美国在全世界许多国家建立了几百个军事基地。中国领土台湾、黎巴嫩以及所有美国在外国的军事基地,都是套在美帝国主义脖子上的绞索。不是别人而是美国人自己制造这种绞索,并把它套在自己的脖子上,而把绞索的另一端交给了中国人民、阿拉伯各国人民和全世界一切爱和平反侵略的人民。美国侵略者在这些地方停留得越久,套在它的头上的绞索就将越紧。

毛泽东主席又说,美帝国主义在全世界到处制造紧张局势。以期达到它侵略和奴役各国人民的目的。美帝国主义自以为紧张局势总是对它自己有利,但是事实是,美国制造的这些紧张局势走向了美国人愿望的反面,它起了动员全世界人民起来反对美国侵略者的作用。毛主席说,美国垄断资本集团如果坚持推行它的侵略政策和战争政策,势必有一天要被全世界人民处以绞刑。其他美国帮凶也将是这样。

毛主席对于中美两国在华沙即将开始的大使级代表的谈判寄予希望。他说:如果双方具有解决问题的诚意的话,谈判可能会取得某些成果。现在全世界人民都在注视着两国代表将要进行的谈判。
\mnitem{2}杜勒斯,时任美国国务卿。
\mnitem{3}李承晚,时任南朝鲜即韩国总统。吴庭艳(一九〇一——一九六三),时任“越南共和国”总统兼总理和国防部长。
\mnitem{4}指北大西洋公约组织。一九四九年四月,美国、英国、法国、荷兰、比利时、卢森堡、挪威、葡萄牙、意大利、丹麦、冰岛和加拿大在华盛顿签署《北大西洋公约》。同年八月二十四日公约生效,北大西洋公约军事集团建立。希腊和土耳其于一九五二年,德意志联邦共和国于一九五五年,西班牙于一九八二年,波兰、捷克和匈牙利于一九九九年,正式加入该组织。
\mnitem{5}指巴格达条约组织,是英、美两国为控制中东地区和遏制苏联而组织的军事集团。一九五五年十一月根据《巴格达条约》而成立,一九五九年八月改称中央条约组织。其成员国有土耳其、伊拉克、英国、伊朗和巴基斯坦,美国以“观察员”身分参加。一九五八年七月伊拉克王朝被推翻后,新政府于次年三月正式宣布退出。随着成员国在一系列国际问题上分歧的日益扩大,一九七九年三月伊朗、巴基斯坦、土耳其三国也宣布退出,同年九月二十八日该组织解散。
\mnitem{6}指东南亚条约(即马尼拉条约)组织。一九五四年九月八日,在美国策动下,由美国、英国、法国、澳大利亚、新西兰、菲律宾、泰国和巴基斯坦在菲律宾首都马尼拉签订了《东南亚集体防务条约》,又称《马尼拉条约》。这是一个军事同盟条约,条约声明要用“自助和互助的办法”“抵抗武装进攻”。条约附有美国提出的“谅解”,对“侵略和武装进攻的意义”解释为“只适用于共产党的侵略”。条约还以议定书的形式,把柬埔寨、老挝和南越划为它的“保护地区”。一九五五年二月十九日条约生效时成立了东南亚条约组织。一九六二年七月日内瓦会议通过的《关于老挝中立的宣言》,不承认它对老挝的所谓保护。一九六七年起法国拒绝派正式代表团参加该组织的部长级理事会。一九七二年十一月八日巴基斯坦宣布退出。一九七七年六月该组织宣布解散。
\mnitem{7}锡兰,今斯里兰卡。
\mnitem{8}纳赛尔,时任阿拉伯联合共和国总统。
\mnitem{9}一九五八年四月二十八日至五月十四日,美国副总统尼克松先后对乌拉圭、阿根廷、巴拉圭、玻利维亚、秘鲁、厄瓜多尔、哥伦比亚和委内瑞拉等八个拉丁美洲国家进行访问。访问期间,这些国家相继发生了反对美国的拉丁美洲政策的激烈的抗议行动。尼克松到委内瑞拉时被迫提前结束访问回国。
\mnitem{10}指炮击金门:一九五八年七月,台湾国民党当局在美国的支持下叫嚷“反攻大陆”,并不断炮击福建沿海村镇。为严惩国民党军,反对美国侵犯中国主权,人民解放军福建前线部队奉命于八月二十三日开始对国民党军金门防卫部和炮兵阵地等军事目标进行炮击,封锁了金门岛,中断国民党军的补给。九月初,美国向台湾海峡地区大量增兵,派军舰、飞机直接为国民党军运输舰护航,公然入侵中国领海。为打击美国的侵略行径,人民解放军前线部队又于九月八日对金门国民党军和海上舰艇进行全面炮击。至一九五九年一月七日,共进行七次大规模炮击,十三次空战,三次海战,击落击伤国民党军飞机三十六架,击沉击伤军舰十七艘,毙伤国民党军七千余人。
\mnitem{11}戴高乐(一八九〇——一九七〇),时任法国总理。一九五八年十二月当选为法兰西第五共和国总统。
\mnitem{12}何应钦(一八九〇——一九八七),贵州兴义人。抗日战争时期曾任国民党政府军事委员会参谋总长兼军政部部长。
\mnitem{13}这里所说的十九个国家,指当时已同中国建立外交关系的阿富汗、巴基斯坦、柬埔寨、缅甸、尼泊尔、锡兰(斯里兰卡)、也门、伊拉克、印度、印度尼西亚、阿拉伯联合共和国(一九五八年二月由埃及和叙利亚合并建立)、丹麦、芬兰、荷兰、挪威、瑞典、瑞士、列支敦士登和英国。社会主义阵营十一个国家,指朝鲜、蒙古、越南、阿尔巴尼亚、保加利亚、波兰、德意志民主共和国、捷克斯洛伐克、罗马尼亚、苏联和匈牙利。
\mnitem{14}尼赫鲁,即贾瓦哈拉尔·尼赫鲁(一八八九——一九六四),印度民族独立运动领袖。时任印度总理。
\mnitem{15}艾森豪威尔,时任美国总统。
\mnitem{16}指美国和台湾当局订立的《共同防御条约》。一九五〇年六月朝鲜战争爆发后,美国总统杜鲁门在公开宣布武装干涉朝鲜内战的同时,命令其海军第七舰队侵入台湾海峡。美国为使侵略中国领土的行为“合法化”,一九五四年十二月二日与台湾当局签署了《共同防御条约》。该条约规定:美国帮助台湾当局维持并发展武装部队;台湾遭到“武装攻击”时,“美国将采取行动”,对付“共同危险”;美国有在台湾、澎湖及其附近部署陆、海、空军的权利,还可扩及到经双方协议所决定的“其他领土”。一九五五年三月三日条约生效。一九七八年十二月十五日,美国政府就美利坚合众国和中华人民共和国建交发表的声明宣布,美台《共同防御条约》将予以终止。一九八〇年一月一日起该条约正式废除。
\mnitem{17}杜鲁门,一九四五年至一九五三年任美国总统。
\mnitem{18}一九五八年九月七日,比克利指挥美国第七舰队的几艘巡洋舰和驱逐舰,为载运军火增援金门的国民党军运输舰护航。
\mnitem{19}斯摩特,时任驻台湾美军司令。
\mnitem{20}马来亚,今属马来西亚。
\mnitem{21}阿拉伯联合共和国,简称阿联,一九五八年由埃及、叙利亚合并组成。一九六一年叙利亚脱离阿拉伯联合共和国,成立阿拉伯叙利亚共和国。一九七一年阿拉伯联合共和国改名为阿拉伯埃及共和国。
\end{maonote}
