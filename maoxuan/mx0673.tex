
\title{胜利的信念是从斗争中得来的}
\date{一九六五年十月二十日}
\thanks{这是毛泽东同志同越南民主共和国党政代表团谈话的主要部分。}
\maketitle


你们的仗打得很好,南方和北方都打得很好。全世界人民都支持你们,包括那些已经觉悟的人和一部分尚未觉悟的人。现在的世界是个不太平的世界。这并不是你们越南人到美国去进行侵略,也不是中国人到美国去进行侵略。

不久前,日本的《朝日新闻》、《读卖新闻》登载了日本记者从南越发回来的几篇报道,美国报纸说这些报道不公正,于是展开了一场争论。我说的不是日本共产党的《赤旗报》,而是日本资产阶级的报纸。这就可见舆论的方向是不利于美国的。最近,美国人民反对美国政府对越南政策的示威发展起来了。目前主要是美国的知识分子在闹。

不过,这些都是外部条件。实际上,解决问题还是靠你们打仗,当然也可以谈判。过去在日内瓦曾经谈过,但是谈了以后,美国人可以不算数。我们和蒋介石、美国都谈过。腊斯克\mnote{1}就说,美国和中国谈判的次数最多。可是,我们要啃住一条,它一定要从台湾撤走,其他问题都好解决。这条它不答应。中、美两国已经谈判十年了\mnote{2},还是重复老话。这一条,我们放松不得。美国曾想和我们交换新闻代表团。它说,从小事情开始,大的问题好解决。我们说,要从大的问题开始,小的问题才好解决。

原来你们曾按照日内瓦协议\mnote{3},把在南方的武装力量都撤出来。结果,敌人就在那里杀人,你们又重新搞起武装斗争来。一开始,你们是以政治斗争为主,武装斗争为辅,我们赞成。到了第二阶段,你们是政治斗争和武装斗争同时进行,我们也赞成。第三阶段,你们是以武装斗争为主,政治斗争为辅,我们又赞成了。依我看,敌人在逐步升级,你们也逐步升级。在今后两三年内你们可能会困难一些,但也难说,也可能不是这样,可是应该把这点估计进去。做好了一切准备,即使发生最困难的情况,也不会离原来的估计相差太远,这不是很好吗?所以,根本的就是这两条:一是争取最有利的局面;二是准备应付最坏的情况。

阿尔及利亚的经验可以供你们参考。大概是当他们打到第四五个年头时,有些领导人就有点忧愁了。那时他们的总理阿巴斯来和我们谈。他们说,阿尔及利亚的人口很少,只有一千万,已经死了一百万;敌人的军队有八十万,而他们的正规军只有三四万,连游击队算在一起不到十万。我当时对他们说,敌人一定要垮台,坚持到胜利之后,人口会增加起来的。后来经过谈判,法国撤了军,现在已撤完,只留下一点海军基地。阿尔及利亚是资产阶级领导的民族民主革命。你们和我们一样都是共产党,对于发动群众和进行人民战争的问题,你们和我们同阿尔及利亚有所不同。

在文章中讲的人民战争,有些属于具体问题,是一二十年前的事了。现在,你们已经有了一些新的情况,你们的很多方法和我们过去的不同了,应该有所不同。我们打仗也是逐步学会的,开始时打了些败仗,不像你们这样顺利。

你们和美国谈些什么问题,我还没有留意到。我只注意如何打美国人,怎样把美国人赶出去。到一定时候也可以谈判,但总是不要把调子降下来,要把调子提得高一点。要准备敌人欺骗你们。

我们支持你们取得最后的胜利。胜利的信念是打出来的,是斗争中间得出来的。比如,美国人是可以打的,这是一条经验。这条经验,只有打才能取得。美国人是可以打的,而且是可以打败的。要打破那种美国人不可打、不可以打败的神话。我们都有很多经验。你们和我们都打过日本人,你们还打过法国人,现在你们正同美国人打。

美国人训练和教育了越南人,教育了我们,也教育了全世界人民。依我看,没有美国人就是不好,这个教员不可少。要打败美国人,就要跟美国人学。马克思的著作里没有教我们怎么打美国人,列宁的书里也没有写。这主要是靠我们向美国人学。

中国人民和全世界人民支持你们。朋友愈多愈好。

\begin{maonote}
\mnitem{1}腊斯克,时任美国国务卿。
\mnitem{2}指中美大使级会谈。一九五五年四月二十三日,周恩来总理在亚非会议八国代表团团长会议上声明:中国政府愿意同美国政府谈判,讨论和缓远东紧张局势问题,特别是和缓台湾地区紧张局势问题。同年七月二十五日,中美双方就举行大使级会谈达成协议,并于八月一日在瑞士日内瓦进行首次会谈。此后由于美方缺乏诚意,会谈中断。一九五八年八月对金门炮击开始后,美国政府公开表示准备恢复会谈,双方随即于九月十五日在波兰华沙恢复会谈。迄至一九七〇年二月二十日,中美大使级会谈共举行了一百三十六次。由于美方坚持干涉中国内政的立场,会谈在和缓和消除台湾地区紧张局势问题上未取得任何进展。
\mnitem{3}日内瓦协议,指一九五四年四月二十六日至七月二十一日在瑞士日内瓦召开的讨论和平解决朝鲜问题和恢复印度支那和平问题的国际会议。中、苏、美、英、法五国参加所有两项议题的讨论。朝鲜北南双方及美、英、法以外的其他十二个侵略朝鲜北方的国家参加了朝鲜问题的讨论,越南民主共和国、老挝、柬埔寨和南越政权参加了印度支那问题的讨论。关于朝鲜问题没有达成任何协议;关于恢复印度支那和平问题,分别达成关于在印度支那三国停止敌对行动的协定和《日内瓦会议最后宣言》(总称日内瓦协议),实现了印度支那的停战。
\end{maonote}
