
\title{与日本首相田中角荣的谈话}
\date{一九七二年九月二十七日}
\thanks{这是毛泽东同志同日本内阁总理大臣、自民党总裁田中角荣谈话的一部分。}
\maketitle


(毛泽东与田中握手)

\mxsay{毛泽东:}(用日语)晚上好!

\mxsay{田中:}晚上好,毛泽东主席。

(就坐后)

\mxsay{毛泽东:}你们吵完了吗?吵吵架对你们有好处。

\mxsay{田中:}吵是吵了一些,但是已经基本上解决了问题。”不,不,我们进行了友好的谈话。

\mxsay{周恩来:}两位外长很努力。

\mxsay{田中:}是的,两位外长很努力。

\mxsay{毛泽东:}不打不成交嘛!

\mxsay{田中:}是,是。

\mxsay{毛泽东:}(对着大平外相)你把他(指姬鹏飞)\mnote{1}打败了吧?

\mxsay{大平正芳:}(日本外相)没有,我们是平等的。

\mxsay{田中:}我们进行了非常圆满的会谈。

\mxsay{毛泽东:}那就好了,你们那个“麻烦/{\mingchao 迷惑}”\mnote{2}问题是怎么解决的?

\mxsay{田中:}我们准备按中国的习惯修改。

\mxsay{毛泽东:}一些女同志不满意啊,特别是这个“美国人”\mnote{3},她是代表尼克松说话的。年轻人坚持说“添了麻烦”这样的话不够分量。因为在中国,只有像出现不留意把水溅到妇女的裙子上,表示道歉时才用这个词。

\mxsay{田中:}“{\mingchao 迷惑}”这个词汇虽然是从中国传到日本的,可是日语“{\mingchao 迷惑}”是在百感交集地道歉时,也可以使用的。

\mxsay{毛泽东:}明白了。“迷惑/{\mingchao 迷惑}”这个词你们用得好。中日有二千多年的来往。历史记载中第一次见于中国历史的是后汉嘛。

\mxsay{田中:}所以,我们一直听说日中交流的历史有二千年。

\mxsay{毛泽东:}你们到北京这么一来,全世界都战战兢兢。主要是一个苏联,一个美国,这两个大国。它们不大放心了,晓得你们在那里捣什么鬼啊。

\mxsay{田中:}美国声明支持我们到中国。

\mxsay{毛泽东:}基辛格也通知我们了,不设障碍。

\mxsay{田中:}是的,我同大平外务大臣一同去夏威夷见过美国总统尼克松。美国也承认日本来访中国是符合世界潮流的、必然发展趋势的。因此,美国支持日中两国改善关系。

\mxsay{毛泽东:}美国好一点,但也有一点不那么舒服,说是他们今年二月来了没建交\mnote{4},你们跑到他们前头去了,心里总有点不那么舒服就是了。可以几十年、百把年达不成协议,也可以在几天之内解决问题。

\mxsay{田中:}啊,对不起啊,我们发动了侵略战争,使中国受到很大的伤害。

\mxsay{毛泽东:}不要对不起啊,你们“有功”啊,为啥“有功”呢?因为你们要不是发动侵华战争的话,我们共产党怎么能够强大?我们怎么能够夺权哪?怎么能够把蒋介石打败呀?\mnote{5}

\mxsay{田中:}谢谢。

……

\mxsay{毛泽东:}(指着在座的廖承志\mnote{6})他是在日本出生的,田中先生,这次你把他领回去吧!

\mxsay{田中:}廖先生在日本很有名气,他要是参加日本参议院选举,一定会当选的。

\mxsay{毛泽东:}你在日本竞选时,角逐很激烈吧?

\mxsay{田中:}二十五年内搞了十一次选举,每次选举都要搞街头演说。不和选民握手,是很难取胜的。

\mxsay{毛泽东:}到街头去作竞选演说,谈何容易啊!在街上演说可是件苦差事,我半个世纪前也曾在长沙经常这样做,可要当心啊!你们的议会制度怎么样?

\mxsay{田中:}很费劲,一出差错,就得解散,进行重新大选。

\mxsay{毛泽东:}日本不容易呀。

\mxsay{田中:}我是否可以抽烟?

\mxsay{毛泽东:}(指着自己的雪茄)你抽不抽我的烟?

\mxsay{田中:}这个就行了,我本人已经戒烟了,但由于同周恩来总理谈判的时间长了,又抽上了。(用火柴给毛主席点烟)

\mxsay{毛泽东:} Thank you(谢谢)。声明\mnote{7}什么时候发表啊?

\mxsay{周恩来:}可能明天,今天晚上还要共同研究定稿。要搞中日两种文本,还有英文本。

\mxsay{毛泽东:}你们速度很快啊。

\mxsay{田中:}是的,只要时机一成熟,就可以得到解决。只要双方不玩弄外交手腕,诚心诚意地进行谈判,一定可以取得圆满的结果。

\mxsay{毛泽东:}现在彼此都有这个需要,这也是尼克松总统跟我讲的。他问是否彼此都有需要,我说是的。我说,我这个人现在勾结右派,名誉不好。你们国家有两个党,据说民主党比较开明,共和党比较右。我说民主党不怎么样,我不赏识,不感兴趣。我对尼克松说,你竞选的时候,我投了你一票,你还不知道啊。

这回我们也投了你的票啊。正如你讲的,你这个自民党主力不来,那怎么能解决中日复交问题呢?\mnote{8}

\mxsay{田中:}按照日本宪法的规定,内阁有权处理外交事务,而且,内阁的成员要共同对日本国民负责。所以这次我们三人来中国,谈定联合声明后要报告内阁,取得内阁的承认。

\mxsay{毛泽东:}所以有些人骂我们专门勾结右派。我说,你们日本在野党不能解决问题,解决中日复交问题还是靠自民党的政府。

……

\mxsay{田中:}看来,毛主席身体很健康,今天能见到毛主席很荣幸。

\mxsay{毛泽东:}不行了,我这个人要见上帝了。

\mxsay{周恩来:}(指靠墙的书架)他每天读很多文件,你看有这么多书。

\mxsay{毛泽东:}我是中了书毒了,离不了书,你看(指周围书架及桌上的书)这是《稼轩》,那是《楚辞集注》,没有什么礼物,把这个(楚辞集注)送给你。

(田中首相、日本外相大平正芳、内阁官方长官二阶堂进都站起来,看毛主席的各种书)

\mxsay{田中:}多谢,多谢!毛主席知识渊博,还这样用功,我不能再喊忙了,要更多地学习。祝你健康长寿。

\begin{maonote}
\mnitem{1}姬鹏飞,时任外交部部长。
\mnitem{2}“迷惑/麻烦”的风波,一九七二年九月二十五日,田中角荣首相访问北京,当天晚上在周恩来总理主持的欢迎宴会上,田中发表祝酒词,有这样一句:“{\mingchao わが国が中国国民に多大なご迷惑をおかけしたことについて,私は改めて深い反省の念を表明するものであります}。”日方提供的中文稿稿将“{\mingchao 迷惑}”直译为“麻烦”,田中的这句话翻成如下的中文:“我国给中国国民添了很大的麻烦,我对此再次表示深切反省之意。”

二十六日,针对田中首相昨晚在周总理欢迎宴会上的讲话中关于军国主义给中国人民“添了麻烦”的提法,周总理说,日本军国主义的侵略战争给中国人民带来了沉重的灾难,日本人民也深受其害,用“添麻烦”来表述,中国人民是通不过的,这句话引起了中国人民强烈的反感,因为普通的事情也可以说是“添麻烦”,“麻烦”在汉语里意思很轻。

田中解释说,从日文来说,“添麻烦”是诚心诚意地表示谢罪之意,而且包含着以后不重犯、请求原谅的意思。他表示,这个表述如果从汉语看不合适,可按中国的习惯改。

最后《中日共同声明》说:“日本方面痛感日本国过去由于战争给中国人民造成的重大损害的责任,表示深刻的反省。”
\mnitem{3}唐闻生,在美国纽约布鲁克林出生,是毛泽东晚年的重要翻译。
\mnitem{4}一九七二年二月美国总统尼克松首次访问中国,毛泽东会见了他。他同周恩来总理就两国关系正常化和双方关心的其他问题进行了会谈。二十七日,中美双方在上海发表联合公报,使中美关系开始走向正常化。但双方并未建交,直到一九七九年一月一日中美才正式建交。
\mnitem{5}毛泽东所说的“感谢”日本侵略或者日本侵略“有功”的表述,是有特定含义的。虽然他每次的谈话不尽相同,但基本意思,就是“日本帝国主义当了我们的好教员”,是说日本帝国主义的侵略在客观上起了促使中国人民觉醒的反面教员的作用。实际上,关于反面教员和反面教员作用的话,毛泽东在当时说得很多。一九六四年七月九日,毛泽东与亚洲、非洲、大洋洲一些国家和地区参加第二次亚洲经济讨论会的代表谈话中,在阐述日本侵略在客观上产生了对中国人民的教育作用时,说“日本帝国主义当了我们的好教员”。并接着说:“我们的第二个教员,帮了我们忙的是美帝国主义。第三个帮了我们忙的教员是蒋介石。”关于蒋介石的反面教员作用,他也说得很多。一九五六年在与南斯拉夫共产主义者联盟代表团谈话中说:“蒋介石是中国最大的教员,教育了全国人民,教育了我们全体党员。他用机关枪上课”;一九五八年九月五日在第十五次最高国务会议上,他指出:“没有‘蒋委员长’,六亿人民教育不过来的,单是共产党正面教育不行的。”所谓杜勒斯“是世界上最好的一个教员”,也是同样的含义。在这里,毛泽东之所以称他们是“教员”,指的是日本侵略中国,美国政府扶蒋反共和仇视、阴谋扼杀新中国,蒋介石反共独裁和屠杀人民、打内战等行径,对中国人民的“教育”作用,使中国人民认识清楚了他们的本来面目,起来与之进行斗争,是在强调他们的反面教员的作用。

通读毛泽东的上述谈话,他的话意所指是十分明确的,即在日本帝国主义侵略中国、促使了中国人民觉醒、团结和反抗的这个特殊意义上,毛泽东说了“感谢”日本侵略等话。
\mnitem{6}廖承志,廖承志于一九〇八年出生在日本东京,父亲廖仲恺是国民党革命元勋,与廖承志母亲何香凝同为国民党内重要左派人物,一九二七年留学日本,进入早稻田大学第一高等学府学习,建国年前多数时间在国统区工作,曾七次被逮捕,一九五二年任中共中央统战部主任。一九五八年任国务院外事办公室副主任,华侨事务委员会主任。二十世纪五十年代中亦曾两度出访日本,以其与日本的独特关系,建立民间中日交流渠道。文革中曾被审查其海外关系和被捕经历,一九七二年复出后,任外交顾问协助周恩来处理外交事务。
\mnitem{7}指后来于一九七二年九月二十九日发布的《中华人民共和国政府和日本国政府联合声明》。
\mnitem{8}一九七二年七月六日,田中角荣刚组阁完政府,便在东京永田町总理官邸二层办公室声称:“我要坚决实现日中邦交正常化,谈判的对手必须是可以依赖的人。毛泽东、周恩来是几十次死里逃生的创业者,从这一点看,他们是信得过的,也可以商谈的。因此,非在毛泽东、周恩来健在的时候一鼓作气实现不可。”田中角荣这次访华期间,中日双方就两国邦交正常化问题进行了会谈,于九月二十九日发表两国政府联合声明,宣告“中华人民共和国政府和日本国政府决定自一九七二年九月二十九日起建立外交关系”。
\end{maonote}
