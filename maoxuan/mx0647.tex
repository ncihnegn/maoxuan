
\title{干部参加劳动的伟大意义}
\date{一九六三年五月九日}
\thanks{这是毛泽东同志写在中共浙江省委办公厅一九六三年四月编印的《一批干部参加劳动的材料》上的批语。毛泽东在审阅这批材料时,将封页上的题目改为《浙江省七个关于干部参加劳动的好材料》。后来,这个批语连同浙江省七个材料作为中发(63)347号中共中央文件(即《前十条》〉的附件七印发。}
\maketitle


浙江省这七个材料\mnote{1},都是很好的。文字也不难看,建议发到各中央局、各省、地、县、社,给干部们阅读。可以从中选两三件向识字不多的干部宣读和讲解,以便引起他们的注意,逐步加深广大干部特别是县、社、大队、生产队四级干部对于参加生产劳动的伟大革命意义的认识,减少许多思想落后的干部的抵抗和阻力。中央曾在今年三月二十三日发出山西省昔阳县全县四级干部无例外地参加生产劳动的模范事例,并作了批语,对于这个重大问题,有些同志是注意了,例如浙江,在全省党代表大会上着重讨论了并且作了具体安排,其他地方,则反映尚少。建议各级领导同志利用适当机会,对于干部参加劳动这个极端重大的问题在今年内进行几次讨论,并普遍宣读山西昔阳县那个文件。各省、市、自治区,一定有自己的好范例,应当选出一些(不要太多)让干部学习。我们希望争取在三年内能使全国全体农村支部书记认真参加生产劳动,而在第一年,能争取有三分之一的支部书记参加劳动,那就是一个大胜利。城市工厂支部书记也应当是生产能手。阶级斗争、生产斗争和科学实验,是建设社会主义强大国家的三项伟大革命运动,是使共产党人免除官僚主义,避免修正主义和教条主义,永远立于不败之地的确实保证,是使无产阶级能够和广人劳动群众联合起来,实行民主专政的可靠保证。不然的话,让地、富、反、坏、牛鬼蛇神一齐跑了出来,而我们的干部则不闻不问,有许多人甚至敌我不分,互相勾结,被敌人腐蚀侵袭,分化瓦解,拉出去,打进来,许多工人、农民和知识分子也被敌人软硬兼施,照此办理,那就不要很多时间,少则几年、十几年,多则几十年,就不可避免地要出现全国性的反革命复辟,马列主义的党就一定会变成修正主义的党,变成法西斯党,整个中国就要改变颜色了。请同志们想一想,这是一种多么危险的情景啊!

解决这个问题是不是很困难呢?并不很困难。只要看到问题的严重性.经过调查研究收集了可靠的材料,明了了情况,下定了决心,政策和方法又都是正确的,又有政治上强有力的几个同志作为核心领导,那末,就一个公社的范围来说,有几个星期就够了,就一个县来说,有几个月也就够了,就一个省来说,分期分批,搞好搞透,大约需要一年、二年,或者更多一点时间。因为这一次社会主义教育运动是一次伟大的革命运动,不但包括阶级斗争问题,而且包括干部参加劳动的问题,而且包括用严格的科学态度,经过实验,学会在企业和事业中解决一批问题这样的工作。看起来很困难,实际上只要认真对待,并不难解决。这一场斗争是重新教育人的斗争,是重新组织革命的阶级队伍,向着正在对我们猖狂进攻的资本主义势力和封建势力作尖锐的针锋相对的斗争,把他们的反革命气焰压下去,把这些势力中间的绝大多数人改造成为新人的伟大的运动,又是干部和群众一道参加生产劳动和科学实验,使我们的党进一步成为更加光荣、更加伟大、更加正确的党,使我们的干部成为既懂政治、又懂业务、又红又专,不是浮在上面、做官当老爷、脱离群众,而是同群众打成一片,受群众拥护的真正的好干部。这一次教育运动完成以后,全国将会出现一种欣欣向荣的气象。差不多占地球四分之一的人类出现了这样的气象,我们的国际主义的贡献也就会更大了。

\begin{maonote}
\mnitem{1}这七个材料是:(一)中共平阳县城西人民公社委员会书记廖锡龙写的《我们是怎样坚持参加生产、领导生产的》;(二)钱天镇写的《应四官劳动好、工作也好》,应四官是中共宁海县越溪公社越溪大队支部书记;(三)章轶仲写的中共金华县汤溪公社汤溪大队支部书记、老劳模陈双田访问记《怎样才能更多地参加劳动?》;(四)中共桐庐县委副书记娄秉宜写的《严如湛同志三下后进队》,严如湛是中共桐庐县俞赵公社俞家大队支部副书记兼大队长;(五)中共瑞安县委书记季殿凯写的《隆山公社生产大队干部参加劳动》;(六)中共余杭县委农村工作部的调查报告《余杭县五常公社大队干部参加劳动好》;(七)中共青田县委书记袁长泽写的《五年来干部坚持种试验田的体会》。
\end{maonote}
