
\title{评国民党对战争责任问题的几种答案}
\date{一九四九年二月十八日}
\maketitle


“政府自抗战结束以后,即以和平建国方针力谋中共问题之和平解决。经过一年半之时间,一切协议皆为中共所破坏,故中共应负破坏和平之责任。今日中共反而提出所谓战犯名单,将政府负责人士尽皆列入,更要求政府先行逮捕,其蛮横无理,显而易见。中共如不改变此种作风,则和平商谈之途径,势难寻觅。”以上是一九四九年二月十三日国民党中央宣传部所发《特别宣传指示》中关于战争责任问题的全部论点。

这个论点,不是别人的,是第一名战争罪犯蒋介石的。蒋介石在其元旦声明里说:“中正为三民主义的信徒,秉承国父的遗教,本不愿在对日作战之后再继之以剿匪的军事,来加重人民的痛苦。所以抗日战事甫告结束,我们政府立即揭举和平建国的方针,更进而以政治商谈、军事调处的方法解决共党问题。不意经过了一年有半的时间,共党对于一切协议和方案都横加梗阻,使其不能依预期的步骤见诸实施。而最后更发动其全面武装叛乱,危害国家的生存。我政府迫不得已,乃忍痛动员,从事戡乱。”

在蒋介石发表这个声明的前七天,即一九四八年十二月二十五日,即有中共权威人士提出了四十三个战犯名单,赫然列在第一名的,就是这个蒋介石。战犯们又要求和,又要逃避责任,只有将责任推在共产党身上一个法子。可是这是不调和的。共产党既然应负发动战争的责任,那末,就应当惩办共产党。既然是“匪”,就应当“剿匪”。既然“发动其全面武装叛乱”,就应当“戡乱”。“剿匪”,“戡乱”,是百分之百的对,为什么可以不剿不戡了呢?为什么从一九四九年一月一日以后,一切国民党的公开文件一律将“共匪”改成了“共党”呢?

孙科觉得有些不妥,他在蒋介石发表元旦声明的同一天的晚上,发表广播演说,关于战争责任问题,提出了一个不同的论点。孙科说:“回忆三年前,当抗战胜利的初期,由于人民需要休养生息,由于国家需要积极建设,由于各党派对国家和人民的需要尚有共同的认识,我们曾经集合各方代表和社会贤达于一堂,举行过政治协商会议。经过三星期的努力,更多谢杜鲁门总统的特使马歇尔先生的善意调协,我们也曾经商定了一个和平建国纲领和解决各种争端的具体方案。假如当时我们能将各种方案及时实行,试问今日的中国应该是如何的繁荣,今天的中国人民应该是如何的幸福啊!可惜当时各方既未能完全放弃小我的利害,全国人民亦未能用最大的努力去促进这个和平运动的成功,遂致战祸复发,生灵涂炭。”

孙科比较蒋介石“公道”一点。你看,他不是如同蒋介石那样,将战争责任一塌括子推在共产党身上,而是采取了“平均地权”的办法,将责任平分给“各方”。这里也有国民党,也有共产党,也有民主同盟\mnote{1},也有社会贤达。不宁唯是,而且有“全国人民”,四亿七千五百万同胞一个也逃不了责任。蒋介石是专打共产党的板子,孙科是给各党各派无党无派全国同胞每人一板子,连蒋介石,也许还有孙科,也得挨上一板子。你看,两个国民党人,孙科和蒋介石,在这里打架。

第三个国民党人跑上来说:不然,照我的意见,责任应全归国民党。这个人的名字叫做李宗仁。一九四九年一月二十二日,李宗仁以“代总统”的身份,发表了一个声明。关于战争责任问题,他说:“在八年抗战之后,继之以三年之内战,不仅将抗战胜利后国家可能复兴之一线生机毁灭无遗,而战祸遍及黄河南北,田园庐舍悉遭摧毁荒废,无辜人民之死伤成千累万,妻离子散啼饥号寒者到处皆是。此一惨绝人寰之浩劫,实为我国内战史上空前所未有。”

李宗仁在这里出的是无头告示,他也没有说国民党应负责任,也没有说共产党或者别的方面应负责任,但是他说出了一个事实,这个“惨绝人寰的浩劫”,不是出在别的地方,而是出在“黄河南北”。查黄河以南直至长江,黄河以北直至松花江,谁在这里造成这个“惨绝人寰的浩劫”呢?难道是这里的人民和人民的军队自己打自己造成的吗?李宗仁是做过北平行营主任的,桂系的军队是和蒋系军队一道打到过山东省的沂蒙山区的\mnote{2},所以他有确实的情报,知道这种“浩劫”的地点和情况。如果说,李宗仁别的什么都不好,那末,他说出了这句老实话,总算是好的。而且他对这场战争起的名称,不叫“戡乱”或“剿匪”,而叫“内战”,这在国民党方面来说,也算得颇为别致。

根据李宗仁自己的逻辑,在同一个声明里,他说:“中共方面所提八条件,政府愿即开始商谈。”李宗仁知道八条的第一条,就是惩办战犯,而且也有他自己的大名在内。战犯的应当惩办,是“浩劫”的逻辑的结论。为了这一点,至今国民党死硬派还在吞吞吐吐地埋怨李宗仁,即所谓“毛泽东一月十四日声明所提八点为亡国条件,政府原不应接受”。

死硬派的埋怨之所以只能是吞吞吐吐,而不敢明目张胆,是有原因的。当蒋介石还没有“引退”时,死硬派原来想批驳八条,后来蒋介石一想不妥,决定不驳,大概是认为驳了就绝了路了,这是一月十九日的事情。当着一月十九日早上,张君劢从南京回到上海,发表谈话,说了“关于中共所提八项条件,政府不久即可能发布另一文告,提出答复”这句话的时候,中央社即于晚间发出通报说:“顷播沪电张君劢谈话一稿,请于电文末加注按语如下:张氏谈话中所说政府不久即发布另一文告一点,中央社记者顷自有关方面探悉,政府并无发布另一文告之拟议。”一月二十一日蒋介石发表“引退”声明,并无只字批评八条,并且把他自己的五条\mnote{3}也取消了,改变为“使领土主权克臻完整,历史文化与社会秩序不受摧残,人民生活与自由权利确有保障,在此原则之下,以致和平之功”。宪法、法统、军队等项都不敢再提了。因此,李宗仁在一月二十二日敢于承认以中共的八条为谈判基础,国民党死硬派也不敢明目张胆地出面反对,只能吞吞吐吐地说一声“政府原不应接受”。

孙科的“平均地权”政策是否坚持不变呢?也不。一九四九年二月五日孙科“迁政府于广州”以后,二月七日发表演说,关于战争责任问题,他说:“半年以来,因战祸蔓延,大局发生严重变化,人民痛苦万状。凡此种种,均系过去所犯错误、失败及不合理现象种下前因,以致有今日局势严重之后果。吾人深知中国需要三民主义。三民主义一日不能实现,则中国之问题始终不能解决。追忆本党总理二十年以前以三民主义亲自遗交本党,冀其逐步得以实行。苟获实行,绝不致演至今日不可收拾之局面。”人们请看,国民党政府的行政院长在这里,不是平分责任给一切党派和全国同胞,而是由国民党自己担负起来了。孙科将一切板子都打在国民党的屁股上,使人们觉得甚为痛快。至于共产党呢?孙院长说:“吾人试观中共能以诱惑及麻醉人民,亦无非仅以实行三民主义之民生主义一部分,即平均地权一节为号召。吾人实应深感惭愧,而加强警惕,重新检讨过去之错误。”谢谢亲爱的院长,共产党虽然尚有“诱惑及麻醉人民”的罪名,总算没有别的滔天大罪,致邀免打,获保首领及屁股而归。

孙院长的可爱,还不止此。他在同一演说里又说:“今日共党势力之蔓延,亦即系因吾人信仰之主义未能实行之故。本党在过去最大之错误,即系党内若干人士过分迷信武力,对内则争权倾轧,坐贻敌人分化离间之机会。及至八年抗战结束,本为实现和平统一千载难逢之时机,政府方面亦原有以政治方式解决国内纠纷之计划,不幸未能贯彻实施。人民于连年战乱之后,已亟待休养生息。刀兵再起,民不聊生,痛苦殊深,亦影响士气之消沉,以致军事步步失利。蒋总统俯顺民情,鉴于军事方法之未能解决问题,乃于元旦发表文告,号召和平。”好了,孙科这一名战争罪犯,没有被捕,也没有被打,即自动招供,而且忠实无误。谁是迷信武力,发动战争,及至军事方法未能解决问题,方始求和的呢?就是国民党,就是蒋介石。孙院长用字造句也很正确,他说过分迷信武力的是他们党内的“若干人士”。这一点,对于中共仅仅要求惩办若干国民党人,把他们称之为战争罪犯,而不要求惩办更多的更不是全体的国民党人,是互相一致的。

我们和孙科之间,在这个数目字上并无争论。不同的是在结论上。我们认为,对于这些“迷信武力”,使得“刀兵再起,民不聊生”的国民党的“若干人士”,必须当作战犯加以惩办。孙科则不同意这样做。他说:“现共方之迟迟不行指派代表,一味拖延,显示共方亦正迷信武力,自以为目前业已羽毛丰满,可以凭借武力征服全国,故拒绝先行停战,其用心亦极显然。余兹须郑重提出者,即为求获得永久之和平,双方必须以平等资格进行商谈,条件则应公平合理,为全国人所能接受者。”这样看来,孙院长又有些不可爱了。他似乎认为惩办战争罪犯一项条件不算公平合理。他的这些话,和二月十三日国民党宣传部的《特别宣传指示》对于战犯问题所表示的态度,是一样地吞吞吐吐,不敢明目张胆地提出反对,较之李宗仁敢于承认以惩办战犯为谈判的基础条件之一,大不相同。

但是孙院长仍旧有可爱的地方,这即是他说共产党“亦正迷信武力”,是表现在“迟迟不行指派代表”和“拒绝先行停战”这两点上,而不是如同国民党那样在一九四六年就迷信武力发动惨绝人寰的战争。夫“迟迟不行指派代表”者,是因为确定战犯名单是一件大事,要是“为全国人所能接受者”,少了,多了,都不合实际,“全国人”(但不包括战犯及其帮凶)不能接受,故须和各民主党派人民团体互相商量,以此“拖延”了一段时间,并且未能迅速指派代表,引起了孙科之流颇为不快。但是这也不能一口断定即为“亦正迷信武力”。大约不要很久,战犯名单就可公布,代表就可指派,谈判就可开始,孙院长就不能说我们“迷信武力”了。

至于“拒绝先行停战”,这是服从蒋总统元旦文告而采取的正确的态度。蒋总统元旦文告说:“只要共党一有和平的诚意,能作确切的表示,政府必开诚相见,愿与商讨停止战事、恢复和平的具体方法。”孙科的行政院,于一月十九日,做出了一个违反蒋介石上述文告的决议,说什么“立即先行无条件停战,并各指定代表进行和平商谈”。中共发言人曾于一月二十一日给了这个不通的决议以严正的批评\mnote{4}。不料该院长充耳不闻,又于二月七日乱说什么中共“拒绝先行停战”,就是表示中共“亦正迷信武力”。连蒋介石那样的战争罪犯,也知道停止战争,恢复和平,没有商谈是不可能的,孙科在这点上比蒋介石差远了。

人们知道孙科之所以成为战犯,是因为他一向赞助蒋介石发动战争,并坚持战争。直到一九四七年六月二十二日他还说:“在军事方面,只要打到底,终归可以解决。”“目前已无和谈可言,政府必须打垮共党,否则即是共党推翻国民政府。”\mnote{5}他就是国民党内迷信武力的“若干人士”之一。现在他站在一旁说风凉话,好像他并没有迷信过武力,三民主义没有实行他也不负责任。这是不忠实的。无论正国法,或者在国民党内正党法,孙科都逃不了挨板子。


\begin{maonote}
\mnitem{1}见本书第三卷\mxnote{论联合政府}{16}。
\mnitem{2}沂蒙山区,指山东省的沂山、蒙山一带地区。曾经和蒋系军队一道进攻这一地区的桂系军队是第四十六军。该军于一九四六年十月从海南岛由海上输送到青岛登陆,一九四七年二月在山东省的莱芜地区全部被歼灭。
\mnitem{3}见本卷\mxnote{中共发言人关于命令国民党反动政府重新逮捕前日本侵华军总司令冈村宁次和逮捕国民党内战罪犯的谈话}{6}。
\mnitem{4}见本卷\mxart{中共发言人评南京行政院的决议}。
\mnitem{5}这是孙科一九四七年六月二十二日在南京接见美联社、国民党中央日报和新民报的记者时发表的谈话中说的,当时孙科任国民党政府副主席。
\end{maonote}
