
\title{对上海文化大革命的指示}
\date{一九六七年二月十二日、十八日}
\thanks{这是毛泽东同志接见张春桥姚文元时的谈话纪要。}
\maketitle


\section*{一}

二月至四月是无产阶级文化大革命的关键时刻。这三个月中,文化大革命要看见眉目。

上海的工作总的方向是很好的。

上海的工人在安亭事件\mnote{1}的时候,你第一次去的时候不是只有一、两千人吗?现在已经到了一百多万人啦!说明上海的工人发动得比较成功。

\section*{二}

你们那个时候学生都不是到了码头吗?现在那些学生是否还在码头上啊?

(张、姚答:“还在”。)

那很好。以前学生和工人结合没有真结合,现在才是真结合了。

\section*{三}

我们现在这个革命,无产阶级文化大革命,这是无产阶级专政下的革命,是我们自己搞起来的。

这是因为,我们的无产阶级专政的机构中间有一部分被篡夺了,这一部分不是无产阶级的,而是资产阶级的,所以要革命。\mnote{2}

\section*{四}

一定要三结合。福建的问题不大,贵州问题也不大,内蒙古问题也不大,乱就乱一些。现在山西省有百分之五十三是革命群众,百分之二十七是部队,百分之二十是机关干部。上海应向他们学习。一月革命\mnote{3}胜利了,但二、三、四月更关键、更重要。

“怀疑一切”、“打倒一切”的人一定要走向反面,一定被人家打倒,干不了几天。\mnote{4}我们这儿还有一个单位,连副科长都不要。这种副科长都不要的人,这种人搞不了几天的。\mnote{5}

应该相信百分之九十五以上的群众,百分之九十五以上的干部是会跟着我们的,中国的小资产阶级相当多,中农占的数量很大。城市里小资产阶级、小手工业者,包括以至于小业主,这个数量相当大。只要我们善于领导,他们也是会跟着我们走的。我们要相信大多数。

一个大学生,领导一个市,刚刚毕业,有的大学生还没有毕业,就管一个上海市是很难的。我看当个大学校长也不行。当个大学校长,学校很复杂,你是一个刚刚毕业或还没有毕业的人,学校情况很复杂。照我看,当一个系主任也不行。系主任总要有一点学问吧!你这个学问还没有学完,大学刚刚毕业,学问还不多,而且没有教书的经验,没有管理一个系的经验。搞个系主任,我们已经培养了一批助教,讲师,原来的领导干部,总要选些人出来。这些老的人,也不能够都不要。恐怕周谷城\mnote{6}不行了吧!周谷城再教书不行了吧!

\section*{五}

巴黎公社是一八七一年成立的,到现在诞生九十六年了。如果巴黎公社不是失败而是胜利了,那据我看,现在已经变成资产阶级的公社了,因为法国资产阶级不允许法国的工人阶级掌握政权这么久。这是巴黎公社。再一个苏联的政权的形式,苏维埃政权一出来,列宁当时很高兴,是工农兵的伟大创造,是无产阶级专政的新形式,但当时他没有想到这种形式,工农兵可以用,资产阶级也能用,赫鲁晓夫也可以用。从现在的苏维埃看来,已从列宁的苏维埃变成赫鲁晓夫的苏维埃了。

英国是君主制,不是有国王吗?美国是总统制,本质是一样的,都是资产阶级专政。还有很多例子:南越伪政权是总统制,它旁边柬埔寨西哈努克是国王,哪一个比较好一点呢?恐怕是西哈努克比较好一点;印度是总统制,它旁边的尼泊尔是王国,这两个,哪一个好一些?看起来还是王国比较印度好点,就现在的表现来看啊。中国古代是三皇五帝,周朝叫王,秦朝叫皇帝,太平天国叫天王,唐高宗叫天皇。你看,名称变来变去。我们不能看名称,问题不在名称,而在实际,不在形式,而在内容。总统制、国王制、君主制,所谓资产阶级民主制,这些都是形式。我们不在名称,而在实际;不在形式,而在内容。总统这名称在英文里和校长是一个名称,是一个词,好象校长就低得多,总统就高得多,在英文里是一样。

名称不宜改得太多。还有历史上的王莽,这个人是最喜欢改名字的了。他一当皇帝就把所有的官职,如现在有人不喜欢那个“长”啊,都统统改了,把全国的县名统统改了,有点象红卫兵把北京街道名称全改了差不多。他改了后仍不记得,还是记老名子,王莽皇帝下诏书都困难了,改得连县名都不知道了,那么下诏书他就要写上,譬如咸阳是陕西一个县,改成祁酉,诏书里就写上“祁酉即咸阳”,把老名称写在诏书里面,这样,使公文来往非常麻烦。

话剧这形式,中国可以用,外国也可以用;无产阶级可以用,资产阶级也可以用。主要经验是巴黎公社、苏维埃。也可以设想,中华人民共和国,两个阶级都可以用,资产阶级可以用,无产阶级也可以用。如我们被推翻了,资产阶级上台,他们也可以不改名字,还叫中华人民共和国,但不是无产阶级专政,是资产阶级专政。和苏联一样,他都不改,还叫苏联共产党,还叫苏维埃共和国。这个问题主要是看哪一个阶级掌握政权。谁掌权,这是根本的问题。所以,是不是咱们还是稳当一点好,不要改名字了。

各省、市都叫“人民公社”,这样就发生改变政体、国家体制、国号问题。是否要改成“中华人民公社”呢?中华人民共和国的主席是否改变成中华人民公社的主任或叫社长?出来这个问题,紧跟着改。不但出来这个问题,如大小都要改,就发生外国承认不承认的问题。因为改变国号,外国的大使都作废了,重新换大使,重新承认。这个问题,我估计苏联是不会承认的,他不敢承认,因为承认了会给苏联带来麻烦,怎么中国出了个中华人民公社问题?资产阶级国家可能承认。如果都叫公社,那么党怎么办呢?党放入在那里呢?因为公社里有党员、非党员、公社的委员里有党员和非党员,那么党委放在那里呢?总得有个党嘛!有个核心嘛!不管叫什么,叫共产党也好,叫社会民主党也好,叫社会民主工党也好,叫国民党也好,叫一贯道\mnote{7}也好,它总得有个党,一贯道也是个党。公社总要有个党,公社能不能代替党呢?我看还是不要改名字吧!不要叫公社吧!还是按照老的办法,将来还是要人民代表大会,还是选举人民委员会。这些名字改来改去,都是形式上的改变,不解决内容问题。现在建立的临时权力机构是不是还叫革命委员会?大学里还是叫文化革命委员会好,因为这是十六条\mnote{8}所规定的。

上海人民很喜欢人民公社啊!很喜欢这个名字,怎么办啊?你们是不是回去商量一下,想几个办法,第一种是不改名字,仍用“人民公社”,这个办法有好处,也有缺点。好处是可以保护上海人民的革命热情,缺点是全国只有你们一家,那不是很孤立吗?现在又不可以在《人民日报》上登载。一登,全国都要叫。否则,人家有意见:为什么只准上海叫,不准我们叫?这样不好办。第二种是全国都改,那么就要发生上面的问题,改变政体,改变国号、旗号,承认不承认的问题。第三个办法,上海改一下,和全国一致了。可以早一些改,也可以晚一点改,不一定马上改。如果大家说不想改,那么你们就叫一些时候吧!你们怎么样?能不能说得通啊?。

\begin{maonote}
\mnitem{1}安亭事件,一九六六年十一月九日,以王洪文为首的“上海工人革命造反总司令部”(简称“工总司”)成立,上海市委拒绝承认其合法性,十日,王洪文率领数千人要上北京告状,被市委限令停在上海安亭火车站,于是他们卧轨拦车。张春桥代表中央去处理“安亭事件”,认为“工总司”是革命行动,并要求市委检查错误。

十一月九日上海国棉十七厂(王洪文)、三十厂(王秀珍)、三十一厂(黄金海)、上海良工阀门厂(陈阿大)、上海玻璃机械厂(潘国平)、上海合成纤维研究所(叶昌明)等十七家工厂造反组织组成的“工总司”在文化广场召开成立大会。会议受到冲击。会前他们曾要求上海市委承认其组织、上海市长曹荻秋到会接受批判、提供宣传工具,却得到“不参加、不承认、不支持”“三不”答复。

“工总司”十日率队赴京告状。潘国平率一千人登上去北京的特快列车,王洪文率一千人登上驶往郑州的69次特快(运行中被改为602次慢车)。先后受阻于南京和安亭。当日中午12点他们将14次特快列车拦阻于安亭站,希望中央解决问题。事件当即令北京高度关注,外国媒体亦广为报道。

中共华东局第三书记韩哲一、上海副市长李干成十一日到安亭劝说。同时陈伯达发来电报:“你们的这次行动,不单影响本单位的生产,而且大大影响全国的交通”,希望“立即回到上海去,有问题就地解决”。张春桥十一日深夜乘专机抵沪,与“工总司”负责人接触,允诺回上海解决他们提出的各项要求。十二日下午大部分造反队员回沪。

十三日上午张春桥与工总司会谈前,参加的上海市委常委会再次重申“三不”决定。下午张春桥即在工总司“承认组织合法、承认上京告状是革命行动、告状后果由上海市委负责、曹荻秋公开检查、对工总司提供各方面条件”的五项条件上签字,并被印成传单全市散发。中央文革小组、毛泽东获悉消息后迅速同意了这一处理意见。十六日毛泽东说“春桥同志的处理是正确的,可以先斩后奏。先有事实,然后有概念。没有事实,怎么能形成概念?没有实际,哪能有理论?有时理论与实际是并行的。有时理论先行,但是实际总归是第一位的。工人不先把革命闹起来,哪儿来的几条规定?”
\mnitem{2}一九六七年二月二十四日,张春桥姚文元在上海“高举毛泽东思想伟大红旗进一步开展三结合夺权斗争誓师大会”上的讲话中传达本段时解释到:

照我们的理解,主席说,这次革命是无产阶级专政下的革命,这就是说我们十七年来的政权是无产阶级专政的政权。昨天红旗杂志第四期的社论里面说了这个意思。就是十七年来,在政治上说,是毛主席的革命路线占了统治地位,不是刘、邓路线占了统治地位。这十七年来,我们的工作,中华人民共和国成立以来,我们建立了无产阶级政权,这个政权是无产阶级专政的,是实行毛主席的革命路线的。现在不能说十七年来,我们的政权是资产阶级政权或者是刘、邓路线占统治地位,不能做这样的估计。对于主席这句话,前半段说是无产阶级专政下的革命,对无产阶级专政这一点是肯定的,那末后面这半段又是革命,在无产阶级专政下面怎么要革命呢?为什么要革命呢?毛主席说,那许多问题好解释了,他说,这是因为我们的无产阶级专政机构中间,有一部分被篡夺了,这一部分不是无产阶级的,而是资产阶级的,所以,就要革命!毛主席的这句话是辩证法的。他先总结了是无产阶级专政,又说还要革命。为什么要革命呢?就是因为我们的专政中间有一部分、不是全部是资产阶级的,被篡夺了,被党内走资本主义道路的当权派,有那么一小撮人篡夺了,所以说需要革命。(口号)有些地方,提出一个口号,叫做“彻底改善无产阶级政权”,这个口号是个反动的口号!为什么要彻底改善?就是,实际上是要推翻无产阶级专政,建立资本主义专政。所谓彻底改善,就是说我们的无产阶级专政不行了,要彻底改善嘛。正确的说法,只能够是部分的改善,这是讲对无产阶级专政下的革命这个问题。为什么说这个革命是我们自己搞出来的呢?事实是如此。是我们自己搞起来的,是毛主席自己发动的。为什么要自己发动这一场革命,是因为我们国家已经统一了十七年了,这十七年比较稳定,大家都很满意,觉得中国至少一百多年没有统一,四分五裂,很不稳定;那么现在呢?经过民主革命,建立中华人民共和国以后,我们国家稳定了,统一了。凡是事情总是这样,统一、稳定有它好的方面,但是也带来了问题。就是在这种稳定中间,实际上也不稳定。而且有一部分政权、党权、财权、文权被走资本主义道路的当权派篡夺了。毛主席发动这一场无产阶级文化大革命,正是为了改善无产阶级专政,加强无产阶级专政,使得我们的无产阶级专政更加巩固,能够把修正主义根子挖掉,能够使得我们的祖国——社会主义祖国不变颜色。我们感觉到,毛主席提出这样一个思想,这是个很重大的问题。譬如说,现在这个革命是无产阶级专政下的革命。从这一句话,主席常常是这样,他把许多复杂的事物,最后归结成一句话,而这一句话非常重要,能够启发我们去想各种问题。我们正确地认识这一点,对我们当前夺权斗争非常重要。如果我们不认识这一句话,我们就看不到发动这一切革命的必要性,也看不到这场斗争究竟要解决什么问题,怎么样解决这些问题?如果我们很好地懂得了主席这一句话的意思,那么,我们就能正确地按照主席的思想来办事。(口号)这是关于夺权里边,我想传达的第一点。
\mnitem{3}一月革命,一九六七年一月,造反派夺取当地党政机关权力的浪潮席卷中国各地。一月四日,上海造反派夺了《文汇报》的权,五日,又夺了《解放日报》的权,在一月六日又联合召开了“打倒以陈丕显、曹荻秋为首的上海市委大会”,八日,在毛泽东指示下,中央文革小组为中共中央、国务院起草了致上海各“革命造反团体”的贺电,并号召全国学习上海“造反派”的经验。九日,《人民日报》发表了上海“造反派”的《告上海全市人民书》,并加《编者按》传达了毛泽东的意见。十日,黑龙江造反派组成省级机关接管委员会,夺了省级机关的权。一月十四日,山西革命造反总指挥部发布“第一号通告”,宣布已于十二日夺了省委、省人委、市委、市人委等党政机关的权;该通告于同月二十五日在《人民日报》刊出,该报并发表社论“山西省无产阶级文化大革命的伟大胜利”。一月十六日,哈尔滨军工学院等二十三个单位的“红色造反者”在哈军工集会,宣布成立“哈尔滨红色造反者联合总部”,并发表夺取省、市党政财文大权的“红色造反者联合接管公告”。一月十七日,湖南长沙市造反派夺了市委、市人委的权。一月十八日,全国财贸系统造反派举行“反对经济主义”誓师大会;新华社于二十一日以“把财贸系统的大权夺回到无产阶级革命派手里”为题加以报道。同日,聂元梓部署北京大学“新北大公社”连夜派出大批人员到高教部、中宣部、统战部、中监委、中组部及北京市人委各部夺权。一月十九日,文化部被造反派夺权。同日,首都出版系统造反派向全国出版系统造反派发出紧急呼吁书,其中第一条就是“立即夺取出版毛主席著作的大权,夺取每一个出版阵地”;新华社二十一日全文播发此呼吁书,《人民日报》二十二日转载时还配发了评论员文章“出版毛主席著作的大权我们掌”。一月二十一日,广东造反派宣布夺了省委、省人委及广州市委、市人委的权,次日在《南方日报》刊登了夺权通告。同日,徐州市两大红卫兵组织“毛泽东主义红卫兵造反总部”和“八一红卫兵革命造反司令部”联合宣布接管市委、市人委和公检法一切权力。一月三十一日晚中央人民广播电台广播了《红旗》杂志一九六七年第三期社论“论无产阶级革命派的夺权斗争”,这篇社论说:“造反派向走资派夺权“这个革命的大风暴是从上海开始的。上海的革命群众把它叫做伟大的‘一月革命’”。
\mnitem{4}一九六七年二月二十四日,张春桥姚文元在上海“高举毛泽东思想伟大红旗进一步开展三结合夺权斗争誓师大会”上的讲话中传达本段时解释到:

在这次谈话时候毛主席反复的讲了“打倒一切”和“怀疑一切”这个问题。“打倒一切”、“怀疑一切”这个口号相当的普遍相当流行,这是一种无政府主义思想,是反马克思列宁主义的反毛泽东思想的是反动的。这种思想在我们革命派内部有了影响,这种思想不是我们革命造反派的思想,但是影响了我们,影响了我们一部分同志,因为在这个,这个我们理解呢,说是文化大革命里面看到的那些人哪,那种顽固劲哪,就使人发生了一种错觉,以为到处都是坏人,所以因此呢,当别人宣传打倒一切、怀疑一切的时候,自己也觉得有道理,觉得有道理,其实啊,同志们,这种口号我们只要想一想就根本不能成立,你说打倒一切嘛,那提出这口号的人,他自己就不打倒他自己,他怎么打倒一切呢,他才不打倒一切呢,他还是打倒一部分,说怀疑一切,是他真的怀疑一切吗?他也不怀疑他自己,他不怀疑他,这个口号对不对,那么应该首先怀疑怀疑我这个“怀疑一切”,对不对啊,他也不怀疑的,实际上他也是怀疑一部分。世界上没有那样的事是“打倒一切”、“怀疑一切”的,只能是这个阶级,两个阶级对垒,无产阶级打倒资产阶级,资产阶级打倒无产阶级,那能那么打倒一切呀,资产阶级打倒资产阶级呀,那是他们内部发生了矛盾,这一种口号啊,今天阻碍着我们的大联合,特别阻碍我们的三结合,如果这个问题不能够坐下来真正解决,三结合是搞不起来的,那么你看到的这个人也要怀疑,那个人也要怀疑,这个人要打倒,那个人也要打倒,那你还有什么三结合呢?这些打倒一切的人,怀疑一切的人,最后势必被人家怀疑,被人家打倒,一定是走到反面,这一点毛主席说得非常确切,他说:“一定走向反面,一定被人家打倒,干不了几天。”
\mnitem{5}一九六七年二月二十四日,张春桥姚文元在上海“高举毛泽东思想伟大红旗进一步开展三结合夺权斗争誓师大会”上的讲话中传达本段时解释到:

就是中央气象局有一个单位,那个单位啊,到我们这里讲。说我们一到上海就碰到一个问题,我们一和哪一个造反派谈话,我们问,你们那个造反派怎么样啊,他说我们那个造反派很好,一个科长都没有,(笑声)就表示他这个组织最纯洁啦,我们一说这些话主席就说啦,说:“我们这还有一个单位呀连付科长也不要”。那个单位三十多个付科长,也不是个很大的单位,一个都不要,明明的有很多同志是很好的,也不要,这个都不作阶级分析呀,而是根据职务,从担任的职务,那一级职务以上,这那里是阶级呢?阶级,有无产阶级、资产阶级,当权派有无产阶级当权派,有资产阶级当权派,科长,有赞成毛主席,拥护毛主席跟着毛主席走的,也有反对毛主席的,应该是这样分嘛。
\mnitem{6}周谷城,中国历史学家、社会活动家,原任上海市历史学会会长,反对用阶级斗争的观点来研究历史。

一九六七年二月二十四日,张春桥姚文元在上海“高举毛泽东思想伟大红旗进一步开展三结合夺权斗争誓师大会”上的讲话中传达本段时解释到:

我在主席那里谈到,我们说,一个大学生啊,还刚刚毕业,有的还没有毕业,就要管一个上海市是很难的,主席他就说,我看当个大学校长也不行!当个大学校长?这个学校很复杂呀!你是作为学校里一个刚刚毕业、还没有毕业的同学,学校里全面的情况也不了解。后来,主席甚至于讲,他说,我看呀,当个系主任也不行!当个系主任总要有点学问吧!那么,你这学生,专业还没有学完,或大学刚刚毕业,学问还不多,而且没有教书的经验,没有管理一个系的经验,他说要搞个系主任那恐怕也是不行的,我们也已经培养了一些助教、讲师及一些原来的领导的干部,所以都要这些人出来。针对大家刚才讲的,主席说:“有一些老的人,也不能够都不要。”当然啦,他也说了,周谷城恐怕不行了吧,说“周谷城啊,如果再教书不行了吧,看那一些人还行哪”,主席很关心这事。同志们,特别是青年同志啊!你们不要泄气呀!怎么主席又怎么说了这些话啊!主席又说:“青年同志很有希望”,在这次文化大革命中,他们做了很大贡献,但是现在马上来接这些班子,接省委书记、市委书记,这个班还有些困难,这有个过程,还得学习。“三结合”呢,我的理解,也是帮助青年同志学习的一种方式,过去,我们不是老是考虑怎么样培养接班人吗?这个班怎么样接呢?看来三结合不但是夺权的一种形式,而且是老带新的一种形式。大家在一块工作,学几年,七年、八年、十年。现在廿来岁,学十年,三十几岁,作省委书记,那还是很年青嘛!所以,主席对青年同志从来就是很热爱的,对青年人评价很高。但现在呢?不能够单单强调这一方面,因为我现在讲的是三结合问题,所以就不能不说到三结合的必要性,不能不说到青年工人同志,我们的学生同志,在现马上接这个班还是很困难。拿我们上海的情况来看,我们上海的革命群众的组织力量是很强大的。这事我刚才已经传达了,主席做了很高的评价。但是,我们可以想一下,如果我们的人民公社临时委员会在成立的时候,不是实行了三结合,那我看人民公社的牌子早就叫人家砸掉了。你们说没有这个危险吗?有这个危险的,因为第二个、第三个人民公社已经在那里筹备了嘛。他不要砸掉这块牌子?在外滩的人民公社委员会办公的那个地方啊,警卫连的同志他们都向警备区的首长请示,说这个牌子有人来砸,怎么办?警备区首长下了坚决的命令,谁要砸这牌子当然是反革命,抓起来!如果没有人民解放军,参加了公社的三结合,那末如果,我不是在这里过高地估计我和姚文元同志的力量。我们两个人没什么,但是因为我们两个是参加中央文革小组工作的,就这么一点身分我们参加了工作,这么一点就使得有些人要反对人民公社,不能不考虑一下。所以,在这里也可以看出来,为什么那么多人去夺市委的权就夺不过来,为什么我们只要三结合这个权就过来了呢?就是因为三结合。如果不是这样的三结合,权也是夺不过来的,我们上海市革命委员会,那么它今后也是这样,也是三结合的。我下面还讲这个三结合,它还会继续加以充实、加以提高。反正基本力量是三方为代表,这种基本形式已经形成。那我们再加以充实、加以提高,这个,市的革命委员会就会更加巩固。谁要反对这样一个组织,那就不但是所有的革命组织会起来反对它,我们的人民解放军就会来保卫上海市革命委员会。那这样子,这样一个临时的权力机关它就可以形成,是一个有威信的、有权威的,可以来领导上海的文化大革命的,这个机构,现在是有权威的,在将来会更加有权威,会更加有力量。有些人是想来试一试,想来反对,过去想反对人民公社,现在还想来反对上海市革命委员会,那么就请来试一试吧!我们也准备好啦!谁愿意要来碰一碰,我们就会等着他们,对付他们!(掌声、口号)以上这一些,这是我们讲的第二点,就是要实行三结合,三结合是主席跟我们反复讲的一个问题,先讲了要在无产阶级专政下的革命,要搞这个革命就要大联合,大联合最好的形式三结合,我们现在的问题就是全面地贯彻执行毛主席的这个指示,我们现在应该在市一级扩大这个三结合,同时在以后,各个夺权单位,需要夺权的地方,在进行夺权的时候,我们都要实行三结合。
\mnitem{7}一贯道,在中国现代史上,一贯道是组织最严密、流传最广、信徒最多、活动最猖獗、危害最严重的一个反动会道门组织,一九五〇年被取缔。
\mnitem{8}十六条,指八届十一中全会通过的《中国共产党中央委员会关于无产阶级文化大革命的决定》,简称十六条。
\end{maonote}
