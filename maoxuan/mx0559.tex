
\title{在中国共产党全国宣传工作会议上的讲话}
\date{一九五七年三月十二日}
\maketitle


各位同志!这次会议开得很好。会议中间提出了很多问题,使我们知道了很多事情。我现在就同志们所讨论的问题讲几点意见。

我们现在是处在一个社会大变动的时期。中国社会很久以来就处在大变动中间了。抗日战争时期是大变动,解放战争时期也是大变动。但是就性质来说,现在的变动比过去的变动深刻得多。我们正在建设社会主义。有几亿人口进入社会主义的改造运动。全国各个阶级的相互关系都在起变化。农业和手工业方面的小资产阶级和工商业资产阶级,都发生了变化。社会经济制度变化了,个体经济变为集体经济,资本主义私有制正在变为社会主义公有制。这样的大变动当然要反映到人们的思想上来。存在决定意识。在不同的阶级、阶层、社会集团的人们中间,对于这个社会制度的大变动,有各种不同的反映。广大人民群众热烈地拥护这个大变动,因为现实生活证明,社会主义是中国的唯一的出路。推翻旧的社会制度,建立新的社会制度,即社会主义制度,这是一场伟大的斗争,是社会制度和人的相互关系的一场大变动。应该说,情况基本上是健康的。但是,新的社会制度还刚刚建立,还需要有一个巩固的时间。不能认为新制度一旦建立起来就完全稳固了。那是不可能的。需要逐步地巩固。要使它最后巩固起来,必须实现国家的社会主义工业化,坚持经济战线上的社会主义革命,还必须在政治战线和思想战线上,进行经常的、艰苦的社会主义革命斗争和社会主义教育。除了这些以外,还要有各种国际条件的配合。在我国,巩固社会主义制度的斗争,社会主义和资本主义谁战胜谁的斗争,还要经过一个很长的历史时期。但是,我们大家都应该看到,这个社会主义的新制度是一定会巩固起来的。我们一定会建设一个具有现代工业、现代农业和现代科学文化的社会主义国家。这是我要讲的第一点。

第二点:关于我国知识分子的情况。中国究竟有多少知识分子,没有精确的统计。有人估计,各类知识分子,包括高级知识分子和普通知识分子在内,大约有五百万左右。这五百万左右的知识分子中,绝大多数人都是爱国的,爱我们的中华人民共和国,愿意为人民服务,为社会主义的国家服务。有少数知识分子对于社会主义制度是不那么欢迎、不那么高兴的。他们对社会主义还有怀疑,但是在帝国主义面前,他们还是爱国的。对于我们的国家抱着敌对情绪的知识分子,是极少数。这种人不喜欢我们这个无产阶级专政的国家,他们留恋旧社会。一遇机会,他们就会兴风作浪,想要推翻共产党,恢复旧中国。这是在无产阶级和资产阶级两条路线、社会主义和资本主义两条路线中间,顽固地要走后一条路线的人。这后一条路线,在实际上是不能实现的,所以他们实际上是准备投降帝国主义、封建主义和官僚资本主义的人。这种人在政治界、工商界、文化教育界、科学技术界、宗教界里都有,这是一些极端反动的人。这种人在五百万左右的人数中间,大约只占百分之一、二、三。绝大部分的知识分子,占五百万总数的百分之九十以上的人,都是在各种不同的程度上拥护社会主义制度的。在这些拥护社会主义制度的人的中间,有许多人对于在社会主义制度下如何工作,许多新问题如何了解,如何对待,如何答复,还不大清楚。

五百万左右的知识分子,如果拿他们对待马克思主义的态度来看,似乎可以这样说:大约有百分之十几的人,包括共产党员和党外同情分子,是比较熟悉马克思主义,并且站稳了脚跟,站稳了无产阶级立场的。就五百万的总数来说,这些人是少数,但是他们是核心,有力量。多数人想学习马克思主义,并且也学了一点,但是还不熟悉。其中有些人还有怀疑,还没有站稳脚跟,一遇风浪就会左右摇摆。在五百万总数中占大多数的这部分知识分子,还是处在一种中间的状态。坚决反对马克思主义、对于马克思主义抱着仇视态度的人,是占极少数。有一些人虽然不公开表示不赞成马克思主义,但是实际上不赞成。这种人在很长的时间内都会有的,我们应该允许他们不赞成。例如一部分唯心主义者,他们可以赞成社会主义的政治制度和经济制度,但是不赞成马克思主义的世界观。宗教界的爱国人士也是这样。他们是有神论者,我们是无神论者。我们不能强迫这些人接受马克思主义世界观。总而言之,可以这样说,五百万左右的知识分子在对待马克思主义的状况是:赞成而且比较熟悉的,占少数;反对的也占少数;多数人是赞成但是不熟悉,赞成的程度又很不相同。这里有三种立场,坚定的,动摇的,反对的三种立场。应该承认,这种状况在一个很长的时间内还会存在。如果不承认这种状况,我们就会对别人要求过高,又会把自己的任务降低。我们作宣传工作的同志有一个宣传马克思主义的任务。这个宣传是逐步的宣传,要宣传得好,使人愿意接受。不能强迫人接受马克思主义,只能说服人接受。如果在今后几个五年计划的时间内,在我们的知识分子中间能够有比较多的人接受马克思主义,能够有比较多的人通过工作和生活的实践,通过阶级斗争的实践、生产的实践、科学的实践,懂得比较多的马克思主义,这样就好了。这是我们的希望。

第三点:知识分子的改造问题。我们的国家是一个文化不发达的国家。五百万左右的知识分子对于我们这样一个大国来说,是太少了。没有知识分子,我们的工作就不能做好,所以我们要好好地团结他们。在社会主义社会里,主要的社会成员是三部分人,就是工人、农民和知识分子。知识分子是脑力劳动者。他们的工作是为人民服务的,也就是为工人农民服务的。知识分子,就大多数来说,可以为旧中国服务,也可以为新中国服务,可以为资产阶级服务,也可以为无产阶级服务。在为旧中国服务的时候,知识分子中的左翼是反抗的,中间派是摇摆的,只有右翼是坚定的。现在转到为新社会服务,就反过来了。左翼是坚定的,中间派是摇摆的(这种摇摆和过去不同,是在新社会里的摇摆),右翼是反抗的。知识分子又是教育者。我们的报纸每天都在教育人民。我们的文学艺术家,我们的科学技术人员,我们的教授、教员,都在教人民,教学生。因为他们是教育者,是当先生的,他们就有一个先受教育的任务。在这个社会制度大变动的时期,尤其要先受教育。过去几年,他们受了一些马克思主义的教育,有些人并且很用功,比以前大有进步。但是就多数人来说,用无产阶级世界观完全代替资产阶级世界观,那就还相差很远。有些人读了一些马克思主义的书,自以为有学问了,但是并没有读进去,并没有在头脑里生根,不会应用,阶级感情还是旧的。还有一些人很骄傲,读了几句书,自以为了不起,尾巴翘到天上去了,可是一遇风浪,他们的立场,比起工人和大多数农民来,就显得大不相同。前者动摇,后者坚定,前者暧昧,后者明朗。因此,如果认为教人者不需要再受教育了,不需要再学习了,如果认为社会主义改造只是要改造别人,改造地主、资本家,改造个体生产者,不要改造知识分子,那就错误了。知识分子也要改造,不仅那些基本立场还没有转过来的人要改造,而且所有的人都应该学习,都应该改造。我说所有的人,我们这些人也在内。情况是在不断地变化,要使自己的思想适应新的情况,就得学习。即使是对于马克思主义已经了解得比较多的人,无产阶级立场比较坚定的人,也还是要再学习,要接受新事物,要研究新问题。知识分子如果不把头脑里的不恰当的东西去掉,就不能担负起教育别人的任务。我们当然只能是一面教,一面学,一面当先生,一面当学生。要做好先生,首先要做好学生。许多东西单从书本上学是不成的,要向生产者学习,向工人学习,向贫农下中农学习,在学校则要向学生学习,向自己教育的对象学习。据我看,在我们的知识分子中间,多数人是愿意学习的。我们的任务是,在他们自愿学习的基础上,好心地帮助他们学习,通过适当的方式来帮助他们学习,而不要用强制的方法勉强他们学习。

第四点:知识分子同工农群众结合的问题。知识分子既然要为工农群众服务,那就首先必须懂得工人农民,熟悉他们的生活、工作和思想。我们提倡知识分子到群众中去,到工厂去,到农村去。如果一辈子都不同工人农民见面,这就很不好。我们的国家机关工作人员、文学家、艺术家、教员和科学研究人员,都应该尽可能地利用各种机会去接近工人农民。有些人可以到工厂农村去看一看,转一转,这叫“走马观花”,总比不走不看好。另外一些人可以在工厂农村里住几个月,在那里作调查,交朋友,这叫“下马看花”。还有些人可以长期住下去,比如两年、三年,或者更长一些时间,就在那里生活,叫做“安家落户”。有一些知识分子本来就是生活在工人农民里面的,例如工业技术人员本来就在工厂,农业技术人员本来就在农村。他们应该把工作做好,和工人农民打成一片。我们要把接近工农群众这件事,造成一种风气,就是说要有很多知识分子这样做。当然不能是百分之百,有些人有各种原因不能去,但是我们希望尽可能有比较多的人去。也不能大家一下子都去,可以逐步地分批地去。让知识分子直接接触工人农民,过去在延安时期曾经这样做过。那时候,延安的许多知识分子思想很乱,有各种怪议论。我们开了一次会,劝大家到群众里面去。后来许多人去了,得到很好的效果。知识分子从书本上得来的知识在没有同实践结合的时候,他们的知识是不完全的,或者是很不完全的。知识分子接受前人的经验,主要是靠读书。书当然不可不读,但是光读书,还不能解决问题。一定要研究当前的情况,研究实际的经验和材料,要和工人农民交朋友。和工人农民交朋友,这并不是一件容易的事情。现在也有一些人到工厂农村去,结果是有的有收获,有的就没有收获。这中间有一个立场问题或者态度问题,也就是世界观问题。我们提倡百家争鸣,在各个学术部门可以有许多派、许多家,可是就世界观来说,在现代,基本上只有两家,就是无产阶级一家,资产阶级一家,或者是无产阶级的世界观,或者是资产阶级的世界观。共产主义世界观就是无产阶级的世界观,它不是任何别的阶级的世界观。我们现在的大多数的知识分子,是从旧社会过来的,是从非劳动人民家庭出身的。有些人即使是出身于工人农民的家庭,但是在解放以前受的是资产阶级教育,世界观基本上是资产阶级的,他们还是属于资产阶级的知识分子。这些人,如果不把过去的一套去掉,换一个无产阶级的世界观,就和工人农民的观点不同,立场不同,感情不同,就会同工人农民格格不入,工人农民也不会把心里的话向他们讲。知识分子如果同工农群众结合,和他们做了朋友,就可以把他们从书本上学来的马克思主义变成自己的东西。学习马克思主义,不但要从书本上学,主要地还要通过阶级斗争、工作实践和接近工农群众,才能真正学到。如果我们的知识分子读了一些马克思主义的书,又在同工农群众的接近中,在自己的工作实践中有所了解,那末,我们大家就有了共同的语言,不仅有爱国主义方面的共同语言、社会主义制度方面的共同语言,而且还可以有共产主义世界观方面的共同语言。如果这样,大家的工作就一定会做得好得多。

第五点:关于整风。整风就是整顿思想作风和工作作风。共产党内的整风,在抗日战争时期进行过一次,以后在解放战争时期进行过一次,在中华人民共和国成立初期又进行过一次。现在共产党中央作出决定,准备党内在今年开始整风。党外人士可以自由参加,不愿意的就不参加。这一次整风,主要是要批判几种错误的思想作风和工作作风:一个是主观主义,一个是官僚主义,还有一个是宗派主义。这次整风的方法同抗日时期的整风一样,就是先研究一些文件,每个人在学习文件的基础上检查自己的思想和工作,开展批评和自我批评,揭发缺点和错误的方面,发扬优点和正确的方面。在整风中间,一方面要严肃认真,对于缺点和错误,一定要进行认真的而不是敷衍的批评和自我批评,而且一定要纠正;另一方面又要和风细雨,惩前毖后,治病救人,反对采取“一棍子把人打死”的办法。

我们的党是一个伟大的党,光荣的党,正确的党。这是必须肯定的。但是我们还有缺点,这个事实也要肯定。不应该肯定我们的一切,只应该肯定正确的东西;同时,也不应该否定我们的一切,只应该否定错误的东西。在我们的工作中间成绩是主要的,但是缺点和错误也还不少。因此我们要进行整风。我们自己来批评自己的主观主义、官僚主义和宗派主义,这会不会使我们的党丧失威信呢?我看不会。相反的,会增加党的威信。抗日时期的整风就是证明。它增加了党的威信,增加了同志们的威信,增加了老干部的威信,新干部也有了很大的进步。一个共产党,一个国民党,这两个党比较起来,谁怕批评呢?国民党害怕批评。它禁止批评,结果并没有挽救它的失败。共产党是不怕批评的,因为我们是马克思主义者,真理是在我们方面,工农基本群众是在我们方面。我们过去说过,整风运动是一个“普遍的马克思主义的教育运动”\mnote{1}。整风就是全党通过批评和自我批评来学习马克思主义。在整风中间,我们一定可以更多地学到一些马克思主义。

中国的改革和建设靠我们来领导。如果我们把作风整顿好了,我们在工作中间就会更加主动,我们的本事就会更大,工作就会做得更好。我们国家要有很多诚心为人民服务、诚心为社会主义事业服务、立志改革的人。我们共产党员都应该是这样的人。在从前,在旧中国,讲改革是要犯罪的,要杀头,要坐班房。但是在那些时候,有一些立志改革的人,他们无所畏惧,他们在各种困难的条件下面,出版书报,教育人民,组织人民,进行不屈不挠的斗争。人民民主专政的政权,给我国的经济和文化的迅速发展开辟了道路。我们的政权的建立还不过短短几年,人们可以看到,不论在经济方面,在文化、教育、科学方面,都已经出现了空前繁荣的局面。为了达到建设新中国的目的,对于什么困难我们共产党人也是无所畏惧的。但是仅仅依靠我们还不够。我们需要有一批党外的志士仁人,他们能够按照社会主义、共产主义的方向,同我们一起来为改革和建设我们的社会而无所畏惧地奋斗。要使几亿人口的中国人生活得好,要把我们这个经济落后、文化落后的国家,建设成为富裕的、强盛的、具有高度文化的国家,这是一个很艰巨的任务。我们所以要整风,现在要整风,将来还要整风,要不断地把我们身上的错误东西整掉,就是为了使我们能够更好地担负起这项任务,更好地同党内外的一切立志改革的志士仁人共同工作。彻底的唯物主义者是无所畏惧的,我们希望一切同我们共同奋斗的人能够勇敢地负起责任,克服困难,不要怕挫折,不要怕有人议论讥笑,也不要怕向我们共产党人提批评建议。“舍得一身剐,敢把皇帝拉下马”,我们在为社会主义、共产主义而奋斗的时候,必须有这种大无畏的精神。在共产党人方面,我们要给这些合作者创造有利的条件,要同他们建立同志式的良好的共同工作关系,要团结他们一起奋斗。

第六点:片面性问题。片面性就是思想上的绝对化,就是形而上学地看问题。对于我们的工作的看法,肯定一切或者否定一切,都是片面性的。这样看问题的人,现在在共产党里面还是不少,党外也有很多。肯定一切,就是只看到好的,看不到坏的,只能赞扬,不能批评。说我们的工作似乎一切都好,这不合乎事实。不是一切都好,还有缺点和错误。但是也不是一切都坏,这也不合乎事实。要加以分析。否定一切,就是不加分析地认为事情都做得不好,社会主义建设这样一个伟大事业,几亿人口所进行的这个伟大斗争,似乎没有什么好处可说,一团糟。许多具有这种看法的人,虽然和那些对社会主义制度心怀敌意的人还不相同,但是这种看法是很错误的,很有害的,它只会使人丧失信心。不论是用肯定一切的观点或者否定一切的观点来看我们的工作,都是错误的。对于这些片面地看问题的人,应该进行批评,当然要以惩前毖后、治病救人的态度去批评,要帮助他们。

有人说,既然要整风,要大家提意见,就必然要有片面性,提出克服片面性,好像就是不让人讲话。这种说法对不对呢?要求所有的人都不带一点片面性,这是困难的。人们总是根据自己的经验来观察问题,处理问题,发表意见,有时候就难免带上一些片面性。但是,可不可以要求人们逐步地克服片面性,要求看问题比较全面一些?我看应该这样要求。如果不是这样,不要求一天一天地、一年一年地有较多的人采用比较全面地看问题的方法,那末,我们就停滞了,我们就是肯定片面性了,就是同整风的要求背道而驰了。所谓片面性,就是违反辩证法。我们要求把辩证法逐步推广,要求大家逐步地学会使用辩证法这个科学方法。我们现在有些文章,神气十足,但是没有货色,不会分析问题,讲不出道理,没有说服力。这种文章应该逐渐减少。当着自己写文章的时候,不要老是想着“我多么高明”,而要采取和读者处于完全平等地位的态度。你参加革命的时间虽然长,讲了错话,人家还是要驳。你的架子摆得越大,人家越是不理你那一套,你的文章人家就越不爱看。我们应该老老实实地办事,对事物有分析,写文章有说服力,不要靠装腔作势来吓人。

有人说,发长篇大论可以避免片面性,写短篇的杂文就不能避免片面性。杂文是不是一定会带片面性?我在上面讲了,片面性往往是难免的,有些片面性也不是不得了。要求所有的人看问题都必须很全面,这样就会阻碍批评的发展。但是,我们还要求努力做到看问题比较全面,不管长文也好,短文也好,杂文包括在内,努力做到不是片面性的。有人说,几百字、一二千字一篇的杂文,怎么能作分析呢?我说,怎么不能呢?鲁迅不就是这样的吗?分析的方法就是辩证的方法。所谓分析,就是分析事物的矛盾。不熟悉生活,对于所论的矛盾不真正了解,就不可能有中肯的分析。鲁迅后期的杂文最深刻有力,并没有片面性,就是因为这时候他学会了辩证法。列宁有一部分文章也可以说是杂文,也有讽刺,写得也很尖锐,但是那里面就没有片面性。鲁迅的杂文绝大部分是对敌人的,列宁的杂文既有对敌人的,也有对同志的。鲁迅式的杂文可不可以用来对付人民内部的错误和缺点呢?我看也可以。当然要分清敌我,不能站在敌对的立场用对待敌人的态度来对待同志。必须是满腔热情地用保护人民事业和提高人民觉悟的态度来说话,而不能用嘲笑和攻击的态度来说话。

不敢写文章怎么办?有的人说,有文章不敢写,写了怕得罪人,怕受批评。我看这种顾虑可以消除。我们的政权是人民民主政权,这对于为人民而写作是有利的环境。百花齐放、百家争鸣的方针,对于科学和艺术的发展给了新的保证。如果你写得对,就不用怕什么批评,就可以通过辩论,进一步阐明自己正确的意见。如果你写错了,那末,有批评就可以帮助你改正,这并没有什么不好。在我们的社会里,革命的战斗的批评和反批评,是揭露矛盾,解决矛盾,发展科学、艺术,做好各项工作的好方法。

第七点:“放”还是“收”?这是个方针问题。百花齐放,百家争鸣,这是一个基本性的同时也是长期性的方针,不是一个暂时性的方针。同志们在讨论中间是不赞成收的,我看这个意见很对。党中央的意见就是不能收,只能放。

领导我们的国家可以采用两种不同的办法,或者说两种不同的方针,这就是放和收。放,就是放手让大家讲意见,使人们敢于说话,敢于批评,敢于争论;不怕错误的议论,不怕有毒素的东西;发展各种意见之间的相互争论和相互批评,既容许批评的自由,也容许批评批评者的自由;对于错误的意见,不是压服,而是说服,以理服人。收,就是不许人家说不同的意见,不许人家发表错误的意见,发表了就“一棍子打死”。这不是解决矛盾的办法,而是扩大矛盾的办法。两种方针,放还是收呢?二者必取其一。我们采取放的方针,因为这是有利于我们国家巩固和文化发展的方针。

我们准备用这个放的方针来团结几百万知识分子,改变他们现在的面貌。像我在上面所说的,我国绝大部分的知识分子是愿意进步的,愿意改造的,是可以改造的。在这里,我们所采取的方针有很大作用。知识分子的问题首先是思想问题,对于思想问题采取粗暴的办法、压制的办法,那是有害无益的。知识分子的改造,特别是他们的世界观的改变,要有一个长时期的过程。我们的同志一定要懂得,思想改造的工作是长期的、耐心的、细致的工作,不能企图上几次课,开几次会,就把人家在几十年生活中间形成的思想意识改变过来。要人家服,只能说服,不能压服。压服的结果总是压而不服。以力服人是不行的。对付敌人可以这样,对付同志,对付朋友,绝不能用这个方法。不会说服怎么办?这就要学习。我们一定要学会通过辩论的方法、说理的方法,来克服各种错误思想。

百花齐放是一种发展艺术的方法,百家争鸣是一种发展科学的方法。百花齐放、百家争鸣这个方针不但是使科学和艺术发展的好方法,而且推而广之,也是我们进行一切工作的好方法。这个方法可以使我们少犯错误。有许多事情我们不知道,因此不会解决,在辩论中间,在斗争中间,我们就会明了这些事情,就会懂得解决问题的方法。各种不同意见辩论的结果,就能使真理发展。对于那些有毒素的反马克思主义的东西,也可以采取这个方法,因为同那些反马克思主义的东西进行斗争,就会使马克思主义发展起来。这是在对立面的斗争中的发展,是合于辩证法的发展。

人们历来不是讲真善美吗?真善美的反面是假恶丑。没有假恶丑就没有真善美。真理是同谬误对立的。在人类社会和自然界,统一体总要分解为不同的部分,只是在不同的具体条件下,内容不同,形式不同罢了。任何时候,总会有错误的东西存在,总会有丑恶的现象存在。任何时候,好同坏,善同恶,美同丑这样的对立,总会有的。香花同毒草也是这样。它们之间的关系都是对立的统一,对立的斗争。有比较才能鉴别。有鉴别,有斗争,才能发展。真理是在同谬误作斗争中间发展起来的。马克思主义就是这样发展起来的。马克思主义在同资产阶级、小资产阶级的思想作斗争中发展起来,而且只有在斗争中才能发展起来。

我们主张放的方针,现在还是放得不够,不是放得过多。不要怕放,不要怕批评,也不要怕毒草。马克思主义是科学真理,不怕批评,它是批评不倒的。共产党、人民政府也是这样,也不怕批评,也批评不倒。错误的东西总会有的,并不可怕。最近一个时期,有一些牛鬼蛇神被搬上舞台了。有些同志看到这个情况,心里很着急。我说,有一点也可以,过几十年,现在舞台上这样的牛鬼蛇神都没有了,想看也看不成了。我们要提倡正确的东西,反对错误的东西,但是不要害怕人们接触错误的东西。单靠行政命令的办法,禁止人接触不正常的现象,禁止人接触丑恶的现象,禁止人接触错误思想,禁止人看牛鬼蛇神,这是不能解决问题的。当然我并不提倡发展牛鬼蛇神,我是说“有一点也可以”。某些错误东西的存在是并不奇怪的,也是用不着害怕的,这可以使人们更好地学会同它作斗争。大风大浪也不可怕。人类社会就是从大风大浪中发展起来的。

在我国,资产阶级和小资产阶级的思想,反马克思主义的思想,还会长期存在。社会主义制度在我国已经基本建立。我们已经在生产资料所有制的改造方面,取得了基本胜利,但是在政治战线和思想战线方面,我们还没有完全取得胜利。无产阶级和资产阶级之间在意识形态方面的谁胜谁负问题,还没有真正解决。我们同资产阶级和小资产阶级的思想还要进行长期的斗争。不了解这种情况,放弃思想斗争,那就是错误的。凡是错误的思想,凡是毒草,凡是牛鬼蛇神,都应该进行批判,决不能让他们自由泛滥。但是,这种批判,应该是充分说理的、有分析的、有说服力的,而不应该是粗暴的、官僚主义的,或者是形而上学的、教条主义的。

长时间以来,人们对于教条主义作过很多批判。这是应该的。但是,人们往往忽略了对于修正主义的批判。教条主义和修正主义都是违反马克思主义的。马克思主义一定要向前发展,要随着实践的发展而发展,不能停滞不前。停止了,老是那么一套,它就没有生命了。但是,马克思主义的基本原则又是不能违背的,违背了就要犯错误。用形而上学的观点来看待马克思主义,把它看成僵死的东西,这是教条主义。否定马克思主义的基本原则,否定马克思主义的普遍真理,这就是修正主义。修正主义是一种资产阶级思想。修正主义者抹杀社会主义和资本主义的区别,抹杀无产阶级专政和资产阶级专政的区别。他们所主张的,在实际上并不是社会主义路线,而是资本主义路线。在现在的情况下,修正主义是比教条主义更有害的东西。我们现在在思想战线上的一个重要任务,就是要开展对于修正主义的批判。

最后一点,第八点:各个省、市、自治区的党委应该把思想问题抓起来。这一点是在座有些同志希望我讲的。现在许多地方的党委还没有抓思想问题,或者抓得很少。这主要是因为忙。但是一定要抓。所谓“抓”,就是要把这个问题提到议事日程上,要研究。我们国内革命战争时期的大规模的急风暴雨式的群众阶级斗争已经基本结束,但是还有阶级斗争,主要是政治战线和思想战线上的阶级斗争,而且还很尖锐。思想问题现在已经成为非常重要的问题。各地党委的第一书记应该亲自出马来抓思想问题,只有重视了和研究了这个问题,才能正确地解决这个问题。各地方要召开像这次宣传会议一样的会议,讨论当地的思想工作和有关思想工作的各方面的问题。这种会不但要有党内的同志参加,而且要有党外的人参加,要有不同意见的人参加。我们这次会议的经验证明,这对于会议的进行,只有好处,没有坏处。


\begin{maonote}
\mnitem{1}见本书第三卷\mxart{论军队生产自给,兼论整风和生产两大运动的重要性}。
\end{maonote}
