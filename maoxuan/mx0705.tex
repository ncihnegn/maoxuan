
\title{部队文艺工作座谈会纪要}
\date{一九六六年四月十日}
\thanks{这是毛泽东同志审定修改的部队文艺工作座谈会纪要\mnote{1}的主要部分。}
\maketitle


一、十六年来,文化战线上存在着尖锐的阶级斗争。

事实上,在我国革命的两个阶段,即新民主主义阶段和社会主义阶段,文化战线上都存在两个阶级、两条路线的斗争,即无产阶级和资产阶级在文化战线上争夺领导权的斗争。我们党的历史上,反对“左”右倾机会主义的斗争,也都包括文化战线上的两条路线斗争。王明\mnote{2}路线是一种曾经在我们党内泛滥过的资产阶级思想。一九四二年开始的整风运动中,毛主席先在理论上彻底地批判了王明的政治路线、军事路线和组织路线;紧接着,又在理论上彻底地批判了以王明为代表的文化路线。毛主席的《新民主主义论》、《在延安文艺座谈会上的讲话》和《看了〈逼上梁山〉以后写给延安平剧院的信》\mnote{3},就是对文化战线上的两条路线斗争的最完整、最全面、最系统的历史总结,是马克思列宁主义世界观和文艺理论的继承和发展。在我国革命进入社会主义阶段以后,毛主席又发表了《关于正确处理人民内部矛盾的问题》和《在中国共产党全国宣传工作会议上的讲话》两篇著作,这是我国和各国革命思想运动、文艺运动的历史经验的最新的总结,是马克思列宁主义世界观和文艺理论的新发展。毛主席的这五篇著作,够我们无产阶级用上一个长时期了。

毛主席的前三篇著作发表到现在已经二十几年了,后两篇也已经发表将近十年了。但是,文艺界在建国以来,却基本上没有执行,被一条与毛主席思想相对立的反党反社会主义的黑线专了我们的政,这条黑线就是资产阶级的文艺思想、现代修正主义的文艺思想和所谓三十年代文艺的结合。“写真实”论、“现实主义广阔的道路”论、“现实主义的深化”论、反“题材决定”论、“中间人物”论、反“火药味”论、“时代精神汇合”论,等等,就是他们的代表性论点,而这些论点,大抵都是毛主席《在延安文艺座谈会上的讲话》中早已批判过的。电影界还有人提出所谓“离经叛道”论,就是离马克思列宁主义、毛泽东思想之经,叛人民革命战争之道。在这股资产阶级、现代修正主义文艺思想逆流的影响或控制下,十几年来,真正歌颂工农兵的英雄人物,为工农兵服务的好的或者基本上好的作品也有,但是不多;不少是中间状态的作品;还有一批是反党反社会主义的毒草。我们一定要根据党中央的指示,坚决进行一场文化战线上的社会主义大革命,彻底搞掉这条黑线。搞掉这条黑线之后,还会有将来的黑线,还得再斗争。所以,这是一场艰巨、复杂、长期的斗争,要经过几十年甚至几百年的努力。这是关系到我国革命前途的大事,也是关系到世界革命前途的大事。

过去十几年的教训是:我们抓迟了。只抓过一些个别问题,没有全盘的系统的抓起来,而只要我们不抓,很多阵地就只好听任黑线去占领,这是一条严重的教训。一九六二年十中全会\mnote{4}作出要在全国进行阶级斗争这个决定之后,文化方面的兴无灭资的斗争也就一步一步地开展起来了。

二、近三年来,社会主义的文化大革命已经出现了新的形势,革命现代京剧的兴起就是最突出的代表。从事京剧革命的文艺工作者,在以毛主席为首的党中央的领导下,以马克思列宁主义和毛泽东思想为武器,向封建阶级、资产阶级和现代修正主义文艺展开了英勇顽强的进攻,锋芒所向,使京剧这个最顽固的堡垒,从思想到形式,都发生了极大的革命,并且带动文艺界发生着革命性的变化。革命现代京剧《红灯记》《沙家浜》《智取威虎山》《奇袭白虎团》等和芭蕾舞剧《红色娘子军》、交响音乐《沙家浜》、泥塑《收租院》等,已经得到广大工农兵群众的批准,在国内外观众中,受到了极大的欢迎。这是一个创举,它将会对社会主义文化革命产生深远的影响。它有力地证明:京剧这个最顽固的堡垒也是可以攻破的,可以革命的;芭蕾舞、交响乐这种外来的古典艺术形式,也是可以加以改造,来为我们所用的,对其它艺术的革命就更应该有信心了。有人说革命现代京剧丢掉了京剧的传统,丢掉了京剧的基本功。事实恰恰相反,革命现代京剧正是对京剧传统的批判地继承,是真正的推陈出新。京剧的基本功不是丢掉了,而是不够用了,有些不能够表现新生活的,应该也必须丢掉。而为了表现新生活,正急需我们从生活中去提炼,去创造,去逐步发展和丰富京剧的基本功。同时,这些事实也有力地回击了形形色色的保守派,和所谓“票房价值”论、“外汇价值”论、“革命作品不能出口”论,等等。

近三年来,社会主义文化革命的另一个突出表现,就是工农兵在思想、文艺战线上的广泛的群众活动。从工农兵群众中,不断地出现了许多优秀的、善于从实际出发表达毛泽东思想的哲学文章;同时,还不断地出现了许多优秀的、歌颂我国社会主义革命的伟大胜利,歌颂社会主义建设各个战线上的大跃进,歌颂我们的新英雄人物,歌颂我们伟大的党,伟大的领袖英明领导的文艺作品,特别是工农兵发表在墙报、黑板报上的大量诗歌,无论内容和形式都划出了一个完全崭新的时代。

当然,这些都还只是社会主义文化革命的初步成果,是万里长征的第一步。为保卫和发展这一成果,把社会主义文化革命进行到底,还需要我们作长期的、艰苦的努力。

三、文艺战线两条道路的斗争,必然要反映到军队内部来,军队不是生活在真空里,决不可能例外。中国人民解放军是中国无产阶级专政的主要工具,是中国人民和世界革命人民的依靠和希望。没有人民的军队,就没有革命的胜利,就没有无产阶级专政,就没有社会主义,也就没有人民的一切。因此,敌人一定会从各方面破坏它,也一定会利用文艺的武器,企图对它进行思想腐蚀。而有的人却在毛主席指出文艺界十五年来基本上没有执行党的方针以后,还说部队文艺方向已经解决了,主要是提高艺术水平的问题,这种观点是错误的,是缺乏具体分析的。事实上,军队的文艺有的方向对,艺术水平也比较高;有的方向对,艺术水平低;有的在政治方向和艺术水平方面都有严重的缺点或错误;也有的是反党反社会主义的毒草。八一电影制片厂就拍摄了《抓壮丁》\mnote{5}这样的坏影片。这说明军队的文艺工作也在不同程度上受到了黑线的影响。同时,我们自己培养的真正过得硬的创作人材还比较少;创作思想问题还很多;组织上也还有些不纯。对这些问题,我们必须作出恰当的分析和解决。

四、在社会主义文化革命中解放军要起重要作用。林彪同志主持军委工作以来,对文艺工作抓得很紧,作了很多很正确的指示;在中共中央军委扩大会议《关于加强军队政治思想工作的决议》中,明确地规定部队文艺工作的任务是:“必须密切结合部队的任务和思想情况,为兴无灭资、巩固和提高战斗力服务”;军队中有一批我们自己培养的、经过革命战争锻炼的文艺骨干;也创作了一些好的作品。因此,在社会主义文化革命中,解放军一定要起应起的作用,勇敢地、坚定不移地,为贯彻执行文艺为工农兵服务、为社会主义服务的方针而斗争。

五、文化革命要有破有立,领导人要亲自抓,搞出好的样板。资产阶级有反动的所谓“创新独白”,我们要标新立异,我们的标新立异是标社会主义之新,立无产阶级之异。要努力塑造工农兵的英雄人物,这是社会主义文艺的根本任务。我们有了这样的样板,有了这方面成功的经验,才有说服力,才能巩固地占领阵地,才能打掉反动派的棍子。

在这个问题上,不要有自卑感,而应当有自豪感。

要破除对所谓三十年代文艺的迷信。那时,左翼文艺运动政治上是王明的“左倾”机会主义路线,组织上是关门主义和宗派主义,文艺思想实际上是俄国资产阶级文艺评论家别林斯基、车尔尼雪夫斯基、杜勃罗留波夫以及戏剧方面的斯坦尼斯拉夫斯基的思想,他们是俄国沙皇时代资产阶级民主主义者,他们的思想不是马克思主义,而是资产阶级思想。资产阶级民主革命,是一个剥削阶级代替另一个剥削阶级的革命,只有无产阶级的社会主义革命,才是最后消灭一切剥削阶级的革命,因此,决不能把任何一个资产阶级革命家的思想,当成我们无产阶级思想运动、文艺运动的指导方针。三十年代也有好的,那就是以鲁迅为首的战斗的左翼文艺运动。到了三十年代的中期,那时左翼的某些领导人在王明的右倾投降主义路线的影响下,背离马克思列宁主义的阶级观点,提出了“国防文学”\mnote{6}的口号。这个口号,就是资产阶级的口号,而“民族革命战争的大众文学”这个无产阶级的口号,却是鲁迅提出的。有些左翼文艺工作者,特别是鲁迅,也提出了文艺要为工农服务和工农自己创作文艺的口号,但是并没有系统地解决文艺同工农兵相结合这个根本问题。当时的左翼文艺工作者,绝大多数还是资产阶级民族民主主义者,有些人民主革命这一关就没过去,有些人没有过好社会主义这一关。

要破除对中外古典文学的迷信。斯大林是个伟大的马克思列宁主义者,他对资产阶级的现代派文艺的批评是很尖锐的,但是,他对俄国和欧洲的所谓经典著作却无批判地继承,后果不好。中国的古典文艺,欧洲(包括俄国)古典文艺,甚至美国电影,对我国文艺界的影响是不小的,有些人就当作经典,全盘接受。我们应当接受斯大林的教训。古人、外国人的东西也要研究,拒绝研究是错误的,但一定要用批判的眼光去研究,做到古为今用,外为中用。

对十月革命后出现的一批比较优秀的苏联革命文艺作品,也要有分析,不能盲目崇拜,更不要盲目的模仿。盲目的模仿不能成为艺术。文学艺术只能来源于人民生活,只有人民生活才是文学艺术的唯一源泉,古今中外的文学艺术的历史过程,证明了这一点。

世界上从来是新生力量战胜腐朽力量。我们人民解放军开头也是弱小的,终于转弱为强,战胜了美蒋反动派。面对着国内外大好的革命形势和光荣的任务,我们应该以做一个彻底的革命派而感到自豪。要有信心,有勇气,去做前人所没有做过的事,因为我们的革命,是一次最后消灭剥削阶级、剥削制度,和从根本上消除一切剥削阶级毒害人民群众的意识形态的革命。我们要在党中央和毛主席的领导下,在马克思列宁主义和毛泽东思想的指导下,去创造无愧于我们伟大的国家,伟大的党,伟大的人民,伟大的军队的社会主义的革命新文艺。这是开创人类历史新纪元的、最光辉灿烂的新文艺。

但是,要搞出好的样板决不是一件轻而易举的事。对创作中的困难,我们在战略上一定要蔑视它,而在战术上却一定要重视它。创作一部好的作品是一个艰苦的过程,抓创作的同志决不能采取老爷式的态度,决不可掉以轻心,要同创作者同甘共苦,真正下一番苦功夫。要尽可能地掌握第一手材料,不可能时也要掌握第二手材料。要不怕失败、不怕犯错误,要允许失败、允许犯错误,还要允许改正错误。要依靠群众,从群众中来,到群众中去,经过长时间的反复实践,精益求精,力求达到革命的政治内容和尽可能完美的艺术形式的统一。在实践中及时总结经验,逐步掌握各种艺术的规律。不这样,就不可能搞出好的样板。

我们应当十分重视社会主义革命和社会主义建设的题材,忽视这一点,是完全错误的。

辽沈、淮海、平津三大战役及其他重大战役的文艺创作,也要趁着领导、指挥这些战役的同志健在,抓紧搞起来。许多重要的革命历史题材和现实题材,急需我们有计划、有步骤地组织创作。《南海长城》\mnote{7}一定要拍好。《万水千山》\mnote{8}一定要改好。并通过这些创作,培养锻炼出一支真正无产阶级的文艺骨干队伍。

六、在文艺工作中,不论是领导人员,还是创作人员,都要实行党的民主集中制,提倡“群言堂”,反对“一言堂”,要走群众路线。过去有些人搞出一个作品,就逼着领导人鼓掌、点头,这是一种很坏的作风。至于抓创作的干部,对待文艺创作,应该经常记住这样两点:第一,要善于倾听广大群众的意见;第二,要善于分析这些意见,好的就吸收,不好的就不吸收。完全没有缺点的作品是没有的,只要基调还好,要指出其缺点错误,把它改好。坏作品不要藏起来,要拿出来交给群众去评论。我们不要怕群众,要坚决地相信群众,群众会给我们提出许多宝贵意见的。另外,也可以提高群众的鉴别能力。摄制一部电影要花费几十万元或者上百万元,把坏片子藏起来,白白地浪费掉了,为什么不拿出来放映,从而教育创作人员和人民群众,又可以弥补国家经济上的损失,做到思想、经济双丰收呢?影片《兵临城下》\mnote{9}演了好久,也没有人批评,《解放军报》是否可以写篇文章批评一下。

七、要提倡革命的战斗的群众性的文艺批评,打破少数所谓“文艺批评家”(即方向错误的和软弱无力的那些批评家)对文艺批评的垄断,把文艺批评的武器交给广大工农兵群众去掌握,使专门批评家和群众批评家结合起来。在文艺批评中,要加强战斗性,反对无原则的庸俗捧场。要改造文风,提倡多写通俗的短文,把文艺批评变成匕首和手榴弹,练出二百米内的硬功夫\mnote{10};当然也要写一些系统的,有理论深度的较长的文章。反对用名词术语吓人。只有这样,才能缴掉那些所谓“文艺批评家”的械。《解放军报》《解放军文艺》要开辟定期的或不定期的文艺评论专栏,对好的或者基本上好的作品要热情支持,也可以善意地指出它的缺点;对坏作品,要进行原则性的批评。对于文艺理论方面一些有代表性的错误论点,和某些人在一些什么《中国电影发展史》《中国话剧运动五十年史料集》《京剧剧目初探》之类的书中企图伪造历史、抬高自己,以及所散布的许多错误论点,都要有计划地进行彻底的批判。不要怕有人骂我们是棍子,对人家说我们简单粗暴要有分析。我们有的批评基本正确,但是分析不够,论据不充分,说服力差,应该改进。有的人是认识问题,他们先说我们简单粗暴,后来就不说了。对敌人把我们正确的批评骂做是简单粗暴,就一定要坚决顶住。文艺评论要成为经常的工作,成为开展文艺斗争的重要方法,也是党领导文艺工作的重要方法。没有正确的文艺评论,就不可能繁荣创作。

八、文艺上反对外国修正主义的斗争,不能只捉丘赫拉依\mnote{11}之类小人物。要捉大的,捉肖洛霍夫\mnote{12},要敢于碰他。他是修正主义文艺的鼻祖。他的《静静的顿河》《被开垦的处女地》《一个人的遭遇》对中国的部分作者和读者影响很大。军队是否可以组织一些人加以研究,写出有分析的、论据充分的、有说服力的批判文章。这对中国,对世界都有很大影响。对国内的作品,也应当这样做。

九、在创作方法上,要采取革命的现实主义和革命的浪漫主义相结合的方法,不要搞资产阶级的批判现实主义和资产阶级的浪漫主义。

在党的正确路线指引下涌现的工农兵英雄人物,他们的优秀品质是无产阶级阶级性的集中表现。我们要满腔热情地、千方百计地去塑造工农兵的英雄形象。要塑造典型,毛主席说:“文艺作品中反映出来的生活却可以而且应该比普通的实际生活更高,更强烈,更有集中性,更典型,更理想,因此就更带普遍性。”不要受真人真事的局限。不要死一个英雄才写一个英雄,其实,活着的英雄要比死去的英雄多得多。这就需要我们的作者从长期的生活积累中,去集中概括,创造出各种各样的典型人物来。

写革命战争,要首先明确战争的性质,我们是正义的,敌人是非正义的。作品中一定要表现我们的艰苦奋斗、英勇牺牲,但是,也一定要表现革命的英雄主义和革命的乐观主义。不要在描写战争的残酷性时,去渲染或颂扬战争的恐怖;不要在描写革命斗争的艰苦性时,去渲染或颂扬苦难。革命战争的残酷性和革命的英雄主义,革命斗争的艰苦性和革命的乐观主义,都是对立的统一,但一定要弄清楚什么是矛盾的主要方面,否则,位置摆错了,就会产生资产阶级和平主义倾向。此外,在描写人民革命战争的时候,不论是在以游击战为主,运动战为辅的阶段,还是以运动战为主的阶段,都要正确地表现党领导下的正规军、游击队和民兵的关系,武装群众和非武装群众的关系。

选择题材要深入生活,很好地调查研究,才能选对、选准。编剧要长期地、无条件地深入到火热的斗争生活中去,导演、演员、摄影、美术、作曲等人员也要深入生活,很好地进行调查研究。过去,有些作品,歪曲历史事实,不表现正确路线,专写错误路线;有些作品,写了英雄人物,但都是犯纪律的,或者塑造起一个英雄形象却让他死掉,人为地制造一个悲剧的结局;有些作品,不写英雄人物,专写中间人物,实际上是落后人物,丑化工农兵形象;而对敌人的描写,却不是暴露敌人剥削、压迫人民的阶级本质,甚至加以美化;还有些作品,则专搞谈情说爱,低级趣味,说什么“爱”和“死”是永恒主题。这些都是资产阶级的、修正主义的东西,必须坚决反对。

十、重新教育文艺干部,重新组织文艺队伍。由于历史的原因,在全国解放前,我们无产阶级在敌人的统治下培养自己的文艺工作者要困难一些。我们的文化水平比较低,我们的经验比较少,我们的许多文艺工作者,是受资产阶级的教育培养起来的,在从事革命文艺活动的过程中,有些人又经不起敌人的迫害叛变了,或者经不起资产阶级思想的腐蚀烂掉了。在根据地,我们培养过相当数量的革命文艺工作者,特别是《在延安文艺座谈会上的讲话》发表以后,他们有了正确的方向,走上同工农兵相结合的道路,在革命过程中起过积极的作用。缺点是,在全国解放后,进了大城市,许多同志没有抵抗住资产阶级思想对我们文艺队伍的侵蚀,因而有的在前进中掉队了。我们的文艺是无产阶级的文艺,是党的文艺。无产阶级的党性原则是我们区别于其他阶级的最显著标志。须知其他阶级的代表人物也是有他们的党性原则的,并且很顽强。不论是创作思想方面,组织路线方面,工作作风方面,都要坚持无产阶级的党性原则,反对资产阶级思想的侵蚀。同资产阶级思想必须划清界线,决不能和平共处。现在文艺界存在的各种问题,对大多数人来讲,是思想认识问题,是教育提高的问题。要认真学习毛主席著作,活学活用,联系思想,联系实际,带着问题学,才能真正学得懂、学得通、学到手。要长期深入生活,和工农兵相结合,提高阶级觉悟,改造思想,不为名,不为利,全心全意地为人民服务。要教育我们的同志,读一辈子马克思列宁主义和毛主席的书,革一辈子命。特别要注意保持无产阶级的晚节,一个人能保持晚节是很不容易的。

\begin{maonote}
\mnitem{1}一九六六年二月四日江青以受林彪委托的名义,在上海召开了由主管宣传、文化工作的总政治部副主任刘志坚、总政文化部长谢镗忠、副部长陈亚丁、宣传部长李曼村、秘书刘景涛、《星火燎原》编辑部编辑黎明等参加的“部队文艺工作座谈会”。座谈会纪要经毛泽东同志多次修改。一九六六年四月十日,中共中央批准《林彪同志委托江青同志召开的部队文艺工作座谈会纪要》。林彪时任中共中央副主席、中央军委副主席、国防部长,主持军委工作。一九六七年五月二十九日,人民日报全文刊发了又经毛泽东多处修改的《纪要》全文。
\mnitem{2}王明,一九二五年加入中国共产党。自一九三一年一月中共六届四中全会起,任中共中央委员、中央政治局委员。此后至一九三五年一月遵义会议前,是党内“左”倾冒险主义错误的主要代表。

一九三一年十一月到苏联担任中共驻共产国际代表。一九三七年十一月回国后,参加中共中央政治局十二月会议,会后任中共中央长江局书记,在此期间,犯有右倾投降主义错误。他相信国民党超过相信共产党,不敢放手发动群众斗争,不敢放手发展人民军队,不敢在日本占领地区扩大解放区,主张“一切经过统一战线”、“一切服从统一战线”,将抗日战争的领导权送给国民党。由于毛泽东为代表的正确路线已经在全党占领导地位,王明的这些错误只在局部地区一度产生过影响。一九三八年九月至十一月中共中央召开的扩大的六届六中全会,批判了王明的右倾投降主义错误,确立了全党独立自主地领导抗日武装斗争的方针和政策。
\mnitem{3}《新民主主义论》见毛选第二卷,文章提出“所谓新民主主义的文化,一句话,就是无产阶级领导的人民大众的反帝反封建的文化”,“民族的科学的大众的文化,就是人民大众反帝反封建的文化,就是新民主主义的文化,就是中华民族的新文化。”

《在延安文艺座谈会上的讲话》见毛选第二卷,文章提出“我们的文学艺术都是为人民大众的,首先是为工农兵的,为工农兵而创作,为工农兵所利用的。”

《看了〈逼上梁山〉以后写给延安平剧院的信》,写于一九四四年一月九日,作为附件随纪要一起转发,一九六七年五月二十五日《人民日报》曾刊发,内容不长:

“看了你们的戏,你们做了很好的工作,我向你们致谢,并请代向演员同志们致谢!历史是人民创造的,但在旧戏舞台上(在一切离开人民的旧文学旧艺术上)人民却成了渣滓,由老爷太太少爷小姐们统治着舞台,这种历史的颠倒,现在由你们再颠倒过来,恢复了历史的面目,从此旧剧开了新生面,所以值得庆贺。你们这个开端将是旧剧革命的划时期的开端,我想到这一点就十分高兴,希望你们多编多演,蔚成风气,推向全国去!”

《关于正确处理人民内部矛盾的问题》见毛选第五卷,文章提出“百花齐放、百家争鸣的方针,是促进艺术发展和科学进步的方针,是促进我国的社会主义文化繁荣的方针”,“我国社会主义和资本主义之间在意识形态方面的谁胜谁负的斗争,还需要一个相当长的时间才能解决”,“在批判教条主义的时候,必须同时注意对修正主义的批判。”

《在中国共产党全国宣传工作会议上的讲话》见毛选第五卷,文章提出“知识分子同工农群众结合的问题”,“提倡知识分子到群众中去,到工厂去,到农村去”。
\mnitem{4}指一九六二年九月二十四日至二十七日在北京举行的中共八届十中全会发布的公报,“八届十中全会指出,在无产阶级革命和无产阶级专政的整个历史时期,在由资本主义过渡到共产主义的整个历史时期(这个时期需要几十年,甚至更多的时间)存在着无产阶级和资产阶级之间的阶级斗争,存在着社会主义和资本主义这两条道路的斗争。被推翻的反动统治阶级不甘心于灭亡,他们总是企图复辟。同时,社会上还存在着资产阶级的影响和旧社会的习惯势力,存在着一部分小生产者的自发的资本主义倾向,因此,在人民中,还有一些没有受到社会主义改造的人,他们人数不多,只占人口的百分之几,但一有机会,就企图离开社会主义道路,走资本主义道路。在这些情况下,阶级斗争是不可避免的。这是马克思列宁主义早就阐明了的一条历史规律,我们千万不要忘记。这种阶级斗争是错综复杂的、曲折的、时起时伏的,有时甚至是很激烈的。这种阶级斗争,不可避免地要反映到党内来。国外帝国主义的压力和国内资产阶级影响的存在,是党内产生修正主义思想的社会根源。在对国内外阶级敌人进行斗争的同时,我们必须及时警惕和坚决反对党内各种机会主义的思想倾向。”
\mnitem{5}《抓壮丁》,国民党反动派打着抗日的招牌,在四川农村抓壮丁。某师管区负责抓壮了的卢队长,借机对老百姓敲诈勒索,奸污妇女,无恶不作。地头蛇王保长也贪赃在法,乘机大发横财。佃农姜国富为使独生子不被抓壮丁,用变卖家产的钱托地主李老栓向王保长求情。谁知李老栓为使自己的二娃子免抓壮丁,竞用姜国富的钱买通王保长将姜的独生子抓去当替身,无依无靠的姜国富无处伸冤,被逼自尽。贪得无厌的王保长又侵吞了李老栓大娃子寄给家里的一笔钱,还欺侮李老栓的儿媳妇三嫂子,由此引起一场狗咬狗的争斗。李老栓的大娃子在国民党军队里当上副官后回乡拉队伍,他封官许愿,与卢队长、王保长达成默契,要他们抓更多的壮丁,然后委任卢队长为营长、王保长为连长。虽然结尾处,受尽欺压的农民终于起来造反,以武装斗争维护自己的生存权,但本剧把佃农等一些群众也进行了丑化,从剧中看不出一点光明,整个中华民族被描写成了一团漆黑。
\mnitem{6}“国防文学”和“民族革命战争的大众文学”的口号之争,指一九三六年上海左翼文学界关于国防文学和民族革命战争的大众文学这两个口号的论争。这两个口号都是因日寇扩大对华侵略和国内阶级关系的新变化,为适应党中央关于建立抗日民族统一战线的策略要求而提出的。国防文学口号先由上海文学界地下党领导周扬提出,并由此开展了国防文学运动和国防戏剧、国防诗歌活动。民族革命战争的大众文学口号由党中央特派员冯雪峰到上海和鲁迅、胡风等商量后由胡风撰文提出的。受到主张国防文学的一些作家的指责而发生论争。鲁迅撰文提出两个口号可以“并存”,批评了主张国防文学的一些左翼领导人的关门主义、宗派主义错误。这是左翼文学界在新形势下围绕建立文艺界统一战线由于某些思想分歧而发生的论争。

鲁迅认为,“民族革命战争的大众文学”这个口号,在本身上,比“国防文学”的提法,意义更明确,更深刻,更有内容。在《论现在我们的文学运动》中,鲁迅进一步强调说:“民族革命战争的大众文学,正如无产革命文学的口号一样,大概是一个总的口号罢。在总口号之下,再提些随时应变的具体的口号,例如“国防文学”“救亡文学”“抗日文艺”……等等,我以为是无碍的。不但没有碍,并且是有益的,需要的。自然,太多了也使人头昏,浑乱。”并苦口婆心地跟徐懋庸那帮人作了解释:“中国的唯一的出路,是全国一致对日的民族革命战争。懂得这一点,则作家观察生活,处理材料,就如理丝有绪;作者可以自由地去写工人,农民,学生,强盗,娼妓,穷人,阔佬,什么材料都可以,写出来都可以成为民族革命战争的大众文学。也无需在作品的后面有意地插一条民族革命战争的尾巴,翘起来当作旗子……。”

在这里,鲁迅正确地说明了“民族革命战争的大众文学”与无产阶级革命文学的关系,而且针对左翼文学队伍中有的人忽视、放弃无产阶级领导权的错误,特别强调了统一战线中无产阶级领导权的重要意义。鲁迅同时认为,“国防文学”是“目前文学运动的具体口号之一”,这个口号“颇通俗,已经有很多人听惯,它能扩大我们政治的和文学的影响,加之它可以解释为作家在国防旗帜下联合,为广义的爱国主义的文学的缘故”,因此,“它即使曾被不正确的解释,它本身含义上的缺陷,它仍应当存在,因为存在对于抗日运动有利益。”

国防文学明显存在着右的错误和宗派主义倾向。他们否认统一战线内部的斗争,不提无产阶级在统一战线中的领导权。统一战线的‘主体’并不是特定的,‘领导权’并不是谁所专有的。各派的斗土,应该在共同的目标下,共同负起领导的责任来。他们以是否赞成国防文学作为加入联合战线的条件,宣称凡是反对、阻碍或曲解国防文学的,都是其敌人,并错误地把当时的文艺划分为国防文艺和汉奸文艺。在民族矛盾急遽上升的历史转折关头,“国防文学”的某些倡导者还不能充分认识阶级矛盾与民族矛盾的辩证关系,暴露出若干“左”的或右的不正确观点。

三十年代中期的这场争论是革命作家内部的论争,但当时却形成了几乎对垒的形势。周扬等人把持的《光明》、《文学界》等刊物主要发表“国防文学”口号的文章;《夜莺》、《现实文学》等刊物则主要发表“民族革命战争的大众文学”口号的文章。拥护“国防文学”口号的作家,成立中国文艺家协会,发表《中国文艺家协会宣言》;赞成“民族革命战争的大众文学”口号的作家,发表《中国文艺工作者宣言》,其间双方发表的论战文章,竞有四百八十篇之多!出现了“一条战线,两个阵容”的不正常的状况。
\mnitem{7}《南海长城》,一九六四年六月十九日,毛泽东和江青观看了热门话剧《南海长城》,并予以充分肯定。该剧讲述了一九六二年国庆节前夕,大陆沿海大南港民兵连长区英才,率领甜女等守岛民兵,消灭了国民党派遣特务入侵骚扰的故事。同年八月,八一电影制片厂决定将这台话剧搬上银幕,特选派本厂实力派导演严寄洲执导,同时,还聘请江青为艺术顾问。由于创作思想存在着较多的分歧,在影片筹拍和外景摄制阶段,严寄洲常常与江青发生激烈的争论,致使影片创作进程迟缓。一九七五年十月,《南海长城》重拍摄制组成立,根据新的文艺原则“三突出”标准拍摄,由刘晓庆主演,一九七六年九月中旬拍摄完成,国庆后公映了。
\mnitem{8}《万水千山》新中国第一部长征题材的电影,由八一电影制片厂于一九五九年摄制。由孙谦、成荫根据陈其通同名话剧改编。影片反映了举世闻名的中国工农红军二万五千里长征的战斗生活,表现红军指战员在长征途中经受了严酷的战争和恶劣的自然环境的种种考验,胜利地完成了长征任务。影片以严肃的现实主义态度,真实地再现了当年长征途中飞夺泸定桥、强渡大渡河、爬雪山、过草地、腊子口战役等雄伟壮观的战斗场景,以高亢激越的笔调讴歌了中国工农红军“万水千山只等闲”的革命英雄主义精神和革命乐观主义精神。但本版电影只反映了一方面军,没有反映二方面军和四方面军。毛泽东曾批评:“写了分裂主义,只写了一方面军,不写二、四方面军。草地一场,凄惨低沉,一个教导员还死了。”一九七七年严寄洲担任导演重拍了此片,增加了红二方面军和红四方面军的内容,是一部完整的红军长征史。
\mnitem{9}《兵临城下》,解放战争时期,孤守东北某城的国民党军队处于东北民主联军的包围之中。为了争取国民党非嫡系部队三六九师师长赵崇武起义,民主联军释放了被俘的赵崇武的亲信团长郑汉臣夫妇,并答应代为寻找他俩失散的孩子。郑汉臣深受感动。但他回到城里后,受到国民党嫡系部队二〇三师的参谋长钱孝正的怀疑,使他十分反感。不久,郑太太被二〇三师某连长所辱。消息传来,郑汉臣怒不可遏,欲与二〇三师拼命,被赵崇武制止。民主联军姜部长以为郑汉臣送孩子为由,乔装前往三六九师驻地,力劝赵崇武认清形势,弃旧图新。赵崇武虽有起义之意,却无决心。这时,国民党军胡高参亲临孤城督战,当晚,钱孝正命令部下包围郑汉臣的家,准备逮捕姜部长和郑汉臣。但姜部长早已转移至赵崇武家,并连夜化装出城。胡高参命令三六九师担任突围主攻任务,二〇三师执行破坏工厂和水电站计划。但三六九师突围时遭到民主联军迎头痛击,赵崇武负伤。民主联军向城内步步进逼,赵崇武深知大势已去,又看到蒋介石命令突围后将他铲除的密电,决定率部起义。他们逮捕了胡高参,击毙了钱孝正,二〇三师被迫投降,孤城宣告解放。但这部电影采用的是旧的艺术形式,过分渲染惊险情节,对地下党的工作表现的也不符合实际,过分强调敌人起义是因为敌人内部派系之争,没有突出毛泽东思想,没有突出是我强大军事威力和政策威力的结果。
\mnitem{10}二百米内的硬功夫,当时,我军在单兵战术训练上,重点放在培养战士的勇敢精神上,在二百米内掌握射击、投弹、刺杀三大基本技术。这里指具有勇敢战斗精神的文学作品。
\mnitem{11}丘赫拉依,苏联导演,曾导演《第四十一》,剧中描写了红军女战士爱上了白匪俘虏,当白匪军来救这个俘虏时,红军女战士还是枪杀了俘虏,用来展示人性的复杂。
\mnitem{12}肖洛霍夫,前苏联著名作家,作者对战争持一概否定的态度,过分地渲染了死亡和恐怖,因而产生了比较明显的资产阶级和平主义倾向,曾把伟大的反法西斯战争描绘成是“人类的悲剧”和“灾难”,是“兄弟的自相残杀”。
\end{maonote}
