
\title{对反革命不要杀,保留活证据}
\date{一九七一年十一月十四日}
\thanks{这是毛泽东同志在接见参加成都地区座谈会的人员时的讲话。}
\maketitle


红军的传统历来不能自己打自己。你\mnote{1}就是王佐\mnote{2}部队的一个兵。王佐这个人无论如何是有功劳的,而且还加入了党。就是舍不得那个山头,吃不了苦。王佐这个人被彭德怀杀了,这就不好了嘛!哪有共产党的部队打共产党的部队?五军团的季振同\mnote{3}也不该杀。人还是少杀一点好。我们对反革命,不杀,保存起来对党有益,因为他们是活证据嘛。

你把解方\mnote{4}这样的人杀了干什么?

(周恩来:那个王实味\mnote{5}也可以不杀。)

不杀,现在多好。国民党抓刘少奇\mnote{6}、审判刘少奇的人,抓陈伯达、审判陈伯达的人还活着,这些都是活证据嘛。

\begin{maonote}
\mnitem{1}指张国华,时任成都军区政治委员、军区党委第一书记、四川省革命委员会主任,中共四川省委第一书记。
\mnitem{2}王佐,与袁文才共同领导井冈山的绿林武装。毛泽东进入井冈山后,对他们进行了改造,部队进行了整编,但袁王部队和地方永新县委存在着很深的误会和矛盾,加上当地长期存在的土客矛盾,一九三〇年二月,永新县委盗用毛泽东名义将袁王二人骗至永新县城,利用彭德怀的红五军将两人杀死,袁、王被杀的后果是极其严重的,其残部无路可走,向国民党保安团投降,之后带领保安团攻打井冈山,井冈山就此失陷。后中央红军多次攻打,均因地势险峻而不能得手,中国革命的第一块根据地从此直到全国解放都一直是白区了。全国解放后,江西省委第一书记陈正人就派人来为袁文才和王佐平反。袁文才儿子袁耀烈和王佐的儿子王寿生一起被中央邀请参加了开国大典。一九五六年五月,毛泽东上井冈山时,特地将袁文才烈士的妻子谢梅香接到井冈山宾馆,亲切地称她“袁文嫂子”,向她表示亲切慰问,并一起照了相。
\mnitem{3}季振同,原任国民党二十六路军十四师师长、七十四旅旅长。一九三一年十二月十四日与赵博生、董振堂等发动了著名的宁都起义,改编为中国红军第五军团,任总指挥。一九三四年八月,由于党内“左”倾错误的影响,被误定为反革命分子,入狱监禁。十月,长征前夕,被杀于江西省瑞金县叶坪镇。
\mnitem{4}解方,曾担任中国人民志愿军参谋长,参与第一至五次战役的指挥,并作为朝中方面代表之一参加开城停战谈判,回国后历任中央军委军训部副部长,越南停战谈判顾问,南京军事学院科学研究部部长、副教育长,高等军事学院教育长、副院长,后勤学院副院长等职。早年留学日本,原为东北军部下,曾任张学良副官,在文革中被查出西安事变前后参加了张学良秘密组织的东北革命军人同志会,在一九四六年曾与吕正操、万毅、张学思等东北军旧部四十二人联名给蒋介石发电报,要求张学良回东北,被定为“东北叛党集团”成员,几乎被判死刑,后被毛泽东和周恩来制止。
\mnitem{5}王实味,作家,在延安写了杂文《野百合花》,恶毒攻击边区生活,后被国民党利用宣传我党的黑暗,被定为托派分子,一九四七年,解放战争期间,中央机关撤出延安,行军途中,因国民党飞机炸毁了关押王实味的看守所,战争状态下被晋绥公安总局秘密处死。
\mnitem{6}刘少奇,原任中共中央副主席、中华人民共和国主席。一九六八年被诊断为“肺炎杆菌性肺炎”,在七月中旬的一次发病后,虽经尽力抢救,从此丧失意识,一九六八年十月中共八届十二中全会通过《关于叛徒、内奸、工贼刘少奇罪行的审查报告》。这次全会公报,宣布了中央“把刘少奇永远开除出党,撤销其党内外的一切职务”的决议。一九六九年十月,在战备大疏散中被疏散到开封,同年十一月十二日逝世。
\end{maonote}
