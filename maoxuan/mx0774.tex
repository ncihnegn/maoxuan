
\title{同英国前首相希思的谈话}
\date{一九七四年五月二十五日}
\thanks{这是毛泽东同志同英国前首相希思\mnote{1}的谈话。}
\maketitle


\mxsay{爱德华·希思:}早上好!

\mxsay{毛泽东:}好!

\mxsay{希思:}见到你非常高兴,非常荣幸。

\mxsay{毛泽东:}谢谢你,欢迎。

\mxsay{希思:}机场的欢迎十分动人,色彩鲜艳,情绪活跃。

\mxsay{毛泽东:}嗯。(面向周恩来)为什么没有仪仗队?

\mxsay{周恩来:}因为照顾他不是现任首相,怕引起误会,使现任首相不高兴。

\mxsay{毛泽东:}我看还是要有。

\mxsay{周恩来:}走的时候加。

\mxsay{王海容\mnote{2}:}不怕得罪威尔逊\mnote{3}啊?

\mxsay{毛泽东:}不怕!(面向希思)我是投你的票的!

\mxsay{希思:}我想苏联有很多困难,有经济困难、农业困难,领导内部也有分歧。但是他们领导内部的分歧是在策略和时机问题上,而不是在长期战略的问题上。

\mxsay{毛泽东:}我看它自顾不暇。它不能对付欧洲、中东、南亚、中国、太平洋。我看它会输的。

\mxsay{希思:}但是它的军事力量却在不断增长。虽然苏联在世界许多地方遇到了问题,但它的实力正在不断得到加强。所以我们认为这是主要威胁。主席是否认为,苏联对中国不构成威胁?

\mxsay{毛泽东:}我们准备它来。但是它来了,它就垮台了呢!它只有那么几个兵,你们欧洲人那么怕!西方有一部分舆论每天都想把苏联这一股祸水引向中国。你们的老前辈张伯伦\mnote{4},包括法国的达拉第\mnote{5},就是推德国向东。

\mxsay{希思:}我当时是反对张伯伦先生的。

\mxsay{毛泽东:}我讲的主要是美国的舆论。你们英国舆论讲到苏联怎么样要进攻中国,我看到的不多,很少。

\mxsay{希思:}如果欧洲是软弱的,苏联就有可能实现对中国的企图。因而一个强大的欧洲是很重要的,它可以使苏联发愁。

\mxsay{毛泽东:}你们欧洲强大起来,我们高兴啊!

\mxsay{希思:}中苏的分歧主要是在思想方面呢,还是苏联的强权政治所致?主席如何判断苏联对中国的目的和动机?

\mxsay{毛泽东:}中苏的分歧要从一九五四年开始算起。因为一九五五年阿登纳\mnote{6}到莫斯科,赫鲁晓夫\mnote{7}就对阿登纳说,中国不得了了。阿登纳的回忆录上是这么写的。你见过阿登纳吗?

\mxsay{希思:}见过,我见过他很多次。有一次他去意大利休假时,我和他谈了一整天。阿登纳总认为苏联会企图接管欧洲。

\mxsay{毛泽东:}欧洲、亚洲、非洲,但是它力不从心。

\mxsay{希思:}它在非洲很不成功。

\mxsay{毛泽东:}丢了埃及。

\mxsay{希思:}在阿拉伯世界中,它的影响很小。

\mxsay{毛泽东:}在我们这里影响更小呢!

\mxsay{希思:}我想在你们这里,毫无影响。

\mxsay{毛泽东:}有些影响,林彪\mnote{8}就是它的人呢。

\mxsay{希思:}我能否再问主席一个问题?中美关系今年将如何发展?自从尼克松\mnote{9}总统访华以来,中美关系似乎停滞了。

\mxsay{毛泽东:}那不要紧,还是比较好的。你可不可以劝一下尼克松,帮他一个忙,叫他渡过水门\mnote{10}难关?

\mxsay{希思:}如果他当时征求我的意见,我在十八个月以前就会劝他把这件事彻底压下去,但是他那时没有问我。

\mxsay{毛泽东:}他也有缺点呢!

\mxsay{希思:}我们都有缺点。

\mxsay{毛泽东:}我的缺点更大呢!八亿人口要吃饭,工业又不发达。不能吹中国怎么样。你们英国还可以吹一下。你们算发达国家,我们是不发达的国家。要看他们年轻的这一辈怎么样。我已经接了上帝的请帖,要我去访问上帝。

\mxsay{希思:}我希望主席在相当长的时间内不要接受这个邀请。

\mxsay{毛泽东:}还没答复呢!

\mxsay{希思:}我对刚才主席所讲的很感兴趣。中国的农业生产发展了,你们的粮食几乎自给自足,工业正在开始发展,可能我们英国能在技术和技能方面向你们提供你们所需的帮助,但主席是如何鼓舞了七亿多人民团结一致这样工作的?

\mxsay{毛泽东:}说来话长。但是你们帮助我们,我们高兴。

\mxsay{希思:}好。我们始终乐于帮助你们。

\mxsay{毛泽东:}好,很好。你们的艾登\mnote{11}现在还在吗?

\mxsay{希思:}在,他很好,现在七十六、七岁了。他至今仍对外交事务、国际问题极有兴趣。

\mxsay{毛泽东:}他在苏伊士运河问题\mnote{12}上吃了亏。

\mxsay{希思:}是的,吃了大亏。

\mxsay{毛泽东:}美国人拆台。美国呢,就是手伸得太长了。你看,它伸到日本、南朝鲜、菲律宾、台湾、东南亚、南亚、伊朗、土耳其、中东、地中海、欧洲。

\mxsay{希思:}是的,这是美国当时遏制世界其他地区的意图的一部分。现在它已经意识到这是不可能的。

\mxsay{毛泽东:}那么怕共产主义干什么呢?我看,欧洲、亚洲,包括日本,都不要吵架。要吵呢,可以,不要大吵。

\mxsay{希思:}我完全同意。

\mxsay{毛泽东:}美国人骂了我们二十多年。

\mxsay{希思:}美国人和你们的关系是又爱又恨。现在他们害怕你们的心理减少了,也就更爱你们了。

\mxsay{毛泽东:}怕得要死!基辛格\mnote{13}第一次到北京,好像中国人要吃他。他自己说,第一次很紧张,第二次也还有点紧张,第三次不紧张了。但是我们对美国人比较放心。

\mxsay{希思:}我们欧洲对此很高兴。主席对日本比较放心吗?

\mxsay{毛泽东:}对。

\mxsay{希思:}你们是否相信日本人所说的和平意图?

\mxsay{毛泽东:}在可以估计到的一段时间内是信的。将来很难说。但是我们不怕你们欧洲。

\mxsay{希思:}没有理由要怕欧洲。

\mxsay{毛泽东:}过去怕呢!

\mxsay{希思:}那是很久以前的事了。

\mxsay{毛泽东:}没有冤仇。过去不仅英国,还有法国、意大利、德国、奥匈帝国……,八国联军。

\mxsay{周恩来:}还有俄国、日本、美国,八个国家代表十二国,那是一九〇〇年。

\mxsay{毛泽东:}都成了历史了。你们剩下一个香港问题\mnote{14}。现在我想谈谈香港及其前途。香港是割让的,九龙是租借的,还有二十四年。

\mxsay{希思:}希望香港平稳交接。

\mxsay{毛泽东:}这也是我想要的,而且一定要办到!我们现在也不谈(交接细节)。到时候怎么办,我们再商量吧。(转向总理)九七年时你我都不在了,那是他们\mnote{15}的事了。

\begin{maonote}
\mnitem{1}爱德华·希思(一九一六年一二〇〇五年),曾于一九七〇年六月十九日至一九七四年三月四 日任英国首相。在任期间,中英两国于一九七二年建交。
\mnitem{2}王海容,一九四二年生,时任外交部礼宾司副司长。
\mnitem{3}威尔逊指詹姆斯·哈罗德·威尔逊,一九一六年生,工党领袖,一九七四年至一九七六年任英国首相。
\mnitem{4}张伯伦,(一八六九一一九四〇),指亚瑟·涅维尔·张伯伦,英国保守党领袖,一九三七年至一九四〇年任英国首相。一九三八年九月,代表英国政府同法国、德国、意大利政府首脑签订《慕尼黑协定》,采取纵容德、意法西斯侵略的“绥靖”政策。一九三九年第二次世界大战爆发后不久,张伯伦下台。
\mnitem{5}指爱德华·达拉第,一九三八年至一九四〇年任法国政府总理兼国防部长。在此期间,执行纵容德、意法西斯侵略的“绥靖”政策。一九三八年九月,代表法国政府签订《慕尼黑协定》。一九四〇年德国侵占法国后被监禁,一九四五年获释。
\mnitem{6}阿登纳,一九五五年时任德意志联邦共和国总理。
\mnitem{7}赫鲁晓夫,一九五五年时任苏共中央第一书记。
\mnitem{8}林彪(一九〇七——一九七一),湖北黄冈人。一九二五年加入中国共产党。一九五八年五月在中共八届五中全会上被增选为中共中央副主席、政治局常务委员。一九五九年任中央军委副主席、国防部长,主持中央军委工作。在九届二中全会上主张设国家主席(毛泽东主席明确表示要改变国家体制不设国家主席),并组织人企图压服中央,犯了错误,被毛泽东主席识破,对其进行了警告和批评,并等待其认错达一年之久(从一九七〇年九月到一九七一年九月),不料,其子林立果狂妄自大,趁毛泽东南巡之时,妄图谋杀毛泽东主席,事情败露后,九月十三日夜,林立果挟制林彪和叶群驾机逃往苏联,最后坠毁于蒙古温都尔汗,史称“九一三”事件。后,林立果制定的《“五七一”工程纪要》被发现,因此,中央认定,林彪叛国。一九七三年八月中共中央决定,开除他的党籍。
\mnitem{9}尼克松,时任美国总统。一九七二年二月,尼克松首次访问中国,并在上海同中国方面发表中美联合公报,中美关系开始走向正常化。
\mnitem{10}一九七二年六月十七日,美国共和党内为尼克松筹划竞选连任总统的一些成员,潜入华盛顿水门大厦民主党总部安置窃听器。尼克松连任总统后不久此事被揭发。一九七四年七月,美国众议院司法委员会根据收集到的证据通过弹劾尼克松的条款。一九七四年八月尼克松被迫辞职。
\mnitem{11}罗伯特·安东尼·艾登(一八九七——一九七七),英国政治家、外交家,英国政治家、外交家。第二次世界大战时期曾任英国国防委员会委员、陆军大臣、外交大臣和副首相等职。但他在一九五六年的苏伊士运河危机中企图再次强硬的对待埃及纳赛尔的时候,却招致世界一片反对,最终辞职下台。
\mnitem{12}苏伊士运河问题,一九五六年七月二十六日,埃及政府宣布将苏伊士运河公司收归国有,公司全部财产移交埃及政府。英法为夺得苏伊士运河的控制权,与以色列勾结,于一九五六年十月二十九日,对埃及发动了突然袭击,这就是第二次中东战争,又称苏伊士运河战争,苏伊士运河危机、西奈战役或卡代什行动。英法和以色列的行为受到国际社会的强烈谴责。苏联宣市:如英法不停火,苏将对英实施核攻击。美国为把英法势力挤出运河区,命令全球美军进入戒备状态,威胁英法。最终,苏伊士战争以英法侵略军的失败告终,其在中东地区数百年殖民统治的传统势力丧失殆尽。苏伊士运河危机不但导致了艾登政府的垮台,英法两国在全球的庞大帝国加速瓦解。此外,美国和苏联两个超级大国成为真正主宰中东乃至全世界的力量。
\mnitem{13}基辛格,一九二三年生。一九七三年至一九七七年任美国国务卿。一九七一年七月以美国总统尼克松的国家安全事务助理的身份第一次到北京与周恩来总理就两国关系正常化等问题举行会谈,后曾多次来中国访问。
\mnitem{14}香港问题是历史遗留的问题。香港(包括香港岛、九龙和新界)自古以来就是中国领土。一八四〇年英国发动鸦片战争,强迫清政府于一八四二年签订《南京条约》,永久割让香港岛。一八五六年英法联军发动第二次鸦片战争,一八六〇年英国迫使清政府缔结《北京条约》,永久割让九龙半岛尖端。一八九八年英国又乘列强在中国划分势力范围之机,逼迫清政府签订《展拓香港界址专条》,强行租借九龙半岛大片土地以及附近二百多个岛屿(后统称“新界”),租期九十九年,一九九七年六月三十日期满。中国人民一直反对上述三个不平等条约。

中华人民共和国成立后,中国政府的一贯立场是:香港是中国的领土,中国不承认帝国主义强加的三个不平等条约,主张在适当时机通过谈判解决这个问题,未解决前暂时维持现状。
\mnitem{15}指在座的年青同志,陪同会见的有周恩来(七十六岁)、王洪文(三十八岁)、邓小平(七十岁)、乔冠华(六十一岁)、唐闻生(三十一岁)、王海容(三十六岁)。
\end{maonote}
