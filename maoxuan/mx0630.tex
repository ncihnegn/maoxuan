
\title{争取比较长的和平时间是可能的}
\date{一九五九年十月十八日}
\thanks{这是毛泽东同志同日本共产党代表团谈话的一部分。}
\maketitle


整个国际形势是在好转。西方的高压政策、实力地位政策,或者说是冷战政策,已难以继续下去了。西方统治集团,比如美国集团、英国集团的大部分,都对打第三次世界大战抱有恐惧。如果说冷战形势有所缓和,那是因为以往的冷战政策对他们不利了,所以才有些改变,才使形势缓和下来。但情况并不是那么简单的,他们有两手:使形势有所缓和,这是一手;另外一手,当缓和对他们不利的时候,又挑起紧张局势。这就是资产阶级的两面性。他们的“爱好和平”和我们的爱好和平是不完全一致的。比如美国垄断资本同日本垄断资本,日本垄断资本同日本人民,都是有区别的。垄断资本本身也有区别,有卖国部分和其他部分的区别。即便一个集团内部,例如艾森豪威尔\mnote{1}集团内部,也不那么简单,也有两面性。我们有困难,他们也有,他们的困难比我们更多。我们利用他们的困难争取和平是可能的,而且和平时间不会是很短的。就是说,争取比较长的和平时间是可能的。你们应当利用他们内部的矛盾。他们是一定有困难的,正因为有困难才要缓和,否则何必缓和呢?西方国家之间也不一致。西方国家内爱好和平的人同他们的政府之间是有区别的,无产阶级同资产阶级之间也有区别。此外,亚洲、非洲和拉丁美洲国家的人民都是反对帝国主义控制的。当然,也有帝国主义的走狗,但广大人民是反对帝国主义的。

社会主义国家是团结的,阵营加强了。帝国主义发动战争已不是那么容易。苏联加强了,社会主义国家都靠在一块儿,而且巩固了。这样,帝国主义要发动战争就不能不考虑。

我们历来是这样估计的,整个国际形势是向好发展,不是向坏。只是有个情况也要估计到,那就是疯子要打第三次世界大战怎么办?所以,战争的情况也要估计到。和平有可能被破坏,缓和之后又会搞紧张,搞突袭,打大战,等等。对这些情况都估计到了之后,我们说总的看来,形势是向好的方面发展的。从总的情况来看,争取到十年至十五年的和平时间是可能的。假如这种情况实现了,那时要打世界大战,他们就比现在更加困难了。那时社会主义阵营的力量要比现在大得多。西方国家的矛盾,日美矛盾,由基地和条约而造成的许多矛盾,都很难解决。

如果反对修改日美“安全条约”\mnote{2}未能取得成功,条约修改了,那末,十年后这个条约还将提到日本人民的面前,那将会教育日本人民更进一步地团结起来进行斗争。

一般讲,美、英、法同德、日是有区别的。就是说,美、英、法有不少殖民地或半殖民地。它们要守卫的地方很多,从台湾和南朝鲜到土耳其,都要守。从这一点上说,它们很富。美国没有殖民地,但有很多半殖民地,这样一些国家和地区,包括南朝鲜和台湾,要保守,要维持现状,以便继续控制下去。西德和日本就不同了。德国在第一次世界大战时殖民地被搞掉了,第二次世界大战时未搞到殖民地。日本经过第一次世界大战殖民地扩大了,第二次世界大战后殖民地全被剥夺了。它们很不甘心,想继续侵占殖民地,但现在受美国的控制,还未准备好。它们想摆脱美国,再搞扩张。这些看法是否对,愿同你们交换意见。

对冒险集团要有个估计。最强大的就是美国冒险集团。他们在目前发动侵略战争还是有困难的,因为他们还未准备好。西德和日本处在美国的控制之下,要发动战争不那么容易。岸信介\mnote{3}要修改宪法第九条,是因为第九条束缚了他进行扩张,他要复活军国主义。你们党的纲领中有和平和独立的口号,这是最合乎实际情况的。

\begin{maonote}
\mnitem{1}艾森豪威尔(一八九〇——一九六九),美国共和党人,时任美国总统。
\mnitem{2}日美“安全条约”,即《日美共同合作和安全条约》,是一九五一年九月八日在旧金山签订的军事同盟条约。条约以维持日本“安全”为由,规定美国有权在日本驻扎军队和建立军事基地。一九六〇年一月,日美修改了该条约,重新签订了《日美共同合作和安全条约》。主要内容是:发展日本“抵抗武装进攻的能力”;在“应付共同危险”时,日本负有保护驻日美军的义务;美国继续有权在日本驻军和使用军事基地;“鼓励两国经济合作”等。
\mnitem{3}岸信介(一八九六——一九八七),时任日本内阁首相。
\end{maonote}
