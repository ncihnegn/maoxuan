
\title{中央军委关于支左工作十条命令}
\date{一九六七年四月六日}
\thanks{这是毛泽东同志审定的中央军委命令。}
\maketitle


一、对群众组织,无论是革命的,或者被反动分子控制的,或者情况不清楚的,都不准开枪,只能进行政治工作。

二、不准随意捕人,更不准大批捕人。对于确实查明的反革命分子要逮捕。但必须证据确凿,经过批准手续。

三、不准任意把群众组织宣布为反动组织,加以取缔。更不准把革命组织宣布为反革命组织。对于犯有某些错误的群众组织,要积极进行帮助教育。对于确实查明被反动分子控制的群众组织要作分化争取工作,孤立其最坏的头头,争取被蒙蔽的群众。必须公开宣布其为反动组织加以取缔的,要经过中央批准。

四、对于过去冲击军事机关的群众,无论左、中、右,概不追究。只对业已查明特别坏的右派头头,要追究。但应尽量缩小打击面,不能仅仅根据是否冲击军事机关这一点来划分左、中、右。

五、对待较大的群众组织采取什么态度,应就地深入调查研究,进行阶级分析;采取重大行动前,应向中央文革和全军文革请示报告。

六、一概不要进行群众性的“请罪”运动。也不要强迫群众写检讨。群众写的检讨书,退还其本人。有些长期不觉悟并且坚持错误观点的群众,不要急于要他们认错,而要给以时间,让他们在斗争中自己教育自己。不允许体罚和变相体罚。例如:戴高帽、挂黑牌、游街、罚跪等等。

七、在军队中要深入进行以毛主席为代表的无产阶级革命路线同资产阶级反动路线的两条路线斗争的教育。学习毛主席著作,必须结合两条路线的斗争。广泛搜集节录反动路线和一小撮党内走资本主义道路当权派的各种具体材料,印发到连队进行教育,使广大指战员了解他们的反动本质,进行彻底批判,肃清其恶劣影响。

八、对派到地方去或主持支左的干部,要详细交代政策。要防止赵永夫式的的反革命分子(赵永夫\mnote{1}原青海军区副司令员、是一个混进党内军内的反革命分子、他玩弄阴谋手段、篡夺军权、对革命群众组织进行残酷的武装镇压)或思想很右的人主持支左工作。

九、在支左工作中,要学会做群众工作,相信群众、依靠群众,有事同群众商量,善于采取说服教育的方式,而不应采取简单粗暴和命令方式。

十、对业已违反了上述诸条件作法的,都要立即改正,积极进行善后处理。今后,坚决按以上各条办事。

这个命令,要在我军所有机关,连队内部用电报电话迅速传达,广泛张贴。

中共中央军事委员会

一九六七年四月六日

\begin{maonote}
\mnitem{1}赵永夫,制造了一九六七年“二·二三”惨案。当时,赵永夫作为西宁驻军“联办”领导小组副组长,将青海省革命造反组织“八·一八红卫战斗队”定为反革命组织予以取缔,确定了“敌人开枪,我还击”的原则,于二月二十三日调动武装部队夺占“八·一八”掌权的青海日报社,导致部队开枪的严重事件。
\end{maonote}
