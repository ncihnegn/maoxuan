
\title{同斯诺\mnote{1}的谈话——中国人的精神面貌改变了}
\date{一九六〇年十月二十二日}
\thanks{这是毛泽东同志同美国进步作家斯诺的谈话。}
\maketitle


\mxsay{毛泽东:}你是哪一年离开延安的?

\mxsay{斯诺:}一九三九年。一般说来,我对美国的情况不十分了解。我在中国住了十二年,第二次世界大战期间,我又是驻苏联和欧洲的战地记者,只在一九五〇年我才真正住在美国。在这个期间,美国经历了巨大的变化。在技术方面,这种变化尤其巨大。

\mxsay{毛:}我们在对美国的研究方面有很大的缺点。科学院应该有一个所,至少有一个专门研究美国问题的组。要有一批人专门研究美国,注意美国各阶层的情况。除了上层的以外,你所说的那些中层、下层的舆论我们也要注意。

\mxsay{斯:}关于台湾问题,不知主席有没有看到在美国所进行的一场激烈辩论?是肯尼迪\mnote{2}和尼克松\mnote{3}两个人关于马祖和金门问题以及美国对远东政策问题所进行的辩论。

\mxsay{毛:}看了一些。

\mxsay{斯:}他们争论得那么激烈,报上经常出现马祖和金门的名字,所以有一个人就编了一个笑话,说人们已经忘了两个总统候选人的名字,忘记了他们叫尼克松和肯尼迪,而以为他们叫马祖和金门。

\mxsay{毛:}他们拿这个问题用在他们的竞选上面,这是因为美国人怕打仗。这两个岛靠大陆太近,肯尼迪就用这点想争取选票。

\mxsay{斯:}但是,这也反映了一个事实,就是说在这个问题上,美国舆论有很大的分歧。一般说来,人们对这次竞选反应冷淡,但这个问题却引起了极大的兴趣,因为很多人反对美国的现行政策,所以这是一个真正的问题。

\mxsay{毛:}尼克松有他的想法,他说非保护这两个岛不可。他也是为了争选票。这个问题使美国竞选有了生色。尼克松讲过了头,他说得好像美国政府有义务保护这两个岛。美国国务院说没有义务保护这两个岛。究竟保护不保护,要看时局,要按照当时的情况,由总统作决定。这是艾森豪威尔\mnote{4}两年前的声明。

\mxsay{斯:}有人提出这样一个问题:根据美国的宪法,新总统在十一月初选出后,还不马上上任,而要等到明年一月。他们说,如果肯尼迪当选,而中国却在十一月六日去占领金门和马祖,那时怎么办?

\mxsay{毛:}他们是这样提问题的?

\mxsay{斯:}直到明年一月,艾森豪威尔还是总统。

\mxsay{毛:}我们不是这样看待这两个岛屿的。我们对这个问题有过公开声明,就是让蒋介石守住这两个岛屿。我们也不切断他们的给养。如果他们给养不够,我们还可以接济他们。我们要的是整个台湾地区,是台湾和澎湖列岛,包括金门和马祖,这都是中国的领土。关于这两个岛屿,现在在蒋介石手里,还可以让他们守住。看来,美国竞选的人还没有查清这个材料。

\mxsay{斯:}很可能。

\mxsay{毛:}这个问题有什么可争的?我们要的不只是金门、马祖这两个岛屿,而是整个台湾和澎湖列岛。这个问题可能要搅很长的时间。现在已经搅十一年了,比方再过两个十一年吧,或更长的时间,都有可能。因为美国政府不愿意放弃台湾。它不愿意放弃,我们也不去打,我们和它谈判\mnote{5},先在日内瓦,后来在华沙。它在台湾,我们也不会打。我们要谈判解决,不要武力解决。这条道理美国政府早已知道。金门、马祖我们也不去打,我们过去有过公开声明的。因此,战争的危险是没有的,美国可以放心继续霸占台湾。今年已经是十一年了,又过十一年,再过十一年,不是三十三年了吗?也许在第三十二年,美国会放弃台湾的。

\mxsay{斯:}我想主席是要等到蒋介石的士兵都成了三条腿的人的时候。

\mxsay{毛:}主要是美国政府的问题,不是蒋介石或者其他人的问题。蒋介石的人如果成了三条腿,台湾还是有人的,还是有两条腿的人。人是能够随便找到的。

\mxsay{斯:}主席是否真的认为,美国的立场还要十一年,甚或二十二年才会有改变?美国的局势现在发展得非常快,要变起来也会是很快的。这种变化当然同外来因素有关。总之,局势会起变化的。

\mxsay{毛:}也许。你在你的文章里有一条,说我们对美国承认中国的兴趣比我们对进联合国的兴趣小,好像我们对进联合国的兴趣要大一些。我看,不是这样,不能这么讲。在联合国里,是不应该由蒋介石代表中国的,应该由我们代表,早就应该如此。但是,美国政府组织了多数国家,不让我们去。这也没有什么不好,我们并不急于进入联合国。急于要我们进入联合国的是另外一些国家,当然不包括美国在内。英国现在不得不听美国的话。但是,英国的本意可能就是你所说的那个,就是如果我们在联合国外无法无天,不如把我们套在联合国里守规矩好。有相当多的国家希望中国守规矩些。你知道,我们打过游击,野惯了。那么多规矩,令人难受,是不是?不进联合国,对我们有什么损失呢?没有什么损失。进联合国有多少好处呢?当然,有一些好处,但说有很多好处就不见得。有些国家争着要进联合国,我们不甚了解这种情绪。我们的国家就是一个“联合国”,我们的一个省就比有的国家大。

\mxsay{斯:}我也经常这样说。

\mxsay{毛:}他们对我们进行经济封锁,就和国民党那时对我们的经济封锁一样。很感谢国民党对我们的经济封锁,使得我们没有办法,只好自己搞,致使我们各个根据地都搞生产。国民党在一九三七、一九三八、一九三九年还给我们发饷,从一九四〇年开始就实行封锁。我们要感谢他们,是他们使我们自己搞生产,不依赖他们。现在美国也对我们实行封锁,这个封锁对我们有益处。

\mxsay{斯:}我记得在一九三九年的时候主席就对我说过:我们有八点要感谢国民党的。一点是,因为共产党发展太慢,所以国民党就实行经济封锁,迫使我们更快地发展;另一点是,因为共产党的军队新兵太少,所以蒋介石就把更多的人关到监狱里去,等等。后来,主席的这几点意见都被证明是正确的。事实上,愈是压迫人民,人民的力量就发展得愈快。

\mxsay{毛:}就是这个道理。

\mxsay{斯:}你在你的一篇文章\mnote{6}里说:帝国主义的规律是,反对殖民地人民争取自由的努力,捣乱、失败,再捣乱,再失败。他们对中国的封锁肯定是失败了。但是,这并没有使他们放弃这种想法。现在他们又在酝酿对古巴实行经济封锁。我认为,这也是要失败的。很难理解他们想从此得到什么结果,不过看样子,他们还是要对古巴实行禁运的。

\mxsay{毛:}现在是部分的禁运,这对古巴没有多大影响;有可能走到全面禁运,影响就比较大些。但是他们要把古巴卡死也不可能,古巴是有路可走的。现在古巴总比过去我们在延安好。

\mxsay{斯:}在你一生中,当你观察中国革命的命运时,哪个时期使你感到是最黑暗的时期?

\mxsay{毛:}我们是有过那样的时候的,比如,打败仗的时候,当然不高兴。我们打过败仗的。在长征中,我们的人员减少了,当然也不高兴了。但是总的来说,我们觉得是有希望的,不管怎样困难。那时的困难主要不在外部,而是在内部。张国焘闹分裂\mnote{7},那是最大的困难。那个困难我们也克服了。我们用适当的政策,把张国焘率领的部队争取过来了。在你来的时候,我们已经合在一起了。还有贺龙率领的部队\mnote{8},是在我们这一方面的。

\mxsay{斯:}主席一年有多少时间在北京,外地去过哪些地方?

\mxsay{毛:}我在北京一年顶多四个月。好多地方我还没有去过。你去过西安,我就没有去过。新疆、青海、宁夏、甘肃、云南、贵州,我都没有去过。我每到一个地方,愿意看一看,和工人、农民、干部谈一谈。

\mxsay{斯:}主席访问美国的可能性如何?

\mxsay{毛:}(对马海德\mnote{9})你们在医学上发明一个延长寿命的办法。我今年六十七岁,如果把我的寿命延长到九十七岁,那我就有希望访问美国了。

\mxsay{斯:}我相信主席不必等到九十七岁就会访问美国的。

\mxsay{毛:}也许我错了,你正确。我希望在这一点上是我错了。事情很难预先料到。我没有想到,抗战胜利后只要四年,解放战争就会胜利。战争开始的时候,我们只提持久战,提战争的长期性,不敢提要打几年。打了两年后,我们肯定地说,解放战争打五年就可以了,那是从一九四六年七月算起。结果三年就胜利了。

\mxsay{斯:}我记得在保安\mnote{10}的时候,曾经问过主席革命要多少时间才能取得胜利。

\mxsay{毛:}那时还没有抗日的事情。

\mxsay{斯:}我提出,两年吗?当时主席笑了一笑。后来我又问,五年?十年?当时主席说至少十年。我当时觉得那等于永久也胜利不了。

\mxsay{毛:}你太性急了。结果我们用了十二年多。

中国过去半殖民地、半封建的情况实在难以忍受。自己没有工业,粮食不够得进口,棉花得进口,工业品也得进口。百分之七十的人都是很穷的,就是工人、贫农、雇农,革命主要是靠这些人。还有百分之二十是城市里的小资产阶级,农村中的中农和富裕中农。剩下的百分之十左右就是地主、富农、城市资产阶级和知识分子,中国有文化的主要是这一部分人。还有百分之十左右的人是识字的,就是上层小资产阶级和富裕中农。百分之八十的人过去都是文盲。当然文盲不等于就没有文化,他们有些人会做很好的手工艺品,我们这房间里的雕刻可能都是不识字的人干的。

\mxsay{斯:}对,不识字不等于没有文化。我看这就是中国为什么能够在短短的十年内使全国的文化水平这样迅速提高的原因。我回到中国以后,发现已经无法区别农民、工人、城市居民、学生和所谓资产阶级出身的人了。

\mxsay{毛:}区别还是有的。只能说有所改变,生活水平也有所改善,但还没有基本改变。美国人的情况我不知道,欧洲人每人每年要吃上几十公斤的肉。中国人基本上是吃素的,肉类也吃一点,但吃得不多。要改变这种状况,至少要几十年。如果在本世纪内,就是说在今后四十年内,能够改变那就算很好了。如果加上过去的十年,就是五十年,半个世纪。再快也难。

\mxsay{斯:}主席刚才讲的正好回答了我想要问的一个问题。你们十年来的变化是非常大的。按照现在的发展速度,也许要不了那么长时间。也许我在时间计算方面又过分乐观了。所以我想问,主席认为中国需要多少时间才能在每人每年平均收入方面达到美国现有的水平?美国现在每人每年平均收入是两千美元,也就是说五千元人民币左右。

\mxsay{毛:}那就难计算。半个世纪够不够,现在也还不能回答。美国独立有一百八十五年了,美国的建设花了一百八十五年的时间。除了独立战争\mnote{11}时期的反英战争和为了解放黑奴而进行的南北内战\mnote{12}以外,你们国内就没有经历过什么战争。你们的地方特别好,气候和地理条件都很好,两个大洋保护着你们。但主要条件是你们一百七十多年前就把英国殖民者赶走了。另外还有一个条件,就是你们国内没有过封建地主所有制。

\mxsay{斯:}十年来你们一直生活在巨大变革中。你们觉得这一切都是很自然的,而感觉不出自己的发展速度。我的情况不同,我在四十年代离开中国以后,现在又回来,对我来说,这些变化是惊人的。在写报道时,除非我加上许多背景材料,否则很难使人相信我看到的一切。

\mxsay{毛:}有所变化,但是还没有基本变化。中国的变化主要是在革命方面,这方面是基本变化了。至于建设方面,现在才刚刚开始。你如果讲得神乎其神,人家就不相信你,因为不合乎事实。比如我看到外国报纸上有人说,中国现在没有苍蝇、没有蚊子了。这就不合乎事实。

所以,说有所改变是正确的,说基本改变了是不符合事实的。你说人的精神面貌改变了,这是合乎事实的。革命工作的结果,把人解放出来了。至于第二个革命,就是产业革命或者说经济革命,过去的十年才是开始。

我们的基本情况就是一穷二白。所谓穷就是生活水平低。为什么生活水平低呢?因为生产力水平低。什么是生产力呢?除人力以外就是机器。工业、农业都要机械化,工业、农业要同时发展。所谓“白”,就是文盲还没有完全消灭,不但是识字的问题,还有提高科学水平的问题。有很多科学项目,我们还没有着手进行。因此,我们说我们是一个一穷二白的国家。但是,比起蒋介石统治时期,我们是前进了一步。比蒋介石时期好,但并没有解决问题。还要多少时间呢?还要几个五年计划才能基本上解决机械化的问题和工农业扩大的问题。所以,我们说中国有进步,初步有些成就,但是并未根本改变中国的经济面貌。根本改变中国的经济面貌需要一个很长的时间。

\mxsay{斯:}我还想提一个问题。再过十年到二十年,你们就会达到工业化的目标。到那个时候,由于原子能和电子学的广泛应用,世界的经济基础将会有很大的改变。当然到那个时候,或者比那个时候要早得多,中国也会有原子能。有些美国人认为,中国要得到原子能,那是遥远的将来的事。另一方面,他们又害怕中国一旦有了原子弹,就会马上不负责任地使用它。

\mxsay{毛:}不会的。原子弹哪里能乱甩呢?如果我们有,也不能乱甩,乱甩就要犯罪。

\mxsay{斯:}尽管中美之间现在并没有和平条约和协定,尽管有些美国人认为美国和中国之间实际上处于半战争状态,但是全世界的和平每天都取决于中国的责任感。这种责任感首先是对中国人民的,其次也是对全世界的,而中国是其中的一部分。您同意我这种说法吗?

\mxsay{毛:}对。不管美国承认不承认我们,不管我们进不进联合国,世界和平的责任我们是要担负的。我们不会因为不进联合国就无法无天,像孙悟空大闹天宫那样。我们要维持世界和平,不要打世界大战。我们主张国与国之间不要用战争来解决问题。但是,维持世界和平不但中国有责任,美国也有责任。解决台湾问题是中国的内政,这点我们是要坚持的。虽然如此,我们不打。美国人在那里,我们去打吗?我们不打。美国人走后,我们就一定打吗?那也不一定。我们要用和平的方法解决台湾问题。我国好多地方就是用和平方法解决的。北京就是用和平方法解决的,还有湖南、云南、新疆。外面有一种说法,好像在各国共产党中,中国共产党特别调皮,不守规矩,不讲道理,是乱来的。你来了几个月,那种话不可全信。你讲过外面有人说,中国是一个大兵营和一个大监狱。对蒋介石的中国这样说,确实是像的,当时北京、南京、上海确实都是兵营。解放后,通过改造、教育,中国大有不同了。

\mxsay{斯:}我的确能够说,我的印象是中国现在同过去大为不同了。

\begin{maonote}
\mnitem{1}斯诺,即埃德加·斯诺(一九〇五——一九七二),美国进步作家、记者。一九三六年到陕北革命根据地访问,毛泽东会见了他。一九三七年发表《红星照耀中国》(一九三八年复社出版中译本时将书名改为《西行漫记》)。新中国成立后,于一九六〇年、一九六四年和一九七〇年三次到中国访问。
\mnitem{2}肯尼迪(一九一七——一九六三),美国民主党人,当时是美国参议员、民主党总统候选人。
\mnitem{3}尼克松(一九一三——一九九四),美国共和党人,时任美国副总统,共和党总统候选人。
\mnitem{4}艾森豪威尔,时任美国总统。
\mnitem{5}指中美大使级会谈,一九五五年四月二十三日,周恩来总理在亚非会议八国代表团团长会议上声明:中国政府愿意同美国政府谈判,讨论和缓远东紧张局势问题,特别是和缓台湾地区紧张局势问题。同年七月二十五日,中美双方就举行大使级会谈达成协议,并于八月一日在瑞士日内瓦进行首次会谈。此后由于美方缺乏诚意,会谈中断。一九五八年八月对金门炮击开始后,美国政府公开表示准备恢复会谈,双方随即于九月十五日在波兰华沙恢复会谈。迄至一九七〇年二月二十日,中美大使级会谈共举行了一百三十六次。由于美方坚持干涉中国内政的立场,会谈在和缓和消除台湾地区紧张局势问题上未取得任何进展。
\mnitem{6}指《丢掉幻想,准备斗争》(《毛泽东选集》第四卷,人民出版社1991年版,第1483—1489页)。
\mnitem{7}一九三五年六月红军第一、第四方面军长征在四川懋功(今小金)地区会师后,张国焘任红军总政治委员。他反对中央关于红军北上的决定,擅自率部南下,进行分裂党和红军的活动,并另立中央。一九三六年六月被迫取消第二中央,随后与红军第二、第四方面军一起北上,十二月到达陕北。一九三八年四月逃出陕甘宁边区,投入国民党特务集团,成为中国革命的叛徒。
\mnitem{8}贺龙、任弼时领导的红二、六军团经过长征,一九三六年七月同红四方面军会师于甘孜。中央革命军事委员会令红二、六军团加上红三十二军合编为红二方面军,由贺龙任总指挥,任弼时任政委。
\mnitem{9}马海德(一九〇一——一九八八),祖籍黎巴嫩,生于美国。一九三三年来中国从事医疗工作。一九三六年到达陕北革命根据地。中华人民共和国成立后任卫生部顾问。
\mnitem{10}保安,今志丹县。
\mnitem{11}独立战争,指一七七五年至一七八三年北美十三个殖民地人民推翻英国殖民统治、争取独立的战争。一七七五年五月殖民地代表召开会议,任命华盛顿为殖民地反英军队总司令,并于一七七六年发表《独立宣言》。一七八三年双方签订《巴黎和约》,正式承认十三个殖民地脱离英国独立。
\mnitem{12}指南北战争,是一八六一年至一八六五年由美国南部种植园主奴隶制与北部资产阶级雇佣劳动制之间的矛盾所引起的资产阶级民主革命战争。在战争过程中,代表北部资产阶级利益的联邦政府总统林肯颁布了《宅地法》和《解放黑奴宣言》,并采取其他民主措施,激发了工人、农民和黑人的革命斗志,因而联邦政府取得了战争的胜利。
\end{maonote}
