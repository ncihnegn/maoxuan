
\title{团结到底}
\date{一九四〇年七月五日}
\thanks{这是毛泽东为延安《新中华报》写的纪念抗日战争三周年的文章。}
\maketitle


抗日战争的三周年,正是中国共产党的十九周年。我们共产党人今天来纪念抗战,更感到自己的责任。中华民族的兴亡,是一切抗日党派的责任,是全国人民的责任,但在我们共产党人看来,我们的责任是更大的。我党中央已发表了对时局的宣言,这个宣言的中心是号召抗战到底,团结到底。这个宣言希望得到友党友军和全国人民的赞同,而一切共产党员尤其必须认真地执行这个宣言中所示的方针。

一切共产党员须知:只有抗战到底,才能团结到底,也只有团结到底,才能抗战到底。因此,共产党员要作抗战的模范,也要作团结的模范。我们所反对的,只是敌人和坚决的投降分子、反共分子,对其它一切人,都要认真地团结他们。而所谓坚决的投降分子、反共分子,在任何地方都只占少数。我调查了一个地方政府的成分,在那里办事的有一千三百人,其中坚决反共的只有四十至五十人,即是说,不足百分之四,其余都是希望团结抗战的。我们对于坚决的投降分子和反共分子,当然是不能容忍的,对他们容忍,就是让他们破坏抗战,破坏团结;所以必须坚决反对投降派,对于反共分子的进攻必须站在自卫立场上坚决地打退之。如果我们不是这样做,那就是右倾机会主义,是对于团结抗战不利的。但对于凡非坚决投降和坚决反共的人,则必须采取团结政策。其中有些人是两面派,有些人是被迫的,又有些人是一时之错,对于这些人都应争取他们,继续团结抗战。如果我们不是这样做,那就是“左”倾机会主义,也是对于团结抗战不利的。一切共产党员须知:我们发起了抗日民族统一战线,我们必须坚持这个统一战线。现在国难日深,世界形势大变,中华民族的兴亡,我们要负起极大的责任来。我们一定要战胜日本帝国主义,我们一定要把中国造成独立、自由、民主的共和国;而要达此目的,必须团结全国最大多数有党有派和无党无派的人。共产党人不许可同人家建立无原则的统一战线,因此,必须反对所谓溶共、限共、防共、制共的一套,必须反对党内的右倾机会主义。但同时,任何共产党员也不许可不尊重党的统一战线政策,因此,一切共产党员必须在抗日原则下团结一切尚能抗日的人,必须反对党内的“左”倾机会主义。

为此目的,在政权问题上,我们主张统一战线政权,既不赞成别的党派的一党专政,也不主张共产党的一党专政,而主张各党、各派、各界、各军的联合专政,这即是统一战线政权。共产党员在敌人后方消灭敌伪政权建立抗日政权之时,应该采取我党中央所决定的“三三制”,不论政府人员中或民意机关中,共产党员只占三分之一,而使其它主张抗日民主的党派和无党派人士占三分之二。无论何人,只要不投降不反共,均可参加政府工作。任何党派,只要是不投降不反共的,应使其在抗日政权下面有存在和活动之权。

在军队问题上,我党宣言中已表明:继续执行“不在一切友军中发展党的组织”的决定。某些地方党部尚未严格执行此决定的,应即加以纠正。凡不向八路军新四军举行军事磨擦的军队,应一律采取友好态度。即对某些举行过磨擦的军队,在其停止了磨擦之时,亦应恢复友好关系。这就是在军队问题上实行统一战线政策。

其它财政、经济、文化、教育、锄奸各方面的政策,为着抗日的需要,均必须从调节各阶级利益出发,实行统一战线政策,均必须一方面反对右倾机会主义,一方面反对“左”倾机会主义。

目前的国际形势,是帝国主义战争正向世界范围内扩大,由帝国主义战争所造成的极端严重的政治危机和经济危机,将必然引起许多国家革命的爆发。我们是处在战争和革命的新时代。没有卷入帝国主义战争漩涡的苏联,是全世界一切被压迫人民和被压迫民族的援助者。这些都是有利于中国抗战的。但同时,日本帝国主义正在准备向南洋侵略,加紧向中国进攻,势将勾引中国一部分动摇分子对其投降,投降危险是空前地加重了。抗战的第四周年将是最困难的一年。我们的任务是团结一切抗日力量,反对投降分子,战胜一切困难,坚持全国抗战。一切共产党员必须和友党友军团结一致去完成这个任务。我们相信,在我党全体党员和友党友军及全体人民共同努力之下,克服投降,战胜困难,驱除日寇,还我河山的目的,是能够达到的,抗战的前途是光明的。
