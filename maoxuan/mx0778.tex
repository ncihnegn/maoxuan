
\title{关于《水浒》的评论}
\date{一九七五年八月十四日}
\thanks{这是毛泽东同志关于学习《水浒》\mnote{1}的谈话。}
\maketitle


《水浒》这部书,好就好在投降。做反面教材,使人民都知道投降派。

《水浒》只反贪官,不反皇帝。屏晁盖\mnote{2}于一百零八人之外。宋江\mnote{3}投降,搞修正主义,把晁的“聚义厅”改为“忠义堂”,让人招安了。宋江同高俅\mnote{4}的斗争,是地主阶级内部这一派反对那一派的斗争。宋江投降了,就去打方腊\mnote{5}。

这支农民起义队伍的领袖不好,投降。李逵、吴用、阮小二、阮小五、阮小七\mnote{6}是好的,不愿意投降。

鲁迅评《水浒》评得好,他说:“一部《水浒》,说得很分明:因为不反对天子,所以大军一到,便受招安,替国家打别的强盗——不‘替天行道’的强盗去了。终于是奴才。”(《三闲集·流氓的变迁》)

金圣叹把《水浒》砍掉了二十多回。砍掉了,不真实。

鲁迅非常不满意金圣叹,专写了一篇评论金圣叹的文章《谈金圣叹》(见《南腔北调集》)。

《水浒》百回本、百二十回本和七十一回本,三种都要出。把鲁迅的那段评语印在前面。

\begin{maonote}
\mnitem{1}《水浒》,即《水浒传》,又称《忠义水浒传》,明初小说家施耐庵根据《大宋宣和遗事》及有关话本与民间流传的水浒故事加工整理而成。主要版本有明嘉靖年间武定侯郭勋刻本《忠义水浒传》,一百回本;明杨定见刻本《忠义水浒全传》,一百二十回本;清金圣叹评点本《第五才子书水浒传》,七十一回本。

这三种版本,人民文学出版社后来都出过重排本。
\mnitem{2}晁盖,《水浒传》中梁山早期领袖。
\mnitem{3}宋江,《水浒传》中为第一号人物,为梁山起义军领袖,力主接受招安,并最终导致了起义军和他自己的悲剧命运。
\mnitem{4}高俅,《水浒传》中朝廷方代表,反面角色,贪官污吏。
\mnitem{5}方腊,《水浒传》中另一支农民起义军。
\mnitem{6}李逵、吴用、阮小二、阮小五、阮小七,《水浒传》中的英雄好汉,反对招安。
\end{maonote}
