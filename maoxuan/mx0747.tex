
\title{赞成越南又打又谈的方针}
\date{一九六八年十一月十七日}
\thanks{这是这是毛泽东同越南民主共和国总理范文同谈话的主要部分。}
\maketitle


因为最近没有什么仗打,所以你们想同美国谈判。可以谈,要把它谈走也困难。美国也想同你们谈判,因为它的处境相当困难。它要顾及三个地区的问题,一个是美洲、美国,一个是欧洲,一个是亚洲。但是它把重兵放在亚洲搞这么几年,已经不平衡了,在欧洲投资的美国资本家在这方面就不满意。同时美国在历史上历来是让别国打,打到半路它再参加。只是在第二次世界大战后才先打朝鲜战争,然后打越南战争,它一国当头,别的国家很少参加。不管它叫什么特种战争,还是局部战争,对美国来说都是全力以赴的。现在它对别国顾不上,例如它在欧洲的军队就哇哇叫,说人少了,有经验的战士和指挥员给抽走了,好的装备也抽走了。不论是它在日本、朝鲜还是在亚洲其他地方的军队,还不是照样抽吗?它自己国家不是说有两亿人口吗?但是它经不起打,只出几十万兵,兵力有限。

你们打了十几年之后,就不要单看自己的困难了,要看到敌人的困难。日本在一九四五年投降到现在已有二十三年了,你们的国家照样存在。曾经有三个帝国主义国家侵略你们:日本、法国、美国。但是,你们的国家照样存在,而且还得到了发展。

帝国主义当然要打。它打的目的,一是为了灭火。你们那个地方起火,它要灭火。二是为了军火资本。为了灭火,就要制造灭火机械,就可以赚钱。美国每年在你们那里要消耗三百亿美元以上。

美国的规律是不愿意打长。他们的战争大概都是四年左右。你们那里,火灭不下去,反而烧得更大了。他们资本家分成派别,这个集团得利多,那个集团得利少,分赃不匀,内部就要闹乱子。这些矛盾都可以利用。赚钱较少的垄断资本家不愿意坚持打下去。从两派的竞选演说中可以看出这个问题来。特别是,美国有个记者叫李普曼\mnote{1},最近发表了一篇文章,说要提防再掉进一个陷阱。他说在越南已经掉进一个陷阱了,现在的问题是要想办法如何爬出这个陷阱。他还怕掉到别的陷阱里去。所以你们的事业是有希望的。

一九六六年,我在杭州同胡志明主席谈话,那时美国对北越已经又打起来了,但还没有恢复轰炸。我说美国大概打到今年就差不多了,因为今年是美国的选举年。不管哪个总统上台都有这个问题:它是继续打下去,还是现在退出?我看继续打下去它的困难会增多。整个欧洲的国家都没有参加打,这同朝鲜战争时的情况不同。日本大概不会参加打,它经济上帮一点儿忙是可能的,制造军火它是赚钱的。我看美国人过去是把自己的力量估计太大了。现在美国又是过去的做法,把兵力过于分散了。这不只是我们这样说,就是尼克松也这样说。它不但在美洲、欧洲把兵力分得这么散,就是在亚洲也是把兵力分散的。我原来不相信它会打北越,后来它轰炸北越,这话不灵了。现在它不轰炸了,这话又灵了。也许我的话就又不灵了。但总有一天要灵的,它又要停止轰炸。所以你们做几种打算我看是好的。

总而言之,这么多年来,美国的陆军是没有进攻北越的,它也没有封锁海防,也没有轰炸河内市区。它是留了一手的。它有个时候说是要“穷追”,可是你们的飞机从我们国内飞来飞去,它也不“穷追”。所以它那是说的空话。你们的飞机在我们的机场来往,它根本不提。又例如,中国有那么多人在你们那里工作,它是知道的,但它一个字也不提,好像没有这么回事似的。至于我们在你们那里的现在没有用的那部分人,可以撤回来,你们讨论过没有?如果它再来,我们再出去嘛!你们考虑一下,哪些可以留,哪些不要留,对你们有用的就留,现在没有用的就撤。等到将来又有用了再去。这和你们的飞机利用中国的飞机场一样,需要利用时就利用,现在不需要了就不利用。大体就是这样。

我赞成你们又打又谈的方针。我们有那么一些同志,就是怕你们上美国人的当。我看不会。这个谈判不是同打仗一样吗?在打仗中间取得经验得出规律嘛!有时是要上些当的,正如你们所说,美国人说话不算数。约翰逊曾公开说,甚至条约有时也不算数。但是事物总是有个规则的。例如你们的谈判,难道要谈一百年吗?我们的总理说,尼克松再谈两年不解决问题,他下一届再当总统就困难了。

还有一点,是南越傀儡政府非常害怕南越民族解放阵线。美国有人说,真正有效的,在南越人中有影响的,不是西贡政府,而是解放阵线。这话不是在美国的国会里讲的,而是记者报道的,但谁说的又不讲名字,只说是所谓美国官方人士。这话就提出这么个问题,在南越谁是真正有威信的政府?是阮文绍\mnote{2}还是阮友寿\mnote{3}?所以,名义上美国吹阮文绍怎么了不起,并说他不去巴黎参加谈判,实际上不是那么一回事。美国知道没有南越民族解放阵线参加谈判,就不能解决问题。

\begin{maonote}
\mnitem{1}沃尔特·李普曼(Walter Lippmann,一八八九——一九七四),美国作家、记者、政治评论家,美国历史上成就和名气最大的记者和专栏作家,传播学史上具有重要影响的学者之一,在宣传分析和舆论研究方面享有很高的声誉。曾做过十二位美国总统的顾问。作为美国著名的新闻评论家和作家,六十年的工作使他成为世界上最有名的政治专栏作家之一,他的专栏评论被不止二百五十家美国报纸和大约二十五家国外报纸刊用,同时它还分别为五十多家杂志撰稿。他获得了一九五八度普利策新闻奖。就大众媒体在构成舆论方面的作用而言,他是最有权威的发言者。所著《舆论学》被公认为是传播领域的奠基之作。
\mnitem{2}阮文绍,时任南越西贡傀儡政府“越南共和国”总统。
\mnitem{3}阮友寿,时任越南南方民族解放阵线中央委员会主席团主席。
\end{maonote}
