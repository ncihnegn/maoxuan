

\title{中国共产党在抗日时期的任务}
\date{一九三七年五月三日}
\thanks{这是毛泽东在一九三七年五月二日至十四日在延安召开的中国共产党全国代表会议上的报告。}
\maketitle


\section{民族矛盾和国内矛盾的目前发展阶段}

(一)由于中日矛盾成为主要的矛盾、国内矛盾降到次要和服从的地位而产生的国际关系和国内阶级关系的变化,形成了目前形势的新的发展阶段。

(二)中国很久以来就是处在两种剧烈的基本的矛盾中——帝国主义和中国之间的矛盾,封建制度和人民大众之间的矛盾。一九二七年以国民党为代表的资产阶级叛变革命,出卖民族利益于帝国主义,造成了工农政权和国民党政权尖锐对立,以及民族和民主革命的任务不能不由中国共产党单独负担的局面。

(三)一九三一年九一八事变\mnote{1}特别是一九三五年华北事变\mnote{2}以来的形势,使这些矛盾发生了如下的变化:

甲、由一般帝国主义和中国的矛盾,变为特别突出特别尖锐的日本帝国主义和中国的矛盾。日本帝国主义实行了完全征服中国的政策。因此,便把若干其它帝国主义和中国的矛盾推入次要的地位,而在这些帝国主义和日本帝国主义之间,扩大了矛盾的裂口。因此,便在中国共产党和中国人民面前提出了中国的抗日民族统一战线和世界的和平阵线相结合的任务。这就是说,中国不但应当和中国人民的始终一贯的良友苏联相联合,而且应当按照可能,和那些在现时愿意保持和平而反对新的侵略战争的帝国主义国家建立共同反对日本帝国主义的关系。我们的统一战线应当以抗日为目的,不是同时反对一切帝国主义。

乙、中日矛盾变动了国内的阶级关系,使资产阶级甚至军阀都遇到了存亡的问题,在他们及其政党内部逐渐地发生了改变政治态度的过程。这就在中国共产党和中国人民面前提出了建立抗日民族统一战线的任务。我们的统一战线是包括资产阶级及一切同意保卫祖国的人们的,是举国一致对外的。这个任务不但必须完成,而且是可能完成的。

丙、中日矛盾变动了全国人民大众(无产阶级、农民和城市小资产阶级)和共产党的情况和政策。人民更大规模地起来为救亡而斗争。共产党发展了在“九一八”后在三个条件(停止进攻革命根据地,保障人民的自由权利,武装人民)下和国民党中愿意同我们合作抗日的部分订立抗日协定的政策,成为建立全民族的抗日统一战线的政策。这就是我党一九三五年八月宣言\mnote{3},十二月决议\mnote{4},一九三六年五月放弃“反蒋”口号\mnote{5},八月致国民党书\mnote{6},九月民主共和国决议\mnote{7},十二月坚持和平解决西安事变\mnote{8},一九三七年二月致国民党三中全会电\mnote{9}等等步骤之所由来。

丁、由于帝国主义势力范围政策和中国半殖民地经济状况而来的中国军阀割据和军阀内战,在中日矛盾面前也起了变化。日本帝国主义赞助这种割据和内战,以便利其独占中国。若干其它帝国主义为了它们自己的利益,暂时地赞助中国的统一与和平。中国共产党和中国人民则以极大的努力反对内战与分裂,争取和平与统一。

戊、中日民族矛盾的发展,在政治比重上,降低了国内阶级间的矛盾和政治集团间的矛盾的地位,使它们变为次要和服从的东西。但是国内阶级间的矛盾和政治集团间的矛盾本身依然存在着,并没有减少或消灭。中国和日本以外其它帝国主义国家之间的矛盾亦然。因此,就在中国共产党和中国人民面前提出了下列的任务:适当地调整国内国际在现时可能和必须调整的矛盾,使之适合于团结抗日的总任务。这就是中国共产党要求和平统一、民主政治、改良生活及与反对日本的外国进行谈判种种方针之所由来。

(四)从一九三五年十二月九日开始的中国革命新时期\mnote{10}的第一阶段,至一九三七年二月国民党三中全会\mnote{11}时,告一段落。此阶段内的重大事变,是学生界、文化界、舆论界的救亡运动,红军的进入西北,共产党的抗日民族统一战线政策的宣传和组织工作,上海和青岛的反日罢工\mnote{12},英国对日政策之趋向比较的强硬\mnote{13},两广事变\mnote{14},绥远战争和援绥运动\mnote{15},南京在中日谈判中的比较强硬的态度\mnote{16},西安事变,最后是南京国民党的三中全会。这些事变,统统都是围绕着中国和日本对立这一基本矛盾的,都是直接围绕着建立抗日民族统一战线这个历史要求的。这一阶段的革命基本任务,是争取国内和平,停止国内的武装冲突,以便团结一致,共同抗日。共产党在此阶段内提出了“停止内战,一致抗日”的号召,这一号召是基本上实现了,这就构成了抗日民族统一战线实际组成上的第一个必要条件。

(五)国民党的三中全会,由于其内部有亲日派的存在,没有表示它的政策的明确和彻底的转变,没有具体地解决问题。然而由于人民的逼迫和国民党内部的变动,国民党不能不开始转变它过去十年的错误政策,这即是由内战、独裁和对日不抵抗的政策,向着和平、民主和抗日的方向转变,而开始接受抗日民族统一战线政策,这种初步转变,在国民党三中全会上是表现出来了。今后的要求是国民党政策的彻底转变。这就需要我们和全国人民更大地发展抗日和民主的运动,进一步地批评、推动和督促国民党,团结国民党内的主张和平、民主、抗日的分子,推动动摇犹豫分子,排除亲日分子,才能达到目的。

(六)目前的阶段,是新时期的第二个阶段。前一阶段和这一阶段都是走上全国性对日武装抗战的过渡阶段。如果前一阶段的任务主要地是争取和平,则这一阶段的任务主要地是争取民主。必须知道,为了建立真正的坚实的抗日民族统一战线,没有国内和平固然不行,没有国内民主也不行。所以争取民主,是目前发展阶段中革命任务的中心一环。看不清民主任务的重要性,降低对于争取民主的努力,我们将不能达到真正的坚实的抗日民族统一战线的建立。

\section{为民主和自由而斗争}

(七)对于中国本部的侵略,日本帝国主义正在加紧准备着。和希特勒、墨索里尼在西方加紧准备的强盗战争相呼应,日本在东方正在用尽一切气力在确定的步骤上准备一举灭亡中国的条件——国内军事的、政治的、经济的、思想的条件,国际外交条件,中国亲日势力的扶植。所谓“中日提携”的宣传和某些外交步骤的缓和,正是出于战争前夜日本侵略政策的战术上的必要。中国正迫近着判定自己存亡的关头,中国的救亡抗战,必须用跑步的速度去准备。我们并不反对准备,但反对长期准备论,反对文恬武嬉饱食终日的亡国现象,这些都是实际上帮助敌人的,必须迅速地清除干净。

(八)政治上、军事上、经济上、教育上的国防准备,都是救亡抗战的必需条件,都是不可一刻延缓的。而争取政治上的民主自由,则为保证抗战胜利的中心一环。抗战需要全国的和平与团结,没有民主自由,便不能巩固已经取得的和平,不能增强国内的团结。抗战需要人民的动员,没有民主自由,便无从进行动员。没有巩固的和平与团结,没有人民的动员,抗战的前途便会蹈袭阿比西尼亚\mnote{17}的覆辙。阿比西尼亚主要地是因为封建制度的统治,不能巩固内部的团结,不能发动人民的积极性,所以失败了。中国真正的坚实的抗日民族统一战线的建立及其任务的完成,没有民主是不行的。

(九)中国必须立即开始实行下列两方面的民主改革。第一方面,将政治制度上国民党一党派一阶级的反动独裁政体,改变为各党派各阶级合作的民主政体。这方面,应从改变国民大会的选举和召集上违反民主的办法,实行民主的选举和保证大会的自由开会做起,直到制定真正的民主宪法,召集真正的民主国会,选举真正的民主政府,执行真正的民主政策为止。只有这样做,才能真正地巩固国内和平,停止国内的武装敌对,增强国内的团结,以便举国一致抗御外敌。可能有这种情况发生,不待我们改革完毕,日本帝国主义的进攻就到来了。因此,为着随时能够抵抗日本的进攻并彻底地战胜之,我们必须迅速地进行改革,并准备在抗战的过程中进到彻底改革的程度。全国人民及各党派的爱国分子,必须抛弃过去对于国民大会和制定宪法问题的冷淡,而集中力量于这一具体的带着国防意义的国民大会运动和宪法运动,严厉地批判当权的国民党,推动和督促国民党放弃其一党派一阶级的独裁,而执行人民的意见。今年的几个月内,全国必须发起一个广大的民主运动,这运动的当前目标,应当放在国民大会和宪法的民主化的完成上。第二方面,是人民的言论、集会、结社自由。没有这种自由,就不能实现政治制度的民主改革,就不能动员人民进入抗战,取得保卫祖国和收复失地的胜利。当前几个月内,全国人民的民主运动,必须争取这一任务的某种最低限度的完成,释放政治犯、开放党禁等等,都包括在内。政治制度的民主改革和人民的自由权利,是抗日民族统一战线纲领上的重要部分,同时也是建立真正的坚实的抗日民族统一战线的必要条件。

(一〇)我们的敌人——日本帝国主义、中国汉奸、亲日派、托洛茨基派\mnote{18},对于中国的和平统一、民主自由和对日抗战的每一个步骤,都竭尽全力来破坏。当我们过去力争和平统一的时候,他们就竭力挑拨内战和分裂。当我们现在和最近将来力争民主自由的时候,他们无疑地又要来破坏。其总目标,就在使我们保卫祖国的抗战任务不能成功,而使他们灭亡中国的侵略计划达到目的。今后在争取民主自由的斗争中,不但要向国民党顽固派和人民中的落后成分努力做宣传鼓动和批评的工作,而且要针对着日本帝国主义以及充任日本侵华走狗的亲日派和托洛茨基派的阴谋,作尽量的揭破和坚决的斗争。

(一一)为了和平、民主和抗战,为了建立抗日的民族统一战线,中国共产党曾在致国民党三中全会电中向他们保证下列四项:(1)共产党领导的陕甘宁革命根据地的政府改名为中华民国特区政府,红军改名为国民革命军,受南京中央政府及军事委员会的指导;(2)在特区政府区域内,实行彻底的民主制度;(3)停止武力推翻国民党的方针;(4)停止没收地主的土地。这些保证,是必需的和许可的。因为只有如此,才能根据民族矛盾和国内矛盾在政治比重上的变化而改变国内两个政权敌对的状态,团结一致,共同赴敌。这是一种有原则有条件的让步,实行这种让步是为了去换得全民族所需要的和平、民主和抗战。然而让步是有限度的。在特区和红军中共产党领导的保持,在国共两党关系上共产党的独立性和批评自由的保持,这就是让步的限度,超过这种限度是不许可的。让步是两党的让步:国民党抛弃内战、独裁和对外不抵抗政策,共产党抛弃两个政权敌对的政策。我们以后者换得前者,重新与国民党合作,为救亡而奋斗。如果说这是共产党的投降,那只是阿Q主义\mnote{19}和恶意的污蔑。

(一二)共产党是否同意三民主义?我们的答复:是同意的\mnote{20}。三民主义有它的历史变化。孙中山先生的革命的三民主义,曾经因为孙先生与共产党合作加以坚决执行而取得人民的信仰,成为一九二四年至一九二七年的胜利的革命的旗帜。但是一九二七年国民党排斥共产党(清党运动\mnote{21}和反共战争),实行相反的政策,招致革命的失败,陷民族于危险的地位,于是三民主义也就失去了人民的信仰。现在民族危机极端严重,国民党已不能照旧不变地统治下去,因而全国人民和国民党中的爱国分子,又有两党合作的迫切要求。因此,重新整顿三民主义的精神,在对外争取独立解放的民族主义、对内实现民主自由的民权主义和增进人民幸福的民生主义之下,两党重新合作,并领导人民坚决地实行起来,是完全适合于中国革命的历史要求,而应为每一个共产党员所明白认识的。共产党人决不抛弃其社会主义和共产主义的理想,他们将经过资产阶级民主革命的阶段而达到社会主义和共产主义的阶段。中国共产党有自己的政治经济纲领。其最高的纲领是社会主义和共产主义,这是和三民主义有区别的。其在民主革命时期的纲领,亦比国内任何党派为彻底。但是共产党的民主革命纲领,与国民党第一次全国代表大会所宣布的三民主义的纲领,基本上是不相冲突的。因此,我们不但不拒绝三民主义,而且愿意坚决地实行三民主义,而且要求国民党和我们一道实行三民主义,而且号召全国人民实行三民主义。我们认为,共产党、国民党、全国人民,应当共同一致为民族独立、民权自由、民生幸福这三大目标而奋斗。

(一三)我们过去的工农民主共和国的口号是否错了呢?没有错的。资产阶级尤其是大资产阶级既然退出革命,而且投靠帝国主义和封建势力,变为人民的敌人,则革命的动力便只剩下了无产阶级、农民和城市中的小资产阶级;革命的政党,便只剩下了共产党;革命的组织责任,便不得不落在唯一的革命政党共产党的肩上。仅仅共产党继续高举革命的旗帜,保持革命的传统,提出工农民主共和国的口号,且为此口号而艰苦奋斗了许多年。工农民主共和国的口号,不是违背资产阶级民主革命任务的,而是坚决地执行资产阶级民主革命任务的。我们在实际斗争中没有一项政策不适合这种任务。我们的政策,包括没收地主土地和实行八小时工作制在内,并没有超出资本主义范畴内私有财产制的界限以外,并没有实行社会主义。新的民主共和国所包括的成分是什么呢?它包括无产阶级、农民、城市小资产阶级、资产阶级及一切国内同意民族和民主革命的分子,它是这些阶级的民族和民主革命的联盟。这里的特点是包括了资产阶级,这是因为资产阶级在今天的环境下,又有重新参加抗日的可能,所以无产阶级政党不应该拒绝他们,而应该招致他们,恢复和他们共同斗争的联盟,以利于中国革命的前进。为了停止国内的武装冲突,共产党愿意停止使用暴力没收地主土地的政策,而准备在新的民主共和国建设过程中,用立法和别的适当方法去解决土地问题。中国土地属于日本人,还是属于中国人,这是首先待解决的问题。既是在保卫中国的大前提之下来解决农民的土地问题,那末,由暴力没收方法转变到新的适当方法,就是完全必要的。

工农民主共和国口号,过去的提出和今天的放弃,都是正的。

(一四)为了建立民族统一战线共同对敌,国内的某些矛盾,必须给予适当的解决,其原则是应当有助于抗日民族统一战线的增强和扩大,而不是使其削弱和缩小。在民主革命阶段内,国内阶级间、党派间、政治集团间的矛盾和斗争是无法避免的,但是可以而且应该停止那些不利于团结抗日的斗争(国内战争,党派敌对,地方割据,一方面封建的政治压迫和经济压迫,一方面暴动政策和不利于抗日的过高的经济要求等等),而保存那些有利于团结抗日的斗争(批评的自由,党派的独立性,人民政治条件和经济条件的改善等等)。

(一五)在为抗日民族统一战线和统一的民主共和国而斗争的总任务之下,红军和抗日根据地的任务是:(1)使红军适合抗日战争的情况,应即改组为国民革命军,并将军事的政治的文化的教育提高一步,造成抗日战争中的模范兵团。(2)根据地改为全国的一个组成部分,实行新条件下的民主制度,重新编制保安部队,肃清汉奸和捣乱分子,造成抗日和民主的模范区。(3)在此区域内实行必要的经济建设,改善人民的生活状况。(4)实行必要的文化建设。

\section{我们的领导责任}

(一六)在某种历史环境能够参加反对帝国主义和反对封建制度的中国资产阶级,由于它在经济上政治上的软弱性,在另一种历史环境就要动摇变节,这一规律,在中国历史上已经证明了。因此,中国反帝反封建的资产阶级民主革命的任务,历史已判定不能经过资产阶级的领导,而必须经过无产阶级的领导,才能够完成。并且只有充分发扬无产阶级在民主革命中的坚持性和彻底性,才能克服资产阶级的那种先天的动摇性和不彻底性,而使革命不至于流产。使无产阶级跟随资产阶级呢,还是使资产阶级跟随无产阶级呢?这个中国革命领导责任的问题,乃是革命成败的关键。一九二四年至一九二七年的经验,表明了当资产阶级追随着无产阶级的政治领导的时候,革命是如何地前进了;及至无产阶级(由共产党负责)在政治上变成了资产阶级的尾巴\mnote{22}的时候,革命又是如何地遭到了失败。这种历史不应当重复了。依现时的情况说来,离开了无产阶级及其政党的政治领导,抗日民族统一战线就不能建立,和平民主抗战的目的就不能实现,祖国就不能保卫,统一的民主共和国就不能成功。在今天,以国民党为代表的资产阶级还带着很多的被动性和保守性,对于共产党发起的抗日民族统一战线,在长久的时期中表示不敢接受,就是证据。这种情况,加重了无产阶级及其政党的政治领导责任。抗日救国的总参谋部的职务,共产党是责无旁贷和义不容辞的。

(一七)无产阶级怎样经过它的政党实现对于全国各革命阶级的政治领导呢?首先,是根据历史发展行程提出基本的政治口号,和为了实现这种口号而提出关于每一发展阶段和每一重大事变中的动员口号。例如我们提出了“抗日民族统一战线”和“统一的民主共和国”这样的基本口号,又提出了“停止内战”、“争取民主”、“实现抗战”的口号,作为全国人民一致行动的具体目标,没有这种具体目标,是无所谓政治领导的。第二,是按照这种具体目标在全国行动起来时,无产阶级,特别是它的先锋队——共产党,应该提起自己的无限的积极性和忠诚,成为实现这些具体目标的模范。在为抗日民族统一战线和民主共和国的一切任务而奋斗时,共产党员应该作到最有远见,最富于牺牲精神,最坚定,而又最能虚心体会情况,依靠群众的多数,得到群众的拥护。第三,在不失掉确定的政治目标的原则上,建立与同盟者的适当的关系,发展和巩固这个同盟。第四,共产党队伍的发展,思想的统一性,纪律的严格性。共产党对于全国人民的政治领导,就是由执行上述这些条件去实现的。这些条件是保证自己的政治领导的基础,也就是使革命获得彻底的胜利而不被同盟者的动摇性所破坏的基础。

(一八)和平实现与两党合作成立之后,过去在两个政权敌对路线下的斗争方式、组织方式和工作方式,应当有所改变。这种改变,主要是从武装的转到和平的,非法的转到合法的。这种转变是不容易的,需要重新学习。重新训练干部,成为主要的一环。

(一九)关于民主共和国的性质和前途的问题,许多同志已提出来了。我们的答复是:其阶级性是各革命阶级的联盟,其前途可能是走向社会主义。我们的民主共和国,是在执行民族抗战任务的过程中建立起来的,是在无产阶级领导之下建立起来的,是在国际新环境之下(苏联社会主义的胜利,世界革命新时期的前夜)建立起来的。因此,按照社会经济条件,它虽仍是资产阶级民主主义性质的国家,但是按照具体的政治条件,它应该是一个工农小资产阶级和资产阶级联盟的国家,而不同于一般的资产阶级共和国。因此,它的前途虽仍然有走上资本主义方向的可能,但是同时又有转变到社会主义方向的可能,中国无产阶级政党应该力争这后一个前途。

(二〇)向关门主义和冒险主义、同时又向尾巴主义作斗争,是执行党的任务的必要的条件。我们党在民众运动中,有严重的关门主义、高慢的宗派主义和冒险主义的传统倾向,这是一个妨碍党建立抗日民族统一战线和争取多数群众的恶劣的倾向。在每一个具体的工作中肃清这个倾向是完全必要的。我们的要求是依靠多数和照顾全局。陈独秀尾巴主义的复活是不能容许的,这是资产阶级改良主义在无产阶级队伍中的反映。降低党的立场,模糊党的面目,牺牲工农利益去适合资产阶级改良主义的要求,将必然引导革命趋于失败。我们的要求是实行坚决的革命政策,争取资产阶级民主革命的彻底胜利。为了达到克服上述这些不良倾向的目的,在全党中提高马克思列宁主义的理论水平是完全必要的,因为只有这种理论,才是引导中国革命走向胜利的指南针。


\begin{maonote}
\mnitem{1}见本卷\mxnote{论反对日本帝国主义的策略}{4}。
\mnitem{2}华北事变指一九三五年日本帝国主义侵略华北和以蒋介石为首的国民党政府出卖华北主权的一连串事件。这一年五月,日本帝国主义向国民党政府提出了对华北统治权的无理要求;国民党政府在华北的代表何应钦开始与日方会商。七月六日,何应钦正式致函日本华北驻屯军司令官梅津美治郎,接受了日方要求,这就是所谓“何梅协定”。六月二十七日,国民党察哈尔省政府代理主席秦德纯与日本沈阳特务机关长土肥原以换文方式达成协议,通称“秦土协定”。按照这些协定,中国在河北和察哈尔(现在分属河北、山西两省)的主权大部丧失。随后,日本帝国主义更策动汉奸制造所谓“华北五省自治运动”,企图使河北、察哈尔、绥远(现属内蒙古自治区)、山东、山西五省脱离中国政府的管辖。十月,日本帝国主义在河北省香河县指使汉奸暴动,一度占领了县城。十一月,汉奸殷汝耕在通县成立所谓“冀东防共自治委员会”(一个月后改称“冀东防共自治政府”)。十二月,国民党政府指派宋哲元等成立“冀察政务委员会”,以适应日本帝国主义关于“华北政权特殊化”的要求。
\mnitem{3}这是指一九三五年八月一日中国共产党驻共产国际代表团以中国苏维埃中央政府、中国共产党中央委员会名义发表的《为抗日救国告全体同胞书》,通称“八一宣言”。这个宣言的要点是:“今当我亡国灭种大祸迫在眉睫之时,共产党再一次向全体同胞呼吁:无论各党派间在过去和现在有任何政见和利害的不同,无论各界同胞间有任何意见上或利益上的差异,无论各军队间过去和现在有任何敌对行动,大家都应当有‘兄弟阋于墙外御其侮’的真诚觉悟,首先大家都应当停止内战,以便集中一切国力(人力、物力、财力、武力等)去为抗日救国的神圣事业而奋斗。共产党特再一次郑重宣言:只要国民党军队停止进攻红军的行动,只要任何部队实行对日抗战,不管过去和现在他们与红军之间有任何旧仇宿怨,不管他们与红军之间在对内问题上有何分歧,红军不仅立刻对之停止敌对行为,而且愿意与之亲密携手共同救国。”“共产党愿意作成立这种国防政府的发起人,共产党愿意立刻与中国一切愿意参加抗日救国事业的各党派,各团体(工会、农会、学生会、商会、教育会、新闻记者联合会、教职员联合会、同乡会、致公堂、民族武装自卫会、反日会、救国会等等),各名流学者,政治家,以及一切地方军政机关,进行谈判共同成立国防政府问题。谈判结果所成立的国防政府,应该作为救亡图存的临时领导机关。这种国防政府,应当设法召集真正代表全体同胞(由工农军政商学各界,一切愿意抗日救国的党派和团体,以及国外侨胞和中国境内各民族,在民主条件下选出的代表)的代表机关,以便更具体地讨论关于抗日救国的各种问题。共产党绝对尽力赞助这一全民代表机关的召集,并绝对执行这一机关的决议。”“抗日联军应由一切愿意抗日的部队合组而成。在国防政府领导之下,组成统一的抗日联军总司令部。这种总司令部或由各军抗日长官及士兵选出代表组成,或由其它形式组成,也由各方代表及全体人民公意而定。红军绝对首先加入联军,以尽抗日救国的天职。为了使国防政府真能担当起国防重任,为了使抗日联军真能担负起抗日重责,共产党号召全体同胞:有钱的出钱,有枪的出枪,有粮的出粮,有力的出力,有专门技能的贡献专门技能,以便我全体同胞总动员,并用一切新旧式武器,武装起千百万民众来。”
\mnitem{4}这是指一九三五年十二月中共中央在陕北瓦窑堡举行政治局会议期间,于二十五日通过的《关于目前政治形势与党的任务决议》。这个决议全面地分析了当时国内外的形势和阶级关系的变化,批判了成为当时党内主要危险的关门主义,确定了建立抗日民族统一战线的策略方针。下面是这个决议的一部分:“目前的形势告诉我们,日本帝国主义并吞中国的行动,震动了全中国与全世界。中国政治生活中的各阶级、阶层、政党以及武装势力,重新改变了与正在改变着它们之间的相互关系。民族革命战线与民族反革命战线是在重新改组中。因此,党的策略路线,是在于发动、团结与组织全中国全民族一切革命力量去反对当前主要的敌人——日本帝国主义与卖国贼头子蒋介石。不论什么人,什么派别,什么武装队伍,什么阶级,只要是反对日本帝国主义与卖国贼蒋介石的,都应该联合起来开展神圣的民族革命战争,驱逐日本帝国主义出中国,打倒日本帝国主义的走狗在中国的统治,取得中华民族的彻底解放,保持中国的独立与领土的完整。只有最广泛的反日民族统一战线(下层的与上层的),才能战胜日本帝国主义与其走狗蒋介石。当然,不同的个人,不同的团体,不同的社会阶级与阶层,不同的武装队伍,他们参加反日的民族革命,各有他们不同的动机与立场。有的是为了保持他们原有的地位,有的是为了要争取运动的领导权使运动不至超出他们所容许的范围之外,有的真是为了中华民族的彻底解放。正因为他们的动机与立场各有不同,有的在斗争开始时就要动摇叛变的,有的会在中途消极或退出战线的,有的愿意奋斗到底的。但是,我们的任务,是在不但要团结一切可能的反日的基本力量,而且要团结一切可能的反日同盟者,是在使全国人民有力出力,有钱出钱,有枪出枪,有知识出知识,不使一个爱国的中国人不参加到反日的战线上去。这就是党的最广泛的民族统一战线策略的总路线。只有按照这种路线,我们才能动员全国人民的力量去对付全国人民的公敌:日本帝国主义与卖国贼蒋介石。中国工人阶级与农民,依然是中国革命的基本动力。广大的小资产阶级群众,革命的知识分子,是民族革命中最可靠的同盟者。工农小资产阶级的坚固联盟,是战胜日本帝国主义与汉奸卖国贼的基本力量。一部分民族资产阶级与军阀,不管他们怎样不同意土地革命与红色政权,在他们对于反日反汉奸卖国贼的斗争采取同情,或善意中立,或直接参加之时,对于反日战线的开展都是有利的。因为这就使他们离开了总的反革命力量,而扩大了总的革命力量。为达到此目的,党应该采取各种适当的方法与方式,争取这些力量到反日战线中来。不但如此,即在地主买办阶级营垒中间,也不是完全统一的,由于中国过去是许多帝国主义互相竞争的结果,产生了各国帝国主义在中国的互相竞争的卖国贼集团,他们中间的矛盾与冲突,党亦应使用许多的办法使某些反革命力量暂时处于不积极的反对反日战线的地位。对于日本帝国主义以外的其它帝国主义的策略也是如此。党在发动团结与组织全中国人民的力量以反对全中国人民的公敌时,应该坚决不动摇地同反日统一战线内部一切动摇、妥协、投降与叛变的倾向做斗争。一切破坏中国人民反日运动者,都是汉奸卖国贼,应该群起而攻之。共产党应该以自己彻底的正确的反日反汉奸卖国贼的言论与行动去取得自己在反日战线中的领导权。也只有在共产党领导之下,反日运动才能得到彻底的胜利。对反日战争中的广大民众,应该满足他们基本利益的要求(农民的土地要求,工人、士兵、贫民、知识分子等改良生活待遇的要求)。只有满足了他们的要求,才能动员更广大的群众走进反日的阵地上去,才能使反日运动得到持久性,才能使运动走到彻底的胜利。也只有如此,才能取得党在反日战争中的领导权。”
\mnitem{5}见一九三六年五月五日中国红军要求南京政府停战议和一致抗日的通电。中国共产党在这个通电中开始放弃“反蒋”口号。通电内容如下:“南京国民政府军事委员会,全体海陆空军,全国各党、各派、各团体、各报馆,一切不愿意当亡国奴的同胞们:自从中国红军革命军事委员会组织中国人民红军抗日先锋军渡河东征以来,所向皆捷,全国响应。但正当抗日先锋军占领同蒲铁路,积极准备东出河北与日本帝国主义直接作战之时,蒋介石氏竟以十师以上兵力开入山西,协同阎锡山氏阻拦红军抗日去路,并命令张学良杨虎城两氏及陕北军队向陕甘红色区域挺进,扰乱我抗日后方。中国人民红军抗日先锋军,本当集中全力消灭蒋氏拦阻抗日去路的部队,以达到对日直接作战之目的。但红军革命军事委员会一再考虑,认为国难当前,双方决战,不论胜负属谁,都是中国国防力量的损失,而为日本帝国主义所称快。且在蒋介石、阎锡山两氏的部队中,不少愿意停止内战、一致抗日的爱国军人,目前接受两氏的命令,拦阻红军抗日去路,实系违反自己良心的举动。因此,红军革命军事委员会为了保存国防实力,以便利于迅速进行抗日战争,为了坚决履行我们每次向国人宣言停止内战、一致抗日的主张,为了促进蒋介石氏及其部下爱国军人们的最后觉悟,故虽在山西取得了许多胜利,仍然将人民抗日先锋军撤回黄河西岸。以此行动向南京政府、全国海陆空军、全国人民表示诚意,我们愿意在一个月内与所有一切进攻抗日红军的武装队伍,实行停战议和,以达到停战抗日的目的。红军革命军事委员会特慎重地向南京政府诸公进言,在亡国灭种紧急关头,理应翻然改悔,以‘兄弟阋于墙外御其侮’的精神,在全国范围、首先在陕甘晋停止内战,双方互派代表磋商抗日救亡的具体办法。此不仅诸公之幸,实亦民族国家之福。如仍执迷不悟,甘为汉奸卖国贼,则诸公的统治,必将最后瓦解,必将为全国人民所唾弃所倾覆。语云:‘千夫所指,无疾而死。’又云:‘放下屠刀,立地成佛。’愿诸公深思熟虑之。红军革命军事委员会更号召全国凡属不愿意做亡国奴的团体、党派、人民,赞助我们停战议和、一致抗日的主张,组织停止内战促进会,派遣代表隔断双方火线,督促并监视这一主张的完全实现。”
\mnitem{6}见本卷\mxnote{关于蒋介石声明的声明}{9}。
\mnitem{7}指一九三六年九月十七日中共中央通过的《关于抗日救亡运动的新形势与民主共和国的决议》。一九三五年十二月中共中央政治局会议通过的《关于目前政治形势与党的任务决议》和毛泽东的《论反对日本帝国主义的策略》的报告,提出了人民共和国的口号。随后,根据情况的需要,中国共产党采取了逼蒋抗日的政策,估计人民共和国这个口号不会为蒋介石集团所接受,于是在一九三六年八月致国民党信中,改用了民主共和国的口号。接着又在同年九月十七日中共中央通过的决议中,对于民主共和国的口号作了具体的说明。两个口号形式上虽有不同,实质上却是一致的。下面是一九三六年九月十七日中共中央通过的决议中关于民主共和国问题的两节:“中央认为在目前形势之下,有提出建立民主共和国口号的必要,因为这是团结一切抗日力量来保障中国领土完整和预防中国人民遭受亡国灭种的惨祸的最好方法,而且这也是从广大人民的民主要求产生出来的最适当的统一战线的口号。民主共和国是较之一部分领土上的工农民主专政制度在地域上更普及的民主,较之全中国主要地区上国民党的一党专政大大进步的政治制度,因此便更能保障抗日战争的普遍发动与彻底胜利。同时,民主共和国不但能够使全中国最广大的人民群众参加到政治生活中来,提高他们的觉悟程度与组织力量,而且也给中国无产阶级及其首领共产党为着将来的社会主义的胜利而斗争以自由活动的舞台。因此,中国共产党宣布:积极赞助民主共和国运动。并且宣布:民主共和国在全中国建立、依据普选的国会实行召集之时,红色区域即将成为它的一个组成部分,红色区域人民将选派代表参加国会,并将在红色区域内完成同样的民主制度。”“中央着重指出:只有继续开展全中国人民的抗日救亡运动,扩大各党各派各界各军的抗日民族统一战线,加强中国共产党在民族统一战线中的政治领导作用,极大地巩固红色政权与红军,同一切丧权辱国及削弱民族统一战线力量的言论行动进行坚决的斗争,我们才能推动国民党南京政府走向抗日,才能给民主共和国的实现准备前提。没有艰苦的持久的斗争,没有全中国人民的发动与革命的高涨,民主共和国的实现是不可能的。中国共产党在为民主共和国而斗争的过程中,应该使这个民主共和国从实行本党所提出的抗日救国十大纲领开始,一直到中国资产阶级民主革命的基本任务彻底的完成。”
\mnitem{8}参见本卷\mxnote{关于蒋介石声明的声明}{1}。
\mnitem{9}这个电报于一九三七年二月十日发出,内容如下:“中国国民党三中全会诸先生鉴:西安问题和平解决,举国庆幸,从此和平统一团结御侮之方针得以实现,实为国家民族之福。当此日寇猖狂,中华民族存亡千钧一发之际,本党深望贵党三中全会,本此方针,将下列各项定为国策:(一)停止一切内战,集中国力,一致对外;(二)保障言论、集会、结社之自由,释放一切政治犯;(三)召集各党各派各界各军的代表会议,集中全国人材,共同救国;(四)迅速完成对日抗战之一切准备工作;(五)改善人民的生活。如贵党三中全会果能毅然决然确定此国策,则本党为着表示团结御侮之诚意,愿给贵党三中全会以如下之保证:(一)在全国范围内停止推翻国民政府之武装暴动方针;(二)工农民主政府改名为中华民国特区政府,红军改名为国民革命军,直接受南京中央政府与军事委员会之指导;(三)在特区政府区域内,实行普选的彻底民主制度;(四)停止没收地主土地之政策,坚决执行抗日民族统一战线之共同纲领。”
\mnitem{10}参见本卷\mxnote{论反对日本帝国主义的策略}{8}。
\mnitem{11}指在西安事变和平解决后,一九三七年二月十五日至二十二日在南京举行的国民党第五届中央执行委员会第三次全体会议。会议主要商讨对中国共产党和对日本的政策。迫于全国人民停止内战、一致抗日的要求,会议实际上接受了国共两党合作抗日的政策,确定了和平统一、修改选举法、扩大民主、释放政治犯等原则。
\mnitem{12}一九三六年十一月八日,上海的日本纱厂工人开始罢工。参加这次罢工的,先后共达四万五千余人。罢工坚持二十天左右,获得胜利。日本资本家被迫同意自十一月份起增加工资百分之五,不无故开除工人,不打骂工人,等等。十一月十九日,青岛的日本纱厂工人,为响应上海工人的斗争,也开始罢工。十二月三日,日本海军陆战队在青岛登陆,罢工工人遭到镇压。
\mnitem{13}一九三三年日本帝国主义侵占山海关进入华北以后,特别是自一九三五年《何梅协定》以后,英美帝国主义在华北华中的利益,直接受到日本帝国主义的打击,因此,英美就开始改变对于日本的态度,并且给国民党政府的对日政策以影响。一九三六年西安事变发生的时候,英国曾经主张拒绝日本所提出的不利于英国在华利益的要求,甚至表示只要国民党政府还能够继续统治中国人民,就不妨和“共产党采取某种形式的联合”,以便打击日本企图独占全中国的政策。
\mnitem{14}广东地方实力派陈济棠和广西地方实力派李宗仁、白崇禧等在一九三六年六月间发表通电,宣布“北上抗日”,联合起来反对蒋介石。蒋介石用分化利诱等手段,收买了陈济棠的军队。七月,陈济棠被迫下台。九月,李宗仁、白崇禧同蒋介石达成协议,事变和平解决。
\mnitem{15}一九三六年七月底至八月初,日本侵略军和伪蒙军向绥远(现划归内蒙古自治区)东北部进攻,当地驻军傅作义所部在全国人民抗日救亡运动的推动下,奋起抗战,击退这次进犯。十一月,日伪军发动更大规模的进攻,当地驻军再次进行抵抗。全国人民开展援绥运动,组织后援会和战区服务团,并且募集款项、棉衣等支援前线。在全国人民的支援下,绥远的中国驻军击溃了日伪军,收复了当时日伪军在绥北的主要基地百灵庙。
\mnitem{16}一九三六年,由于中国人民抗日潮流的压力和英美同日本争夺中国的矛盾日益尖锐,南京国民党政府对于日本帝国主义的侵略采取了比较强硬的态度。这一年的三月,国民党政府外交部长同日本驻华大使举行会谈,最后发表共同公告,宣布双方意见“未能全部一致”。在同年九月至十二月的中日谈判中,国民党政府又采用拖延的办法,使谈判未获结果而停顿。
\mnitem{17}阿比西尼亚,即埃塞俄比亚。
\mnitem{18}参见本卷\mxnote{论反对日本帝国主义的策略}{33}。抗日战争时期,托派在宣传上主张抗日,但是攻击中国共产党的抗日民族统一战线政策。把托派与汉奸相提并论,是由于当时在共产国际内流行着中国托派与日本帝国主义间谍组织有关的错误论断所造成的。
\mnitem{19}阿Q是中国伟大作家鲁迅的著名小说《阿Q正传》中的主角。他的突出特点是习惯于用自己安慰自己的方法,在任何情形下都自以为是胜利者即“精神胜利”者。阿Q主义就是指这种“精神上的胜利法”。
\mnitem{20}这里所说的三民主义,是指孙中山在《中国国民党第一次全国代表大会宣言》中所重新解释的三民主义。中国共产党在资产阶级民主革命阶段中同意孙中山重新解释的三民主义中革命的民族主义、民权主义和民生主义这三个政治原则,但并不同意他所代表的资产阶级和小资产阶级的宇宙观或理论体系。参见本书第二卷\mxart{新民主主义论}第九节和第十节。
\mnitem{21}一九二四年,孙中山在共产党人的帮助下,将国民党改组成各阶级的民主革命的联盟。当时,中国共产党的许多党员曾以个人名义参加国民党。一九二七年蒋介石、汪精卫相继叛变革命以后,在全国各地屠杀共产党人以及国民党内许多真正拥护孙中山三大政策的左派分子,他们称之为“清党运动”。从此,国民党基本上变成代表大地主大资产阶级的反动集团。
\mnitem{22}这里是指一九二七年上半年陈独秀右倾投降主义的领导所造成的情况。
\end{maonote}
