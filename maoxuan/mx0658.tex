
\title{关于教育革命的谈话}
\date{一九六四年二月十三日、一九六五年十二月二十一日}
\thanks{这是毛泽东同志一九六四年二月十三日在春节座谈会上的讲话和一九六五年十二月二十一日在杭州会议上讲话的整理,最初收录在《毛泽东论教育革命》(人民出版社依一九六七年十二月出版),出版前经毛泽东审定。}
\maketitle


我早就说过,我们的教育方针,应该使受教育者在德育、智育、体育几方面都得到发展,成为有社会主义觉悟的有文化的劳动者。

现在课程多,害死人,使中小学生、大学生天天处于紧张状态。课程可以砍掉一半。学生成天看书,并不好,可以参加一些生产劳动和必要的社会劳动。

现在的考试,用对付敌人的办法,搞突然袭击,出一些怪题、偏题,整学生。这是一种考八股文的办法,我不赞成,要完全改变。我主张题目公开,由学生研究、看书去做。例如,出二十个题,学生能答出十题,答得好,其中有的答得很好,有创见,可以打一百分;二十题都答了,也对,但是平平淡淡,没有创见的,给五十分、六十分。考试可以交头接耳,无非自己不懂,问了别人懂了。懂了就有收获,为什么要死记硬背呢?人家做了,我抄一遍也好。可以试试点。

旧教学制度摧残人材,摧残青年,我很不赞成。孔夫子出身没落奴隶主贵族,也没有上过什么中学、大学,开始的职业是替人办丧事,大约是个吹鼓手。人家死了人,他去吹吹打打。他会弹琴、射箭、架车子,也了解一些群众情况。开头作过小官,管理粮草和管理牛羊畜牧。后来他在鲁国当了大官,群众的事就听到了。他后来办私塾,反对学生从事劳动。

明朝李时珍长期自己上山采药,才写了《本草纲目》。更早些的,有所发明的祖冲之,也没有上过什么中学、大学。美国的佛兰克林是印刷所学徒,也卖过报,他是电的大发明家。英国的瓦特是工人,是蒸汽机的大发明家。高尔基的学问完全是自学的,据说他只上过两年小学。

现在一是课多,一是书多,压得太重。有些课程不一定要考。如中学学一点逻辑、语法,不要考,知道什么是语法,什么是逻辑就可以了,真正理解,要到工作中去慢慢体会。课程讲的太多,是烦琐哲学。烦琐哲学总是要灭亡的。如经学,搞那么多注解,现在没有用了。我看这种方法,无论中国的也好,其他国家的也好,都要走向自己的反面,都要灭亡的。

书不一定读得很多。马克思主义的书要读,读了要消化。读多了,又不能消化,可能走向反面,成为书呆子,成为教条主义者、修正主义者。

现在学校课程太多,对学生压力太大。讲授又不甚得法。考试方法以学生为敌人,举行突然袭击。这三项都是不利于培养青年们在德、智、体诸方面生动活泼地主动地得到发展。整个教育制度就是那样,公开号召去争取那个五分,就有那么一些人把分数看透了,大胆主动地去学。把那一套看透了,学习也主动了。

据说某大学有个学生,平时不记笔记,考试时得三分半到四分,可是毕业论文在班里水平最高。在学校是全优,工作上不一定就是全优。中国历史上凡是中状元的,都没有真才实学,反倒是有些连举人都没有考取的人优点真才实学。不要把分数看重了,要把精力集中在培养分析问题和解决问题的能力上,不要只是跟在教员的后面跑,自己没有主动性。

反对注入式教学法,连资产阶级教育家在五四时期就早已提出来了,我们为什么不反?只要不把学生当成打击对象就好了。你们的教学就是灌,天天上课,有那么多可讲的?教员应该把讲稿印发给你们。怕什么?应该让学生自己去研究讲稿。讲稿还对学生保密?到了讲堂才让学生抄,把学生束缚死了。

大学生,尤其是高年级,主要是自己研究问题,讲那么多干什么?教改的问题,主要是教员问题。教员就那么点本事,离开讲稿什么也不行。为什么不把讲稿发给你们,与你们一起研究问题?高年级学生提出的问题,教员能答百分之五十,其它的说不知道,和学生一起商量,这就是不错了。不要装着样子去吓唬人。

学生负担太重,影响健康,学了也无用。建议从一切活动总量中,砍掉三分之一。请邀学校师生代表,讨论几次,决定实行。如何请酌。

现在这种教育制度,我很怀疑。从小学到大学,一共十六、七年,二十多年看不见稻、菽、麦、黍、稷,看不见工人怎样做工,看不见农民怎样种田,看不见商品是怎么交换的,身体也搞坏了,真是害死人。我曾给我的孩子说:“你下乡去跟贫下中农说,就说我爸爸说的,读了几十年书,越读越蠢。请叔叔伯伯、姐妹兄弟做老师,向你们来学习。”其实,入学前的小孩。一岁到七岁,接触事物很多。二岁学说话,三岁哇啦哇啦跟人吵架,再大一点就拿小工具挖土,模仿大人劳动。这就是观察世界。小孩子已经学会了一些概念。狗,是个大概念。黑狗、黄狗是小些的概念。他家里的那条黄狗,就是具体的。人,这个概念已经舍掉了许多东西,舍掉了男人、女人的区别,大人、小孩的区别,中国人与外国人的区别,只剩下了区别于其它动物的特点。谁见过“人”?只能见到张三、李四。“房子”的概念谁也看不见,只看到具体的房子,天津的洋房,北京的四合院。

大学教育应当改造,上学的时间不要那么多。文科不改造不得了。不改造能出哲学家吗?能出文学家吗?能出历史学家吗?

现在的哲学家搞不了哲学,文学家写不了小说,历史学家搞不了历史,要搞就是帝王将相。要改造文科大学,要学生下去搞工业、农业、商业。至于工科、理科,情况不同,他们有实习工厂,有实验室,在实习工厂做工,在实验室做实验,但也要接触社会实际。
