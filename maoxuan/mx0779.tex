
\title{中美关系会一点点好起来}
\date{一九七五年十二月二日}
\thanks{这是毛泽东同志会见美国总统福特时的谈话。}
\maketitle


\mxsay{福特:}主席先生,我们今天早上有一次很不错的讨论。

\mxsay{毛泽东:}你们讨论什么?

\mxsay{福特:}我们讨论了我们同苏联之间的问题,还有我们在全球范围内的对策,以及我的国家和你的国家有必要合作,去实现那些对我们两国都有利的目标。

\mxsay{毛泽东:}我们没有这种本事,我们只能放放空炮。

\mxsay{福特:}我不相信,主席先生。

\mxsay{毛泽东:}如果说到骂人,这种本事我们倒是有一点。

\mxsay{福特:}我们也能。

\mxsay{毛泽东:}你们也能?那么我们应该达成一项协议。

\mxsay{福特:}我们还可以使用武力,对付那些制造麻烦的国家。

\mxsay{毛泽东:}这个不坏。那么我们可以达成另外一项协议。

\mxsay{福特:}今天早上,我们讨论的时候,有很明确的讨论目标。

\mxsay{毛泽东:}除了苏联帝国主义者,不可能是别人了。

\mxsay{福特:}主席先生,今天早上,我们使用了一些很强硬的语言。

\mxsay{毛泽东:}(指着邓小平\mnote{1})你们批评他了?

\mxsay{福特:}我们强烈批评的是另外一个国家。

\mxsay{毛泽东:}那个在北面的国家。

\mxsay{福特:}对。

\mxsay{毛泽东:}(对驻美大使黄镇)黄镇\mnote{2}同志,事情怎么样了?你还回美国吗?

\mxsay{黄镇:}我听从主席的指示。

\mxsay{毛泽东:}总统先生,你们要他吗?

\mxsay{福特:}我们当然要他回去。我们的关系好极了。大使先生回美国是很重要的,就像(驻华大使)布什\mnote{3}先生要是留在北京一样。

\mxsay{毛泽东:}(对布什)你留在北京吗?

\mxsay{布什:}只有几天了。

\mxsay{毛泽东:}你升职了。

\mxsay{福特:}对,他升职了。我们将在一个月里,提名继任者。

\mxsay{毛泽东:}我们不想让他走。

\mxsay{福特:}他是一个出色的人,所以我要他回美国。但我们会找一个同样好的人来接替他。

\mxsay{毛泽东:}那样挺好。我觉得,黄镇同志返回美国更有利。

\mxsay{黄镇:}我将坚决执行主席的指示。我确实想回国,因为我驻外太久了。但是我将按照主席说的做。

\mxsay{毛泽东:}你应该再在那里待一到二年。

\mxsay{黄镇:}好的,我肯定会回去,坚决执行主席的指示。

\mxsay{毛泽东:}有一些年轻人批评他(指黄镇)。这两个年轻人(指在场的两个翻译唐闻生、王海容)对乔老爷\mnote{4}也有一些批评。不能忽视这些人啊,否则你会在他们身上吃苦头的。这是一场内战。现在外面有很多大字报出来。你不妨去清华和北大看看。

\mxsay{福特:}我不太明白这些东西。主席先生,我希望,你让大使回美国再待两年,意味着我们还能继续发展我们两国之间的良好关系。

\mxsay{毛泽东:}对,对,我们之间的关系还要继续发展。我觉得,现在我们两国之间没有什么大问题,你的国家和我的国家。今年、明年、后年,可能都不会有大问题发生在我们两国之间。以后的情况可能会变得更好一点吧。

\mxsay{福特:}主席先生,我认为我们不得不在国际事务上加强合作,应对来自诸如苏联那样国家的挑战。

\mxsay{毛泽东:}对,我们对苏联一点信心也没有。邓小平同志也不喜欢苏联。

\mxsay{福特:}我们也是这样。他们在全世界扩张,掠夺领土和经济资源。我们必须面对他们的挑战。

\mxsay{毛泽东:}好。我们也要面对挑战。

\mxsay{福特:}我们希望从明年开始,我们的双边关系可以得到改善。我们认为,现在正是我们两国关系取得实质性进展的时候。

\mxsay{毛泽东:}你是说我们两国之间?

\mxsay{福特:}对。

\mxsay{毛泽东:}那样很好。

\mxsay{福特:}主席先生,如果你的国家和我的国家一起合作,应对来自苏联的挑战,那么美国和中华人民共和国外交关系正常化,就会在我的国家得到支持。

\mxsay{毛泽东:}很好。不过,我们在这里只是说说而已,苏联到底怎么行动,还需要观察。

\mxsay{福特:}主席先生,我们必须让苏联对美国和中华人民共和国的合作留下深刻印象,言辞是不起作用的,必须有实际行动。我们将不断对他们施加压力。我希望,来自您的压力,和我们的压力一样强大。

\mxsay{毛泽东:}我们就是放放空炮,骂骂人而已。

\mxsay{福特:}主席先生,我们将要做的就不止与此了,过去也是如此。美国人民希望他们的总统坚强有力。我们将来就会这样行动。而不是只说不做、放空炮。

\mxsay{毛泽东:}你们有实心炮?

\mxsay{福特:}对,我们将一直准备火药,如果他们来挑战我们,那么我们的火药就不会闲着。

\mxsay{毛泽东:}很好,那样也不坏。不过,现在你们是和平共处。

\mxsay{福特:}但是,这不意味着我们会对任何扩张主义行为坐视不管。事实上,我们已经在应对这些挑战了,并将继续如此。

\mxsay{毛泽东:}那样很好。我们需要签署协议吗?

\mxsay{福特:}对,我们可以一起努力,达到同样的效果。你从东边施加压力,我从西边施加压力。

\mxsay{毛泽东:}可以。这是君子协定。

\mxsay{福特:}这是最好的取得成功的方法,对付那些不是君子的小人。

\mxsay{毛泽东:}他们是小人。

\mxsay{福特:}我们今天早上用的词,比这个还要强烈。

\mxsay{毛泽东:}我非常感谢总统先生来看我。我希望未来我们两国能够更加友好。

\mxsay{福特:}主席先生,这正是美国人民和我本人的真诚愿望。我希望你们清楚地了解,过去三年来,我们两国之间历史性的行动是得到美国人民全力支持的。美国人民像我们一样,认识到必须采取强力行动,阻止苏联那样的扩张主义国家。我们将维持我们的军事力量,并且准备使用它。在我们看来,这是让世界稳定和发展的最好方法。

\mxsay{毛泽东:}好的。我们之间没有任何矛盾。

\mxsay{福特:}很正确。如果我们之间有矛盾,我们就会坐下来谈判,试图互相理解和消除矛盾。

\mxsay{毛泽东:}很好。我们两国有不同的社会制度和意识形态,矛盾总是存在的。

\mxsay{福特:}但是这不影响我们之间的紧密合作,这对我们两国和所有人都是有利的。

\mxsay{毛泽东:}举例来说,我们和苏联之间就没有我们和你们之间这样的谈话。我去过苏联两次,赫鲁晓夫\mnote{5}到过北京三次。没有一次谈话是真的令人满意的。

\mxsay{福特:}主席先生,我与勃列日涅夫\mnote{6}先生见过两次。有时谈得很顺利,有时谈得不好。我想这正是我们坚定态度的一种外在表现,我们没法同意他们的有些提议,将来也不会同意。我们将保持坚定,维持军事力量。他们理解这一点,我想对我们两国最有利的方法就是,我们一起采取坚定立场。这正是我们打算要做的事。

\mxsay{毛泽东:}好的。

\mxsay{毛泽东:}你们现在同日本的关系怎么样?比以前好吗?

\mxsay{福特:}对,比以前好。主席先生,正如你所知,我大约一年前访问了日本。这是美国在任总统第一次访问日本。大约一个月,日本天皇和皇后访问了美国,这也是他们的元首第一次来我们的国家。我们感到,美日关系正处在第二次世界大战以来的最好时期。

\mxsay{毛泽东:}日本也受到苏联的威胁。

\mxsay{福特:}我同意这一点,因此,主席先生,我认为中日关系的不断改善很重要,就像美日关系在不断改善一样。事实上,美日关系现在是历史上最好的。

\mxsay{毛泽东:}对日本来说,同你们的关系是第一位的,同我们的关系是第二位的。

\mxsay{福特:}你们同日本的关系好吗?

\mxsay{毛泽东:}不算很糟,但也不是很好。

\mxsay{福特:}你们希望中日关系得到改善,对吗?

\mxsay{毛泽东:}对。他们国内有亲苏派,反对谈论霸权。

\mxsay{基辛格\mnote{6}:}也可能是害怕谈论。

\mxsay{毛泽东:}正是这样。

\mxsay{福特:}你们同西欧国家的关系怎么样,主席先生?

\mxsay{毛泽东:}比起我们同日本的关系,好太多、太多了。

\mxsay{福特:}这是非常重要的,我们同西欧、以及你们同西欧都保持良好关系,这样才能应对任何苏联在西欧的扩张。

\mxsay{毛泽东:}对,对,这是我们同你们的一个共同点。我们同欧洲国家没有利益冲突。

\mxsay{福特:}主席先生,事实上,我们中有些人相信,中国比有些欧洲国家更能使得欧洲保持团结,使得北约加强力量。

\mxsay{毛泽东:}他们心不齐。

\mxsay{福特:}有些欧洲国家不像表面上那样强大和坦率。

\mxsay{毛泽东:}我觉得瑞典不坏,西德也不坏。南斯拉夫也很好。荷兰和比利时就稍微差一点。

\mxsay{福特:}很正确。苏联正在设法利用葡萄牙和意大利的弱点。我们必须阻止它,我们打算这样做。

\mxsay{毛泽东:}对,现在葡萄牙看上去更稳定一些了。形势好像在朝好的方面发展。\mnote{8}

\mxsay{福特:}过去四十八小时里,形势发展很令人鼓舞。我们支持的一方正在得势,采取行动稳定局势。我们同意你的看法,南斯拉夫是重要的,在抵抗苏联上发挥着重要的作用。但是我们很关注,铁托\mnote{9}逝世后可能发生的事情。

\mxsay{毛泽东:}对。也许铁托之后会是卡德尔\mnote{10}。

\mxsay{基辛格:}但是,我们关注的是国内外可能产生的反应。我们正在着手做这件事。不同的因素都会对外部集团产生影响。

\mxsay{毛泽东:}对,南斯拉夫有许多省,它是由很多以前的小国家组成的。

\mxsay{福特:}主席先生,今年夏天,我到罗马尼亚进行了一次有趣的旅行。我对齐奥赛斯库\mnote{11}总统的力量和独立性,留下深刻印象。

\mxsay{毛泽东:}很好。

\mxsay{福特:}主席先生,我们对西班牙的局势也很关注。我们支持西班牙国王。我们希望,他能够处理好那些破坏西班牙王国的因素。我们将和他一起工作,在这个转变时期,争取对局势最基本的控制。

\mxsay{毛泽东:}如果欧洲共同体\mnote{12}吸纳西班牙,我们觉得,不管怎么样,这是一件好事。欧洲共同体想要西班牙和葡萄牙吗?

\mxsay{福特:}主席先生,我们正在敦促北约对西班牙更友好一些,哪怕法国反对这样。我们希望,在新国王的领导下,西班牙能够更容易被北约接受。此外,我们感到欧洲共同市场应该对西班牙政府有更多的回应,朝着西欧一体的目标努力。我们将在这两个方向上,尽我们所能。

\mxsay{基辛格:}对于欧洲人来说,他们不够激进。

\mxsay{毛泽东:}是这样吗?对了,以前他们总是打来打去的。还有,以前你们不骂法国。

\mxsay{福特:}对,我们支持西班牙新国王。因为西欧的南部必须保持强大,包括葡萄牙、西班牙、意大利、希腊、土耳其、南斯拉夫。如果我们要抵御苏联的扩张主义,这些国家都必须得到增强。

\mxsay{毛泽东:}好的。我们认为希腊应该变得比现在好。

\mxsay{福特:}对,他们经过了一段困难的日子,我们感到新政府正在正确的方向上前进。我们会帮助他们的。我们希望最终他们会回到北约,成为一个成员国。

\mxsay{毛泽东:}那样不错。

\mxsay{福特:}当然,希腊有一些激进的想法,从我们的观点看,不是很好,会削弱北约,给苏联以可乘之机。

\mxsay{毛泽东:}是吗?!

\mxsay{福特:}让我们看看中东地区,主席先生。我们认为西奈协议\mnote{13}有助于减少苏联在该地区的影响,但是我们也认识到必须尽快实现该地区的广泛和平。等到下一次美国大选一结束,我们就将努力争取实现该地区广泛的和持久的和平。

\mxsay{毛泽东:}持久和平很难做到。

\mxsay{福特:}对,几百年来他们都无法实现和平。但是,只要我们努力实现它,一旦成功,就将消除苏联在该地区的大部分影响。如果局势保持停滞,那么苏联就有机会制造麻烦。因此,我们相信必须不断前进。西奈协议有助于我们同埃及实现良好关系。在下一次大选后,如果我们继续前进,争取更大范围的和平,这对将苏联的影响赶出该地区就会有重大影响。

\mxsay{毛泽东:}我不反对那样做。

\mxsay{福特:}关于南亚次大陆,我们希望通过在迪戈加西亚岛的基地保持我们的影响。当然,我们将不断改善与巴基斯坦的关系。我们已经取消了对他们的武器禁运,所以他们自己可以想办法发展足够的军事力量,使得印度相信,发动任何军事行动都不值得尝试。

\mxsay{毛泽东:}那样很好。

\mxsay{福特:}主席先生,你对孟加拉国的局势有何看法?

\mxsay{毛泽东:}那里的情况正在变好,但还不稳定。我们准备派个大使过去。也许他需要不少时间才能到那里。

\mxsay{福特:}你是否感到担忧,印度可能会插手,对孟加拉国采取军事行动,取得主导权?

\mxsay{毛泽东:}有这种危险。我们必须警惕。

\mxsay{福特:}主席先生,印度总是会做一些不明智的事情,反对其他国家。我希望他们不会在孟加拉国上面犯错误。

\mxsay{毛泽东:}确实如此。如果他们在那里采取行动,我们将会反对他们。

\mxsay{福特:}我们正在同巴基斯坦和伊朗合作,阻止出现这种情况。我们将谴责任何印度的此类行动。

\mxsay{毛泽东:}好,我们又达成了另一项协议。

\mxsay{福特:}我肯定你同我们一样,很关心苏联在印度洋的存在,以及他们在东非的活动。这一类举动,我们都强烈反对。我在这里特别要说一下安哥拉\mnote{14},我们正在那里采取先发制人的行动,防止苏联在那里得到一个在非洲大陆上的据点。

\mxsay{毛泽东:}你们看上去没什么办法,我们也没有。

\mxsay{福特:}主席先生,我想我们双方都可以做得更好。

\mxsay{毛泽东:}我赞同把苏联赶出去。

\mxsay{福特:}如果我们一起好好努力,我们就能做到这一点。

\mxsay{毛泽东:}可以通过刚果和扎伊尔的金沙萨。

\mxsay{邓小平:}(对毛泽东说)这里主要的问题是南非,它会卷进来。这会触怒整个黑非洲。这使得整个事情都变复杂了。

\mxsay{毛泽东:}南非名声\mnote{15}不好。

\mxsay{福特:}但是他们正在阻止苏联扩张。我们觉得这是令人钦佩的。我们给了赞比亚和扎伊尔许多钱。我们想象,如果我们自己采取行动,并且中国和其他国家也采取行动,我们就能够防止苏联获得一个重要的海军基地,防止他们控制安哥拉的主要资源。我们强烈反对古巴的介入。现在,他们在安哥拉有五六千人。我们觉得这不是好事。我们也是这样看待苏联。

\mxsay{邓小平:}你的意思是,你们钦佩南非?

\mxsay{福特:}不。他们强烈反对苏联。他们这样做,完全是为了他们自己的利益。这件事美国没有插手。

\mxsay{邓小平:}在安哥拉。

\mxsay{福特:}南非反对安哥拉人民解放军。

\mxsay{毛泽东:}这个问题需要研究。

\mxsay{福特:}刻不容缓。

\mxsay{毛泽东:}我觉得安哥拉人民解放军不会成功。

\mxsay{福特:}我们也希望他们不会成功。

\mxsay{基辛格:}如果其他两个派别足够训练有素,我们就可以提供他们武器。这样我们就能防止安哥拉人民解放军获胜。那两个派别需要得到训练,需要理解游击战。我们可以提供他们武器,但是需要其他人训练他们。

\mxsay{毛泽东:}过去,我们通过坦桑尼亚支持他们。但是坦桑尼亚会扣留某些东西。也许现在我们可以通过扎伊尔。

\mxsay{邓小平:}可能还是通过扎伊尔比较好。

\mxsay{基辛格:}那就通过扎伊尔。中国方面也许可以使用它和莫桑比克的影响。对于非洲来说,如果莫桑比克反对苏联集团和安哥拉人民解放军,那么就有巨大的象征意义。

\mxsay{福特:}但是,你知道,莫桑比克支持安哥拉人民解放军。这可能很难吧。

\mxsay{邓小平:}不可能的事。

\mxsay{基辛格:}我知道。不过他们可能不理解他们正在干什么,因为他们非常尊敬中国。

\mxsay{毛泽东:}我们可以试一下。

\mxsay{基辛格:}我不觉得莫桑比克理解安哥拉到底发生了什么事。他们需要有人给他们提建议,他们更愿意听中国的话,而不是我们的话。

\mxsay{毛泽东:}我们可以试一下。

\mxsay{邓小平:}我们可以试,但是不太可能成功。

\mxsay{基辛格:}确实如此。

\mxsay{毛泽东:}扎伊尔可能更可靠一些。

\mxsay{基辛格:}扎伊尔应该成为一个提供支援的基地。我们从莫桑比克得不到帮助,但是也许他们会置身事外。我们不指望从莫桑比克得到帮助,但是可能他们至少会保持中立。

\mxsay{毛泽东:}我们试一下。

\mxsay{福特:}我要再说一次,刻不容缓。因为其他两个派别需要得到支援。直到不久前,他们都做得很好。目前,局势陷入僵局。如果在我们、你们和其他国家做出努力之后,安哥拉人民解放军依然占据优势,那将是一场悲剧。

\mxsay{毛泽东:}这很难说。你觉得事情就是这样了?

\mxsay{福特:}我可能会说,对于安哥拉就是这样。在动身离开华盛顿前,我刚刚批准对那两个派别提供三千五百万美元的援助。这充分表明了,我们将面对苏联的挑战,击败安哥拉人民解放军。

\mxsay{毛泽东:}好的。

\mxsay{福特:}主席先生,我要谢谢你,给了我这样一个机会讨论世界局势,表达了我们发展双边关系、在许多问题上一起合作、解决许许多多世界冲突的愿望。

\mxsay{毛泽东:}对,现在有一些记者说,我们两方的关系非常坏。也许你应该向他们透露一些消息,给他们一些内幕。

\mxsay{基辛格:}这需要双方努力。他们在北京也打探到了一些消息。

\mxsay{毛泽东:}但是这与我们无关,是一些外国人透露出去的。

\mxsay{福特:}主席先生,我们不相信报纸上的话。我想,非常重要的是,我们两个国家要一起给全世界留下一个印象,那就是我们的关系很好。当我返回美国的时候,我就会说我们有很好的关系。我希望你的人也这样做。重要的不仅仅是我们两方关系好,而是还要让世界相信我们的关系好。

\mxsay{毛泽东:}我们会一点点来的。

\mxsay{福特:}我们也会努力的。

\mxsay{毛泽东:}那就这样吧。

\begin{maonote}
\mnitem{1}邓小平,时任中共中央副主席、国务院副总理、中央军委副主席、中国人民解放军总参谋长。
\mnitem{2}黄镇,一九七三年五月中美两国在对方首都互建联络处,建立大使级外交关系,黄镇任驻美联络处主任。
\mnitem{3}乔治·赫伯特·沃克·布什,又称老布什,一九七四年至一九七五年任美国驻北京联络处主任,后在一九八九年一月至一九九三年一月期间,担任美国第四十一届总统。
\mnitem{4}乔老爷,指乔冠华,时任外交部长。
\mnitem{5}赫鲁晓夫,原任苏共中央第一书记。一九六四年下台。
\mnitem{6}勃列日涅夫,当时的苏共中央第一书记。
\mnitem{7}基辛格,时任美国国务卿。
\mnitem{8}一九七四年四月二十五日,葡萄牙左派军队发动政变,推翻了持续四十二年的萨拉查法西斯专制统治,开始民主化进程。在政变期间,军人用康乃馨花来代替枪中子弹,又称“康乃馨革命”。一九七五年政府宣布放弃非洲殖民地的管辖权,一九七六年举行第一次选举,选出了首相,实现了葡萄牙的自由民主化。
\mnitem{9}铁托(一八九二——一九八〇),南斯拉夫社会主义联邦共和国总统。铁托是不结盟运动的发起者。他反对大国垄断国际事务,反对战争,维护和平,为南斯拉夫赢得了国际尊重。一九九一年,南斯拉夫社会主义联邦共和国的斯洛文尼亚、克罗地亚、波斯尼亚和黑塞哥维纳和马其顿四个共和国相继宣布独立。一九九二年四月二十七日,南斯拉夫社会主义联邦共和国议会通过新宪法,宣布塞尔维亚和黑山两个共和国联合成立南斯拉夫联盟共和国。南斯拉夫社会主义联邦共和国正式解体。
\mnitem{10}爱德华·卡德尔,南斯拉夫党和国家杰出领导人,著名的马克思主义理论家,国际共产主义运动的活动家。时任南斯拉夫联邦政府副主席兼外交部长。一九七九年去世。
\mnitem{11}齐奥赛斯库,罗马尼亚社会主义共和国总统。
\mnitem{12}欧洲共同体,西欧国家推行欧洲经济、政治一体化,并具有一定超国家机制和职能的国际组织。一九六五年四月八日,欧洲煤钢共同体、欧洲原子能共同体和欧洲经济共同体统一起来,称欧洲共同体,简称欧共体。欧共体向欧洲联盟的前身。
\mnitem{13}一九六七年第三次中东战争中,以色列占领了埃及的西奈半岛。此后,美国为了排除苏联在中东的影响,致力于撮合埃及和以色列进行关于西奈半岛问题的和平谈判。

一九七五年九月二日,埃及和以色列关于西奈的协议签字。这是继一九七四年一月埃以达成两国军队脱离接触协议之后,关于西奈问题的第二项协议。在协议中,双方都作出了让步。以色列将从西奈部分领土上撤走并放弃了结束战争状态的要求。埃及则同意对以色列的非军事物资开放苏伊士运河。
\mnitem{14}安哥拉,从二十世纪五十年代起,安哥拉人民为反抗葡殖民统治,先后成立了三个民族解放组织:安哥拉人民解放运动(简称安人运)、安哥拉民族解放阵线(简称安解阵)和争取安哥拉彻底独立全国联盟(简称安盟),并于六十年代相继开展争取民族独立的武装斗争并得到国际社会的广泛支持。一九七四年四月二十五日,葡萄牙爆发了“康乃馨革命”,推翻了萨拉查法西斯专制统治,新政府宣布放弃殖民主义政策。一九七五年一月十五日,上述三个组织同葡政府达成关于安哥拉独立的《阿沃尔协议》,并于一月三十一日同葡当局共同组成过渡政府。不久,安人运、安盟和安解阵之间发生武装冲突,过渡政府解体。同年十一月十一日,安人运成立安哥拉人民共和国;安解阵和安盟则成立了安哥拉人民民主共和国。同日,安两派三方武装冲突扩大为全面内战。安人运受苏联支持,另外两个派别受美国与南非支持,国内仍持续呈现战争状态,直到二〇〇二年四月四日签订停战协定,才结束了长达二十七年的内战。
\mnitem{15}南非共和国在一九四八年至一九九〇年实行种族隔离制度,这个制度对白人与非白人(包括黑人、印度人、马来人、及其他混血门族)进行分隔并在政治经济等各方面给予歧别待遇。南非的种族隔离政策不但引发国内的反弹与抗争,更引发国际社会的攻击与经济制裁;一九七三年十一月三十日联合国大会通过的《禁止并惩治种族隔离罪行国际公约》宣布:种族隔离违反国际法原则,特别是联合国宪章的宗旨和原则,对国际和平和安全构成严重的威胁,是危害人类的罪行。凡是有种族隔离行为的组织、机构或个人即为犯罪。缔约国承担义务,禁止、预防并惩治犯有这种罪行的人。
\end{maonote}
