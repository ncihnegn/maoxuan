
\title{要安定团结不要分裂,不要搞阴谋诡计}
\date{一九七五年五月三日}
\thanks{这是毛泽东同志召集在京中共中央政治局委员的谈话纪要。}
\maketitle


多久不见了\mnote{1},有一个问题,我与你们商量,一些人思想不一致,个别的人。我自己也犯了错误,春桥那篇文章\mnote{2},我没有看出来,只听了一遍,我是没有看,我也不能看书,讲了经验主义的问题我放过了。新华社的文件,文元给我看了,对不起春桥\mnote{3}。还有上海机床厂的“十条经验”\mnote{4},都说了经验主义,一个马克思主义都没有,也没有说教条主义。办了一个大学,很多知识分子,他们觉得外国月亮比中国的好。

要安定,要团结。无论什么问题,无论经验主义也好,教条主义也好,都是修正马列主义,都要用教育的方法。现在要安定团结。

现在我们的一部分同志犯了错误要批评。三箭齐发,批林,批孔,批走后门\mnote{5}。批林批孔都要这些人来干,没有这些人批林批孔就不行。走后门这样的人有成百万,包括你们(王海容、唐闻生)在内,我也是一个,我送了几个女孩子,到北大上学,我没办法,我说你们去上学,他们当了五年工人,现在送她们上大学了,我送去的,也是走后门,我也有资产阶级法权,我送去,小谢\mnote{6}不得不收,这些人不是坏人。

在这里我同小平\mnote{7}同志谈过一次。

你们\mnote{8}只恨经验主义,不恨教条主义,二十八个半统治了四年之久,打着共产国际的旗帜,吓唬中国党,凡不赞成的就要打,俘虏了一批经验主义。你(周恩来)一个,朱德一个,还有别的人,主要是林彪、彭德怀。我讲恩来、朱德不够,没有林彪、彭德怀还没有力量。林彪写了短促突击\mnote{9},称赞华夫\mnote{10}文章,反对邓、毛、谢、古。邓是你(邓小平),毛是毛泽覃,谢是谢唯俊,古是古柏,其他的人(除邓以外)都牺牲了,我只见过你一面(邓小平),你就是毛派的代表。

教育界、科学界、新闻界、文化艺术界,还有好多了,还有医学界,外国人放个屁都是香的。害得我有两年不能吃鸡蛋,因苏联人发表了一篇文章,说里面有胆固醇。后来又一篇文章说胆固醇不要紧,又说可以吃啦。月亮也是外国的好,不要看低教条主义。

有经验主义的人多,无非是不认识几个字,马列也不能看,他们只好凭经验办事。历来对经验主义是没有办法,我是没有办法,慢慢来,还要十年、八年,二十年,三十年可以好一些。太急了不好,不要急,一些观念连不起来。

我说的是安定团结,(既要批)教条主义、经验主义、修正主义,又要批评资产阶级法权,不能过急,你们谁要过急就要摔下来。

不要分裂,要团结。要马列主义,不要搞修正主义;要团结,不要分裂;要光明正大,不要搞阴谋诡计。不要搞四人帮,你们不要搞了,为什么照样搞呀?为什么不和二百多个中央委员搞团结,搞少数人不好,历来不好。这次犯错误,还是自我批评。这次和庐山会议不同,庐山会议反对林彪是对的。这一次还是三条,要马列不要修正,要团结不要分裂,要光明正大,不要搞阴谋诡计,就是不要搞宗派主义。这三条我重复一遍,要搞马列主义,不要搞修正主义;要团结,不要分裂;要光明正大,不要搞阴谋诡计。其他的事你们去议,治病救人,不处分任何人,一次会议解决不了。我的意见,我的看法,有的同志不信三条,也不听我的,这三条都忘记了。九大、十大都讲这三条,这三条要大家再议一下。

教育界、科学界、文艺界、新闻界、医务界,知识分子成堆的地方,其中也有好的,有点马列的。你们外交部也是知识分子成堆的地方,讲错了没有?(面向王海容、唐闻生)你们两个是臭知识分子,你们自己承认,臭老九,老九不能走。

我看问题不大,不要小题大作,但有问题要讲明白,上半年解决不了,下半年解决;今年解决不了,明年解决;明年解决不了,后年解决。我看批判经验主义的人,自己就是经验主义,马列主义不多,有一些,不多,跟我差不多。不作自我批评不好:要人家作,自己不作。

江青同志党的一大半没有参加,陈独秀、瞿秋白、李立三、罗章龙、王明、张国焘,都没有参加斗争,没有参加长征,所以也难怪。我看江青就是一个小小的经验主义者,教条主义谈不上,她不像王明那样写了一篇文章《更加布尔什维克化》,也不会象张闻天那样写《机会主义的动摇》\mnote{11}。不要随便,要有纪律,要谨慎,不要个人自作主张,要跟政治局讨论,有意见要在政治局讨论,印成文件发下去,要以中央的名义,不要用个人的名义,比如也不要以我的名义,我是从来不送什么材料的。这一回跑了十个月,没有讲过什么话,没有发表什么意见,因为中央没有委托我,我在外面养病,我一面养病,一面听文件,每天都有飞机送。现在上帝还没要我去,我还能想,还能听,还能讲,讲不行还能写。我能吃饭,能睡觉。

要守纪律,军队要谨慎,中央委员更要谨慎。我跟江青谈过一次,我跟小平谈过一次。王洪文要见我,江青又打来电话要见我,我说不见,要见请大家一起来。

完了。对不起,我就是这样,我没有更多的话,就是三句,九次、十次代表大会都是三句,要马列不要修正,要团结不要分裂,要光明正大不要搞阴谋诡计。不要搞什么帮,什么广东帮、湖南帮,粤汉铁路长沙修理厂不收湖南人,只收广东人,广东帮。

\begin{maonote}
\mnitem{1}毛泽东自一九七四年七月去南方休养了十个月,刚刚回到北京。
\mnitem{2}春桥,即张春桥,时任中共中央政治局常委、国务院副总理、中国人民解放军总政治部主任。毛泽东曾责成张春桥、姚文元写文章论述无产阶级专政。姚文元署名在《红旗》杂志一九七五年三月一日第三期发表了《论林彪反党集团的社会基础》,提出反经验主义问题。张春桥署名在《红旗》杂志一九七五年四月一日第四期发表了《论对资产阶级的全面专政》,提出要对资产阶级实行全面专政。两篇文章实为两人合作写成送毛泽东审定。

毛泽东说的这篇文章指《论林彪反党集团的社会基础》。
\mnitem{3}文元,即姚文元,时任中共中央政治局委员。

一九七五年二月九日,《人民日报》发表题为《学好无产阶级专政的理论》的社论,“学习无产阶级专政理论”运动是这时整个舆论宣传的中心。

一九七五年三月一日,张春桥在全军各大单位政治部主任座谈会提出“对经验主义的危险,恐怕还是要警惕”,毛主席一九五九年关于反经验主义的话“现在仍然有效”,要把反对经验主义“当作纲”。同一天,姚文元发表的《论林彪反党集团的社会基础》提出了“现在,主要危险是经验主义”这个意见。此后,江青也提出“经验主义是当前的大敌”,“党现在最大的危险不是教条主义而是经验主义。”并在一次政治局会议上正式提出这个问题。三四月间,《人民日报》、《光明日报》、《文汇报》、《解放日报》等报刊发表了许多反“经验主义”的文章。邓小平表示,他不同意关于“经验主义是当前主要危险”的提法。

姚文元曾于一九七五年四月二十日送审新华社《关于报道学习无产阶级专政理论问题请示报告》的文件,这个报告中说,在今后一段时间里,我们要大力报道各级干部认真读书,刻苦钻研,决心弄通无产阶级专政的理论,反对学习中不求甚解的作风。特别要注意宣传各级干部通过学习,认识和批判经验主义的危害,自觉克服经验主义。

毛泽东批语:“提法似应提反对修正主义,包括反对经验主义和教条主义,二者都是修正马列主义的,不要只提一项,放过另一项。各地情况不同,都是由于马列水平不高而来的。不论何者都应教育,应以多年时间逐渐提高马列为好。我党真懂马列的不多,有些人自以为懂了,其实不大懂,自以为是,动不动就训人。这也是不懂马列的一种表现。此问题请提政治局一议。”

四月二十七日,政治局开会讨论,批评了江青、张春桥等人。《论林彪反党集团的社会基础》这篇文章发表前曾送毛泽东审定,所以毛泽东现在批评张春桥又说对不起张春桥。
\mnitem{4}指江青等呈送的上海机床厂搞的批“经验主义十条表现”的材料。
\mnitem{5}一九七四年的批林批孔运动,是毛主席为首的党中央领导和发起的,是周总理、叶剑英分别组织、动员地方和军队开展的一次运动。

一九七四年一月十八日,中共中央以当年一号文件正式下达关于印发《林彪与孔孟之道(材料之一)》的通知。通知说:林彪“和历代行将灭亡的反动派一样,尊孔反法,攻击秦始皇,把孔孟之道作为阴谋篡党夺权、复辟资本主义的反动思想武器。北京大学、清华大学选编的这个材料,对于继续深入批林,批判林彪路线的极右实质,对于继续开展对尊孔反法思想的批判,对于加强思想和政治路线方面的教育,会有很大帮助”。

一九七四年一月二十四日,叶剑英亲自召集并主持的驻京部队批林批孔动员大会。

一九七四年一月二十五日,周恩来在北京召集并主持召开党中央和国务院机关“批林批孔”动员大会。迟群等人发言称:“批林批孔”所要联系的实际之一,就是揭批“走后门”问题;“走后门”实际上“就是对马列主义的背叛”。二月十五日,毛泽东针对大批“走后门”一事指出:“此事甚大,从支部到北京牵涉几百万人。开后门来的也有好人,从前门来的也有坏人。现在,形而上学猖獗,片面性,批林批孔,又夹着(批)走后门,有可能冲淡批林批孔。”

二月二十日,中央根据毛泽东的批示发出通知:对批林批孔运动中不少单位提出的领导干部“走后门”送子女参军、入学等问题,应进行调查研究,确定政策,放运动后期妥善解决。
\mnitem{6}小谢,谢静宜,时任清华大学革命委员会副主任,中共北京市委常委、书记,北京市革委会副主任。
\mnitem{7}小平,指邓小平,一九七五年一月五日,中共中央发出文件,任命邓小平为中共中央军委副主席兼中国人民解放军总参谋长。随后,在一月八日至十日举行的中共十届二中全会上,邓小平当选为中共中央副主席、中央政治局常委。在一月十三日至十七日举行的四届全国人大一次会议上,邓小平被任命为副总理。时周总理病重。会后,邓小平实际上开始主持工作。
\mnitem{8}你们,指江青、王洪文、张春桥、姚文元。
\mnitem{9}林彪(一九〇七——一九七一),湖北黄冈人。一九二五年加入中国共产党。一九五八年五月在中共八届五中全会上被增选为中共中央副主席、政治局常务委员。一九五九年任中央军委副主席、国防部长,主持中央军委工作。在九届二中全会上主张设国家主席(毛泽东主席明确表示要改变国家体制不设国家主席),并组织人企图压服中央,犯了错误,被毛泽东主席识破,对其进行了警告和批评,并等待其认错达一年之久(从一九七〇年九月到一九七一年九月),不料,其子林立果狂妄自大,趁毛泽东南巡之时,妄图谋杀毛泽东主席,事情败露后,九月十三日夜,林立果挟制林彪和叶群驾机逃往苏联,最后坠毁于蒙古温都尔汗,史称“九一三”事件。后,林立果制定的《“五七一”工程纪要》被发现,因此,中央认定,林彪叛国。一九七三年八月中共中央决定,开除他的党籍。

林彪一九三四年六月十七日发表了《论短促突击》的文章,提出了二十七条实施措施和注意事项,文章最后说:最后让我引用华夫同志的话来结束我这篇文字吧!“我们要特别指出最危险的简单化及机械化的应用战术原则。敌人的和我们的战术都是在发展中变更中成就中,若以这些原则引以为足时,那就要在目前的战斗环境中算落伍了。因此我们必须估计每次战斗的经验,来补足及变更我们的战术。在这创造的工作中我们应造成最后的胜利最重要的前提之一”。
\mnitem{10}即李德,被共产国际派到中国,担任中共中央的军事顾问。他在担任军事顾问期间,推行军事教条主义,他不懂得中国的国情,也不认真分析战争的实际情况,只凭在军事课本上学到的条条框框,在苏区进行指挥。当时任临时中央书记的博古,把军事指挥大权交给李德,由他一人凭着地图指挥战斗。当时的地图大部分是一些简单的草图,极不准确,所以他的指挥往往与前线实际情况差距很大。在红军的第五次反“围剿”,都是按李德“短促突击”、“两个拳头作战”、“御敌于国门之外”、建立正规军打阵地战这一套错误路线进行的。他曾以华夫为笔名在《革命与战争》杂志上发表了不少评论,他是第三次“左”倾机会主义错误在军事上的推行者。一九三五年一月党中央在贵州遵义召开政治局扩大会议,李德列席了会议,在受到会议批判后,被取消了最高军事指挥权,撤销了军事顾问的职务。
\mnitem{11}王明,实际主持当时临时中央工作,一九三一年七月十五日发表了《为中共更加布尔什维克化而斗争》的小册子,照搬苏联经验,推行“左”倾教条主义路线。一九三二年四月四日,时任临时中央政治局委员和政治局常委的张闻天发表《在争取中国革命在一省与数省的首先胜利中中国共产党内机会主义的动摇》长文,系统批判毛泽东,指出当前苏区的主要危险,“是对国民党统治的过分估计,与对于革命力量的估计不足的右倾机会主义。”
\end{maonote}
