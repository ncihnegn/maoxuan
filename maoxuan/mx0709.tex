
\title{同胡志明的谈话——不搞清官贪官,搞文化大革命}
\date{一九六六年六月十日}
\thanks{这是毛泽东同志在杭州会见胡志明\mnote{1}的谈话节选。}
\maketitle


\section*{一}

\mxsay{毛泽东:}一切事物都是一分为二,对立统一。事物总是有两个对立面。你们党如果只有完全的团结,没有对立面,就不符合实际。全世界的党都分裂嘛。马克思、恩格斯没有料到他们的接班人伯恩斯坦、考茨基成为反马克思主义者,他们创立和领导的党——德国社会民主党、法国社会党等,在他们死后,就成为资产阶级的党。这条不注意要吃亏的。

天下乌鸦一般黑。只要理解了,我们有准备,全党大多数人有准备,不怕。我们都是七十以上的人了,总有一天被马克思请去。接班人究竟是谁,是伯恩斯坦、考茨基\mnote{2},还是赫鲁晓夫,不得而知。要准备,还来得及。总之,是一分为二,不要看现在都是喊“万岁”的。

\section*{二}

\mxsay{胡志明:}现在中国的四个省正在帮助越南的七个省,如广东帮助我们的广宁省等。

\mxsay{毛泽东:}帮助些什么?

\mxsay{胡志明:}帮助搞农业生产、水利、改良稻种、牛种,还帮助办小型发电厂、小工厂,搞改良农具等等。

\mxsay{毛泽东:}你们的炼钢厂搞起来了吗?

\mxsay{胡志明:}已开始投入生产。敌机也已开始在附近轰炸。

\mxsay{毛泽东:}年产钢多少?

\mxsay{胡志明:}很少,还没有轧钢设备。

\mxsay{毛泽东:}你们那里有煤矿、铁矿吗?铁砂从那里去的?从中国去吗?

\mxsay{胡志明:}越南有煤、有铁。中国也去一点。

\mxsay{毛泽东:}没有钢,没有机械工业,国家就没有整套的工业。

\mxsay{胡志明:}可能你还记得,是我在见你后,在你的鼓舞下,才建钢厂的。

\mxsay{毛泽东:}我最关心钢铁工业和机械工业。

\mxsay{胡志明:}我们太原钢厂的设备和专家都是中国的。

\mxsay{毛泽东:}可以从小型开始,逐步发展。有个什么十几年,就可以搞成。小型轧钢机可以从中国弄去,炼钢可以采用新的技术。我们已经开始搞用氧气炼钢。有些新技术也可以从中国弄去。初步,不要搞急了,搞多了,我们吃了搞急了、搞多了的亏,一年搞了一千七百个基本建设项目。搞了几年不行,然后缩小下来,变成七百多个。你看,减了一千个,有的已经搞成了,没有搞成的基建单位,就象癞痢头一样。那时就是贪大、贪多、贪全。可惜你没有到锦州去看看。那里搞了许多小工业。没有资金,干部、工人每人凑一点,没有钱盖房子,就搭个草棚。现在出了许多新产品。

\section*{三}

\mxsay{毛泽东:}我们最近这场斗争\mnote{2},是从去年十一月开始的,已经七个多月了。最初,姚文元发难。他是个青年人。讨论清官等问题。你不是赞成清官吗?你说世界上有清官,我就没有见过。无官不贪,只有多少之别,没有真正的清官。

\mxsay{胡志明:}我的父亲当了知县,他没有贪。

\mxsay{毛泽东:}不见得,那时你还小,他贪你不知道。当知县可了不起。

\mxsay{胡志明:}当了几个月,他就被撤职了。

\mxsay{毛泽东:}那是他来不及贪,当上一两年知县,我看他不大贪也小贪。现在我们不搞清官、贪官这件事了,搞文化大革命。搞教育界、文艺界、学术界、哲学界、史学界、出版界、新闻界。文艺界又分好多界,有戏剧界、电影界、音乐界、美术界、雕刻界;戏剧界又分京戏和几百种地方戏。

\mxsay{毛泽东:}今天,我只睡了两个小时,因为心里有事,要见你这胡伯伯。我打听你几点钟睡,知道你五点钟起床,好,我七点见你。前天我睡了八个小时,昨天睡了八个小时,今天睡两个小时够了。夏天,有时我几天不睡觉。现在,主要是看大字报;报纸上也很热闹。大字报厉害得很,有群众性,轰轰烈烈。你可以到浙江大学去看一看嘛,晚上,化装去,戴上口罩去看一看嘛;这是发动群众整反动分子的一个好办法。

\mxsay{胡志明:}一九五七年时我也在中国看过大字报。

\mxsay{毛泽东:}没有这一次深入、广泛。这次是大大小小可能要整倒几百人、几千人,特别是学术界,教育界、新闻界、出版界、文艺界、大学、中学、小学。因为当时我们没有人,把国民党的教员都接受下来了。大、中、小学教员,办报的,唱戏的,写小说的,画画的,搞电影的,我们很少,把国民党的都包下来。这些人都钻到我们党内来了。这样一说,你就知道文化大革命的道理了。

\mxsay{胡志明:}中国有的,越南也有。中国搞的,越南也要搞,虽然越南的规模要小一些。越南的情况同中国是一样的。

\mxsay{毛泽东:}可能都是一样。你们也有小学、中学、大学教师,这些人还不都是旧知识分子。党内的人也是来自五湖四海,各种人都有。我们党有百分之八十的人是一九四九年以后入党的。他们没有经过什么风浪,没有经过斗争,其中好的也有,坏的也有。

\mxsay{胡志明:}所以有矛盾。

\mxsay{毛泽东:}就是有矛盾,我同很多人有矛盾。

\mxsay{胡志明:}从你谈的历史情况来看,问题真是复杂。

\mxsay{毛泽东:}斗争很复杂,但党并没有灭亡。

\mxsay{胡志明:}听了毛主席、刘主席\mnote{4}等同志的谈话,我吸取了一些经验,也比过去更加担忧了。

\mxsay{毛泽东:}一方面要担忧,一方面要乐观。党不会灭亡,天塌不下来,山上的树木照样长,水里的鱼照样游,女人照样生孩子。若不信,你看看嘛。难道出了赫鲁晓夫,天就会塌下来,山上的树木就不长,水里的鱼就不游,女人就不生孩子了吗?我就不信!事物的发展不断地转向它的反面。马克思、恩格斯死后,他们的接班人成为反马克思主义者。列宁死后还有斯大林一代。斯大林没有料到,他死了之后,赫鲁晓夫反对他,而且反得那样不近人情。

\begin{maonote}
\mnitem{1}胡志明,越南共产主义革命家,时任越南劳动党主席,越南民主共和国主席。胡志明本人的汉语说得极为流利,仅略带广东口音。
\mnitem{2}一八九五年恩格斯去世后,爱德华·伯恩斯坦打着发展马克思主义的旗号,发表大量理论文章,形成一套修正主义的理论体系,代表作是《社会主义的前提和社会民主党的任务》一书,对马克思主义的三个组成部分进行了全面的修正。由于伯恩斯坦是以发展马克思主义的名义而对马克思主义进行全面“修正”的,所以,修正主义一词即由此而来。此后,凡是披着马克思主义外衣,打着“发展”马克思主义的旗号而实际上则篡改马克思主义基本原理的机会主义思潮,都被称作修正主义。伯恩斯坦也理所当然的成为世界修正主义的鼻祖。

在第一次世界大战前夕,考茨基在关于时代、战争与和平、无产阶级革命和无产阶级专政等一系列重大问题上,形成了一套完整的机会主义理论,成为当时国际共产主义运动的主要危险。
\mnitem{3}指文化大革命。
\mnitem{4}刘主席,指刘少奇,时任国家主席。
\end{maonote}
