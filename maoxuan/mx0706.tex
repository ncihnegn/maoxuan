
\title{为什么要搞文化大革命}
\date{一九六六年五月五日}
\thanks{这是毛泽东同志在接见阿尔巴尼亚党政代表团代表团时的谈话节选。}
\maketitle


事物的发展是不以人的意志为转移的。马克思、恩格斯就没有料到手创的社会民主党在他们死后被他们的接班人篡夺领导权,变为资产阶级政党,这是不以马克思、恩格斯的意志为转移的。他们那个党开始是革命的,他们一死变成反革命的了。苏联也不以列宁的意志为转移,他也没有料到会出赫鲁晓夫修正主义\mnote{1}。

事物不断地走向反面。不仅是量变,而且要起质变;只有量变,不起质变,那就是形而上学。我们也准备着。你晓得哪一天修正主义占领北京?现在这些拥护我们的人摇身一变,就可以变成修正主义。这是第一种可能。第二种可能是部分分化。

鉴于这些情况,我们这批人一死,修正主义很可能起来。

我们是黄昏时候了,所以,现在趁着还有一口气的时候,整一整这些资产阶级复辟。

总之,要把两个可能放在心里:头一个可能是反革命专政,反革命复辟。把这个放在头一种可能,我们就有点着急了,不然就不着急,太平无事。如果你不着急,太平无事,那就好了?才不是那样。光明的一面现在看出来了,还有更主要的一面,有黑暗的一面。他们在做地下工作。列宁讲过,被打败了的剥削阶级长期还强于胜利的无产阶级。列宁又讲,农民、小资产阶级每日、每时都生长资本主义。打败了的阶级是哪些人?帝国主义、封建主义、资本主义。而群众就是工人、农民、城市小资产阶级。还有中国的民族资产阶级和他们的知识分子,我们都包进来了。还有地主阶级的儿女。过去我们的大学生大多数是资产阶级、地主阶级的儿女。工人、贫农、下中农都进不起学校,小学都进不上,进上小学进不上中学,何况进大学?旧的知识分子至少有几百万人。群众的文化教育掌握在他们手里,我们没有掌握。那么多小学,我们没有小学教员,只好用国民党留下来的小学教员;我们也没有自己的中学教员、大学教授、工程师、演员、画家,也没有搞出版社和开书店的人员。那些旧人有一部分钻到党内来,暂时潜伏不动,待机而起。等于赫鲁晓夫潜伏不动,待机而起一样。

第二个可能就是剥笋政策,一层一层地剥掉,剩下的是好的,把坏的剥掉。从一九二一年到一九六六年四十五年了,我们就初步地剥了一遍,剥掉了不少反动的:陈独秀、瞿秋白、李立三、王明、张国焘、张闻天、高岗、饶漱石、彭德怀、罗瑞卿\mnote{2}、彭真等等前后几十个中央委员,还有睡在我们身边没有发现的。

不要怕反革命。有的时候我也很忧虑。说不想、不忧虑,那是假的。但是睡觉起来,找几个同志开个会,议论议论,又想出办法来了。

\begin{maonote}
\mnitem{1}赫鲁晓夫修正主义,赫鲁晓夫(一八九四——一九七一),曾任苏联党和国家主要领导人,斯大林生前吹捧斯大林是自己的“生身父亲”,斯大林去世后发布秘密报告,咒骂斯大林是“凶手”、“刑事犯”、“强盗”、“赌棍”、“伊凡雷帝式的暴君”、“俄国历史上最大的独裁者”、“混蛋”、“白痴”,全盘否定斯大林,并把斯大林的遗体从列宁墓中迁出。

斯大林领导社会主义阵营三十年,是公认的革命导师和领袖:

他领导苏联共产党和苏联人民,同国内外的一切敌人进行了坚决的斗争,保卫了并且巩固了世界上的第一个社会主义国家;他领导苏联共产党和苏联人民,在国内坚持了社会主义工业化和农业集体化的路线,取得了社会主义改造和社会主义建设的伟大成就;他领导苏联共产党、苏联人民和苏联军队,进行了艰苦卓绝的战斗,取得了反法西斯战争的伟大胜利;他的一系列理论著作,是马克思列宁主义的不朽文献,对国际共产主义运动作出了不可磨灭的贡献,捍卫和发展了马克思列宁主义;他领导的苏联党和政府,从总的方面来说,实行了符合无产阶级国际主义的对外政策,对世界各国人民的革命斗争给了巨大的援助,指导帮助了中国、朝鲜、越南、东欧等多个国家的革命斗争,建立了社会主义阵营。

赫鲁晓夫否定斯大林,其实就否定了苏共自己,否定了苏联,否定了社会主义阵营,否定了共产主义信仰。

从苏共第二十次代表大会开始,赫鲁晓夫提出了同十月革命道路根本对立的所谓“和平过渡”道路,也就是“通过议会的道路向社会主义过渡”。

赫鲁晓夫认为,无产阶级只要取得议会中的多数,就等于取得政权,就等于粉碎资产阶级的国家机器,修正了马列主义的基本原理:“一切革命的根本问题是国家政权问题。无产阶级革命的根本问题,就是用暴力夺取政权,打碎资产阶级国家机器,建立自己的阶级专政,用无产阶级国家代替资产阶级国家。”

对赫鲁晓夫修正主义的批判详见由毛泽东同志主持编写的《九评苏共中央的公开信》。
\mnitem{2}罗瑞卿,时任中共中央书记处书记、国务院副总理、中央军委常委、中央军委秘书长、中国人民解放军总参谋长、国防部副部长、国防工业办公室主任,一九六五年十二月八日至十五日,党中央在上海召开了会议,揭发和批判了罗瑞卿的错误,叶剑英、聂荣臻、谢富治、箫克、杨成武、刘志坚等相继发言,批判了罗瑞卿“反对突出政治、把林彪同志实际当作敌人对待、不尊重各元帅、个人独断”的错误,最后,毛泽东提出:“如果没有这三条(指反林、向党伸手、反对突出政治),可以把问题先挂起来。中国有很多问题都是挂起来的。挂几百年不行,可以挂一万年。有什么就检讨什么。”

一九六六年一月五日,毛泽东同江西省党政负责人杨尚奎、方志纯谈到了罗瑞卿的问题,他说:“这个人就是盛气凌人,锋芒毕露。”“我也同罗瑞卿说过,要他到哪个省去搞个省长,他不干。军队工作是不能做了。要调动一下,可以到地方上去做些工作,也不一定调到江西来。”

一九六六年三月三日,在京西宾馆召开的军委扩大会议对他进行了批判,三月十八日,罗瑞卿从办公三楼跳下,摔断了腿,毛泽东听说后,骂了句“没出息”。一九六六年六月二十七日下午四时,在人民大会堂安徽厅,刘少奇在中共中央召集的民主人士座谈会上通报“彭真、罗瑞卿、陆定一、杨尚昆”错误时,是这样说的:“在今年二、三月间,召集了几十个人,有地方、军队干部参加的会议,进一步讨论罗瑞卿的问题,他在会上讲了一次话,大家不满意,没有让他过关,此时他就在自己住的三层楼跳楼自杀,受了点伤,没有死,现在住在医院里,本来,自杀要有点技术,应该是头重脚轻,他却是脚先落地,脚坏了点,头部没有伤。(邓小平:就象女跳水运动员那样,跳了一根冰棍)他的这种行动,是对抗情绪,是严重地对抗党,对抗同志们的批评。”

一九六六年四月三十日中央工作组向中共中央提交了《关于罗瑞卿错误问题的报告》报告,由中共中央一九六六年五月十六日转发全党,指出罗的错误有:

第一、敌视和反对毛泽东思想,诽谤和攻击毛泽东同志。

第二、推行资产阶级军事路线,反对毛主席军事路线,擅自决定三军大比武,反对突出政治。

第三、目无组织纪律,个人专断,搞独立王国,破坏党的民主集中制。“他对林彪同志、贺龙、聂荣臻、陈毅、刘伯承、叶剑英、徐向前和已故的罗荣桓同志,妄加议论、攻击和诽谤。”

第四、品质恶劣,投机取巧,坚持剥削阶级立场,资产阶级个人主义登峰造极。

第五、公开向党伸手,逼迫林彪同志“让贤”让权,进行篡军反党的阴谋活动。“罗瑞卿迫不及待地要林彪同志交位“让贤”。当时他跑到林彪同志处,在谈干部问题时,借题发挥,指桑骂槐地说:“病号嘛!就是养病,还管什么事!病号,让贤,不要干扰!”他走出房门外在走廊里还叫嚷说:“不要挡路”。”
\end{maonote}
