
\title{为争取千百万群众进入抗日民族统一战线而斗争}
\date{一九三七年五月七日在苏区党代表大会上的结论}
\thanks{毛泽东这篇文章在官方毛选中题目改成为《中国共产党在抗日时期的任务》。这篇文章后来收入官方毛选时只作了个别的修改,但是修改之后,在政治口气上同原文仍有分别,特别是在《革命前途问题》的那一节。}
\maketitle


同志们!对于我的报告——抗日民族统一战线在目前阶段的任务,经这几天的讨论之后,除了个别同志提出了不同意见之外,都是一致同意的。但是这些不同意见,颇带重要性,因此我的结论,首先在答复他们的问题,然后说到一些其它问题。

\section{一 和平问题}

我们党为国内和平而斗争,差不多两年的时间了。国民党三中全会后,我们说和平已经取得,“争取和平”阶段已经过去,新的任务是“巩固和平”,并指出这是同“争取民主”相关联的——从争取民主去巩固和平。我们的这种意见,按照几个同志的说法却不能成立,他们的结论必是相反的,或者是动摇于两者之间的。因为他们说
:“日本后退了,南京更动摇了,民族矛盾下降,国内矛盾上升。”根据这种估计,当然无所谓新阶段与新任务,情况回到旧阶段,或者还不如。这种意见,我以为是不对的。

我们说和平取得,并不是说和平巩固了,相反,我们说他是不巩固的,和平取得与和平巩固是两件事。历史暂时的走回头路是可能的,和平发生波折是可能的,原因就在于日本帝国主义与汉奸亲日派之存在。然而西安事变后和平是事实,这种情况是由多方面促成的(日本进攻的基本方针,苏联及英美法之赞助和平,中国人民的逼迫,共产党在西安事变中的和平方针及取消两个政权对立的政策,资产阶级的分化,国民党的分化等等),不是蒋介石一个人所能决定与推翻的。要推翻和平必须同多方面势力作战,并且必须同日本帝国主义与亲日派靠拢,才能成功。没有问题,日本同亲日派还在企图中国的内战,和平没有巩固,正是因为这一点。在这种情况下,我们的结论不是回到“停止内战”或“争取和平”的口号与阶段去,而是前进一步,指出“争取民主”的口号与阶段,只有这样才能巩固和平,也只有这样才能实现抗战。为什么提出“巩固和平”“争取民主”与“实现抗战”这样三位一体的口号,为的是把我们的革命车轮推进一步,为的是情况己经允许我们进一步了。如果否认新阶级与新任务,否认国民党的“开始转变”,并且逻辑的结论也将不得不否认一年半以来一切为争取和平而斗争的各派势力之努力的成绩,那么只是把自己停顿在旧位置,一步也没有前进。

为什么作出这种不妥当的估计呢?原因在他们不但从根本之点(日本后退、南京更动摇,民族矛盾下降,国内矛盾上升)出发,而且也从许多局部与一时的现象(佐藤外交,苏州审判,压制罢工,东北军东调,杨虎城出洋等等)出发,并把二者联贯起来,形成一幅暗淡的画图。我们说国民党已经开始转变,但我们同时即说国民党并没有彻底转变。国民党的十年反动政策,要他彻底转变而不用我们同人民新的更多与更大的努力,这是不能设想的事情。不少号称“左倾”的人们,平日痛骂国民党,在西安事变中主张杀蒋与“打出潼关去”,及至和平刚刚实现又发现苏州审判等事,就用惊诧的口气发问道
:“为什么蒋介石又这样干?”这些人们须知:共产党员同蒋介石都不是神仙,且都不是孤立的个人,而是处于一个党派,一个阶级,一个民族里头的分子。共产党有本领把革命逐步的推向前进。但没有本领把全国的坏事在一个早晨去掉干净。蒋介石或国民党已经开始看见了他们的转变,但没有全国人民的更大努力,也决不会在一个早晨把他们十年的污浊洗掉得干净。我们说运动的方向是向着和平民主与抗战,但不是说不经努力能够把内战独裁与不抵抗的旧毒去掉干净。旧毒、污浊,革命进程中的某些波折,以及可能的回头路,只有斗争与努力才能够克服,而且需要长期的斗争与努力。

“他们是一心要破坏我们。”对的,他们总是在企图破坏我们,我完全承认这种估计的正确,不估计这一点就等于睡觉。但问题在破坏的方式是否有了改变?我以为是有了改变的,从战争与屠杀政策改变到改良与欺骗政策,从硬的政策改变到软的政策,从消灭政策改变到争取政策,从军事政策改变到政治政策。为甚么有这种改变?资产阶级处在日本帝国主义面前不得不向无产阶级找同盟军,也和我们向资产阶级找同盟军一样,观察问题应从这一点出发。国际上,法苏世仇变为盟友,同此道理。我们的任务,亦是从军事的变到政治的。我们不需要阴谋鬼计,我们的目的在团结资产阶级及国民党一切同情抗日分子,共同战胜日本帝国主义。

\section{二 民主问题}

“强调民主是错误的,仅仅应该强调抗日,没有抗日的直接行动,就不能有民主运动,多数人只要抗日不要民主,再来一个十二月九号才是对的。”让我首先发出一点问题:能够在过去阶段中(十二月九号到三中全会)说,多数人只要抗日不要和平吗?过去强调和平是错了吗?没有抗日的直接行动就不能有和平运动吗(西安事变与三中全会正在绥战结束之后,现在也还没有绥战或十二月九号)?谁人不知为抗日而要和平,无和平不能抗日,和平是抗日的条件。前一阶段一切直接间接的抗日行动(从十二月九号起到三中全会止)都围绕着争取和平,和平是前一阶段中心一环,是抗日运动在前一阶段中的最本质的东西。

对于抗日任务,民主也是新阶段中最本质的东西,为民主即是为抗日。抗日与民主互为条件,同抗日与和平、民主与和平、互为条件一样。民主是抗日的保证,抗日能给与民主运动发展的有利条件。

新阶段中,我们希望有也将会有许多直接的间接的反日斗争,这些将推动对日抗战,也大有助于民主运动。然而历史给与我们的革命任务,中心的本质的东西是争取民主。“民主民主”是错的吗?我以为是不错的。

“日本退后了,英日向着平衡,南京更动摇了。”这是一种不知历史发展规律而发生的不适当的忧愁。日本如因国内革命而根本后退,这是有助于中国革命的,是我们所希望的,是世界侵略战线崩溃的开始,为什么还忧愁?然而暂时还不是,佐藤外交是大战的准备,大战在我们面前。英国的动摇政策只能向着无结果,这是英国与侵略国的不同利害决定了的。南京如果是长期动摇,便变为全国人民之敌,也为南京的利益所不许。一时的后退现象,不能代替总的历史规律。因此不能否认新阶段,也不能否认民主任务的提出。况且无论什么情况,民主的口号都能适应,民主对于中国人是缺乏而不是多余,这是人人明白的。何况实际情况已经表现,指出新阶段与提出民主任务,是向抗战接近一步的东西。时局已经前进了,不要把他拉向后退。

“为什么强调国民大会?”因为他是可能牵涉到全部生活的东西,因为他是从独裁到民主的桥梁;因为他带着国防性,因为他是合法的。收复冀东察北,反对走私,反对经济提携等等,如像同志们所提出的,都是很对的,但这丝毫也不与民主任务及国民大会相矛盾,二者正是互相完成的,但中心的东西是国民大会与人民自由。

日常的反日斗争与人民生活斗争,要同民主运动相配合,这是完全对的,也没有任何争论的。但目前阶段里中心同本质的东西,是民主与自由。

\section{三 革命前途问题}

有几个同志发出了这个问题,我的答复只能是简单的。

两篇文章,上篇与下篇,只有上篇做好,下篇才能做好。坚决的领导民主革命,是争取社会主义胜利的条件,我们是为着社会主义而斗争,这是与任何革命的三民主义者不相同的。今日的努力是朝着明日的大目标的,失掉这个大目标,就不是共产党员了。然而放松今日的努力,也就不是共产党员。

我们是革命转变论者,主张民主革命转变到社会主义方向去。民主革命中将有几个发展阶段,都在民主共和国口号下面,而不在苏维埃口号下面。从资产阶级占优势到无产阶级占优势,这是一个斗争的长过程,争取领导权的过程。依靠着共产党对无产阶级觉悟与组织程度的提高,对农民,对小资产阶级觉悟与组织程度的提高。

无产阶级坚固的同盟者是农民,其次是小资产阶级。同我们争领导权的是资产阶级。

对资产阶级的动摇与不彻底性的克服,依靠群众的力量与正确的政策,否则资产阶级将反过来克服无产阶级。

健全的转变(不流血的)是我们所希望的,我们应该力争这一着,结果将看群众的力量如何而定。

我们是革命转变论者,不是托洛茨基主义的不断革命论者,也不是半托洛茨基主义的立三主义。我们主张经过民主共和国的一切必要的阶段,到达于社会主义。我们反对尾巴主义,但又反对冒险主义与急性病。

因为资产阶级的暂时性而不要资产阶级,指联合资产阶级的革命派(在半殖民地)为投降主义,这是托洛茨基主义的说法,我们是不能同意的。今天的联合资产阶级革命派,正是走向社会主义的必经的桥梁。

\section{四 干部问题}

担负着指导伟大的革命任务,要有伟大的党,要有伟大的领袖与干部。在一个四万万五千万人的中国里面,进行历史空前的大革命,如果领导者是一个狭隘的小团体是不行的,党内仅有一般委琐不识大体,没有远见,没有能力的领袖与干部也是不行的。中国共产党早就是一个大政党,经过反动时期的损失它依然是一个大政党,它有了许多好的领袖与干部,但是还不够。我们党的组织要向全国发展,要自觉的造就成万数的干部要有数百个最好的群众领袖。这些干部同领袖懂得马克思列宁主义,有政治远见,有工作能力,富于牺牲精神,能独立解决问题,在困难中不动摇,并拿出全部忠心为民族为阶级为党而工作。党的路线依靠着这些人而联系到党员与群众,依靠着这些人对于群众的坚强领导而达到打倒敌人之目的。这些人不要自私自利,不要个人英雄主义与风头主义,不要懒惰与消极性,不要自高自大的宗派主义。他们是大公无私的民族的与阶级的英雄,这就是共产党员,党的干部、党的领袖应该有的性格与作风。我们死去的若干万数的党员,若干千数与数十个最好的领袖遗留给我们的精神,也就是这些东西。我们应该学习这些东西,把自己改造得更好一些,把自己提高到更高的革命水平,是无疑地需要的。但是还不够,还要作为一种任务,向全党与全国找寻许多新的干部与领袖。我们的革命依靠干部,好像斯大林同志所说的话,“干部是决定一切的。”

\section{五 党内民主问题}

要达到这种目的,党内的民主是必要的。要党有力量,依然实行党的民主集中制去发动全党的积极性。在反动与内战时期,集中制表现的多一些。在新时期,集中制应该密切联系于民主制。从民主制的实行,发挥全党的积极性。从发挥全党的积极性,创造出大批的干部与领袖,肃清派别观念的残余,团结全党像钢铁一样。

\section{六 大会的团结与全党的团结}

大会中政治问题上的不同意见,经过说明已经归于一致了,过去中央路线与个别同志领导的退却路线之间的分歧,也己经没有了,表示了我们的党已经团结得很坚固的。这种团结是当前民族与民主革命的最重要的基础,因为只有经过共产党的团结,才能达到全阶级与全民族的团结,只有经过全阶级全民族的团结,才能战胜敌人,实现民族与民主革命的任务。

\section{七 为争取千百万群众进入抗日民族统一战线而斗争}

我们的正确的政治方针与坚固的团结,唯一的是向着争取千百万群众进入抗日民族统一战线这个目的。无产阶级、农民、小资产阶级的广大群众,有待于我们宣传、鼓动、与组织的工作。资产阶级革命派的进入和我们的同盟,也还待我们的进一步工作。把党的方针变为群众的方针,还须要我们长期坚持的、百折不挠的、艰苦卓绝的、耐心而不怕麻烦的努力。没有这一努力是一切都不成功的。抗日民族统一战线的组成、巩固、及其任务的完成,民主共和国在中国的实现,丝毫也不能离开这一争取群众的努力。如果经过这种努力而得到千百万群众在我们领导之下的话,那我们的全部革命任务就能够迅速的实现。日本帝国主义甚么也不怕我们,但他独怕我们的这种努力。我们的努力将确定地打倒日本帝国主义,并实现全部的民族解放与社会解放。
