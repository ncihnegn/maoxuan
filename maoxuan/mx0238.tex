
\title{为皖南事变发表的命令和谈话}
\date{一九四一年一月二十日}
\maketitle


\section{中国共产党中央革命军事委员会命令}

国民革命军新编第四军抗战有功,驰名中外。军长叶挺,领导抗敌,卓著勋劳;此次奉令北移,突被亲日派阴谋袭击,力竭负伤,陷身囹圄。迭据该军第一支队长陈毅、参谋长张云逸等电陈皖南事变经过,愤慨之余,殊深轸念。除对亲日派破坏抗日、袭击人民军队、发动内战之滔天罪行,另有处置外,兹特任命陈毅为国民革命军新编第四军代理军长,张云逸为副军长,刘少奇为政治委员,赖传珠为参谋长,邓子恢为政治部主任。着陈代军长等悉心整饬该军,团结内部,协和军民,实行三民主义,遵循《总理遗嘱》,巩固并扩大抗日民族统一战线,为保卫民族国家、坚持抗战到底、防止亲日派袭击而奋斗。

\section{中国共产党中央革命军事委员会发言人对新华社记者的谈话}

此次皖南反共事变,酝酿已久。目前的发展,不过是全国性突然事变的开端而已。自日寇和德意订立三国同盟\mnote{1}之后,为急谋解决中日战争,遂积极努力,策动中国内部的变化。其目的,在借中国人的手,镇压中国的抗日运动,巩固日本南进的后方,以便放手南进,配合希特勒进攻英国的行动。中国亲日派首要分子,早已潜伏在国民党党政军各机关中,为数颇多,日夕煽诱。至去年年底,其全部计划乃准备完成。袭击皖南新四军部队和发布一月十七日的反动命令,不过是此种计划表面化的开端。最重大的事变,将在今后逐步演出。日寇和亲日派的整个计划为何?即是:

(一)用何应钦、白崇禧名义,发布致朱彭叶项的“皓”“齐”两电\mnote{2},以动员舆论;

(二)在报纸上宣传军纪军令的重要性,以为发动内战的准备;

(三)消灭皖南的新四军;

(四)宣布新四军“叛变”,取消该军番号。以上诸项,均已实现。

(五)任命汤恩伯、李品仙、王仲廉、韩德勤等为华中各路“剿共”军司令官,以李宗仁为最高总司令,向新四军彭雪枫、张云逸、李先念诸部实行进攻,得手后,再向山东和苏北的八路军新四军进攻,而日军则加以密切的配合。这一步骤,已开始实行。

(六)寻找借口,宣布八路军“叛变”,取消八路军番号,通缉朱彭。这一步骤,目前正在准备中。

(七)取消重庆、西安、桂林等地的八路军办事处,逮捕周恩来、叶剑英、董必武、邓颖超诸人。这一步骤也正开始实施,桂林办事处已被取消。

(八)封闭《新华日报》\mnote{3};

(九)进攻陕甘宁边区,夺取延安;

(十)在重庆和各省大批逮捕抗日人士,镇压抗日运动;

(十一)破坏各省共产党的组织,大批逮捕共产党员;

(十二)日军从华中华南撤退,国民党政府宣布所谓“收复失地”,同时宣传实行所谓“荣誉和平”的必要性;

(十三)日军将原驻华中华南的兵力向华北增加,最残酷地进攻八路军,与国民党军队合作,全部消灭八路军新四军;

(十四)除一刻也不放松对于八路军新四军进攻之外,在各战场上的国民党军队和日军继续去年的休战状态,以便转到完全停战议和的局面;

(十五)国民党政府同日本订立和平条约,加入三国同盟。以上各步,正在积极准备推行中。

以上,就是日本和亲日派整个阴谋计划的大纲。中国共产党中央曾于前年七月七日的宣言上指出:投降是时局最大的危险,反共是投降的准备步骤。在去年七月七日的宣言中则说:“空前的投降危险和空前的抗战困难,已经到来了。”朱彭叶项在去年十一月《佳电》中更具体地指出:“国内一部分人士正在策动所谓新的反共高潮,企图为投降肃清道路。……欲以所谓中日联合‘剿共’,结束抗战局面。以内战代抗战,以投降代独立,以分裂代团结,以黑暗代光明。其事至险,其计至毒。道路相告,动魄惊心。时局危机,诚未有如今日之甚者。”故皖南事变及重庆军事委员会一月十七日的命令,不过是一系列事变的开始而已。特别是一月十七日的命令,包含着严重的政治意义。因为发令者敢于公开发此反革命命令,冒天下之大不韪,必已具有全面破裂和彻底投降的决心。盖中国软弱的大地主大资产阶级的政治代表们,没有后台老板,是一件小事也做不成的,何况如此惊天动地的大事?在目前的时机下,欲改变发令者此种决心似已甚难,非有全国人民的紧急努力和国际外交方面的重大压力,改变决心的事,恐怕是不可能的。故目前全国人民的紧急任务,在于以最大的警惕性,注视事变的发展,准备着对付任何黑暗的反动局面,绝对不能粗心大意。若问中国的前途如何,那是很明显的。日寇和亲日派的计划即使实现,我们中国共产党和中国人民,不但有责任,而且自问有能力,挺身出来收拾时局,决不让日寇和亲日派横行到底。时局不论如何黑暗,不论将来尚须经历何种艰难道路和在此道路上须付何等代价(皖南新四军部队就是代价的一部分),日寇和亲日派总是要失败的。其原因,则是:

(一)中国共产党已非一九二七年那样容易受人欺骗和容易受人摧毁。中国共产党已是一个屹然独立的大政党了。

(二)中国其它党派(包括国民党在内)的党员,懔于民族危亡的巨祸,必有很多不愿意投降和内战的。有些虽然一时受了蒙蔽,但时机一到,他们还有觉悟的可能。

(三)中国的军队也是一样。他们的反共,大多数是被迫的。

(四)全国人民的大多数,不愿当亡国奴。

(五)帝国主义战争现时已到发生大变化的前夜,一切依靠帝国主义过活的寄生虫,不论如何蠢动于一时,他们的后台总是靠不住的,一旦树倒猢狲散,全局就改观了。

(六)许多国家革命的爆发,只是时间问题,这些国家的革命和中国革命必然互相配合,共同争取胜利。

(七)苏联是世界上第一个大力量,它是决然帮助中国抗战到底的。

因为上述种种原因,故我们还是希望那班玩火的人,不要过于冲昏头脑。我们正式警告他们说:放谨慎一点吧,这种火是不好玩的,仔细你们自己的脑袋。如果这班人能够冷静地想一想,他们就应该老老实实地并且很快地去做下列几件事:

第一、悬崖勒马,停止挑衅;

第二、取消一月十七日的反动命令,并宣布自己是完全错了;

第三、惩办皖南事变的祸首何应钦、顾祝同、上官云相三人;

第四、恢复叶挺自由,继续充当新四军军长;

第五、交还皖南新四军全部人枪;

第六、抚恤皖南新四军全部伤亡将士;

第七、撤退华中的“剿共”军;

第八、平毁西北的封锁线\mnote{4};

第九、释放全国一切被捕的爱国政治犯;

第十、废止一党专政,实行民主政治;

第十一、实行三民主义,服从《总理遗嘱》;

第十二、逮捕各亲日派首领,交付国法审判。

如能实行以上十二条,则事态自然平复,我们共产党和全国人民,必不过为已甚。否则,“吾恐季孙之忧,不在颛臾,而在萧墙之内”\mnote{5},反动派必然是搬起石头打他们自己的脚,那时我们就爱莫能助了。我们是珍重合作的,但必须他们也珍重合作。老实说,我们的让步是有限度的,我们让步的阶段已经完结了。他们已经杀了第一刀,这个伤痕是很深重的。他们如果还为前途着想,他们就应该自己出来医治这个伤痕。“亡羊补牢,犹未为晚。”这是他们自己性命交关的大问题,我们不得不尽最后的忠告。如若他们怙恶不悛,继续胡闹,那时,全国人民忍无可忍,把他们抛到茅厕里去,那就悔之无及了。关于新四军,中国共产党中央革命军事委员会已于一月二十日下了命令,任命陈毅为代理军长,张云逸为副军长,刘少奇为政治委员,赖传珠为参谋长,邓子恢为政治部主任。该军在华中及苏南一带者尚有九万余人,虽受日寇和反共军夹击,必能艰苦奋斗,尽忠民族国家到底。同时,它的兄弟部队八路军各部,决不坐视它陷于夹击,必能采取相当步骤,予以必要的援助,这是我可以率直地告诉他们的。至于重庆军委会发言人所说的那一篇,只好拿“自相矛盾”四个字批评它。既在重庆军委会的通令中说新四军“叛变”,又在发言人的谈话中说新四军的目的在于开到京、沪、杭三角地区创立根据地。就照他这样说吧,难道开到京、沪、杭三角地区算是“叛变”吗?愚蠢的重庆发言人没有想一想,究竟到那里去叛变谁呢?那里不是日本占领的地方吗?你们为什么不让它到那里去,要在皖南就消灭它呢?啊,是了,替日本帝国主义尽忠的人原来应该如此。于是七个师的聚歼计划出现了,于是一月十七日的命令发布了,于是叶挺交付审判了。但是我还要说重庆发言人是个蠢猪,他不打自招,向全国人民泄露了日本帝国主义的计划。


\begin{maonote}
\mnitem{1}指一九四〇年九月二十七日德、意、日在柏林订立三国军事同盟条约。
\mnitem{2}“皓”“齐”两电,是蒋介石以国民政府军事委员会参谋总长何应钦、副参谋总长白崇禧的名义,在一九四〇年十月十九日(皓)和十二月八日(齐)发出的两个电报。《皓电》对坚持敌后抗战的八路军、新四军大肆诬蔑,强迫命令黄河以南的八路军、新四军在一个月内撤到黄河以北。中共中央为顾全大局,挽救危亡,以第十八集团军总司令朱德、副总司令彭德怀和新四军军长叶挺、副军长项英的名义,于十一月九日(佳)电复何、白,除据实驳斥《皓电》的造谣外,同意将江南新四军部队移至长江以北。(关于《佳电》的内容,另见本卷\mxnote{关于打退第二次反共高潮的总结}{8}。)《齐电》是针对朱、彭、叶、项的《佳电》而发,为蒋介石发动第二次反共高潮进一步做舆论上的动员。
\mnitem{3}见本卷\mxnote{关于国际新形势对新华日报记者的谈话}{1}。
\mnitem{4}西北封锁线,指国民党反动派包围陕甘宁边区的封锁线。在皖南事变前夜,国民党在边区周围已经修筑了五道包括沟墙和碉堡的封锁线,西起甘肃、宁夏,南沿泾水,东迄黄河,绵亘数省。同时,包围边区的国民党军队也增加到二十余万人。
\mnitem{5}见《论语·季氏》。季孙,鲁国大夫。颛臾,春秋时小国。萧墙是古代宫室内当门的小墙。季孙将伐颛臾,孔子以为季孙之忧不在外而在内。
\end{maonote}
