
\title{论反对官僚主义}
\date{一九六三年五月二十九日}
\thanks{这是毛泽东同志为中共中央起草的党内指示。}
\maketitle


我们这些机关高高在上,官僚主义是容易犯的弊病,有了官僚主义,必然对上闹分裂主义。比如“跃进号”\mnote{1}抓了才清楚的。下边也闹地方主义,根子都是官僚主义。前年下放权利那么多,文件是我起草的,造成了分散主义。有人说要反对,顶不住,问题还是官僚主义。

官僚主义是一个剥削阶级遗留下来的东西。党外部长和我们一道,也希望借重一下你们的归劝。

官僚主义,思想上表现和个人主义、分散主义、本位主义、自由主义、命令主义、事务主义、组织上的宗派主义、自由主义相结合的,因此官僚主义也必然联系到这些主义。总之,要集中的反对剥削阶级的思想作风。三月一日,中央五反指示中说:“官僚主义在抬头”。我看带有普遍性。

我尝归纳官僚主义二十种表现:

一、高高在上,孤陋寡闻。不了解下情,不作调查研究,因而脱离实际,脱离领导。不作政治思想工作,不抓具体政策,上脱离领导,下脱离实际,一旦发号施令,必然祸国殃民。这是脱离领导、脱离群众的官僚主义。

二、狂妄自大,骄傲自满,空谈政策,不抓业务,主观片面,不听人言,蛮横专断,强迫命令,不顾实际,盲目指挥。这是强迫命令式的官僚主义。

三、从早到晚,忙忙碌碌,一年到头,辛辛苦苦,但是作事不调查,对人不考察,发言无准备,工作无计划。这是无头无脑,迷失方向的官僚主义。

四、官气熏天,唯我独尊,不可亲近,望而生畏,对干部颐指气使,作风粗暴,动辄骂人。这是官老爷式的官僚主义。

五、不学无术,耻于下问,浮夸谎报,弄虚作暇,欺上瞒下,文过饰非,功则归己,过则归人。这是不老实的官僚主义。

六、不学政治,不钻业务,遇事推委,怕负责任,办事拖拉,长期不决,工作上讨价还价,政治上麻木不仁。这是不负责任的官僚主义。

七、遇事敷衍,得过且过,与人为争,老于事故,上捧下拉,两面俱圆,八面玲珑。这是作官混饭吃的官僚主义。

八、政治学不成,业务钻不进,人云亦云,语气无味,尸位素餐。领导无方,滥竿充数。这是满预无能的官僚主义。

九、糊糊涂涂,混混沌沌,人云亦云,得过且过,饱食终日,无所用心。这是糊涂无用的官僚主义。

十、送文件不看就批,批错了不承认,文件听别人读,别人读他睡着了,心中无数,不和人商量事情推来推去不负责,对下不懂装懂,指手划脚,对同级貌合神离,同床异梦。这是懒汉误事的官僚主义。

十一、机构庞大,人事庞杂,层次重迭,浪费资产,人多事乱,遇事团团转,不务正业,

人多事少,工作效率低。这是机关式的官僚主义。

十二、指示多不看,报告多不批,会议多不传,报表多不用,往来多不谈。这叫“五多”的官僚主义。

十三、图享受,好伸手,走“后门”,怕艰苦,一人得道鸡犬升天,一人作官全家享福,内外不一请客送礼。这是特殊化的官僚主义。

十四、官越作越大,脾气越来越坏,房子越来越大,陈设越来越好,生活要求越高,供应越多,分配东西越多,价钱越低。这是摆官架子的官僚主义。

十五、自私自利,假公济私,以私作公,监守自盗,知法犯法,多吃多占,不退不还。这是自私自利的官僚主义。

十六、争名夺利,向党伸手,对待遇斤斤计较,对工作挑肥拣瘦,对同志拉拉扯扯,对群众漠不关心。这是争权夺利的官僚主义。

十七、多头领导,互不团结,政出多门,工作散乱,上下隔离,互相排挤,既不集中,又不民主。这是闹不团结的官僚主义。

十八、目无组织,任用私人,结党营私,互相包庇,个人利益、派别利益高于一切,损大公肥小私。这是闹宗派的官僚主义。

十九、革命意志衰颓,政治生活蜕化,靠老资格吃饭,摆官架子,好逸恶劳,游山玩水,既不用脑,又不动手,不关心国家和人民利益。这是蜕化的官僚主义。

二十、助长歪风邪气,纵容坏人坏事,打击报复,压制民主,欺压群众,包庇坏人,敌我不分,作奸犯科。这是助长歪风邪气的官僚主义。

总之,使干部脱离实际,脱离群众,漠视群众利益,使党的路线政策受损失。不作为普通劳动者,不同群众同甘共苦,政治上空谈,不老实,不负责任,不能、无用,埋头于事务主义,搞特殊化,自私自利,闹不团结,搞宗派,最后发展蜕化变质。

官僚主义的思想根源、阶级社会根源、历史根源、思想根源,是剥削阶级的思想作风,既有资产阶级的个人主义、实用主义,也有封建的家长制。(红楼梦四大家族,农奴主四十人,官僚占三分之二的人。)

阶级社会根源;新的资产阶级,老的资产阶级,还有城乡封建势力。在国际上有资本主义包围,而且帝国主义、修正主义联合起来了。

历史根源:我们的革命打碎了旧的国家机器,建立了新的国家机器,但旧的统治势力,传统影响,旧人员包下来,政策是对的,但带来了副作用,一九五一年“三反”重点是反贪污,一九五七年重点是反右,去年主要是批判了分散主义,所以历年来没有把官僚主义当成重点来搞。现在滋生官僚主义的土壤是肥沃的,也是修正主义、教条主义的土壤。

\begin{maonote}
\mnitem{1}跃进号事件,一九五八年十一月二十七日,由大连造船厂建造的新中国第一艘万吨远洋货轮下水,交通部将这艘远洋货轮命名为“跃进号”。这艘由苏联设计的万吨巨轮的庞大船体,是在一九五八年年国庆节前夕正式铺上船台开工建造的,到十一月二十七日建成下水只用了五十八天时间,这样的速度在当时是很快的,甚至超过了造船速度最快的日本建造万吨级货轮平均三个月的船台周期。

一九六三年四月三十日,“跃进”轮载着一万三千四百吨吨玉米、矿产品和杂货,于下午三时五十八分从青岛启航,按“青岛——上海长江口(不进港)——日本门司”航线迂回曲折行驶。

五月一日十三时五十五分,“跃进”轮报告了在韩国济州岛南八十海里左右、近日本海的苏岩礁海域的船位,随即又发出“我轮受击、损坏严重”的密电。十四时十分,该轮再次发出国际通用的SOS明码求救电讯,接着该轮即在海面上消失,完全断绝了与国内的联系。

“跃进号”上的五十九名船员分乘三艘救生艇在海上漂流时,被日本渔船“壹岐丸号”救起,船上二副为推卸责任,说是被鱼雷击中,甚至说“看见一只潜艇在沉船区域露出水面,有三个美国兵样子的人在舷楼哈哈大笑”。事后的调查证明,这纯属子虚乌有,显然是为了推卸责任而杜撰。五月一日夜间,日本的全亚细亚广播电台在新闻节目里向全世界披露:中国国产的第一艘万吨级远洋货轮“跃进”号在来日本途中因腹部命中三发鱼雷而沉没,这一消息犹如原子弹爆炸一样,迅速传遍全球,立即掀起轩然大波,世界各国均感震惊。须知,在和平时期攻击商船,意味着对该国的宣战,将可能引起局部乃至世界性的战争,况且当时正处于冷战状态的国际形势,更具有导火线式的危险性。同时,攻击商船的行为也为国际惯例所不准,将受全世界舆论的强烈谴责。所以,一下子引起各国舆论的纷纭传说和猜测,各国政府立即作出不同的反响,国际形势顿时微妙地紧张起来。

美国政府迅速发表声明:“五月一日美国潜艇没有到过苏岩礁海域,没有对中国货船发动过攻击。”台湾国民党当局也发表声明,说他们的军舰从来没有到过“跃进号”失事的海区。韩国和苏联也发表了类似的声明。世界各国都在密切关注着事态的进一步发展。

事件惊动了毛泽东主席和周恩来总理,周总理亲自组织人调查,调查查明,当时虽然有海图标明航线上有苏岩礁,但是由于船员计算船位失误,还是撞上了礁石并沉没。六月三日,新华社奉命发表声明:“经过周密调查,已经证实“跃进”号是因触礁而沉没的。”
\end{maonote}
