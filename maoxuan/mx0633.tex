
\title{坚决制止重刮“共产风”等违法乱纪行为}
\date{一九六〇年三月二十三日}
\thanks{这是毛泽东同志为山东六级干部会议秘书处编印材料的批语。}
\maketitle


\mxname{各省、市、自治区党委,中央一级各部委、各党组:}

此件\mnote{1}请各同志阅读,并请转发到县级党委。山东发现的问题,肯定各省、各市、各自治区都有,不过大同小异而已。问题严重,不处理不行。在一些县、社中,去年三月郑州决议\mnote{2}忘记了,去年四月上海会议十八个问题的规定\mnote{3}也忘记了,“共产风”、浮夸风、命令风又都刮起来了。一些公社工作人员很狂妄,毫无纪律观点,敢于不得上级批准,一平二调\mnote{4}。另外还有三风:贪污、浪费、官僚主义,又大发作,危害人民。什么叫做价值法则,等价交换,他们全不理会。所有以上这些,都是公社一级干的。范围多大,不很大,也不很小。是否有十分之一的社这样胡闹,要查清楚。中央相信,大多数公社是谨慎、公正、守纪律的,胡闹的只是少数。这个少数公社的所有工作人员,也不都是胡闹的,胡闹的只有其中一部分。对于这些人,应当分别情况,适当处理。教育为主,惩办为辅。对于那些最胡闹的,坚决撤掉,换上新人。平调方面的处理,一定要算账,全部退还,不许不退。对于大贪污犯,一定要法办。一些县委为什么没有注意这些问题呢?他们严重地丧失了职守,以后务要注意改正。对于少数县委实在不行的,也要坚决撤掉,换上新人。同志们须知,这是一个长期存在的问题,是一个客观存在。出现这些坏事,是必然不可避免的,是旧社会坏习惯的残余,要有长期教育工作,才能克服。因此,年年要整风,一年要开两次六级干部大会。全国形势大好,好人好事肯定占十分之九以上。这些好人好事,应该受到表扬。对于犯错误而不严重、自己又愿意改正的同志,应当采用教育方法,帮助他们改正错误,照样做工作。我们主张坚决撤掉或法办的,是指那些错误极严重、民愤极大的人们。在工作能力上实在不行、无法继续下去的人们,也必须坚决撤换。

\begin{maonote}
\mnitem{1}指山东省六级干部会议秘书处编印的《会议情况》第二期。这期简报反映的主要问题有:一、关于过渡问题。主要存在以下几种情况,1.有的公社干部存在急于过渡的苗头,有的打算明年“走到社有制”,有的打算“秋后搞过渡”;2.穷队盼过渡,富队怕过渡,生产不积极;3.对过渡的根据和一些问题的政策界限认识不清。二、关于发展社有经济问题。对于一平二调,社干部与队干部有不同看法。例如队干部批评公社无偿上调物资,公社干部却有抵触情绪顾虑算账问题。三、干部作风问题。有些县存在虚报浮夸现象;有的地方用大量生产资金和物料搞盖大礼堂、办公大楼和宾馆等非生产性建设;干部中存在吃喝浪费和贪污行为。
\mnitem{2}指一九五九年二月二十七日至三月五日在郑州召开的中共中央政治局扩大会议形成并下发的《郑州会议记录》。
\mnitem{3}指一九五九年三月二十五日至四月一日在上海召开的中共中央政治局扩大会议形成的会议纪要《关于人民公社的十八个问题》。这个纪要于同年四月二日至五日在上海举行的中共八届七中全会上原则通过。纪要对人民公社管理体制问题作了若干原则规定,进一步明确人民公社的所有制基本上是生产队所有制;并且将《郑州会议记录》中关于人民公社化过程中平调财物的旧账一般不算的规定,改为旧账一般要算,凡是县社调用生产队的劳力、资财,或者社队调用社员的私人财物,都要进行清理,如数归还,或者折价补偿。
\mnitem{4}一平二调是人民公社化运动中“共产风”的主要表现,即:在公社范围内实行贫富拉平平均分配;县、社两级无偿调走生产队(包括社员个人)的某些财物。三收款,指银行将过去发放给农村的贷款统统收回。
\end{maonote}
