
\title{和中央社、扫荡报、新民报\mnote{1}三记者的谈话}
\date{一九三九年九月十六日}
\maketitle


\mxsay{记者问:}有几个问题请教。今天在《新中华报》\mnote{2}上看了毛先生九月一日的谈话,有些问题已经说到了,有些尚请毛先生补充。问题分三部分,就是写在纸上的,请逐一赐教。

\mxsay{毛答:}可以根据先生们的问题表,分别来讲。

先生们提到抗战的相持阶段是否到来的问题。我以为,相持阶段是有条件地到来了。就是说,在国际新形势之下,在日本更加困难和中国绝不妥协的条件之下,可以说已经到来了。这里并不否认敌人还可能有比较大的战役进攻,例如进攻北海、长沙,甚至进攻西安,都是可能的。说敌人的大规模战略进攻和我们的战略退却在一定条件下基本上已经停止,并不是说一切进攻的可能和一切退却的可能都没有了。至于新阶段的具体内容,就是准备反攻,一切都可以包括在这一概念之中。这就是说,中国要在相持阶段中准备一切力量,以备将来的反攻。说准备反攻,并不是立即反攻,条件不够是不能反攻的。而且这讲的是战略的反攻,不是战役的反攻。战役上的反攻,例如对付敌人在晋东南的军事“扫荡”,我们把他打退,这样的战役反攻不但会有,而且是必不可少的。但是战略上的大举反攻时期,现在还没有到,现在是对于这种大举反攻作积极准备的时期。在这个时期内,还要打退正面敌人一些可能的战役进攻。

如果把新阶段的任务分别来讲,那末,在敌人后方,一定要坚持游击战争,粉碎敌人的“扫荡”,破坏敌人的经济侵略;在正面,一定要巩固军事防御,打退敌人可能的战役进攻;在大后方\mnote{3},主要的是积极改革政治。这许多,都是准备反攻的具体内容。

改革国内政治之所以非常重要,是因为敌人在目前,主要的是政治进攻,我们就要特别加强政治抵抗。这就是说,民主政治的问题,应当快点解决,才能加强政治上的抵抗力,才能准备军事力量。中国抗战主要地依靠自力更生。如果过去也讲自力更生,那末,在新的国际环境下,自力更生就更加重要。自力更生的主要内容,就是民主政治。

\mxsay{问:}刚才毛先生说,为了自力更生达到抗战胜利,民主政治是必要的,那末,在现在的环境下,用什么方法来实现这个制度?

\mxsay{答:}军政、训政、宪政三个时期的划分\mnote{4},原是孙中山先生说的。但孙先生在逝世前的《北上宣言》\mnote{5}里,就没有讲三个时期了,那里讲到中国要立即召开国民会议。可见孙先生的主张,在他自己,早就依据情势,有了变动。现在在抗战这种严重的局面之下,要避免亡国惨祸,并把敌人打出去,必须快些召集国民大会,实行民主政治。关于这个问题,有各种不同的议论。有些人说:老百姓没有知识,不能实行民主政治。这是不对的。在抗战中间,老百姓进步甚快,加上有领导,有方针,一定可以实行民主政治。例如在华北,已经实行了民主政治。在那里,区长、乡长、保甲长,多是民选的。县长,有些也是民选的了,许多先进的人物和有为的青年,被选出来当县长了。这样的问题,应该提出让大家讨论。

先生们提出的第二部分问题里,有关于所谓“限制异党”的问题,就是说,关于各地磨擦的问题。先生们关心这件事是很对的。关于这件事,近来情况虽然比较好一点,但是根本上没有什么变化。

\mxsay{问:}共产党对这个问题的态度,曾向中央政府表示过没有?

\mxsay{答:}我们已经提出抗议。

\mxsay{问:}用什么方式提出的?

\mxsay{答:}还是在七月间,我们党的代表周恩来同志,已经写信给蒋委员长。八月一日,延安各界又打了电报给蒋委员长和国民政府,要求取消那个秘密流行成为各地磨擦根源的所谓《限制异党活动办法》\mnote{6}。

\mxsay{问:}中央政府有无答复?

\mxsay{答:}没有答复。听说这个东西,国民党里面也有一些人不赞成。你们知道,共同抗日的军队叫做友军,不叫做“异军”,那末,共同抗日的党派就是友党,不是“异党”。抗战中间有许多党派,党派的力量有大小,但是大家同在抗战,完全应该互相团结,而决不应该互相“限制”。什么是异党?日本走狗汪精卫\mnote{7}的汉奸党是异党,因为它和抗日党派在政治上没有丝毫共同之点,这样的党,就应该限制。国民党、共产党,在政治上是有共同之点的,这就是抗日。所以现在是如何集中全力反日防日和反汪防汪的问题,而不是集中全力反共防共的问题。口号只能是这样提。现在汪精卫有三个口号:反蒋、反共、亲日。汪精卫是国共两党和全国人民的共同敌人。共产党却不是国民党的敌人,国民党也不是共产党的敌人,不应该互相反对,互相“限制”,而应该互相团结,互相协助。我们的口号一定要和汪精卫的口号有区别,一定要和汪精卫的口号对立起来,而决不能和他相混同。他要反蒋,我们就要拥蒋;他要反共,我们就要联共;他要亲日,我们就要抗日。凡是敌人反对的,我们就要拥护;凡是敌人拥护的,我们就要反对。现在许多人的文章上常常有一句话,说是“无使亲痛仇快”。这句话出于东汉时刘秀的一位将军叫朱浮的写给渔阳太守彭宠的一封信,那信上说:“凡举事无为亲厚者所痛,而为见仇者所快。”朱浮这句话提出了一个明确的政治原则,我们千万不可忘记。

先生们的问题表中还问到共产党对待所谓磨擦的态度。我可以率直地告诉你们,我们根本反对抗日党派之间那种互相对消力量的磨擦。但是,任何方面的横逆如果一定要来,如果欺人太甚,如果实行压迫,那末,共产党就必须用严正的态度对待之。这态度就是:人不犯我,我不犯人;人若犯我,我必犯人。但我们是站在严格的自卫立场上的,任何共产党员不许超过自卫原则。

\mxsay{问:}华北的磨擦问题怎样?

\mxsay{答:}那里的张荫梧、秦启荣\mnote{8},是两位磨擦专家。张荫梧在河北,秦启荣在山东,简直是无法无天,和汉奸的行为很少区别。他们打敌人的时候少,打八路军的时候多。有许多铁的证据,如像张荫梧给其部下进攻八路军的命令等,我们已送给蒋委员长了。

\mxsay{问:}新四军方面有无磨擦?

\mxsay{答:}也是有的,平江惨案\mnote{9}就是惊动全国的大事件。

\mxsay{问:}有些人说,统一战线是重要的,但是按照统一,边区政府就应该取消。关于这,先生以为如何?

\mxsay{答:}各种胡言乱语到处都有,如所谓取消边区,即是一例。陕甘宁边区是民主的抗日根据地,是全国政治上最进步的区域,取消的理由何在?何况边区是蒋委员长早已承认了的,国民政府行政院也早在民国二十六年冬天就正式通过了。中国确实需要统一,但是应该统一于抗战,统一于团结,统一于进步。如果向相反的方面统一,那中国就会亡国。

\mxsay{问:}由于对于统一的了解不同,国共是否有分裂的可能?

\mxsay{答:}如果只说到可能性的话,那末,团结和分裂两种可能性都有,要看国共两党的态度如何,尤其要看全国人民的态度如何来决定。我们共产党方面,关于合作的方针,早经讲过,我们不但希望长期合作,而且努力争取这种合作。听说蒋委员长在国民党五中全会中也说过,国内问题不能用武力来解决。大敌当前,国共两党又都有了过去的经验,大家一定要长期合作,一定要避免分裂。但是要给长期合作找到政治保证,分裂的可能性才能彻底避免,这就是坚持抗战到底和实行民主政治。如果能这样做,那末,就能继续团结而避免分裂,这是要靠两党和全国人民共同努力的,也是一定要这样努力的。“坚持抗战、反对投降”,“坚持团结、反对分裂”,“坚持进步、反对倒退”,这是我们党在今年的《七七宣言》里提出来的三大政治口号。我们认为只有这样做,中国才能避免亡国,并把敌人打出去;除此没有第二条路好走。


\begin{maonote}
\mnitem{1}中央社是国民党的中央通讯社。《扫荡报》是国民党政府军事系统的报纸。《新民报》是代表民族资产阶级的一种报纸。
\mnitem{2}《新中华报》的前身是中华苏维埃共和国中央政府机关报《红色中华》,一九三七年一月二十九日改为此名,在延安出版。同年九月九日改为陕甘宁边区政府的机关报。一九三九年二月七日起改组为中国共产党中央委员会的机关报。一九四一年五月十五日终刊。
\mnitem{3}指国民党统治区。抗日战争时期,人们习惯称未被日本侵略军占领而在国民党统治下的中国西南部和西北部的广大地区为“大后方”。
\mnitem{4}孙中山在《建国大纲》中,曾经将“建国”程序划分为“军政”、“训政”、“宪政”三个时期。以蒋介石为首的国民党反动派,长期利用“军政”、“训政”的说法,作为实行反革命专政和剥夺人民一切自由权利的借口。
\mnitem{5}一九二四年十月,直系军阀在第二次直奉战争中失败,它控制的北京中央政权垮台,冯玉祥等北方实力派电请孙中山入京,共商国是。孙中山于十一月十三日应邀北上。在离开广州前,孙中山发表《北上宣言》,重申反对帝国主义和军阀的主张,号召召集国民会议。这个宣言受到全国人民的欢迎。
\mnitem{6}见本卷\mxnote{必须制裁反动派}{5}。
\mnitem{7}见本书第一卷\mxnote{论反对日本帝国主义的策略}{31}。
\mnitem{8}见本卷\mxnote{团结一切抗日力量,反对反共顽固派}{5}和\mxnotex{6}。
\mnitem{9}见本卷\mxnote{必须制裁反动派}{1}。
\end{maonote}
