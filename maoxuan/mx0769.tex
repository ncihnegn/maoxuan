
\title{接见基辛格时的谈话\mnote{1}}
\date{一九七三年二月、十一月}
\thanks{这是毛泽东同志在接见美国国务卿基辛格时的谈话节选。}
\maketitle


\section*{(一)}

\mxsay{基辛格:}我看主席这次比上次好得多了。

\mxsay{毛泽东:}看起来是这样,实际上上帝给我下了请帖。(转头朝洛德\mnote{2}说话)你真年轻。

\mxsay{洛德:}我渐渐老了。

\mxsay{毛泽东:}在坐的要属我最老了。

\mxsay{周恩来:}我是第二老的。

\mxsay{毛泽东:}当年英军有人反对你们国家独立。蒙哥马利\mnote{3}元帅则是反对你们政策的人士之一。

\mxsay{基辛格:}是的。

\mxsay{毛泽东:}他也反对杜勒斯\mnote{4}的政策。不过,他大概不会再反对你们了。当时,你们也反对我们,我们也反对你们。所以我们彼此是敌人(大笑)。

\mxsay{基辛格:}以前的敌人。

\mxsay{毛泽东:}现在我们的关系说是叫做什么Friendship(友谊)。

\mxsay{基辛格:}这是我们的感想。

\mxsay{毛泽东:}也正是我说的。

\mxsay{基辛格:}我对周总理说过,我们还没跟其他国家会谈得像跟你们会谈时,这般的坦白和开放。

\mxsay{毛泽东:}不要讲假话,不要搞鬼。你的文件我们是不偷的。你故意放到那里试试看嘛。我们也不搞窃听器那一套,搞那些小动作没用,有些大动作也没用。我跟你们的一个记者Edgar Snow(埃德迦·斯诺)\mnote{5}谈过,我说你们的中央情报局大事也不行,不管用。

\mxsay{基辛格:}这确实是真的。我们的经验是这样的。

\mxsay{毛泽东:}因为,当你们下令时,譬如说,你们的总统下令,你需要关于某些问题的信息,情报单位的报告却像雪片般飞来。我们也有情报局,情形也一样。他们做得不好(周恩来在一旁笑了),例如,他们就不了解林彪\mnote{6}(周继续笑着)。同样的,他们也不知道你想来中国。

\mxsay{毛泽东:}你的事情干得好,到处飞。你是燕子,还是鸽子?越南问题\mnote{7}可以算是基本解决了。

\mxsay{基辛格:}我们感觉是这样,我们现在需要一个走向平静的过渡时期。

\mxsay{毛泽东:}对的。

\mxsay{基辛格:}基本问题都解决了。

\mxsay{毛泽东:}我们也需要嘛。你们的总统坐在这里讲的(手指基的座位),我们两家出于需要,所以就这样,(把两只手握在一起)HAND IN HAND(手携手)。

\mxsay{基辛格:}是的,我们都面对一样的危险。有时我们可能会用不同的方法,但目标是一样的。

\mxsay{毛泽东:}这就好。只要目标相同,我们也不损害你们,你们也不损害我们,共同对付一个王八蛋!实际上是这样。有时候我们也要批你们一回,你们也要批我们一回。你们总统说是叫“思想力量”的“影响”。就是说,“共产党去你的吧!共产主义去你的吧!”我们就说,“帝国主义去你的吧!”有时我们也要讲点呢,不讲不行呢。

\mxsay{基辛格:}我认为我们双方应该忠于各自的基本原则,实际上如果双方讲同样一个调子,只会使局势混乱。我对总理说过,在欧洲,你们出于自己的原则,可以讲得比我们更坚定。

\mxsay{毛泽东:}我们希望你们跟欧洲、跟日本合作。有些事情吵吵闹闹可以,但是根本上要合作啊。

\mxsay{基辛格:}主席先生,从我们这方面说,你们和我们之间虽然有时要进行批评,但我们仍要同你们协调行动。我们在任何时候都不会参与企图孤立你们的行动。至于日本和欧洲,我们同意在一切实质问题上要同他们合作,但目前欧洲的领导很弱。

\mxsay{毛泽东:}他们不团结。

\mxsay{基辛格:}不团结,而且不象你们那样有远见。当他们面临危险的时候,他们总想不费力气就使危险消失。

周恩来(对毛):帮助蓬皮杜\mnote{8},他(指基辛格)同意。

\mxsay{基辛格:}我们正在尽量这样做,回去后我们要争取做得更多。

\mxsay{毛泽东:}现在蓬皮杜受到威胁,就是那个社会党和共产党联合起来顶他。

\mxsay{基辛格:}两家联合。

\mxsay{毛泽东:}两家联合,苏联想要共产党登台。那个共产党我不喜欢,就象你们的共产党我也不喜欢一样。我喜欢你们,不喜欢你们的共产党。(基笑)

你们西方历来有条政策,两次世界大战开始都是推动德国打俄国。

\mxsay{基辛格:}但是,推动俄国打中国不是我们的政策。因为如果在中国爆发战争,对我们来说,其危险性和在欧洲爆发战争一样。

\mxsay{毛泽东:}我正是要讲这句话:是不是你们现在是推动西德\mnote{9}跟俄国讲和,然后又推俄国向东进。我怀疑整个西方有这么一条路线。向东,主要向我们,而且向日本,也有一部分向你们,在太平洋和印度洋。

\mxsay{基辛格:}我们并不赞成德国的政策。我们宁愿德国的反对党上台,德国的反对党不奉行这个政策。

\mxsay{毛泽东:}我跟一个外国朋友谈过,我说要搞一条横线,就是纬度,美国,日本、中国、巴基斯坦、伊朗、土耳其、欧洲。

\mxsay{基辛格:}我曾告诉过总理,你们的行动方式要比我们直截了当和英勇一些。我们有时要采用复杂的方法,这是由于国内的形势所造成的。

\mxsay{毛泽东:}英文什么叫做行动方式啊?

\mxsay{沈若云:}Modeofacton(行动方式)。

\mxsay{基辛格:}不管公众舆论如何,我们对基本目标会有决断的。如果称霸的意图活跃起来,那末真正的危险会发展。不管哪里有这种意图,我们肯定都将予以抗衡。总统曾对主席说过,我们这样做是为了我们本身的利益,而不是为了对别的任何人表示善意。

\mxsay{毛泽东:}你说的这个是老实话。

\mxsay{基辛格:}这是我们的立场。

(毛泽东开始抽雪茄,并试着递雪茄给基辛格和洛德,洛德表示他不抽烟)

\mxsay{基辛格:}我们并未计划在未来四年大量裁减驻扎在欧洲的美军。

(毛泽东望向周恩来)

\mxsay{周恩来:}说到裁军,你的意思是最多裁百分之十到十五。

\mxsay{基辛格:}完全正确。

\mxsay{毛泽东:}美国在欧洲的驻军有多少?他们大都是导弹部队吧。

\mxsay{周恩来:}大概在三十至三十五万间,包括地中海的驻军。

\mxsay{毛泽东:}这大概不包括海军在内。

\mxsay{基辛格:}不包括海军。在欧洲中部约有二十七万五千人,但这不含部署在地中海的第六舰队。

\mxsay{毛泽东:}你们部署在亚洲和太平洋的军队散布很广。你们在南朝鲜\mnote{10}有军队,我听说大约有三十万人。

\mxsay{基辛格:}大约四万。

\mxsay{毛泽东:}蒋介石那儿大约有八到九千人。

\mxsay{周恩来:}在台湾吧。

\mxsay{毛泽东:}听说日本还有两地驻军,四万人在琉球、二至三万人在日本本土。我不知道菲律宾有多少美军,但现在越南的美军只有一万多人。

\mxsay{基辛格:}但他们很快都会撤回。

\mxsay{毛泽东:}对,我听说你们在泰国有四万人。

\mxsay{基辛格:}对的。但主席你刚才说的大都是空军部队,所以恐怕不能光以人数衡量。

\mxsay{毛泽东:}你们也有地面部队,例如在南朝鲜。

\mxsay{基辛格:}我们在南朝鲜确有地面部队。

\mxsay{毛泽东:}你经过日本时,最好多花点时间和他们会谈。你只和他们谈一天,他们的面子很挂不住。

\mxsay{基辛格:}主席,我们希望此行的重点是北京的会谈。稍后我会单独再去一趟东京。

\mxsay{毛泽东:}很好。对他们说清楚些。你知道日本对苏联的感觉也不是很好。

\mxsay{基辛格:}他们有点爱恨交加。

\mxsay{毛泽东:}(比手势)一句话,这是日本田中首相告诉周总理的,苏联做的事就像看到有人要上吊,就立刻把人家脚下的椅子抽走。

\mxsay{基辛格:}是的。

\mxsay{毛泽东:}也可以这么说,他们未发一枪一弹就抢了一大片土地。(周恩来轻轻的笑了)他们抢了蒙古人民共和国,他们抢了一半的新疆和东北的满洲,还说这是他们的势力范围。\mnote{11}

\mxsay{基辛格:}他们还夺走当地所有的工业。

\mxsay{毛泽东:}对呀。他们还抢走库页岛和千岛群岛\mnote{12}。

\mxsay{基辛格:}日本被苏联的经济发展性所迷惑。

\mxsay{毛泽东:}(点头)他们希望从苏联那儿拿回些什么。

\mxsay{基辛格:}但我们将加强日本和美国关系,同时也希望和中国加强关系。

\mxsay{毛泽东:}我们认为,要是日本和苏联加强关系,不如和美国加强关系,这样会比较好些。

\mxsay{基辛格:}日本和苏联如果形成紧密政治关系是很危险的事。

\mxsay{毛泽东:}这在现实上,似乎不可能成真。

\mxsay{周恩来:}(对毛泽东)我们已经决定在双方首都设立联络办事处,以维持黄华\mnote{13}和白宫的联系。

\mxsay{毛泽东:}(对周恩来)重要性何在?

\mxsay{周恩来:}联络办事处将处理一般民众的交流事务。至于保密性强以及紧急事务则不包括在内,这将交由黄华大使的管道处理。

\mxsay{毛泽东:}黄华命苦(周大笑),他在你们那干得很好,现在赶回上海,背还扭伤。

\mxsay{基辛格:}他返回任所时,我们会给他找个医生。

\mxsay{毛泽东:}好啊。(周大笑)黄华好像在你们那比较安全,他一回到上海就摔跤。从你们总统观看中国杂技团演出开始,我想越南问题快解决了吧。还有谣传说,你也快摔跤了(笑声),对这件事,在场女士们可不太满意(笑声,尤其是女士们),有人说,如果博士垮了,我们也将没活干。

\mxsay{毛泽东:}中国和美国贸易量少得可怜,但逐渐在增加。你要知道,中国是很穷的。

\mxsay{毛泽东:}中国人排外得很。你们可以容纳很多民族。我们中国没有几个外国人。你们不同的民族有多少,你们美国有六十万中国人,而我们恐怕连六十个美国人都没有,我也不知道是什么道理,你们研究一下吧。从来就不喜欢外国人。包括日本人也很少。印度人一个都没有。

\mxsay{周恩来:}非常少。

\mxsay{基辛格:}这是因为历史上你们同外国人打交道不幸运。

\mxsay{毛泽东:}有这么点理由。恐怕过去一百年,主要是八国联军,后来日本人占领中国十三年,占领大部分领土。他们过去占领中国领土不算,占领了北京还要赔偿。

\mxsay{基辛格:}是的,还有治外法权。

\mxsay{毛泽东:}治外法权,多哩!现在我们对日本,不要它赔偿。没法算,谁也算不清。赔不起。

\mxsay{周恩来:}一赔就要增加人民负担。

\mxsay{毛泽东:}只有以这种方法我们才能消除敌意,改善两国人民之间的关系;要化解中日人民间的敌意比化解你我之间的敌意困难。

\mxsay{基辛格:}是的,美国人民对中国人民没有任何敌意,相反的,我们之间现在只有一个判断性问题。

\mxsay{毛泽东:}对。

\mxsay{基辛格:}未来几年我们将解决这个问题,但是一个强大的利益共同体很快就会开始运作。

\mxsay{毛泽东:}是吗?

\mxsay{基辛格:}在中国和美国之间。

\mxsay{毛泽东:}你所谓的利益共同体是什么?是指台湾?

\mxsay{基辛格:}是指其他有这种意向的国家。

\mxsay{周恩来:}你是指苏联?

\mxsay{基辛格:}我是指苏联。

\mxsay{周恩来:}沈小姐了解你说的意思。

\mxsay{毛泽东:}(看着翻译沈若云)这个人的英文能力很好。(对周恩来说)她是谁?

\mxsay{周恩来:}她是沈若云。

\mxsay{毛泽东:}我们的翻译实在太少了。

\mxsay{基辛格:}不过,我们遇到的翻译,他们都做得很称职。

\mxsay{毛泽东:}你遇到的翻译和我们现在的翻译现在只有二、三十岁,如果他们老了以后,就无法翻译像现在这么好了。

\mxsay{周恩来:}我们应该送一些人出国。

\mxsay{毛泽东:}我们应该送一些像这样高的小孩(用手比了一下)出国,年龄不要太大。

\mxsay{基辛格:}我们准备设立一些人员交换计划,让你们送学生到美国。

\mxsay{毛泽东:}一百个学生当中如果有十个人学好外语,那就非常成功,即使有十多个学生不想回国,例如一些女孩想留在美国,那也没关系,因为你们美国人不像中国人那么排外。过去,中国到外国却不肯学当地语言,(看着翻译唐闻生)她的祖父母就拒绝学英语。他们就是这么顽固。你知道中国人是非常顽固和保守。许多老一代华侨不肯说当地语言,但年轻一代好多了。

\mxsay{毛泽东:}(比个手势并指着他的书)假如苏联丢了炸弹并杀死三十岁以上的中国人,那将会帮我们解决问题,因为像我一样的老人不会学英文,我们只会读中文,我大部分的书是中文,只有少数的字典是外文,其他大部分的书是中文。

\mxsay{基辛格:}主席现在正在学英文吗?

\mxsay{毛泽东:}我听说外面传说我正在学英文,我不在意这些传闻,它们都是假的,我认识几个英文单词,但不懂文法。

\mxsay{唐闻生:}主席发明了一个英文字。

\mxsay{毛泽东:}是的,我发明了一个英文辞汇——纸老虎。

\mxsay{基辛格:}纸老虎。对了,那是指我们。(笑)

\mxsay{毛泽东:}假如俄国攻击中国,我现在告诉你我们将采用游击战和持久战,我们会让他们到任何想到的地方。(周恩来笑了)他们想到黄河流域,那好啊!很好啊。(笑)假如他们进一步到长江流域,那也不坏啊。

\mxsay{基辛格:}不过,如果他们使用炸弹而不派兵呢?(笑)

\mxsay{毛泽东:}我们要怎么办?也许你可以组成一个委员会去研究这个问题,我们将让他们猛攻一番,而他们将损失许多资源。他们说他们是社会主义者,我们也是社会主义者,那么苏联进攻中国就是社会主义者攻击社会主义者。

\mxsay{基辛格:}假如他们攻击中国,我们肯定会基于我们的理由反对他们。

\mxsay{毛泽东:}但是你的人民并未觉醒,欧洲和你们都会认为祸水流向中国将是一件好事。

\mxsay{基辛格:}欧洲想什么我无法判断,他们不会做任何事,因为他们基本上与此事无关。我们考虑的是假如苏联占领中国,将影响其他国家的安全并造成我们的孤立。

\mxsay{毛泽东:}(笑)那会怎样?因为自从深陷越南后,你们遭遇这么多的困难,你想如果苏联深陷在中国,他们会感到舒服吗?

\mxsay{基辛格:}苏联?

\mxsay{唐闻生:}苏联。

\mxsay{毛泽东:}那时候你们可以让苏联深陷在中国,半年、一年、两年、三年或四年,戳苏联的背后,那时候你们的口号将是寻求和平,你们将以和平之名瓦解社会主义帝国,也许你们将以作生意帮助他们,并向他们表示可以提供一切协助反对中国。

\mxsay{基辛格:}主席先生,我们了解彼此的意图实在非常重要,我们绝对不会联手攻击中国。

\mxsay{毛泽东:}(打断基的谈话)不,不是这样,你正在进行的目标是瓦解苏联。

\mxsay{基辛格:}那是一件非常危险的事。(笑)

\mxsay{毛泽东:}(用两手作势)苏联的目标是占领欧亚两个大陆。

\mxsay{基辛格:}我们希望吓阻苏联的攻击,而不是击败他,我们希望阻止他。

(周恩来看表)

\mxsay{毛泽东:}世事难料,我们宁愿事情如此发展,这样的发展对世界来说比较好。

\mxsay{基辛格:}哪种方式?

\mxsay{毛泽东:}那就是苏联进攻中国并且被击败,我们必须作最坏的考虑。

\mxsay{基辛格:}那是你的必然性推论。

\mxsay{毛泽东:}无论如何,老天已经送给我一封邀请函。

\mxsay{基辛格:}我真的发现主席今年气色比去年好。

\mxsay{毛泽东:}是的,我的情况比去年好。请代我向尼克松总统致意,同时也向尼克松夫人致意,很抱歉无法与她及罗杰国务卿见面。

\mxsay{基辛格:}我一定会转达。

\date{一九七三年十一月十二日}
\section*{(二)}

\mxsay{毛泽东:}我们现在来讨论讨论台湾问题。美国跟我们的关系,应当和我们与台湾之间的关系分开来处理。

\mxsay{基辛格:}原则上……

\mxsay{毛泽东:}只要你们跟台湾切断外交关系,就可以来解决我们两国之间的外交关系问题。这跟我们和日本之间的处理方式是一样的。至于我们跟台湾的关系,就相当的复杂。我看没办法和平解决。

他们\mnote{14}都是一伙儿反革命分子,怎么会跟我们合作?我说我们可以暂时不要台湾,过一百年再去管他。对世事不要太急。有什么好急的呢?那只不过是个千把万人口的小岛罢了。

\mxsay{周恩来:}他们现在有一千六百万人。

\mxsay{毛泽东:}至于你们跟我们的关系,我想用不到一百年的时间处理。

\mxsay{基辛格:}我相信是这样。我们应该会快得多。

\mxsay{毛泽东:}不过这是你们自己要决定的事情。我们也不会催你们。要是你们觉得有必要,我们就配合。如果你们觉得现在还不行,那我们也可以缓一点。

\mxsay{基辛格:}在我们而言,我们希望能够跟中华人民共和国建立外交关系。困难在于我们不能马上就切断与台湾的关系。有几个理由,而这些理由全都跟我们的国内情势有关。我跟总理说过,我们希望在一九七六年之前,或者在一九七六年期间完成这个程序。所以,我们要找到某种方式,来建立外交关系,也可以当作贵我加强关系的象征,因为就技术层面来说,联络处\mnote{15}用处非常大。

\mxsay{毛泽东:}这是可以的。

\mxsay{基辛格:}什么可以?

\mxsay{毛泽东:}可以照现在的办法来做,因为现在你们还需要台湾。

\mxsay{基辛格:}这不是需不需要台湾的问题,而是实际上可不可行的问题。

\mxsay{毛泽东:}都一样(大笑)。我们现在对香港也不急(大笑)。我们甚至不会碰澳门。如果我们真的想要碰澳门,顶多也只碰一点点。因为葡萄牙从明朝就开始紧紧地掌控着澳门(大笑)。赫鲁晓夫也骂过我,说为什么我们连香港和澳门都不要\mnote{16}。而且我也跟日本说过,我们不只赞同他们(日本)要求(苏联)归还北方四岛,而且还包括历史上,苏联自中国割去的一百五十万平方公里土地。

\mxsay{基辛格:}主席,我觉得外交关系上的问题就是这个。在台湾问题上,我想我们都很了解各自的立场。所以我们现在的问题是……,而且联络处也发挥了应有的功能。所以唯一的问题是,我们是不是都认为,或者在某一个程度上认为,应该象征性地显示,中美关系在各方面说来都是正常的。果真如此,我们就应该想个办法来做,不过这不是非做不可。

\mxsay{毛泽东:}我们也有和苏联与印度建交啊,不过关系也不怎么样。甚至还没有我们与你们的关系好。所以这不是很重要。整个国际情势才真的重要。

\mxsay{基辛格:}我完全同意主席的说法,也同意我们必须彼此了解,我相信我们对彼此都有实质的了解。

\mxsay{毛泽东:}我们联络处主任已经跟你们说过大原则,也提过当年乔治·华盛顿反抗英国的故事。

\mxsay{基辛格:}对,他在几个星期之前对我作过精彩的演说,我以前也从总理那里听过。

\mxsay{毛泽东:}那套说法可以不用再提了。我们现在有一个矛盾:一方面,我们一直支持阿拉伯国家反对以色列的建国运动,另一方面,我们必须对美国在当地阻止苏联的做法表示欢迎,以免苏联控制中东地带。我们黄镇大使提过我们支持阿拉伯世界,不过他不了解阻止苏联势力扩张的重要性。

\mxsay{基辛格:}嗯,我那时的说法让他很惊讶,而他也重述了联合国的官式立场(大笑)。而我了解你们必须公开地采取某些立场。这些并不违反我们的共同立场。不过,事实上,我们会进一步解决中东问题,但我们也想显示这不是因为苏联施压才使然。

所以,每当苏俄施压,我们必须抗拒,除了争执本身有其优点之外,我们可说是为了抗拒而抗拒。然后当我们击败他们时,我们甚至有可能朝同样的方向前进,我们不是反对阿拉伯国家的想法,我们只是反对借着苏联施压来实现他们的愿望。

\mxsay{毛泽东:}没错。

\mxsay{基辛格:}而且我们现在的策略正是如此。

\mxsay{毛泽东:}下次你再到中国来,除了政治,还可以同我谈一点哲学。

\mxsay{基李格:}我很乐意,主席先生。哲学研究是我的初恋。

\mxsay{毛泽东:}或许当了国务卿后就比较难研究哲学了。

\mxsay{基辛格:}是的。

\mxsay{毛泽东:}赫鲁晓夫说我们像是好斗的公鸡。

\mxsay{基辛格:}他一九五九年到这里访问时,不太成功。

\mxsay{毛泽东:}我们在一九五八年决裂。他们一九五八年想要控制中国海岸和中国军港时,我们开始失和。我和他们、他们的大使讨论的时候,我差一点拍桌子,我骂了他一顿。(笑)在莫斯科报告后,赫鲁晓夫就来了。当时,赫鲁晓夫提出联合舰队的构想,就是苏联和中国共组一个联合海军舰队。那就是他说的。那时候,他口气还很大,因为他见过当时的美国总统艾森豪威尔将军:就沾了点所谓的“戴维营精神”。他在北京同我夸口,说他认识美国总统,讲到艾森豪威尔总统时还说了两个英文字,说他是“我的朋友”(my friend)。(问布鲁斯大使;你知道这回事吗?)

\mxsay{布鲁斯:}不,我从不知道这回事。

\mxsay{毛泽东:}还有一些消息。从此以后,赫鲁晓夫没有再来过。但他曾到过海参崴,他是从中国到海参巍的。

\mxsay{周恩来:}他在那里发表了一篇反中国的演讲。

\mxsay{毛泽东:}现在的苏联领导人没有一个到过像海参崴那么远的东边。柯西金自己说过,他对西伯利亚的事情不太清楚。

(中方查看时间)

\mxsay{周恩来:}已经两个半小时了。

\mxsay{毛泽东:}我本来还想同你谈另外一个问题。不过今天我们好像谈得太久了。超过两个半钟头。我们占用了原本为别的活动安排的时间。我想讨论的是,我很怀疑,如果民主党掌权,他们会采用孤立主义。

\mxsay{基辛格:}这是一个非常严肃的问题,主席先生。我认为目前的知识分子和一些民主党,可能会走向孤立主义。不过,现实状况会让他们了解,我们目前的政策就是唯一的选择。目前我还不知道,在他们了解这点以前会产生多大伤害,还有,他们是否将继续同样错综复杂的战术。但是我想,他们不会偏离现在的轨道。

\mxsay{毛泽东:}那你和我似乎是同一类的。我们好像多少都有点疑神疑鬼的。

\mxsay{基辛格:}我是怀疑,我对某些领袖也有些疑问。但是我想形势所需,使我们只有回到我们现在在进行的政策。

不过,主席先生,这也是为什么我们应该趁着大家都还在位,而且都了解这情势的时候,尽量加强关系,不要让其他政策有机可趁。

\mxsay{毛泽东:}这个主要从一点上就可以证明,就是主张从欧洲撤军。

\mxsay{基辛格:}是的。

\mxsay{毛泽东:}这是帮苏联的大忙了。

\mxsay{基辛格:}我们将不会在我们尼克松总统任内执行。只有两种可能的情形:军队从欧洲撤退,如遇挑战,也不太愿意极为粗暴、极为迅速的解决。

\mxsay{毛泽东:}你所说的“粗暴”意思是指可能开战。

\mxsay{基辛格:}如果有必要,但……

\mxsay{毛泽东:}我不喜欢你同我耍外交辞令。

\mxsay{基辛格:}如果有必要,但是根据我们的经验,只要他们知道我们将发动战争,他们就会松手。直到现在,他们一直很怕我们。

\mxsay{毛泽东:}因为我也认为最好不要开战。我也不喜欢战争。虽然人家都说我是个好战分子。(笑)如果你们和苏联开战,我也觉得不要。如果你们真的打,最好只用传统武器,核武器只是备用,尽量不要碰。

\mxsay{基辛格:}我们绝对不会发动战争。

\mxsay{毛泽东:}那就好。我听说你们以前这个方法是为了争取时间。

\mxsay{毛泽东:}我们是想争取时间,可是,我们还有另一个立场,如果苏联攻击我们刚刚讨论过的任何一个重要地区,我们要能反击。我们必须防患未然。

\mxsay{毛泽东:}完全正确。至于苏联,他们欺弱怕强。

\begin{maonote}
\mnitem{1}一九七三年二月十七日,星期六晚间十一点半至隔天凌晨一点二十分。毛泽东接见了美国国家安全事务助理基辛格,在座的有周恩来总理、外交部部长助理王海容、翻译唐闻生、翻译沈若云、美国国家安全委员会委员洛德。
\mnitem{2}洛德,总统国家安全事务助理基辛格的特别助理,时年三十六岁。
\mnitem{3}蒙哥马利(一八八七——一九七六),英国陆军元帅。第二次世界大战期间是盟军指挥官之一。后曾任英军总参谋长、北大西洋公约组织盟军最高副总司令。
\mnitem{4}杜勒斯,美国共和党人,一九五三年至一九五九年任美国国务卿。他在国际活动中,鼓吹冷战,推行“战争边缘”“大规模核报复”以及对社会主义国家进行“和平演变”等战略。一九五零年他参与策划美国政府利用朝鲜战争武装侵占中国领土台湾。一九五四年他又策划美国政府同台湾当局签订《美台共同防御条约》,企图使霸占台湾的行为合法化,将台湾长期作为美国的军事基地。他一贯敌视社会主义国家和民族解放运动,坚持不承认中国、非法排斥中国在联合国的合法地位、对中国实行封锁禁运,明目张胆地进行制造“两个中国”的阴谋活动。
\mnitem{5}斯诺,(一九〇五——一九七二年)美国新闻记者、作家,埃德加·斯诺,美国进步作家、记者。一九二八年第一次到中国。一九三六年到陕北革命根据地访问,见到了毛泽东等中共和红军的领导人,后写了《西行漫记》等书。新中国成立后,在一九六〇年、一九六四年、一九七〇年访问中国。一九七二年二月十五日因病在瑞士日内瓦逝世。
\mnitem{6}林彪(一九〇七——一九七一),湖北黄冈人。一九二五年加入中国共产党。一九五八年五月在中共八届五中全会上被增选为中共中央副主席、政治局常务委员。一九五九年任中央军委副主席、国防部长,主持中央军委工作。在九届二中全会上主张设国家主席(毛泽东主席明确表示要改变国家体制不设国家主席),并组织人企图压服中央,犯了错误,被毛泽东主席识破,对其进行了警告和批评,并等待其认错达一年之久(从一九七〇年九月到一九七一年九月),不料,其子林立果狂妄自大,趁毛泽东南巡之时,妄图谋杀毛泽东主席,事情败露后,九月十三日夜,林立果挟制林彪和叶群驾机逃往苏联,最后坠毁于蒙古温都尔汗,史称“九一三”事件。后,林立果制定的《“五七一”工程纪要》被发现,因此,中央认定,林彪叛国。一九七三年八月中共中央决定,开除他的党籍。
\mnitem{7}一九七三年一月二十七日,越南民主共和国、美国、越南南方共和临时革命政府、西贡伪政权四方代表在巴黎签订的《关于在越南结束战争、恢复和平的协定》,简称《越南问题的巴黎协定》。根据规定,美国政府必须在六十天内撤走全部军队,从而为越南人民自己解决自己的问题创造了有利的条件。

由于基辛格对这份协定的贡献,一九七三年十月十六日,挪威议会选出的诺贝尔和平奖五人委员会宣布基辛格与黎德寿(越方领导人)获得诺贝尔和平奖。
\mnitem{8}蓬皮杜,时任法国总统。
\mnitem{9}西德,德意志联邦共和国,简称西德或联邦德国,第二次世界大战后,德国分裂为民主德国(东德)和联邦德国(西德),一九九〇年十月三日民主德国正式加入联邦德国,两德统一,称德国。
\mnitem{10}南朝鲜,今大韩民国,简称韩国。
\mnitem{11}二十世纪二十年代到四十年代,二十年间,苏联趁中国内乱无力北顾,逐渐蚕食中国领土,利用二战日本入侵中国、国共相争的时机,逼迫中国国民政府承认外蒙古(蒙古人民共和国)独立,并取得了在新疆和东北的一系列特权。外蒙古独立是指外蒙古于二十世纪前期从中国独立出去的一直影响到现在中国的巨大历史事件,是中国历史上最黑暗的时代之一。
\mnitem{12}一九四五年二月,苏美英关于战后日本问题的《雅尔塔协定》规定,整个千岛群岛,包括日本方面所说的“北方四岛”,择捉、国后、色丹和齿舞群岛,都划归苏联。
\mnitem{13}黄华,时任常驻联合国及其安全理事会的代表。
\mnitem{14}指台湾的蒋介石集团。
\mnitem{15}一九七三年五月中美两国在对方首都互建联络处,建立大使级外交关系,资深外交家、原驻法大使黄镇任驻美联络处主任,美国资深外交家布鲁斯任驻华联络处主任。
\mnitem{16}一九六二年十二月,赫鲁晓夫在苏联最高苏维埃会议上发表讲话时提出:为什么中华人民共和国对收回由于侵略战争和不平等条约落入葡萄牙和英国帝国主义监督之下的澳门和香港不感兴趣,而却对与印度的边界怀有敌意呢?
\end{maonote}
