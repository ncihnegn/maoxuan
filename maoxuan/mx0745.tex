
\title{广大干部下放劳动是一种重新学习的极好机会}
\date{一九六八年九月三十日}
\thanks{这是毛泽东同志对《柳河“五·七”干校为机关革命化走出条新路》\mnote{1}一文的批语。}
\maketitle


此件似可在人民日报发表。广大干部下放劳动,这对干部是一种重新学习的极好机会,除老弱病残者外都应这样做。在职干部也应分批下放劳动。

\begin{maonote}
\mnitem{1}《柳河“五·七”干校为机关革命化走出条新路》这个材料最初在一九六八年九月二十九日《文革情况汇编》第六二八期上发表。“革命委员会”的成立,机构精简产生出许多精简出来的干部,加上原来一些“牛棚”里的干部也回来了,这些干部的安置便成了一个大问题。这时,黑龙江省出了一个新事物。原来,黑龙江省革命委员会也一直在为如何安排精简下来的干部和“牛棚”里的干部而发愁。一开始,黑龙江省革命委员会负责人只是安排这些人打扫卫生,清理街道的垃圾,干一些杂活。但这些活毕竟有限,干完这些活,这些人仍然闲着无事做。后来,他们想出一个办法:把这些干部集中安排到农村,办一个农场,保留工资待遇,让他们在体力劳动中“改造”自己。并且经过考察,最后选定了庆安县的柳河,作为办这种农场的试点。

柳河这个地方,自然资源比较好,也有一部分空出的房舍,安置第一批干部不成问题。第一批干部到达柳河后,还可以继续建房、开荒、种树、办小工厂,为安置第二批干部创造条件。去柳河的干部,采取分期分批的形式,且在那里的时间有长有短;干部劳动的时间,也根据体力而有长有短;又保留原工资待遇,干部们很容易接受。于是,黑龙江省革命委员会在一九六八年五月七日,毛泽东的“五七指示”发表两周年之际,正式启动这一计划,当天组织第一批干部到达柳河,并把这个农场命名为——柳河“五七”干校。

柳河“五七”干校初办时还算顺利。到达那里的干部,与当地农民建立了深厚的感情,他们的生活和劳动也得到了当地农民的帮助。这些干部在柳河种了不少地,又新开了一些荒地,盖了不少新房子,还种树,搞副业生产。黑龙江省革命委员会把办柳河“五七”干校的经验介绍材料进行了上报。经验材料中说:办柳河“五七”干校,“为机关革命化,改革上层建筑走出了一条新路。干校共有学员一百四十一人,主要是原省、市委机关干部和革命委员会的工作人员。机关干部办农场,走与工农相结合的道路,深受广大贫下中农的欢迎。不少干部到干校后,亲临三大革命第一线,接近了贫下中农,增强了对劳动人民的思想感情。目前,干校耕种土地三千余亩,农、林、牧、副、渔全面发展,并自力更生办起了小型工厂、企业。实践证明,‘五七’干校是改造和培养干部的好地方,是实现机关革命化,搞好斗、批、改的一种好办法”。

毛泽东看到这个材料后,于九月三十日写了本则批语。

十月五日,《人民日报》刊登了《柳河“五七”干校为机关革命化提供了新的经验》的报道,引用了毛泽东的批语,同时刊发了姚文元写的编者按:毛主席关于柳河“五七”干校经验的批语,“对反修、防修,对搞好斗、批、改,有十分重大的意义,应引起我们各级革命干部和广大革命群众的高度重视。希望广大干部(除老弱病残者外),包括那些犯过错误的干部,遵照毛主席的指示,在下放劳动的过程中重新学习,使自己的精神面貌来一个比较彻底的革命化。在革命委员会中工作的新老革命干部,也要执行毛主席的指示,分期分批下放劳动,使自己不脱离劳动人民,既当‘官’,又当老百姓。新干部要特别注意不要染上脱离群众、脱离劳动、一切依靠秘书、做官当老爷的剥削阶级坏作风,要保持无产阶级朝气蓬勃的革命的青春”。

柳河“五七”干校成了毛泽东肯定的典型后,各地纷纷仿效,办起了许多干校。中央各机关也都在外地寻找地点,分别办起了各自的“五七”干校。一开始,黑龙江省办的柳河“五七”干校,主要是为了安置干部。因为毛泽东主席在批示中明确指出,“除老弱病残者外”,广大干部都应该下放劳动,“在职干部也应分批下放劳动”。而且,他把自己的女儿也送到中央办公厅办的“五七”干校去锻炼。《人民日报》的编者按中说得更明确:“在革命委员会中工作的新老干部,也要执行毛主席的指示,分期分批下放劳动。”所以实际上,“五七”干校实际上成了一种新型的“党校”。

后来有人疯狂地谩骂“五七”干校,他们认为自己做了官就只能上不能下,再也不可以与劳动人民一样参加劳动。这才是他们极力扭曲“五七”干校的真正原因。
\end{maonote}
