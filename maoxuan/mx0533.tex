
\title{改造资本主义工商业的必经之路}
\date{一九五三年九月七日}
\thanks{毛泽东同志在一九五三年九月七日同民主党派和工商界部分代表进行了谈话。这是毛泽东同志写的谈话要点。}
\maketitle


经过国家资本主义,完成由资本主义到社会主义的改造。

(1)过去三年多,做了一些工作,但忙别的去了,用力不多,现在起要多做些工作。

(2)有了三年多的经验,已经可以肯定:经过国家资本主义完成对私营工商业的社会主义改造,是较健全的方针和办法。

(3)《共同纲领》第三十一条\mnote{1}的方针,现在应明确起来和逐步地具体化。所谓“明确起来”,是说在中央及地方的领导人物的头脑中,首先肯定国家资本主义是改造资本主义工商业和逐步完成社会主义过渡的必经之路。这一点无论在共产党和民主人士方面,都还没做到,此次会议的目的,应当做到这一点。

(4)稳步前进,不能太急。将全国私营工商业基本上(不是一切)引上国家资本主义轨道,至少需要三年至五年的时间,因此不应该发生震动和不安。

(5)公私合营、全部出原料收产品的加工订货和只收大部产品,是国家资本主义在私营工业方面的三种形式。

(6)私营商业亦可以实行国家资本主义,不可能以“排除”二字了之。这方面经验较少,尚须研究。

(7)占有大约三百八十万工人、店员的私营工商业,是国家的一项大财富,在国计民生中有很大的作用。私营工商业不仅对国家供给产品,而且可以为国家积累资金,可以为国家训练干部。

(8)有些资本家对国家保持一个很大的距离,他们仍没有改变唯利是图的思想;有些工人前进得太快了,他们不允许资本家有利可得。我们应向这两方面的人们进行教育,使他们逐步地(争取尽可能快些)适合国家的方针政策,即使中国的私营工商业基本上是为国计民生服务的、部分地是为资本家谋利的——这样就走上国家资本主义的轨道了。

关于国家资本主义企业的利润分配,有一个表:

\begin{center}
\begin{tabular}{ccccc}
\toprule
所得税 & 福利费 & 公积金 & 资方红利 & 总计 \\
\midrule
34.5\% & 15\%  & 30\%   & 20.5\%   & 100.0\%  \\
\bottomrule
\end{tabular}
\end{center}

(9)需要继续在资本家中间进行爱国主义教育,为此需要有计划地培养一部分眼光远大的、愿意和共产党和人民政府靠近的、先进的资本家,以便经过他们去说服大部分资本家。

(10)实行国家资本主义,不但要根据需要和可能(《共同纲领》),而且要出于资本家自愿,因为这是合作的事业,既是合作就不能强迫,这和对地主不同。

(11)全国各民族、各民主阶级、各民主党派、各人民团体在过去几年中已有很大的进步,相信再有三年至五年,这种进步将更大,所以三年至五年内基本上完成将私营工商业引上国家资本主义轨道是有可能的。国营企业的优胜,则是完成这一任务在物质方面的保证。

(12)至于完成整个过渡时期,即包括基本上完成国家工业化,基本上完成对农业、对手工业和对资本主义工商业的社会主义改造,则不是三五年所能办到的,而需要几个五年计划的时间。在这个问题上,既要反对遥遥无期的思想,又要反对急躁冒进的思想。

(13)一个是领导者,一个是被领导者,一个是不谋私利者,一个是还要谋一部分私利者,等等,这些是不相同的。但私营工商业基本上是为国计民生服务的(就利润分配上说,约占四分之三左右),因此可以和应当说服工人,和国营企业一样,实行增产节约、劳动竞赛,提高劳动生产率,降低成本,提高数量质量,这样对公私、劳资都有利。


\begin{maonote}
\mnitem{1}共同纲领第三十一条规定:“国家资本与私人资本合作的经济为国家资本主义性质的经济。在必要和可能的条件下,应鼓励私人资本向国家资本主义方向发展,例如为国家企业加工,或与国家合营,或用租借形式经营国家的企业,开发国家的富源等。”
\end{maonote}
