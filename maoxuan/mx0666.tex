
\title{打破核垄断,消灭核武器}
\date{一九六四年十月二十二日}
\thanks{这是毛泽东同志为人民日报起草的社论。}
\maketitle


我国成功地爆炸了第一颗原子弹以后,在全世界引起了巨大的反响。一切反对帝国主义、爱好和平的人民,特别是亚洲、非洲、拉丁美洲的革命人民,都欢欣鼓舞,热烈赞扬我国人民的这个重大成就,支持我国人民为反对美帝国主义核讹诈和核威胁而采取的正当措施。

社会主义各国人民、全世界爱好和平的人民、特别是亚洲、非洲、拉丁美洲的革命人民深信,社会主义中国手中的核武器,是保卫世界和平的强大力量。他们把中国人民的这一胜利,看作是他们自己的胜利。

中国的核试验长了全世界革命人民的志气,灭了美帝国主义的威风。在我国第一颗原子弹爆炸成功以后,美国总统约翰逊在不到三小时之内就发表声明加以反对,并且说中国的原子弹意思不大,不足以动摇美国的核霸权;接着,他在十八日的电视演说中再次对中国进行攻击。这一次他却说,“不应该把这件事等闲视之”。美国政府在重大的国际事件上,表现得如此罕见地慌乱,前言不搭后语,正好说明了中国原子弹的爆炸是对美国核霸王的当头一棒。

美帝国主义的核讹诈政策是建立在核垄断的基础之上的。美国的核垄断地位被进一步打破之后,美国的核讹诈政策就吃不开了。中国掌握了核武器,当然使美帝国主义感到万分恼火。他们反对中国进行核试验,掌握核武器,是一点也不奇怪的。奇怪的是,一贯敌视中国人民的约翰逊,这一回却装出一副假仁假义的姿态,似乎美国之所以反对中国拥有核武器,并不是由于中国打破了美国的核垄断,而是出于对中国人民利益的关怀。

据约翰逊说,中国的核武器对于中国人民来说是“一个悲剧”,因为中国的“有限的资源”被用来制造核武器,而不能“用来改善中国人民的福利”。

约翰逊的意思无非是说,中国是个穷国,搞不起核武器。帝国主义总是低估人民力量的。从新中国成立的第一天起,它们就一直在嘲笑中国的贫穷和落后,说中国这也搞不成,那也搞不成。似乎中国人民不听凭他们的摆布,不依靠他们的援助和恩赐,就什么也干不了。但是,站起来了的中国人民,是有志气的,是勇敢勤劳的。我们深深地懂得,如果不能有效地抵抗帝国主义的侵略,我们的一切资源,就都不过是帝国主义的囊中物;我们的和平劳动,就毫无保障。正是美帝国主义的核讹诈和核威胁,迫使中国人民自力更生,奋发图强,终于克服了重重困难,取得了抵制美国核威胁的手段。中国实现了第一颗原子弹的爆炸,如果说是什么悲剧的话,不是别人的悲剧,而是美帝国主义的悲剧:它要使中国人民沦为核奴隶的迷梦破灭了,它的核垄断地位从根本上动摇了。

约翰逊还装出十分关心中国安全的样子,说什么中国有了核武器,“只会增加中国人民的不安全感”。

这是怎么回事呢?谁都知道,长期以来,美帝国主义一直对中国人民进行核威胁,甚至把核武器摆到中国的大门口。配备着核武器的美国第七舰队在中国沿海晃来晃去,美国在中国周围建立了一个又一个的核基地,美国军政头目嚣张跋扈地扬言要向中国扔核弹。照约翰逊的说法,美国的核威胁能够增加中国人民的安全感,而中国人民拥有对付美国核威胁的核武器,反倒会增加中国人民的不安全感,这真是奇怪的逻辑!按照这种逻辑,中国人民要想安安稳稳过日子,就应当彻底解除武装,托庇于美国的“核保护伞”之下,除此别无出路。

老实告诉你,约翰逊先生,中国在没有核武器的时候,根本就没有慑服于你们的核威胁。中国现在有了核武器,固然可以增强我国的国防力量,但是我们从来没有把我们的安全感建筑在核武器上。中国有比原子弹更强大的东西,这就是:战无不胜的毛泽东思想,光荣的伟大的正确的中国共产党的领导,高度觉悟和坚强团结的六亿五千万人民,和优越的社会主义制度。依靠这些,我们就能战胜任何强大的敌人。

约翰逊不仅断定中国的核试验对中国人民没有好处,而且“对和平事业没有帮助”。

照约翰逊说来,似乎只有在核大国的把持和垄断下,才能保持世界和平;一旦中国有了核武器,打破了核大国的垄断,天下就要出乱子。但是谁都知道,美国发展核武器,是为了发动侵略、称霸世界;中国发展核武器,是为了保卫自己、维护和平。为什么侵略成性的美帝国主义有了核武器,倒是和平的“福音”,而爱好和平的社会主义中国有了核武器,却成为人类的“灾难”?美国为了推行侵略政策和战争政策,拼命发展核武器,已经二十年了,进行过几百次的核试验。为什么你们这种疯狂的核军备竞赛,就是对和平事业有帮助的,而中国为了自卫进行核试验,就是对和平事业没有帮助的呢?真是颠倒是非,岂有此理!

约翰逊说了这么多,意思只有一个,那就是:只应该美国有核武器,中国不应该有核武器。这真是一脸恶霸相,十足流氓腔。

老实说,中国并不醉心于拥有核武器。只要美帝国主义不搞核武器,中国也不搞;可是,只要美帝国主义手中还有核弹,中国就非有不可。不管约翰逊欺骗也好,恫吓也好,污蔑也好,都改变不了中国人民的这个主张。

当然,中国手里的核武器同美帝国主义手里的核武器,有着本质的不同。中国是一个社会主义国家,我们一向是根据中国人民的利益、社会主义阵营的利益、亚洲、非洲和拉丁美洲民族解放运动的利益、世界革命人民的利益和世界和平的利益来确定我们的外交政策的。有了核武器之后,我们仍将一如既往,奉行和平外交政策。我们既不会用这个东西去吓唬别人,进行任何冒险;也不会把它当作参加“核俱乐部”的入场券,做任何损害世界人民革命利益和世界和平利益的事情。新中国成立以来十五年的历史,证明了在反对帝国主义的侵略政策和战争政策、支持各国人民革命运动、保卫世界和平的斗争中,社会主义的中国是完全可以信赖的。

约翰逊以小人之心度君子之腹,竟然说中国要以自己“少量的核力量”来同美国的“强大的(核)武库作交易”。总统先生,你完全想错了。中国发展核武器,并不是想以此作为资本来同你们讨价还价,做一笔什么买卖,而是要打破你们的核垄断,进而消灭核武器,以便永远消除笼罩着人类的核战争危险。这一点在中国政府十月十六日的声明中是说得明明白白的。

中国在核武器问题上的立场是一贯的。过去,当中国没有核武器的时候,我们主张全面禁止和彻底销毁核武器;现在,中国有了核武器,我们还是这样主张。在中国的第一颗原子弹爆炸以后,中国政府立即郑重声明,中国在任何时候、任何情况下,都不会首先使用核武器。中国政府的这个立场,最鲜明不过地说明了中国发展核武器完全是为了自卫,为了抵抗美国的核威胁,归根到底,是为了全面禁止和彻底销毁核武器。

中国政府还郑重建议:召开世界各国首脑会议,讨论全面禁止和彻底销毁核武器问题。作为第一步,各国首脑会议应当达成协议,即拥有核武器的国家和很快可能拥有核武器的国家承担义务,保证不使用核武器,不对无核国家使用核武器,不对无核武器区使用核武器,彼此也不使用核武器。

中国政府的关于首先达成不使用核武器的协议的具体建议是现实的,是合情合理的,是简单易行而不牵涉到监督问题的。如果有关国家都愿意承担这个义务,就可以立即减少发生核战争的危险。这就向全面禁止和彻底销毁核武器的最终目标迈开了重大的第一步。在这之后,可以讨论停止一切核试验,禁止输出、输入、扩散、生产、贮存和销毁核武器问题。显然,美国政府如果还有一点和平的意愿,就没有任何理由拒绝这个建议。

约翰逊在他的声明和电视演说中,避开中国政府的建议,而夸夸其谈,说什么要中国参加三国禁止部分核试验条约;要通过核查的协议来结束一切种类的核试验;要争取避免核扩散;要其他无核国家接受美国“核保护伞”的庇护。约翰逊提出这么一大套东西,拿腊斯克\mnote{1}攻击中国的话来说,不过是一种烟幕,用来掩饰美国不敢承担不首先使用核武器的顽固而又虚弱的立场。

三国条约\mnote{2}的作用,经过一年多时间的检验,早就真相大白了。它是巩固美国核垄断地位的骗局。早在三国条约签订的时候,我们就没有上当。现在我们有了打破美国核垄断的手段,还能指望我们会自己钻进这个圈套吗?这简直是异想天开。

约翰逊好像十分热心于防止核扩散。事实上,真正在搞核扩散的不是别人,正是美国自己。约翰逊政府正在积极推行所谓多边核力量计划,把核武器交给北大西洋集团国家,特别是西德复仇主义者的手中。美国的这种做法,既是为了准备核战争,也是为了加强对它的盟国的核控制。为了反对美国的核威胁,为了反对美国在它的侵略集团中间扩散核武器,一定会有越来越多的爱好和平的国家响应和支持中国政府的主张和建议。美国核垄断的局面已经维持不下去了。这对于全面禁止和彻底销毁核武器是大有好处的。防止核战争、禁止核武器的希望,绝不在于巩固美国的核垄断,而在于打破这种核垄断。把美国核垄断的地位打破得越彻底,全面禁止和彻底销毁核武器的可能性也就越大。事物发展的辩证法就是这样的。

核武器不是上帝造出来的。既然人类能够制造核武器,也就一定能够消灭核武器。我们深信,通过各国人民的联合斗争,核战争是能够防止的,核武器是可以禁止的。社会主义阵营各国人民要团结起来,亚洲非洲拉丁美洲人民要团结起来,全世界人民要团结起来,为彻底粉碎美帝国主义的核讹诈和核威胁,为争取实现全面、彻底、干净、坚决地禁止和销毁核武器的崇高目标而奋斗到底!

\begin{maonote}
\mnitem{1}腊斯克,一九〇九年生,美国民主党人,时任美国国务卿。
\mnitem{2}三国条约,指苏美英关于一九六三年八月五日在莫斯科正式签署了《禁止在大气层、外层空间和水下进行核试验条约》。禁止在大气层、水下和宇宙空间进行核试验。但是并不禁止地下核试验。
\end{maonote}
