
\title{中国共产党中央委员会关于无产阶级文化大革命的决定}
\date{一九六六年八月八日}
\thanks{这是毛泽东同志主持起草的中共中央的决定。}
\maketitle


\section{一、社会主义革命的新阶段}

当前开展的无产阶级文化大革命,是一场触及人们灵魂的大革命,是我国社会主义革命发展的一个更深入、更广阔的新阶段。毛泽东同志在党的八届十中全会上说过,凡是要推翻一个政权,总要先造成舆论,总要先作意识形态方面的工作。革命的阶级是这样,反革命的阶级也是这样。实践证明,毛泽东同志的这个论断是完全正确的。资产阶级虽然已被推翻,但是,他们企图用剥削阶级的旧思想,旧文化,旧风俗,旧习惯,来腐蚀群众,征服人心,力求达到他们复辟的目的。无产阶级恰恰相反,必须迎头痛击资产阶级在意识形态领域里的一切挑战,用无产阶级自己的新思想,新文化,新风俗,新习惯,来改变整个社会的精神面貌。在当前,我们的目的是斗跨走资本主义道路的当权派,批判资产阶级的反动学术“权威”,批判资产阶级和一切剥削阶级的意识形态,改革教育,改革文艺,改革一切不适应社会主义经济基础的上层建筑,以利于巩固和发展社会主义制度。

\section{二、主流和曲折}

广大的工农兵、革命的知识分子和革命的干部,是这场文化大革命的主力军。一大批本来不出名的革命青少年成了勇敢的闯将。他们有魄力、有智慧。他们用大字报、大辩论的形式,大鸣大放,大揭露,大批判,坚决地向那些公开的、隐蔽的资产阶级代表人物进行了进攻。在这样大的革命运动中,他们难免有这样那样的缺点,但是,他们的革命大方向始终是正确的。这是无产阶级文化大革命的主流。无产阶级文化大革命正在沿着这个大方向继续前进。

文化革命既然是革命,就不可避免地会有阻力。这种阻力,主要来自那些混进党内的走资本主义道路的当权派,同时也来自旧的社会习惯势力。这种阻力目前还是相当大的,顽强的。但是,无产阶级文化大革命毕竟是大势所趋,不可阻挡。大量事实说明,只要群众充分发动起来了,这种阻力就会迅速被冲垮。

由于阻力比较大,斗争会有反复,甚至可能有多次反复。这种反复,没有什么害处。它将使无产阶级和其它劳动群众,特别是年轻一代,得到锻炼,取得经验教训,懂得革命的道路是曲折的,不平坦的。

\section{三、“敢”字当头,放手发动群众}

党的领导敢不敢放手发动群众,将决定这场文化大革命的命运。目前党的各级组织,对文化革命运动的领导,存在着四种情况。

(一)能够站在运动的最前面,敢于放手发动群众。他们是“敢”字当头、无所畏惧的共产主义战士,是毛主席的好学生。他们提倡大字报、大辩论,鼓励群众揭露一切牛鬼蛇神,同时也鼓励群众批评自己工作中的缺点和错误。这种正确领导就是由于突出无产阶级政治,由于毛泽东思想领先。

(二)有许多单位的负责人,对于这场伟大的斗争的领导,还很不理解,很不认真,很不得力,因而处于软弱无能的地位。他们是“怕”字当头,墨守旧的章法,不愿意打破常规,不求进取。对于群众的革命新秩序,他们感到突然,以致领导落后于形势,落后于群众。

(三)有些单位的负责人,平时有这样那样的错误,他们更是“怕”字当头,怕群众起来抓住他们的辫子。实际上,他们只要认真进行自我批评,接受群众批评,是会被党和群众谅解的。不这样做,就会继续犯错误,以致成为群众运动的绊脚石。

(四)有些单位是被一些混进党内的走资本主义道路的当权派把持着。这些当权派极端害怕群众揭露他们,因而找各种借口压制群众运动。他们采用转移目标、颠倒黑白的手段,企图把运动引向斜路。当他们感到非常孤立,真混不下去的时候,还进一步耍阴谋,放暗箭,造谣言,极力混淆革命和反革命的界限,打击革命派。

党中央对各级党委的要求,就是要坚持正确的领导,“敢”字当头,放手发动群众,改变那种处于软弱无能的状态,鼓励那些有错误而愿意改正的同志放下包袱,参加战斗,撤换那些走资本主义道路的当权派,把那里的领导权夺回到无产阶级革命派手中。

\section{四、让群众在运动中自己教育自己}

无产阶级文化大革命,只能是群众自己解放自己,不能采用任何包办代替的办法。

要信任群众,依靠群众,尊重群众的首创精神。要去掉“怕”字。不要怕出乱子。毛主席经常告诉我们,革命不能那样雅致,那样文质彬彬、温良恭俭让。要让群众在这个大革命运动中,自己教育自己,去识别哪些是对的,哪些是错的,哪些作法是正确的,哪些作法是不正确的。

要充分运用大字报、大辩论这些形式,进行大鸣大放,以便群众阐明正确的观点,批判错误的意见,揭露一切牛鬼蛇神。这样,才能使广大群众在斗争中提高觉悟,增长才干,辨别是非,分清敌我。

\section{五、坚决执行党的阶级路线}

谁是我们的敌人?谁是我们的朋友?这个问题是革命的首要问题。党的领导要善于发现左派,发展和壮大左派队伍,坚决依靠革命的左派。这样,才能够在运动中,彻底孤立最反动的右派,争取中间派,团结大多数,经过运动,最后达到团结百分之九十五以上的干部,团结百分之九十五以上的群众。

集中力量打击一小撮极端反动的资产阶级右派分子、反革命修正主义分子,充分地揭露和批判他们的反党反社会主义反毛泽东思想的罪行,把他们最大限度地孤立起来。

这次运动的重点,是整党内那些走资本主义道路的当权派。

注意把反党反社会主义的右派分子,同拥护党和社会主义,但也说过一些错话,作过一些错事或写过一些不好文章不好作品的人,严格区别开来。

注意把资产阶级的反动学阀、反动“权威”,同具有一般的资产阶级学术思想的人,严格区别开来。

\section{六、正确处理人民内部矛盾}

必须严格分别两类不同性质的矛盾:是人民内部矛盾,还是敌我矛盾?不要把人民内部矛盾搞成敌我矛盾,也不要把敌我矛盾搞成人民内部矛盾。

人民群众中有不同意见,这是正常现象。几种不同意见的争论,是不可避免的,是必要的,是有益的。群众会在正常的充分的辩论中,肯定正确,改正错误,逐步取得一致。

在辩论中,必须采取摆事实、讲道理、以理服人的方法。对于持有不同意见的少数人,也不准采取任何压服的办法。要保护少数,因为有时真理在少数人手里。即使少数人的意见是错误的,也允许他们申辩,允许他们保留自己的意见。

在进行辩论的时候,要用文斗,不用武斗。

在辩论中,每个革命者都要善于独立思考,发扬敢想、敢说、敢做的共产主义风格。革命的同志,在大方向一致的前提下,不要在枝节问题上争论不休,以便加强团结。

\section{七、警惕有人把革命群众打成“反革命”}

有些学校、有些单位、有些工作组的负责人,对给他们贴大字报的群众,组织反击,甚至提出所谓反对本单位或工作组领导人就是反对党中央,就是反党反社会主义,就是反革命等类口号。他们这样做,必然要打击到一些真正革命的积极分子。这是方向的错误,路线的错误,决不允许这样做。

有些有严重错误思想的人们,甚至有些反党反社会主义的右派分子,利用群众运动中的某些缺点和错误,散布流言蜚语,进行煽动,故意把一些群众打成“反革命”。要谨防扒手,及时揭穿他们耍弄的这套把戏。

在运动中,除了确有证据的杀人、放火、放毒、破坏、盗窃国家机密等现行反革命分子,应当依法处理外,大学、专科学校、中学和小学学生中的问题,一律不整。为了防止转移斗争的主要目标,不许用任何借口,去挑动群众斗争群众,挑动学生斗争学生,即使是真正的右派分子,也要放到运动的后期酌情处理。

\section{八、干部问题}

干部大致可分为以下四种:

(一)好的。

(二)比较好的。

(三)有严重错误,但还不是反党反社会主义的右派分子。

(四)少量的反党反社会主义的右派分子。

在一般情况下,前两种人(好的,比较好的)是大多数。对反党反社会主义的右派分子,要充分揭露,要斗倒,斗垮,斗臭,肃清他们的影响,同时给以出路,让他们重新做人。

\section{九、文化革命小组、文化革命委员会、文化革命代表大会}

无产阶级文化大革命运动中,开始涌现了许多新事物。在许多学校、许多单位,群众所创造的文化革命小组、文化革命委员会等组织形式,就是一种有伟大历史意义的新事物。

文化革命小组、文化革命委员会、文化革命代表大会是群众在共产党领导下自己教育自己的最好的新组织形式。它是我们党同群众密切联系的最好的桥梁。它是无产阶级文化大革命的权力机构。

无产阶级同过去几千年来一切剥削阶级遗留下来的旧思想、旧文化、旧风俗、旧习惯的斗争需要经历很长很长的时期。因此,文化革命小组、文化革命委员会、文化革命代表大会不应当是临时性的组织,而应当是长期的常设的群众组织。它不但适用于学校、机关,也基本上适用于工矿企业、街道、农村。

文化革命小组、文化革命委员会和文化革命代表大会的代表的产生,要象巴黎公社那样,必须实行全面的选举制。候选名单,要由革命群众充分酝酿提出来,在经过群众反复讨论后进行选举。

当选的文化革命小组、文化革命委员会和文化革命代表大会的代表,可以由群众随时提出批评,如果不称职,经过群众讨论,可以改选、撤换。

在学校中,文化革命小组、文化革命委员会、文化革命代表大会,应该以革命学生为主体,同时,要有一定数量的革命教师职工的代表参加。

\section{十、教学改革}

改革旧的教育制度,改革旧的教育方针和方法,是这场无产阶级文化大革命的一个极其重要的任务。

在这场文化大革命中,必须彻底改变资产阶级知识分子统治我们学校的现象。

在各类学校中,必须彻底贯彻执行毛泽东同志提出的教育为无产阶级政治服务、教育与生产劳动相结合的方针,使受教育者在德育、智育、体育几方面都得到发展,成为有社会主义觉悟的有文化的劳动者。

学制要缩短。课程设置要精简。教材要彻底改革,有的首先删繁就简。学生以学为主,兼学别样。也就是不但要学文,也要学工、学农、学军,也要随时参加批判资产阶级的文化革命的斗争。

\section{十一、报刊上点名批判的问题}

在进行文化革命群众运动的时候,必须把对无产阶级世界观的传播,对马克思列宁主义、毛泽东思想的传播,同对资产阶级和封建阶级的思想批判很好地结合起来。

要组织对那些有代表性的混进党内的资产阶级代表人物和资产阶级的反动学术“权威”,进行批判,其中包括对哲学、历史学、政治经济学、教育学、文艺作品、文艺理论、自然科学理论等战线上的各种反动观点的批判。

在报刊上点名批判,应当经过同级党委讨论,有的要报上级党委批准。

\section{十二、关于科学家、技术人员和一般工作人员的政策}

对于科学家、技术人员和一般工作人员,只要他们是爱国的,是积极工作的,是不反党反社会主义的,是不里通外国的,在这次运动中,都应该继续采取团结、批评、团结的方针。对于有贡献的科学家和科学技术人员,应该加以保护。对他们的世界观和作风,可以帮助他们逐步改造。

\section{十三、同城乡社会主义教育运动相结合的部署问题}

大中城市的文化教育单位和党的领导机关,是当前无产阶级文化大革命运动的重点。

文革使城乡社会主义教育运动更加丰富、更加提高了。必须把两者结合起来进行。各地区、各部门可以根据具体情况进行部署。

在农村和城市企业进行社会主义教育运动的地方,如果原来的部署是合适的,又做得好,就不要打乱它,继续按照原来的部署进行。但是,当前无产阶级文化大革命运动提出的问题,应当在适当的时机,交给群众讨论,以便进一步大兴无产阶级思想,大灭资产阶级的思想。

有的地方,以无产阶级文化大革命为中心,带动社会主义教育运动,清政治、清思想、清组织,清经济。这样做,如果那里党委认为合适,也是可以的。

\section{十四、抓革命,促生产}

无产阶级文化大革命,就是为的要使人的思想革命化,因而使各项工作做得更多、更快、更好、更省。只要充分发动群众,妥善安排,就能够保证文化革命和生产两不误,保证各项工作的高质量。

无产阶级文化大革命是使我国社会生产力发展的一个强大的推动力。把文化大革命同发展生产对立起来,这种看法是不对的。

\section{十五、部队}

部队的文化革命运动和社会主义教育运动,按照中央军委和总政治部的指示进行。

\section{十六、毛泽东思想是无产阶级文化大革命的行动指南}

在无产阶级文化大革命中,要高举毛泽东思想的伟大红旗,实行无产阶级政治挂帅,要在广大工农兵、广大干部和广大知识分子中,开展活学活用毛主席著作的运动,把毛泽东思想作为文化革命的行动指南。

各级党委,在这样错综复杂的文化大革命中,更必须认真地活学活用毛主席著作。特别是要反复学习毛主席有关文化革命和党的领导方法的著作,例如,《新民主主义论》、《在延安文艺座谈会上的讲话》、《关于正确处理人民内部矛盾的问题》、《在中国共产党全国宣传工作会议上的讲话》、《关于领导方法的若干问题》、《党委会的工作方法》。

各级党委,要遵守毛主席历来的指示,贯彻执行从群众中来、到群众中去的群众路线,先做学生,后做先生。要努力避免片面性和局限性。要提倡唯物辩证法,反对形而上学和烦琐哲学。

在以毛泽东同志为首的党中央领导下,无产阶级文化大革命必将取得伟大的胜利。
