
\title{在庐山会议上的多次讲话、批语}
\date{一九五九年七月十日、二十三日、二十六日、八月二日、十六日}
\maketitle


\date{一九五九年七月十日}
\section{一、统一认识,搞好团结,承认错误}

团结问题。

对形势的认识不一致,就不能团结,要党内团结,首先要把问题搞清楚,要思想统一。有些同志对形势缺乏全面分析,要帮助他们认识,得的是什么,失的是什么。

要把问题搞清楚,有人说总路线根本不对,所谓总路线,无非是多快好省,多快好省根本不会错。

我们把道理讲清楚,把问题摆开,总可以有百分之七十的人在总路线下面。

要承认缺点错误,从一个局部来讲,从一个问题来讲,可能是十个指头,九个指头,七个指头,或者三个指头、两个指头,但是从全局来讲,只是一个指头的问题,从总的形势来讲,就是这样,九个指头和一个指头。

我总是同外国同志说,请他们隔十年时间再来看看我们是否正确。因为路线的正确与否,是实践的问题,要有时间,从实践的结果来证明,我们对建设说应该还没有经验,至少还要十年。这一年来的会议,我们总是把问题加以分析,加以解决,坚持真理,修正错误。党内有些同志不了解整个形势,要向他们说明:从某些具体事实说来,确实有得不偿失的,但总的来说,不能说得不偿失。取得经验总是要付学费的。

\date{一九五九年 七月二十三日}
\section{二、反对两种倾向,维护党的团结\mnote{1}}

你们讲了那么多,允许我讲点吧,可以不可以?吃了三次安眠药,睡不着。

讲点这样的意见。我看了同志们的记录、发言、文件,并和一部分同志们谈了话。我感觉到有两种倾向,这里讲一讲。

一种是触不得,大有一触即发之势。吴稚晖\mnote{2}说:孙科一触即跳\mnote{3}。因此有部分同志感到有压力,即是不愿人家讲坏话,只愿人家讲好话,不愿听坏话。我劝这些同志要听。话有三种,嘴有两用。人有一个嘴巴:一曰吃饭,二曰讲话之义务。长一对耳朵就要听。他要讲,你有什么办法?有一部分同志就是不爱听坏话。好话坏话都是话,都要听。话有三种,一是正确的,二是基本正确的或不甚正确的,三是基本不正确或不正确的。两头是对立的。正确与不正确是对立的。

现在党内外夹攻我们。右派讲:秦始皇为什么倒台?就是因为修长城。现在我们修天安门,要垮台了,这是右派讲的。党内一部分意见还没有讲完,集中表现在江西党校的反映,各地都有。所有右派的言论都拿出来了,龙云、陈铭枢、罗隆基、章伯钧为代表\mnote{4}。江西党校是党内的代表\mnote{5},有些人就是右派、动摇分子。他们看得不完全,做点工作可以转变过来。有些人历史上有问题,挨过批评,也认为一塌糊涂,如广州军区的材料\mnote{6}。这些话都是会外的讲话。我们是会内外结合。可惜庐山地方太小,不能把他们都请来。像江西党校(的人),罗隆基,陈铭枢,这是江西人的责任,房子太小吆!

不分什么话,无非是讲得一塌糊涂。这很好,越讲得一塌糊涂越好,越要听。我们在整风中创造了“硬着头皮顶住”这样一个名词。我和有些同志讲过,要顶住,硬着头皮顶住。顶多久?一个月、三个月,半年、一年、三年、五年、十年、八年,我们有的同志说“持久战”,我很赞成。这种同志占多数。

在座诸公,你们都有耳朵,听嘛!难听是难听,欢迎!你这么一想就不难听了!为什么要让人家讲呢?其原因,神州不会陆沉,天不会塌下来。因为我们做了些好事,腰杆子硬。我们多数同志腰杆子要硬起来。为什么不硬?无非是一个时期蔬菜太少,头发卡子太少,没有肥皂,比例失调,市场紧张,以致搞得人心紧张。我看没有什么紧张的。我也紧张,说不紧张是假的。上半夜你紧张紧张,下半夜安眠药一吃就不紧张了。

说我们脱离了群众,其实群众还是拥护我们的。我看困难是暂时的,就是三个月。春节前后,我看群众和我们结合得很好。小资产阶级狂热性有那么一点,但是并不那么多。我同意同志们的意见,问题是公社运动,我到遂平详细地谈了两个多钟头,嵖岈山公社党委书记告诉我,七、八、九三个月,平均每天三千人参观,十天三万人才,三个月三十万人。徐水、七里营听说也有这么多人参观,除了西藏都来看了。唐僧取经嘛。这些人都是县、社、队干部,也有地专干部。他们的想法是:河南人河北人创造了经验,打破了罗斯福免于贫困的“自由”\mnote{7}。搞共产主义,这股热情,怎么看法?小资产阶级狂热性吗?我看不能那么说,要想多一点,无非是想多一点。这种分析是否恰当?三个月当中,三十万朝山进香,这种广泛的群众运动,不能泼冷水,只能劝说:“同志们,你们的心是好的,事情难以办到,不能性急,要有步骤。吃肉只能一口一口地吃,要一口吃成个胖子不行”。林彪一天吃一斤肉还不胖,十年也不行。总司令和我的胖并非一朝一夕之功。这些干部率领几亿人民,至少百分之三十是积极分子,百分之三十是消积分子及地、富、反、坏、官僚、中农和部分贫农,百分之四十随大流。百分之三十是多少人,一亿几千万人。他们要求办公社,办食堂,搞大协作,非常积极。他们愿搞,你能说这是小资产阶级狂热性吗?这不是小资产阶级,是贫农、下中农、无产阶级,半无产阶级。随大流者也可以,不愿搞的百分之三十,总之百分之三十加百分之四十为百分之七十,三亿五千万人在一个时期有狂热性,他们要搞。到春节前后有两个多月,他们不高兴了,变了,干部下乡都不讲话了,请吃地瓜、稀饭,面无笑容,这叫刮“共产风”(造成的后果)。对刮“共产风”也要分析,其中有小资产阶级狂热性,这是什么人?“共产风”主要是县社两级干部,特别是公社一部分干部,刮生产队和小队的“共产风”,这是不好的,群众不欢迎。坚决纠正,说服他们,用一个月的功夫,三、四月间把风压下去,该退的退,社与队的账清了。这一个月的算账教育是有好处的,极短的时间使他懂得平均主义不行,“一平二调三提款”是不行的。听说现在大多数人转过来了,只有一部分人还留恋“共产风”还舍不得。哪里找这样一个大学校,短期训练班,使几亿人几百万干部受到教育。东西要交回,不能说你的就是我的,拿起就走了。从古以来,没有这个规矩,一万年以后也不能拿起就走。有,只有青红帮,青偷红劫,明火执仗,无代价地削剥人家劳动,破坏等价交换。宋江的政府叫“忠义堂”,劫富济贫,理直气壮,可以拿起就走,拿的是土豪劣绅的,那个章程,我看可以的。宋江\mnote{8}劫的是“生辰纲”,就是我们打土豪劫的是不义之财,“劫之无碍”,刮自农民,归到农民。我们已长期不打土豪了,打土豪,分田地归公,那也可以,因为那也是不义之财。我们刮共产风,取生产队、小队之财,肥猪、大白菜拿起就走,这样是错误的。我们对帝国主义财产还有三种办法:征购、购买、挤垮,怎么能剥削劳动人民的财产呢?为什么一个多月就熄下这股风呢?证明我们的党是伟大的,光荣的、正确的。三、四月份和五月有几百万干部和几亿农民受到教育,讲清了,他们想通了。主要是干部,不懂得这个财是义财,分不清界线,没有读好政治经济学,未搞通价值法则,等价交换,按劳分配,几个月就说通了,不办了。

十分搞通的,未必有,几分通,七、八分通,教科书还没懂,叫他们读,公社一级不懂点政治经济学是不行的。不识字,可以讲。通几分,可以不读书,用事实来教育。梁武帝有个宰相陈发之,一字不识,强迫他作诗,他口念叫别人写,他说有些读书人,还不如老夫的用耳学。当然我不反对扫文盲,柯老说全民进大学,我也赞成,不过十五年得延长。还有南北朝有个姓曹的将军,打了仗以后要作诗:“出师儿女悲,归来笳鼓霓,借问过路人,何为霍去病”。还有北朝斛律金敕勒歌“敕勒川,阴山下,天似穷庐,笼盖四野。天苍苍,野茫茫,风吹草低见牛羊”。这也是一字不识的人。一字不识的人可以当宰相,为什么我们的公社干部,农民不可以听政治经济学呢?我看大家可以学,讲讲,政治经济学不识字可以讲,讲讲就懂了。他们比知识分子容易懂。教科书我就没有看,略为看了一点,才有发言权,要挤出时间,全党来个学习运动。

我们不晓得作了多少次检查了。从去年郑州会议以来,大作特作,有六级会议、五级会议,都要作检查。北京来的人哇啦哇啦,他们就听不进去,我们检讨多次,你们就没有听到?我就劝这些同志,人家有嘴巴么!要人家讲么!要听听人家的意见。我看这次会议有些问题不解决,有些人不会放弃他们的观点,无非拖着么!一年、二年、三年、五年,听不得怪话不行,要养成习惯。我说就是硬着头皮顶住呵!无非是骂祖宗三代。这也难,我少年、中学时代,也是一听到坏话就一肚子气,人不犯我,我不犯人,人若犯我,我必犯人,人先犯我,我后犯人,这个原则,现在也不放弃。现在学会了听,硬着头皮顶住,听他一个、两个星期,再反击。劝同志们要听,你们赞成不赞成是你们的事,不赞成,如我错,我作自我批评。

第二方面,我劝另外一部分同志,在这样的紧急关头,不要动摇,据我观察,有一部分同志是动摇的。他们也说大跃进、总路线、人民公社都是正确的,但要看讲话的思想方面站在那一边?向那方面讲,这部分人是第二种人,“基本正确,部分不正确”的这一类人,但有些动摇。有些人在关键时就是动摇的,在历次大风大浪中就是不坚决的。历史上有四条路线,陈独秀路线、立三路线、王明路线、高饶路线,现在是一条总路线,在大风浪时,有些同志站不稳,扭秧歌。蒋帮不是叫我们做秧歌王朝吗?这部分同志扭秧歌,他们忧心如焚,想把国家搞好,这是好的。这叫什么阶级呢?资产阶级还是小资产阶级?我现在不讲。在南宁会议,成都会议,党代大会讲过,一九五六年、一九五七年的动摇,不戴高帽子,讲成思想方法问题。如果讲小资产阶级的狂热性,反过来讲,那时的反冒进,就是资产阶级的冷冷清清,惨惨戚戚的泄气性,悲观性。我们不戴高帽子,因为这些同志和右派不同,他们也搞社会主义,只不过是没有经验,一有风吹草动就站不住脚,就反冒进。那次反冒进的人这次站住脚了。如恩来同志劲很大,受过那次教训,相信陈云同志也会站住脚的。恰恰是那次批判恩来同志的,他们那一部分人这次取他们的地位而代之。不讲冒进了,可是有反冒进的味道。比如说:“有失有得”\mnote{9}。“得”放往后面是经过斟酌了的,如果戴高帽子,这是资产阶级动摇性,或降一等是小资产阶级动摇性。因为右的性质,往往受资产阶级的影响,在帝国主义、资产阶级压力下右起来了。

一个生产队一条错误,七十几万个生产队七十几万条错误都登出来,一年登到头,登得完登不完?还有文章长短,我看至少要一年,这样结果如何?我们的国家就垮台了,那时候帝国主义不来,国内人民也会起来把我们统统打倒。你办那个报纸天天登坏事,无心工作,不要说一年,就是一个星期也要灭亡的。登七十万条,专登坏事,那就不是无产阶级了,那就是资产阶级国家了,就是资产阶级的章伯均的设计院了,当然在座没有人这样主张,我是用夸大说法。假如办十件事,九件是坏的,都登在报刊上,一定灭亡,应当灭亡,那我就走,到农村去,率领农民推翻政府,你解放军不跟我走,我就找红军去。我看解放军会跟我走的。

我劝一部分同志讲话的方向要注意,不过讲话的内容基本正确,部分不妥。要别人坚定,首先自己坚定,要别人不动摇,首先自己不动摇。这又是一次教训。这些同志据我看不是右派是中间派,不是左派(不加引号的左派)\mnote{10}。我所谓方向,是因为一些人碰了一些钉子了,头破血流,忧心如焚,站不住脚,动摇了,站到中间去了,究竟中间偏左偏右,还要分析。重复五六年下半年、五七上半年犯错误同志的道路,他们不是右派,可是自己把自己抛到右派边缘去了,距右派还有三十公里,因为右派很欢迎这个论凋,现在有些同志的论调,右派不欢迎才怪。这种同志采取边缘政策,相当危险,不相信,将来看。这些话是在大庭广众当中讲的,有些伤人,现在不讲,对这些同志不利。

我出的题目中加一个题目,团结问题。还是单独写一段,拿着团结的旗帜,人民的团结,民族的团结,党的团结。我不讲对这些同志是有益是有害?有害!所以还是要讲。我们是马克思主义政党。第一方面的人要听人家讲,第二方面的人也要听人家讲。两方面的人都要听人家讲,我说还是要讲吗?一条是要讲,一条是要听人家讲。我不忙讲,硬着头皮顶住,我为什么现在不硬着头皮顶了呢?顶了二十多天,快散会了,索性开到月底。马歇尔\mnote{11}八上庐山,蒋介石三上庐山,我们一上庐山,为什么不可以?有此权利。

食堂问题,食堂是个好东西,无可厚非,我赞成积极办好。自愿参加,粮食到户,节约归己。我看在全国保持三分之一我就满意了,一讲这个,吴芝圃\mnote{12}就紧张了,不要怕,河南等省有百分之五十的食堂还在,那也可以试试看,不要搞掉,我是就全国范围来讲的。三分之一农民,一亿五千万坚持下去就了不起了。第二个希望,一半左右,二亿五千万,多几个河南、四川、湖南、云南、上海等等。取得经验,有些散了,还得恢复,食堂并不是我们发明的,是群众创造的,河北一九五六年公社化以前就有办的,一九五八年办得很快,曾希圣\mnote{13}说:食堂节省劳动力,我看还有一条,节省物质,如果没有后面这一条,就不能持久。可否办到?可以办到。我建议河南同志把一套机械化搞起来,比如自来水,搞个东西不用挑,这样一来可以节省劳力,可以省物质,现在散掉一半左右有好处。总司令我赞同你的说法,但又和你的说法有区别。不可不散,不可全散。我是个中间派,河南、四川、湖北等是左派,可是有个右派出来了,科学院昌黎调查组说食堂没有一点好处,攻其一点,不及其余,学“登徒子好色赋”的办法。登徒子攻宋玉三条:漂亮,好色、会说话,不能到后宫去,很危险。宋玉反驳说:“漂亮是父母所生,会说话是先生所教,好色无此事。天下佳人不如楚,楚国出丽者,莫若臣里,臣里之美者,莫若臣东家之子,增一分过长,减一分过短。……”登徒子是大夫,大夫就是今天的部长,是大部,如冶金部长,煤炭部长,还有什么农业部长,科学院调查组是攻其一点,不及其余,攻其一点的办法,无非是猪肉、头发卡子(少了)。食堂哪能没有缺点?无论什么人都有缺点,孔夫子也有错误,我也看过列宁的手稿,改得一塌糊涂,没有错误,为什么要改?食堂可以多一点,再试试看,试它一年、二年,估计可以办成。人民公社会不会垮台?现在没有垮一个,准备垮一半、垮七分,还有三分,要垮就垮,办得不好,一定要垮。共产党就是要办好,办好公社,办好一切事业,办好农业,办好工业,办好交通运输,办好商业,办好文化教育。

许多事情根本料不到,以前不是说党不管党吗?现在计划机关不管计划,至少是一个时期不管计划。计划机关不只是计委,还有其它各部,还有地方,一个时期不管综合平衡。地方可以原谅,计委同中央各部十年了,忽然在北戴河会议后开始不管了,名曰计划指标,等于不要计划,所谓不管计划,就是不要综合平衡,根本不去算要多少煤,要多少铁,要多少交通,煤铁不能自己走路,要车马运,这点没料到。我这样的人,少奇、总理这样的人,根本没有管,或者略略一管。我不是开脱也是开脱,因为我不是计委主任,去年八月以前,主要精力放在革命方面,对建设根本外行,对工业计划一点不懂,在西楼(中南海西楼)时曾经说过不要写“英明领导”,管都没管还说什么英明?但是,同志们,一九五八、一九五九主要责任在我身上,过去责任在别人,过去说总理、陈云,现在应该说我,实在有一大堆事没管。大办钢铁的发明权是柯庆施\mnote{14}还是我?我说是我,我和柯庆施谈过一次话,你说要搞六百万吨(整个华东经济协作区)。以后我找大家谈话,有王鹤寿\mnote{15}也觉得可行,我六月讲一〇七〇万吨,后来去做,薄一波\mnote{16}建议搞在北戴河公报上,我觉得可行,从此闯下大祸,九千万人上阵,补贴四十亿,搞小土群、小洋群,“得不偿失”,“得失相当”等等说法,即由此而来。

看了很多讨论,大家讲还可以搞(小土群),铁还可以炼,要提高质量,降低成本,降低硫的成分,要为出真正好铁而努力奋斗。只要抓,也有可能。共产党有个方法叫抓,共产党和蒋介石都有两只手,共产党的手是共产主义者的手,一抓就抓起来了。钢铁要抓,粮油、棉、麻、丝麻、糖、药,还有烟果盐,农、林、牧、付、渔有十二项要抓,要综合平衡,各地不同,不能每县都一个模子。有些地方不长茶,不长甘庶,要因地制宜。苏联不是搞过回民地区养猪么,岂有此理?工业计划搞了一篇文章,写得还好。至于党不管党,计划机关不管计划,不搞综合平衡,搞什么去了?总理着急他们却根本不急。人不着急就没有一股神气,没有一股热情就办不好事情。

有人批评计委李富春\mnote{17}是“足将进而趑趄,口将言而嗫嚅”也,不要像李逵,太急了也不行,列宁热情磅礴实在好,群众很欢迎。口将言而嗫嚅,无非有各种顾虑,这个我看要改,有话就要讲,上半月顾虑甚多,现在展开了,有话讲出来了,记录为证,口说无凭,立此存照。你们有话讲出来嘛!有话就讲出来嘛,你们抓住,就整我嘛。成都会议上我说过不要怕穿小鞋,穿小鞋有什么要紧。还讲过几条,甚至说不要怕坐班房,不要怕杀头,不要怕开除党籍,一个共产党员,高级干部,那么多的顾虑,有些人就是怕讲得不妥挨整,这叫“明哲保身”啊!病从口入,祸从口出,我今天要闯祸,两种人都不高兴我,一种是触不得听不得坏话的,一种是方向有点问题的。如果你们不赞成,你们就驳,说主席不能驳,我看不对,事实上已经纷纷在驳,不过不指名,江西党校,中央党校一些意见就是驳,说“始作俑者,其无后乎”。一个是一〇七〇万吨钢。一〇七〇万吨钢是我建议,我下的决心,其结果是九千万人上阵,四十亿人民币,“得不偿失”。第二个是人民公社,人民公社我无发明之权,有建议之权。北戴河决议是我建议写的,当时嵖岈山章程如获至宝。你讲我是小资产阶级狂热性,也是有一点,不然为什么如获至宝呢?要上《红旗》杂志呢?我在山东,一个记者问我,“人民公社好不好?”我说“好”,他就登了报。这个没关系,你登也好,不登也好,到北戴河我提议要作决议的。小资产阶级狂热性有一点,以后新闻记者要离开(远点)。

我有两条罪状,一条叫一〇七〇万吨,大炼钢铁,你们赞成也可以给我分一点,但是始作俑者是我,推不掉,主要责任是我,人民公社,全世界反对,苏联也反对,还有总路线是虚的实的,你们分一点,见之于行动是工业、农业。至于其他一些大炮,别人也要分担一点,你们那大炮也相当多,放的不准心血来潮,不谨慎,共产共的快。在河南讲起,江苏、浙江的记录传得快,说话不谨慎,把握不大,要谨慎一点。长处是一股干劲,肯负责任。比那凄凄惨惨戚戚要好,但放大炮在重大问题要慎重,我也放了三大炮,公社,炼钢,总路线。彭德怀说他粗中无细,我是张飞粗中有点细。人民公社我说集体所有制。我说集体所有制到共产主义全民所有制的过程,两个五年计划太短了一些,也许要二十个五年计划。

要快之事,马克思也犯过不少错误。我搬出马克思来,使同志们得到一点安慰。这个马克思,天天想革命快,一见形势来了就说欧洲革命来了,无产阶级革命来了,后头又没有来;过一阵又说要来,又没有来。总之,反反复复。马克思死了好多年,列宁时代才来,那还不是急性病?小资产阶级狂热性?马克思也有呵!巴黎公社起义爆发之后,马克思就赞成了,但他估计会失败。他看出这是第一个无产阶级专政,哪怕只存在三个月也好。要讲经济核算的话,划不来。我们还有广州公社,一九二七年大革命失败,等等。我们现在的经济工作,是否会像一九二七年那样失败?像万里长征那样,大部分根据地丧失,红军和党都缩小到十分之一,或者还不到?我看不能这样讲,大家也是这么个意见,参加庐山会议的同志都豪无例外地说有所得,没有完全失败。是否大部分失败了?我看也不能讲。大部分没有失败,一部分失败了。就是所谓多付了代价、多用点劳力、多付一点钱、刮了一次“共产风”,可是全国人民受了教育,清醒了。

现在要研究政治经济学,过去谁人去读政治经济学教科书?我就不读。斯大林的书(按指《苏联社会主义经济问题》),我读了一遍,根本没有味道。那个时候搞革命,搞什么社会主义经济。唉,一到郑州,我就读了两遍,我就讲学,就有资格讲学了,不过刚刚在火车上读了两遍,我讲了两章,没有造谣吧!现在不够,现在要深入研究,不然我们的事业不能发展、不能够巩固、不能够前进。

如讲责任,李富春、王鹤寿有点责任。农业部谭老板\mnote{18}有点责任,第一个责任是我。柯老,你的发明权有没有责任?(柯老:有)是否比较轻?你那是意识形态问题。我是一个一〇七〇万吨钢,九千万人上阵,这个乱子就闹大了,自己负责。同志们自己的责任都要分析一下,有屎拉出来,有屁放出来,肚子就舒服了。

\date{一九五九年七月二十六}
\section{三、统一认识,团结同志,改善工作}

收到一封信\mnote{19},是一个有代表性的文件。信的作者在我们的经济工作搜集了一些材料,这些材料专门属于缺点方面的。作者只对这一方面的材料有兴趣;而对另一方面的材料,成绩方面的材料,可以说根本不发生兴趣。他认为,从一九五八年第四季度以来,党的工作中缺点错误是主流。因此做出结论说,党犯了“左倾冒险主义”,“机会主义”的错误,而其根源在于一九五七年整风反右斗争没有“同时”反对左倾冒险主义的危险。作者李云仲同志(他是国家计委一个副局长,不久前调任东北协作区委员会办公厅综合组组长)的基本观点是错误的。他几乎否定一切。他认为几千万人上阵大炼钢铁损失极大而毫无效益,人民公社也是错误的,对基本建设极为悲观,对农业他提到水利,认为党的“左倾冒险主义、机会主义”错误是办水利引起的,他对前冬去春几亿农民在党的领导下大办水利没有好评。他是一个“得不偿失”论者,某些地方简直是“有失无得”论者。作者的这些结论性的观点放在第一段,篇幅不多。这个同志的好处是把自己的思想和盘托出。他跟我们看见的另一些同志,他们对党和人民的工作基本上不是高兴,而是不满,对成绩估计很不足,对缺点估计过高,为现在的困难所吓倒、对干部不是鼓劲而是泄气,对前途信心不足,甚至丧失信心,但是不愿意讲出自己的想法和看法,或者讲一点,留一点,而采取“足将进而趔趄,口将开而嗫嚅”,躲躲闪闪的态度大不相同。

李云仲同志和这些人不同,他不隐瞒自己的政治观点,他满腔热情地写信给中央同志,希望中央采取步骤克服现在的困难。他认为困难是可以克服的,不过时间要长些,这些看法是正确的。信的作者对计划工作中缺点的批评占了大部分篇幅,我认为很中肯。十年来还没有一个愿意和敢于向中央中肯地有分析地系统地揭露我们计划工作的缺点,因而求得改正的同志。我就没有看见这样一个人。我知道这种人是有的,他们就是不敢越衙上告。因此我建议,将此信在中央一级和地方一级(省、市、自治区)共两级的党组织中,特别是计划机关中,予以传阅并且展开辩论,将一九五八年、一九五九年两年自己所做的工作的长短,利害得失,加以正确的分析,以利统一认识,团结同志,改善工作,鼓足干劲,奋勇前进,争取经济工作及其他工作(政治工作、军事工作、文教卫生工作、党的各级组织的领导工作,工、青,妇工作)的新的伟大胜利。党中央从去年十一月第一次郑州会议\mnote{20}以来至此次庐山会议\mnote{21},对于在自己领导下的各项重大工作中的错误缺点在足够地估计成绩(成绩是主要的,缺点是第二位的)的条件下,进行了严肃的批判,这次批判工作已经有九个月了。必须看到,这种批判是完全必要的,而且是迅速地见效和逐步地见效的,又必须看到,这种严肃的认真的批判,必定而且已经带来了一定的付作用,就是对于某些同志有泄气。错误必须批判,泄气必须防止。气可鼓而不可泄。人而无气,不知其可也。我们必须坚持今年三月第二次郑州会议纪录上所说的,在满腔热情地保护干部的精神下,引导那些在工作中犯有错误者,存在缺点者,批判和改正自己的缺点错误。错误并不可怕,就怕不肯批评,不肯改正,就怕因批评而泄了气,必须顾到改错和鼓劲两方面,必须看到批评整改虽然已经几个月了、一切未完工作还必须坚持做完,不可留尾巴。

但是现在党内外出现了一种新的事物,就是右倾情绪、右倾思潮、右倾活动已经增加,大有猖狂进攻之势,这表现在此次会议印发各同志的许多材料上。这种情况还没有达到一九五七年党内外右派猖狂进攻那种程度,但是苗头和趋势已经很清楚,已经出现在地平线上了,这种情况是资产阶级性质的,另一种情况是无产阶级内部的思想性质的,他们和我们一样都要社会主义,不要资本主义,这是我们和这些同志的基本相同点。但是这些同志的观点和我们的观点是有分歧的。他们的情绪有些不正常,他们把党的错误估计得过大了一些,而对几亿人民在党的领导下所创造出来的伟大成绩估计得过小了些,他们做出了不适当的结论,他们对于克服当前的困难信心很不足。他们把他们的位置不自觉地摆得不恰当,摆在左派和右派的中间。他们是典型的中间派。他们是“得失相当”论者。他们在紧要关头不坚定,摇摇摆摆。我们不怕右派猖狂进攻,却怕这些同志的摇摆。因为这种摇摆不利党和人民的团结,不利于全党一致鼓足干劲,克服困难,争取胜利。我们相信,这些同志的态度是可能改变的。我们的任务是团结他们,争取他们改变态度。为要达到此目的,必须对此种党内的动态作必要的估计。不可估计太高,认为他们有力量可以把党和人民的大船在风浪中摇翻。他们没有这样大的力量。他们占相对的少数,而我们则占大多数。我们和人民中的大多数(工人、贫农、下中农、一部分上中农和知识分子)是团结一致的。党的总路线和体现总路线的方针、政策、工作方法,是受到广大党员、广大干部和广大人民群众的欢迎的。但也不可把他们的力量估计过低,他们有相当一些人。他们的错误观点在受到批判、接受批判、端正态度以前,是不会轻易放弃自己的观点的,这一点必须看到。党内遇到大问题有争论表现不同的观点,有些人暂时摇摆,站在中间,有些人站到右边去,是正常的现象,无须大惊小怪。归根结底,错误观点乃至错误路线一定会被克服,大多数的人,包括暂时摇摆,甚至犯路线错误的人,一定会在新的基础上团结起来。我们党三十八年的历史就是这样走过来来的。反右必出“左”,反“左”必出右,这是必然的。时然若言。现在是讲这一点的时候了,不讲对团结不利,于党于个人都不利。现在这一次争论,可能会被证明是一次意义重大的争论,如同我们在革命时期,各次重大争论一样,在新的历史时期——社会主义建设时期,不可能没有争论的,风平浪静的。庐山会议可能被证明是一次意义重大的会议。“团结——批评——团结”,“惩前毖后,治病救人”,是我们解决党内矛盾的正确的已被历史证明的有效方法,我们一定要坚持这种方法。

我的这些意见,大体已在七月二十三日的全体会议上讲了,但有些未讲完,作为那次讲话的补充,又写了这些话。

\date{一九五九年八月二日}
\section{四、中央全会的团结,关系到中国社会主义的命运\mnote{22}}

中委、候补中委一百九十一人,到会一百四十七人,列席十五人,共一百六十五人,会议议程:

改指标问题:武汉六中全会决定了今年的指标,上海七中全会有人主张改指标,多数不同意,看来改也改不彻底,现在还有五个月,改了好经过人大常委会,高指标是自己定的,自己立了个菩萨自己拜,现在还得打破,打破了不符合实际的指标,钢、煤、粮、棉等。

路线问题:有些同志发生怀疑,究竟对不对?上庐山前不清楚,上庐山后有部分人要求民主,要求自由,说不敢讲话,有压力,当时摸不着头脑,不知所说的不民主是为的什么?前半个月是神仙会议,没有紧张局势。他们说没有自由,就是要攻击总路线,破坏总路线,他们要自由,就是破坏总路线的自由,要批评总路线的言论自由,他们要求紧张的局势。以批评去年为主,也批评今年的工作。说去年的工作做坏了,自去年十一月第一次郑州会议以来,纠正了“刮共产风”,纠正了“一平二调三提款”等一些“左”的倾向。他们对于九个月来的工作,看不到,不满意,要求重新议论,否则就认为压制民主。他们对政治局扩大会议嫌不过瘾,说民主少了,现在开全会,民主大些,准备明年开党代表大会。看形势,如需要,今年九、十月开也可以。五七年不是要求大民主,大鸣、大放、大辩论吗?庐山会议已经开了一个月了,新来的同志不知道怎么一回事,先开几天小会,再开大会,最后作决议。

开会的方法,用大家所赞成的方法,从团结的愿望出发。中央全会的团结,关系到中国社会主义的命运。在我们看来,我们应该团结,现在有一种分裂的倾向。去年八大我说过,危险无非是:一、世界大战,二、党分裂,当时还没有显着的迹象,现在有这种迹象了。团结的方法,从团结的愿望出发,经过批评与自我批评,在新的基础上达到新的团结的目的。对犯错误的同志,采取惩前毖后,治病救人的方针,给犯错误的同志一条出路,允许犯错误的同志改正错误,继续革命,不要像“阿Q正传”上的赵太爷,不许阿Q革命。对犯错误的同志要一看二帮,只看不帮,不作工作是不好的。我们反对错误,毒药吃不得,我们不是欣赏错误的臭味。批评斗争他们是使他们离我们近一点,使缺点错误离我们越远越好。对于犯错误的同志要有分析,无非是两种可能,一个是能改,一个是不能改。所谓看,就是看能不能改,所谓帮,就是帮助他改。有些同志一时跟到那边去,经过批评说服,加上客观情况的改变,许多同志改变过来了,又脱离了那些人。立三路线、王明路线,遵义会议上纠正了,以后经过十年时间,一直到七大,中间经过了整风,经过十年是必要的。一个人要改正错误要有几个过程。你强迫一下改正不行。马克思说:“商品是经过千百次交换才认识其两重性的。”洛甫\mnote{23}开始不承认路线错误,七大经过斗争,洛甫承认了路线错误。那场斗争,王明没有改,洛甫也没有改,又旧病复发,他还在发疟疾,一有机会出来了。大多数同志改好了。从路线错误来说,历史事实证明是可以改变的,要有这种信心。不能改的是个别的。可见釆取治病救人的方针是见效的,要有好心帮助他们。对人有情,对错误的东西应当无情的,那是毒药,要有深恶痛绝的态度,但不用武松、鲁智深、李逵的方法。他们很坚决。可以参加共产党,他们的缺点是不大策略,不会作政治工作。要釆取摆事实讲道理的方法,大辩论,大字报,中字报,庐山会议的简报。

\date{一九五九八月十六日年}
\section{五、右倾机会主义者挑起了斗争\mnote{24}}

犯右倾机会主义错误的同志,不在去年十一月郑州会议上提出意见,更不在北戴河会议上对高指标提出意见,也不在去年十二月武昌会议上提出意见,也不在三月底四月初上海会议上提出意见,而在这庐山会议上提出意见。

这些同志为什么不在那个时候提?因为他们的一套,那时提不出,如果他们有一套正确的见解,比我们高明,在北戴河就提嘛!他们等到中央把问题解决了,或者大部分解决了,才来提,认为这时不提就不好提了,因为他们感觉现在不提,再等几个月后,形势要好转,时间过了,就更不好了,故急于发动。

党内斗争反映了社会上的阶级斗争,这是毫不足怪的,没有这种斗争才是不可思议。这个道理过去没有讲透,很多同志不明白,一旦出了问题,例如一九五三年高饶问题,现在的彭、黄、周、张问题,就有许多人感到惊奇,这种惊奇,是可以理解的。因为社会矛盾是由隐到显的。人们对于社会主义时代的阶级斗争的理解,是要通过自己的斗争和实践才会逐步深入的。特别是一些党内斗争,例如高、饶、彭、黄这一类斗争具有曲折复杂的性质。昨日还是“功臣”,今日变成祸首,“怎么搞的,是不是弄错了?”人们不知道他们的历史变化,不知道他们历史的复杂和曲折,这不是很自然的吗?应当逐步地、正确地向同志们说清楚这种复杂和曲折的性质。再则,处理这类事件,不可用简单的方法,不可以把它当作敌我矛盾去处理,而必须把它当作人民内部矛盾去处理。必须采取“团结——批评——团结”,“惩前毖后,治病救人”,“批判从严,处理从宽”,“一曰看,二曰帮”的政策。不但要把他们留在党内,而且要把他们留在省委员会内、中央委员会内,个别同志还应当留在中央政治局内。这样,是否有危险呢?可能有。只要我们采取正确的政策,可能避免。他们的错误,无非是两个可能:第一,改过来;第二,改不过来。改过来的条件是充分的。首先,他们有两面性,一面,革命性,一面,反革命性。直到现在,他们与叛徒陈独秀、罗章龙、张国焘、高岗是有区别的,一是人民内部矛盾,一是敌我矛盾。人民内部矛盾可能转化为敌我矛盾,如果双方采取的态度和政策不适当的话;可能不转化为敌我矛盾,而始终当作人民内部矛盾,予以彻底解决,如果我们能够把这种矛盾及时适当地加以解决的话。下面的这些条件是重要的。全党全民的监督,中央和地方的大多数干部的政治水平,比较一九五三年高、饶事件时期大为提高了,懂事了。庐山会议上这一场成功的斗争,不就是证据吗?还有,我们对待他们的态度和政策,一定要是符合情况的马克思主义的态度和政策,而我们已经有了这样的态度和政策。改不过来的可能性也是存在的。无非是继续捣乱,自取灭亡。那也没有什么不得了。向陈独秀、罗章龙、张国焘、高岗队伍里增加几个成员,何损于我们伟大的党和我们伟大的民族呢?但是,我们相信,一切犯错误的同志,除陈、罗、张、高一类极少数人以外,在一定的条件下,积以时日,总是可以改变的。这一点,我们必须有坚定的信心。我党三十八年的历史提供了充分的证据,这是大家所知道的。为了帮助犯错误的同志改正错误,就要仍然把他们当同志看待,当作兄弟一样看待,给以热忱的帮助,给他们以改正错误的时间和继续从事革命工作的出路。必须留有余地。必须有温暖,必须有春天,不能老留在冬天过日子。我认为,这些都是极为重要的。


\begin{maonote}
\mnitem{1}这是毛泽东同志在庐山举行的中共中央政治局扩大会议全体会议上的讲话。
\mnitem{2}吴稚晖,吴敬恒(一八六五年——一九五三年十月三十日),字稚晖,国民党右翼政客,积极反共。
\mnitem{3}孙科(一八九一年十月二十日——一九七三年九月十三日),字哲生,广东省香山县(今中山市)翠亨村人,孙中山唯一的儿子,其母是孙中山的元配夫人卢慕贞。
\mnitem{4}章伯钧、罗隆基、龙云等,当时发表了许多尖锐的、讽刺性的意见。章伯钧说,一九五八年搞错了,炼钢失败了,食堂办不起来了,大办水利是瞎来。罗隆基说,物资供应紧张是社会制度造成的。私营工商业改造有毛病。现在人民怨愤已达极点。共产党说唯物,实际上最唯心。龙云说,解放后只是整人,人心丧尽。内政还不如台湾。全国干部数量,比蒋介石时代成百倍增加。陈铭枢说,供求相差惊人,几年之内也难恢复正常供应。要是过去发生这种情况,早就该“下诏引咎”了。他们实行的不是列宁主义,而是斯大林主义。于学忠说,共产党的政策忽冷忽热,大跃进的成绩全是假话。天安门的工程,像秦始皇修万里长城。
\mnitem{5}当时有一份文件《江西省中级党校学员对人民公社的各种看法》,一九五九年五月间,当讨论郑州会议、上海会议巩固公社方针时,江西党校八十多个县委一级干部初步鸣放后,对一九五八年大跃进有如下看法:

1.大跃进是吹起来的,是浮夸、谎报的结果;

2.大炼钢铁是劳民伤财,是得不偿失;

3.粮食、副食品供应的紧张,就是农副业没有大跃进的证明。

对公社化运动提出这样一些问题:

1.是“早产儿”,“群众不是自觉入社,是被风刮进来的”;

2.违反了客观必然性,“是根据上级指示人为的产物”;

3.没有高级社优越,“农民只说高级社好,没听说人民公社好”;

4.搞人民公社化根本没有条件,“公社的缺点大于优点,现在是空架子,金字招牌”。
\mnitem{6}广州军区据四十二军政治部报告(这个材料是彭德怀提供的),“少数营团干部对经济生活有抵触情绪”。他们认为经济紧张是全面的,长期不能解决的。有的人讲怪话:“现在除了水和空气以外,其他一切都紧张。”“中国大跃进举世闻名,但我怀疑,市场紧张就是证明。”有人甚至认为我们的事业后退了,说:“一九五六年好,一九五七年较好,一九五八年成问题。”他们否定成立人民公社的必然性和优越性,说“公社成立得太快了,太早了,不合乎规律”。“人民的觉悟没有跟上来,工人、农民和军官都对成立公社有意见”。“苏联建国40年还允许私人有房子,我们建国不到十年,就什么都‘公有化’了”。“公社的优越性是宣传出来的”。他们认为经济生活紧张是由于路线上有错误。说:“去年不仅是工作方法上有问题,而是带有路线性质的错误,中央要负责任。”在少数连排干部中,也有类似情况,有位排长听战士唱《社会主义好》这支歌时,不耐烦地说:“算了,不要唱了,我看这支歌非修改不可。”海南军区一个指导员说:“什么敌人一天天烂下去,我们一天天好起来,我看社会主义建设倒是一年不如一年!”有位排长讲怪话:“在公社劳动,还不如给地主干活,给地主干活有饭吃,还给钱。”讲这些话的人,都有名有姓有职务。
\mnitem{7}美国总统佛兰克林·罗斯福一九四一年在美国国会大厦发表演说时提出的公民有“言论自由、信仰自由、免于贫困及免于恐惧的自由”。
\mnitem{8}宋江,这里是口误,应该指的是晁盖。
\mnitem{9}“有失有得”,见《彭德怀同志的意见书》(一九五九年七月十四日):

\mxname{主席:}

这次庐山会议是重要的。我在西北小组有几次插言,在小组会还没有讲完的一些意见,特写给你作参考。但我这个简单人类似张飞,确有其粗,而无其细。因此,是否有参考价值请斟酌。不妥之处,烦请指示。

甲、一九五八年大跃进的成绩是肯定无疑的。

根据国家计委几个核实后的指标来看,一九五八年较一九五七年工农业总产值增长了百分之四十八点四,其中工业增长了百分之六十六点一,农副业增长了百分之二十五(粮棉增产百分之三十是肯定的),国家财政收入增长了百分之四十三点五。这样的增长速度,是世界各国从未有过的,突破了社会主义建设速度的成规。特别是像我国经济基础薄弱,技术设备落后,通过大跃进,基本上证实了多快好省的总路线是正确的。不仅是我国伟大的成就,在社会主义阵营也将长期地起积极作用。

一九五八年的基本建设,现在看来有些项目是过急过多了一些,分散了一部分资金,推迟了一部分必成项目,这是一个缺点。基本原因是缺乏经验,对这点体会不深,认识过迟。因此,一九五九年就不仅没有把步伐放慢一点,加以适当控制,而且继续大跃进,这就使不平衡现象没有得到及时调整,增加了新的暂时困难。但这些建设,终究是国家建设所需要的,在今后一两年内或者稍许长一点时间,就会逐步收到效益的。现在还有一些缺门和薄弱环节,致使生产不能成套,有些物资缺乏十分必要的储备,使发生了失调现象和出现新的不平衡就难以及时调整,这就是当前困难的所在。因此,在安排明年度(一九六零年)计划时,更应当放在实事求是和稳妥可靠的基础上,加以认真考虑。对一九五八年和一九五九年上半年有些基本建设项目实在无法完成的,也必须下最大决心暂时停止,在这方面必须有所舍,才能有所取,否则严重失调现象将要延长,某些方面的被动局面难以摆脱,将妨碍今后四年赶英和超英的跃进速度。国家计委虽有安排,但因各种原因难予决断。

一九五八年农村公社化,是具有伟大意义的,这不仅使我国农民将彻底摆脱穷困,而且是加速建成社会主义走向共产主义的正确途径。虽然在所有制问题上,曾有一段混乱,具体工作中出现了一些缺点错误,这当然是严重的现象。但是经过武昌、郑州、上海等一系列会议,基本已经得到纠正,混乱情况基本上已经过去,已经逐步地走上按劳分配的正常轨道。

在一九五八年大跃进中,解决了失业问题,在我们这样人口众多的、经济落后的国度里,能够迅速得到解决,不是小事,而是大事。

在全民炼钢铁中,多办了一些小土高炉,浪费了一些资源(物力、财力)和人力,当然是一笔较大损失。但是得到对全国地质作了一次规模巨大的初步普查,培养了不少技术人员,广大干部在这一运动中得到了锻炼和提高。虽然付出了一笔学费(贴补二十余亿)。即在这一方面也是有失有得的。

仅从上述几点来看,成绩确是伟大的。但也有不少深刻的经验教训,认真地加以分析,是必要的有益的。

乙、如何总结工作中的经验教训:

这次会议,到会同志都正在探讨去年以来工作中的经验教训,并且提出了不少有益的意见。通过这次讨论,将会使我们党的工作得到极大好处,变某些方面的被动为主动,进一步体会社会主义经济法则,使经常存在着的不平衡现象,得到及时调整,正确地认识“积极平衡”的意义。

据我看,一九五八年大跃进中所出现的一些缺点错误,有一些是难以避免的。如同我们党三十多年来领导历次革命运动一样,在伟大成绩中总是有缺点的,这是一个问题的两个方面。现时我们在建设工作中所面临的突出矛盾,是一于比例失调而引起各方面的紧张。就其性质看,这种情况的发展已影响到工农之间、城市各阶层之间和农民各阶层之间的关系,因此也是具有政治性的。是关系到我们今后动员广大群众继续实现跃进的关键所在。

过去一个时期工作中所出现的一些缺点错误,原因是多方面的。其客观因素是我们对社会主义建设工作不熟悉,没有完整的经验。对社会主义有计划按比例发展的规律体会不深,对两条腿走路的方针,没有贯彻到各方面的实际工作中去。我们在处理经济建设中的问题时,总还没有像处理炮击金门、平定西藏叛乱等政治问题那样得心应手。另方面,客观形势是我国一穷(还有一部分人吃不饱饭,去年棉布平均每人还只十八尺,可缝一套单衣和两条裤衩)二白的落后状态,人民迫切要求改变现状。其次是国际形势的有利趋势。这些也是促使我们大跃进的重要因素。利用这一有利时机,适应广大人民要求,加速我们的建设工作,尽快改变我们一穷二白的落后面貌,创造更为有利的国际局面,是完全必要和正确的。

过去一个时期,在我们的思想方法和工作作风方面,也暴露出不少值得注意的问题。这主要是:

1.浮夸风气较普遍地滋长起来。去年北戴河会议时,对粮食产量估计过大,造成了一种假象。大家都感到粮食问题已经得到解决,因此就可以腾出手来大搞工业了。在对发展钢铁的认识上,有严重的片面性,没有认真地研究炼钢、轧钢和碎石设备,煤炭、矿石、炼焦设备,坑木来源,运输能力,劳动力增加,购买力扩大,市场商品如何安排等等。总之,是没有必要的平衡计划。这些也同样是犯了不够实事求是的毛病。这恐怕是产生一系列问题的起因。浮夸风气,吹遍各地区各部门,一些不可置信的奇迹也见之于报刊,确使党的威信蒙受重大损失。当时从各方面的报告材料看,共产主义大有很快到来之势,使不少同志的脑子发起热来。在粮棉高产。钢铁加番的浪潮中,铺张浪费就随着发展起来,秋收粗糙,不计成本,把穷日子当富日子过。严重的是相当长的一段时间,不容易得到真实情况,直到武昌会议和今年一月省市委书记会议时,仍然没有全部弄清形势真象。产生这种浮夸风气,是有其社会原因的,值得很好的研究。这也与我们有些工作只有任务指标,而缺乏具体措施是有关系的。虽然主席在去年就已经提示全党要把冲天干劲和科学分析结合起来,和两条腿走路的方针,看来是没有为多数领导同志所领会,我也是不例外的。

2.小资产阶级的狂热性,使我们容易犯左的错误。在一九五八年的大跃进中,我和其他不少同志一样,为大跃进的成绩和群众运动的热情所迷惑,一些左的倾向有了相当程度的发展,总想一步跨进共产主义,抢先思想一度占了上风,把党长期以来历形成的群众路线和实事求是作风置诸脑后了。在思想方法上,往往把战略性的布局和具体措施、长远性的方针和当前步骤、全体与局部、大集体与小集体等关系混淆起来。如主席提出的“少种、高产、多收”、“十五年赶上英国”等号召,都是属于战略件、长远性的方针,我们则缺乏研究,不注意研究当前具体情况,把工作安排在积极而又是稳妥可靠的基础上。有些指标逐级提高,层层加码,把本来需要几年或者十几年才能达到的要求,变成一年或者几个月就要做到的指标。因此就脱离了实际,得不到群众的支持。诸如过早否定等价交换法则,过早提出吃饭不要钱,某些地区认为粮食丰产了,一度取消统销政策,提倡放开肚皮吃饭,以及某些技术不经鉴定就贸然推广,有些经济法则和科学规律轻易被否定等,都是一种左的倾向。在这些同志看来,只要提出政治挂帅,就可以代替一切,忘记了政治挂帅是提高劳动自觉、保证产品数量质量的提高,发挥群众的积极性和创造性,从而加速我们的经济建设。政治挂帅不可能代替经济法则,更不能代替经济工作中的具体措施。政治挂帅与经济工作中的确切有效措施,两者必须并重,不可偏重偏废。纠正这些左的现象,一般要比反掉右倾保守思想还要困难些,这是我们党的历史经验所证明了的。去年下半年,似乎出现了一种空气,注意了反右倾保守思想,而忽略了主观主义左的方面,经过去年冬郑州会议以后一系列措施,一些左的现象基本上纠正过来了,这是一个伟大的胜利。这个胜利既教育了全党同志,又没有损伤同志们的积极性。

观在对国内形势已基本上弄清楚了,特别是经过最近几次会议,党内大多数同志的认识已基本一致。目前的任务,就是全党团结一致,继续努力工作。我觉得,系统地总结一下我们去年下半年以来工作中的成绩和教训,进一步教育全党同志,甚有益处。其目的是要达到明辨是非,提高思想,一般的不去追究个人责任。反之,是不利于团结,不利于事业的。属于对社会主义建设的规律等问题的不熟悉方面,经过去年下半年以来的实践和探讨,有些问题是可以弄清楚的。有些问题再经过一段时间的学习摸索,也是可以学会的。属于思想方法和工作作风方面的问题,已经有了这次深刻教训,使我们较易觉醒和体会了。但要彻底克服,还是要经过一番艰苦努力的。正如主席在这次会议中所指示的:“成绩伟大,问题很多,经验丰富,前途光明。”主动在我,全党团结起来艰苦奋斗,继续跃进的条件是存在的。今年明年和今后四年计划必将胜利完成,十五年赶上英国的奋斗目标,在今后四年内可以基本实现,某些重要产品也肯定可以超过英国。这就是我们伟大的成绩和光明的前途。

顺致敬礼!

彭德怀

一九五九年七月十四日。
\mnitem{10}左派是正确的,带引号的“左派”表示极左是错误的。
\mnitem{11}马歇尔(一八八〇——一九五九),美国民主党人,前国务卿和国防部长。一九四五年十二月曾被美国总统派为驻华特使,以“调处”为名,参与国共谈判,支持国民党政府发动内战。一九四六年八月宣布“调处”失败,不久返回美国。
\mnitem{12}吴芝圃,时任河南省委第一书记。
\mnitem{13}曾希圣,时任安徽省委第一书记。
\mnitem{14}柯庆施,时任上海市委第一书记,华东协作区主任委员。
\mnitem{15}王鹤寿,时任冶金工业部部长。
\mnitem{16}薄一波,时任副总理兼国家经济委员会主任。
\mnitem{17}李富春,时任计划委员会主任。
\mnitem{18}谭震林,外号“谭老板”,时任国务院副总理,分管农业。
\mnitem{19}这是对李云仲一九五九年六月九日写给毛泽东的关于目前经济生活中的一些问题的一封信的批语,李云仲的信中说,我想对目前经济生活中所发生的问题,联系一些思想作风问题,提出一些意见,供参考。(一)我觉得最近一年来,我们在工作中犯有“左”倾冒险主义的错误,其原因主要是,我们在思想战线上忽略了两条战线的斗争,即在反对右倾保守思想的同时,忽视“左”倾冒险主义的侵袭。我们党的历史经验中最重要的一条是:在党内思想战线上不断进行两条战线的斗争,这就是要时而反对“左”倾冒险主义对革命的危害,时而反对右倾机会主义对党的侵袭。(转发这封信时,毛泽东在这句话后面加括号写了以下批注:“毛注:时而反对这样,时而反对那样,时(然)而后言,可见不是同时。”)……(二)在各级干部中反对主观主义的思想作风,教育全体党员坚持党的原则,增强党性,是当前党的政治思想战线上一个很重要的问题。……(三)关于农民问题和工农关系问题。……(四)关于计划工作问题。……(五)关于体制问题。……(六)关于树立节俭、朴实的风气问题。铺张浪费,最近又有发展,具体表现在:第一,豪华的高级宾馆、饭店建得太多。第二,会议伙食标准太高。第三,领导干部生活上过分特殊的风气,有些地方仍未改变。
\mnitem{20}指一九五八年十一月二日至十日毛泽东在郑州召集部分中央领导人、大区负责人和部分省市委书记参加的工作会议。
\mnitem{21}指当时正在庐山举行的中共中央政治局扩大会议。
\mnitem{22}这是毛泽东在庐山召开的中共八届八中全会开幕时的讲话。
\mnitem{23}洛甫:张闻天。
\mnitem{24}这是毛泽东同志在庐山召开的中共八届八中全会闭幕式上讲话的节选。
\end{maonote}
