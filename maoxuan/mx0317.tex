
\title{组织起来}
\date{一九四三年十一月二十九日}
\thanks{这是毛泽东在中共中央招待陕甘宁边区劳动英雄大会上的讲话。}
\maketitle


今天共产党中央招待陕甘宁边区从农民群众中、工厂中、部队中、机关学校中选举出来的男女劳动英雄,以及在生产中的模范工作者,我代表中央来讲几句话。我想讲的意思,拿几个字来概括,就是“组织起来”。边区的农民群众和部队、机关、学校、工厂中的群众,根据去年冬天中共中央西北局所召集的高级干部会议的决议,今年进行了一年的生产运动。这一年的生产,在各方面都有了很大的成绩和很大的进步,边区的面目为之一新。事实已经完全证明:高级干部会议的方针是正确的。高级干部会议方针的主要点,就是把群众组织起来,把一切老百姓的力量、一切部队机关学校的力量、一切男女老少的全劳动力半劳动力,只要是可能的,就要毫无例外地动员起来,组织起来,成为一支劳动大军。我们有打仗的军队,又有劳动的军队。打仗的军队,我们有八路军新四军;这支军队也要当两支用,一方面打仗,一方面生产。我们有了这两支军队,我们的军队有了这两套本领,再加上做群众工作一项本领,那末,我们就可以克服困难,把日本帝国主义打垮。如果边区去年以前的生产运动的成绩还不够大,还不够显着,还不足以完全证明这一点,那末今年的成绩,就完全证明了这一点,这是大家亲眼看见了的。

边区的军队,今年凡有地的,做到每个战士平均种地十八亩,吃的菜、肉、油,穿的棉衣、毛衣、鞋袜,住的窑洞、房屋,开会的大小礼堂,日用的桌椅板凳、纸张笔墨,烧的柴火、木炭、石炭,差不多一切都可以自己造,自己办。我们用自己动手的方法,达到了丰衣足食的目的。每个战士,一年中只需花三个月工夫从事生产,其余九个月时间都可以从事训练和作战。我们的军队既不要国民党政府发饷,也不要边区政府发饷,也不要老百姓发饷,完全由军队自己供给;这一个创造,对于我们的民族解放事业,该有多么重大的意义啊!抗日战争六年半中,敌人在各抗日根据地内实行烧、杀、抢的“三光”政策,陕甘宁边区则遭受国民党的重重封锁,财政上经济上处于非常困难的地位,我们的军队如果只会打仗,那是不能解决问题的。现在我们边区的军队已经学会了生产;前方的军队,一部分也学会了,其它部分正在开始学习。只要我们全体英勇善战的八路军新四军,人人个个不但会打仗,会作群众工作,又会生产,我们就不怕任何困难,就会是孟夫子说过的:“无敌于天下。”\mnote{1}我们的机关学校,今年也大进了一步,向政府领款只占经费的一小部分,由自己生产解决的占了绝大部分;去年还只自给蔬菜百分之五十,今年就自给了百分之一百;喂猪养羊大大增加了肉食;又开设了许多作坊生产日用品。部队机关学校既然自己解决了全部或大部的物质问题,用税收方法从老百姓手中取给的部分就减少了,老百姓生产的结果归自己享受的部分就增多了。军民两方大家都发展生产,大家都做到丰衣足食,大家都欢喜。还有我们的工厂,发展了生产,清查了特务,生产效率也大大提高了。整个边区,产生了许多农业劳动英雄、工业劳动英雄、机关学校劳动英雄,军队中也出了许多劳动英雄,边区的生产,可以说是走上了轨道。凡此,都是实行把群众力量组织起来的结果。

把群众力量组织起来,这是一种方针。还有什么与此相反的方针没有呢?有的。那就是缺乏群众观点,不依靠群众,不组织群众,不注意把农村、部队、机关、学校、工厂的广大群众组织起来,而只注意组织财政机关、供给机关、贸易机关的一小部分人;不把经济工作看作是一个广大的运动,一个广大的战线,而只看作是一个用以补救财政不足的临时手段。这就是另外一种方针,这就是错误的方针。陕甘宁边区过去是存在过这种方针的,经过历年的指正,特别是经过去年的高级干部会议和今年的群众运动,大概现在还作这样错误想法的人是少了。华北华中各个根据地,因为战争紧张,也因为领导机关注意不够,群众的生产运动还没有广大的开展。但是在中央今年十月一号的指示\mnote{2}以后,各个地方也都在准备发动明年的生产运动了。前方的条件,比陕甘宁边区更困难,不但有严重的战争,有些地方还有严重的灾荒。但是为了支持战争,为了对付敌人的“三光”政策,为了救济灾荒,就不能不动员全体党政军民,一面打击敌人,一面实行生产。前方的生产,过去几年已经有了一些经验,加上今年冬天的思想准备、组织准备和物质准备,明年可能造成广大的运动,并且必须造成广大的运动。前方处于战争环境,还不能做到“丰衣足食”,但是“自己动手,克服困难”,则是完全可以做到,并且必须做到的。

目前我们在经济上组织群众的最重要形式,就是合作社。我们部队机关学校的群众生产,虽不要硬安上合作社的名目,但是这种在集中领导下用互相帮助共同劳动的方法来解决各部门各单位各个人物质需要的群众的生产活动,是带有合作社性质的。这是一种合作社。

在农民群众方面,几千年来都是个体经济,一家一户就是一个生产单位,这种分散的个体生产,就是封建统治的经济基础,而使农民自己陷于永远的穷苦。克服这种状况的唯一办法,就是逐渐地集体化;而达到集体化的唯一道路,依据列宁所说,就是经过合作社\mnote{3}。在边区,我们现在已经组织了许多的农民合作社,不过这些在目前还是一种初级形式的合作社,还要经过若干发展阶段,才会在将来发展为苏联式的被称为集体农庄的那种合作社。我们的经济是新民主主义的,我们的合作社目前还是建立在个体经济基础上(私有财产基础上)的集体劳动组织。这又有几种样式。一种是“变工队”、“扎工队”这一类的农业劳动互助组织\mnote{4},从前江西红色区域叫做劳动互助社,又叫耕田队\mnote{5},现在前方有些地方也叫互助社。无论叫什么名称,无论每一单位的人数是几个人的,几十个人的,几百个人的,又无论单是由全劳动力组成的,或有半劳动力参加的,又无论实行互助的是人力、畜力、工具,或者在农忙时竟至集体吃饭住宿,也无论是临时性的,还是永久性的,总之,只要是群众自愿参加(决不能强迫)的集体互助组织,就是好的。这种集体互助的办法是群众自己发明出来的。从前我们在江西综合了群众的经验,这次我们在陕北又综合了这样的经验。经过去年高级干部会议的提倡,今年一年的实行,边区的劳动互助就大为条理化和更加发展了。今年边区有许多变工队,实行集体的耕种、锄草、收割,收成比去年多了一倍。群众看见了这样大的实效,明年一定有更多的人实行这个办法。我们现在不希望在明年一年就把全边区的几十万个全劳动力和半劳动力都组织到合作社里去,但是在几年之内是可能达到这个目的的。妇女群众也要全部动员参加一定分量的生产。所有二流子都要受到改造,参加生产,变成好人。在华北华中各抗日根据地内,都应该在群众自愿的基础上,广泛组织这种集体互助的生产合作社。

除了这种集体互助的农业生产合作社以外,还有三种形式的合作社,这就是延安南区合作社式的包括生产合作、消费合作、运输合作(运盐)、信用合作的综合性合作社,运输合作社(运盐队)以及手工业合作社。

我们有了人民群众的这四种合作社,和部队机关学校集体劳动的合作社,我们就可以把群众的力量组织成为一支劳动大军。这是人民群众得到解放的必由之路,由穷苦变富裕的必由之路,也是抗战胜利的必由之路。每一个共产党员,必须学会组织群众的劳动。知识分子出身的党员,也必须学会;只要有决心,半年一年工夫就可以学好的。他们可以帮助群众组织生产,帮助群众总结经验。我们的同志学会了组织群众的劳动,学会了帮助农民做按家生产计划,组织变工队,组织运盐队,组织综合性合作社,组织军队的生产,组织机关学校的生产,组织工厂的生产,组织生产竞赛,奖励劳动英雄,组织生产展览会,发动群众的创造力和积极性,加上旁的各项本领,我们就一定可以把日本帝国主义打出去,一定可以协同全国人民,把一个新国家建立起来。

我们共产党员,无论在什么问题上,一定要能够同群众相结合。如果我们的党员,一生一世坐在房子里不出去,不经风雨,不见世面,这种党员,对于中国人民究竟有什么好处没有呢?一点好处也没有的,我们不需要这样的人做党员。我们共产党员应该经风雨,见世面;这个风雨,就是群众斗争的大风雨,这个世面,就是群众斗争的大世面。“三个臭皮匠,合成一个诸葛亮”,这就是说,群众有伟大的创造力。中国人民中间,实在有成千成万的“诸葛亮”,每个乡村,每个市镇,都有那里的“诸葛亮”。我们应该走到群众中间去,向群众学习,把他们的经验综合起来,成为更好的有条理的道理和办法,然后再告诉群众(宣传),并号召群众实行起来,解决群众的问题,使群众得到解放和幸福。如果我们做地方工作的同志脱离了群众,不了解群众的情绪,不能够帮助群众组织生产,改善生活,只知道向他们要救国公粮,而不知道首先用百分之九十的精力去帮助群众解决他们“救民私粮”的问题,然后仅仅用百分之十的精力就可以解决救国公粮的问题,那末,这就是沾染了国民党的作风,沾染了官僚主义的灰尘。国民党就是只问老百姓要东西,而不给老百姓以任何一点什么东西的。如果我们共产党员也是这样,那末,这种党员的作风就是国民党的作风,这种党员的脸上就堆上了一层官僚主义的灰尘,就得用一盆热水好好洗干净。我觉得,在无论哪一个抗日根据地的地方工作中,都存在有这种官僚主义的作风,都有一部分缺乏群众观点因而脱离群众的工作同志。我们必须坚决地克服这种作风,才能和群众亲密地结合起来。

此外,在我们的军队工作中,还存在有一种军阀主义作风,这也是一种国民党的作风,因为国民党军队是脱离群众的。我们的军队必须在军民关系上、军政关系上、军党关系上、官兵关系上、军事工作和政治工作关系上、干部相互关系上,遵守正确的原则,决不可犯军阀主义的毛病。官长必须爱护士兵,不能漠不关心,不能采取肉刑;军队必须爱护人民,不能损害人民利益;军队必须尊重政府,尊重党,不能闹独立性。我们的八路军新四军是人民的军队,历来是好的,现在也是好的,是全国军队中一支最好的军队。但是近年来确实生长了一种军阀主义的毛病,一部分军队工作同志养成了一种骄气,对士兵,对人民,对政府,对党,横蛮不讲理,只责备做地方工作的同志,不责备自己,只看见成绩,不看见缺点,只爱听恭维话,不爱听批评话。例如陕甘宁边区,就有这种现象。经过去年的高级干部会议和军政干部会,又经过今年春节的拥政爱民运动和拥军运动,这个倾向是根本地克服下去了,还有一些残余,还必须继续去克服。华北华中各根据地内,这种毛病都是有的,那里的党和军队必须注意克服这种毛病。

无论在地方工作中,在军队工作中,无论是官僚主义倾向或军阀主义倾向,其毛病的性质都是一样,就是脱离群众。我们的同志,绝对大多数都是好同志。对于有了毛病的人,一经展开批评,揭发错误,也就可以改正。但是必须开展自我批评,正视错误倾向,认真实行改正。如果在地方工作中不批评官僚主义倾向,在军队工作中不批评军阀主义倾向,那就是愿意保存国民党作风,愿意保存官僚主义灰尘和军阀主义灰尘在自己清洁的脸上,那就不是一个好党员。如果我们在地方工作中去掉官僚主义倾向,在军队工作中去掉军阀主义倾向,那就一切工作都会顺利地开展,生产运动当然也是这样。

我们边区的生产,无论在农民群众方面、机关学校方面、军队方面、工厂方面,都得到了很大的成绩,在军民关系上也有了很大进步,边区的面目,和以前大不相同了。所有这些,都是我们的同志的群众观点已经加强,同群众的结合大进一步的表现。但是我们不应该自满,我们还要继续作自我批评,还要继续求进步。我们的生产也要继续求进步。我们脸上有灰尘,就要天天洗脸,地上有灰尘,就要天天扫地。尽管我们在地方工作中的官僚主义倾向,在军队工作中的军阀主义倾向,已经根本上克服了,但是这些恶劣倾向又可以生长起来的。我们是处在日本帝国主义和中国反动势力的层层包围之中,我们是处在散漫的小资产阶级的包围之中,极端恶浊的官僚主义灰尘和军阀主义灰尘天天都向我们的脸上大批地扑来。因此,我们决不能一见成绩就自满自足起来。我们应该抑制自满,时时批评自己的缺点,好像我们为了清洁,为了去掉灰尘,天天要洗脸,天天要扫地一样。

各位劳动英雄和模范生产工作者,你们是人民的领袖,你们的工作是很有成绩的,我希望你们也不要自满。我希望你们回到关中去,回到陇东去,回到三边去,回到绥德去,回到延属各县去,回到机关学校部队工厂去,领导人民,领导群众,把工作做得更好,首先是按自愿的原则把群众组织到合作社里来,组织得更多,更好。希望你们回去实行这一条,宣传这一条,使明年再开劳动英雄大会的时候,我们能够得到更大的成绩。


\begin{maonote}
\mnitem{1}见《孟子·公孙丑上》。
\mnitem{2}见本卷\mxart{开展根据地的减租、生产和拥政爱民运动}。
\mnitem{3}参见列宁《论合作社》(《列宁全集》第43卷,人民出版社1987年版,第361—368页)。
\mnitem{4}“变工队”和“扎工队”,都是陕甘宁边区建立在个体经济基础上的农业劳动互助组织。“变工”即换工,是农民相互间调剂劳动力的方法,有人工换人工、畜工换畜工、人工换畜工等等。参加变工队的农民,各以自己的劳动力或者畜力,轮流地给本队各家耕种。结算时,多出了人工或者畜工的由少出了的补给工钱。“扎工队”一般是由土地不足的农民组成。参加扎工队的农民,除相互变工互助以外,主要是集体出雇于需要劳动力的人家。
\mnitem{5}见本书第一卷\mxnote{我们的经济政策}{2}。
\end{maonote}
