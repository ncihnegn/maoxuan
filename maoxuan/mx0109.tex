
\title{怎样分析农村阶级}
\date{一九三三年十月}
\thanks{这个文件,是毛泽东一九三三年十月为纠正在土地改革工作中发生的偏向、正确地解决土地问题而写的,曾由当时中央工农民主政府通过,作为划分农村阶级成分的标准。}
\maketitle


\section{一 地主}

占有土地,自己不劳动,或只有附带的劳动,而靠剥削农民为生的,叫做地主。地主剥削的方式,主要地是收取地租,此外或兼放债,或兼雇工,或兼营工商业。但对农民剥削地租是地主剥削的主要的方式。管公堂和收学租\mnote{1}也是地租剥削的一类。

有些地主虽然已破产了,但破产之后仍不劳动,依靠欺骗、掠夺或亲友接济等方法为生,而其生活状况超过普通中农者,仍然算是地主。

军阀、官僚、土豪、劣绅是地主阶级的政治代表,是地主中特别凶恶者。富农中亦常有较小的土豪、劣绅。

帮助地主收租管家,依靠地主剥削农民为主要的生活来源,其生活状况超过普通中农的一些人,应和地主一例看待。

依靠高利贷剥削为主要生活来源,其生活状况超过普通中农的人,称为高利贷者,应和地主一例看待。

\section{二 富农}

富农一般占有土地。但也有自己占有一部分土地,另租入一部分土地的。也有自己全无土地,全部土地都是租入的。富农一般都占有比较优裕的生产工具和活动资本,自己参加劳动,但经常地依靠剥削为其生活来源的一部或大部。富农的剥削方式,主要是剥削雇佣劳动(请长工)。此外,或兼以一部土地出租剥削地租,或兼放债,或兼营工商业。富农多半还管公堂。有的占有相当多的优良土地,除自己劳动之外并不雇工,而另以地租债利等方式剥削农民,此种情况也应以富农看待。富农的剥削是经常的,许多富农的剥削收入在其全部收入中并且是主要的。

\section{三 中农}

中农许多都占有土地。有些中农只占有一部分土地,另租入一部分土地。有些中农并无土地,全部土地都是租入的。中农自己都有相当的工具。中农的生活来源全靠自己劳动,或主要靠自己劳动。中农一般不剥削别人,许多中农还要受别人小部分地租债利等剥削。但中农一般不出卖劳动力。另一部分中农(富裕中农)则对别人有轻微的剥削,但非经常的和主要的。

\section{四 贫农}

贫农有些占有一部分土地和不完全的工具;有些全无土地,只有一些不完全的工具。一般都须租入土地来耕,受人地租、债利和小部分雇佣劳动的剥削。中农一般不要出卖劳动力,贫农一般要出卖小部分的劳动力,这是区别中农和贫农的主要标准。

\section{五 工人}

工人(雇农在内)一般全无土地和工具,有些工人有极小部分的土地和工具。工人完全地或主要地以出卖劳动力为生。


\begin{maonote}
\mnitem{1}旧中国农村中有许多的公共土地。有些是政治性的,例如一些区乡政府所有的土地。有些是宗族性的,例如各姓祠堂所有的土地。有些是宗教性的,例如佛教、道教、天主教、伊斯兰教的寺、观、教堂、清真寺所有的土地。有些是社会救济或者社会公益性的,例如义仓的土地和为修桥补路而设置的土地。有些是教育性的,例如学田。所有这些土地,大部分都掌握在地主富农手里,只有一小部分,农民有权干预。
\end{maonote}
