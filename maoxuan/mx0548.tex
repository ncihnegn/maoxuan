
\title{《中国农村的社会主义高潮》的按语}
\date{一九五五年九月、十二月}
\thanks{毛泽东同志在编辑《中国农村的社会主义高潮》一书的时候,写了一百零四篇按语,这里选辑了四十三篇。一九五八年三月,在成都召开的中共中央政治局扩大会议,重印了一部分按语,为此,毛泽东同志在一九五八年三月十九日写了一个说明。这个说明全文如下:\par
“这些按语见《中国农村的社会主义高潮》一书中,是一九五五年九月和十二月写的,其中有一些现在还没有丧失它们的意义。其中说:一九五五年是社会主义与资本主义决战取得基本胜利的一年,这样说不妥当。应当说:一九五五年是在生产关系的所有制方面取得基本胜利的一年,在生产关系的其它方面以及上层建筑的某些方面即思想战线方面和政治战线方面,则或者还没有基本胜利,或者还没有完全胜利,还有待于尔后的努力。我们没有预料到一九五六年国际方面会发生那样大的风浪,也没有预料到一九五六年国内方面会发生打击群众积极性的‘反冒进’事件。这两件事,都给右派猖狂进攻以相当的影响。由此得到教训:社会主义革命和社会主义建设都不是一帆风顺的,我们应当准备对付国际国内可能发生的许多重大困难。无论就国际方面说来,或者就国内方面说来,总的形势是有利的,这点是肯定的;但是一定会有许多重大困难发生,我们必须准备去对付。”}
\maketitle


\section*{一}

这篇文章写得很好,值得作为本书的第一篇向读者们推荐。如象这篇文章在开头所描写的,自己不懂,怕人问,就“绕开社走”的人,现在各地还是不少的。所谓“坚决收缩”,下命令大批地解散合作社的做法,也是“绕开社走”的一种表现。不过他们不是消极地避开,而是索性一刀“砍掉”(这是他们的话)多少个合作社,采取十分积极的态度罢了。他们手里拿着刀,一砍,他们就绕开麻烦问题了。他们说办合作社有怎样怎样的困难,据说简直困难到了不堪设想的地步。全国有不可胜数的事例驳倒了这一种说法。河北省遵化县的经验,不过是这些事例的一个。在一九五二年,这里的人都不懂得怎样办合作社。他们的办法就是学习。他们的口号是“书记动手,全党办社”。其结果就是“从不懂到懂”,“从少数人会到多数人会”,“从区干部办社到群众办社”。河北省遵化县的第十区,十一个乡,四千三百四十三户,从一九五二年到一九五四年,共计三年时间内,已经在半社会主义性质的阶段内基本上完成了合作化,入社农户占全区农户的百分之八十五。这个区的农林牧等项生产的产量,一九五四年同一九五二年比较,粮食增加了百分之七十六,林木增加了百分之五十六点四,果树增加了百分之六十二点八七,羊增加了百分之四百六十三点一。

我们现在有理由向人们提出这样一个问题:为什么这个地方可以这样做,别的地方就不可以这样做呢?如果说不可以,你们的理由在什么地方呢?我看只有一条理由,就是怕麻烦,或者爽直一点,叫做右倾机会主义。因此就是“绕开社走”,就是书记不动手,全党不办社,就是从不懂到不懂,从少数人到少数人,从区干部到区干部。要不然,就是手里拿着刀,见了找麻烦的合作社就给它一砍。只要有了这样一条理由,那就什么事也做不成了。我们提出了“积极领导,稳步前进”,“全面规划,加强领导”这样一些口号,并且赞成遵化县同志们所提出来的“书记动手,全党力、社”这个完全正确的口号。在遵化县,难道不是“积极领导,稳步前进”的吗?难道不是“全面规划,加强领导”的吗?当然是的。这是不是有危险呢?是不是“冒进”了呢?危险在于“绕开社走”,这一点遵化县的同志们已经克服了。危险还在于借口“冒进”,大批地“砍掉”合作社,这一点遵化县那里并没有。所谓“合作社发展速度超过了群众觉悟的水平和干部领导能力的水平”,这对于遵化县的情况怎样解释呢?那里的群众就是要求合作化,那里的干部就是由不懂到懂。人人都有眼睛,谁能在遵化县那里看得出什么危险来呢?难道在三年内,由于一步一步地实现了合作化,粮食增加了百分之七十六,林木增加了百分之五十六点四,果树增加了百分之六十二点八七,羊增加了百分之四百六十三点一,这就算是一种危险吗?这就算是“冒进”吗?这就算是“超过了群众觉悟的水平和干部领导能力的水平”吗?

遵化县的合作化运动中,有一个王国藩合作社,二十三户贫农只有三条驴腿,被人称为“穷棒子社”。他们用自己的努力,在三年时间内,“从山上取来”了大批的生产资料,使得有些参观的人感动得下泪。我看这就是我们整个国家的形象。难道六万万穷棒子不能在几十年内,由于自己的努力,变成一个社会主义的又富又强的国家吗?社会的财富是工人、农民和劳动知识分子自己创造的。只要这些人掌握了自己的命运,又有一条马克思列宁主义的路线,不是回避问题,而是用积极的态度去解决问题,任何人间的困难总是可以解决的。

最后,我们要感谢这篇文章的那位没有署名的作者。他用满腔的热情,生动的笔调,详尽地叙述了一个区的合作化过程,这对于全国的合作化事业会有不小的贡献。我们希望每个省、每个专区、每个县都有一篇到几篇这样的文章。


(《书记动手,全党办社》一文按语)

\section*{二}

在中国,对于许多人来说,一九五五年,可以说是破除迷信的一年。一九五五年的上半年,许多人对于一些事还是那么样坚持自己的信念。一到下半年,他们就坚持不下去了,只好相信新事物。例如:他们认为群众中提出的“三年合作化”不过是幻想;合作化北方可以快一些,南方无法快;落后乡不能办合作社;山区不能办合作社;少数民族地区和民族杂居地区不能办合作社;灾区不能办合作社;建社容易巩固难;农民太穷,资金无法筹集;农民没有文化,找不到会计;合作社办得越多,出乱子就会越多;合作社发展的速度,超过了群众的觉悟水平和干部的经验水平;因为党的粮食统购统销政策和合作化政策,使得农民的生产积极性降低了;在合作化问题上,共产党如果不赶快下马,就有破坏工农联盟的危险;合作化将出现大批的剩余劳动力,找不到出路。如此等等,还可以举出许多。总之都是迷信。这些迷信,经过一九五五年十月中国共产党第七届第六次中央全体会议(扩大)的批判以后,统统都打破了。现在全国农村中已经出现了社会主义改造的高潮,群众欢欣鼓舞。这件事给了一切共产党人一个深刻的教训:群众中蕴藏了这样大的社会主义的积极性,为什么在许多领导机关,在几个月以前,居然没有感觉到,或者感觉的那样少呢?领导者们所想的同广大群众所想的,为什么那样不一致呢?以此为教训,那末,今后对于有相似情况的事件和问题,应当怎样处理才好呢?回答只有一句话,就是不要脱离群众,要善于从本质上发现群众的积极性。


(《所谓落后乡村并非一切都落后》一文按语)

\section*{三}

那些不相信就一个一个的地方来说可以在三年内实现初级形式的合作化的人们(三年合作化的口号是群众提出的,遭到了机会主义者的批评),那些不相信晚解放区可以和老解放区同时实现合作化的人们,请看一看江苏省昆山县的这个乡罢!这里不是三年合作化,而是两年就合作化了。这里不是老解放区,是一个千真万确的晚解放区。这个晚解放区,走到许多老解放区的前面去了。有什么办法呢?难道可以把它拉回来吗?当然不能,机会主义者只有认输一法。群众中蕴藏了一种极大的社会主义的积极性。那些在革命时期还只会按照常规走路的人们,对于这种积极性一概看不见。他们是瞎子,在他们面前出现的只是一片黑暗。他们有时简直要闹到颠倒是非、混淆黑白的程度。这种人难道我们遇见得还少吗?这些只会循着常规走路的人们,老是对于人民的积极性估计过低。一种新事物出现,他们总是不赞成,首先反对一气。随后就是认输,做一点自我批评。第二种新事物出现,他们又按照这两种态度循环一遍。以后各种新事物出现,都按照这个格式处理。这种人老是被动,在紧要的关头老是止步不前,老是需要别人在他的背上击一猛掌,才肯向前跨进一步。那一个年头能使这种人自己有办法走路,并且走得象个样子呢?有一个治好这种毛病的法子,就是拿出一些时间到群众中去走一走,看看群众在想些什么,做些什么,从其中找出先进经验,加以推广。这是一个治好右倾顽症的有效的药方,奉劝人们不妨试一试。


(《这个乡两年就合作化了》一文按语)

\section*{四}

这是一篇好文章。看了这篇文章,使人懂得维吾尔族的农民,对于走合作化道路,积极性是很高的。他们为了实现半社会主义合作化所需要的干部,也已经培养出来了。有人说。在少数民族中不能实行合作化。这是不对的。我们已经看到蒙族,回族,维吾尔族,苗族,壮族和其它一些民族都已经办了不少的合作社,或者是几个族的人民联合办的合作社,并且成绩很好,这就驳斥了那些对于少数民族采取轻视态度的人们的错误观点。


(《乡、村干部有能力领导建社》一文按语)

\section*{五}

这一篇很好,可以说服很多人。这个地方的党组织,在合作化的问题上,从来没有动摇过。它坚决地支持贫苦农民的办社要求,在和富裕中农的竞赛中取得了胜利,由小社变大社,年年增产,不到三年,实现了全村合作化。富裕中农说:“穷光蛋想办合作社哩,没有见过鸡毛能上天。”鸡毛居然飞上天去了。这就是社会主义和资本主义的两条道路的斗争。在中国,富农经济很弱(在土地改革时期,征收了他们的半封建的那部分土地,老富农大多数已无雇工,他们在社会上的名声又很坏),富裕的和比较富裕的中农的力量却是相当强大的,他们占农村人口的百分之二十到三十。在中国的农村中,两条道路的斗争的一个重要方面,是通过贫农和下中农同富裕中农实行和平竞赛表现出来的。在两三年内,看谁增产:是单干的富裕中农增产呢,还是贫农和下中农组成的合作社增产呢?在开头,只是一部分贫农和下中农组成的合作社,同单干的富裕中农在竞赛,大多数的贫农和下中农还在那里看,这就是双方在争夺群众。在富裕中农的后面站着地主和富农,他们是有时公开地有时秘密地支持富裕中农的。在合作社的这面站着共产党,他们应当如同安阳县南崔庄的共产党人那样,坚决地支持合作社。但是,可惜,并不是每一个乡村的党支部都是这样的。在这种情形之下,就造成了混乱。首先是鸡毛能不能上天的舆论问题。这当然是一个严重的问题。几千年以来,谁人看见过鸡毛能够上天呢?这似乎是一个真理。如果党不给以批评,它就会使许多贫农和下中农感到迷惑。其次,在干部方面,又其次,在物质力量例如贷款方面,如果都得不到党和国家的支持,合作社就会发生很大的困难。富裕中农之所以敢于宣传鸡毛不能上天一类的从古以来的真理,就是因为合作社还没有增产,穷社还没有变成富社,个别的孤立的合作社还没有变成成千成万的合作社。就是因为党还没有在全国范围内,大张旗鼓地宣传合作化的好处。还没有明确地指出“鸡毛不能上天”这个古代的真理,在社会主义时代,它已经不是真理了。穷人要翻身了。旧制度要灭亡,新制度要出世了。鸡毛确实要上天了。在苏联,已经上天。在中国,正在上天。在全世界,都是要上天的。我们的许多地方党组织没有能够给贫苦农民以坚决的支持,也不能完全怪它们,上面还没有给机会主义思想以致命的打击,还没有给合作化做出全面的规划,并且在全国范围内加强对于这个运动的领导。一九五五年,我们做了这些工作,几个月工夫,形势就完全不同了。站在那里看的广大群众,一批一批地站到合作化这边来了。富裕中农也改变了腔调。有些要求入社,有些准备人社。最顽固的,也不敢议论鸡毛能不能上天的问题了。地主和富农,一点神气也没有了。这同人民政府惩治了一批破坏治安和破坏合作化的反革命分子,也是有关系的。总之,一九五五年的下半年,我国的阶级力量对比起了一个根本的变化,社会主义大为上升,资本主义大为下降。一九五六年再有一年的努力,过渡时期内的社会主义改造的基础,就可以从基本上奠定了。


(《谁说鸡毛不能上天》一文按语)

\section*{六}

几乎带普遍性地在许多地方存在着的、阻碍广大的贫农和下中农群众走合作化道路的、党内的右倾机会主义分子,同社会上的资本主义势力互相呼应着。对于这样一种情形,这一篇文章算是描写得恰好。作者以极大的愤怒斥责了机会主义者,支持了贫苦农民。有些人虽然顶着共产主义者的称号,却对于现在要做的社会主义事业表现很少兴趣。他们不但不支持热情的群众,反而向群众的头上泼冷水。一九五五年,在中国,正是社会主义和资本主义决胜负的一年。这一决战,是首先经过中国共产党中央召集的五月、七月和十月三次会议表现出来的。一九五五年上半年是那样的乌烟瘴气,阴霸满天。一九五五年下半年却完全变了样,成了另外一种气候,几千万户的农民群众行动起来,响应党中央的号召,实行合作化。到编者写这几行的时候,全国已经有六千万以上的农户加入合作社了。这是大海的怒涛,一切妖魔鬼怪都被冲走了。社会上各种人物的嘴脸,被区别得清清楚楚。党内也是这样。这一年过去,社会主义的胜利就有了很大的把握了。当然还有许多战斗在后头,还要努力作战。


(《机会主义的邪气垮下去,社会主义的正气升上来》一文按语)

\section*{七}

这也是一篇很有兴趣的文章。想要阻挡潮流的机会主义者虽然几乎到处都有,潮流总是阻挡不住的,社会主义到处都在胜利地前进,把一切绊脚石抛在自己的后头。社会就是这样地每天在前进,人们的思想在被改造着,特别在革命高涨的时候是这样。


(《在合作化运动中,工人家属的积极性非常高》一文按语)

\section*{八}

这是一篇动人的叙述,希望读者好好地看一遍。特别要请那些不相信广大农民群众有走社会主义道路的积极性的同志和那些动不动就想拿起刀来“砍掉”合作社的同志好好地看一遍。现在全国农村中,社会主义因素每日每时都在增长,广大农民群众要求组织合作社,群众中涌出了大批的聪明、能干、公道、积极的领袖人物,这种情况十分令人兴奋。最大的缺点,就是在许多地方党的领导还没有主动地赶上去。目前的任务,就是要使各级地方党委在这个问题上采取马克思列宁主义的主动立场,将整个农业合作化的任务拿到自己手里来,用积极的高兴的欢迎的全力以赴的态度去领导这个运动。不要重复叶公好龙那个故事,讲了多少年的社会主义,临到社会主义跑来找他,他又害怕起来了。


(《一个违背领导意愿由群众自动办起来的合作社》一文按语)

\section*{九}

这个地方的路线是正确的。这个乡已经有了五个农业生产合作社,七个互助联组,三个常年互助组,十四个临时互助组,占应当组织的农户百分之九十八点四。在一九五四年十二月以前,这个乡的党支部还没有把自己领导工作的重心放到互助合作方面来,党员对于领导互助组的工作怕困难。支部所依靠的不是“书记动手,全党办社”,而是工作组(似乎是上面派在那里的工作组)。在农业合作化问题上,党的农村支部处在这样一种软弱无能的状态的,现在在全国还是不少的。不但支部,可能还有一些上级党委也是这样。问题就在于这一点。我国的农业的社会主义改造能不能和国家工业化的进度相适应,合作化运动能不能健康地发展,少出毛病,保证增产,就看各级地方党委的领导重心是不是能够迅速地和正确地转移到这一方面来。工作组是必须派的,但是必须讲清楚,他们去是为了帮助那里的党组织,而不是代替它们,使它们自己不动手脑,专门依赖工作组。贵州的这个乡,自从一九五四年十二月转变工作态度以来,只有五个多月,就取得了很大的成绩。他们不是依赖工作组,而是自己动手了,党员也不怕困难了。这种转变,首先要靠各级党委的书记——省委和自治区党委的书记,地委和自治州委的书记,县委和自治县委的书记,区委书记和支部书记,他们应当把整个农业合作化的任务拿起来。怕麻烦,怕困难,面临着这样伟大的任务自己不动手,仅仅委托给农村工作部,或者工作组,这种态度,不但任务不可能完成,而且会要闹出很多的乱子来的。


(《凤冈县崇新乡是怎样在党支部领导下开展互助合作运动的》一文按语)

\section*{十}

本文作者说,自从县里开了合作社主任的联席会议以后,这个合作社,就在临时包工的基础上,实行了季节包工制。可见县的领导是很重要的。我们希望全国二千几百个县的县级领导机关,密切地注意全县合作化运动的发展情况,发现问题,研究解决问题的办法,及时地召开全县的合作社主任的会议,或者全县的重点合作社主任的会议,作出决定,迅速推行。不要等到问题成了堆,闹出了许多乱子,然后才去解决。领导一定要走在运动的前面,不要落在它的后面。在一个县的范围内,党的县委应当起主要的领导作用。


(《季节包工》一文按语)

\section*{十一}

这个材料有很大的说服力。使一个地方健全地达到合作化的问题是党的政策和工作方法的问题。只要我们党对于处理合作化问题上的各项政策是正确的,只要我们党当着发动群众加入合作社的时候所采用的工作方法,不是命令主义的或者简单从事的方法,而是向群众讲道理,作分析,完全依靠群众自觉自愿的方法,那末,完成合作化,并且达到增产,决不是很困难的。河北省邢台县东川口村是老解放区,全村七十户,在一九五二年以前都加入了互助组,有一个强的党支部,又有王志棋那样为群众所信任的领袖人物,各方面的条件都成熟了。所以那个村,在一九五二年,只费了一个多月的时间,就建成了合作社,完成了半社会主义的合作化。在条件没有这个村这样完备的地方怎么办呢?那就是准备条件的问题,有几个月,或者一年,或者更多一点时间,也就可以了。条件可以边做这准备。办一些小社,就是替全村全乡全区的合作化准备条件。东川口这个材料还着重地说明了党支部如何向群众进行宣传教育工作,如何依靠群众的自觉自愿去建立合作社的问题。其中有所谓“倒宣传”\mnote{1},是很值得注意的一点。关于劳动组织和劳动管理方面的问题,这个材料描绘了整个曲折变化的过程,结果出现了逐年增产的巨大的成绩。事实证明,这个合作社是健全的。一切合作社,都要以是否增产和增产的程度,作为检验自己是否健全的主要的标准。


(《只花一个多月时间就使全村合作化》一文按语)

\section*{十二}

这个材料指出了一个真理,就是任何情况混乱的合作社,都是可以整理的。因为加入合作社的都是劳动农民,不管他们各个阶层之间意见怎样不合,总是可以说清楚的。有些合作社,在一个时期内,确是混乱的,唯一的原因是得不到党的领导,党没有向群众讲明自己的政策和办法。“我们知道办社是好事情。但是办起社来,县委、区委、支部都不管我们了。恐怕是嫌我们寨子穷,吃不好,住不好,才不到我们社里来。”所谓混乱,没有别的原因,就是这样一个原因。得不到党的领导,当然就要混乱。领导一加上去,混乱就会立刻停止。这个材料又提出了一个在落后乡村是否可以建立合作社的问题。回答是肯定的。本文作者所说的这个合作社,就是处在一个落后村。全国约有百分之五左右的落后乡村,我们应当都去建立合作社,就在建社的斗争中去消灭这些地方的落后状态。


(《一个混乱的合作社整顿好了》一文按语)

\section*{十三}

这是一个普遍的严重的问题。各级党委和派到农村指导合作化工作的同志们,对于这个问题都应当引起充分的注意。合作社的领导机关必须建立现有贫农和新下中农在领导机关中的优势,而以老下中农和新老两部分上中农作为辅助力量,一才能按照党的政策实现贫农和中农的团结,巩固合作社,发展生产,正确地完成整个农村的社会主义改造。没有这个条件,中农和贫农就不能团结,合作社就不能巩固,生产就不能发展,整个农村的社会主义改造就不能实现。许多同志不懂这个道理。他们认为建立贫农优势的问题,在土地改革时期是必要的,因为那时占农村人口百分之五十、六十到七十的贫农,还没有上升为中农,而中农对于土地改革是动摇的,因此那时确有建立贫农优势的必要。现在是实行农业的社会主义改造的时期,过去的贫农大部分已经上升为新中农,而老中农的生产资料又多,没有老中农参加就不能解决合作社生产资料缺乏的问题。因此,这些同志认为,现在不应当提出依靠贫农或者建立贫农优势的口号,认为这种口号会不利于合作化。我们认为这种意见是错误的。工人阶级和共产党如果要用社会主义精神和社会主义制度去彻底地改造整个农村的小农私有生产资料的制度,便只有依靠过去是半无产阶级的广大的贫农群众,才能比较顺利地办到,否则将是很困难的。因为农村中的半无产阶级,是比较地不固执小农私有生产资料的制度,比较地容易接受社会主义改造的人们。他们中间的大部分现在已经变为新中农,但是他们同老中农比较起来,除了一部分新富裕中农以外,大多数在政治上有较高的觉悟,他们过去的困苦生活还是容易回忆起来。还有,老中农中间的下中农,他们的经济地位和政治态度,和新中农中间的下中农比较接近,而和新老中农中间的上中农,即富裕的和比较富裕的中农不相同。因此,在合作化的过程中,我们必须注意:(一)现在还处于困难地位的贫农,(二)新中农中间的下中农,(三)老中农中间的下中农,这样三部分比较容易接受社会主义改造的人们,首先分批分期引导他们加入合作社,并且选择他们中间觉悟程度较高、组织能力较强的若干人,加以训练,组成合作社的领导骨干,特别要注意从现有贫农和新下中农里面选择这种骨干分子。这不是在农村中重新来一次划分阶级成份的工作,而是在合作化的过程中,党的支部和派到农村做指导工作的同志们应当注意掌握的一种方针,这个方针应当公开告诉农民群众。我们也不是说富裕中农不能入社,而是说等到富裕中农的社会主义觉悟提高了,他们表示愿意入社,并且愿意服从贫农(包括现在的贫农和原来是贫农的全部新下中农)领导的时候,再去吸收他们入社,不要为了打他们的耕牛农具的主意,当着他们不愿意入社的时候强迫他们入社。已经入社而愿意留下来的,继续不变。要求退社,加以说服,愿意留的,也可以国社。生产资料少一点也可以组织合作社,很多贫农和下中农组织的合作社已经证明了。我们也不是说富裕中农一个也不能充当合作社的干部。那些社会主义觉悟程度高,公道能干,为全社大多数人所佩服的个别的富裕中农,也可以充当干部。但是,合作社必须树立贫农(再说一遍,包括现在的贫农和原来是贫农的全部新下中农在内,他们占农村人口的多数,或者大多数)的优势。在组织成份方面,他们应占三分之二左右,中农(包括老下中农和新老两部分上中农)只占也应占三分之一左右。在合作社的指导方针方面,必须实行贫农和中农的互利政策,不应当损害任何人的利益。要做到这一点,也必须建立贫农优势。在中农占优势的合作社里,总是会要排挤贫农和损害贫农的利益的。湖南省长沙县高山乡的经验,充分地告诉我们:建立贫农优势和由此去巩固地团结中农的必要性和可能性,以及如果不这样做,它的危险又会怎么样。本文作者完全懂得党的路线。做法也很对,先去完成紧急的增产任务,后去建立贫农的优势领导。结果,贫农扬眉吐气,中农也心悦诚服。本文作者还告诉我们一件大事,就是一个情况混乱的合作社,究竟是将它解散好呢?还是加以整顿,使它由混乱走到健康好呢?这样的合作社,整顿巩固是不是可能的呢?本文作者很有说服力地告诉我们,不应当去解散那些“三等社”,而应当去做整顿工作。经过工作,三等社是完全可以变为一等社的。这种经验,全国各地已经不少,不止是长沙县高山乡一处。


(《长沙县高山乡武塘农业生产合作社是怎样从中农占优势转变为贫农占优势的》一文按语)

\section*{十四}

这里所说的问题,有普遍的意义。中农是必须团结的,不团结中农是错误的。但是工人阶级和共产党,在农村中,依靠什么人去团结中农,实现整个农村的社会主义改造呢?当然只有贫农。在过去向地主作斗争、实行土地改革的时候是这样,在现在向富农和其它资本主义因素作斗争实行农业的社会主义改造的时候,也是这样。在两个革命时期,中农在开始阶段都是动摇的。等到看清了大势,革命将要胜利的时候,中农才会参加到革命方面来。贫农必须向中农做工作,把中农团结到自己方面来,使革命一天一天地扩大,直到取得最后的胜利。现在的农业生产合作社的社务管理委员会,如同过去的农民协会一样,必须吸引老下中农和一部分觉悟较高的和有代表性的新老上中农参加,但是人数不可太多,以占三分之一左右为适宜。贫农(包括现在的贫农和原来是贫农的新下中农)委员的人数应当占到三分之二左右。社的主要领导干部,除了老下中农和若干觉悟很高、确实公道能干的新老上中农仍然可充当以外,一般应当由贫农(再说一遍,包括现在的贫农和原来是贫农的全部新下中农)来充当。我们对于福建省福安县贫农领导的合作社和中农领导的合作社对社会主义事业表示不同的态度这样一种情况,不应当看作是个别的现象,它是具有普遍意义的。


(《福安县发生“中农社”和“贫农社”的教训》一文按语)

\section*{十五}

这个材料有用,值得普遍注意。这个材料描绘了农村中各个阶层的动态。贫农对于合作化最积极。许多中农要“再看一看”,他们爱“在外边松快”,他们主要地是要看合作社对于他们的生产资料入社是否使他们不吃亏,他们是可以这样也可以那样的。许多富裕中农对于合作化有很大的抵触情绪;其中态度最坏的,在那里变卖生产资料,抽逃资金,组织假合作社,个别的甚至勾结地主富农做坏事。我们希望各地从事农村工作的同志们都注意观察和分析自己那里的各个阶层的动态,以便采取适合情况的政策。这个材料提到了注意合作社忽视互助组的错误倾向,提出了统筹兼顾的意见,这是正确的。“互助合作网”的办法是好的,就是要社、组兼顾,合作社要真正帮助互助组和单干户解决他们的当前生产的困难问题。贫农基金必须迅速发下去。现在还没有加入合作社的贫农,要告诉他们,他们什么时候入社,什么时候就可以取得这笔基金。


(《新情况和新问题》一文按语)

\section*{十六}

这个合作社的方针是正确的。一切合作社都应当这样做。各省应当在自己的关于合作化问题的决议或者指示里面指出,一切合作社有责任帮助鳏寡孤独缺乏劳动力的社员(应当吸收他们入社)和虽然有劳动力但是生活上十分困难的社员,解决他们的困难。目前,有许多合作社,缺乏帮助困难户的社会主义的精神,甚至根本排斥贫农,这是完全错误的。目前,政府已经设立了贫农基金,可以帮助贫农解决耕牛农具的困难,但是还不能解决贫农中有些户缺乏劳动力的困难,也不能完全解决有些户在青黄不接时期缺乏生活资料的困难,这只有依靠合作社广大群众的力量才能解决。


(《湘潭县清风乡党支部帮助贫苦社员解决困难》一文按语)

\section*{十七}

这是一个很有兴趣的故事。社会主义这样一个新事物,它的出生,是要经过同旧事物的严重斗争才能实现的。社会上一部分人,在一个时期内,是那样顽固地要走他们的老路。在另一个时期内,这些同样的人又可以改变态度表示赞成新事物。富裕中农的大多数,在一九五五年上半年,对于合作化还是反对的,下半年就有一部分人改变了态度,表示要入合作社,虽然其中有一些人的目的是为了想要取得合作社的领导权而入社的。另一部分人表现了极大的动摇,口里讲要加入,心里还是不大愿意。第三部分人则是顽固地还要等着看。在这个问题上,农村的党组织对于这个阶层要有等待的耐心。为了建立贫农和新下中农在领导方面的优势,某些富裕中农迟一点加入合作社反而是有利的。


(《他们坚决选择了合作化的道路》一文按语)

\section*{十八}

政治工作是一切经济工作的生命线。在社会经济制度发生根本变革的时期,尤其是这样。农业合作化运动,从一开始,就是一种严重的思想的和政治的斗争。每一个合作社,不经过这样的一场斗争,就不能创立。一个崭新的社会制度要从旧制度的基地上建立起来,它就必须清除这个基地。反映旧制度的旧思想的残余,总是长期地留在人们的头脑里,不愿意轻易地退走的。合作社建立以后,还必须经过许多的斗争,才能使自己巩固起来。巩固了以后,只要一松劲,又可能垮台。山西省解虞县三娄寺合作社,就是在巩固以后,因为松劲,几乎垮了台的。仅在那里的党组织批判了自己的错误,重新向社员群众进行了反对资本主义加强社会主义的教育,恢复了政治工作,方才克服了那里的危机,走上了继续发展的道路。反对自私自利的资本主义的自发倾向,提倡以集体利益和个人利益相结合的原则为一切言论行动的标准的社会主义精神,是使分散的小农经济逐步地过渡到大规模合作化经济的思想的和政治的保证。这一工作是艰巨的,必须根据农民的生活经验,很具体地很细致地去做,不能采用粗暴的态度和简单的方法。它是要结合着经济工作一道去做的,不能孤立地去做。这种工作,在全国范围内,我们已经有了相当丰富的经验。本书所载的作品,几乎每一篇都表现了这一个特点。


(《严重的教训》一文按语)

\section*{十九}

这篇文章的观点是正确的。合作社必须强调做好政治工作。政治工作的基本任务是向农民群众不断地灌输社会主义思想,批评资本主义倾向。


(《张郭庄合作社的政治工作》一文按语)

\section*{二十}

这种情况值得注意。富裕农民中的资本主义倾向是严重的。只要我们在合作化运动中,乃至以后一个很长的时期内,稍微放松了对于农民的政治工作,资本主义倾向就会泛滥起来。


(《必须对资本主义倾向作坚决的斗争》一文按语)

\section*{二十一}

这是一篇很好的整社经验,值得推荐。一个新的社会制度的诞生,总是要伴随一场大喊大叫的,这就是宣传新制度的优越性,批判旧制度的落后性。使我国五亿多农民实行社会主义改造这样一种惊天动地的事业,不可能是在一种风平浪静的情况下出现的,它要求我们共产党人向着背上背着旧制度包袱的广大的农民群众,进行耐心的生动的容易被他们理解的宣传教育工作。目前全国各地都在做这种工作,出现了很多善于做宣传的农村工作同志。这篇文章里所描写的“四对比、五算账”,就是向农民说明两种制度谁好谁坏、使人一听就懂的一种很好的方法。这种方法有很强的说服力。它不是象有些不善于做宣传的同志那样,仅仅简单地提到所谓“或者走共产党的道路,或者走蒋介石的道路”,只是企图拿大帽子压服听众,手里并无动人的货色,而是拿当地农民的经验向农民作细致的分析,这就具有很强的说服力。


(《一个整社的好经验》一文按语)

\section*{二十二}

反革命破坏合作化运动的问题,是一个普遍的问题,不是贵州省都匀县第五区一个地方的问题,但是我们在别省的同类刊物上却很少看到这个问题的反映。在合作化过程中,一切从事农村工作的同志必须充分地注意这个同反革命的破坏活动作斗争的问题。要学都匀县这个区一样,在合作社内,以党团员为骨干,建立保卫组织。在县委的领导和监督之下,党的区委,在研究了情况,向党内外作好了宣传和解释,提高了群众对于反革命破坏活动的警惕以后,对于混入合作社领导机关里的反革命分子和其它坏分子,加以审查、清洗和处理,是完全必要的。不过清洗的必须是真正的反革命分子和真正的坏分子,不能将好人或者只有某些缺点的人说成坏人。处理尤其要恰当,必须经过县的批准。


(《必须和反革命的破坏活动作坚决的斗争》一文按语)

\section*{二十三}

为了建设伟大的社会主义社会,发动广大的妇女群众参加生产活动,具有极大的意义。在生产中,必须实现男女同工同酬。真正的男女平等,只有在整个社会的社会主义改造过程中才能实现。


(《妇女走上了劳动战线》一文按语)

\section*{二十四}

这一篇很好,可作各地参考。青年是整个社会力量中的一部分最积极最有生气的力量。他们最肯学习,最少保守思想,在社会主义时代尤其是这样。希望各地的党组织,协同青年团组织,注意研究如何特别发挥青年人的力量,不要将他们一般看待,抹杀了他们的特点。当然青年人必须向老年人和成年人学习,要尽量争取在老年人和成年人同意之下去做各种有益的活动。老年人和成年人的保守思想是比较多的,他们往往压抑青年人的进步活动,要在青年人做出了成绩以后他们才心服,本文就是很好地描写了这种情况。对于保守思想当然是不应当妥协的,那末好罢,就来试试看,成绩出来了,他们也就同意了。


(《中山县新平乡第九农业生产合作社的青年突击队》一文按语)

\section*{二十五}

这也是一篇好文章,可作各地参考。其中提到组织中学。生和高小毕业生参加合作化的工作,值得特别注意。一切可以到农村中去工作的这样的知识分子,应当高兴地到那里去。农村是一个广阔的天地,在那里是可以大有作为的。


(《在一个乡里进行合作化规划的经验》一文按语)

\section*{二十六}

这里又有一个陈学孟。在中国,这类英雄人物何止成千上万,可惜文学家们还没有去找他们,下乡去从事指导合作化工作的人们也是看得多写得少。


(《合作化的带头人陈学孟》一文按语)

\section*{二十七}

这篇文章写得很好,值得向每个党和团的县委、区委和乡支部推荐,一切合作社都应当这样做。本文作者懂得党的路线,他说得完全中肯。文字也好,使人一看就懂,没有党八股气。在这里要请读者注意,我们的许多同志,在写文章的时候,十分爱好党八股,不生动,不形象,使人看了头痛。也不讲究文法和修辞,爱好一种半文言半白话的体裁,有时废话连篇,有时又尽量简古,好象他们是立志要让读者受苦似的。本书中所收的一百七十多篇文章,有不少篇是带有浓厚的党八股气的。经过几次修改,才使它们较为好读。虽然如此,还有少数作品仍然有些晦涩难懂。仅仅因为它们的内容重要,所以选录了。那一年能使我们少看一点令人头痛的党八股呢?这就要求我们的报纸和刊物的编辑同志注意这件事,向作者提出写生动和通顺的文章的要求,并且自己动手帮作者修改文章。


(《合作社的政治工作》一文按语)

\section*{二十八}

这里介绍的合作社,就是王国藩领导的所谓“穷棒子社”。勤俭经营应当是全国一切农业生产合作社的方针,不,应当是一切经济事业的方针。勤俭办工厂,勤俭办商店,勤俭办一切国营事业和合作事业,勤俭办一切其它事业,什么事情都应当执行勤俭的原则。这就是节约的原则,节约是社会主义经济的基本原则之一。中国是一个大国,但是现在还很穷,要使中国富起来,需要几十年时间。几十年以后也需要执行勤俭的原则,但是特别要提倡勤俭,特别要注意节约的,是在目前这几十年内,是在目前这几个五年计划的时期内。现在有许多合作社存在着一种不注意节约的不良作风,应当迅速地加以改正。每一个省每一个县都可以找到一些勤俭办社的例子,应当把这些例子传开去,让大家照着做。应当奖励那些勤俭的、产量最高的、各方面都办得好的合作社,应当批评那些浪费的、产量很低的、各方面都做得差的合作社。


(《勤俭办社》一文按语)

\section*{二十九}

这是一个全乡一千多户建成一个大合作社(他们叫做集体农庄,即是合作社)的七年远景计划,可作各地参考。为什么要有这样的长远计划,人们看一看它的内容就知道了。人类的发展有了几十万年,在中国这个地方,直到现在方才取得了按照计划发展自己的经济和文化的条件。自从取得了这个条件,我国的面目就将一年一年地起变化。每一个五年将有一个较大的变化,积几个五年将有一个更大的变化。


(《红星集体农庄的远景规划》一文按语)

\section*{三十}

这是一篇好文章,值得大家一读,可供各地合作社做长期计划的参考。本文作者说得很对:“制订生产规划的整个过程,就是先进思想和保守思想斗争的过程。”保守思想现在几乎在到处作怪。为了克服这种保守思想,使生产力和生产向前发展一大步,一切地方,一切合作社,都要做出自己的长期计划来。


(《一个合作社的三年生产规划》一文按语)

\section*{三十一}

这个乡做了一个合作化、增产措施、水利、整党整团、文化教育等项工作的两年计划,全国各乡也应当这样做。有些人说计划难做,为什么这个乡能做呢?一九五六年,全国各县、区、乡都要做一个全面性的计划,包括的项目,比这个计划还应当多一些,例如副业、商业、金融、绿化、卫生等。那怕粗糙一点,不尽符合实际,总比没有好些。一个省只要有一两个县、一两个区、一两个乡做出了相当象样的计划,就可以迅速传播开去,叫其它县其它区其它乡仿照办理。说起来怎样困难,其实是并不那么困难的。


(《沂涛乡的全面规划》一文按语)

\section*{三十二}

这一篇很有用,可作各县参考。每县都应当在自己的全面规划中,做出一个适当的水利规划。兴修水利是保证农业增产的大事,小型水利是各县各区各乡和各个合作社都可以办的,十分需要定出一个在若干年内,分期实行,除了遇到不可抵抗的特大的水旱灾荒以外,保证遇旱有水,遇涝排水的规划。这是完全可以做得到的。在合作化的基础之上,群众有很大的力量。几千年不能解决的普通的水灾、旱灾问题,可能在几年之内获得解决。


(《应当使每人有一亩水地》一文按语)

\section*{三十三}

养猪是关系肥料、肉食和出口换取外汇的大问题,一切合作社都要将养猪一事放在自己的计划内,当然省、专、县、区都应有自己的计划。猪的饲料是容易解决的,某些青草,某些树叶,番薯藤叶和番薯都是饲料,不一定要精料,尤其不一定要用很多的精料。除了合作社公养以外,每个农家都要劝他们养一口至几口猪,分作几年达到这个目的。某些少数民族禁止养猪的和某些个别家庭因为宗教习惯不愿养猪的,当然不在此内。发展养猪事业要有一套奖励办法,浙江省上华合作社的经验可供各地参考。


(《这里养了一大批毛猪》一文按语)

\section*{三十四}

在合作化以前,全国很多地方存在着劳动力过剩的问题。在合作化以后,许多合作社感到劳动力不足了,有必要发动过去不参加田间劳动的广大的妇女群众参加到劳动战线上去。这是出于许多人意料之外的一件大事。过去,人们总以为合作化以后,劳动力一定会过剩。原来已经过剩了,再来一个过剩,怎么办呢!在许多地方,合作化的实践,打破了人们的这种顾虑,劳动力不是过剩,而是不足。有些地方,合作化以后,一时感到劳动力过剩,那是因为还没有扩大生产规模,还没有进行多种经营,耕作也还没有精致化的缘故。对于很多地方说来,生产的规模大了,经营的部门多了,劳动的范围向自然界的广度和深度扩张了,工作做得精致了,劳动力就会感到不足。这种情形,现在还只是在开始,将来会一年一年地发展起来。农业机械化以后也将是这样。将来会出现从来没有被人们设想过的种种事业,几倍、十几倍以至几十倍于现在的农作物的高产量。工业、交通和交换事业的发展,更是前人所不能设想的。科学、文化、教育、卫生等项事业也是如此。中国的妇女是一种伟大的人力资源。必须发掘这种资源,为了建设一个伟大的社会主义国家而奋斗。要发动妇女参加劳动,必须实行男女同工同酬的原则。浙江建德县的经验,一切合作社都可以采用。


(《发动妇女投入生产,解决了劳动力不足的困难》一文按语)

\section*{三十五}

这也是一个带普遍性的问题。根据这两个合作社的情况,按照现在的生产条件,就已经多余了差不多三分之一的劳动力。过去三个人做的工作,合作化以后,两个人做就行了,表示了社会主义的优越性。多余的三分之一甚至更多的劳动力向那里找出路呢?主要地还是在农村。社会主义不仅从旧社会解放了劳动者和生产资料,也解放了旧社会所无法利用的广大的自然界。人民群众有无限的创造力。他们可以组织起来,向一切可以发挥自己力量的地方和部门进军,向生产的深度和广度进军,替自己创造日益增多的福利事业。这里还没有涉及农业机械化。机械化以后,劳动力更会大量节省,是不是有出路呢?根据一些机耕农场的经验仍然是有出路的,因为生产的范围大了,部门多了,工作细了,这就不怕有力无处使。


(《多余劳动力找到了出路》一文按语)

\section*{三十六}

这个县的情况也告诉我们,乡村中的剩余劳动力是能够在乡村中找到出路的。一年内每个男女劳动力的工作日,依照经营方法的改进,生产门路的扩大,还可以增加,不是如同文内所说的男的一百多个工作日,女的几十个工作日,而是可以做到男的二百多个工作日,女的一百多个工作日,或者更多一些。这个数目,现在别处的有些合作社,已经做到了。副业必须要有确实的销路,不能盲目发展,这是对的。农村副业,就全国说来,一个很大的部分是为农村服务的,但是必须有一个不小的部分为城市服务和为出口服务,将来这部分可能扩大起来。问题是国家要有统一的计划,一步一步地去掉盲目性。


(《湘阴县解决了剩余劳动力的出路问题》一文按语)

\section*{三十七}

这一篇很好,各地均可仿照办理。“没有会计”,是反对合作化迅速发展的人们的借口之一。全国合作化,需要几百万人当会计,到那里去找呢?其实人是有的,可以动员大批的高小毕业生和初中毕业生去做这个工作。问题是要迅速地加以训练,并且在工作中提高他们的文化和技术的水平。以区为单位,由生产合作社、供销合作社和信用合作社的会计员组成会计互助网,就是这种提高会计员的文化、技术水平的好办法。彰武县第三区的会计网,不但帮助会计员提高了他们的文化、技术水平,而且做了许多经济工作和政治工作。县和区的党组织,都要注意去领导这项工作。


(《一个由农业生产合作社、供销合作社和信用合作社的会计员组成会计互助网的经验》一文按语)

\section*{三十八}

这个经验应当普遍推行。列宁说过:“在一个文盲充斥的国家内,是建成不了共产主义社会的。”\mnote{2}我国现在文盲这样多,而社会主义的建设又不能等到消灭了文盲以后才去开始进行,这就产生了一个尖锐的矛盾。现在我国不仅有许多到了学习年龄的儿童没有学校可进,而且还有一大批超过学龄的少年和青年也没有学校可进,成年人更不待说了。这个严重的问题必须在农业合作化的过程中加以解决,也只有在农业合作化的过程中才能解决。农民组织了合作社,因为经济上的需要,迫切地要求学文化。农民组织了合作社,有了集体的力量,情况就完全改变了,他们可以自己组织学文化。第一步为了记工的需要,学习本村本乡的人名、地名、工具名、农活名和一些必要的语汇,大约两三百字。第二步,再学进一步的文字和语汇。要编这样两种课本。第一种课本应当由从事指导合作化工作的同志,帮助当地的知识分子,各就自己那里的合作社的需要去编。每处自编一本,不能用统一的课本。这种课本不要审查。第二种课本也应当由从事指导合作化工作的同志,帮助当地的知识分子,根据一个较小范围的地方(例如一个县,或者一个专区)的事物和语汇,加上一部分全省(市、区)的和全国性的事物和语汇编出来,也只要几百字。这种课本,各地也不要统一,由县级、专区级或者省(市、区)级的教育机关迅速地加以审查。做了这样两步之后再做第三步,由各省(市、区)教育机关编第三种通常应用的课本。以后还要有继续提高的课本。中央的文化教育机关应当给这件事以适当的指导。山东宫南县高家柳沟村的青年团支部做了一个创造性的工作。看了这种情况,令人十分高兴。教员是有的。就是本乡的高小毕业生。进度是快的,两个半月就有一百多个青年和壮年学会了两百多字,能记自己的工账,有些人当了合作社的记账员。记工学习班这个名称也很好。这种学习班,各地应当普遍地仿办。各级青年团组织应当领导这一工作,一切党政机关应当予以支持。


(《南县高家柳沟村青年团支部创办记工学习班的经验》一文按语)

\section*{三十九}

这里说的是李顺达领导的金星农林牧生产合作社。这个合作社办了三年,变成了一个包括二百八十三户的大社。这个社所在的地方是那样一个太行山上的穷地方,由于大家的努力,三年工夫,已经开始改变了面貌。劳动力的利用率,比抗日以前的个体劳动时期提高了百分之一百一十点六,比建社以前的互助组时期也提高了百分之七十四。合作社的公共积累已经由第一年的一百二十元,增加到了一万一千多元。一九五五年,社员每人平均收入粮食八百八十四斤,比抗日以前增加了百分之七十七,比建社以前增加了百分之二十五点一。这个社已经做了一个五年计划,实行三年的结果,生产总值已经达到五年计划的百分之一百零点六。这个合作社的经验告诉我们,如果自然条件较差的地方能够大量增产,为什么自然条件较好的地方不能够更加大量地增产呢?


(《勤俭办社,建设山区》一文按语)

\section*{四十}

这是一个办得很好的合作社,可以从这里吸取许多有益的经验。曲阜县是孔夫子的故乡,他老人家在这里办过多少年的学校,教出了许多有才干的学生,这件事是很出名的。可是他不大注意人民的经济生活。他的学生樊返问起他如何从事农业的话,他不但推开不理,还在背后骂樊迟做“小人”\mnote{3}。现在他的故乡的人民办起社会主义的合作社来了。经过了两千多年仍然是那样贫困的人民,办了三年合作社,经济生活和文化生活都开始改变了面貌。这就证明,现在的社会主义确实是前无古人的。社会主义比起孔夫子的“经书”来,不知道要好过多少倍。有兴趣去看孔庙孔林的人们,我劝他们不妨顺道去看看这个合作社。


(《以一个在三年内增产百分之六十七的农业生产合作社》一文按语)

\section*{四十一}

这篇文章写得很好,值得一阅。现在办的半社会主义的合作社,为了易于办成,为了使干部和群众迅速取得经验,二、三十户的小社为多。但是小社人少地少资金少,不能进行大规模的经营,不能使用机器。这种小社仍然束缚生产力的发展,不能停留太久,应当逐步合并。有些地方可以一乡为一个社,少数地方可以几乡为一个社,当然会有很多地方一乡有几个社的。不但平原地区可以办大社,山区也可以办大社。安徽佛子岭水库所在的一个乡,全是山地,纵横几十里,就办成了一个大规模的农林牧综合经营的合作社。当然,这种合并要有步骤,要有适当的干部,要得到群众的同意。


(《大社的优越性》一文按语)

\section*{四十二}

办大型社和高级社最为有利这一点,海南岛红旗合作社的经验也是证明。这个大型合作社还只有一年的历史,它就准备转变为高级社。当然,这不是说,一切合作社都要照这样做,它们仍然要看自己的条件是否成熟,作出自己究竟在何时实行并社升级为宜的决定。但是,一般地说来,有三年时间也就差不多了。重要的是做出榜样给农民看。当着农民看见办大型社和高级社比办小型社和低级社更为有利的时候,他们就会要求并社和升级了。


(《琼山县第一区红旗农业生产合作社,在同自然灾害和同资本主义思想作斗争中巩固起来了》一文按语)

\section*{四十三}

对于条件已经成熟了的合作社,就应当考虑使它们从初级形式转到高级形式上去,以便使生产力和生产获得进一步的发展。因为初级形式的合作社保存了半私有制,到了一定的时候,这种半私有制就束缚了生产力的发展,人们就要求改变这种制度,使合作社成为生产资料完全公有化的集体经营的经济团体。生产力一经进一步解放,生产就会有更大的发展。转变的时间,有些地方可能快些,有些地方可能要慢一点。大约办了三年左右的初级合作社,就基本上具有这种条件了。各省各市和各自治区的党组织对此应有研究和布置,在一九五六和一九五七两年内,应当在群众同意的条件下办一些试点性质的高级社。现在办的合作社一般地是小型社,向高级社转变的时候,应当取得群众同意,把许多小型社合并起来成为大型社。如果能够在这两年使得每个区都有一个至几个这样的合作社,并且在群众中显出它们比较初级社具有更大的优越性,那就可以使以后几年的并社升级工作,获得有利的条件。这个工作,要同发展生产的全面规划配合起来。当着人们看见了大型社和高级社比较小型社和初级社更为有利的时候,当着人们看见长期规划将给他们带来比较现在高得多的物质和文化的生活水平的时候,他们就会同意并社和升级的。城市郊区的升级会要快一些。北京这个合作社的经验,可以作其它具有同类情况的合作社的参考。


(《一个从初级形式过渡到高级形式的合作社》一文按语)


\begin{maonote}
\mnitem{1}这里所说的“倒宣传”,是指在群众普遍发动起来,纷纷要求入社的时候,除了宣传组织合作社的好处和有利条件以外,还要向群众公开说明各种可能遇到的困难和不利情况,让大家充分考虑,做到自愿入社。
\mnitem{2}见列宁《青年团的任务》。
\mnitem{3}参看《论语·子路第十三》。
\end{maonote}
