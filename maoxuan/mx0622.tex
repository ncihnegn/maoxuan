
\title{党内通信——致六级干部的公开信}
\date{一九五九年四月二十九日}
\thanks{这是毛泽东同志关于农业方面六个问题给省级、地级、县级、社级、队级、小队级干部们的一封信,题目是毛泽东拟的。}
\maketitle


\mxname{省级、地级、县级、社级、队级、小队级的同志们:}

我想和同志们商量几个问题,都是关于农业的。

第一个问题,包产问题。南方正在插秧,北方也在春耕。包产一定要落实。根本不要管上级规定的那一套指标。不管这些,只管现实可能性。例如,去年亩产实际只有三百斤的,今年能增产一百斤、二百斤,也就很好了。吹上八百斤、一千斤、一千二百斤,甚至更多,吹牛而已,实在办不到,有何益处呢?又例如,去年亩产五百斤的,今年增加二百斤、三百斤,也就算成绩很大了。再增上去,就一般说,不可能的。

第二个问题,密植问题。不可太稀,不可太密。许多青年干部和某些上级机关缺少经验,一个劲儿要密。有些人竟说愈密愈好。不对。老农怀疑,中年人也有怀疑的。这三种人开一个会,得出一个适当密度,那就好了。既然要包产,密植问题就得由生产队、生产小队商量决定。上面死硬的密植命令,不但无用,而且害人不浅。因此,根本不要下这种死硬的命令。省委可以规定一个密植幅度,不当作命令下达,只给下面参考。此外,上面要精心研究到底密植程度以何为好,积累经验,根据因气候不同,因地点不同,因土、肥、水、种等条件不同,因各种作物的情况不同,因田间管理水平高低不同,做出一个比较科学的密植程度的规定,几年之内达到一个实际可行的标准,那就好了。

第三个问题,节约粮食问题。要十分抓紧,按人定量,忙时多吃,闲时少吃,忙时吃干,闲时半干半稀,杂以番薯、青菜、萝卜、瓜豆、芋头之类。此事一定要十分抓紧。每年一定要把收割、保管、吃用三件事(收、管、吃)抓得很紧很紧,而且要抓得及时。机不可失,时不再来。一定要有储备粮,年年储一点,逐年增多。经过十年八年奋斗,粮食问题可能解决。在十年内,一切大话、高调,切不可讲,讲就是十分危险的。须知我国是一个有六亿五千万人口的大国,吃饭是第一件大事。

第四个问题,播种面积要多的问题。少种、高产、多收的计划,是一个远景计划,是可能的,但在十年内不能全部实行,也不能大部实行。十年以内,只能看情况逐步实行。三年以内,大部不可行。三年以内,要力争多种。目前几年的方针是:广种薄收与少种多收(高额丰产田)同时实行。

第五个问题,机械化问题。农业的根本出路在于机械化,要有十年时间。四年以内小解决,七年以内中解决,十年以内大解决。今年、明年、后年、大后年这四年内,主要依靠改良农具、半机械化农具。每省每地每县都要设一个农具研究所,集中一批科学技术人员和农村有经验的铁匠木匠,搜集全省、全地、全县各种比较进步的农具,加以比较,加以试验,加以改进,试制新式农具。试制成功,在田里实验,确实有效,然后才能成批制造,加以推广。提到机械化,用机械制造化学肥料这件事,必须包括在内。逐年增加化学肥料,是一件十分重要的事。

第六个问题,讲真话问题。包产能包多少,就讲能包多少,不讲经过努力实在做不到而又勉强讲做得到的假话。收获多少,就讲多少,不可以讲不合实际情况的假话。对各项增产措施,对实行八字宪法\mnote{1},每项都不可讲假话。老实人,敢讲真话的人,归根到底,于人民事业有利,于自己也不吃亏。爱讲假话的人,一害人民,二害自己,总是吃亏。应当说,有许多假话是上面压出来的。上面“一吹二压三许愿”,使下面很难办。因此,干劲一定要有,假话一定不可讲。

以上六件事,请同志们研究,可以提出不同意见,以求得真理为目的。我们办农业工业的经验还很不足。一年一年积累经验,再过十年,客观必然性可能逐步被我们认识,在某种程度上,我们就有自由了。什么叫自由?自由是必然的认识。

同现在流行的一些高调比较起来,我在这里唱的是低调,意在真正调动积极性,达到增产的目的。如果事实不是我讲的那样低,而达到了较高的目的,我变为保守主义者,那就谢天谢地,不胜光荣之至。

\begin{maonote}
\mnitem{1}八字宪法,指毛泽东一九五八年提出的农作物增产的八项措施,即土、肥、水、种(推广良种)、密(合理密植)、保(植物保护,防治病虫害)、管(田间管理)、工(工具改革)。
\end{maonote}
