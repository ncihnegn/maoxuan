
\title{经济建设是科学,要老老实实学习}
\date{一九五九年六月十一日}
\thanks{这是毛泽东同志同秘鲁议员团谈话的节录。}
\maketitle


中国是个经济不发达的国家,工农业水平不高,有许多人是文盲。我们现在正在积极组织自己的经济,积极做提高人民文化水平的工作,发展我们的工业,使农业用机器装备起来。现在工作才开始做,仅仅是开始。至于做出成绩来,那需要时间,需要朋友的帮助,不是很短时间内所能做到的。我们这样一个大国要提高经济、文化水平,建设现代化的工业、农业和文化教育,需要一个过程。我们现在提出了“多、快、好、省”这个建设经济、文化的口号。可以快一点,但不可能很快,想很快是吹牛皮,你们不会相信,我也不相信。比如走路,总要花些时间。我们已经干了十年了,但是如你们所知道的,我们工业才这么一点,农业还是手工业式的,也许再有十年才会有些进步。中国不仅要自己料理自己,自己过生活,还应该对别的国家和民族进行帮助,对世界有些益处。同别的国家一样,不仅要为自己而且还要对世界做些贡献。和别的国家互相帮助,发展经济关系,尤其是我们亚洲、非洲、拉丁美洲国家之间互相了解、交流经验,很有必要。搞经济关门是不行的,需要交换。中美、南美有二十个国家,同亚洲的国家例如中国发生经济联系是可能的,文化上的联系也是可能的,互相交换有益的东西。日本是亚洲国家,是我们的邻国,我们同他们打过仗,最近关系也不大好,但是人民间的交往还是有的,现在还有日本民间代表团在中国访问。虽然中国和日本没有外交关系,日本政府对我们不友好,但两国人民还是互相来往,人民间互相来往很自由、很自然。有可能的话,我们还会派代表团去南美洲访问你们国家的。

过去干的一件事叫革命,现在干的叫建设,是新的事,没有经验。怎么搞工业,比如炼铁、炼钢,过去就不大知道。这是科学技术,是向地球开战,当然这只是向地球上的中国部分开战,不会向你们那里开战。如果对自然界没有认识,或者认识不清楚,就会碰钉子,自然界就会处罚我们,会抵抗。比如水坝,如修得不好,质量不好,就会被水冲垮,将房屋、土地都淹没,这不是处罚吗?

可以告诉你们,我们真正认真搞经济工作,是从去年八月才开始的\mnote{1}。我就是一个。去年八月前,主要不是搞建设,而是搞革命。许多同志都是这样,把重点放在革命、社会改革上,而不是放在改造自然界方面。在与自然界作斗争方面,我们的第一个先生是苏联,我们首先要学习苏联,但是美国也是我们的先生。美国炼的钢含硫量是百分之零点零四,我们只有个别地方炼的钢含硫量达到百分之零点零三七,大部分地方炼的钢质量不好。这是新问题,不能调皮,要老老实实学习。如果粗心大意、调皮、充好汉,一定会跌跤子的。革命事业是不容易的,是科学,经济建设也是科学。

搞社会主义建设,很重要的一个问题是综合平衡。比如社会主义建设需要钢、铁等种种东西,缺一样就不能综合平衡。我们有些人办事时总是忘了一两个条件。比如炼铁,没有耐火砖不行,于是他们就把原来做盘子用的陶土拿去搞耐火砖,这样盘子就不够了,因此就要到另外地方去找耐火材料,把原来的陶瓷生产恢复起来。这个事情是很复杂的,每个行业都会有这样的事情。工业、农业、商业、交通事业都可能碰到。农业也要综合平衡,农业包括农、林、牧、副、渔五个方面。

中国有希望就是了。这还要靠你们帮助,靠世界上爱好和平人民的帮助,最主要的是保持和平环境,这是大家的最大利益。你们要和平,我们也是这样。

\begin{maonote}
\mnitem{1}一九五八年八月十七日至三十日在北戴河召开的中共中央政治局扩大会议主要讨论了一九五九年的国民经济计划以及工业生产、农业生产、农村工作和商业工作等问题。会议作出《中共中央关于一九五九年计划和第二个五年计划问题的决定》和《中共中央关于在农村建立人民公社问题的决议》。
\end{maonote}
