
\title{和王海蓉\mnote{1}同志的谈话}
\date{一九六四年六月二十四日}
\thanks{这是毛泽东同志同王海蓉谈话的节选。}
\maketitle


\mxsay{王:}我们学校的阶级斗争很尖锐,听说发现了反动标语,都有用英语的。就在我们英语系的黑板上。

\mxsay{毛:}他写的是什么反动标语?

\mxsay{王:}我就知道这一条,蒋万岁。

\mxsay{毛:}英语怎么讲?

\mxsay{王:}Long live 蒋。

\mxsay{毛:}还写了什么?

\mxsay{王:}别的不晓得,我就知道这一条,章会娴\mnote{2}告诉我的。

\mxsay{毛:}好嘛!让他多写一些贴在外面,让大家看一看,他杀人不杀人?

\mxsay{王:}不知道杀人不杀人,如果查出来,我看要开除他,让他去劳动改造。

\mxsay{毛:}只要他不杀人,不要开除他,也不要让他去劳动改造,让他留在学校里,继续学习,你们可以开一个会,让他讲一讲,蒋介石为什么好?蒋介石做了哪些好事?你们也可以讲一讲蒋介石为什么不好?你们学校有多少人?

\mxsay{王:}大概有三千多人,其中包括教职员。

\mxsay{毛:}你们三千多人中间最好有七、八个蒋介石分子。

\mxsay{王:}出一个就不得了,还要有七、八个,那还了得?

\mxsay{毛:}我看你这个人啊!看到一张反动标语就紧张了。

\mxsay{王:}为什么要七、八个呢?

\mxsay{毛:}多几个就可以树立对立面,可以作反面教员,只要他不杀人。

\mxsay{王:}我们学校贯彻了阶级路线,这次招生,百分之七十都是工人和贫下中农子弟。其它就是干部子弟,烈属子弟等。

\mxsay{毛:}你们这个班有多少工农子弟?

\mxsay{王:}除了我以外还有两个干部子弟,其他都是工人、贫下中农子弟,他们表现很好,我向他们学到很多东西。

\mxsay{毛:}他们和你的关系好不好?他们喜欢不喜欢和你接近?

\mxsay{王:}我认为我们关系还不错,我跟他们合得来,他们也跟我合得来。

\mxsay{毛:}这样就好。

\mxsay{王:}我们班有个干部子弟,表现可不好了,上课不用心听讲,下课也不练习,专看小说,有时在宿舍睡觉,星期六下午开会有时也不参加,星期天也不按时返校,有时星期天晚上,我们班或团员开会,他也不到,大家都对他有意见。

\mxsay{毛:}你们教员允许你们上课打瞌睡,看小说吗?

\mxsay{王:}不允许。

\mxsay{毛:}要允许学生上课看小说,要允许学生上课打瞌睡,要爱护学生身体,教员要少讲,要让学生多看,我看你讲的这个学生,将来可能有所作为。他就敢星期六不参加会,也敢星期日不按时返校。回去以后,你就告诉这学生,八、九点钟回校还太早,可以十一点,十二点再回去,谁让你们星期日晚上开会哪!

\mxsay{王:}原来我在师范学院时,星期天晚上一般不能用来开会的。星期天晚上的时间一般都归同学自己利用。有一次我们开支委会,几个干部商量好,准备在一个星期天晚上过组织生活,结果很多团员反对。有的团员还去和政治辅导员提出来,星期天晚上是我们自己利用的时间,晚上我们回不来。后来政治辅导员接受了团员的意见要我们改期开会。

\mxsay{毛:}这个政治辅导员作得对。

\mxsay{王:}我们这里尽占星期日的晚上开会,不是班会就是支委会,要不就是级里开会,要不就是党课学习小组。这学期从开学到我出来为止,我计算一下没有一个星期天晚上不开会的。

\mxsay{毛:}回去以后,你带头造反。星期天你不要回去,开会就是不去。

\mxsay{王:}我不敢,这是学校的制度规定,星期日一定要回校,否则别人会说我破坏学校制度。

\mxsay{毛:}什么制度不制度,管他那一套,就是不回去,你说:我就是破坏学校制度。

\mxsay{王:}这样做不行,会挨批评的。

\mxsay{毛:}我看你这个人将来没有什么大作为。你怕人家说你破坏制度,又怕挨批评,又怕记过,又怕开除,又怕入不了党。有什么好怕的,最多就是开除。学校就应该允许学生造反。回去带头造反。

\mxsay{王:}人家会说我,主席的亲戚还不听主席的话,带头破坏学校制度。人家会说我骄傲自满,无组织无纪律。

\mxsay{毛:}你这个人哪?又怕人家批评你骄傲自满,又怕人家说你无组织无纪律,你怕什么呢?你说就是听了主席的话,我才造反的。我看你说的那个学生,将来可能比你有所作为,他就敢不服从你们学校的制度。我看你们这些人有些形而上学。

\begin{maonote}
\mnitem{1}王海蓉是外国语学院英语专修科学生,毛泽东的侄孙女。
\mnitem{2}章会娴是章士钊之女,王海蓉的同学。
\end{maonote}
