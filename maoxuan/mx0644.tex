
\title{在七千人大会上的讲话——关于党的民主集中制}
\date{一九六二年一月三十日}
\thanks{这是毛泽东同志在扩大的中央工作会议\mnote{1}上的讲话。}
\maketitle


同志们,我现在讲几点意见。(热烈鼓掌)一共讲六点,中心是讲一个民主集中制的问题,同时也讲到一些其他问题。

第一点,这次会议的开会方法。

这次扩大的中央工作会议,到会的有七千多人。在这次会议开始的时候,刘少奇同志和别的几位同志,准备了一个报告稿子。这个稿子,还没有经过中央政治局讨论,我就向他们建议,不要先开中央政治局会议讨论了,立即发给参加大会的同志们,请大家评论,提意见。同志们,你们有各方面的人、各地方的人,有各个省委、地委、县委的人,有企业党委的人,有中央各部门的人,你们当中的多数人是比较接近下层的,你们应当比我们中央常委、中央政治局和中央书记处的同志更加了解情况和问题。还有,你们站在各种不同的岗位,可以从各种的角度提出问题。因此,要请你们提意见。报告稿子发给你们了,果然议论纷纷,除了同意中央提出的基本方针以外,还提出许多意见。后来又由少奇同志主持,组织了二十一个人的起草委员会,这里面有各中央局的负责同志参加,经过八天讨论,写出了书面报告的第二稿。应当说,报告第二稿是中央集中了七千多人议论的结果。如果没有你们的意见,这个第二稿不可能写成。在第二稿里面,第一部分和第二部分有很大的修改,这是你们的功劳。听说大家对第二稿的评价不坏,认为它是比较好的。如果不是采用这种方法,而是采用通常那种开会的方法,就是先来一篇报告,然后进行讨论,大家举手赞成,那就不可能做到这样好。

这是一个开会的方法问题。先把报告草稿发下去,请到会的人提意见,加以修改,然后再作报告。报告的时候不是照着本子念,而是讲一些补充意见,作一些解释。这样,就更能充分地发扬民主,集中各方面的智慧,对各种不同的看法有所比较,会也开得活泼一些。我们这次会议是要总结十二年的工作经验,特别是要总结最近四年来的工作经验,问题很多,意见也会很多,宜于采取这种方法。

是不是所有的会议都可以采用这种方法呢?那也不是。采用这种方法,要有充裕的时间。我们的人民代表大会的会议,有时也许可以采用这种方法。省委、地委、县委的同志们,你们以后召集会议,如果有条件的话,也可以采用这种方法。当然,你们的工作忙,一般地不能用很长的时间去开会,但是在有条件的时候,不妨试一试看。

这个方法是一个什么方法呢?是一个民主集中制的方法,是一个群众路线的方法。先民主,后集中,从群众中来,到群众中去,领导同群众相结合。这是我讲的第一点。

第二点,民主集中制问题。

看起来,我们有些同志,对于马克思、列宁所说的民主集中制,还不理解。有些同志已经是老革命了,“三八式”的,或者别的什么式的,总之已经作了几十年的共产党员,但是他们还不懂得这个问题。他们怕群众,怕群众讲话,怕群众批评。哪有马克思列宁主义者怕群众的道理呢?有了错误,自己不讲,又怕群众讲。越怕,就越有鬼。

我看不应当怕。有什么可怕的呢?我们的态度是:坚持真理,随时修正错误。\mnote{2}我们工作中的是和非的问题,正确和错误的问题,这是属于人民内部矛盾问题。解决人民内部矛盾,不能用咒骂,也不能用拳头,更不能用刀枪,只能用讨论的方法,说理的方法,批评和自我批评的方法,一句话,只能用民主的方法,让群众讲话的方法。

不论党内党外,都要有充分的民主生活,就是说,都要认真实行民主集中制。要真正把问题敞开,让群众讲话,哪怕是骂自己的话,也要让人家讲。骂的结果,无非是自己倒台,不能做这项工作了,降到下级机关去做工作,或者调到别的地方去做工作,那又有什么不可以呢?一个人为什么只能上升不能下降呢?为什么只能做这个地方的工作而不能调到别个地方去呢?我认为这种下降和调动,不论正确与否,都是有益处的,可以锻炼革命意志,可以调查和研究许多新鲜情况,增加有益的知识。我自己就有这一方面的经验,得到很大的益处。不信,你们不妨试试看。

司马迁说过:“文王拘而演周易,仲尼厄而作春秋。屈原放逐,乃赋离骚。左丘失明,厥有国语。孙子膑脚,兵法修列。不韦迁蜀,世传吕览。韩非囚秦,说难孤愤。诗三百篇,大抵贤圣发愤之所为作也。”\mnote{3}这几句话当中,所谓文王演周易,孔子作春秋,究竟有无其事,近人已有怀疑,我们可以不去理它,让专门家去解决吧,但是司马迁是相信有其事的。文王拘,仲尼厄,则确有其事。司马迁讲的这些事情,除左丘失明一例以外,都是指当时上级领导者对他们作了错误处理的。我们过去也错误地处理过一些干部,对这些人不论是全部处理错了的,或者是部分处理错了的,都应当按照具体情况,加以甄别和平反。但是,一般地说,这种错误处理,让他们下降,或者调动工作,对他们的革命意志总是一种锻炼,而且可以从人民群众中吸取许多新知识。我在这里申明,我不是提倡对干部,对同志,对任何人,可以不分青红皂白,作出错误处理,像古代人拘文王,厄孔子,放逐屈原,去掉孙膑的膝盖骨那样。我不是提倡这样做,而是反对这样做的。我是说,人类社会的各个历史阶段,总是有这样处理错误的事实。在阶级社会,这样的事实多得很。在社会主义社会,也在所难免。不论在正确路线领导的时期,还是在错误路线领导的时期,都在所难免。不过有一个区别。在正确路线领导的时期,一经发现有错误处理的,就能甄别、平反,向他们赔礼道歉,使他们心情舒畅,重新抬起头来。而在错误路线领导的时期,则不可能这样做,只能由代表正确路线的人们,在适当的时机,通过民主集中制的方法,起来纠正错误。至于由于自己犯了错误,经过同志们的批评和上级的鉴定,作出正确处理,因而下降或者调动工作的人,这种下降或者调动,对于他们改正错误,获得新的知识,会有益处,那就不待说了。

现在有些同志,很怕群众开展讨论,怕他们提出同领导机关、领导者意见不同的意见。一讨论问题,就压抑群众的积极性,不许人家讲话。这种态度非常恶劣。民主集中制是上了我们的党章的,上了我们的宪法的,他们就是不实行。同志们,我们是干革命的,如果真正犯了错误,这种错误是不利于党的事业,不利于人民的事业的,就应当征求人民群众和同志们的意见,并且自己作检讨。这种检讨,有的时候,要有若干次。一次不行,大家不满意,再来第二次;还不满意,再来第三次;一直到大家没有意见了,才不再作检讨。有的省委就是这样做的。有一些省比较主动,让大家讲话。早的,在一九五九年就开始作自我批评,晚的,也在一九六一年开始作自我批评。还有一些省是被迫作检讨的,像河南、甘肃、青海。另外一些省,有人反映,好像现在才刚刚开始作自我批评。不管是主动的,被动的,早作检讨,晚作检讨,只要正视错误,肯承认错误,肯改正错误,肯让群众批评,只要采取了这种态度,都应当欢迎。

批评和自我批评是一种方法,是解决人民内部矛盾的方法,而且是唯一的方法。除此以外,没有别的方法。但是,如果没有充分的民主生活,没有真正实行民主集中制,就不可能实行批评和自我批评这种方法。

我们现在不是有许多困难吗?不依靠群众,不发动群众和干部的积极性,就不可能克服困难。但是,如果不向群众和干部说明情况,不向群众和干部交心,不让他们说出自己的意见,他们还对你感到害怕,不敢讲话,就不可能发动他们的积极性。我在一九五七年这样说过,要造成“又有集中又有民主,又有纪律又有自由,又有统一意志、又有个人心情舒畅、生动活泼,那样一种政治局面”\mnote{4}。党内党外部应当有这样的政治局面。没有这样的政治局面,群众的积极性是不可能发动起来的。克服困难,没有民主不行。当然没有集中更不行,但是,没有民主就没有集中。

没有民主,不可能有正确的集中,因为大家意见分歧,没有统一的认识,集中制就建立不起来。什么叫集中?首先是要集中正确的意见。在集中正确意见的基础上,做到统一认识,统一政策,统一计划,统一指挥,统一行动,叫做集中统一。如果大家对问题还不了解,有意见还没有发表,有气还没有出,你这个集中统一怎么建立得起来呢?没有民主,就不可能正确地总结经验。没有民主,意见不是从群众中来,就不可能制定出好的路线、方针、政策和办法。我们的领导机关,就制定路线、方针、政策和办法这一方面说来,只是一个加工工厂。大家知道,工厂没有原料就不可能进行加工。没有数量上充分的和质量上适当的原料,就不可能制造出好的成品来。如果没有民主,不了解下情,情况不明,不充分搜集各方面的意见,不使上下通气,只由上级领导机关凭着片面的或者不真实的材料决定问题,那就难免不是主观主义的,也就不可能达到统一认识,统一行动,不可能实现真正的集中。我们这次会议的主要议题,不是要反对分散主义,加强集中统一吗?如果离开充分发扬民主,这种集中,这种统一,是真的还是假的?是实的还是空的?是正确的还是错误的?当然只能是假的、空的、错误的。

我们的集中制,是建立在民主基础上的集中制。无产阶级的集中,是在广泛民主基础上的集中。各级党委是执行集中领导的机关。但是,党委的领导,是集体领导,不是第一书记个人独断。在党委会内部只应当实行民主集中制。第一书记同其他书记和委员之间的关系是少数服从多数。拿中央常委或者政治局来说,常常有这样的事情,我讲的话,不管是对的还是不对的,只要大家不赞成,我就得服从他们的意见,因为他们是多数。听说现在有一些省委、地委、县委,有这样的情况:一切事情,第一书记一个人说了就算数。这是很错误的。哪有一个人说了就算数的道理呢?我这是指的大事,不是指有了决议之后的日常工作。只要是大事,就得集体讨论,认真地听取不同的意见,认真地对于复杂的情况和不同的意见加以分析。要想到事情的几种可能性,估计情况的几个方面,好的和坏的,顺利的和困难的,可能办到的和不可能办到的。尽可能地慎重一些,周到一些。如果不是这样,就是一人称霸。这样的第一书记,应当叫做霸王,不是民主集中制的“班长”。从前有个项羽,叫做西楚霸王,他就不爱听别人的不同意见。他那里有个范增,给他出过些主意,可是项羽不听范增的话。另外一个人叫刘邦,就是汉高祖,他比较能够采纳各种不同的意见。有个知识分子名叫郦食其,去见刘邦。初一报,说是读书人,孔夫子这一派的。回答说,现在军事时期,不见儒生。这个郦食其就发了火,他向管门房的人说,你给我滚进去报告,老子是高阳酒徒,不是儒生。管门房的人进去照样报告了一篇。好,请。请了进去,刘邦正在洗脚,连忙起来欢迎。郦食其因为刘邦不见儒生的事,心中还有火,批评了刘邦一顿。他说,你究竟要不要取天下,你为什么轻视长者!这时候,郦食其已经六十多岁了,刘邦比他年轻,所以他自称长者。刘邦一听,向他道歉,立即采纳了郦食其夺取陈留县的意见。此事见《史记》郦生陆贾列传。刘邦是在封建时代被历史家称为“豁达大度,从谏如流”的英雄人物。刘邦同项羽打了好几年仗,结果刘邦胜了,项羽败了,不是偶然的。我们现在有些第一书记,连封建时代的刘邦都不如,倒有点像项羽。

这些同志如果不改,最后要垮台的。不是有一出戏叫《霸王别姬》吗?这些同志如果总是不改,难免有一天要“别姬”就是了。(笑声)我为什么要讲得这样厉害呢?是想讲得挖苦一点,对一些同志戳得痛一点,让这些同志好好地想一想,最好有两天睡不着觉。他们如果睡得着觉,我就不高兴,因为他们还没有被戳痛。

我们有些同志,听不得相反的意见,批评不得。这是很不对的。在我们这次会议中间,有一个省,会本来是开得生动活泼的,省委书记到那里一坐,鸦雀无声,大家不讲话了。这位省委书记同志,你坐到那里去干什么呢?为什么不坐到自己房子里想一想问题,让人家去纷纷议论呢?平素养成了这样一种风气,当着你的面不敢讲话,那末,你就应当回避一下。有了错误,一定要作自我批评,要让人家讲话,让人批评。去年六月十二号,在中央北京工作会议的最后一天,我讲了自己的缺点和错误。我说,请同志们传达到各省、各地方去。事后知道,许多地方没有传达。

似乎我的错误就可以隐瞒,而且应当隐瞒。同志们,不能隐瞒。凡是中央犯的错误,直接的归我负责,间接的我也有份,因为我是中央主席。我不是要别人推卸责任,其他一些同志也有责任,但是第一个负责的应当是我。我们的省委书记,地委书记,县委书记,直到区委书记,企业党委书记,公社党委书记,既然作了第一书记,对于工作中的缺点错误,就要担起责任。不负责任,怕负责任,不许人讲话,老虎屁股摸不得,凡是采取这种态度的人,十个就有十个要失败。人家总是要讲的,你老虎屁股真是摸不得吗?偏要摸!

在我们国家,如果不充分发扬人民民主和党内民主,不充分实行无产阶级的民主制,就不可能有真正的无产阶级的集中制。没有高度的民主,不可能有高度的集中,而没有高度的集中,就不可能建立社会主义经济。我们的国家,如果不建立社会主义经济,那会是一种什么状况呢?就会变成修正主义的国家,变成实际上是资产阶级的国家,无产阶级专政就会转化为资产阶级专政,而且会是反动的、法西斯式的专政。这是一个十分值得警惕的问题,希望同志们好好想一想。

没有民主集中制,无产阶级专政不可能巩固。在人民内部实行民主,对人民的敌人实行专政,这两个方面是分不开的,把这两个方面结合起来,就是无产阶级专政,或者叫人民民主专政。我们的口号是:无产阶级领导的、以工农联盟为基础的人民民主专政。无产阶级怎样实行领导呢?经过共产党来领导。共产党是无产阶级的先进部队。无产阶级团结一切赞成、拥护和参加社会主义革命和社会主义建设的阶级和阶层,对反动阶级,或者说,对反动阶级的残余,实行专政。在我们国内,人剥削人的制度已经消灭,地主阶级和资产阶级的经济基础已经消灭,现在反动阶级已经没有过去那么厉害了,比如说,已经没有一九四九年人民共和国建立的时候那么厉害了,也没有一九五七年资产阶级右派猖狂进攻的时候那么厉害了。所以我们说是反动阶级的残余。但是,对于这个残余,千万不可轻视,必须继续同他们作斗争。已经被推翻的反动阶级,还企图复辟。在社会主义社会,还会产生新的资产阶级分子。整个社会主义阶段,存在着阶级和阶级斗争。这种阶级斗争是长期的、复杂的,有时甚至是很激烈的。\mnote{5}我们的专政工具不能削弱,还应当加强。我们的公安系统是掌握在正确的同志的手里的。也可能有个别地方的公安部门,是掌握在坏人手里。还有一些作公安工作的同志,不依靠群众,不依靠党,在肃反工作中不是执行在党委领导下通过群众肃反的路线,只依靠秘密工作,只依靠所谓专业工作。专业工作是需要的,对于反革命分子,侦查、审讯是完全必要的,但是,主要是实行党委领导下的群众路线,特别是对于整个反动阶级的专政,必须依靠群众,依靠党。对于反动阶级实行专政,这并不是说把一切反动阶级的分子统统消灭掉,而是要改造他们,用适当的方法改造他们,使他们成为新人。没有广泛的人民民主,无产阶级专政不能巩固,政权会不稳。没有民主,没有把群众发动起来,没有群众的监督,就不可能对反动分子和坏分子实行有效的专政,也不可能对他们进行有效的改造,他们就会继续捣乱,还有复辟的可能\mnote{6}。这个问题应当警惕,也希望同志们好好想一想。

第三点,我们应当联合哪一些阶级?压迫哪一些阶级?

这是一个根本立场问题。

工人阶级应当联合农民阶级、城市小资产阶级、爱国的民族资产阶级,首先要联合的是农民阶级。知识分子,例如科学家、工程技术人员、教授、教员、作家、艺术家、演员、医务工作者、新闻工作者,他们不是一个阶级,他们或者附属于资产阶级或者附属于无产阶级。对于知识分子,是不是只有革命的我们才去团结呢?不是的。只要他们爱国,我们就要团结他们,并且要让他们好好工作。工人,农民,城市小资产阶级分子,爱国的知识分子,爱国的资本家和其他爱国的民主人士,这些人占了全人口的百分之九十五以上\mnote{7}。这些人,在我们人民民主专政下面,都属于人民的范围。在人民的内部,要实行民主。

人民民主专政要压迫的是地主、富农、反革命分子、坏分子和反共的右派分子。反革命分子、坏分子和反共的右派分子,他们代表的阶级是地主阶级和反动的资产阶级。这些阶级和坏人,大约占全人口的百分之四、五。这些人是我们要强迫改造的。他们是人民民主专政的专政对象。

我们站在哪一边?站在占全人口百分之九十五以上的人民群众一边,还是站在占全人口百分之四、五的地、富、反、坏、右一边呢?必须站在人民群众这一边,绝不能站到人民敌人那一边去。这是一个马克思列宁主义者的根本立场问题。

在国内是如此,在国际范围内也是如此。各国的人民,占人口总数的百分之九十以上的人民大众,总是要革命的,总是会拥护马克思列宁主义的。他们不会拥护修正主义,有些人暂时拥护,将来终究会抛弃它。他们总会逐步地觉醒起来,总会反对帝国主义和各国的反动派,总会反对修正主义。一个真正的马克思列宁主义者,必须坚定地站在占世界人口百分之九十以上的人民大众这一边。

第四点,关于认识客观世界的问题。

人对客观世界的认识,由必然王国到自由王国的飞跃,要有一个过程。例如对于在中国如何进行民主革命的问题,从一九二一年党的建立直到一九四五年党的第七次代表大会,一共二十四年,我们全党的认识才完全统一起来。中间经过一次全党范围的整风,从一九四二年春天到一九四五年夏天,有三年半的时间。那是一次细致的整风,采用的方法是民主的方法,就是说,不管什么人犯了错误,只要认识了,改正了,就好了,而且大家帮助他认识,帮助他改正,叫做“惩前毖后,治病救人”,“从团结的愿望出发,经过批评或者斗争,分清是非,在新的基础上达到新的团结”。“团结——批评——团结”这个公式,就是在那个时候产生的。那次整风帮助全党同志统一了认识。对于当时的民主革命应当怎么办,党的总路线和各项具体政策应当怎么定,这些问题,都是在那个时期,特别是在整风之后,才得到完全解决的。

从党的建立到抗日时期,中间有北伐战争和十年土地革命战争。我们经过了两次胜利,两次失败。北伐战争胜利了,但是到一九二七年,革命遭到了失败。土地革命战争曾经取得了很大的胜利,红军发展到三十万人,后来又遭到挫折,经过长征,这三十万人缩小到两万多人,到陕北以后补充了一点,还是不到三万人,就是说,不到三十万人的十分之一。究竟是那三十万人的军队强些,还是这不到三万人的军队强些?我们受了那样大的挫折,吃过那样大的苦头,就得到锻炼,有了经验,纠正了错误路线,恢复了正确路线,所以这不到三万人的军队,比起过去那个三十万人的军队来,要更强些。刘少奇同志在报告里说,最近四年,我们的路线是正确的,成绩是主要的,我们在实际工作中犯过一些错误,吃了苦头,有了经验了,因此我们更强了,而不是更弱了。情形正是这样。在民主革命时期,经过胜利、失败,再胜利、再失败,两次比较,我们才认识了中国这个客观世界。在抗日战争前夜和抗日战争时期,我写了一些论文,例《中国革命战争的战略问题》、《论持久战》、《新民主主义论》、《〈共产党人〉发刊词》,替中央起草过一些关于政策、策略的文件,都是革命经验的总结。那些论文和文件,只有在那个时候才能产生,在以前不可能,因为没有经过大风大浪,没有两次胜利和两次失败的比较,还没有充分的经验,还不能充分认识中国革命的规律。

中国这个客观世界,整个地说来,是由中国人认识的,不是在共产国际管中国问题的同志们认识的。共产国际的这些同志就不了解或者说不很了解中国社会,中国民族,中国革命。对于中国这个客观世界,我们自己在很长时间内都认识不清楚,何况外国同志呢?

在抗日时期,我们才制定了合乎情况的党的总路线和一整套具体政策。这时候,中国民主革命这个必然王国才被我们认识,我们才有了自由。到这个时候,我们已经干了二十来年的革命。过去那么多年的革命工作,是带着很大的盲目性的。如果有人说,有哪一位同志,比如说中央的任何同志,比如说我自己,对于中国革命的规律,在一开始的时候就完全认识了,那是吹牛,你们切记不要信,没有那回事。过去,特别是开始时期,我们只是一股劲儿要革命,至于怎么革法,革些什么,哪些先革,哪些后革,哪些要到下一阶段才革,在一个相当长的时间内,都没有弄清楚,或者说没有完全弄清楚。我讲我们中国共产党人在民主革命时期艰难地但是成功地认识中国革命规律这一段历史情况的目的,是想引导同志们理解这样一件事:对于建设社会主义的规律的认识,必须有一个过程。必须从实践出发,从没有经验到有经验,从有较少的经验,到有较多的经验,从建设社会主义这个未被认识的必然王国,到逐步地克服盲目性、认识客观规律、从而获得自由,在认识上出现一个飞跃,到达自由王国。

对于社会主义建设,我们还缺乏经验。我向好几个国家的兄弟党的代表团谈过这个问题。我说,对于建设社会主义经济,我们没有经验。

这个问题,我也向一些资本主义国家的新闻记者谈过,其中有一个美国人叫斯诺。他老要来中国,一九六〇年让他来了。我同他谈过一次话。我说:“你知道,对于政治、军事,对于阶级斗争,我们有一套经验,有一套方针、政策和办法;至于社会主义建设,过去没有干过,还没有经验。你会说,不是已经干了十一年了吗?是干了十一年了,可是还缺乏知识,还缺乏经验,就算开始有了一点,也还不多。”斯诺要我讲讲中国建设的长期计划。我说:“不晓得。”他说:“你讲话太谨慎。”我说:“不是什么谨慎不谨慎,我就是不晓得呀,就是没有经验呀。”同志们,也真是不晓得,我们确实还缺少经验,确实还没有这样一个长期计划。一九六〇年,那正是我们碰了许多钉子的时候。一九六一年,我同蒙哥马利\mnote{8}谈话,也说到上面那些意见。他说:“再过五十年,你们就了不起了。”他的意思是说,过了五十年我们就会壮大起来,而且会“侵略”人家,五十年内还不会。他的这种看法,一九六〇年他来中国的时候就对我说过。我说:“我们是马克思列宁主义者,我们的国家是社会主义国家,不是资本主义国家,因此,一百年,一万年,我们也不会侵略别人。至于建设强大的社会主义经济,在中国,五十年不行,会要一百年,或者更多的时间。

在你们国家,资本主义的发展,经过了好几百年。十六世纪不算,那还是在中世纪。从十七世纪到现在,已经有三百六十多年。在我国,要建设起强大的社会主义经济,我估计要花一百多年。”十七世纪是什么时代呢?那是中国的明朝末年和清朝初年。再过一个世纪,到十八世纪的上半期,就是清朝乾隆时代,《红楼梦》的作者曹雪芹就生活在那个时代,就是产生贾宝玉这种不满意封建制度的小说人物的时代。乾隆时代,中国已经有了一些资本主义生产关系的萌芽,但是还是封建社会。这就是出现大观园里那一群小说人物的社会背景。在那个时候以前,在十七世纪,欧洲的一些国家已经在发展资本主义了,经过三百多年,资本主义的生产力有了现在这个样子。社会主义和资本主义比较,有许多优越性,我们国家经济的发展,会比资本主义国家快得多。可是,中国的人口多、底子薄,经济落后,要使生产力很大地发展起来,要赶上和超过世界上最先进的资本主义国家,没有一百多年的时间,我看是不行的。也许只要几十年,例如有些人所设想的五十年,就能做到。果然这样,谢天谢地,岂不甚好。但是我劝同志们宁肯把困难想得多一点,因而把时间设想得长一点。三百几十年建设了强大的资本主义经济,在我国,五十年内外到一百年内外,建设起强大的社会主义经济,那又有什么不好呢?从现在起,五十年内外到一百年内外,是世界上社会制度彻底变化的伟大时代,是一个翻天覆地的时代,是过去任何一个历史时代都不能比拟的。处在这样一个时代,我们必须准备进行同过去时代的斗争形式有着许多不同特点的伟大的斗争。为了这个事业,我们必须把马克思列宁主义的普遍真理同中国社会主义建设的具体实际、并且同今后世界革命的具体实际,尽可能好一些地结合起来,从实践中一步一步地认识斗争的客观规律。要准备着由于盲目性而遭受到许多的失败和挫折,从而取得经验,取得最后的胜利。由这点出发,把时间设想得长一点,是有许多好处的,设想得短了反而有害。

在社会主义建设上,我们还有很大的盲目性。社会主义经济,对于我们来说,还有许多未被认识的必然王国。拿我来说,经济建设工作中间的许多问题,还不懂得。工业、商业,我就不大懂。对于农业,我懂得一点。但是也只是比较地懂得,还是懂得不多。要较多地懂得农业,还要懂得土壤学、植物学、作物栽培学、农业化学、农业机械,等等;还要懂得农业内部的各个分业部门,例如粮、棉、油、麻、丝、茶、糖、菜、烟、果、药、杂等等;还有畜牧业,还有林业。我是相信苏联威廉斯\mnote{9}土壤学的,在威廉斯的土壤学著作里,主张农、林、牧三结合。我认为必须要有这种三结合,否则对于农业不利。所有这些农业生产方面的问题,我劝同志们,在工作之暇,认真研究一下,我也还想研究一点。但是到现时止,在这些方面,我的知识很少。我注意得较多的是制度方面的问题,生产关系方面的问题。至于生产力方面,我的知识很少。社会主义建设,从我们全党来说,知识都非常不够。我们应当在今后一段时间内,积累经验,努力学习,在实践中间逐步地加深对它的认识,弄清楚它的规律。一定要下一番苦功,要切切实实地去调查它,研究它。要下去蹲点,到生产大队、生产队,到工厂,到商店,去蹲点。调查研究,我们从前做得比较好,可是进城以后,不认真做了。一九六一年我们又重新提倡,现在情况已经有所改变。但是,在领导干部中间,特别是在高级领导干部中间,有一些地方、部门和企业,至今还没有形成风气。有一些省委书记,到现在还没有下去蹲过点。如果省委书记不去,怎么能叫地委书记、县委书记下去蹲点呢。这个现象不好,必须改变过来。

从中华人民共和国成立,到现在已经十二年了。这十二年分为前八年和后四年。一九五〇年到一九五七年底,是前八年。一九五八年到现在,是后四年。我们这次会议已经初步总结了过去工作的经验,主要是后四年的经验。这个总结,反映在刘少奇同志的报告里面。我们已经制定、或者正在制定、或者将要制定各个方面的具体政策。已经制定了的,例如农村公社六十条\mnote{10},工业企业七十条\mnote{11},高等教育六十条\mnote{12},科学研究工作十四条\mnote{13},这些条例草案已经在实行或者试行,以后还要修改,有些还可能大改。正在制定的,例如商业工作条例\mnote{14}。将要制定的,例如中小学教育条例\mnote{15}。我们的党政机关和群众团体的工作,也应当制定一些条例。军队已经制定了一些条例\mnote{16}。

总之,工、农、商、学、兵、政、党这七个方面的工作,都应当好好地总结经验,制定一整套的方针、政策和办法,使它们在正确的轨道上前进。

有了总路线还不够,还必须在总路线指导之下,在工、农、商、学、兵、政、党各个方面,有一整套适合情况的具体的方针、政策和办法,才有可能说服群众和干部,并且把这些当作教材去教育他们,使他们有一个统一的认识和统一的行动,然后才有可能取得革命事业和建设事业的胜利,否则是不可能的。对于这一点,我们在抗日时期就有了深刻的认识。在那时候,我们这样做了,就使得干部和群众对于民主革命时期的一整套具体的方针、政策和办法,有了统一的认识,因而有了统一的行动,使当时的民主革命事业取得了胜利,这是大家知道的。在社会主义革命和社会主义建设的时期,头八年内,我们的革命任务,在农村是完成对封建主义的土地制度的改革和接着实现农业合作化;在城市是实现对资本主义工商业的社会主义改造。

在经济建设方面,那时候的任务是恢复经济和实现第一个五年计划。不论在革命方面和建设方面,时候都有一条适合客观情况的、有充分说服力的总路线,以及在总路线指导下的一整套方针、政策和办法,因此教育了干部和群众,统一了他们的认识,工作也就比较做得好。这也是大家知道的。但是,那时候有这样一种情况,因为我们没有经验,在经济建设方面,我们只得照抄苏联,特别是在重工业方面,几乎一切都抄苏联,自己的创造性很少。这在当时是完全必要的,同时又是一个缺点,缺乏创造性,缺乏独立自主的能力。这当然不应当是长久之计。从一九五八年起,我们就确立了自力更生为主、争取外援为辅的方针。在一九五八年党的八大二次会议上,通过了“鼓足干劲,力争上游,多快好省地建设社会主义”的总路线,在那一年又办起了人民公社,提出了大跃进的口号。在提出社会主义建设总路线的一个相当时间内,我们还没有来得及、也没有可能规定一整套适合情况的具体的方针、政策和办法,因为经验还不足。在这种情形下,干部和群众,还得不到一整套的教材,得不到系统的政策教育,也就不可能真正有统一的认识和统一的行动。要经过一段时间,碰过一些钉子,有了正、反两方面的经验,才有这样的可能。现在好了,有了这些东西了,或者正在制定这些东西。这样,我们就可以更加妥善地进行社会主义革命和社会主义建设。

在总路线指导之下,制定一整套的具体的方针、政策和办法,必须通过从群众中来的方法,通过作系统的周密的调查研究的方法,对工作中的成功经验和失败经验,作历史的考察,才能找出客观事物所固有的而不是人们主观臆造的规律,才能制定适合情况的各种条例。这件事很重要,请同志们注意到这点。

工、农、商、学、兵、政、党这七个方面,党是领导一切的。党要领导工业、农业、商业、文化教育、军队和政府。我们的党,一般说来是很好的。我们党员的成分,主要的是工人和贫苦农民。我们的绝大多数干部都是好的,他们都在辛辛苦苦地工作。但是,也要看到,我们党内还存在一些问题,不要想象我们党的情况什么都好。我们现在有一千七百多万党员,这里面差不多有百分之八十的人是建国以后入党的,五十年代入党的。建国以前入党的只占百分之二十。在这百分之二十的人里面,一九三〇年以前入党的,二十年代入党的,据前几年计算,有八百多人,这两年死了一些,恐怕只有七百多人了。不论在老的和新的党员里面,特别是在新党员里面,都有一些品质不纯和作风不纯的人。他们是个人主义者、官僚主义者、主观主义者,甚至是变了质的分子。还有些人挂着共产党员的招牌,但是并不代表工人阶级,而是代表资产阶级。党内并不纯粹,这一点必须看到,否则我们是要吃亏的。

上面是我讲的第四点。就是讲,我们对于客观世界的认识,要有一个过程。先是不认识或者不完全认识,经过反复的实践,在实践里面得到成绩,有了胜利,又翻过斤斗,碰了钉子,有了成功和失败的比较,然后才有可能逐步地发展成为完全的认识或者比较完全的认识。到那个时候,我们就比较主动了,比较自由了,就变成比较聪明一些的人了。自由是对必然的认识和对客观世界的改造。\mnote{17}只有在认识必然的基础上,人们才有自由的活动。这是自由和必然的辩证规律。所谓必然,就是客观存在的规律性,在没有认识它以前,我们的行动总是不自觉的,带着盲目性的。这时候我们是一些蠢人。最近几年我们不是干过许多蠢事吗?

第五点,关于国际共产主义运动。这个问题,我只简单地讲几句。

不论在中国,在世界各国,总而言之,百分之九十以上的人终究是会拥护马克思列宁主义的。在世界上,现在还有许多人,在社会民主党的欺骗之下,在修正主义的欺骗之下,在帝国主义的欺骗之下,在各国反动派的欺骗之下,他们还不觉悟。但是,他们总会逐步地觉悟过来,总会拥护马克思列宁主义。马克思列宁主义这个真理,是不可抗拒的。人民群众总是要革命的。世界革命总是要胜利的。不准革命,像鲁迅所写的赵太爷、钱太爷、假洋鬼子不准阿Q革命那样,总是要失败的。

苏联是第一个社会主义国家,苏联共产党是列宁创造的党。虽然,苏联的党和国家的领导现在被修正主义者篡夺了,但是,我劝同志们坚决相信,苏联广大的人民、广大的党员和干部,是好的,是要革命的,修正主义的统治是不会长久的。无论什么时候,现在,将来,我们这一辈子,我们的子孙,都要向苏联学习,学习苏联的经验。不学习苏联,要犯错误。人们会问:苏联被修正主义者统治了,还要学吗?我们学习的是苏联的好人好事,苏联党的好经验,苏联工人、农民和联系劳动人民的知识分子的好经验。至于苏联的坏人坏事,苏联的修正主义者,我们应当看作反面教员,从他们那里吸取教训。\mnote{18}

我们永远要坚持无产阶级的国际主义团结的原则,我们始终主张社会主义国家和世界共产主义运动一定要在马克思列宁主义的基础上巩固地团结起来。

国际修正主义者在不断地骂我们。我们的态度是,由他骂去。在必要的时候,给以适当的回答。我们这个党是被人家骂惯了的。从前骂的不说,现在呢,在国外,帝国主义者骂我们,反动的民族主义者骂我们,各国反动派骂我们,修正主义者骂我们;在国内,蒋介石骂我们,地、富、反、坏、右骂我们。历来就是这么骂的,已经听惯了。我们是不是孤立的呢?我就不感觉孤立。我们在座的有七千多人,七千多人还孤立吗?(笑声)我们国家有六亿几千万人民,我国人民是团结的,六亿几千万人还孤立吗?世界各国人民群众已经或者将要同我们站到一起,我们会是孤立的吗?

最后一点,第六点,要团结全党和全体人民。要把党内、党外的先进分子、积极分子团结起来,把中间分子团结起来,去带动落后分子,这样就可以使全党、全民团结起来。只有依靠这些团结,我们才能够做好工作,克服困难,把中国建设好。要团结全党、全民,这并不是说我们没有倾向性。有些人说共产党是“全民的党”,我们不这样看。我们的党是无产阶级政党,是无产阶级的先进部队,是用马克思列宁主义武装起来的战斗部队。我们是站在占总人口的百分之九十五以上的人民大众一边,绝不站在占总人口百分之四、五的地、富、反、坏、右那一边。

在国际范围内也是这样,我们是同一切马克思列宁主义者、一切革命人民、全体人民讲团结的,绝不同反共反人民的帝国主义者和各国反动派讲什么团结。只要有可能,我们也同这些人建立外交关系,争取在五项原则基础上和平共处。但是这些事,跟我们和各国人民的团结是不同范畴的两回事情。

要使全党、全民团结起来,就必须发扬民主,让人讲话。在党内是这样,在党外也是这样。省委的同志,地委的同志,县委的同志,你们回去,一定要让人讲话。在座的同志们要这样做,不在座的同志们也要这样做,一切党的领导人员都要发扬党内民主,让人讲话。界限是什么呢?一个是,遵守党的纪律,少数服从多数,全党服从中央。另一个是,不准组织秘密集团。我们不怕公开的反对派,只怕秘密的反对派,这种人,当面不讲真话,当面讲的尽是些假的、骗人的话,真正的目的不讲出来。只要不是违反纪律的,只要不是搞秘密集团活动的,我们都允许他讲话,而且讲错了也不要处罚。讲错了话可以批评,但是要用道理说服人家。说而不服怎么办?让他保留意见。只要服从决议,服从多数人决定的东西,少数人可以保留不同的意见。在党内党外,容许少数人保留意见,是有好处的。错误的意见,让他暂时保留,将来他会改的。许多时候,少数人的意见,倒是正确的。历史上常常有这样的事实,起初,真理不是在多数人手里,而是在少数人手里。马克思、恩格斯手里有真理,可是他们在开始的时候是少数。列宁在很长一个时期内也是少数。我们党内也有这样的经验,在陈独秀\mnote{19}统治的时候,在“左”倾路线统治的时候,真理都不在领导机关的多数人手里,而是在少数人手里。历史上的自然科学家,例如哥白尼、伽利略、达尔文\mnote{20},他们的学说曾经在一个长时间内不被多数人承认,反而被看作错误的东西,当时他们是少数。我们党在一九二一年成立的时候,只有几十个党员,也是少数人,可是这几十个人代表了真理,代表了中国的命运。

有一个捕人、杀人的问题,我还想讲一下。在现在的时候,在革命胜利还只有十几年的时候,在被打倒了的反动阶级分子还没有被改造好,有些人并且企图阴谋复辟的时候,人总会要捕一点、杀一点的,否则不能平民愤,不能巩固人民的专政。但是,不要轻于捕人,尤其不要轻于杀人。有一些坏人,钻到我们队伍里面的坏分子,蜕化变质分子,这些人,骑在人民的头上拉屎拉尿,穷凶极恶,严重地违法乱纪。这是些小蒋介石。对于这种人得有个处理,罪大恶极的,也要捕一些,还要杀几个。因为对这样的人,完全不捕、不杀,不足以平民愤。这就是所谓“不可不捕,不可不杀”。但是绝不可以多捕、多杀。凡是可捕可不捕的,可杀可不杀的,都要坚决不捕、不杀。有个潘汉年\mnote{21},此人当过上海市副市长,过去秘密投降了国民党,是个CC派人物,现在关在班房里头,我们没有杀他。像潘汉年这样的人,只要杀一个,杀戒一开,类似的人都得杀。还有个王实味\mnote{22},是个暗藏的国民党探子。在延安的时候,他写过一篇文章,题名《野百合花》,攻击革命,诬蔑共产党。后头把他抓起来,杀掉了。那是保安机关在行军中间,自己杀的,不是中央的决定。对于这件事,我们总是提出批评,认为不应当杀。他当特务,写文章骂我们,又死不肯改,就把他放在那里吧,让他劳动去吧,杀了不好。人要少捕、少杀。动不动就捕人、杀人,会弄得人人自危,不敢讲话。在这种风气下面,就不会有多少民主。

还不要给人乱戴帽子。我们有些同志惯于拿帽子压人,一张口就是帽子满天飞,吓得人不敢讲话。当然,帽子总是有的,刘少奇同志的报告里面不是就有许多帽子吗?“分散主义”不是帽子吗?但是,不要动不动就给人戴在头上,弄得张三分散主义,李四分散主义,什么人都是分散主义。

帽子,最好由人家自己戴,而且要戴得合适,最好不要由别人去戴。他自己戴了几回,大家不同意他戴了,那就取消了。这样,就会有很好的民主空气。我们提倡不抓辫子、不戴帽子、不打棍子,目的就是要使人心里不怕,敢于讲意见。

对于犯了错误的人,对于那些不让人讲话的人,要采取善意帮助的态度。不要有这样的空气:似乎犯不得错误,一犯错误就不得了,一犯错误,从此不得翻身。一个人犯了错误,只要他真心愿意改正,只要他确实有了自我批评,我们就要表示欢迎。头一、二次自我批评,我们不要要求过高,检讨得还不彻底,不彻底也可以,让他再想一想,善意地帮助他。人是要有人帮助的。应当帮助那些犯错误的同志认识错误。如果人家诚恳地作了自我批评,愿意改正错误,我们就要宽恕他,对他采取宽大政策。只要他的工作成绩还是主要的,能力也还行,就还可以让他在那里继续工作。

我在这个讲话里批评了一些现象,批评了一些同志,但是没有指名道姓,没有指出张三、李四来。你们自己心里有数。(笑声)我们这几年工作中的缺点、错误,第一笔账,首先是中央负责,中央又是我首先负责;第二笔账,是省委、市委、自治区党委的;第三笔账,是地委一级的;第四笔账,是县委一级的;第五笔账,就算到企业党委、公社党委了。总之,各有各的账。

同志们,你们回去,一定要把民主集中制健全起来。县委的同志,要引导公社党委把民主集中制健全起来。首先要建立和加强集体领导。不要再实行长期固定的“分片包干”的领导方法了,那个方法,党委书记和委员们各搞各的,不能有真正的集体讨论,不能有真正的集体领导。要发扬民主,要启发人家批评,要听人家的批评。自己要经得起批评。应当采取主动,首先作自我批评。有什么就检讨什么,一个钟头,顶多两个钟头,倾箱倒箧而出,无非是那么多。如果人家认为不够,请他提出来,如果说得对,我就接受。让人讲话,是采取主动好,还是被动好?当然是主动好。已经处在被动地位了怎么办?过去不民主,现在陷于被动,那也不要紧,就请大家批评吧。白天出气,晚上不看戏,白天晚上都请你们批评。(笑声)这个时候我坐下来,冷静地想一想,两三天晚上睡不着觉。想好了,想通了,然后诚诚恳恳地作一篇检讨。这不就好了吗?总之,让人讲话,天不会塌下来,自己也不会垮台。不让人讲话呢?那就难免有一天要垮台。

我今天的讲话就讲这一些。中心是讲了一个实行民主集中制的问题,在党内、党外发扬民主的问题。我向同志们建议,仔细考虑一下这个问题。有些同志还没有民主集中制的思想,现在就要开始建立这个思想,开始认识这个问题。我们充分地发扬了民主,就能把党内、党外广大群众的积极性调动起来,就能使占总人口百分之九十五以上的人民大众团结起来。做到了这些,我们的工作就会越做越好,我们遇到的困难就会较快地得到克服,我们事业的发展就会顺利得多。(热烈鼓掌)

\begin{maonote}
\mnitem{1}一九六二年一月十一日至二月七日,中共中央在北京召开扩大的中央工作会议。参加会议的有县委以上的各级党委主要负责人七千人,因此这次大会又称“七千人大会”。
\mnitem{2}“我们的态度是:坚持真理,随时修正错误”这句话,是一九六六年二月印发这个讲话稿时加进去的。
\mnitem{3}见司马迁《报任少卿书》。
\mnitem{4}这段话引自毛泽东一九五七年七月在青岛召开的省市委书记会议期间写的《一九五七年夏季的形势》一文。原文是:“我们的目标,是想造成一个又有集中又有民主,又有纪律又有自由,又有统一意志、又有个人心情舒畅、生动活泼,那样一种政治局面,以利于社会主义革命和社会主义建设,较易于克服困难,较快地建设我国的现代工业和现代农业,党和国家较为巩固,较为能够经受风险。”
\mnitem{5}“已经被推翻的反动阶级,还企图复辟。在社会主义社会,还会产生新的资产阶级分子。整个社会主义阶段,存在着阶级和阶级斗争。这种阶级斗争是长期的、复杂的,有时甚至是很激烈的”这四句话,是一九六二年九月二十八日党的八届十中全会公报中的提法,一九六六年二月中共中央将毛泽东这个讲话印发给党内领导干部阅读时,经他本人审阅同意,加进了讲话中。
\mnitem{6}一九六二年四月中共中央印发的这个讲话中,这句话是:“没有民主,没有把群众发动起来,没有群众的监督,反动阶级的残余,坏分子,不可能被改造,而且他们还会继续捣乱,还有复辟的可能。”
\mnitem{7}一九六二年四月中共中央印发的这个讲话中,这里是“百分之九十以上”,另外还有几处“百分之九十几”,在一九六六年二月印发的讲话稿中均改为“百分之九十五以上”。
\mnitem{8}蒙哥马利,曾任英国陆军元帅。一九六〇年、一九六一年曾两次访问中国。
\mnitem{9}威廉斯,苏联土壤学家和农学家。他提出了一套关于恢复和提高土壤肥力的方法和理论,主张农、林、牧相结合,提倡草田轮作制。
\mnitem{10}指中共中央一九六一年三月制定的《农村人民公社工作条例(草案)》,共六十条。一九六二年九月,中共八届十中全会通过了《农村人民公社工作条例(修正草案)》。
\mnitem{11}指中共中央一九六一年九月制定的《国营工业企业工作条例(草案)》,共七十条。
\mnitem{12}指中共中央一九六一年九月原则批准的《中华人民共和国教育部直属高等学校暂行工作条例(草案)》,共六十条。
\mnitem{13}指国家科学技术委员会党组和中国科学院党组一九六一年六月提出、中共中央七月批准试行的《关于自然科学研究机构当前工作的十四条意见(草案)》。
\mnitem{14}中共中央一九六一年六月十九日下达了《关于改进商业工作的若干规定(试行草案)》,共四十条。随后,中央根据调查研究的情况,准备拟定一个比较全面的商业工作条例,一九六二年曾起草,后因故没有形成正式文件。
\mnitem{15}指教育部拟定的《全日制中学暂行工作条例(草案)》和《全日制小学暂行工作条例(草案)》。一九六三年三月二十三日,中共中央将这两个工作条例草案发给各地讨论和试行。
\mnitem{16}指中华人民共和国国防部一九六一年六月十九日命令颁布的《中国人民解放军连队管理教育工作条例》和同年十一月十七日在全军政治工作会议上通过的《中国人民解放军连队政治指导员工作条例》、《中国共产党连队支部工作条例》、《中国共产主义青年团连队支部工作条例》、《中围人民解放军连队革命军人委员会工作条例》。
\mnitem{17}“自由是对必然的认识和对客观世界的改造”这一句,是一九六六年二月印发这个讲话稿时修改的,一九六二年四月中共中央印发的这个讲话中是“自由是被认识了的必然”。
\mnitem{18}这一段中,“虽然,苏联的党和国家的领导现在被修正主义者篡夺了,但是,我劝同志们坚决相信,苏联广大的人民、广大的党员和干部,是好的,是要革命的,修正主义的统治是不会长久的”,“苏联被修正主义者统治了”,“至于苏联的坏人坏事,苏联的修正主义者,我们应当看作反面教员,从他们那里吸取教训”几句话,是一九六六年二月印发这个讲话稿时加写和改写的。
\mnitem{19}陈独秀,五四新文化运动的主要领导人之一,中国共产党的主要创建人之一。在党成立后的最初六年中,是党的主要领导人,在第一次国内革命战争后期,犯了严重的右倾投降主义错误。第一次大革命失败后,他对于革命前途悲观失望,接受托派观点,在党内成立小组织,进行反党活动,一九二九年十一月被开除出党。
\mnitem{20}哥白尼,波兰天文学家,太阳中心说的创始人。伽利略,意大利物理学家、天文学家,是哥白尼太阳中心说的支持者和论证者。达尔文,英国生物学家,生物进化论的奠基者。
\mnitem{21}潘汉年,一九二五年加入中国共产党。一九三六、一九三七年间,曾任中国共产党同国民党谈判的代表。与潘汉年一案密切关切的一个人是胡均鹤,胡均鹤早年加入中国共产党,一九三二年秋天被国民党抓获后叛变,后投靠日伪。

一九三九年夏末,中共中央正式任命潘任中社部负责情报的副部长,并命令其组建华南情报局,统一领导和管理从上海到香港的中共情报系统,由于情报工作的复杂性,潘汉年通过胡均鹤与汪伪特务机构“76号”头目李士群建立了联系,这些人都是多面间谍。

一九四三年,抗日战争处于最艰难时刻,日伪计划对新四军根据地进行残酷的大“扫荡”。当时担任新四军政委兼中共华中局书记的饶漱石向潘汉年提出,要他到上海去一次,重新部署那里的情报工作,加强与李士群的联系,尤其是要搞清楚日伪这次大“扫荡”的准确情况。潘汉年在胡均鹤的陪同下分别到苏州、南京去见李士群,并怀着复杂的心情随着李士群与胡均鹤一同驱车到汪公馆会见了汪精卫,并和汪精卫作了简短的会见,汪表示欢迎中共参加他主张的政治议会,建立联合政府。潘答复似乎没有可能,但愿意将此意禀报延安方面。并表示新四军主要图谋自身发展,倘若将来汪需要转寰,新四军不会跟他过不去的。

当晚,汪精卫身边的中统、军统人员就将谈话内容密报重庆,国民党迫不及待地在新闻媒体方式上公开,将污水泼向中共,公开攻击共产党派代表潘汉年与“汪逆”勾结,极力损害共产党“坚决抗日”的良好形象,中共中央由于不知情,认定国民党是“造谣生事,栽赃陷害”,立马公开决然否认,并几次以中共中央名义通过报界郑重辟谣。

四月初潘汉年返回淮南根据地,向饶漱石报告了上海的情报工作以及关于日伪军“扫荡”计划暂时还不会有大动作等情况,但隐瞒了其会见汪精卫的事情。

而后,饶漱石对潘汉年在日伪地盘的举动也有察觉,他给延安打报告认为潘汉年“违犯纪律”,与敌伪关系过于密切,来往已属不正常。

一九四五年初潘汉年赴延安参加党的七大期间,毛泽东接见潘汉年,并单独向他了解了他在敌后从事情报工作的情况,潘汉年又一次隐瞒了他在南京秘密会见汪精卫的情况。于是,毛泽东于二月二十三日签发了刘少奇、康生署名给饶漱石的电报:“饶:……至于敌伪及国民党各特务机关说汉年到南京与日方谈判并见过汪精卫等等,完全是造谣诬蔑。在利用李士群的过程中,汉年也绝无可怀疑之处,相反的还得到了许多成绩。这类工作今后还要放手去做,此次汉年来延安,毛主席已向他解释清楚。……国民党中统局经常制造谣言说延安派李富春、华中派潘汉年到南京与敌伪勾结,又常造谣说汉年已被华中局扣押,极尽造谣挑拨之伎,望告情报系统的同志们千万勿听信此种谣言,以致中敌人奸计。”

一九四五年八月,抗日战争胜利后,胡均鹤被判处十年徒刑,一九四九年初,国民党兵败,为给即将接管政权的共产党增添麻烦,就将监狱中关押的罪犯统统放了出来,胡均鹤被释放后,国民党命他作潜伏特务。胡均鹤经潘汉年介绍,被委任为上海市公安局情报委员会主任和专员,被委任为上海市公安局情报委员会主任和专员,为党做了一些有益的工作。

但由于国民党潜伏特务的身份问题,一九五四年九月胡均鹤被逮捕,一九五五年三月下旬,中国共产党召开了一次全国代表大会,身为上海市副市长的潘汉年出席了这次代表大会。这次会议的一项重要议题是解决高岗、饶漱石反党联盟问题。一些曾受高、饶影响,或与高、饶有过牵连的人先后在会上做了自我批评和交代。毛泽东在会上要求高级干部本人历史上有什么问题要交代的,应主动向中央讲清楚。会议印发的胡均鹤被逮捕并被送往北京隔离审查等情况,使潘汉年意识到了胡均鹤有可能交代自己密会汪精卫一事,迫于强大的压力,经过反复考虑,潘汉年找到陈毅,详细讲述了自己十二年前在李士群、胡均鹤“挟持”下去见汪精卫的经过,检讨了自己长时期没有向组织上汇报的原因,并将一份此事的经过情况和检讨交给陈毅转报中央。

陈毅亲自赴中南海向毛泽东报告并转交书面材料。毛泽东看后大为震怒,批示:此人从此不可信用。

一九五五年四月三日,潘汉年被逮捕,值得一提的是,一九五五年四月二十九日,李克农向中共中央政治局和书记处写了正式报告。报告分两部分,第一部分认为潘汉年历史有七处可疑之处,同意审查;第二部分认为潘汉年没有背叛党,并提出了有力的反证材料:

“潘汉年为中央提供了大量决策性的情报”、“潘汉年所参与、了解的组织机密,一直未被泄露,直到上海解放”。

一九六四年六月,最高人民法院作出终审判决,以“内奸”罪,判处潘汉年有期徒刑十五年,剥夺政治权利终身。中央认定潘汉年“投降了国民党,充当国民党特务”、“秘密投靠了日本特务机关,当了日本特务,并与大汉奸汪精卫进行勾结”、“掩护以胡均鹤为首的大批中统潜伏特务和反革命分子”。

不过,潘汉年除了对私见汪精卫承认之外,对于其他指控一概否认。

一九七七年四月十四日,潘汉年因病在湖南长沙省医学院第二附属医院去世,终年七十岁。

胡均鹤于一九八三年保外就医,一九八四年释放,一九九三年九月在上海去世。

一九八〇年以后,走资派控制了党中央,为了反毛和夺权需要,一风吹的翻案,一九八二年八月二十三日,中共中央发出《关于为潘汉年同志平反昭雪,恢复名誉的通知》,认定:“一九四三年,潘汉年做汉奸李士群的策反工作,突然被李挟持去见汪精卫。会见汪精卫是被挟持的。”

静火有言:潘汉年私自会见汪精卫,事前不请示,事后不报告,严重违犯纪律,且故意隐瞒党中央十二年,在毛主席当面询问的时候仍然否认,对党不忠诚。私自会见汪精卫是一个严重的政治错误,被国民党和日伪利用,损害了共产党的形象,给党的事业造成了严重损失。在胡均鹤被捕的情况下,迫于压力潘汉年才交代这件事,即使他辩解的都是事实,但这种严重违纪的事情,是做秘密工作绝对不能允许的。况且,谁知道他究竟还有多少事在隐瞒组织呢?潘汉年终究是为党工作多年的人,可以优待善待,但潘汉年所犯错误可着实不小,毛主席的批示更是一针见血。
\mnitem{22}王实味,翻译家,还写过一些文学评论和杂文。曾在延安中央研究院文艺研究室任特别研究员。因发表《野百合花》等文章,一九四二年在整风中受到批判,同年十月被开除党籍,年底被关押。一九四六年被定为“反革命托派奸细分子”。一九四七年七月,在战争环境中被处决。据查,关于他是暗藏的国民党探子、特务一事,不能成立。关于反革命托派奸细问题,一九九一年二月七日,公安部《关于对王实味同志托派问题的复查决定》中说,“在复查中没有查出王实味同志参加托派组织的材料。因此,一九四六年定为‘反革命托派奸细分子’的结论予以纠正,王在战争环境中被错误处决给予平反昭雪。”
\end{maonote}
