
\title{关于九评苏共中央的公开信的节录}
\date{一九六三年、一九六四年}
\thanks{这是毛泽东同志主持的著名的《九评苏共中央公开信》的节录。}
\maketitle


\date{一九六三年九月十三日}
\section{一、关于斯大林问题——二评苏共中央的公开信}

斯大林问题,是一个世界范围内的大问题,曾经引起了世界各国一切阶级的反响,至今还在议论纷纷。各个不同的阶级,代表各个不同阶级的政党或政治派别,意见不同。估计在本世纪内,这个问题还不可能作出定论。但是,在国际工人阶级和革命人民范围之内,多数人的意见其实是相同的,他们不赞成全盘否定斯大林,而且越来越怀念斯大林。就是在苏联,也是如此。我们同苏共领导人的争论,是同一部分人的争论。我们希望说服这一部分人,以利于推进革命事业。这就是我们写这篇文章的目的。

中国共产党一向认为,赫鲁晓夫同志利用所谓“反对个人迷信”,全盘否定斯大林,是完全错误的,是别有用心的。

中国共产党历来认为,斯大林是有过一些错误的。这些错误,有思想认识的根源,也有社会历史的根源。如果站在正确的立场,采取正确的方法,批判斯大林确实犯过的错误,而不是凭空加给他的所谓错误,是必要的。

斯大林的思想方法,在一些问题上,离开了辩证唯物主义,陷入了形式主义和主观主义,因而有时脱离了实际情况,脱离了群众。他在党内和党外的斗争中,有的时候,有的问题上,混淆了敌我矛盾和人民内部矛盾这两类不同性质的矛盾和处理这两类矛盾的不同方法。他领导的肃清反革命的工作,正确地惩办了很多必须惩办的反革命分子,但是也错判了一些好人,在一九三七年和一九三八年,曾经造成过肃反扩大化的错误。

对于斯大林的只占第二位的一些错误方面,应当作为历史教训,使苏联共产党人和各国共产党人引以为戒,不再重犯,或者少犯一些,这也是有益的。正、反两面的历史经验,只要是总结得正确,合乎历史实际,而不加以任何歪曲,对于一切共产党人,都是有益的。

倍倍尔、卢森堡\mnote{1}等人在历史上所起的作用,远不能和斯大林相比。斯大林是一个历史时代的无产阶级专政和国际共产主义运动的伟大的领导人,对他的评价,应当更加慎重些。

早在二十年代末期和整个三十年代,随后又在四十年代的初期和中期,以毛泽东同志和刘少奇同志为代表的中国马克思列宁主义者,就在抵制斯大林的某些错误的影响,并且逐步克服了“左”倾和右倾机会主义的错误路线,终于把中国革命引导到胜利。但是,由于斯大林的一些错误主张,是被某些中国同志所接受和实行的,中国人自己应当负责,所以我们党进行的反对“左”倾和右倾机会主义的斗争,从来只限于批评我们自己的犯了错误的那些同志,而没有把责任推到斯大林身上。我们进行批评的目的,是为了分清是非,接受教训,推进革命事业。对于犯错误的同志,只要改了就好了。如果他们不改,也可以等待他们在实践经验中逐步觉悟过来,只要他们不组织秘密集团,暗中进行破坏活动。我们采取的方法是党内的批评和自我批评的正常方法,是从团结的愿望出发,经过批评或斗争,在新的基础上达到新的团结,因而收到了良好的效果。我们认为,这是人民内部的矛盾,不是敌我矛盾,所以应当采取这样的方法来处理。

绝大多数苏联人,不赞成这样谩骂斯大林。他们越来越怀念斯大林。苏共领导人严重地脱离了群众。他们时时刻刻感觉到斯大林的阴魂不散,在威胁着他们,其实是广大人民群众对于全盘否定斯大林表示非常不满意。赫鲁晓夫在苏共第二十次代表大会上所作的全盘否定斯大林的秘密报告,至今不敢拿出来同苏联人民和整个社会主义阵营各国人民见面,其原因就在于这个报告是一个见不得人的报告,是一个严重脱离群众的报告。

特别值得人们注意的是,苏共领导人在百般咒骂斯大林的同时,却对艾森豪威尔、肯尼迪\mnote{2}之流“表示尊重和信任”!咒骂斯大林是“伊凡雷帝\mnote{3}式的暴君”、“俄国历史上最大的独裁者”,却恭维艾森豪威尔和肯尼迪都“得到绝大多数美国人民的支持”!咒骂斯大林是“白痴”,却歌颂艾森豪威尔和肯尼迪“明智”!

斯大林做错了事,是能够做自我批评的。例如,他对中国革命曾经出过一些错误的主意,在中国革命胜利以后,他承认了自己的错误。对于清党工作中的一些错误,斯大林在一九三九年联共第十八次代表大会上的报告中,也是承认了的。

斯大林是在一九五三年逝世的,三年以后,苏共领导在苏共第二十次代表大会上大反斯大林;八年以后,苏共领导在苏共第二十二次代表大会上,又一次大反斯大林,并且搬尸焚尸。

无产阶级政党的领袖,不论是中央委员会委员,还是地方委员会委员,是在阶级斗争中、在群众的革命运动中产生的,是对群众忠心耿耿,同群众有血肉联系的,是善于把群众的意见正确地集中起来和坚持下去的。这样的领袖,是无产阶级的真正代表。这样的领袖,是群众公认的。

我们劝告赫鲁晓夫同志一句诚恳的话,希望你迷途知返,从完全错误的道路,回到马克思列宁主义的道路上来。

\date{一九六三年十一月十九日}
\section{二、在战争与和平问题上的两条路线——五评苏共中央的公开信}

社会实践是检验真理的唯一标准。那些在战争与和平问题的认识上有错误观点的人,在帝国主义和反动派的反面教育之下,我们相信,有很多人会改变过来。对此,我们寄予很大的希望。

\date{一九六四年二月四日}
\section{三、苏共领导是当代最大的分裂主义者——七评苏共中央的公开信}

苏共领导的修正主义和分裂主义,是国内资产阶级因素泛滥和增长起来的产物,也是帝国主义政策的产物,特别是美帝国主义的核讹诈政策和“和平演变”政策的产物。

马克思列宁主义是科学,科学是不怕论战的,怕论战的不是科学。

\date{一九六四年七月十四日}
\section{四、关于赫鲁晓夫的假共产主义及其在世界历史上的教训——九评苏共中央的公开信}

无产阶级专政的历史教训节选

(怎样才能防止资本主义复辟呢?在这个问题上,毛泽东同志根据马克思列宁主义的基本原理,总结了中国无产阶级专政的实践经验,也研究了国际的主要是苏联的正面的和反面的经验,提出了系统的理论和政策,从而丰富了和发展了马克思列宁主义关于无产阶级专政的学说。)

毛泽东同志在这方面提出的理论和政策的主要内容是:

第一,必须用马克思列宁主义的对立统一的规律来观察社会主义社会。事物的矛盾规律,即对立统一规律,是唯物辩证法的最根本的规律。这个规律,不论在自然界、人类社会和人们的思想中,都是普遍存在的。矛盾着的对立面又统一又斗争,由此推动事物的运动和变化。社会主义社会也不例外。在社会主义社会中,存在着两类社会矛盾,人民内部矛盾和敌我矛盾。这两类社会矛盾性质完全不同,处理方法也应当不同。正确处理这两类社会矛盾,将使无产阶级专政日益巩固,将使社会主义社会日益巩固和发展。许多人承认对立统一的规律,但是不能应用这个规律去观察和处理社会主义社会的问题。他们不承认社会主义社会有矛盾,不承认在社会主义社会中,不仅有敌我矛盾,而且有人民内部矛盾,不懂得正确地区别和正确地处理这两类社会矛盾,这样也就不能正确地处理无产阶级专政问题。

第二,社会主义社会是一个很长的历史阶段。社会主义社会还存在着阶级和阶级斗争,存在着社会主义和资本主义这两条道路的斗争。单有在经济战线上(在生产资料所有制)的社会主义革命,是不够的,并且是不巩固的,必须还有一个政治战线上和一个思想战线上的彻底的社会主义革命。在政治思想领域内,社会主义同资本主义之间谁胜谁负的斗争,需要一个很长的时间才能解决。几十年内是不行的,需要一百年到几百年的时间才能成功。在时间问题上,与其准备短些,宁可准备长些;在工作问题上,与其看得容易些,宁可看得困难些。这样想,这样做,较为有益,而较少受害。如果对于这种形势认识不足,或者根本不认识,那就要犯绝大的错误。在社会主义这个历史阶段中,必须坚持无产阶级专政,把社会主义革命进行到底,才能防止资本主义复辟,进行社会主义建设,为过渡到共产主义准备条件。

第三,无产阶级专政,是工人阶级领导的,是以工农联盟为基础的。无产阶级专政,就是工人阶级和在它领导下的人民,对反动阶级、反动派和反抗社会主义改造和社会主义建设的分子实行专政。在人民内部是实行民主集中制。我们的这种民主是任何资产阶级国家所不能有的最广大的民主。

第四,社会主义革命和社会主义建设,必须坚持群众路线,放手发动群众,大搞群众运动。“从群众中来,到群众中去”的群众路线,是我们党一切工作的根本路线。必须坚定地相信群众的多数,首先是工农基本群众的多数。要善于同群众商量办事,任何时候也不要离开群众。反对命令主义和恩赐观点。我国人民在长期革命斗争中创造出来的大鸣、大放、大辩论,是依靠人民群众,解决人民内部矛盾和敌我矛盾的一种重要的革命斗争形式。

第五,不论在社会主义革命中,或者在社会主义建设中,都必须解决依靠谁、争取谁、反对谁的问题。无产阶级和它的先锋队必须对社会主义社会做阶级分析,依靠坚决走社会主义道路的真正可靠的力量,争取一切可能争取的同盟者,团结占人口百分之九十五以上的人民群众,共同对付社会主义的敌人。在农村中,在农业集体化以后,也必须依靠贫农、下中农,才能巩固无产阶级专政,才能巩固工农联盟,才能击败资本主义自发势力,不断地巩固和扩大社会主义阵地。

第六,必须在城市和乡村中普遍地、反复地进行社会主义教育运动。在这个不断地教育人的运动中,要善于组织革命的阶级队伍,提高他们的阶级觉悟,正确地处理人民内部矛盾,团结一切可以团结的人。在这个运动中,要向那些敌视社会主义的资本主义势力和封建势力,向那些地主、富农、反革命分子、资产阶级右派分子,向那些贪污盗窃分子和蜕化变质分子,进行尖锐的针锋相对的斗争,打败他们对社会主义的进攻,把他们中间的大多数人改造成为新人。

第七,无产阶级专政的基本任务之一,就是努力发展社会主义经济。必须在以农业为基础、工业为主导的发展国民经济总方针的指导下,逐步实现工业、农业、科学技术和国防的现代化。必须在发展生产的基础上,逐步地普遍地改善人民群众的生活。

第八,全民所有制经济,同集体所有制经济的两种形式。从集体所有制过渡到全民所有制,从两种所有制过渡到单一的全民所有制,需要有一个相当长的发展过程。集体所有制本身也有一个由低级向高级、由小到大的发展过程。中国人民创造的人民公社,就是解决这个过渡问题的一种适宜的组织形式。

第九,百花齐放、百家争鸣的方针,是促进艺术发展和科学进步的方针,是促进社会主义文化繁荣的方针。教育必须为无产阶级政治服务,必须同生产劳动相结合。劳动人民要知识化,知识分子要劳动化。在科学、文化、艺术、教育队伍中,兴无产阶级思想,灭资产阶级思想,也是长期的、激烈的阶级斗争。我们要经过文化革命,经过阶级斗争、生产斗争和科学实验的革命实践,建立一支广大的、为社会主义服务的、又红又专的工人阶级知识分子的队伍。

第十,必须坚持干部参加集体生产劳动的制度。我们党和国家的干部是普通劳动者,而不是骑在人民头上的老爷。干部通过参加集体生产劳动,同劳动人民保持最广泛的、经常的、密切的联系。这是社会主义制度下一件带根本性的大事,它有助于克服官僚主义,防止修正主义和教条主义。

第十一,绝不要实行对少数人的高薪制度。应当合理地逐步缩小而不应当扩大党、国家、企业、人民公社的工作人员同人民群众之间的个人收入的差距。防止一切工作人员利用职权享受任何特权。

第十二,社会主义国家的人民武装部队必须永远置于无产阶级政党的领导和人民群众的监督之下,永远保持人民军队的光荣传统,军民一致,官兵一致。坚持军官当兵的制度。实行军事民主、政治民主和经济民主。同时,普遍组织和训练民兵,实行全民皆兵的制度。枪杆子要永远掌握在党和人民手里,绝不能让它成为个人野心家的工具。

第十三,人民公安机关必须永远置于无产阶级政党的领导和人民群众的监督之下。在保卫社会主义成果和人民利益的斗争中,要实行依靠广大人民群众和专门机关相结合的方针,不放过一个坏人,不冤枉一个好人。有反必肃,有错必纠。

第十四,在对外政策方面,必须坚持无产阶级国际主义,反对大国沙文主义和民族利己主义。社会主义阵营是国际无产阶级和劳动人民斗争的产物。社会主义阵营不仅属于社会主义各国人民,而且属于国际无产阶级和劳动人民。必须真正实行“全世界无产阶级者联合起来”和“全世界无产者和被压迫民族联合起来”的战斗口号,坚决反对帝国主义和各国反动派的反共、反人民、反革命的政策,援助全世界被压迫阶级和被压迫民族的革命斗争。社会主义国家之间的关系,应当建立在独立自主、完全平等和无产阶级国际主义的相互支持和相互援助的原则的基础上。每一个社会主义国家的建设事业,主要地应当依靠自力更生。如果社会主义国家在对外政策上实行民族利己主义,甚至热衷于同帝国主义合伙瓜分世界,那就是蜕化变质,背叛无产阶级国际主义。

第十五,作为无产阶级先锋队的共产党必须同无产阶级专政一起存在。共产党是无产阶级的最高组织形式。无产阶级的领导作用,就是通过共产党的领导来实现的。在一切部门中,都必须实行党委领导的制度。在无产阶级专政时期,无产阶级政党必须保持和发展它同无产阶级和广大劳动群众的密切联系,保持和发扬它的生气勃勃的革命风格,坚持马克思列宁主义的普遍真理同本国的具体实践相结合的原则,坚持反对修正主义、反对教条主义和反对一切机会主义的斗争。


\begin{maonote}
\mnitem{1}倍倍尔,德国社会民主党和第二国际的创始人和领导人之一。卢森堡,德国社会民主党和第二国际的左翼领袖之一,德国共产党的创始人之一。
\mnitem{2}艾森豪威尔,美国共和党人。一九五三年至一九六一年任美国总统。肯尼迪,美国民主党人。一九六一年一月起任美国总统,一九六三年十一月遇刺身亡。
\mnitem{3}伊凡雷帝,即伊凡四世,俄国第一个沙皇。
\end{maonote}
