
\title{中国人民解放军布告}
\date{一九四九年四月二十五日}
\maketitle


国民党反动派业已拒绝接受和平条件\mnote{1},坚持其反民族反人民的罪恶的战争立场。全国人民希望人民解放军迅速消灭国民党反动派。我们已命令人民解放军奋勇前进,消灭一切敢于抵抗的国民党反动军队,逮捕一切怙恶不悛的战争罪犯,解放全国人民,保卫中国领土主权的独立和完整,实现全国人民所渴望的真正的统一。人民解放军所到之处,深望各界人民予以协助。兹特宣布约法八章,愿与我全体人民共同遵守之。

(一)保护全体人民的生命财产。各界人民,不分阶级、信仰和职业,均望保持秩序,采取和人民解放军合作的态度。人民解放军则采取和各界人民合作的态度。如有反革命分子或其它破坏分子,乘机捣乱、抢劫或破坏者,定予严办。

(二)保护民族工商农牧业。凡属私人经营的工厂、商店、银行、仓库、船舶、码头、农场、牧场等,一律保护,不受侵犯。希望各业员工照常生产,各行商店照常营业。

(三)没收官僚资本。凡属国民党反动政府和大官僚分子所经营的工厂、商店、银行、仓库、船舶、码头、铁路、邮政、电报、电灯、电话、自来水和农场、牧场等,均由人民政府接管。其中,如有民族工商农牧业家私人股份经调查属实者,当承认其所有权。所有在官僚资本企业中供职的人员,在人民政府接管以前,均须照旧供职,并负责保护资财、机器、图表、账册、档案等,听候清点和接管。保护有功者奖,怠工破坏者罚。凡愿继续服务者,在人民政府接管后,准予量才录用,不使流离失所。

(四)保护一切公私学校、医院、文化教育机关、体育场所,和其它一切公益事业。凡在这些机关供职的人员,均望照常供职,人民解放军一律保护,不受侵犯。

(五)除怙恶不悛的战争罪犯和罪大恶极的反革命分子外,凡属国民党中央、省、市、县各级政府的大小官员,“国大”代表,立法、监察委员,参议员,警察人员,区镇乡保甲人员,凡不持枪抵抗、不阴谋破坏者,人民解放军和人民政府一律不加俘虏,不加逮捕,不加侮辱。责成上述人员各安职守,服从人民解放军和人民政府的命令,负责保护各机关资财、档案等,听候接收处理。这些人员中,凡有一技之长而无严重的反动行为或严重的劣迹者,人民政府准予分别录用。如有乘机破坏,偷盗,舞弊,携带公款、公物、档案潜逃,或拒不交代者,则须予以惩办。

(六)为着确保城乡治安、安定社会秩序的目的,一切散兵游勇,均应向当地人民解放军或人民政府投诚报到。凡自动投诚报到,并将所有武器交出者,概不追究。其有抗不报到,或隐藏武器者,即予逮捕查究。窝藏不报者,须受相当的处分。

(七)农村中的封建的土地所有权制度,是不合理的,应当废除。但是废除这种制度,必须是有准备和有步骤的。一般地说来,应当先行减租减息,后行分配土地,并且需要人民解放军到达和工作一个相当长的时期之后,方才谈得到认真地解决土地问题。农民群众应当组织起来,协助人民解放军进行各项初步的改革工作。同时,努力耕种,使现有的农业生产水平不致降低,然后逐步加以提高,借以改善农民生活,并供给城市人民以商品粮食。城市的土地房屋,不能和农村土地问题一样处理。

(八)保护外国侨民生命财产的安全。希望一切外国侨民各安生业,保持秩序。一切外国侨民,必须遵守人民解放军和人民政府的法令,不得进行间谍活动,不得有反对中国民族独立事业和人民解放事业的行为,不得包庇中国战争罪犯、反革命分子及其它罪犯。否则,当受人民解放军和人民政府的法律制裁。

人民解放军纪律严明,公买公卖,不许妄取民间一针一线。希望我全体人民,一律安居乐业,切勿轻信谣言,自相惊扰。切切此布。

中国人民革命军事委员会主席毛泽东

中国人民解放军总司令朱德


\begin{maonote}
\mnitem{1}见本卷\mxnote{向全国进军的命令}{1}。
\end{maonote}
