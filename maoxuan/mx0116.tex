
\title{为争取千百万群众进入抗日民族统一战线而斗争}
\date{一九三七年五月八日}
\thanks{这是毛泽东在一九三七年五月二日至十四日在延安召开的中国共产党全国代表会议上所作的结论。}
\maketitle


同志们!对于我的报告——《中国共产党在抗日时期的任务》,经这几天的讨论,除了个别同志提出了不同意见之外,大家都已表示同意。他们这些不同的意见,颇带重要性,因此我的结论,首先就来讨论这些意见,然后再说到一些其它的问题。

\section{和平问题}

我们党为国内和平而斗争,差不多有两年的时间了。国民党三中全会\mnote{1}后,我们说和平已经取得,“争取和平”的阶段已经过去,新的任务是“巩固和平”,并指出这是同“争取民主”相关联的——用争取民主去巩固和平。我们的这种意见,按照几个同志的说法却不能成立。他们的结论必是相反的,或者是动摇于两者之间的。因为他们说:“日本后退了\mnote{2},南京更动摇了,民族矛盾下降,国内矛盾上升。”根据这种估计,当然无所谓新阶段和新任务,情况回到旧阶段,或者还不如。这种意见,我以为是不对的。

我们说和平取得了,并不是说和平巩固了,相反,我们说它是不巩固的。和平实现与和平巩固是两件事。历史暂时地走回头路是可能的,和平发生波折是可能的,原因就在于日本帝国主义和汉奸亲日派的存在。然而西安事变\mnote{3}后和平实现是事实,这种情况是由多方面促成的(日本进攻的基本方针,苏联和英美法的赞助和平,中国人民的逼迫,共产党在西安事变中的和平方针及停止两个政权敌对的政策,资产阶级的分化,国民党的分化等等),不是蒋介石一个人所能决定和推翻的。要推翻和平必须同多方面势力作战,并且必须同日本帝国主义和亲日派靠拢,才能成功。没有问题,日本帝国主义和亲日派还在企图使中国继续内战。和平没有巩固,正是因为这一点。在这种情况下,我们的结论不是回到“停止内战”或“争取和平”的旧口号去,而是前进一步,提出“争取民主”的新口号,只有这样才能巩固和平,也只有这样才能实现抗战。为什么提出“巩固和平”、“争取民主”、“实现抗战”这样三位一体的口号?为的是把我们的革命车轮推进一步,为的是情况已经允许我们进一步了。如果否认新阶段和新任务,否认国民党的“开始转变”,并且逻辑的结论也将不得不否认一年半以来一切为争取和平而斗争的各派势力努力的成绩,那末,只是把自己停顿在旧位置,一步也没有前进。

为什么这些同志作出这种不妥当的估计呢?原因在于他们观察时局不从根本之点出发,而从许多局部和一时的现象(佐藤外交,苏州审判\mnote{4},压制罢工,东北军东调\mnote{5},杨虎城出洋\mnote{6}等等)出发,于是形成一幅暗淡的画图。我们说国民党已经开始转变,但我们同时即说国民党并没有彻底转变。国民党的十年反动政策,要它彻底转变而不用我们和人民的新的更多更大的努力,这是不能设想的事情。不少号称“左”倾的人们,平日痛骂国民党,在西安事变中主张杀蒋和“打出潼关去”\mnote{7},及至和平刚刚实现又发现苏州审判等事,就用惊诧的口气发问道:“为什么蒋介石又这样干?”这些人们须知:共产党员和蒋介石都不是神仙,且都不是孤立的个人,而是处于一个党派、一个阶级里头的分子。共产党有本领把革命逐步地推向前进,但没有本领把全国的坏事在一个早晨去掉干净。蒋介石或国民党已经开始了他们的转变,但没有全国人民的更大努力,也决不会在一个早晨把他们的十年污浊洗掉得干净。我们说运动的方向是向着和平、民主和抗战,但不是说不经努力能够把内战、独裁和不抵抗的旧毒扫除干净。旧毒,污浊,革命进程中的某些波折,以及可能的回头路,只有斗争和努力才能够克服,而且需要长期的斗争和努力。

“他们是一心要破坏我们。”对的,他们总是在企图破坏我们,我完全承认这种估计的正确,不估计这一点就等于睡觉。但问题在破坏的方式是否有了改变。我以为是有了改变的。从战争和屠杀的政策改变到改良和欺骗的政策,从硬的政策改变到软的政策,从军事政策改变到政治政策。为什么有这种改变?资产阶级和国民党处在日本帝国主义面前不得不暂时向无产阶级找同盟军,也和我们向资产阶级找同盟军一样。观察问题应从这一点出发。国际上,法国政府由仇苏变为联苏\mnote{8},同此道理。我们在国内的任务,也从军事的变到政治的。我们不需要阴谋诡计,我们的目的在团结资产阶级和国民党中一切同情抗日的分子,共同战胜日本帝国主义。

\section{民主问题}

“强调民主是错误的,仅仅应该强调抗日;没有抗日的直接行动,就不能有民主运动;多数人只要抗日不要民主,再来一个‘一二九’就对了。”

让我首先发出一点问题:能够在过去阶段中(一九三五年一二九运动\mnote{9}到一九三七年二月国民党三中全会)说,多数人只要抗日不要和平吗?过去强调和平是错了吗?没有抗日的直接行动就不能有和平运动吗?(西安事变和国民党三中全会正在绥远抗战\mnote{10}结束之后,现在也还没有绥远抗战或“一二九”。)谁人不知:要抗日就要和平,无和平不能抗日,和平是抗日的条件。前一阶段一切直接间接的抗日行动(从“一二九”起到国民党三中全会止)都围绕着争取和平,和平是前一阶段的中心一环,是抗日运动在前一阶段中的最本质的东西。

对于抗日任务,民主也是新阶段中最本质的东西,为民主即是为抗日。抗日与民主互为条件,同抗日与和平、民主与和平互为条件一样。民主是抗日的保证,抗日能给予民主运动发展以有利条件。

新阶段中,我们希望有、也将会有许多直接的间接的反日斗争,这些将推动对日抗战,也大有助于民主运动。然而历史给予我们的革命任务,中心的本质的东西是争取民主。“民主”,“民主”是错的吗?我以为是不错的。

“日本退后了,英日向着平衡,南京更动摇了。”这是一种不了解历史发展规律而发生的不适当的忧虑。日本如因国内革命而根本后退,这是有助于中国革命的,是我们所希望的,是世界侵略战线崩溃的开始,为什么还忧虑?然而究竟还不是这样;佐藤外交是大战的准备,大战在我们面前。英国的动摇政策只能落得无结果,这是英国和日本的不同利害决定了的。南京如果是长期动摇,便变为全国人民之敌,也为南京的利益所不许。一时的后退现象,不能代替总的历史规律。因此不能否认新阶段,也不能否认民主任务的提出。况且无论什么情况,民主的口号都能适应,民主对于中国人是缺乏而不是多余,这是人人明白的。何况实际情况已经表明,指出新阶段和提出民主任务,是向抗战接近一步的东西。时局已经前进了,不要把它拉向后退。

“为什么强调国民大会?”因为它是可能牵涉到全部生活的东西,因为它是从反动独裁到民主的桥梁,因为它带着国防性,因为它是合法的。收复冀东察北、反对走私、反对“经济提携”等等,如像同志们所提出的,都是很对的,但这丝毫也不与民主任务和国民大会相矛盾,二者正是互相完成的,但中心的东西是国民大会和人民自由。

日常的反日斗争和人民生活斗争,要和民主运动相配合,这是完全对的,也是没有任何争论的。但目前阶段里中心和本质的东西,是民主和自由。

\section{革命前途问题}

有几个同志发出了这个问题,我的答复只能是简单的。

两篇文章,上篇与下篇,只有上篇做好,下篇才能做好。坚决地领导民主革命,是争取社会主义胜利的条件。我们是为着社会主义而斗争,这是和任何革命的三民主义者不相同的。现在的努力是朝着将来的大目标的,失掉这个大目标,就不是共产党员了。然而放松今日的努力,也就不是共产党员。

我们是革命转变论\mnote{11}者,主张民主革命转变到社会主义方向去。民主革命中将有几个发展阶段,都在民主共和国口号下面。从资产阶级占优势到无产阶级占优势,这是一个斗争的长过程,争取领导权的过程,依靠着共产党对无产阶级觉悟程度组织程度的提高,对农民、对城市小资产阶级觉悟程度组织程度的提高。

无产阶级的坚固的同盟者是农民,其次是城市小资产阶级。同我们争领导权的是资产阶级。

对资产阶级的动摇和不彻底性的克服,依靠群众的力量和正确的政策,否则资产阶级将反过来克服无产阶级。

不流血的转变是我们所希望的,我们应该力争这一着,结果将看群众的力量如何而定。

我们是革命转变论者,不是托洛茨基主义的“不断革命”论\mnote{12}者。我们主张经过民主共和国的一切必要的阶段,到达于社会主义。我们反对尾巴主义,但又反对冒险主义和急性病。

因为资产阶级参加革命的暂时性而不要资产阶级,指联合资产阶级的抗日派(在半殖民地)为投降主义,这是托洛茨基主义的说法,我们是不能同意的。今天的联合资产阶级抗日派,正是走向社会主义的必经的桥梁。

\section{干部问题}

指导伟大的革命,要有伟大的党,要有许多最好的干部。在一个四亿五千万人的中国里面,进行历史上空前的大革命,如果领导者是一个狭隘的小团体是不行的,党内仅有一些委琐不识大体、没有远见、没有能力的领袖和干部也是不行的。中国共产党早就是一个大政党,经过反动时期的损失它依然是一个大政党,它有了许多好的领袖和干部,但是还不够。我们党的组织要向全国发展,要自觉地造就成万数的干部,要有几百个最好的群众领袖。这些干部和领袖懂得马克思列宁主义,有政治远见,有工作能力,富于牺牲精神,能独立解决问题,在困难中不动摇,忠心耿耿地为民族、为阶级、为党而工作。党依靠着这些人而联系党员和群众,依靠着这些人对于群众的坚强领导而达到打倒敌人之目的。这些人不要自私自利,不要个人英雄主义和风头主义,不要懒惰和消极性,不要自高自大的宗派主义,他们是大公无私的民族的阶级的英雄,这就是共产党员、党的干部、党的领袖应该有的性格和作风。我们死去的若干万数的党员,若干千数的干部和几十个最好的领袖遗留给我们的精神,也就是这些东西。我们无疑地应该学习这些东西,把自己改造得更好一些,把自己提高到更高的革命水平。但是还不够,还要作为一种任务,在全党和全国发现许多新的干部和领袖。我们的革命依靠干部,正像斯大林所说的话:“干部决定一切。”\mnote{13}

\section{党内民主问题}

要达到这种目的,党内的民主是必要的。要党有力量,依靠实行党的民主集中制去发动全党的积极性。在反动和内战时期,集中制表现得多一些。在新时期,集中制应该密切联系于民主制。用民主制的实行,发挥全党的积极性。用发挥全党的积极性,锻炼出大批的干部,肃清宗派观念的残余,团结全党像钢铁一样。

\section{大会的团结和全党的团结}

大会中政治问题上的不同意见,经过说明已经归于一致了;过去中央路线和个别同志领导的退却路线之间的分歧,也已经解决了\mnote{14},表示了我们的党已经团结得很坚固。这种团结是当前民族和民主革命的最重要的基础;因为只有经过共产党的团结,才能达到全阶级和全民族的团结,只有经过全阶级全民族的团结,才能战胜敌人,完成民族和民主革命的任务。

\section{为争取千百万群众进入抗日民族统一战线而斗争}

我们的正确的政治方针和坚固的团结,是为着争取千百万群众进入抗日民族统一战线这个目的。无产阶级、农民、城市小资产阶级的广大群众,有待于我们宣传、鼓动和组织的工作。资产阶级抗日派的和我们建立同盟,也还待我们的进一步工作。把党的方针变为群众的方针,还须要我们长期坚持的、百折不挠的、艰苦卓绝的、耐心而不怕麻烦的努力。没有这样一种努力是一切都不成功的。抗日民族统一战线的组成、巩固及其任务的完成,民主共和国在中国的实现,丝毫也不能离开这一争取群众的努力。如果经过这种努力而争取千百万群众在我们领导之下的话,那我们的革命任务就能够迅速地完成。我们的努力将确定地打倒日本帝国主义,并实现全部的民族解放和社会解放。


\begin{maonote}
\mnitem{1}见本卷\mxnote{中国共产党在抗日时期的任务}{11}。
\mnitem{2}西安事变以后,日本帝国主义为了破坏当时已开始实现的中国国内和平和正在逐渐形成中的抗日民族统一战线,在加紧准备以武力征服中国的同时,表面上对国民党当局暂时采取了和缓姿态。一九三六年十二月和一九三七年一月,日本帝国主义曾两次唆使伪蒙古军政府发表通电,拥护国民党政府集中军力进攻红军和张学良、杨虎城部队,宣称同国民党军队停止作战。一九三七年三月,日本外相佐藤尚武诡称要调整中日两国的关系,协助中国的“统一和复兴”。日本财阀儿玉谦次等还组织了所谓“经济考察团”来华,诡称要协助中国建成现代国家。所谓“佐藤外交”和“日本后退”,就是指当时日本帝国主义玩弄的这一套骗人的阴谋。
\mnitem{3}参见本卷\mxnote{关于蒋介石声明的声明}{1}。
\mnitem{4}一九三六年十一月,国民党政府逮捕了全国各界救国联合会领导人沈钧儒、章乃器、邹韬奋、李公朴、王造时、沙千里、史良等七人,随后又把他们押到苏州,在国民党江苏高等法院看守所内监禁。一九三七年四月,这个法院的检察官对沈等提出“公诉”,并于六月十一日和六月二十五日两次开庭审判,说他们违犯了所谓“危害民国紧急治罪法”。
\mnitem{5}西安事变以前,东北军驻在陕西、甘肃境内,同西北红军直接接触,深受中国共产党抗日民族统一战线政策的影响,促成了西安事变的发生。一九三七年三月,国民党反动派为了隔离红军和东北军的关系,并且乘机分裂东北军内部,强令东北军东调河南、安徽和苏北地区。
\mnitem{6}杨虎城(一八九三——一九四九),陕西蒲城人,原西北军爱国将领。曾任国民党军第十七路军总指挥、西安绥靖公署主任。一九三六年十二月和张学良一起发动西安事变。张学良在释放蒋介石后送蒋回南京,即被长期囚禁。杨虎城也被国民党反动派迫令于一九三七年四月二十七日辞职,六月二十九日出国“考察”。抗日战争爆发后,杨虎城于一九三七年十一月回国准备参加抗日工作,但不久也被蒋介石逮捕长期监禁,到一九四九年九月人民解放军迫近重庆的时候,在集中营内遇害。
\mnitem{7}潼关是陕西、河南、山西三省交界处的军事重地。西安事变时,国民党中央军驻在潼关以东,准备进攻东北军和西北军。当时某些号称“左”倾的人们(张国焘是其中之一),主张“打出潼关去”,向国民党中央军进攻。这种主张是同中共中央和平解决西安事变的方针相反的。
\mnitem{8}一九一八年至一九二〇年,法国政府积极地参加了十四个国家对苏维埃共和国的武装干涉,并在这次干涉失败以后继续执行孤立苏联的反动政策。直到一九三五年五月,由于苏联的日益强大及其和平外交政策在法国人民中的影响,由于法西斯德国对法国的威胁,法国政府才同苏联缔结了互助条约。但是,法国政府后来并未忠实地执行这个条约。
\mnitem{9}见本卷\mxnote{论反对日本帝国主义的策略}{8}。
\mnitem{10}见本卷\mxnote{中国共产党在抗日时期的任务}{15}。
\mnitem{11}参见马克思、恩格斯《共产党宣言》第四部分(《马克思恩格斯选集》第1卷,人民出版社1972年版,第284—286页),列宁《社会民主党在民主革命中的两种策略》第十二、十三部分(《列宁全集》第11卷,人民出版社1987年版,第76—97页)和《联共(布)党史简明教程》第三章第三节(人民出版社1975年版,第68—84页)。
\mnitem{12}参见斯大林《论列宁主义基础》第三部分,《十月革命和俄国共产党人的策略》第二部分,《论列宁主义的几个问题》第三部分(《斯大林选集》上卷,人民出版社1979年版,第199—214、279—293、400—402页)。
\mnitem{13}见一九三五年五月四日斯大林在克里姆林宫举行的红军学院学员毕业典礼上的讲话。原文如下:“人才,干部是世界上所有宝贵的资本中最宝贵最有决定意义的资本。应该了解:在我们目前的条件下,‘干部决定一切’。”(《斯大林选集》下卷,人民出版社1979年版,第373页)
\mnitem{14}这里所说的分歧,指一九三五年至一九三六年间党中央路线和张国焘退却路线之间的分歧。一九三六年十月,红军第四方面军到达甘肃会宁地区同红军第一方面军会合。一九三七年三月下旬,中共中央在延安召开政治局扩大会议,讨论国内政治形势和党的任务,对张国焘路线的错误及其危害进行了系统的批判和总结,使党和红军在思想上、政治上和组织上达到新的一致,标志着这个分歧已经解决。至于后来张国焘公开叛党,堕落为反革命,那已不是领导路线上的问题,而只是个人的叛变行动了。
\end{maonote}

