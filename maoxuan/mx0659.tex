
\title{以农业为基础,以工业为主导}
\date{一九六四年六月六日、六月八日、六月十六日}
\thanks{这是毛泽东同志一九六四年五月十五日至六月十七日在北京举行的中共中央工作会议上讲话的节录。}
\maketitle


\date{一九六四年六月六日}
\section{一、以农业为基础,以工业为主导}

制定计划的方法,过去基本上是学苏联的,比较容易做:先定下来多少钢,然后根据这来计算要多少煤,多少电,多少运输力量,等等;根据这些再计算增加多少城市人口、多少生活福利,是摇计算机的办法。钢的产量一变少,别的一律跟着削减。这种方法是一种不合实际的方法,行不通。这样计算把老天爷就计划不进去。天灾来了,偏不给你那么多粮食,城市人口不能增加那么多,别的就都落空。打仗,也计划不进去。我们不是美国的参谋长,不晓得他什么时候要打。还有各国的革命,也难计划进去。有的国家的人民革命成功了,就需要我们的经济援助,这如何能预计到?

要改变计划方法。这是一个革命。学上了苏联的方法以后,成了习惯势力,似乎很难改变。

这几年,我们摸索出来了一些方法。我们的方针是;以农业为基础,以工业为主导。按照这个方针,制定计划时先看可能生产多少粮食,再看需要多少化肥、农药、机械、钢铁。

年成,如何计划?五年中,按一丰、二平、三欠来定。这样比较切实可靠。先确定,在这样能够生产的粮食、棉花和其他经济作物的基础上,可能搞多少工业。如果年成好些,那就更好。

还要考虑到打仗。要有战略部署,各地党委,不可只管文不管武,只管钱不管枪。只要有帝国主义存在,就有战争危险。要建立战略后方。沿海不是不要了,也要好好安排,发挥支援建设新基地的作用。

两个拳头,一个屁股。基础工业是一个拳头,国防是一个拳头。要使拳头有劲,屁股就要坐稳,屁股就是农业。

基础工业,现在主要解决品种、质量问题。去年钢的数量虽然比过去少了,但品种比过去多了,质量比过去好了,用处比过去还大。关键不在数量上。苏联就是以数量为标准,如果钢的数量标准完不成,就好像整个社会主义建设就不行了。他们年年要增加产量指标,年年搞虚夸。其实数量计划完不成,国家垮不了台。有一定的数量,品种更多了,质量更好了,基础就更巩固了。

农业主要靠大寨精神\mnote{1},自力更生。这不是说可以不要工业支援。水利、化肥、农药都是需要基础工业的。

要按照我们掌握的客观的比例关系安排计划。

计划不能只靠加、减、乘、除。计算出来了,各部门、各地区,就分数字、争人、争钱、打官司,要政治挂帅,要有全局观点,不是根据那个地区自己的愿望,而是根据客观存在,事物本身的规律,来安排计划。

不要老是争钱,争来了钱,就乱花钱。

争取几年内做到不再进口粮食,节省下外汇来多买技术设备,技术资料。

不能乱花钱。不要看到情况好转了,又随便“大办”。“留有余地”过去说了多少次,不照办。这两年照办了。不要情况好了又不照办了。

机关工作人员,大部分可以做到半工作半劳动。这办法值得提倡。懒是出修正主义的根源之一。

文艺界为什么弄那么多协会摆在北京?无所事事,或者办些乱七八糟的事。文艺会演,军队的第一,地方的第二,北京(中央)的最糟。这个协会,那个协会,这一套也是从苏联搬来的,中央文艺团体,还是洋人、死人统治着。一定要深入生活。老搞死人洋人,我们的国家是要亡的。要为工人、贫下中农服务。体育,也要对革命斗争和建设有益处的。

一般干部中,“三门”干部很多(出家门、进学校门、进机关门),“三门”不能很好培养干部。国家将来靠这种干部掌握,就危险。靠“小学门、中学门、大学门”干部也不行。不读书不行,读书太多了也不行。本事,光靠读书不行,要靠实践。我们的国家主要靠在实践中读书的干部掌握。

各省都要搞军事工业。要从工业、农业、文教挤出钱来。不要办那么多正规学校。清华,学生一万多,教职员、家属四万多,这样,领导精神会大大浪费。

\date{一九六四年 六月八日}
\section{二、防修反修,搞三线建设}

我们对赫鲁晓夫开始没有准备他会叛变。现在世界上有两种共产党。一种是真的,一种是假的。十月革命,我们知道修正主义出在苏联有伟大意义。南斯拉夫出修正主义不行,苏联是搞了四十多年,列宁领导的,南斯拉夫是偶然的,苏联不是偶然的了。

我们已经出了,白银厂,小站\mnote{2},过去我们不注意上面的根子。

传下去,传到县,如果出了赫鲁晓夫怎么办?中国出修正主义中央怎么办?要县委顶修正主义中央。

搞一、二、三线,打起仗来准备打烂。

要有第三线,要搞西南后方,要搞快些,但不要毛草。钱就那么多,这就不要把摊铺得那么大,铁路两头铺就快些。

\date{一九六四年六月十六日}
\section{三、对帝国主义不要怕,培养无产阶级革命接班人}

讲二个问题:一个是地方党委抓军事问题,二是要搞接班人。

\subsection*{(一)}

地方党委要搞军事,光看表演可不行。要把民兵很好整顿一下。从组织上、政治上、军事上整顿。组织上整顿就是基干民兵、普通民兵有多少?组织上确定下来,有战士、班、排、连、营、团、师长,而且真正起作用。还有政治工作人员也要组织起来一旦有事,拿起枪来就走。有人说,当三个月民兵精神面貌大不同啦。民兵组织要有组织,有兵、有官,要落实。现在许多地方不落实,要做政治工作,做人的工作。政治落实要有政治机构,有政委、教导员、指导员。政治工作就是做人的工作。要分清民兵中的好人坏人,把坏人清理出去。

无论出什么大事都不要慌慌张张。原子弹打下来就和他干。“自古皆有死,人无信不立。”原子弹都炸光了,帝国主义也不干,他没有剥削对象了。

要教育人民都不要慌。站着死趴着死都一样。

对帝国主义不要怕,怕也不行,越怕越没劲,有准备,不怕,就有劲。

\subsection*{(二)}

要准备后事,接班人问题,帝国主义说我们第一代没问题,第二代也变不了,第三代第四代就有希望了。帝国主义这个希望能不能实现呢?帝国主义这话灵不灵?希望讲得不灵,但也可能灵,苏联就是第三代出了苏联赫鲁晓夫修正主义的,我们也可能出修正主义。如何防止修正主义?怎样培养无产阶级的革命接班人?我看有五条:

第一条,要教育干部懂得一些马列主义,懂得多一些更好。就是说,要搞马列主义,不搞修正主义。

第二条,要为大多数人民谋利益,为中国人民大多数谋利益,为世界人民大多数谋利益,不是为少数人,不是为剥削阶级,不是为资产阶级,不是为地、富、反、坏、右。

没有这一条,不能当支部书记,更不能当中央委员。赫鲁晓夫是为少数人的利益,我们是为大多数人的利益。

第三条,要能够团结大多数人。所谓团结大多数人,包括从前反对自己反对错了的人,不管他是哪个山头的,不要记仇,不能“一朝天子一朝臣”。我们的经验证明,如果不是“七大”的正确的团结方针,我们的革命就不能胜利。

对于搞阴谋诡计的人要注意,如中央就出了高、饶、彭、黄\mnote{3}等人。事物都是一分为二的。有的人就是要搞阴谋,他要搞,有什么办法,现在还有要搞的嘛!搞阴谋的人,是客观存在,不是我们喜欢不喜欢的问题。

一切事物都是对立的统一。五个指头,四个指头向一边,大拇指向另一边,这才捏得拢。

完全的纯是没有的,这个道理许多人没有想通。不纯才成其为自然界,成其为社会。完全的纯就不成其为自然界,不成其为社会,不合乎辩证规律。不纯是绝对的,纯是相对的,这就是对立的统一。扫地,一天到晚扫二十四个钟头,还是有尘土。你们看,我们党的历史上哪年纯过吗?但是却没有把我们搞垮。帝国主义也好,我们党里冒出来的修正主义也好,都没有把我们搞垮。解放以后出了高岗、饶漱石、彭德怀,搞垮了我们没有?没有。搞垮我们是不容易的,这是历史经验。

人是可以改变的。有少数人变不了,吃了饭就骂人,各省都有一点,是极少数,不变也可以,让他们去骂。对那些犯错误的人,要劝他们改好。要帮助人家改正。只要他认真改正了,就不要老是批评没完。

要团结广大群众,团结广大干部,团结这两个百分之九十五。

第四条,有事要跟同志们商量,要充分酝酿,要听各种意见,反对的意见也可以让他讲出来。要讲民主,不要“一言堂”,一开会就自己讲几个钟头,不让人家讲话。不要开会时赞成,会后又翻案,又说不赞成。共产党人要搞民主作风,不能搞家长作风。

第五条,自己有了错误,要作自我批评。一个指挥员指挥打仗,三个仗,胜二个,败一个,就可以当下去。打主意,对的多,错的少一点,就行了。不要总是以为自己对,好像真理都在自己手里。不要总是认为只有自己才行,别人什么都不行,好像世界上没有自己,地球就不转了。自然界和人类社会都是按照自己的规律前进的。无产阶级的大人物,像马克思、恩格斯、列宁、斯大林不是都逝世了吗?世界革命还是在前进。

但是,接班人的问题还是要部署一下。要准备好接班人。无产阶级的革命接班人总是要在大风大浪中成长的。


\begin{maonote}
\mnitem{1}大寨的名称是因为北宋时,宋军在此驻兵抗击辽兵,因此得名。全村共有一百六十多户人家五百一十口人。村东西长约两公里,南北宽约一公里,总面积约为一点八八平方公里,海拔为一千一百六十二点六米。全村共有七百多亩地,但被山梁、沟壑分割成四千八百多块,恶劣的地形俗称“七沟八梁一面坡”。全年无霜期只有五个多月,十年九旱,平均亩产只有七八十斤,自然环境非常恶劣。

一九四六年成立了互助组,一九五二年陈永贵担任大寨村的党支部书记。一九五三年办起了农业生产合作社,一九五八年又率先成立了人民公社。这一期间在陈永贵的带领下,大寨人从山下担土到山上,造起了汗涝保收的人工梯田,平均个劳动力搬运土石方作业量达一千多立方米,担土八十多万担;每人每年担石头八百八十多担,担粪、担庄稼十万斤。一九六四年二月十日,《人民日报》刊登了新华社记者的通讯报道《大寨之路》,介绍了大寨村的先进事迹,并配发社论《用革命精神建设山区的好榜样》,号召中国人民学习大寨的战天斗地的精神,全国掀起了学习大寨热。周恩来总理将大寨精神概括为三句话,即:“政治挂帅、思想领先的原则,自力更生、艰苦奋斗的精神,爱国家、爱集体的共产主义风格。”并将它庄重地写入了三届人大的政府工作报告。
\mnitem{2}指天津市委《关于小站地区夺权斗争的报告》和甘肃省委和冶金工业部党组《关于夺回白银有色金属公司的领导权的报告》。
\mnitem{3}高饶彭黄指高岗、饶漱石、彭德怀、黄克诚。
\end{maonote}
