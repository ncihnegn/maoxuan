
\title{为人民服务}
\date{一九四四年九月八日}
\thanks{这是毛泽东在中共中央警备团追悼张思德的会上的讲演。}
\maketitle


我们的共产党和共产党所领导的八路军、新四军,是革命的队伍。我们这个队伍完全是为着解放人民的,是彻底地为人民的利益工作的。张思德\mnote{1}同志就是我们这个队伍中的一个同志。

人总是要死的,但死的意义有不同。中国古时候有个文学家叫做司马迁的说过:“人固有一死,或重于泰山,或轻于鸿毛。”\mnote{2}为人民利益而死,就比泰山还重;替法西斯卖力,替剥削人民和压迫人民的人去死,就比鸿毛还轻。张思德同志是为人民利益而死的,他的死是比泰山还要重的。

因为我们是为人民服务的,所以,我们如果有缺点,就不怕别人批评指出。不管是什么人,谁向我们指出都行。只要你说得对,我们就改正。你说的办法对人民有好处,我们就照你的办。“精兵简政”这一条意见,就是党外人士李鼎铭\mnote{3}先生提出来的;他提得好,对人民有好处,我们就采用了。只要我们为人民的利益坚持好的,为人民的利益改正错的,我们这个队伍就一定会兴旺起来。

我们都是来自五湖四海,为了一个共同的革命目标,走到一起来了。我们还要和全国大多数人民走这一条路。我们今天已经领导着有九千一百万人口的根据地\mnote{4},但是还不够,还要更大些,才能取得全民族的解放。我们的同志在困难的时候,要看到成绩,要看到光明,要提高我们的勇气。中国人民正在受难,我们有责任解救他们,我们要努力奋斗。要奋斗就会有牺牲,死人的事是经常发生的。但是我们想到人民的利益,想到大多数人民的痛苦,我们为人民而死,就是死得其所。不过,我们应当尽量地减少那些不必要的牺牲。我们的干部要关心每一个战士,一切革命队伍的人都要互相关心,互相爱护,互相帮助。

今后我们的队伍里,不管死了谁,不管是炊事员,是战士,只要他是做过一些有益的工作的,我们都要给他送葬,开追悼会。这要成为一个制度。这个方法也要介绍到老百姓那里去。村上的人死了,开个追悼会。用这样的方法,寄托我们的哀思,使整个人民团结起来。


\begin{maonote}
\mnitem{1}张思德,四川仪陇人,中央警备团的战士。他在一九三三年参加红军,经历长征,负过伤,是一个忠实为人民服务的共产党员。一九四四年九月五日在陕北安塞县山中烧炭,因炭窑崩塌而牺牲。
\mnitem{2}司马迁,中国西汉时期著名的文学家和历史学家,着有《史记》一百三十篇。此处引语见《汉书·司马迁传》中的《报任少卿书》,原文是:“人固有一死,死有重于泰山,或轻于鸿毛。”
\mnitem{3}李鼎铭(一八八一——一九四七),陕西米脂人,开明绅士。他在一九四一年十一月陕甘宁边区第二届参议会上提出“精兵简政”的提案,并在这次会议上当选为陕甘宁边区政府副主席。
\mnitem{4}这是指当时陕甘宁边区和华北、华中、华南各抗日根据地所拥有的人口的总数。
\end{maonote}
