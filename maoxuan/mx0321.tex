
\title{文化工作中的统一战线}
\date{一九四四年十月三十日}
\thanks{这是毛泽东在陕甘宁边区文教工作者会议上所作的讲演。}
\maketitle


我们的一切工作都是为了打倒日本帝国主义。日本帝国主义和希特勒一样,快要灭亡了。但是还须我们继续努力,才能最后地消灭它。我们的工作首先是战争,其次是生产,其次是文化。没有文化的军队是愚蠢的军队,而愚蠢的军队是不能战胜敌人的。

解放区的文化已经有了它的进步的方面,但是还有它的落后的方面。解放区已有人民的新文化,但是还有广大的封建遗迹。在一百五十万人口的陕甘宁边区内,还有一百多万文盲,两千个巫神,迷信思想还在影响广大的群众。这些都是群众脑子里的敌人。我们反对群众脑子里的敌人,常常比反对日本帝国主义还要困难些。我们必须告诉群众,自己起来同自己的文盲、迷信和不卫生的习惯作斗争。为了进行这个斗争,不能不有广泛的统一战线。而在陕甘宁边区这样人口稀少、交通不便、原有文化水平很低的地方,加上在战争期间,这种统一战线就尤其要广泛。因此,在教育工作方面,不但要有集中的正规的小学、中学,而且要有分散的不正规的村学、读报组和识字组。不但要有新式学校,而且要利用旧的村塾加以改造。在艺术工作方面,不但要有话剧,而且要有秦腔\mnote{1}和秧歌。不但要有新秦腔、新秧歌,而且要利用旧戏班,利用在秧歌队总数中占百分之九十的旧秧歌队,逐步地加以改造。在医药方面,更是如此。陕甘宁边区的人、畜死亡率都很高,许多人民还相信巫神。在这种情形之下,仅仅依靠新医是不可能解决问题的。新医当然比旧医高明,但是新医如果不关心人民的痛苦,不为人民训练医生,不联合边区现有的一千多个旧医和旧式兽医,并帮助他们进步,那就是实际上帮助巫神,实际上忍心看着大批人畜的死亡。统一战线的原则有两个:第一个是团结,第二个是批评、教育和改造。在统一战线中,投降主义是错误的,对别人采取排斥和鄙弃态度的宗派主义也是错误的。我们的任务是联合一切可用的旧知识分子、旧艺人、旧医生,而帮助、感化和改造他们。为了改造,先要团结。只要我们做得恰当,他们是会欢迎我们的帮助的。

我们的文化是人民的文化,文化工作者必须有为人民服务的高度的热忱,必须联系群众,而不要脱离群众。要联系群众,就要按照群众的需要和自愿。一切为群众的工作都要从群众的需要出发,而不是从任何良好的个人愿望出发。有许多时候,群众在客观上虽然有了某种改革的需要,但在他们的主观上还没有这种觉悟,群众还没有决心,还不愿实行改革,我们就要耐心地等待;直到经过我们的工作,群众的多数有了觉悟,有了决心,自愿实行改革,才去实行这种改革,否则就会脱离群众。凡是需要群众参加的工作,如果没有群众的自觉和自愿,就会流于徒有形式而失败。“欲速则不达”\mnote{2},这不是说不要速,而是说不要犯盲动主义,盲动主义是必然要失败的。在一切工作中都是如此;在改造群众思想的文化教育工作中尤其是如此。这里是两条原则:一条是群众的实际上的需要,而不是我们脑子里头幻想出来的需要;一条是群众的自愿,由群众自己下决心,而不是由我们代替群众下决心。


\begin{maonote}
\mnitem{1}秦腔,又名梆子腔,是流行于中国西北地区的具有悠久历史的地方戏曲。
\mnitem{2}见《论语·子路》。
\end{maonote}
