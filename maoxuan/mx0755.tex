
\title{帝国主义怕第三世界}
\date{一九七〇年七月十一日}
\thanks{这是毛泽东同志同坦桑尼亚政府代表团和赞比亚政府代表团谈话的一部分。}
\maketitle


英联邦像是个俱乐部,中国过去也是这个俱乐部的。英国第一次世界大战时还行,第二次世界大战以后它顾不了了,把中国让给了美国。所以,日本投降之后,美国帮助蒋介石打我们。而我们这些人就被叫做共产党“匪徒”,跟普通的土匪不同,是共产党“匪徒”。这样,我们就没有资格进那个联合国。哪有“匪徒”能够进联合国的?我们也想了一想,我们这个国家也算个联合国嘛!我不到你那个联合国去也可以嘛!我们这些人,帝国主义都是不高兴的。无论是你们的两位总统也好,还是我们这些人也好,都不大中它们的意。那有什么办法呢?你们到北京来也没有通知英国、美国吧?

实际上现在世界上帝国主义的日子不大好过。它们怕第三世界,既怕你们这些人,也怕我们这些人。要破除迷信,不要迷信那个什么帝国主义。当然,我不是说帝国主义国家的人民都要反对,也不是说帝国主义国家的技术不可以学习,而是说对帝国主义的政治的迷信,对它们那套欺骗,要破除。要破除对帝国主义的这种迷信不容易,它在一些人中根深蒂固。你看,帝国主义多了不起,它们有那么多原子弹、氢弹,飞机到处飞,海军到处跑,到处占领别人的国家,比如美国出兵柬埔寨\mnote{1}。但它们那个办法是老牌帝国主义英国的做法。英国不是到处占领吗?现在它比较乖乖的了。

\begin{maonote}
\mnitem{1}一九七〇年三月,美国在柬埔寨策动军事政变,颠覆西哈努克政府,扶植以朗诺为首的右翼政府,四月三十日,美国政府以柬埔寨有越南南方人民武装的“庇护所”为借口,派遣美国军队和南越雇佣军入侵柬埔寨。
\end{maonote}
