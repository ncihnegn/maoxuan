
\title{坚定地相信群众的大多数}
\date{一九五七年十月十三日}
\thanks{这是毛泽东同志在最高国务会议第十三次会议上的讲话。}
\maketitle


现在整风找出了一种形式,就是大鸣,大放,大辩论,大字报。这是群众创造的一种新形式,跟我们党历史上采取过的形式是有区别的。延安那一次整风,也出了一点大字报,但是那个时候我们没有提倡。后来“三查三整”,也没有采取这种形式。在革命战争时期,没有人给我们发饷,没有制造枪炮的工厂,我们的党和军队就是依靠战士,依靠当地人民,依靠群众。所以,长期以来,形成了一种民主作风。但是,那个时候,就没有现在这样的大鸣,大放,大辩论,大字报。这是什么理由?就是那个时候金鼓齐鸣,在打仗,阶级斗争那么尖锐,如果内部这么大闹,那就不好了。现在不同了,战争结束了,全国除台湾省外都解放了。所以,就出现了这种新形式。新的革命内容,它要找到新的形式。现在的革命是社会主义革命,是为了建设社会主义国家,它找到了这种新形式。这种形式,可以很快普及,很快学会,几个月就可以学会。

对大鸣、大放、大辩论、大字报,主要有两怕:一个是怕乱。你们怕不怕乱?我看有许多人是怕乱的。还有一个是怕下不得台。当工厂厂长的,当合作社主任的,当学校校长的,当党委书记的,怕一放出来,火一烧,怎么下台呀?现在容易说通了,在五月间那个时候,就很不容易说服人。北京三十四个大专院校,开了很多会才放开。为什么可以不怕?为什么放有利?大鸣大放有利,还是小鸣小放有利?或者不鸣不放有利?不鸣不放是不利的,小鸣小放不能解决问题,还是要大鸣大放。大鸣大放,一不会乱,二不会下不得台。当然,个别的人除外,比如丁玲,她就下不得台。还有冯雪峰,他在那里放火,目的是要烧共产党,就下不得台。那是少数人,是右派。其它的人就不要怕下不得台,可以下台的。无非是官僚主义、宗派主义、主观主义之类的毛病,有则改之,不应当怕。基础就是要相信群众的大多数,相信人民中间的大多数是好人。工人的大多数是好人,农民的大多数是好人。共产党里,青年团里,大多数是好人。他们不是想要把我们国家搞乱。资产阶级知识分子、资本家、民主党派成员的多数,是可以改造的。所以我们不要怕乱,不会乱,乱不了。应当相信多数,这里所谓多数,是不是百分之五十一呢?不是的,是百分之九十到百分之九十八。

社会主义革命对我们都是新的。我们过去只搞过民主革命,那是资产阶级性质的革命,不破坏个体所有制,不破坏民族资本主义所有制,只破坏帝国主义所有制,封建主义所有制,官僚资本主义所有制。所以,有许多人,民主革命这一关可以过来。这里头,有些人对彻底的民主革命就不热心,是勉强过来的;有些人对彻底的民主革命是肯干的,这一关过来了。现在是过社会主义的关,有些人就难过。比如,湖北有那么一个雇农出身的党员,他家是三代要饭,解放后翻身了,发家了,当了区一级干部。这回他非常不满意社会主义,非常不赞成合作化,要搞“自由”,反对统购统销。现在开了他的展览会,进行阶级教育,他痛哭流涕,表示愿意改正错误。为什么社会主义这个关难过呢?因为这一关是要破资本主义所有制,使它变为社会主义全民所有制,要破个体所有制,使它变为社会主义集体所有制。当然,这个斗争要搞很多年的,究竟多长时间叫过渡时期,现在也还很难定。今年是斗争的一个洪峰。以后是不是年年要来一个洪峰?象每年黄河的洪峰要来一样,我看恐怕不是那样。但是,这样的洪峰,以后也还会有的。

现在,全国究竟有多少人不赞成社会主义?我和许多地方同志摸了这个底。在全国总人口中间,大概有百分之十的人,是不赞成或者反对社会主义的。这里包括地主阶级,富农,一部分富裕中农,一部分民族资产阶级,一部分资产阶级知识分子,一部分城市上层小资产阶级,甚至个别的工人、贫下中农。六亿人口的百分之十是多少呢?是六千万人。这个数目不小,不要把它看小了。

我们说要坚定地相信群众的大多数,有两个出发点:第一,我们有百分之九十的人赞成社会主义。这里包括无产阶级,农村里头半无产阶级的贫农,下中农,还有上层小资产阶级的多数,资产阶级知识分子的多数,以及一部分民族资产阶级。第二,在不赞成或者反对社会主义的人里边,最顽固的分子,包括极右派,反革命,搞破坏的,还有不搞破坏但很顽固的,可能要带着顽固头脑到棺材里面去的,这样的人有多少呢?大概只有百分之二左右。全国人口的百分之二是多少呢?就是一千二百万。一千二百万人,如果集合起来,手里拿了枪,那是个很大的军队。但是,为什么天下又不会大乱呢?因为他们是分散在这个合作社,那个合作社;这个农村,那个农村;这个工厂,那个工厂;这个学校,那个学校;这个共产党支部,那个共产党支部;这个青年团支部,那个青年团支部;这个民主党派的支部,那个民主党派的支部;是分散在各处,不能集合,所以天下不会大乱。

社会主义革命是在一个什么范围内的革命,是一些什么阶级之间的斗争呢?就是无产阶级领导劳动人民同资产阶级之间的斗争。我国无产阶级数目比较小,但是它有广大的同盟军,最主要的就是农村里头的贫农、下中农,他们占农村人口的百分之七十或者还要多一点。富裕中农大约占农村人口的百分之二十。现在的富裕中农大体分三部分:赞成合作化的,占百分之四十;动摇的,占百分之四十;反对的,占百分之二十。这几年来,经过教育改造,地主、富农也有分化,现在也有不完全反对社会主义的。对资产阶级和资产阶级知识分子,也要加以分析,不要以为他们都是反对社会主义的,事实不是那样。在全国总人口中,赞成社会主义的,有百分之九十。我们要相信这个多数。经过工作,经过大辩论,还可能争取百分之八,就变成百分之九十八。坚决反社会主义的死硬派,只有百分之二。当然,要注意,刚才邓小平同志讲了,它还是一个很大的力量。

富农是农村的资产阶级,他们在农村说话没有什么人听。地主的名声更臭。买办资产阶级早就臭了。资产阶级和资产阶级知识分子,农村的上层小资产阶级(富裕中农),城市的上层小资产阶级(一些比较富裕的小业主)和他们的知识分子,这些人就有些影响了。特别是这个知识分子吃得开,那一样都缺不了他。办学校要有大学教授、中小学教员,办报纸要有新闻记者,唱戏要有演员,搞建设要有科学家、工程师、技术人员。现在知识分子有五百万人,资本家有七十万人,加在一起,约计六百万人,五口之家,五六就是三千万人。资产阶级和他们的知识分子是比较最有文化的,最有技术的。右派翘尾巴也在这里。罗隆基不是讲过吗,无产阶级的小知识分子就领导不了他这个小资产阶级的大知识分子。他不说他是资产阶级,一定要说他是小资产阶级,是小资产阶级的大知识分子。我看,不仅是无产阶级的小知识分子,就是大字不认得几个的工人、农民,也比他罗隆基高明得多。

资产阶级和他们的知识分子,上层小资产阶级和他们的知识分子,他们里头的右派和中间派,对于共产党、无产阶级的领导是不服气的。讲拥护共产党,拥护宪法,那也是拥护的,手也是举的,但是心里是不那么服气的。这里头就要分别了,右派是对抗的,中间派是半服半不服的。不是有人讲共产党这样也不能领导,那样也不能领导吗?不仅右派有这个思想,中间派有些人也有。总而言之,照他们的说法,差不多就完了,共产党非搬到外国不可,无产阶级非上别的星球不可。因为你这样也不行,那样也不行嘛!无论讲那一行,右派都说你不行。这一次辩论的主要目的,就是争取半服半不服的中间派,使他们懂得这个社会发展规律究竟是一件什么事,还是要听文化不高的无产阶级的话,在农村里头要听贫农、下中农的话。讲文化,无产阶级、贫农、下中农不如他们,但是讲革命,就是无产阶级、贫农、下中农行。这可不可以说服多数人?可以说服多数人。资产阶级的多数,资产阶级知识分子的多数,上层小资产阶级的多数,是可以说服的。大学教授、中小学教员、艺术家、文学家、科学家、工程师中的多数,是可以说服的。不大服气的,过若干年,慢慢就会服气了。

在多数人拥护社会主义这个基础上,在现在这个时候,出现大鸣、大放、大辩论、大字报这种形式,很有益处。这种形式是没有阶级性的。什么大鸣、大放、大字报,右派也可以搞。感谢右派,“大”字是他们发明的。我在今年二月二十七日的讲话中,并没有讲什么大鸣,大放,大辩论,没有这个“大”字。去年五月,我们在这里开会讲百花齐放,那是一个“放”,百家争鸣,那是一个“鸣”,就没有这个“大”字,并且是限于文学艺术上的百花齐放,学术问题上的百家争鸣。后来右派要涉及政治,就是什么问题都要鸣放,叫作鸣放时期,而且要搞大鸣大放。可见,这个口号无产阶级可以用,资产阶级也可以用,左派可以用,中间派可以用,右派也可以用。大鸣、大放、大辩论、大字报,究竟对那个阶级有利?归根结底,对无产阶级有利,对资产阶级右派不利。原因是,百分之九十的人不愿意国家乱,而愿意建成社会主义,百分之十不赞成或者反对社会主义的人中间,有许多人是动摇的,至于坚决反社会主义的分子,只有百分之二。你乱得了呀?所以,大鸣大放的口号,大鸣、大放、大辩论、大字报的方式和方法,归根结底有利于多数人,有利于多数人的自我改造。两条道路,一条社会主义,一条资本主义,归根结底有利于社会主义。

我们不要怕乱,也不要怕下不得台。右派是下不了台的,但也还是可以下台。按照辩证法,我看右派会一分为二。可能有相当多的右派分子,大势所趋,他们想通了,转好了,比较老实,比较不十分顽固了,那个时候把帽子一摘,就不要叫右派了,并且还要安排工作。少数极顽固的,可能死不改悔,戴着右派帽子进棺材,那也没有什么了不起,这样的人总是会有的。

右派这么闹一下,使我们摸了一个底:一方面,赞成社会主义的,是百分之九十,可能争取到百分之九十八;另一方面,不赞成或者反对社会主义的,是百分之十,其中坚决反社会主义的死硬派只有百分之二。摸了这样的底,就心中有数了。在无产阶级政党的领导下,在多数人拥护社会主义的基础上,用我们这个大鸣、大放、大辩论、大字报的办法,可以避免匈牙利那样的事件,也可以避免现在波兰发生的那样的事件。我们不需要象波兰那样封一个刊物,我们只要党报发一两篇社论就行了。对文汇报,我们写了两篇社论批评它,头一篇不彻底,没有讲透问题,再发第二篇社论,它就自己改。新民报也是它自己改。在波兰就不行,他们那里反革命的问题没有解决,右派的问题没有解决,走那条道路的问题没有解决,又不抓对资产阶级思想的斗争,所以封一个刊物就惹起事来了。我看中国的事情好办,我是从来不悲观的。我不是说过,乱不了,不怕乱吗?乱子可以变成好事。凡是放得彻底的地方,鬼叫一个时候,大乱一阵,事情就更好办了。

我国解放以前只有四百万产业工人,现在是一千二百万工人。工人阶级人数虽然少,但只有这个阶级有前途,其它的阶级都是过渡的阶级,都要过渡到工人阶级那方面去。农民头一步过渡到集体化的农民,第二步要变为国营农场的工人。资产阶级要灭掉。不是讲把人灭掉,是把这个阶级灭掉,人要改造。资产阶级知识分子要改造,小资产阶级知识分子也要改造,可以逐步地改造过来,改造成无产阶级知识分子。我讲过,“皮之不存,毛将焉附”,知识分子如果不附在无产阶级身上,就有作“梁上君子”的危险。现在许多人进了工会,有人说进了工会岂不就变成工人阶级了吗?不。有的人进了共产党,他还反共,丁玲、冯雪峰不就是共产党员反共吗?进了工会不等于就是工人阶级,还要有一个改造过程。现在民主党派的成员、大学教授。文学家、作家,他们没有工人朋友,没有农民朋友,这是一个很大的缺点。比如费孝通,他找了二百多个高级知识分子朋友,北京、上海、成都、武汉、无锡等地都有。他在那个圈子里头出不来,还有意识地组织这些人,代表这些人大鸣大放。他吃亏就在这个地方。我说,你可不可以改一改呀?不要搞那二百个,要到工人、农民里头去另找二百个。我看知识分子都要到工农群众中去找朋友,真正的朋友是在工人、农民那里。要找老工人做朋友。在农民中,不要轻易去找富裕中农做朋友,要找贫农、下中农做朋友。老工人辨别方向非常之清楚,贫农、下中农辨别方向非常之清楚。

整风有四个阶段:放,反,改,学。就是一个大鸣大放,一个反击右派,一个整改,最后还有一个,学点马克思列宁主义,和风细雨,开点小组会,搞点批评和自我批评。今年五月一比中共中央发表的整风文件中讲和风细雨,当时许多人不赞成,主要是右派不赞成,他们要来一个急风暴雨,结果很有益处。这一点我们当时也估计到了。因为延安那一次整风就是那样,你讲和风细雨,结果来了个急风暴雨,但是,最后还是归结到和风细雨。一个工厂,大字报一贴,几千张,那个工厂领导人也是很难受的。有那么十天左右的时间,有些人就不干了,想辞职,说是受不了,吃不下饭,睡不着觉。北京那些大学的党委书记就吃不下饭,睡不着觉。那时候右派他们说,你们不能驳,只能他们鸣放。我们也讲,要让他们放,不要驳。所以,五月我们不驳,六月八日以前,我们一概不驳,这样就充分鸣放出来了。鸣放出来的东西,大概百分之九十以上是正确的,有百分之几是右派言论。在那个时候,就是要硬着头皮听,听了再反击。每个单位都要经过这么一个阶段。这个整风,每个工厂,每个合作社都要搞。现在军队也是这样搞。这样搞一下很必要。只要你不搞,“自由市场”又要发展的。世界上有些事就是那么怪,三年不整风,共产党、青年团、民主党派、大学教授、中小学教员、新闻记者、工程师、科学家里头,又要出许多怪议论,资本主义思想又要抬头。比如房子每天要打扫,脸每天要洗一样,整风我看以后大体上一年搞一次,一次个把月就行了。也许那时候还要来一点洪峰。现在这个洪峰不是我们造成的,是右派造成的。我们不是讲过吗?共产党里头出了高岗,你们民主党派一个高岗都没有呀?我就不信。现在共产党又出了丁玲、冯雪峰、江丰这么一些人,你们民主党派不是也出了吗?

资产阶级和资产阶级知识分子,要承认有改造的必要。右派就不承认自己有改造的必要,而且影响其它一些人也不大愿意改造,说自己已经改造好了。章乃器说,改造那怎么得了,那叫做抽筋剥皮。我们说要脱胎换骨,他说脱胎换骨就会抽筋剥皮。这位先生,谁人去抽他的筋,剥他的皮?许多人忘记了我们的目的是干什么,为什么要这么搞,社会主义有什么好处。为什么要思想改造?就是为了要资产阶级知识分子建立无产阶级世界观,改造成为无产阶级知识分子。那些老知识分子会要逼得非变不可,因为新知识分子起来了。讲学问,你说他现在不行,他将来是会行的。这批新的人出来了,就对老科学家、老工程师、老教授、老教员将了一军,逼得他们非前进不可。我们估计,大多数人是能够前进的,一部分是能够改造成无产阶级知识分子的。

无产阶级必须造就自己的知识分子队伍,这跟资产阶级要造就它自己的知识分子队伍一样。一个阶级的政权,没有自己的知识分子那是不行的。美国没有那样一些知识分子,它资产阶级专政怎么能行?我们是无产阶级专政,一定要造就无产阶级自己的知识分子队伍,包括从旧社会来的经过改造真正站稳工人阶级立场的一切知识分子。右派中间那些不愿意变的,大概章乃器算一个。你要他变成无产阶级知识分子,他就不干,他说他早已变好了,是“红色资产阶级”。自报公议嘛,你自报可以,大家还要公议。我们说,你还不行,你章乃器是白色资产阶级。有人说,要先专后红。所谓先专后红,就是先白后红。他在这个时候不红,要到将来再红,这个时候不红,他是什么颜色呀?还不是白色的。知识分子要同时是红的,又是专的。要红,就要下一个决心,彻底改造自己的资产阶级世界观。这并不是要读很多书,而是要真正弄懂什么叫无产阶级,什么叫无产阶级专政,为什么只有无产阶级有前途,其它阶级都是过渡的阶级,为什么我们这个国家要走社会主义道路,不能走资本主义道路,为什么一定要共产党领导等等问题。

我在四月三十日讲的那些话,许多人就听不进去。“皮之不存,毛将焉附”?我说中国有五张皮。旧有的三张:帝国主义所有制,封建主义所有制,官僚资本主义所有制。过去知识分子就靠这三张皮吃饭。此外,还靠一个民族资本主义所有制,一个小生产者所有制即小资产阶级所有制。我们的民主革命,是革前三张皮的命,从林则徐算起,一直革了一百多年。社会主义革命是革后两张皮:民族资本主义所有制和小生产者所有制。这五张皮现在都不存了。老皮三张久已不存,另外两张也不存了。现在有什么皮呢?有社会主义公有制这张皮。当然,这又分两部分,一个全民所有制,一个集体所有制。现在靠谁吃饭?民主党派也好,大学教授也好,科学家也好,新闻记者也好,是吃工人阶级的饭,吃集体农民的饭,是吃全民所有制和集体所有制的饭,总起来说,是吃社会主义公有制的饭。那五张旧皮没有了,这个毛呢,现在就在天上飞,落下来也不扎实。知识分子还看不起这张新皮,什么无产阶级、贫农、下中农,实在是太不高明了,上不知天文,下不知地理,“三教九流”\mnote{1}都不如他。他不愿接受马克思列宁主义。这个马克思列宁主义,过去反对的人多,帝国主义反对,蒋介石天天反,说是“共产主义不适合中国国情”,害得大家生怕这个东西。知识分子接受马克思列宁主义,把他的资产阶级世界观改造成无产阶级世界观,这要一个过程,而且要有一个社会主义思想革命运动。今年这个运动,就是开辟这条道路。

现在有些机关、学校,反过右派之后,风平浪静,他就舒舒服服,对提出来的许多正确意见就不肯改了。北京的一些机关、学校就发生这个问题。我看,这个整改又要来一个鸣放高潮。把大字报一贴,你为什么不改?将一军。这个将军很有作用。整改,要有一个短时期,比如一两个月。还要学,学点马克思列宁主义,和风细雨,搞点批评和自我批评,那是在第四个阶段。这个学,当然不是一两个月了,只是讲这个运动告一段落,引起学习的兴趣。

反击右派总要告一个段落嘛!这一点,有的右派估计到了。他说,这个风潮总要过去就是了。很正确呀,你不能老反右派,天天反,年年反。比如,现在北京这个反右派的空气,就比较不那么浓厚了,因为反得差不多了,不过还没有完结,不要松劲。现在有些右派死不投降,象罗隆基、章乃器就是死不投降。我看还要说服他,说几次,他硬是不服,你还能天天同他开会呀?一部分死硬派,他永远不肯改,那也就算了。他们人数很少,摆到那里,摆他几十年,听他怎么办。多数人总是要向前进的。

是不是要把右派分子丢到海里头去呢?我们一个也不丢。右派,因为他们反共反人民反社会主义,所以是一种敌对的力量。但是,现在我们不把他们当作地主、反革命分子那么对待,其基本标志,就是不取消他们的选举权。也许有个别的人,要取消他的选举权,让他劳动改造。我们采取不提人,又不剥夺选举权的办法,给他们一个转弯的余地,以利于分化他们。不是刚才讲分两种人吗?一种是改正了以后,可以把右派分子帽子摘掉,归到人民的队伍;一种就是顽固到底,一直到见阎王。他说,我是不投降的,阎王老爷你看我多么有“骨气”呀!他是资产阶级的忠臣。右派跟封建残余、反革命是有联系的,通气的,彼此呼应的。那个文汇报,地主看了非常高兴,他就买来对农民读,吓唬农民说,你看报纸上载了的呀!他想倒算。还有帝国主义、蒋介石跟右派也是通气的。比如台湾、香港的反动派,对储安平的“党天下”,章伯钧的“政治设计院”,罗隆基的“平反委员会”,是很拥护的。美帝国主义也很同情右派。我曾经跟各位讲过,假如美国人打到北京,你们怎么样?采取什么态度?准备怎么做?是跟美国一起组织维持会?还是跟我们上山?我说,我的主意是上山,第一步到张家口,第二步就到延安。说这个话是极而言之,把问题讲透,不怕乱。你美国占领半个中国我也不怕。日本不是占领了大半个中国吗?后来我们不是打出一个新中国来了吗?我跟日本人谈过,要感谢日本帝国主义,他们这个侵略对于我们很有好处,激发了我们全民族反对日本帝国主义,提高了我国人民的觉悟。

右派不讲老实话,他不老实,瞒着我们干坏事。谁晓得章伯钧搞了那么多坏事?我看这种人是官越做得高,反就越造得大。章罗同盟对长期共存、互相监督,百花齐放、百家争鸣这两个口号最喜欢了。他们利用这两个口号反对我们。我们说要长期共存,他们搞短期共存;我们说要互相监督,他们不接受监督。一个时期他们疯狂得很,结果走到反面,长期共存变成短期共存。章伯钩的部长怎么样呀?部长恐怕当不成了。右派当部长,人民恐怕不赞成吧!还有一些著名的右派,原来是人民代表,现在怎么办?恐怕难安排了。比如丁玲,就不能当人民代表了。有些人,一点职务不安排,一点工作不给做,恐怕也不好。比如钱伟长,恐怕教授还可以当,副校长就当不成了。还有一些人,教授恐怕暂时也不能当,学生不听。那末干什么事呢?可以在学校里头分配一点别的工作,让他有所改造,过几年再教书。这些问题都要考虑,是一个麻烦的问题。革命这个事情就是一个麻烦的事情。对右派如何处理,如何安排,这个问题请诸位去议一下。

各民主党派什么情况,基层什么情况,恐怕你们这些负责人也不摸底。坚决的右派分子,在一个时候,在一些单位,可以把水搞得很混,使我们看不见底。一查,其实只有那么百分之一、二。一把明矾放下去,就看见了底。这次整风,就是放一把明矾。大鸣、大放、大辩论之后,就看得见底了。工厂、农村看得见底,学校看得见底,对共产党、青年团、民主党派,也都有底了。

现在,我讲一讲农业发展纲要四十条。经过两年的实践,基本要求还是那个四、五、八,就是粮食亩产黄河以北四百斤,淮河以北五百斤,淮河以南八百斤。十二年要达到这个目标,这是基本之点。整个纲要基本上没有改,只是少数条文改了。有些问题已经解决了,如合作化问题就基本上解决了,相应的条文就作了修改。有些过去没有强调的,如农业机械、化学肥料,现在要大搞,条文上就加以强调了。还有条文的次序有些调动。这个修改过的农业发展纲要草案,经过人大常委和政协常委联席会议讨论以后,要重新公布,拿到全国农村中去讨论。工厂也可以讨论,各界、各民主党派也可以讨论。这个农业发展纲要草案,是中国共产党提出的,是中共中央这个政治设计院设计出来的,不是章伯钧那个“政治设计院”设计出来的。

发动全体农民讨论这个农业发展纲要很有必要。要鼓起一股劲来。去年下半年今年上半年松了劲,加上城乡右派一间,劲就更不大了,现在整风反右又把这个劲鼓起来了。范说,农业发展纲要四十条是比较适合中国国情的,不是主观主义的。原来有些主观主义的东西,现在我们把它改掉了。后的说来,实现这个纲要是有希望的。我们中国可以改造,无知识可以改造得有知识,不振作可以改造得振作。

纲要里头有一个除四害,就是消灭老鼠、麻雀、苍蝇、蚊子。我对这件事很有兴趣,不晓得诸位如何?恐怕你们也是有兴趣的吧!除四害是一个大的清洁卫生运动,是一个破除迷信的运动。把这几样东西搞掉也是不容易的。除四害也要搞大鸣、大放、大辩论、大字报。如果动员全体人民来搞,搞出一点成绩来,我看人们的心理状态是会变的,我们中华民族的精神就会为之一振。我们要使我们这个民族振作起来。

计划生育也有希望做好。这件事也要经过大辩论,要几年试点,几年推广,几年普及。

我们要做的事情很多。农业发展纲要四十条里头就有很多事情要做。那仅是农业计划,还有工业计划和文教计划。三个五年计划完成以后,我们国家的面貌是会有个改变的。

我们预计,经过三个五年计划,钢的年产量可以搞到两千万吨。今年是五百二十万吨,再有十年大概就可以达到这个目标了。印度一九五二年钢产量是一百六十万吨,现在是一百七十几万吨,它搞了五年只增加十几万吨。我们呢?一九四九年只有十九万吨,三年恢复时期搞到一百多万吨,又搞了五年,达到五百二十万吨,五年就增加三百多万吨。再搞五年,就可以超过一千万吨,或者稍微多一点,达到一千一百五十万吨。然后,搞第三个五年计划,是不是可以达到两千万吨呢?是可能的。

我说我们这个国家是完全有希望的。右派说没有希望,那是完全错误的。他们没有信心,因为他们反对社会主义,那当然没有信心。我们坚持社会主义,我们是完全有信心的。


\begin{maonote}
\mnitem{1}三教指儒教、道教、佛教。九流指儒家、道家、阴阳家、法家、名家、墨家、纵横家、杂家和农家。“三教九流”后来是泛指宗教和学术的各种流派,旧社会中也用来泛称江湖上各种各样的人。
\end{maonote}
