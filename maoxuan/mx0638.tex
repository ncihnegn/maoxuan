
\title{永远不许一平二调\mnote{1}}
\date{一九六〇年十一月二十八日}
\thanks{这是毛泽东同志以中央名义起草的党内指示。}
\maketitle


\mxname{各中央局,各省、市、区党委:}

发去甘肃省委一九六〇年十一月二十五日报告\mnote{2}一件,很有参考价值,值得你们及地、县同志们认真研究一遍至两遍。甘肃省委在作自我批评了,看起来批评得还算切实、认真。看起来甘肃同志开始已经有了真正改正错误的决心了。毛泽东同志对这个报告看了两遍,他说还想看一遍,以便从其中吸取教训和经验。他自己说,他是同一切愿意改正错误的同志同命运、共呼吸的。他说,他自己也曾犯了错误,一定要改正。例如,错误之一,在北戴河决议中写上了公社所有制转变过程的时间,设想得过快了。在那个文件中有一段是他写的\mnote{3},那一段在原则上是正确的,规定由社会主义过渡到共产主义的原则和条件,是马列主义的。但是在那一段的开头几句规定过程的时间是太快了。那一段开头说:“由集体所有制向全民所有制过渡,是一个过程,有些地方可能较快,三、四年内就可以完成;有些地方可能较慢,需要五、六年或者更长的时间。”这种想法是不现实的。现在更正了,改为从现在起,至少(同志们注意,说的是至少)七年时间公社现行所有制不变,即使将来变的时候,也是队共社的产,而不是社共队的产。又规定从现在起至少二十年内社会主义制度(各尽所能,按劳付酬)坚决不变,二十年后是否能变,要看那时情况才能决定。所以说“至少”二十年不变。至于人民公社队为基础的三级所有制规定至少七年不变,也是这样。一九六七年以后是否能变,要看那时情况才能决定,也许再加七年,成为十四年后才能改变。总之,无论何时,队的产业永远归队所有或使用,永远不许一平二调。公共积累一定不能多,公共工程也一定不能过多。不是死规定几年改变农村面貌,而是依情况一步一步地改变农村面貌。甘肃省委这个报告,没有提到生活安排,也没有提到一、二、三类县、社、队的摸底和分析,这是缺点,这两个问题关系甚大,请大家注意。

\begin{maonote}
\mnitem{1}一平二调是人民公社化运动中“共产风”的主要表现,即:在公社范围内实行贫富拉平平均分配;县、社两级无偿调走生产队(包括社员个人)的某些财物。三收款,指银行将过去发放给农村的贷款统统收回。
\mnitem{2}指中共甘肃省委贯彻中央关于农村人民公社当前政策问题的紧急指示信给中央、西北局并发各地、市、州委和各县委的第四次报告。报告说,我省召开三级干部会议,深入检查了一再发生“共产风”的根源,认为应从省委领导工作中的缺点错误方面去寻找。主要存在三个方面的问题:第一关于执行中央政策问题,研究不够,领会不深,贯彻不力,甚至产生了一些偏差。例如,急于由基本队有制向基本社有制过渡;忽视小队小部分所有制和小队工作;对发展生产队的经济重视不够;收益分配政策定得不恰当等。第二,在指导人民公社发展生产和农村工作安排方面,没有把安排工作和贯彻政策结合起来,提出的任务大、要求急,对需要考虑多,对可能考虑少等等,因而出现一些违反政策的事。第三关于领导作风问题,对农业估产偏高,要求过高过急,作了一些不恰当的宣传,助长了“五风”的出现。
\mnitem{3}指毛泽东一九五八年八月在审阅《中共中央关于在农村建立人民公社问题的决议》稿时加写的一段文字:“人民公社建成以后,不要忙于改集体所有制为全民所有制,在目前还是以采用集体所有制为好,这可以避免在改变所有制的过程中发生不必要的麻烦。实际上,人民公社的集体所有制中,就已经包含有若干全民所有制的成分了。这种全民所有制,将在不断发展中继续增长,逐步地代替集体所有制。由集体所有制向全民所有制过渡,是一个过程,有些地方可能较快,三、四年内就可完成,有些地方,可能较慢,需要五、六年或者更长一些的时间。过渡到了全民所有制,如国营工业那样,它的性质还是社会主义的,各尽所能,按劳取酬。然后再经过多少年,社会产品极大地丰富了,全体人民的共产主义的思想觉悟和道德品质都极大地提高了,全民教育普及并且提高了,社会主义时期还不得不保存的旧社会遗留下来的工农差别、城乡差别、脑力劳动与体力劳动的差别,都逐步地消失了,反映这些差别的不平等的资产阶级法权的残余,也逐步地消失了,国家职能只是为了对付外部敌人的侵略,对内已经不起作用了,在这种时候,我国社会就将进入各尽所能,各取所需的共产主义时代。”
\end{maonote}
