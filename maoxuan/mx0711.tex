
\title{凡是镇压学生运动的人,都没有好下场}
\date{一九六六年七月十九月}
\thanks{这是毛泽东同志同中央几个负责同志的谈话。}
\maketitle


五月二十五日聂元梓大字报\mnote{1}是二十世纪六十年代中国的巴黎公社宣言书,意义超过巴黎公社。这种大字报我们写不出来的。

大字报写得好。

我向大家讲,青年是文化革命的大军,要把他们充分发动起来。

回到北京后,感到很难过,冷冷清清,有的学校大门都关起来了。甚至有些学校镇压学生运动。谁去镇压学生运动?只有北洋军阀。共产党怕学生运动是反马克思主义。有人天天说走群众路线,为人民服务,实际却是走资产阶级路线,为资产阶级服务。

团中央\mnote{2}应该站在学生运动这边,可是他们站在镇压学生运动那边。

谁反对文化大革命?美帝、苏修、日修,反动派。

借口“内外有别”\mnote{3}是怕革命。大字报贴出去又盖起来,这样的情况不允许,这是方向性错误,赶快扭转,把一切框框打得稀巴烂!

我们相信群众,做群众的学生,才能当群众的先生。现在这次文化大革命是个惊天动地的大事情。能不能,敢不敢过社会主义这一关?这一关是最后消灭阶级,缩短三大差别。

反对,特别是资产阶级“权威”思想,这就是破。如果没有这个破,社会主义的立,就立不起来;要做到一斗、二批、三改,也是不可能的。

坐办公室听汇报不行。只有依靠群众,相信群众,闹到底。准备革命革到自己头上来。党政领导、党员负责同志,应当有这个准备。现在要把革命闹到底,从这方面锻炼自己,改造自己,这样才能赶上。不然,就只有靠在外面。

有的同志斗别人很凶,斗自己不行,这样自己永远过不了关。

靠你们引火烧身,煽风点火,敢不敢?因为是烧到自己头上。同志们这样回答:准备好,不行就自己罢自己的官。生为共产党员,死为共产党员。坐沙发、吹风扇的生活不行。

给群众定框框不行。北京大学看到学生起来,定框框,美其名曰“纳入正轨”,其实是纳入邪轨。

有的学校给学生戴反革命帽子。这样就把群众放到对立面去了。不怕坏人,究竟坏人有多少?广大的学生大多数是好人。

(有人提出乱的时候,打乱档案怎么办?)怕什么?坏人来证明是坏人,好人你怕什么?要将“怕”字换成一个“敢”字。要最后证明社会主义关是不是过。

凡是镇压学生运动的人,都没有好下场!

\begin{maonote}
\mnitem{1}一九六六年五月二十五日下午二时许,北京大学哲学系聂元梓、宋一秀、夏剑豸、杨克明、赵正义、高云鹏、李醒尘七人,在大饭厅东墙上贴出了题为《宋硕、陆平、彭珮云在文化革命中究竟干些什么?》的大字报。全文如下:

现在全国人民正以对党对毛主席无限热爱、对反党反社会主义黑帮无限愤怒的高昂革命精神掀起轰轰烈烈的文化大革命,为彻底打垮反动黑帮的进攻,保卫党中央,保卫毛主席而斗争,可是北大按兵不动,冷冷清清,死气沉沉,广大师生的强烈革命要求被压制下来,这究竟是怎么回事?原因在那里?这里有鬼。请看最近的事实吧!

事情发生在五月八日发表了何明、高炬的文章,全国掀起了声讨“三家村”的斗争高潮之后,五月十四日陆平(北京大学校长、党委书记)急急忙忙的传达了宋硕(北京市委大学部副部长)在市委大学部紧急会议上的“指示”,宋硕说:现在运动“急切需要加强领导,要求学校党组织加强领导,坚守岗位。”“群众起来了要引导到正确的道路上去”,“这场意识形态的斗争,是一场严肃的阶级斗争,必须从理论上彻底驳倒反党反社会主义的言论。坚持讲道理,方法上怎样便于驳倒就怎样作,要领导好学习文件,开小组讨论会,写小字报,写批判文章,总之,这场严肃的斗争,要做的很细致,很深入,彻底打垮反党反社会主义的言论,从理论上驳倒他们,绝不是开大会所能解决的。”“如果群众激愤要求开大会,不要压制,要引导开小组会,学习文件,写小字报。”

陆平和彭珮云(北京市委大学部干部、北京大学党委副书记)完全用同一腔调布置北大的运动,他们说:“我校文化革命形势很好”,五月八日以前写了一百多篇文章,运动是健康的……运动深入了要积极引导。”“现在急切需要领导,引导运动向正确的方向发展”,“积极加强领导才能引向正常的发展”,“北大不宜贴大字报”,“大字报不去引导,群众要贴,要积极引导”等等。这是党中央和毛主席制定的文化革命路线吗?不是,绝对不是!这是十足的反对党中央、反对毛泽东思想的修正主义路线。

“这是一场意识形态的斗争”,“必须从理论上彻底驳倒反党反社会主义的言论”,“坚持讲道理”,“要作的细致”。这是什么意思?难道这是理论问题吗?仅仅是什么言论吗?你们要把我们反击反党反社会主义黑帮的你死我活的政治斗争,还要“引导”到那里去呢?邓拓和他的指使者对抗文化革命的一个主要手法,不就是把严重的政治斗争引导到“纯学术”讨论上去吗?你们为什么到现在还这么干?你们到底是些什么人?

“群众起来了,要引导到正确的道路上去”。“引导运动向正确的方向发展”。“要积极领导才能引向正常的发展”。什么是“正确的道路”?什么是“正确的方向”?什么是“正常的发展”?你们把伟大的政治上的阶级斗争“引导”到“纯理论”“纯学术”的圈套里去。不久前,你们不是亲自“指导”法律系同志查了一千五百卷书,一千四百万字的资料来研究一个海瑞“平冤狱”的问题,并大肆推广是什么“方向正确,方法对头”,要大家学习“好经验”吗?实际上这是你们和邓拓一伙黑帮一手制造的“好经验”,这也就是你们所谓“运动的发展是健康的”实质。党中央毛主席早已给我们指出的文化革命的正确道路、正确方向,你们闭口不谈,另搞一套所谓“正确的道路”,“正确的方向”,你们想把革命的群众运动纳入你们的修正主义轨道,老实告诉你们,这是妄想!

“从理论上驳倒他们,绝不是开大会能解决的”。“北大不宜贴大字报”,“要引导开小组会,写小字报”。你们为什么这样害怕大字报?害怕开声讨大会?反击向党向社会主义向毛泽东思想猖狂进攻的黑帮,这是一场你死我活的阶级斗争,革命人民必须充分发动起来,轰轰烈烈、义愤声讨,开大会,出大字报就是最好的一种群众战斗形式。你们“引导”群众不开大会,不出大字报,制造种种清规戒律,这不是压制群众革命,不准群众革命,反对群众革命吗?我们绝对不答应!

你们大喊,要“加强领导,坚守岗位”,这就暴露了你们的马脚。在革命群众轰轰烈烈起来响应党中央和毛主席的号召,坚决反击反党反社会主义黑帮的时候,你们大喊:“加强领导,坚守岗位”。你们坚守的是什么“岗位”,为谁坚守“岗位”,你们是些什么人,搞的什么鬼,不是很清楚吗?直到今天你们还要负隅顽抗,你们还想“坚守岗位”来破坏文化革命。告诉你们,螳臂挡不住车轮,蚍蜉撼不了大树。这是白日作梦!

一切革命的知识分子,是战斗的时候了!让我们团结起来,高举毛泽东思想的伟大红旗,团结在党中央和毛主席的周围,打破修正主义的种种控制和一切阴谋鬼计,坚决、彻底、干净、全部地消灭一切牛鬼蛇神、一切赫鲁晓夫式的反革命的修正主义分子,把社会主义革命进行到底。

保卫党中央!

保卫毛泽东思想!

保卫无产阶级专政!

哲学系:聂元梓、宋一秀、夏剑豸、杨克明、赵正义、高云鹏、李醒尘

一九六六年五月二十五日
\mnitem{2}一九六六年六月三日,刘少奇、邓小平主持召开中央政治局扩大会议,决定把北京市所有中学的“文化大革命”运动交给团中央领导,要团中央派出工作组,以便领导运动。团中央立即抽调了一千八百多名团干部,组成三百多个工作组,迅速派到了各所中学。胡耀邦兼团中央第一书记,胡克实任常务书记。胡克实参加了中央政治局扩大会议。
\mnitem{3}“内外有别”,一九六六年六月三日,刘少奇主持召开中央政治局常委扩大会议,议出八条指示:“八条指示的主要内容是:一,大字报要贴在校内;二,开会不要妨碍工作、教学;三,游行不要上街;四,内外区别对待,不准外国人参观,外国留学生不参加运动;五,不准到被揪斗的人家里闹;六,注意保密;七,不准打人、污蔑人;八,积极领导,坚持岗位。”
\end{maonote}
