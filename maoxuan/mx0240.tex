
\title{关于打退第二次反共高潮的总结}
\date{一九四一年五月八日}
\thanks{这是毛泽东为中共中央起草的对党内的指示。}
\maketitle


这一次的反共高潮,正如三月十八日中央的指示所说,是已经过去了。继之而来的是在国际国内的新环境中继续抗战的局面。在这个新环境中所增加的因素是:帝国主义战争的扩大,国际革命运动的高涨,苏日的中立条约\mnote{1},国民党第二次反共高潮的被击退以及由此产生的国民党政治地位的降低和共产党政治地位的提高,再加上最近日本准备新的对华大举进攻。为了团结全国人民坚持抗日,并继续有效地克服大地主大资产阶级的投降危险和反共逆流起见,研究和学习我党在英勇地胜利地反对这次反共高潮的斗争中所获得的教训,是完全必要的。

(一)在中国两大矛盾中间,中日民族间的矛盾依然是基本的,国内阶级间的矛盾依然处在从属的地位。一个民族敌人深入国土这一事实,起着决定一切的作用。只要中日矛盾继续尖锐地存在,即使大地主大资产阶级全部地叛变投降,也决不能造成一九二七年的形势,重演四一二事变\mnote{2}和马日事变\mnote{3}。上次反共高潮\mnote{4}曾被一部分同志估计为马日事变,这次反共高潮又被估计为四一二事变和马日事变,但是客观事实却证明了这种估计是不正确的。这些同志的错误,在于忘记了民族矛盾是基本矛盾这一点。

(二)在这种情况之下,指导着国民党政府全部政策的英美派大地主大资产阶级,依然是两面性的阶级,它一面和日本对立,一面又和共产党及其所代表的广大人民对立。而它的抗日和反共,又各有其两面性。在抗日方面,既和日本对立,又不积极地作战,不积极地反汪反汉奸,有时还向日本的和平使者勾勾搭搭。在反共方面,既要反共,甚至反到皖南事变和一月十七日的命令\mnote{5}那种地步,又不愿意最后破裂,依然是一打一拉的政策。这些事实,也在这次反共高潮中再度地证明了。极端地复杂的中国政治,要求我们的同志深刻地给以注意。英美派的大地主大资产阶级既然还在抗日,其对我党既然还在一打一拉,则我党的方针便是“即以其人之道,还治其人之身”\mnote{6},以打对打,以拉对拉,这就是革命的两面政策。只要大地主大资产阶级一天没有完全叛变,我们的这个政策总是不会改变的。

(三)和国民党的反共政策作战,需要一整套的战术,万万不可粗心大意。以蒋介石为代表的大地主大资产阶级对于人民革命力量的仇恨和残忍,不但为过去十年的反共战争所证明,更由抗日战争中的两次反共高潮特别是第二次反共高潮中的皖南事变所完全地证明了。任何的人民革命力量如果要避免为蒋介石所消灭,并迫使他承认这种力量的存在,除了对于他的反革命政策作针锋相对的斗争,便无他路可循。这次反共高潮中项英同志的机会主义\mnote{7}的失败,全党应该引为深戒。但是斗争必须是有理、有利、有节的,三者缺一,就要吃亏。

(四)在反对国民党顽固派的斗争中,将买办性的大资产阶级和没有或较少买办性的民族资产阶级加以区别,将最反动的大地主和开明绅士及一般地主加以区别,这是我党争取中间派和实行“三三制”政权的理论根据,这是去年三月以来中央就屡次指出了的。这次反共高潮再一次地证实了它的正确性。我们在皖南事变前所取《佳电》\mnote{8}的立场,对于事变后我们转入政治的反攻是完全必要的,非此即不能争取中间派。因为如果不经过反复多次的经验,中间派对于我党为什么必须向国民党顽固派进行坚决的斗争,为什么只能以斗争求团结,放弃斗争则没有任何的团结,就不能了解。地方实力派的领导成分虽然也是大地主大资产阶级,但是因为他们和统制中央政权的大地主大资产阶级分子有矛盾,故一般地亦须以中间派看待之。上次反共高潮中反共最力的阎锡山,这一次就站在中间立场;而上次居于中间立场的桂系,这一次虽然转到了反共方面,却和蒋系仍然有矛盾,不可视同一律。其它各地方实力派更不待论。我们同志中却有许多人至今还把各派地主阶级各派资产阶级混为一谈,似乎在皖南事变之后整个的地主阶级资产阶级都叛变了,这是把复杂的中国政治简单化。如果我们采取了这种看法,将一切地主资产阶级都看成和国民党顽固派一样,其结果将使我们自陷于孤立。须知中国社会是一个两头小中间大的社会\mnote{9},共产党如果不能争取中间阶级的群众,并按其情况使之各得其所,是不能解决中国问题的。

(五)有些同志由于对于中日矛盾是基本矛盾这一点发生动摇,并因此对国内阶级关系作了错误的估计,因而对党的政策也有时发生动摇。这些同志在皖南事变后,从其“四一二”和马日事变的估计出发,似乎感觉去年十二月二十五日的中央的原则指示\mnote{10},已不适用,或不大适用了。他们认为现在需要的已不是包含一切主张抗日和民主的人们的政权,而只是所谓工人、农民和城市小资产阶级的政权了;已不是抗日时期的统一战线的政策,而是像过去十年内战时期那样的土地革命的政策了。党的正确的政策,在这些同志的心目中,至少是暂时地模糊起来了。

(六)这些同志,当着我党中央令其准备对付国民党的可能的破裂,对付时局发展的最坏的一种可能性的时候,他们就把别的可能性丢掉了。他们不了解向着最坏的一种可能性作准备是完全必要的,但这不是抛弃好的可能性,而正是为着争取好的可能性并使之变为现实性的一个条件。这次我们充分地准备着对付国民党的破裂,就使国民党不敢轻于破裂了。

(七)还有更多的同志不了解民族斗争和阶级斗争的一致性,不了解统一战线政策和阶级政策,从而不了解统一战线教育和阶级教育的一致性。他们认为在皖南事变后需要特别强调所谓统一战线教育以外的阶级教育。他们至今还不明白:我党在整个抗日时期,对于国内各上层中层还在抗日的人们,不管是大地主大资产阶级和中间阶级,都只有一个完整的包括联合和斗争两方面的(两面性的)民族统一战线的政策。即使是伪军、汉奸和亲日派分子,除对绝对坚决不愿悔改者必须采取坚决的打倒政策外,对其余的分子也是这种两面性的政策。我党对党内对人民所施行的教育,也是包括这两方面性质的教育,就是教导无产阶级、农民阶级和其它小资产阶级如何和资产阶级地主阶级的各个不同的阶层在各种不同的形式上联合抗日,又和他们的各种不同程度的妥协性、动摇性、反共性作各种不同程度的斗争。统一战线政策就是阶级政策,二者不可分割,这一点不弄清楚,很多问题是弄不清楚的。

(八)还有一些同志,不了解陕甘宁边区和华北华中各抗日根据地的社会性质已经是新民主主义的。判断一个地方的社会性质是不是新民主主义的,主要地是以那里的政权是否有人民大众的代表参加以及是否有共产党的领导为原则。因此,共产党领导的统一战线政权,便是新民主主义社会的主要标志。有些人以为只有实行十年内战时期那样的土地革命才算实现了新民主主义,这是不对的。现在各根据地的政治,是一切赞成抗日和民主的人民的统一战线的政治,其经济是基本上排除了半殖民地因素和半封建因素的经济,其文化是人民大众反帝反封建的文化。因此,无论就政治、经济或文化来看,只实行减租减息的各抗日根据地,和实行了彻底的土地革命的陕甘宁边区,同样是新民主主义的社会。各根据地的模型推广到全国,那时全国就成了新民主主义的共和国。


\begin{maonote}
\mnitem{1}指一九四一年四月十三日苏联和日本在莫斯科签订的中立条约。
\mnitem{2}四一二事变,是蒋介石于一九二七年四月十二日在上海发动的反革命事变。在这次事变中,蒋介石残酷地屠杀了大批共产党人和革命群众。从此,蒋介石和他的追随者就完全从革命统一战线中分裂出去,随后发动了历时十年之久的内战。
\mnitem{3}见本书第一卷\mxnote{井冈山的斗争}{20}。
\mnitem{4}指一九三九年冬至一九四〇年春蒋介石发动的第一次反共高潮。参见本卷\mxnote{向国民党的十点要求}{5}。
\mnitem{5}见本卷\mxnote{打退第二次反共高潮后的时局}{2}。
\mnitem{6}这是宋朝的著名学者、理学家朱熹(一一三〇——一二〇〇)在《中庸》第十三章注文中所说的话。
\mnitem{7}参见本卷\mxthanks{放手发展抗日力量,抵抗反共顽固派的进攻}{一文的}。
\mnitem{8}《佳电》是中共中央以第十八集团军总司令朱德、副总司令彭德怀和新四军军长叶挺、副军长项英的名义,于一九四〇年十一月九日答复何应钦、白崇禧《皓电》的电报。这个电报,揭发了国民党顽固派的反共投降阴谋,驳斥了何、白强迫黄河以南八路军、新四军在一个月内撤到黄河以北的荒谬命令;同时,为了照顾团结抗日的大局,委曲求全,同意将江南新四军部队移至长江以北,并且进一步要求解决国共间的若干重要悬案。这个电报,曾经取得当时中间派的同情,孤立了蒋介石。
\mnitem{9}毛泽东这个说法,是指领导革命的中国工业无产阶级和反动的中国大地主大资产阶级在中国社会总人口中都只占少数,农民、城市小资产阶级和其它中间阶级在中国社会总人口中占了绝大多数。参见本书第三卷\mxart{在陕甘宁边区参议会的演说}。
\mnitem{10}见本卷\mxart{论政策}。
\end{maonote}
