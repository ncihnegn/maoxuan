
\title{蒋介石在挑动内战}
\date{一九四五年八月十三日}
\thanks{这是毛泽东为新华社写的评论。}
\maketitle


国民党中央宣传部发言人发表谈话说,第十八集团军朱德总司令于八月十日在延安总部所发表的限令敌伪投降的命令\mnote{1},是一种“唐突和非法之行动”。这种评论,荒谬绝伦。根据这种意见,可以逻辑地解释为朱德总司令根据波茨坦公告\mnote{2}和敌人投降的意向,下令给所属部队促使敌伪投降,倒反错了,应该劝使敌伪拒绝投降,才是对的,才算合法。无怪中国法西斯头子独夫民贼蒋介石,在敌人尚未真正接受投降之前,敢于“命令”解放区抗日军队“应就原地驻防待命”,束手让敌人来打。无怪这同一个法西斯头子,又敢于“命令”所谓地下军(实际上就是实行“曲线救国”\mnote{3}的伪军和与敌伪合流的戴笠系特务\mnote{4})和伪军,“负责维持地方治安”,而不许解放区抗日军队向敌伪“擅自行动”。这样的敌我倒置,真是由蒋介石自己招供,活画出他一贯勾结敌伪、消除异己的全部心理了。可是中国解放区的人民抗日军队,绝不会中此毒计。他们知道:朱德总司令的命令,正是坚决地执行波茨坦公告第二项的规定:“对日作战,直至其停止抵抗为止。”而蒋介石的所谓“命令”,正是违反了他自己签字的波茨坦公告。只要拿这一比,就知道谁是不“恪守盟邦共同协议之规定”了。

国民党中央宣传部发言人的评论和蒋介石的“命令”,从头到尾都是在挑拨内战,其目的是在当着国内外集中注意力于日本无条件投降之际,找一个借口,好在抗战结束时,马上转入内战。其实,国民党反动派是蠢得可怜的。他们找了朱德总司令命令敌伪投降缴械当作借口。这难道也算得一个聪明的借口吗?不,这样来找借口,只足以证明国民党反动派把敌伪看得比同胞还可亲些,把同胞看得比敌伪还可恨些。淳化事件\mnote{5},明明是胡宗南军队攻入陕甘宁边区,挑拨内战,国民党反动派却说是中共的“谣言攻势”。淳化事件这个借口,好容易被国民党反动派找着了,却被中外舆论界一下子识破,于是又说八路军、新四军不该要敌伪缴枪了。八年抗战,八路军、新四军受尽了蒋介石和日本人夹击围攻的苦楚,现在抗战瞬将结束,蒋介石又在暗示日本人(加上他亲爱的伪军),叫他们不要向八路军、新四军缴枪,说是只能缴给我蒋介石。蒋介石剩下一句话没有说,这一句就是:好使我拿了这些枪杀共产党,并破坏中国和世界的和平。不是吗?叫日本人缴枪给蒋介石,叫伪军“负责维持地方治安”,这会有什么结果呢?只有一个结果,就是以宁渝合流\mnote{6}、蒋伪合作,去代替“中日提携”、日伪合作;以蒋介石的反共建国,去代替日本人、汪精卫\mnote{7}的反共建国。这难道还不是违背波茨坦公告吗?抗战一旦结束,内战危险立即严重威胁全国人民,这一点难道还有疑义吗?现在我们向全国同胞和世界盟邦呼吁,一致起来,同解放区人民一道,坚决制止这个危及世界和平的中国内战。

究竟谁有权接受日伪的投降呢?中国解放区的抗日军队,在国民党政府毫无接济又不承认的条件下,完全靠自己的努力和人民的拥护,得以独力解放了广大的国土和一万万以上的人民,抗击着侵华敌军百分之五十六和伪军的百分之九十五。要是没有这一个军队,中国绝无今天的局面!实在说,在中国境内,只有解放区抗日军队才有接受敌伪军投降的权利。至于蒋介石,他的政策是袖手旁观,坐待胜利,实在没有丝毫权利接受敌伪投降。

我们要向全国同胞和全世界人民宣布:重庆统帅部,不能代表中国人民和中国真正抗日的军队;中国人民要求,中国解放区抗日军队有在朱德总司令指挥之下,直接派遣他的代表参加四大盟国接受日本投降和军事管制日本的权利,并且有参加将来和会的权利。要不是这样做,中国人民将认为是很不恰当的。


\begin{maonote}
\mnitem{1}一九四五年八月十日,延安总部朱德总司令为日本投降事向各解放区所有武装部队发布命令,全文如下:“日本已宣布无条件投降,同盟国在波茨坦宣言基础上将会商受降办法。因此,我特向各解放区所有武装部队发布下列命令:一、各解放区任何抗日武装部队均得依据波茨坦宣言规定,向其附近各城镇交通要道之敌人军队及其指挥机关送出通牒,限其于一定时间向我作战部队缴出全部武装,在缴械后,我军当依优待俘虏条例给以生命安全之保护。二、各解放区任何抗日武装部队均得向其附近之一切伪军伪政权送出通牒,限其于敌寇投降签字前,率队反正,听候编遣,过期即须全部缴出武装。三、各解放区所有抗日武装部队,如遇敌伪武装部队拒绝投降缴械,即应予以坚决消灭。四、我军对任何敌伪所占城镇交通要道,都有全权派兵接收,进入占领,实行军事管制,维持秩序,并委任专员负责管理该地区之一切行政事宜,如有任何破坏或反抗事件发生,均须以汉奸论罪。”接着,在八月十一日,延安总部又连续发布了六道命令,命令晋绥解放区贺龙领导的武装部队、晋察冀解放区聂荣臻领导的武装部队、冀热辽解放区的武装部队向内蒙和东北进军;命令山西解放区的武装部队肃清同蒲路沿线和汾河流域的日伪军;命令各解放区的武装部队,向一切敌占交通要道展开积极进攻,迫使日伪军投降。各解放区的解放军坚决地执行了这些命令,并取得了重大的胜利。
\mnitem{2}指一九四五年七月二十六日中、美、英三国在波茨坦会议过程中发表的促令日本投降的公告。其主要内容为:盟军对日作战,直至其停止抵抗为止;日本政府应立即宣布无条件投降;日本军国主义必须永远肃清;日本军队必须完全解除武装;日本的军事工业必须拆除;日本的战犯必须交付审判;开罗宣言必须实施,即日本必须放弃前所掠取的土地,如朝鲜,中国的满洲、台湾、澎湖列岛等地,日本的领土限于本州岛、北海道、九州岛、四国和其它小岛之内;同盟国军队占领日本直到日本民主政府建立以后为止。八月八日,苏联对日宣战后,亦签字于该公告。
\mnitem{3}见本书第三卷\mxnote{论联合政府}{9}。
\mnitem{4}戴笠系特务,是以国民政府军事委员会调查统计局(简称“军统”)局长戴笠为首的特务系统。“军统”成立于一九三八年八月,其前身是一九三二年四月成立的复兴社的核心组织力行社的特务处。“军统”是国民党庞大的特务组织之一,它残酷地迫害共产党人和革命人士,破坏人民革命,同时也对国民党军队、警察及行政机关等进行监视和控制。
\mnitem{5}淳化事件,即一九四五年七月国民党军队进犯陕甘宁边区关中分区淳化县爷台山等地的事件。见本卷\mxnote{抗日战争胜利后的时局和我们的方针}{6}。
\mnitem{6}“宁”指当时南京汉奸集团的伪政权,“渝”指当时重庆的蒋介石政权。“宁渝合流”,是当时日本帝国主义与南京方面以及重庆蒋介石方面共同策动的一个政治阴谋。
\mnitem{7}见本书第一卷\mxnote{论反对日本帝国主义的策略}{31}。
\end{maonote}
