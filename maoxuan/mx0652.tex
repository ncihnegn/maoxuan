
\title{革命和建设都要靠自己}
\date{一九六三年九月三日}
\thanks{这是毛泽东同志同由中央委员会主席迪·努·艾地率领的印度尼西亚共产党代表团的谈话的一部分。}
\maketitle


我们困难的时间只有两年半,就是一九六〇年、一九六一年和一九六二年上半年,一九六二年下半年情况就好起来了。粮食去年比前年增产一千多万吨。今年情况更好一点。虽然今年华北地区特别是河北和河南有水灾,但是全国可能比去年增产粮食一千万吨左右。现在我们正集中力量搞棉花、油料、烟叶和糖料。我们已经找到一条道路。我们有两种经验,错误的经验和正确的经验。正确的经验鼓励了我们,错误的经验教训了我们。苏联把专家撤走,撕毁了合同,这对我们有好处。我们没办法,就靠自己,靠自己两只手。后来苏联又后悔了,想再派专家来,要同我们做生意,我们不干。他们再派专家来,有朝一日他们又要撤走专家,撕毁合同。他们已经失去了我们的信任。正是在一九六〇年的这个时候,苏联撤走专家,到现在已经三年了,我们的工业建设搞出了许多自己的经验。离开了先生,学生就自己学。有先生有好处,也有坏处。不要先生,自己读书,自己写字,自己想问题。这是一条真理。过去我们就是由先生把着手学写字,从一九二一年党成立到一九三四年,我们就是吃了先生的亏,纲领由先生起草,中央全会的决议也由先生起草,特别是一九三四年,使我们遭到了很大的损失。从那之后,我们就懂得要自己想问题。我们认识中国,花了几十年时间。中国人不懂中国情况,这怎么行?真正懂得独立自主是从遵义会议开始的,这次会议批判了教条主义。教条主义者说苏联一切都对,不把苏联的经验同中国的实际相结合。马列主义普遍真理与中国具体实践相结合,这个口号就是在延安整风时提出的。这个口号写进了一九五七年莫斯科宣言,那里面说马列主义普遍真理要与各国的具体实践相结合。外国经验,不管是哪一个国家的,只能供参考。
