
\title{在中共八届十一中全会闭幕会上的讲话}
\date{一九六六年八月十二日}
\maketitle


关于第九次大会的问题,恐怕要准备一下。第九次大会什么时候召集的问题,要准备一下。已经多年了,八大二次会议到后年就十年了。现在要开九次大会大概是在明年一个适当的时候再开,现在要准备,建议委托中央政治局同它的常委会来筹备这件事,好不好?

至于这次全会所决定的问题,究竟是正确的还是不正确的,要看以后的实践。我们所决定的那些东西,看来群众是欢迎的。比如中央主要的一个决定就是关于文化大革命,广大的学生和革命教师是支持我们的,而过去那些方针,广大的革命学生跟革命教师是抵抗的,我们是根据这些抵抗来制定这个决定的。但是,究竟这个决定能不能实行,还是要靠我们在座的与不在座的各级领导去做。比如讲依靠群众吧,群众路线,还是有两种可能性:一种是依靠,一种是不依靠;一种是实行群众路线,一种是不实行群众路线。决不要以为,决定上写了,所有的党委,所有的同志就都会实行,总有一小部分人不愿意实行。可能比过去好一些,因为过去没有这样公开的决定,并且这次有组织的保证。这回组织有些改变,政治局委员、政治局候补委员、书记处书记、常委的调整,就保证了中央这个决议以及公报的实行。

对犯错误的同志总是要给他出路,要准许改正错误。不要认为别人犯了错误,就不许他改正错误。我们的政策是惩前毖后,治病救人,一看二帮,团结——批评——团结。我们这个党不是党外无党,我看是党外有党,党内也有派,从来都是如此,这是正常现象。我们过去批评国民党,国民党说党外无党,党内无派,有人就说,“党外无党,帝王思想。党内无派,千奇百怪”。我们共产党也是这样,你说党内无派?它就是有,比如说对群众运动就有两派,不过是占多占少的问题。如果不开这次全会,再搞几个月,我看事情就要坏得多。所以,我看这次会是开得好的,是有结果的。
