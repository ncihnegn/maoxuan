
\title{同澳共总书记夏基的谈话}
\date{一九五九年十月二十六日}
\maketitle


你写的材料提出了如何过渡到社会主义的问题,说是同意中国同志对这个问题的看法。这是重大问题。这里有两个问题,一个是战略问题,一个是策略问题。作为战略问题来说,从长远看,用和平手段能够消灭资产阶级政权是不可想像的。资产阶级怎么能够让工人阶级用和平手段来推翻资产阶级政权,消灭阶级,建设社会主义和共产主义呢!从策略上讲,首先可以说无产阶级愿意用和平手段取得政权,表明我们不是好战的。但是如果资产阶级使用暴力,无产阶级就被迫不得不使用暴力。不要散布幻想,不要在精神上解除自己的武装。不作精神准备,就无法教育人民,无产阶级自己也就没有革命干劲。用和平手段也是要斗争的。其实,革命的大量日常工作都是通过和平手段进行的。但作为革命家,在用和平手段进行日常工作的同时,要想到革命时机到来时怎么办?这个问题,不要每天去讲。在重要时机才提这个问题,提两条,一定要有两条:第一,无产阶级愿意用和平手段取得政权;第二,假使资产阶级使用暴力,无产阶级被迫也得使用暴力。

马克思主义者知道,阶级斗争不经过战争是不能最后解决问题的。自古以来都是这样。明知如此,为何又要提和平手段?因为广大人民群众还不觉悟,资产阶级就利用这一点恐吓人民群众,说共产党专讲暴力和战争。

列宁在一九一七年二月以后的一段时间内,曾经指出俄国存在用和平手段夺取政权的可能性。当时俄国有两个政权并存,列宁根据那时特有的一些条件,设想布尔什维克\mnote{1}通过在苏维埃中取得多数来夺取政权。后来到了七月就不行了,资产阶级政府用武力镇压群众的革命运动,于是列宁和布尔什维克决定准备武装起义的方针,用暴力夺取政权。

在十月革命以前,列宁对全党讲得很清楚,写了许多文章,说明革命要建立无产阶级专政,不用暴力是不行的。他极力反对孟什维克\mnote{2}主张通过议会斗争进行革命的论调。

我们在一九五五年和英共波立特\mnote{3}谈过这个问题,他不赞成我们的意见。他要修改中国同志的著作,要在我的选集的英文版中删掉两段讲革命一定要用暴力的话。\mnote{4}我们不赞成他的这个意见,但结果他还是把那两段挖掉了。

一九五七年十一月莫斯科会议宣言\mnote{5}又提到这个问题,也是提两条:无产阶级愿意用和平手段取得政权;但是如果资产阶级使用暴力时,无产阶级被迫也要使用暴力。这是一个原则问题。法共、意共对这个问题都提两条,日共、印尼共也提两条,大多数党提两条。讲清楚两条,可使资产阶级被动。我们并不是提倡武力,我们只是说你使用武力我们才使用武力。

但是在从理论上讲问题时,就要把这个问题讲清楚,讲彻底。国家是暴力机关,无论奴隶制度的国家、封建制度的国家或资本主义制度的国家都是暴力机关。无产阶级专政的社会主义国家,也不例外。离开了暴力还叫什么国家。也有人拿中国和平改造资本主义工商业的政策作为和平过渡的例子,其实我们是经过了几十年的战争,打倒了国民党政府,建立了强大的无产阶级领导的人民政权,发展到几百万军队,这才有了和平改造的可能。

革命用战争手段和用和平手段也是两条腿走路。实际上大量工作是用和平手段通过日常工作进行的,战争时间并不长,但最后解决问题还是要靠战争。不用两条腿走路,就不能夺取政权。

我们在一九四五年也努力争取过国内的和平,并参加了国民党召开的政治协商会议。同时我们准备了另一面,发展了武装力量,有了一年的准备。一九四六年夏打起来以后,我们也不说绝对不要和平。一九四九年春天国民党曾提出要“和平”。那时他们只有长江以南,他们在长江以北的主力已经被消灭。美国劝他们谈判,以保住江南,准备力量再来打我们。

我们说,你们派代表团到北京来谈。代表团派来了,达成了协议,然后把协议送往南京签字,但他们不肯签。这样很好,拒绝和平的责任就完全在他们身上了。他们第一天拒绝,我们第二天就过长江,一百万大军一夜渡过长江,到九月全国大陆基本解放,十月一日成立了中华人民共和国。

蒋介石跑到台湾去了。我们也还不放弃和平手段,我们提出了和平解放台湾的口号,主张同他们正式谈判。但他们害怕谈判,美国更是害怕蒋介石同我们谈判。在这种情况下,和平口号就很有用了。

现在在一些资本主义国家里,斗争通常是不流血的,但是当事物要发生质变时,就要流血了。经过长期的量变就要发展到质变。要是没有这种质变,旧的上层建筑是不会改变的。上层建筑主要指政权和军队、警察、法院等国家机器,也包括意识形态方面的东西。上层建筑是保护经济基础的。所以首先要用暴力把国家机器这些主要的上层建筑夺取过来,加以粉碎。至于意识形态方面的上层建筑,不能用武力解决,而是要经过长期的改造。

上层建筑是建立在经济基础上的,是用来保护经济基础的,当经济基础失掉保护的时候,我们就可以改变生产关系,即旧的所有制等。当然这也要经过斗争,但不一定要经过战争。

下面谈谈对国际形势的看法。

国际紧张局势是帝国主义制造的,但走向了它的反面。紧张局势的一部分或大部分使他们觉得对他们不利了,不利于他们保存资本主义和消灭社会主义的目的了。杜勒斯\mnote{6}的那一套对他们所要达到的这个目的是不利的,他们想走出这条很窄的路。如果紧张局势有利于他们达到保存资本主义和消灭社会主义的目的,就不能想像他们会有所改变。看来他们了解到这种不利,要有些改变,而且他们害怕战争。大家知道两次世界大战对他们都不利,第三次世界大战对他们将更加不利。像美国这样的国家,战争打起来对它是很不利的。

缓和对社会主义国家和资本主义国家的人民都有利,这是社会主义国家、各国兄弟党以及世界和平力量斗争的结果。再有十年和平是很好的,中国和苏联能再搞几个五年计划那好得很。

但是还要看到另一面,帝国主义为了维持军火工业和夺取外国的利益,需要一定程度的紧张局势。例如在赫鲁晓夫访美\mnote{7}以后,美国就在一些国家建立了火箭基地,又在联合国大叫大嚷西藏问题,可见他们还要制造紧张局势。所以我们要警惕。

\begin{maonote}
\mnitem{1}布尔什维克,指布尔什维克党,原苏联共产党的前身,一九一二年正式建立。
\mnitem{2}孟什维克,是一九〇三年在俄国社会民主工党内形成的与布尔什维克对立的机会主义派别。
\mnitem{3}波立特(一八九〇——一九六〇),时任英国共产党中央执行委员会主席。
\mnitem{4}一九五四年三月二十九日,波立特给中共中央来信,提出他们在翻译《毛泽东选集》第二卷时,准备将《战争和战略问题》一文的头两段从英译本中删去。在这两段中,毛泽东指出:“革命的中心任务和最高形式是武装夺取政权,是战争解决问题。这个马克思列宁主义的革命原则是普遍地对的,不论在中国在外国,一概都是对的。”“但是在同一个原则下,就无产阶级政党在各种条件下执行这个原则的表现说来,则基于条件的不同而不一致。”同年八月二十三日,中共中央复信波立特,明确表示:我们不能同意在《毛泽东选集》英译本中把《战争和战略问题》的头两段删去的提议,“因为毛泽东同志在该文件中所说到的原则,是马列主义的普遍真理,并不因为国际形势的变化,而须要作什么修正。而且《毛泽东选集》已经出版俄文版及其他外国文版,都没有作什么修改”。
\mnitem{5}指一九五七年十一月十四日至十六日在莫斯科召开的社会主义国家共产党和工人党代表会议通过的《社会主义国家共产党和工人党代表会议宣言》(通称《莫斯科宣言》)。宣言总结了国际共产主义运动的经验,提出了各国共产党争取和平和社会主义的斗争任务,规定了社会主义国家和政党之间关系的准则,并要求各国共产党创造性地运用马克思列宁主义。
\mnitem{6}杜勒斯(一八八八——一九五九),美国共和党人,一九五三年一月至一九五九年四月任美国国务卿。一九五六年一月,杜勒斯提出美国“不怕走到战争边缘,但要学会走到战争边缘,又不卷入战争的必要艺术”。这种主张被称为“战争边缘政策”。
\mnitem{7}一九五九年九月十五日至二十七日,苏联共产党中央委员会第一书记、苏联部长会议主席赫鲁晓夫访问美国,在美国总统别墅戴维营同艾森豪威尔就德国问题、柏林问题、裁军、核试验、双边关系等举行了会谈。
\end{maonote}
