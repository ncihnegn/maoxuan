
\title{同斯诺\mnote{1}的谈话——关于越南战争}
\date{一九六五年一月九日}
\thanks{这是毛泽东同志同美国进步作家、记者埃德加·斯诺谈话的主要部分。}
\maketitle


\mxsay{斯诺:}我来这里见主席之前曾看了主席的军事著作,联系南越军事专家的说法,是不是可以说,现在南越战争\mnote{2}已经进入到运动战阶段,像中国的第三次国内革命战争那样?

\mxsay{毛泽东:}我们第三次国内革命战争是全国解放战争,从一九四六年起,那时蒋介石有几百万军队,我们的军队也有一百多万。从规模上说,现在南越战争还没有那么大。你可以劝劝美国政府,何必这么搞?它一到哪个地方,哪个地方的人就学会打仗。但是,叫它走,它又不走。就说吴庭艳\mnote{3}吧,我和胡志明\mnote{4}都认为他还不错,应该帮他维持几年。可是美国将军认为吴庭艳很讨厌,把他干掉,这样天下就太平了?

\mxsay{斯:}当然现在南越解放军的人力不如中国的八路军和后来的解放军,但是西贡\mnote{5}也没有蒋介石那么多的军队。

\mxsay{毛:}没有那么多,也不懂得打仗,比蒋介石更差。

\mxsay{斯:}是不是可以认为,南越有足够的力量能单独克服外来的干涉和反对当地反动派?

\mxsay{毛:}我看是可以的,至少比我们第二次国内革命战争时候的条件好。我们第二次国内革命战争时没有外国直接干涉。现在南越的好处就是有两万名美国人在那里,这就能教育大部分的人民和军队里的战士以及部分军官。现在反对美国军队的那些人,并不全是解放军,吴庭艳也不赞成美国,现政府军队中也有些人不赞成美国。

\mxsay{斯:}很明显。

\mxsay{毛:}他们吵得厉害。

\mxsay{斯:}有可能使南越军队的一部分人参加越共?

\mxsay{毛:}是的,像傅作义\mnote{6}一样,像新疆的陶峙岳\mnote{7}和湖南的程潜、陈明仁\mnote{8}一样。

\mxsay{斯:}从我前次来中国后,国际形势有了很大变化,非洲觉醒了。在这种情况下,是不是可以说,当前的主要矛盾是帝国主义和亚非拉的新兴力量之间的矛盾?这一矛盾是否比帝国主义之间的矛盾更为重要?

\mxsay{毛:}你说呢?我也不大清楚,我又没有到处走。你到处走,是怎么想的?我想请你当教员,把国际情况讲一讲。

\mxsay{斯:}我相信主席可以回答,我无法回答,或者只好等主席的下一本书了。从主席的著作中可以看到,主席特别重视的那些事件,从这里是否可以认为帝国主义和亚非拉的新兴力量的矛盾是主要矛盾?

\mxsay{毛:}我看美国总统也是这么说的。前任总统多次说,美国、加拿大、西欧问题比较少,问题严重的是地球的南半部。肯尼迪\mnote{9}多次讲话都讲这个问题。他提出特种战争、局部战争,就是对付这个的。有消息说,他也看我写的军事文章,这可能是真的。当阿尔及利亚问题不得解决的时候,阿尔及利亚人问我,我的著作他们利用,法国人也利用,怎么办?说这话的是当时的总理阿巴斯,他访问过中国。我说,怎么利用?我根据中国的经验写的书,只能适用于人民的战争,不适用于反人民的战争。蒋介石也研究我们的材料,我们的许多材料在战争中被蒋介石得到,但是无法挽救其失败。法国人也没有因为看我的书而挽救其失败。现在我们也研究美国的军事著作。美国驻南越大使、前参谋长联席会议主席泰勒写了一本书,叫《不定音的号角》。看他那本书的意思,他是不大赞成核武器的。他说,在朝鲜战争中没用过,在中国解放战争中没用过,他怀疑以后的战争能够用这种东西制胜。他要争陆军的人数和用费,但是同时说也要造核武器,二者平行发展。他说陆军是需要的,要保持八十至九十万人。陆军要坚持陆军的人数,空军要多搞飞机、多搞核武器,海军有海军的主张。他代表陆军,要争取陆军的优先权。现在他又得到在南越实验的机会。他是去年六月去南越的,还不到一年,不如在朝鲜呆得久。他会取得经验的。我已经看到美国军队怎么对付南越游击战争的条例,无非是讲南越游击战争的许多长处和许多缺点,然后说消灭游击队是有希望的。

\mxsay{斯:}美国人的弱点不是军事上的,而是政治上的。

\mxsay{毛:}可能是。就是南越政府不得人心,无论吴庭艳政府也好,当今政府也好,都是脱离群众的。帮助这种不得人心的政府是不会有好结果的。不仅我讲的话他们不听,你讲的话他们照样也不听。

\mxsay{斯:}过去没有听我的话,所以我认为他们才有那么多的失败。现在可以看到亚非拉同发达国家在军事、经济方面的距离越来越大,同时,新殖民主义的所作所为使距离更大了。这是不是主要矛盾?所以法国的政策有改变,是不是不仅为了抵抗美国,还为了适应这个主要矛盾?

\mxsay{毛:}我也同法国人讲过。在同法国议员代表团谈话时,我问他们第三世界究竟包不包括法国,他们说不包括。现在发达国家为一方,不发达国家为一方。所谓发达国家就不那么一致,而且从来没有一致过。例如发达的英、法、德、意、日之间,就发生两次世界大战,这还不是发达国家和发达国家打吗?其目的是争所谓不发达的国家。它们为什么要打,是睡不着觉、吃不下饭吗?你没有参加这个战争,你们前任总统参加了第二次世界大战,现任总统也参加了。

\mxsay{斯:}那时我在俄国当战地记者。

\mxsay{毛:}呆了多久?

\mxsay{斯:}两年半。战争结束后,在英国、法国呆过。我从未打死过人,我倒有几次几乎被打死。

\mxsay{毛:}那么危险!也跑到前线去?

\mxsay{斯:}战地记者是战争的附属品。在俄国我没到过前线,在德国、法国到前线去过,打斯大林格勒时去过斯大林格勒。

\mxsay{毛:}打前还是打后?

\mxsay{斯:}希特勒\mnote{10}军队投降时去的。

\mxsay{毛:}那时希特勒可了不起,差不多占领了整个欧洲。除了莫斯科、列宁格勒、斯大林格勒一线以东之外,整个欧洲几乎都被他占领了,但是英国不包括在内。他还占领了北非洲。希特勒犯了错误,如果敦刻尔克\mnote{11}之后他的军队跟着进去,英国也毫无办法。这是一位英国首相\mnote{12}在日内瓦告诉我们周恩来总理的。那时英国根本没有兵了,到处没有设防,可是就是因为有英吉利海峡,德国不敢过去。

\mxsay{斯:}希特勒急于打俄国。关于中美关系改善有没有希望?

\mxsay{毛:}我看有希望,不过需要时间。也许我这辈子没有希望了,我快见上帝了,也许你们有希望。按照辩证法,生命总是有限的。

\mxsay{斯:}我看主席的身体很健康。

\mxsay{毛:}我准备了好多次了,就是不死,有什么办法!多少次好像快死了,包括你说的战争中的危险,把我身边的卫士炸死,血溅到我身上,可是炸弹就是没打到我。

\mxsay{斯:}在延安时?

\mxsay{毛:}好多次。在长征路上也有一次,过了大渡河,遇上飞机轰炸,把我的卫士长炸死,这次血倒没有溅到我身上。

过去我当过小学教员,你是知道的,不仅没有想到打仗,也没有想到搞共产党,同你差不多,是个民主人士。后来就不知道什么原因搞起共产党来了。总之,这不以我们这些人的意志为转移。中国受帝国主义、封建主义和官僚资本主义的压迫,开始还有军阀的压迫,这是事实。

\mxsay{斯:}客观条件使革命变成不可避免的,现在没有这种条件了。现在中国条件不同了,下一代将怎样?

\mxsay{毛:}我也不知道,那是下一代的事。谁知道下一代干些什么事,无非是几种可能:一是把革命继续发展;一是否定革命,干坏事,跟帝国主义讲和,把蒋介石接到大陆上来,同国内百分之几的反革命站在一起,这就叫反革命。你问我的意见,我当然不希望出现反革命。将来的事由将来的人决定。从长远来看,将来的人要比我们聪明,如同资本主义时代的人比封建时代的人要聪明、要好一样。美国没有封建主义,可是欧洲是有的。

\mxsay{斯:}美国不能说完全没有封建主义,南北战争\mnote{13}的原因之一就是反封建。

\mxsay{毛:}就是为了争劳动力,所谓解放黑奴就是开放劳动力市场。

\mxsay{斯:}美国南方封建主统治时间虽然不长,但封建主义思想影响是很深的。

\mxsay{毛:}现在南方还比较落后一些。

\mxsay{斯:}主席还是认为原子弹是纸老虎吗?

\mxsay{毛:}我不过讲讲而已,真打起来会死人的。但是最后它是要被消灭的,那时就变成纸老虎了,它没有了嘛!

\mxsay{斯:}主席一定会听到有人说,主席认为如果发生一场核战争,其他一些国家的人全部消灭,中国还留下几亿人口。

\mxsay{毛:}你说怎么样?

\mxsay{斯:}实际上已由主席间接回答了,在中国同苏联的论战文章里也提到了。

\mxsay{毛:}怎么回答的,我也忘了。

\mxsay{斯:}我恐怕也忘了,不过我记得有一篇文章里说到,这是撒谎,是硬把这些话加到主席身上的。

\mxsay{毛:}怎么说的?

\mxsay{斯:}说如果发生核战争,中国还会留下几亿人口。这是一位南斯拉夫人在五十年代后半期访问中国时,主席向他说的。

\mxsay{毛:}我记不起来了,可能我说过。我记得说的是,我们是不要打仗的,我们没有原子弹,如果别的国家要打,全世界可能遭殃。遭殃无非是要死人,死多少,谁也不知道,总要死一些。我不只是说中国。我就不相信原子弹能把全人类毁灭,什么都毁灭了,找不到任何政府谈和了。这是同尼赫鲁\mnote{14}在北京的一次谈话里谈到的。他说他是印度原子能委员会主任,他知道原子武器的厉害。我说可能不是如他所说,核战争后会找不到任何政府谈和。这个政府倒下去了,那个政府又会起来,总会有人起来的。我没有说过全世界都会毁灭。听说你们有个电影叫《在海滩上》。

\mxsay{斯:}那是假想小说,说全人类毁灭了。

\mxsay{毛:}这怎么得了啊!赫鲁晓夫\mnote{15}说,他手里有一种弹,一种什么死光\mnote{16},可把整个人类、动物、植物统统消灭光,后来又说没有讲过,几次否认。我不否认我说过的话,你不要替我否认这个所谓谣言。

\mxsay{斯:}我的书里也说主席也可能这么说的,目的是看看对方的反映怎样。

\mxsay{毛:}这是因为有一位大国政治家说,那时将找不到任何政府了,我反驳他的意见。

\mxsay{斯:}就是在这样情况下第一次提的?

\mxsay{毛:}是的。这是一九五四年十月的事。美国人说什么原子弹毁灭性严重,赫鲁晓夫也说得很神气,他们都超过我,我比他们落后了。是不是这样?相当落后。最近看见报道说许多美国专家访问比基尼岛\mnote{17},他们登陆后看到老鼠照样在跑来跑去,鱼照样在湖里游来游去,井水还能喝,植物茂盛,鸟类很多,专家在进岛时要开辟道路,砍掉树木。这是在这个岛经历过十二年的核武器爆炸试验,隔了六年再去的。爆炸之后大概有一两年生物是倒霉的,然后又生长起来。为什么老鼠根本不受影响,没有毁灭?因为它们钻到洞里去了。植物为什么那么多,不受影响?大概死了不少,剩下一点又生长起来,经过几年又大大发展起来。

\mxsay{斯:}我看过一部电影,在氢弹爆炸不久,生物都死光了,海龟跑到岸上去下蛋,但是不生小海龟。

\mxsay{毛:}过几年可能又会生的。对人类是不是也这样,就不知道了。

\mxsay{斯:}甲虫生命力最强。

\mxsay{毛:}总之,对那地方的鸟、树、海龟说来,原子弹不过是纸老虎。可能人类要比它们脆弱一点。

\mxsay{斯:}人对自己制造的毒素更易受灾。蚂蚁认为自己统治世界,它是世界之主。

\mxsay{毛:}什么人在蚂蚁看来,都是微不足道的。蚂蚁还是比较大的动物。细菌是在做人所不能做的事。全世界只有这么多的人,三十多亿。但是据土壤学家告诉我,每亩地里含有细菌四百公斤。没有细菌不能造成土壤,也不能生长植物。所以不要看它那么小,什么人的身体里它都能钻进去,不管你是总统也好,新闻记者也好,它可厉害了。

\mxsay{斯:}细菌根本看不见的。

\mxsay{毛:}人离开它是不能生活的,不知人体中有多少细菌,医学上说,有大肠杆菌,口腔里有真菌,没有它对人很不利。我们在这个问题上可以有一致的意见。美国专家在比基尼岛的调查是很好的材料,我们把这个材料印发给人大代表看了。

\mxsay{斯:}这是公开报告?

\mxsay{毛:}不是,是一个中国人引用美国专家材料写的一篇通讯,那篇通讯登在香港的《新闻天地》上。

\mxsay{斯:}虽然如此,主席并不是认为核战争是件好事?

\mxsay{毛:}对。根本不要打核战争,要打就用常规武器打。

\mxsay{斯:}看来亚非拉地区越来越近代化,革命越来越发展。

\mxsay{毛:}可能。

\mxsay{斯:}亚非拉国家的革命是不是可以在没有第三次世界大战的情况下完成?

\mxsay{毛:}完成就难说了,可能要相当长的时间。

\mxsay{斯:}关于印尼退出联合国问题,中国表示支持。这是否会为其他国家退出联合国开创先例?

\mxsay{毛:}开先例的是美国。美国不让中国进入联合国,提出要三分之二多数票通过才能进入。中国没有进入联合国,不是也很好吗?印尼退出联合国,也是它觉得参加联合国没有什么好处。

\mxsay{斯:}能不能说中国不想进入联合国?

\mxsay{毛:}不能。如果联合国三分之二的国家要我们进去,而我们不进去,不是要说我们是民族主义者了吗?但我们要联合国撤销中国是侵略者的诬蔑,同时要指出美国是侵略者,你看这个理由行吗?中国作为一个侵略国家怎么能进联合国?说美国是侵略者,它不会同意的。现在我们还不想进去,美国也不愿意我们进去。我们进去了,美国会感到碍手碍脚。在这点上,双方是有一定的共同之处的。现在还是让蒋委员长留在联合国里代表中国吧!这些话你可不要报道,我们还没有公布过。

\mxsay{斯:}是不是可能有一个没有美国的联合国?

\mxsay{毛:}亚非会议\mnote{18}就没有美国参加。

\mxsay{斯:}还有新兴力量运动会\mnote{19}。

\mxsay{毛:}中国很大,自己要办的事很多,也很忙。中国也是一个“联合国”。我们这个“联合国”接待你,那个联合国还没有接待过你吧?你打算什么时候离开中国?

\mxsay{斯:}再过几天就走。这次我回去,约翰逊\mnote{20}可能找人让我去见他一次。你有什么口信要捎给他吗?

\mxsay{毛:}没有。

\mxsay{斯:}我也可以把这句话捎给他。

西方有些“毛学”专家,互相展开争辩,观点各有不同。我不久前在日内瓦参加了一次“北京问题专家”的会,会上辩论的一个问题是,《矛盾论》是不是对马列主义作出了新的贡献?

\mxsay{毛:}是一些什么人?出版商吗?

\mxsay{斯:}主要是大学教授,俄文、中文专家。辩论中提出的一个问题是,《矛盾论》是不是真的在一九三七年写的,是不是在《辩证唯物主义》\mnote{21}小册子之前写的。

\mxsay{毛:}是一九三七年写的。当时大家都去前线打日本了。

\mxsay{斯:}当时有时间做研究工作?

\mxsay{毛:}那时抗日军政大学要我去讲一讲哲学。

\mxsay{斯:}《矛盾论》是讲演的一部分?

\mxsay{毛:}就是。他们强迫我去讲课,我没有办法。这是写的讲义的一部分。花了几个星期,搜集了些材料,主要是总结中国革命的经验,每天晚上写,白天睡觉。讲课只讲了两个钟头。我讲课的时候,不准他们看书,也不准他们做笔记,我把讲义的大意讲了一下。

\mxsay{斯:}是在写《辩证唯物主义》小册子以前几年?

\mxsay{毛:}我不记得写过那样一本小册子。其实,《矛盾论》不如《实践论》那篇文章好。《实践论》是讲认识过程,说明人的认识是从什么地方来的,又向什么地方去。

\mxsay{斯:}这两篇文章是同时写的吧?

\mxsay{毛:}先后不久。

\mxsay{斯:}是一九三八、一九三九年吗?

\mxsay{毛:}不是一九三八年,三八年忙起来了,是一九三七年。

\mxsay{斯:}现在我可以告诉那些教授们,主席自己是怎么讲的。教授们在进行学术性辩论时,可能还坚持他们的看法。主席看过黑格尔\mnote{22}的文章吗?

\mxsay{毛:}看过一些,还有费尔巴哈\mnote{23}的。海克尔写的一本书\mnote{24}里头有相当丰富的材料,他不承认他自己是唯物主义者,实际上是唯物主义者。

\mxsay{斯:}什么时候读的?

\mxsay{毛:}那很久了,是打游击战争的时候。

\mxsay{斯:}主席一面搞革命,一面给许多教授提供了职业,现在可能很多人成为“毛学”专家。

\mxsay{毛:}中国战国时期有一个人写了一部著作,叫《老子》,后来注解《老子》的在一百家以上。现在我的这些东西,甚至马克思、恩格斯、列宁的东西,在一千年以后看来可能是可笑的了。

\mxsay{斯:}一千年是很长的时间。

\mxsay{毛:}今后的一千年比过去的一千年可能变化大。

\mxsay{斯:}深刻的技术革命,征服宇宙空间。但我相信主席著作的影响将远远超出我们这一代和下一代。

\mxsay{毛:}你可能讲得过分了,我自己都不相信。

\mxsay{斯:}中国历史上没有任何人物像主席经历过这么多的变革,从开始作为一个学生,到参加革命,到革命完成,并成为历史学家、哲学家。

\mxsay{毛:}我不能驳你,也不可能赞成。这要看后人、看几十年后怎么看了。在一些人看来,我是坏人是定了的。帝国主义、修正主义、各国反动派不赞成我,包括蒋介石不赞成我。他不赞成我,我也不赞成他。这就要发生争论,有时要写文章,有时要动武。

\mxsay{斯:}现在中国强调在青年中保持革命精神。重要的是否在于给类似国家树立榜样,促进其他国家革命,以使中国革命得到最后的安全?

\mxsay{毛:}青年们没有见过地主剥削、资本家剥削,也没有打过仗,没有看见过什么是帝国主义。就是现在二十几岁的人,当时也只有十岁左右,对旧社会什么也不知道。所以由他们的父母、老年人讲一讲过去,很有必要,不然不知道过去那段历史。你刚才说的最后这一点,很难讲。你讲有什么安全?现在不是在说裁军吗?究竟哪一年裁军了?不是讲普遍、彻底裁军吗?过去苏联讲,现在美国也讲,我们也赞成过普遍裁军。事实上,现在是普遍、彻底扩军。嘴里说普遍裁军,实际上普遍扩军。

\mxsay{斯:}北大西洋公约组织\mnote{25}就是这样,现在一分为二,二分为四,每个国家都要有自己的原子弹。

\mxsay{毛:}就是不许可中国有原子弹,我们也不希望自己有那么多原子弹,要那么多干什么?稍微有一点也好,做些科学实验。

\mxsay{斯:}主席曾经说过,在江西打倒土豪劣绅时,他们说苏维埃先生是个很坏的家伙。而且在中国整个革命发展过程中,西方都在讲社会主义先生制造很多麻烦,现在又归结于中国的原子弹。

\mxsay{毛:}这证明我的“名誉”不好,中国政府、中国党的“名誉”都不大好。他们为什么要反对中国,搞反华高潮?我们还措手不及,突然肯尼迪不见了。越南人民还措手不及,吴庭艳不见了。再一个措手不及,赫鲁晓夫下台了,真是天晓得,而且搞得那么彻底,他的书、照片一概收起来了。

\mxsay{斯:}好多欧洲的党批评苏联党用这种办法把赫鲁晓夫搞下台。

\mxsay{毛:}我们这里赫鲁晓夫的照片没有多少,书店里照样有赫鲁晓夫的书。世界没有赫鲁晓夫还行!赫鲁晓夫阴魂不散,他这种人总是有的。

\mxsay{斯:}现在的苏共新领导能不能说是三七开,七分是对的?

\mxsay{毛:}苏共现在的领导?这很难说,我不讲这个话。外面讲他们要搞没有赫鲁晓夫的赫鲁晓夫主义。

\mxsay{斯:}赫鲁晓夫下台后,中苏关系有什么改进?

\mxsay{毛:}可能有点,但是不多,使我们丧失了一个写文章批评的对象。

\mxsay{斯:}在俄国有人说中国有个人迷信。

\mxsay{毛:}恐怕有一点。据说斯大林是有的,赫鲁晓夫一点也没有,中国人是有的。这也有点道理。赫鲁晓夫倒台了,大概就是因为他没有个人迷信。

\mxsay{斯:}我认为我能认识你,是极大的荣幸,也为我个人带来许多好处,我也希望我曾把你的思想转告别人。我真心觉得你的成就是伟大的,当然不是一切都是好的,但总之做了许多伟大的事情。感到遗憾的是,由于历史的原因,中美两国、两国人民被分开了。

\mxsay{毛:}由于历史的原因,两国是会接近起来的。要等候,总会有这么一天。

\mxsay{斯:}我不认为中美之间会发生大战。

\mxsay{毛:}这也可能你是对的。中国这个地方,美国军队来可以,不来也可以。来了没有什么很大的搞头,我们不会让美国军队得到好处。因为这点,也许他们就不来了。我们不会打到美国去,这我已经说了,你们可以放心。

\mxsay{斯:}美国常说,南越战争要扩大到北方。

\mxsay{毛:}最近腊斯克\mnote{26}纠正了他的说法,说他没有讲过这种话。

\mxsay{斯:}当然,我认为美国政府不会听我的话。美国有一位议员叫丘奇,他现在提出对美国干涉别国的政策要进行一次大辩论。他是约翰逊的好朋友。美国统治者不了解主席,我本人恐怕也不了解。

\mxsay{毛:}怎么不了解?我们不会打出去,只有美国打进来,我们才打。这点有历史作证。我国忙自己的事还忙不过来,打出去是犯罪的,为什么要打出去?南越根本不需要我们去,他们自己可以对付。

\mxsay{斯:}打南越战争的美国人在说,如果美国撤出南越,中国将占领整个东南亚。

\mxsay{毛:}怎么占领法?我们的军队去占领,还是当地人民去占领?中国人还是占领中国。

\mxsay{斯:}中国在南越有没有军队?

\mxsay{毛:}没有。

\mxsay{斯:}腊斯克说,如果中国和北越放弃在东南亚的侵略政策,美国就撤出南越。

\mxsay{毛:}我们没有什么侵略政策可以放弃,我们没有侵略。可是中国支持革命,不支持不行。哪里发生革命,我们就发表声明支持,并开些大会声援。帝国主义讨厌的就是这个。我们喜欢说空话,放空炮,但不出兵。放空炮,就叫侵略?出了兵的,反而不叫侵略?

\mxsay{斯:}过去说中国受俄国支持,现在说南越受中国支持。

\mxsay{毛:}中国内战取得胜利,主要是靠美国的武器,这证明没有什么外国正面的支持。实际上,南越是从美国取得武器。去年以来,他们不仅能取得武器,还补充兵力,经常俘虏南越伪军。这同过去我们兵源之一是蒋介石的军队一样,他们是受过训练的,是被抓来强迫当兵的,一经俘虏就能参加我军作战。

\mxsay{斯:}为什么?

\mxsay{毛:}他们是被国民党抓壮丁抓去当兵的,他们不喜欢国民党。

\mxsay{斯:}还有一点,中共和全国老百姓是一致的。

\mxsay{毛:}被国民党抓去当兵的是贫苦农民。我们的办法是,花几天工夫开诉苦大会、祭灵大会,谁家有人被国民党害死,就把死者的名字写在纸上祭灵。这样解决问题后,他们就参加了我们的军队,换上一顶帽子。为什么他们一定要戴我们的帽子呢?因为他们怕打死了以后,被人错认为国民党的兵。从头上这顶帽子就可以认出来他们是我们的人了。

\mxsay{斯:}在很大程度上,南越就是这种情况。

\mxsay{毛:}哪里有压迫,不革命就不行。社会主义革命就是这样。资产阶级也是在资本主义发展到一定程度时,就要反对封建主义。美国没有封建制度,有殖民主义——英国,等美国资本主义发展到一定程度,它就要反对英国。世界上的人,不受压迫谁起来革命?美国发生独立战争\mnote{27},就是因为受英国的压迫。美国独立战争差不多有二百年了吧?

\mxsay{斯:}美国独立战争时有很多革命家,提出的口号同后来法国大革命\mnote{28}提出的口号一样。那时美国成为世界上唯一的共和国。当时欧洲国家对美国的看法同今天美国对中国的看法一样。

\mxsay{毛:}华盛顿\mnote{29}的“名誉”不好,我们可以追认他为“共产党”。

\mxsay{斯:}中共会把他看成是个反动的人,所以不会让他参加党。

\mxsay{毛:}不能参加共产党是一件事,那时还没有共产党,华盛顿所起的革命作用,我们应当承认,他当时起了很先进的作用,是很进步的。还有林肯\mnote{30}也一样。

\mxsay{斯:}林肯是一个自相矛盾的人,也是伟大的人。他是一个人道主义者。希望在我走之前,请主席向美国人民说几句话,美国人民对中国是有好感的。

\mxsay{毛:}祝他们进步。如果我祝他们获得解放,他们有些人可能不大赞成。我就祝那些认识到自己还没有解放的、生活上有困难的人获得解放。

\mxsay{斯:}主席的话非常好,特别是同前面的话联系起来,就是中国不会打出去,中国在忙于自己的事。我本人看到了这一点。

\mxsay{毛:}美国人需要再解放,这是他们自己的事。不是从英国的统治下解放,而是从垄断资本的统治下解放出来。

\mxsay{斯:}主席能否给美国总统也提些建议?

\mxsay{毛:}这不好提。美国人的手伸到全世界,我们早已提过要他们收回去一点,他们照例不听。

\mxsay{斯:}美国军队差不多有一半在外国。看样子,好多在外国的美国军队变成了当地人民的人质。

\mxsay{毛:}要走不好,不走也不好,这使美国政府处于困难的境地。要美国撤兵困难,不撤也困难。哪里有点风吹草动,它就要派兵,就这么调来调去。有的时候我们故意这么一叫,例如打金门几炮\mnote{31},就是因为我们打那么几炮,它觉得第七舰队不够了,把第六舰队开过来一部分,把旧金山的海军也开一部分过来。我们又不打炮了,美国军队来了没事干,又要开回去。所以美国军队是可以调动的,叫它怎么样它就怎么样。有点像蒋介石的军队,叫它怎样就怎样。

\mxsay{斯:}美国军队总要有点事干才行。

\mxsay{毛:}有事干的。美国垄断资本就要到有些地方去帮反动派的忙,叫它不帮不行,一定要帮,最后它一定要走,像帮助蒋介石一样。过去上海、青岛、天津、唐山、北京都有美军,后来都走了,而且走得很快,我们军队同它还隔好远,它就赶快走。那时英国就很蠢,派军舰到南京去接兵,被我们打着了。\mnote{32}问题是中国有这么一个不争气的蒋介石总打败仗,又有强大的解放军。不具备这些条件的地方,美国就呆着不走。

\mxsay{斯:}只有在同样情况下美国才会从南越撤走?

\mxsay{毛:}美国在南越的军队现在不走,可能再打一两年,但美国感到没有味道了,走了也难说。

\mxsay{斯:}如果我没有理解错周恩来总理的话,总理同我说,美国军队撤走前,不可能通过开会解决南越问题。是这样的吗?

\mxsay{毛:}我不晓得总理怎么讲的,恐怕两种可能性都要讲。军队撤出前可以谈,军队撤出以后也可以谈。或者根本不谈,南越把美军赶走。甚至谈了美军还不走,像它在朝鲜那样。在日内瓦也是谈过的。日内瓦会议\mnote{33}后,美国把兵开进南越代替法国军队。老实说,美军留在南越是件好事,它锻炼了南越人民,使解放军壮大。只有一个吴庭艳是不行的,就像中国单有一个蒋介石是不行的,必须要日本占领大半个中国,而且占领八年之久,才能锻炼中国人民。

\begin{maonote}
\mnitem{1}埃德加·斯诺,美国进步作家、记者。一九二八年第一次到中国。一九三六年到陕北革命根据地访问,见到了毛泽东等中共和红军的领导人,后写了《西行漫记》等书。新中国成立后,在一九六〇年、一九六四年、一九七〇年访问中国。一九七二年二月十五日在瑞士病逝。
\mnitem{2}南越战争,即越南战争。一九五四年日内瓦会议后,美国为了取代法国在印度支那的宗主国地位,破坏日内瓦协议,在越南南方扶植建立以吴庭艳为总统的亲美的“越南共和国”。一九六一年,美国对南越发动由美国出钱出枪、南越吴庭艳集团出人的“特种战争”,镇压越南南方军民的武装反抗。一九六五年初美国一面对越南北方大规模连续轰炸,一面向越南南方增派大量军队,把侵越战争升级为“局部战争”。
\mnitem{3}吴庭艳(一九〇一——一九六三),原“越南共和国”总统兼总理和国防部长。一九六三年十一月一日在美国策划的军事政变中,同其弟吴庭儒一起被击毙。
\mnitem{4}胡志明,时任越南劳动党中央委员会主席、越南民主共和国主席。
\mnitem{5}西贡,这里指美国扶植的南越政权,其统治中心在西贡。
\mnitem{6}傅作义(一八九五——一九七四),山西荣河安昌村(今属临猗)人。曾任国民党军华北“剿总”总司令、察哈尔省政府主席等职。一九四九年一月率部接受中国人民解放军的和平改编,对北平和绥远的和平解放做出了贡献。中华人民共和国成立后,任中央人民政府委员、国防委员会副主席、政协全国委员会副主席、水利电力部部长等职。
\mnitem{7}陶峙岳(一八九二——一九八八),湖南宁乡人。曾任国民党军新疆警备总司令部总司令。一九四九年九月率部起义。中华人民共和国成立后,任中国人民解放军新疆军区副司令员、新疆生产建设兵团司令员、国防委员会委员、全国人大常委会委员、政协全国委员会副主席等职。
\mnitem{8}程潜(一八八二——一九六八),湖南醴陵人。曾任国民党军长沙绥靖公署主任兼湖南省政府主席。一九四九年八月与陈明仁率部起义,湖南和平解放。中华人民共和国成立后,任中央人民政府委员、全国人大常委会副委员长、国防委员会副主席、政协全国委员会常务委员、湖南省省长等职。陈明仁(一九〇三——一九七四),湖南醴陵人。曾任国民党政府华中军政长官公署副长官兼第一兵团司令官。一九四九年八月与程潜率部起义,湖南和平解放。中华人民共和国成立后,任中国人民解放军湖南军区副司令员,第二十一兵团司令员,国防委员会委员,政协全国委员会常务委员等职。
\mnitem{9}肯尼迪,一九六〇年至一九六三年任美国总统。
\mnitem{10}希特勒(一八八九——一九四五),德国法西斯首领、纳粹党党魁。一九三三年在德国垄断资产阶级支持下出任总理,次年总统兴登堡死后,自称国家元首,实行法西斯统治,积极扩军备战。一九三九年九月派德军入侵波兰,挑起第二次世界大战;一九四一年六月大举进攻苏联。一九四五年四月在苏军解放柏林时自杀。
\mnitem{11}敦刻尔克,是法国北部一海港城市。这里指第二次世界大战初期英、法军队从敦刻尔克撤退一事。一九四〇年五月,德军进攻比利时、荷兰、卢森堡,侵入法国,击败英、法军队。英国远征军二十二万人、法军二十万人被迫退到比利时、法国沿海地区,面临被歼灭的危险。五月二十七日至六月四日,英军及部分法军共三十多万人从敦刻尔克地区越过英吉利海峡撤往英国。虽然撤退时丢弃了大量武器装备,但保存了有生力量。
\mnitem{12}指艾登。一九五四年日内瓦会议时他任英国副首相兼外交大臣,一九五五年至一九五七年任英国首相。
\mnitem{13}南北战争,是一八六一年至一八六五年由美国南部种植园主奴隶制与北部资产阶级雇佣劳动制之间的矛盾所引起的资产阶级民主革命战争。在战争过程中,代表北部资产阶级利益的联邦政府总统林肯颁布了《宅地法》和《解放黑奴宣言》,并采取其他民主措施,激发了工人、农民和黑人的革命斗志,因而联邦政府取得了战争的胜利。
\mnitem{14}尼赫鲁,一九四七年至一九六四年任印度总理。
\mnitem{15}赫鲁晓夫,原为苏联共产党中央委员会第一书记、苏联部长会议主席。一九六四年十月被解除领导职务。
\mnitem{16}死光,即激光。
\mnitem{17}比基尼岛,是马绍尔群岛中的一个珊瑚岛,一九四七年成为美国托管地,曾经是美国的核试验基地。
\mnitem{18}亚非会议,即万隆会议,一九五五年四月十八日至二十四日在印度尼西亚万隆召开。参加会议的有缅甸、锡兰(今斯里兰卡)、印度、印度尼西亚和巴基斯坦五个发起国,以及阿富汗、柬埔寨、中华人民共和国、埃及等,共二十九个亚非国家。会议广泛讨论了民族主权、反殖民主义斗争、世界和平以及与会国之间的经济文化合作等问题,发表了《亚非会议最后公报》,提出了著名的关于促进世界和平与合作的十项原则。
\mnitem{19}新兴力量运动会,是一九六三年十一月十日至二十二日在印度尼西亚首都雅加达举行的新兴力量国家及地区参加的运动会。参加运动会的有来自亚洲、非洲、拉丁美洲和欧洲四十多个国家和地区的二千余名运动员。
\mnitem{20}约翰逊,时任美国总统。
\mnitem{21}指一九三七年七、八月毛泽东为抗日军政大学讲授哲学而写的《辩证法唯物论(讲授提纲)》。后来,毛泽东将其中的两节,整理成为\mxart{实践论}和\mxart{矛盾论}两篇著作,收入《毛泽东选集》。
\mnitem{22}黑格尔(一七七〇——一八三一),德国古典哲学家,客观唯心主义者,辩证法大师。主要著作有《精神现象学》、《逻辑学》、《法哲学原理》等。
\mnitem{23}费尔巴哈(一八〇四——一八七二),德国古典哲学家,唯物主义的代表。主要著作有《黑格尔哲学批判》、《基督教的本质》等。
\mnitem{24}指海克尔的《宇宙之谜》。海克尔(一八三四——一九一九),德国自然科学家、达尔文主义的卓越代表之一。他在《宇宙之谜》一书中,批判了唯心主义和僧侣主义,试图根据最新的科学成就建立严整的唯物主义体系。
\mnitem{25}北大西洋公约组织,一九四九年四月,美国、英国、法国、荷兰、比利时、卢森堡、挪威、葡萄牙、意大利、丹麦、冰岛和加拿大在华盛顿签署《北大西洋公约》。同年八月二十四日公约生效,北大西洋公约军事集团建立。希腊和土耳其于一九五二年,德意志联邦共和国于一九五五年,西班牙于一九八二年,波兰、捷克和匈牙利于一九九九年,正式加入该组织。
\mnitem{26}腊斯克,一九〇九年生,美国民主党人,时任美国国务卿。
\mnitem{27}独立战争,指一七七五年至一七八三年北美十三个殖民地人民推翻英国殖民统治、争取独立的战争。一七七五年五月殖民地代表召开会议,任命华盛顿为殖民地反英军队总司令,并于一七七六年发表《独立宣言》。一七八三年双方签订《巴黎和约》,正式承认十三个殖民地脱离英国独立。
\mnitem{28}法国大革命,指一七八九年至一七九四年法国资产阶级革命。它是在法国封建制度极端腐朽,第一等级(僧侣)和第二等级(贵族)与广大的第三等级(农民、城市平民和资产阶级)之间的矛盾日益尖锐化的情况下爆发的。这次革命推翻了法国封建专制制度,促进了法国资本主义的发展,并推动了欧洲各国的资产阶级革命运动。
\mnitem{29}华盛顿(一七三二——一七九九),美国第一任总统。一七七五年美国独立战争爆发后,任大陆军总司令,将武装落后、组织松散的地方民军整编训练成为能与英军正面抗衡的正规军,领导美国取得独立战争的胜利。
\mnitem{30}林肯(一八〇九——一八六五),美国共和党人,一八六一年至一八六五年任美国总统。他领导了反对南方奴隶制的战争,颁布了著名的《宅地法》和《解放黑奴宣言》。
\mnitem{31}指炮击金门:一九五八年七月,台湾国民党当局在美国的支持下叫嚷“反攻大陆”,并不断炮击福建沿海村镇。为严惩国民党军,反对美国侵犯中国主权,人民解放军福建前线部队奉命于八月二十三日开始对国民党军金门防卫部和炮兵阵地等军事目标进行炮击,封锁了金门岛,中断国民党军的补给。九月初,美国向台湾海峡地区大量增兵,派军舰、飞机直接为国民党军运输舰护航,公然入侵中国领海。为打击美国的侵略行径,人民解放军前线部队又于九月八日对金门国民党军和海上舰艇进行全面炮击。至一九五九年一月七日,共进行七次大规模炮击,十三次空战,三次海战,击落击伤国民党军飞机三十六架,击沉击伤军舰十七艘,毙伤国民党军七千余人。
\mnitem{32}一九四九年四月二十日至二十一日,当人民解放军渡江作战的时候,侵入中国内河长江的紫石英号等四艘英国军舰先后驶向人民解放军防区,妨碍渡江,中英双方发生了军事冲突。英舰开炮打死打伤人民解放军二百五十多人。紫石英号被人民解放军击伤被迫停于镇江附近江中,其他三艘英舰逃走。七月三十日夜紫石英号军舰逃出长江。
\mnitem{33}日内瓦会议,指一九五四年四月二十六日至七月二十一日在瑞士日内瓦召开的讨论和平解决朝鲜问题和恢复印度支那和平问题的国际会议。中、苏、美、英、法五国参加所有两项议题的讨论。朝鲜北南双方及美、英、法以外的其他十二个侵略朝鲜北方的国家参加了朝鲜问题的讨论,越南民主共和国、老挝、柬埔寨和南越政权参加了印度支那问题的讨论。关于朝鲜问题没有达成任何协议;关于恢复印度支那和平问题,分别达成关于在印度支那三国停止敌对行动的协定和《日内瓦会议最后宣言》(总称日内瓦协议),实现了印度支那的停战。
\end{maonote}
