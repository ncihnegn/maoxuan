
\title{在政治局会议上的讲话——八大军区司令员对调}
\date{一九七三年十二月}
\thanks{这是毛泽东同志在政治局会议上有关八大军区对调\mnote{1}的谈话。}
\maketitle

\date{一九七三年十二月十二日}
\section*{(一)}

我提议,议一个军事问题,全国各大军区司令员互相对调。(对叶剑英\mnote{2})你是赞成的。我赞成你的意见,我代表你说话。我先找了总理\mnote{3}、王洪文\mnote{4}两位同志,他们也赞成。一个人在一个地方搞久了,不行呢。搞久了,油了呢。有几个大军区,政治委员不起作用,司令员拍板就算。我想了好几年了,主要问题是军区司令员互相调动,政治委员不走。(带领大家唱《三大纪律八项注意》),步调要一致,不一致就不行。

政治局要议政。军委要议军,不仅要议军,还要议政。军委不议军,政治局不议政,以后改了吧。你们不改,我就要开会,到这里来。我毫无办法,我无非是开个会,跟你们吹一吹,当面讲,在政治局。

\date{一九七三年十二月二十一日}
\section*{(二)}

你陈(锡联)司令,济南的杨得志同志,南京的许世友同志,这几个同志呢,在一个地方搞得太久了。这个李德生同志、曾思玉同志、丁盛同志倒是搞得不那么久呢!你们带个头呢。省军区、军分区、人武部就会照样去做。

到一个新地方有很多困难呢,不熟人,不熟地方,不熟党,不熟军,党政军民都不熟。党政军民学,东西南北中。

(许世友:有党,有同志们,可以学习。)

慢慢来,就会顺手。

你们呢,要交好班呢。有困难啊,人生地不熟。有些人就批你们。大多数呢,舍不得你们走呢。(对韩先楚)我不是跟你讲过嘛,这个世界上这类事啊,心放宽些,胆子壮些。心要宽,胆要大。无非是做官嘛,革命嘛,一个不撤,一个不批吧!你们想一想,总是有些缺点,十个指头有一个指头的缺点。一切错误都是我。我错误大呢,比你们大,所以屡次想辞掉这个主席。八大我还设了一个名誉主席,就是为着我想当个名誉主席,让别人当主席。

我向基辛格\mnote{5}讲了差不多三个小时。其实只有一句话:当心!北极熊\mnote{6}要整你美国!一整太平洋的舰队,二整欧洲,三整中东。

\begin{maonote}
\mnitem{1}八大军区对调,一九七三年十二月二十二日,毛泽东签发命令,八大军区司令员对调。当时全国共有十一个大军区,新疆军区司令员杨勇、成都军区司令员秦基伟、昆明军区司令员王必成都任职时间不长,最长的也才四个月,他们都原封不动。其它八个军区司令员对调:

北京军区司令员李德生与沈阳军区司令员陈锡联对调。

济南军区司令员杨得志与武汉军区司令员曾思玉对调。

南京军区司令员许世友与广州军区司令员丁盛对调。

福州军区司令员韩先楚与兰州军区司令员皮定钧对调。
\mnitem{2}叶剑英,时任中央军委副主席,主持军委日常工作。
\mnitem{3}总理,指周恩来。
\mnitem{4}王洪文,时任中共中央副主席。
\mnitem{5}基辛格,一九七三年二月和十一月两次访华,毛泽东都接见了他。就国际形势尤其是苏联的威胁交换了意见。
\mnitem{6}北极熊,指苏联。
\end{maonote}
