
\title{在中央工作会议上的讲话}
\date{一九六六年十月二十四日、二十五日}
\maketitle


\date{一九六六年十月二十四日}
\section*{(一)}

你们有什么可怕的呢?你们看了李雪峰\mnote{1}的简报没有?他的两个孩子跑出去搞串连,回来教育李雪峰说:“我们这里的老首长为什么那么害怕红卫兵呢?我们又没打你们。”你们就是不检讨!

伍修权\mnote{2}有四个孩子,分为四派。有很多同学到他家里去,有时几个人或十几个人。接触多了,就没有什么可怕的了,觉得他们很可爱。自己要教育人,教育者应先受教育。你们思想搞不通,不敢见红卫兵,不和学生说真话,做官当老爷。先不敢见面,后不敢说话。革了几十年的命,越来越蠢了。少奇给江渭清的信,批评了江渭清\mnote{3},说他蠢,他自己就聪明了?

(刘澜涛\mnote{4}:红卫兵到处抢档案,查我们的历史问题,搞得大过火了。)

你回去打算怎么办?

(刘澜涛:回去看看再说。)

你说话总是那么吞吞吐吐。

(问李井泉\mnote{5}):你们四川那个廖志高\mnote{6}现在怎么样啊?

(李井泉:开始不大通,全会后一段时间比较好了,从历史上看,他还是一贯正确的。)

什么一贯正确?群众起来后你自己就溜了,吓得魂不附体,跑到军区去住。回去要振作精神,好好搞一搞。但万万不能承认反党反社会主义,把中央局、省、市委都打倒,让他们学生来接班,行吗?不知工农业,只读一点书,行吗?

把刘少奇、邓小平的大字报贴到大街上去不好,要允许人家犯错误,要允许人家革命,允许改嘛。让红卫兵看看《阿Q正传》\mnote{7}。

依我看,这次会开得比较好些。上次会\mnote{8}是灌而不进,没有经验。这次会议有了两个月的经验,一共不到五个月的经验。民主革命搞了二十八年,犯了多少错误?!死了多少人?!社会主义革命搞了十七年,文化革命只搞了五个月,最少得五年才能得出经验。一张大字报,一个红卫兵,一个大串联\mnote{9},谁也没料到,连我也没料到,弄得各省市“呜呼哀哉”。学生也犯了一些错误,主要是我们这些老爷们犯了错误。

(问李先念\mnote{10})你们今天会开得怎样?

(李先念:财经学院说,他们明天要开声讨会。我要检查,他们不让我说话)

你明天还去检查,不然人家说你溜了。

(李先念:明天我要出国。)

你要告诉他们一下。过去是“三娘教子”,现在是“子教三娘”。我看你精神有点不足。他们不听你们检讨,你们就偏检讨。他们声讨,你们就承认错误。乱子是中央闹起来的,责任在中央,地方也有责任。

我的责任是分一、二线\mnote{11}。为什么分一、二线呢?一是身体不好,二是苏联的教训。马林科夫\mnote{12}不成熟,在斯大林死前没有当权,每次会议都只是敬酒、吹吹捧捧。我想在我没死之前,树立他们的威信,没有想到反面。

(陶铸:大权旁落。)

这是我故意大权旁落,可他们不但不自觉,反而和我闹独立王国,许多事情不与我商量,如土地会议、天津讲话、山西合作社、否定调查研究、大捧王光美\mnote{13}。本来应经中央讨论作个决议就好了,可是他(刘少奇)偏要自行其是。邓小平从来不找我,从一九五九年到现在,什么事情都不找我。书记处、计委干什么我也不知道。六二年,忽然四个副总理就是李富春、谭震林、李先念、薄一波\mnote{14}到南京来找我,后又到天津,他们要办的事情我马上答应,后来,四个人又回去了,可邓小平就是不来。武昌会议\mnote{15}我不满意,高指标弄得我毫无办法。到北京开会\mnote{16},你们开了六天,我要开一天还不行,完不成任务不要紧,不要如丧考妣。

遵义会议以后,中央比较集中。三八年六中全会以后,项英、彭德怀\mnote{17}搞独立王国,那些事情(皖南事变、百团大战)都不打招呼。七大后中央没有几个人,胡宗南进攻延安,中央分两路,我同周恩来、任弼时在陕北,刘少奇、朱德在华北,还比较集中。进城以后就分散了,各搞一套,特别分一线、二线后,就更分散了。一九五三年财经会议以后,就打过招呼,要大家互相通气,向中央通气,向地方通气。刘、邓两人是搞公开的,不是秘密的,与彭真不同。过去陈独秀、张国焘、王明、罗章龙、李立三都是搞公开的,这不要紧。高岗、饶漱石、彭德怀都是搞两面手法,彭德怀与他们勾结上了,我不知道。彭真、罗瑞卿、陆定一、杨尚昆\mnote{18}是搞秘密的,搞秘密的没有好下场。

犯路线错误的要改,陈独秀、王明、李立三没改。不管什么小集团,什么门头,都要关紧关严,只要改过来,意见一致,团结就好。

要允许刘、邓革命,允许改。你们说我是和稀泥,我就是和稀泥的人。七大时,陈其涵\mnote{19}说不能把犯王明路线的选为中央委员。王明和其他几个人都选上中央委员了。现在只走了一个王明,其他人还在嘛!洛甫\mnote{20}不好,王稼祥我有好感,东崮一战他是赞成的。宁都会议洛甫要开除我,周、朱他们不同意,遵义会议他起了好作用,那个时候没有他们不行,洛甫是顽固的,少奇同志是反对他们的,聂荣臻也是反对他们的。对刘少奇不能一笔抹煞。

你们有错就改嘛!改了就行,回去振作精神,大胆放手工作。这次会议是我建议开的,时间这样短,不知是否通,可能比上次好。我没想到一张大字报,一个红卫兵,一个大串联就闹起来了这么大的事。学生有些出身不太好的,难道我们出身都好吗?不要招降纳叛,我的右派朋友很多,如周谷城、张治中,一个人不去接近几个右派,那怎么行呢?哪有那么干净?接近他们就是调查研究嘛,了解他们的动态。那天在天安门上我特意把李宗仁\mnote{21}拉在一起,这个人不安置比安置好,无职无权好。

民主党派要不要?一个党行不行?学校党组织不能恢复太早。一九四九年以后发展的党员很多,翦伯赞、吴晗、李达都是党员,都那么好吗?民主党派都那么坏?我看民主党派比彭罗陆杨就好。民主党派还要,政协也还要,同红卫兵讲清楚,中国的民主革命是孙中山搞起来的,那时没有共产党,是孙中山领导搞起来的,反康、反梁、反帝制。今年是孙中山诞生一百周年,怎样纪念呢?和红卫兵商量一下,还要开纪念会。我的分一线、二线走向反面。

(康生说:八大政治报告是阶级斗争熄灭论。)

报告我们看了,这是大会通过的,不能单叫他们两个负责。

\date{一九六六年十月二十五日}
\section*{(二)}

邓小平(虽然)耳朵聋,(但)一开会就在我很远的地方坐着。一九五九年以来,六年不向我汇报工作,书记处的工作他就抓彭真。邓小平对我是敬而远之。

对形势的看法,两头小,中间大,“敢”字当头的只有河南,“怕”字当头的占多数。真正“反”字的还是少数。真正四类干部\mnote{22}也就是百分之一、二、三。

(周恩来:现在已经大大超过了。)

多了不怕,将来平反嘛!有的不能在本地工作,可以调到别的地方工作。

河南一个书记搞生产,其余五个书记搞接待。全国只有刘建勋\mnote{23}写了一张大字报,支持少数派,这是好的。

\date{一九六六年十月二十五日}
\section*{(三)}

讲几句话,两件事:

十七年来,有一件事我看做得不好,就是搞一、二线。原来的意思是考虑到国家的安全,鉴于苏联斯大林的教训,搞了一线二线,我处在二线,别的同志在一线。现在看来不那么好,结果很分散,一进城就不能集中了,相当多的独立王国,所以十一中全会作了改变,这是一件事。以前,我处在二线,不主持日常工作,许多事让别人去搞,培养别人的威信,以便我见上帝的时候,国家不会出现那么大的震动,大家赞成这个意见,后来处在一线的同志,有些事情处理得不那么好。有些应当我抓的事情,我没有抓,所以,我也有责任,不能完全怪他们。

为什么说我也有责任呢?

第一,常委分一、二线,搞书记处,是我提议的,大家同意了。再嘛是过于信任别人了。这件事引起警惕,还是在制定二十三条\mnote{24}那个时候。北京就是没有办法,中央也没有办法。去年九、十月提出中央出了修正主义,地方怎么办?我就感到我的意见在北京不能实行。为什么批判吴晗不在北京发起,而在上海发起呢?因为北京没有人办。现在北京问题解决了。

第二件事,文化大革命闯了一个大祸,就是批了北大聂元梓一张大字报,给清华附中写了一封信,还有我自己写了一张《炮打司令部》的大字报。这几件事,时间很短,六、七、八、九、十,五个月不到,难怪同志们还不那么理解。时间很短,来势很猛,我也没有料到。北大大字报一广播,全国都闹起来了,红卫兵信还没有发出,全国红卫兵都动起来了,一冲就把你们冲了个不亦乐乎。我这个人闯了这么个大祸。所以你们有怨言,也是难怪的。

上次开会我是没有信心的,说过决定\mnote{25}通过了不一定能执行,果然很多同志还是不那么理解。现在经过两个月了,有了经验,好一点了。这次会议两个阶段,头一个阶段,大家发言都不那么正常,后一个阶段经中央同志讲话,交流经验,就比较顺了,思想就通了一些。运动只搞五个月,可能要搞两个五个月,也许还要多一点。

民主革命搞了二十八年,开始搞民主革命,谁也不知道怎么个革法,斗争怎么斗争法,以后才摸出一些经验。路也是一步一步从实践中走出来的,总结经验,搞了二十八年嘛。社会主义革命也搞了十七年,文化革命只有五个月嘛,所以就不能要求同志们都就那么理解。去年批判吴晗的文章,许多同志不去看,不那么管。以前批判《武训传》,批判《红楼梦研究》,是个别抓,抓不起来,不全盘抓不行,这个责任在我。个别抓,头痛医头,脚痛医脚,是不能解决问题的。这次文化大革命,前几个月,一、二、三、四月用那么多文章,中央又发了(五·一六)“通知”,可是并没有引起多大注意,还是大字报、红卫兵一冲,引起注意,不注意不行了。革命革到自己头上来了,赶快总结经验,做政治思想工作,为什么两个月之后又开这个会?就是总结经验,做好政治思想工作。你们回去以后有大量的政治思想工作要做。

中央局、省委、地委、县委召开十几天会,把问题讲清楚,也不要以为所有都能讲清楚。有人说,“原则通了,碰到具体问题处理不好。”原来我想不通,原则问题搞通了,具体问题还不好处理?现在看来还是有点道理,恐怕还是政治思想工作没有作好。上次开会回去,有些地方没有来得及很好开会,十个书记有七、八个接待红卫兵,一冲就冲乱了,学生们生了气,自己还不知道,也没有准备回答问题,还以为几十分钟讲一讲,表示欢迎就可以了。人家一肚子气,几个问题一问不能回答就被动了。这个被动是可以改变的,可以变被动为主动的。所以我对这次会议信心增强了。不知你们怎么样?如果回去还是老章程,维护现状,让一派红卫兵对立,拉另一派红卫兵保驾,就搞不好。我看会改变,情况会好转。当然不能过多地要求中央局、省、地、县广大干部全部都那么豁然贯通。不一定,总有那么一些人不通,有少数人是要对立的,但是我相信多数讲得通的。

上面讲了两件事情:

第一件事讲历史,讲十七年历史,一线二线不统一,别人有责任,我也有责任。

第二件事,五个月文化大革命,火是我点起来的,时间很仓促。与二十八年民主革命和十七年社会主义革命比起来,时间是很短的,只有五个月。不到半年,不那么通,有抵触情绪,是可以理解的。为什么不通!你们过去只搞工业、农业、交通,就是没有搞文化大革命,你们外交部也一样,军委也一样。你们没有想到的事情来了,来了就来了,我看冲一下有好处,多少年没有想,一冲就想了。无非是犯错误,什么路线错误,改了就算了,谁要打倒你们!我也是不想打倒你们,我看红卫兵也不要打倒你们。

这次会议发的简报不少,我几乎全部看了。你们过不了关,我也不好过,你们着急,我也着急,不能怪同志们,因为时间太短。有的同志说不是有心犯错误,是糊里糊涂犯了错误,可以原谅,也不能完全怪少奇同志和小平同志,他们有责任,中央也有责任,中央也没有管好,时间太短,新的问题没有精神准备,政治思想工作没有做好,我看十七天会议以后会好一些。

\begin{maonote}
\mnitem{1}李雪峰,时任北京市委第一书记。
\mnitem{2}伍修权,时任中共中央对外联络部副部长。
\mnitem{3}江渭清,时任江苏省委第一书记。一九六四年七月十四日,刘少奇视察江苏,问江渭清对王光美向参加省委四届扩大会议的同志所作的关于“桃园经验”的报告,有什么看法?江渭清说:“从江苏的实际出发,符合情况的,就学习运用,如果不符合江苏情况,就不照搬。”刘少奇说:“那你们江苏就不执行了?”江渭清说:“不盲目执行。”第二天,两人就这个问题再次发生“顶撞”,刘少奇发了脾气。

江渭清于九月八日以个人名义给刘少奇写了封信,报告了江苏省城乡社会主义教育运动的部署,检讨了省委办公厅通知要各地市、县委学习江渭清的一篇讲话,是个严重的错误,并承担了自己的领导责任。九月三十日,刘少奇给江渭清去信批评江渭清“骄傲自满,固步自封”,指出:“你们必须重新了解本省情况,重新组织革命的阶级队伍,才能进行当前的革命斗争。”十月二十日,中央发出《关于认真讨论刘少奇同志答江渭清同志的一封信的指示》。

毛泽东是支持江渭清的,十二月,毛泽东当着江渭清和刘少奇的面说:“少奇同志给你的一封信,是错误的。你的意见是对的。
\mnitem{4}刘澜涛,时任中共中央西北局第一书记,兼兰州军区第一政委,并担任西北三线建设委员会主任。
\mnitem{5}李井泉,时任中共中央西南局第一书记,兼成都军区第一政委。
\mnitem{6}廖志高,时任四川省委第一书记。
\mnitem{7}《阿Q正传》,鲁迅的代表作,其中反映了资产阶级革命没有动员和组织广大农民一起参加革命,注定了失败的结局。
\mnitem{8}上次会,指八届十一中全会。
\mnitem{9}一张大字报,指的是聂元梓等人写的《宋硕、陆平、彭佩云在文化大革命中究竟干了些什么?》;一个红卫兵,指毛泽东在给清华附中红卫兵的信中,对红卫兵表示“热烈支持”,红卫兵运动立即风靡全国;一个大串连,指毛泽东主席曾在天安门八次接见红卫兵,接见来自全国各地的红卫兵一千三百多万。
\mnitem{10}李先念,时任国务院副总理,兼中央财经领导小组副组长。
\mnitem{11}一二线,一九五三年底,中央常委分一线二线,毛泽东为二线,其余常委为一线,刘少奇主持常委会,八大时,中央设立书记处,邓小平任总书记,一九五九年,刘少奇接任国家主席,分一线二线后,中央日常事务由一线领导处理,主要是刘少奇、邓小平、周恩来。
\mnitem{12}马林科夫,斯大林指定的接班人,斯大林死后,一度担任苏联最高领导人,但很快就被赫鲁晓夫篡了权。一九五三年斯大林逝世后,任苏共中央第一书记、苏联部长会议主席。一九五五年二月,改任苏联部长会议副主席兼电站部部长。一九五七年六月被撤职。
\mnitem{13}土地会议,一九四七年夏刘少奇主持的全国土地会议上没有贯彻毛主席的土改方针,错误地实行所谓‘搬石头’的极左做法,提出一套排斥一切干部的做法。

天津讲话,一九四九年春天刘少奇在天津的讲话有右倾错误,有些口号比如提出‘中国不是资本主义大多了,而是资本主义太少了’,‘要发展资本主义剥削,这种剥削是进步的’等等,这些提法在一定程度上制造成思想上的混乱。

山西合作社,指刘少奇于一九五一年七月错误地批评了山西省委关于组织农业生产合作社的决定。一九五一年二月,山西省委决定试办初级合作社,但华北局认为山西在合作社问题上“冒”了,“用积累公积金和按劳分配来逐步动摇、削弱私有基础直至否定私有基础,是和党的新民主主义时期的政策及《共同纲领》精神不相符合的,因而是错误的,……不宜推广”,刘少奇在听了华北局报告后,原则意见与华北局一致,山西省委不同意这种看法,决定进行申述,“认为七届二中全会已经明确半社会主义性质的合作社是新民主主义五种经济成分之一,我们试办这种初级社,不存在违背《共同纲领》精神的问题。”五月七日,刘少奇在全国宣传工作会议上公开批评了山西省委的做法。他说:“山西省委在农村里边要组织农业生产合作社(苏联叫共耕社),这种合作社也是初步的。”“这种合作社是有社会主义性质的,可是单用这一种农业合作社、互助组的办法,使我们中国的农业直接走向社会主义化是不可能的。”“那是一种空想的农业社会主义,是实现不了的。”“我们中国党内有很大的一部分同志存在有农业社会主义思想,这种思想要纠正。”“农业社会化要依靠工业。”七月三日,刘少奇批转山西省委报告,批语比较尖锐。

在这种情况下,感到没有办法了,长治地委第一书记王谦王谦等人就给毛主席写了一封信,全面介绍了长治地区试办农业初级社的情况,毛主席详细了解了各方面意见后,表示,他不能支持华北局,他支持山西省委的意见。批评了互助组不能生长为农业合作社的观点和现阶段不能动摇私有基础的观点。他说:“既然西方资本主义在其发展过程中有一个工场手工业阶段,即尚未采用蒸汽动力机械、而依靠工场分工以形成新生产力的阶段,则中国的合作社,依靠统一经营形成新生产力,去动摇私有基础,也是可行的。”

十个试办初级社当年的粮食生产和其他经济的发展,都获得了历史上空前的丰收。四十多年后,时任华北局第一书记的薄一波在他的《若干重大决策与事件的回顾》一书中也详细谈到这件事情。他说:“少奇同志关于山西省委报告的处理,是有缺点的。我的处理也是有缺点的。”针对刘少奇对山西省委的批评,具体分析了当时处理不当和一些提法不妥的问题同时做了深刻的反思。

否定调查研究,一九六四年六月,刘少奇提出:“现在调查研究,按毛主席的办法不行了,现在的办法,只有放手发动群众,在改造客观世界的过程中认识客观世界。目前,我们这种工作方法,层层听汇报,或者看报表,不行了。这种领导工作、领导革命的方法不行了。我们现在这种靠会议、报表的领导方法,一定要亡国。”

大捧王光美,一九六三年十一月,刘少奇派遣王光美(化名“董朴”),以河北省公安厅秘书的名义到唐山专区抚宁县卢王庄公社桃园大队任“四清”工作组副组长,在刘的直接指示下创立了所谓的“桃园经验”。然后,王光美又在刘的安排授意下,在中共河北省委工作会议上作了“关于一个大队的社会主义教育运动的经验总结”的报告。接着刘少奇又安排王光美到各地作报告,传播他们夫妇共同树立的“桃园经验”,以此作为全国“四清”运动的样板。一九六四年八月十九日,刘少奇又以中共中央的名义拟了批语,九月一日,“桃园经验”便作为一个“有普遍意义”的样板由中共中央正式行文介绍到全国各级党政机关。
\mnitem{14}李富春、谭震林、李先念、薄一波,当时都任国务院副总理。
\mnitem{15}武昌会议,一九五八年十一月二十一日至二十七日在武昌举行的中国共产党政治局扩大会议,参加会议的有部分中央领导人和各省、市、自治区党委第一书记,会议上毛泽东同志对一线领导提出了严厉批评,批评了高指标、浮夸风、穷过渡、做假等现象,要求将指标调下来,但阻力很大。
\mnitem{16}北京开会,中共中央于一九五九年一月十二六日至二月二日,在北京召开了省、市、自治区党委书记会议,刘少奇主持出议,没有让毛泽东参加,制定了《关于一九五九年国民经济发展计划几个重要问题的说明》,中心内容是“钢产量达到二千万吨,是必须力争完成和超额完成的。”
\mnitem{17}项英与皖南事变,项英同志是党和红军早期的领导人之一,新四军的创建人和主要领导人之一,任军委新四军军分会书记,同时兼任新四军唯一的副军长,在实行党委负责制的新四军军中,项英实质是最高领导人。项英对于党中央早在一九三八年五月便已决定的新四军在执行中央向东向北发展的战略方针时,始终抱着将信将疑、既执行又打折扣的错误态度,而是总想实施他自己的向南发展的想法。

皖南事变前,毛泽东电令新四军在一九四〇年十二月底前完成北移,项英致电中共中央,表示还是有困难不能北渡,请求指示,毛泽东代中央书记处致电项英等人,严厉批评道:“全国没有任何一个地方像你们这样迟疑犹豫无办法无决心的。在移动中如遇国民党向你们攻击,你们要有自卫的准备和决心,这个方针也是早已指示你们了,我们不明了你们要我们指示何方针,究竟你们自己有没有方针,现在又提出拖还是走的问题,究竟你们主张的是什么,主张拖还是主张走,似此毫无定见,毫无方向,将来你们要吃大亏的!”

项英对新四军军部和皖南部队的转移未能抓住有利时机,在转移途中,犹豫动摇,处置失当,在战场上竟然开了七个小时的紧急会议,得不出结论,使新四军失去了突围的最后时机,最后决定部队重新由原路折回,改向西南前进,到达茂林后,又擅自行动,一度脱离军部,军长叶挺和副书记饶漱石联名发报给中央,报告项英等“率小部武装不告而去,行踪不明”。事变后,一九四〇年三月十四日凌晨,项英等十余人躲藏在蜜峰洞的山洞,由于携带作为新四军军费的金条,在夜里熟睡时被叛徒刘厚总杀害。一九五二年八月初,刘厚总在江西南昌被处决。

项英在皖南事变前后,对中央指示消极应对,犯了独立王国错误,对新四军在皖南事变中遭受严重损失负有重要责任。

彭德怀与百团大战,一九四〇年七月二十二日,主持八路军总部工作的彭德怀的准备发动百团大战的《战役预备命令》上报中央军委,出于全面考虑,毛泽东和中央军委一直没有批准,彭德怀这样说:“……未等到军委批准(这是不对的),就提早发起了战斗。”百团大战中,八路军共作战二千一百余次,歼日伪军五万余人。不过,八路军也付出了沉重代价,仅在前三个半月中即伤亡指战员一点七万余人,有二万余人中毒。更重要的是,百团大战过早地暴露了中共的军事力量,使日军和国民党顽固派加紧了对共产党的围剿,抗日根据地面积缩小,解放区人口由一亿下降到五千万,干部损失很多,使敌后抗战进入严重困难局面。
\mnitem{18}陆定一,原任中宣部部长、中央书记处书记。一九六六年五月二十四日,中央发出《关于陆定一同志和杨尚昆同志错误问题的说明》,错误如下:

“陆定一同志的妻子严慰冰,是现行反革命分子。现已查明,严慰冰在一九六〇年三月到一九六六年一月的六年期间,连续写了几十封反党反革命的匿名信,其中百分之九十是集中攻击和辱骂毛泽东同志最亲密的战友林彪同志和他的一家。这些信中充满了刻骨的反动阶级仇恨。

“大量的材料证明,陆定一同严慰冰的反革命案件是有密切牵连的。在严慰冰的卧室内书桌上放着一九六五年二月十五日写的一封匿名信的四页底稿和寄给叶群同志的信封,时间达一年零两个月之久,陆定一不可能不知道。而且严慰冰写反革命匿名信达六年之久,在写的时候,陆定一几乎全都在家。当陆定一被告之严慰冰犯了反革命罪行时,陆不仅不表示愤慨,还想诳说严患有神经偏执症为严开脱。”

“陆定一猖狂反对毛泽东思想,把活学活用毛泽东思想骂成‘实用主义’、‘庸俗化’、‘简单化’。”

“在文化革命的问题上,陆定一的立场和观点,是同彭真完全一致的,陆定一垄断中央宣传部的工作,打击左派,包庇右派”。

匿名信是赤裸裸的谩骂,其中有“搂了一个骚婆子,生了两个兔崽子。封官进爵升三级,终年四季怕光照。五官不正双眉倒,六神无主乱当朝。七孔生烟抽鸦片,拔(八)光了头上毛。机关算尽九头鸟,十殿阎罗把魂招。”

逼得林彪在政治局会议上散发证明:

我证明:(一)叶群和我结婚时是纯洁的处女,婚后一贯正派;(二)叶群与王实味、郭希佑根本没有恋爱过;(三)老虎、豆豆是我与叶群的亲生子女;(四)严慰冰的反革命信,所谈一切全是造谣。

林彪

一九六六年五月十四日

二〇〇六年出版的《王光美访谈录》,对严案也有提及:“严慰冰同志写匿名信这件事,我原来一点也不知道。叶群固然很坏,但我觉得严慰冰同志采取这种方式实在不好,有问题可以向组织上反映嘛!而且,她反对叶群可又要把这事往别人头上栽,这不是挑拨吗?她在有的匿名信上署名‘王光’,信里说‘咱俩是同学,谁也知道谁’,还把发信地址故意写作‘按院胡同’。按院胡同是我母亲办的洁如托儿所的地址。这不是有意让人以为写信人是王光美吗?我原先完全蒙在鼓里,好几年都不知道,一直到破案,才大吃一惊。”

杨尚昆,原任中央办公厅主任,一九六六年五月二十四日,中央发出《关于陆定一同志和杨尚昆同志错误问题的说明》,关于杨尚昆的问题,文件定性为:“(一)他不顾中央多次关于禁止安装窃听器的决定,背着中央,私设窃听器,私录毛主席和常委同志的讲话,盗窃党的机密。(二)他把大量的机密文件和档案擅自供给别人抄录,严重地泄露党的核心机密。(三)他同罗瑞卿等人的关系极不正常,积极参加了反党活动。(四)他还有其他严重错误。”

一九六一年四月,毛泽东乘坐专列到南下调查研究。毛泽东专列停在长沙车站,一名通信兵在站台上见到毛泽东机要秘书张玉凤,就模仿毛泽东口音开玩笑,而其所说正是毛泽东在车厢和张玉凤说的话。毛泽东立即盘问该通信兵,得知是罗瑞卿布置。而罗瑞卿则说是杨尚昆根据政治局会议决议要求他在车厢安装窃听器,以便政治局成员及时了解毛泽东的指示以便贯彻执行,罗瑞卿并出示了政治局的决议文件。

这样录音不经过毛泽东本人允许,连生活隐私录音都录下来,就是窃听。
\mnitem{19}陈奇涵(一八九七——一九八一),号圣涯,江西省兴国县人。中国人民解放军高级将领。一九五五年被授予上将军衔,荣获一级八一勋章、一级独立自由勋章、一级解放勋章。“七大”正式代表。
\mnitem{20}洛甫,指张闻天。
\mnitem{21}李宗仁,(一八九一年八月十三日——一九六九年一月三十日)字德邻,民国时期军事家、爱国将领。广西临桂(今桂林)人。黄埔军校南宁分校总负责人。国民党高级将领,中华民国副总统、代总统。一九六五年七月二十日,回国定居。
\mnitem{22}四类干部,《中国共产党中央委员会关于无产阶级文化大革命的决定》将干部分成四类,(一)好的。(二)比较好的。(三)有严重错误,但还不是反党反社会主义的右派分子。(四)少量的反党反社会主义的右派分子。四类干部是指反党反社会主义的右派分子。
\mnitem{22}刘建勋,时任河南省委第一书记。

一九六六年八月十九日,在毛泽东写出《炮打司令部——我的一张大字报》十三天后,中共河南省委第一书记刘建勋也写出了《我的一张大字报》,公开表态支持郑州大学造反学生党言川等“少数派”,并号召“炮打司令部”,一九六六年六月一日晚,中文系学生王相海等十四位学生贴出了《请问学校领导,对文化大革命究竟抱什么态度?》的大字报,批评校党委压制群众运动,单批学术权威,把郑大运动搞得“死气沉沉、冷冷清清,不像样子”。当天夜里,郑州大学党委召开会议准备组织反击。在全校性的反击开始中,王相海等少数学生被戴上“小邓拓”、“小吴晗”、“反革命”、“小右派”、“牛鬼蛇神”等帽子被围攻、批斗,六月六日凌晨,共产党员、团支部书记王相海跳楼自杀身亡。八月六日,中文系四年级学生党言川去北京,向毛主席、党中央反映河南和郑州大学文革情况,接着方复山、刘松盛二同学也去到北京。三人联名发表《致北京革命学生的公开信》,系统地向首都人民说明河南文化大革命存在的问题,同时向国务院接待站作了汇报。

一九六六年八月十四日下午,参加中共八届十一中全会的刘建勋在北京接见党言川等三位同学,明确地讲:“回到河南后,要在郑州召开一、二十万人的群众大会,号召大家炮打省委司令部,首先炮轰我这个司令官。”一九六六年八月十九日,在郑州大学全体师生员工大会,河南省委书记处候补书记兼省委秘书长纪登奎宣读了刘建勋的《我的一张大字报》。大字报中说:党言川等三位同学到北京向党中央、毛主席要求汇报郑大文化革命情况,这不仅不是什么非法行为,而是一种革命行动;他们回校后召开的“赴京情况汇报会”不是什么“黑会”,而是光明正大的革命大会;郑大部分同学认为“校文革”不能代表他们的意见,因而自动酝酿成立联委会,起来闹革命,我认为不能说是非法的……。大字报共十条,不仅旗帜鲜明地支持党言川等少数派,而且按照十六条精神号召群众“炮打省委司令部”。
\mnitem{24}二十三条,一九六四年底到一九六五年一月,中央政治局召集全国工作会议,在毛泽东的主持下讨论制定了《农村社会主义教育运动中目前提出的一些问题》(共“二十三条”),将“四清”的内容规定为清政治、清经济、清组织、清思想,强调这次运动的性质是解决“社会主义和资本主义的矛盾”,提出这次运动的重点是整“党内那些走资本主义道路的当权派。”
\mnitem{25}指八届十一中全会通过的《中国共产党中央委员会关于无产阶级文化大革命的决定》,简称十六条。
\end{maonote}
