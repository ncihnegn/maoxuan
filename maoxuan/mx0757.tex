
\title{改变国家体制,不设国家主席}
\date{一九七〇年三月、四月、七月}
\thanks{这是毛泽东同志对改变国家体制的重要批示。}
\maketitle


改变国家体制,不设国家主席\mnote{1}。

我不能再作此事,此议不妥\mnote{2}。

设国家主席,那是形式,不要因人设事\mnote{3}。

\begin{maonote}
\mnitem{1}这是一九七〇年的三月七日,毛泽东为召开四届“人大”并修改宪法提出的建议。

历史事实证明,毛泽东同志的考虑极富远见的:

党、政、军的权力主要来自宪法和党章。

建国后,毛泽东就开始构思国家的领导体制,一九四九年中国人民政治协商会议制定了《共同纲领》,这被看作是新中国的临时宪法。这部临时宪法主要规定了国家的政治经济制度、公民的权利义务等,也规定了组织国家政权的基本原则,但并没有规定国家机关的设置与职权(“在普选的全国人民代表大会召开以前,由中国人民政治协商会议的全体会议执行全国人民代表大会的职权,制定中华人民共和国中央人民政府组织法,选举中华人民共和国中央人民政府委员会,并付之以行使国家权力的职权。”(第十三条第二款)而对国家机构具体职权的规定是在这次政治协商会议上通过的《中央人民政府组织法》。可以说,《共同纲领》和这部组织法共同构成了新中国的临时宪法。其中组织法明确规定“中央人民政府委员会对外代表中华人民共和国,对内领导国家政权。”(第四条)这个委员会由中央人民政府主席一名、副主席六名、委员会五十六名组成。而组织法明确规定:“中央人民政府主席,主持中央人民政府委员会的会议,并领导中央人民政府委员会的工作。”这意味着“中央人民政府主席”虽属于中央人民政府委员会,但又具有相对的独立性,并居于领导地位。在第一届全国政治协商会议第一次全体会议上,毛泽东当选中央人民政府主席和中央革命军事委员会主席,“中华人民共和国建立统一的军队,即人民解放军和人民公安部队,受中央人民政府人民革命军事委员会统率,实行统一的指挥,统一的制度,统一的编制,统一的纪律。”(第二十条),同时,他还是中国共产党的中央委员会主席。党、政、军权集中于毛泽东一身,形成了“三位一体”的主席制的雏形。

一九五四年宪法是新中国第一部正式宪法,毛泽东对这部宪法的起草倾注了很大心血,完全可以称之为“毛泽东宪法”。在国家领导体制上,毛泽东对当时的苏联宪法和一九四九年临时宪法所确立的领导体制作了根本性改革,即创设了一九三六年苏联宪法和建国初期临时宪法中不存在的一个全新领导职位:“国家主席”。

一九五四年宪法规定中华人民共和国主席有下列职权:

“第四十条、中华人民共和国主席根据全国人民代表大会的决定和全国人民代表大会常务委员会的决定,公布法律和法令,任免国务院总理、副总理、各部部长、各委员会主任、秘书长,任免国防委员会副主席、委员,授予国家的勋章和荣誉称号,发布大赦令和特赦令,发布戒严令,宣布战争状态,发布动员令。

第四十一条、中华人民共和国主席对外代表中华人民共和国,接受外国使节;根据全国人民代表大会常务委员会的决定,派遣和召回驻外全权代表,批准同外国缔结的条约。

第四十二条、中华人民共和国主席统率全国武装力量,担任国防委员会主席。

第四十三条、中华人民共和国主席在必要的时候召开最高国务会议,并担任最高国务会议主席。最高国务会议由中华人民共和国副主席、全国人民代表大会常务委员会委员长、国务院总理和其他有关人员参加。最高国务会议对于国家重大事务的意见,由中华人民共和国主席提交全国人民代表大会、全国人民代表大会常务委员会、国务院或者其他有关部门讨论并作出决定。”

由此来看,五四年宪法规定的国家主席不是一个虚权元首,而是实权元首,在法律上不仅是最高军事统帅,国家元首,而且是政府首脑,人大委员长和总理都直接受其领导。一九五四年九月二十七日,中华人民共和国第一届全国人大召开第一次全体会议,毛泽东以其在党内外、国内外独一无二的历史地位和无人可比的崇高威望当选国家主席。

一九五四年九月二十八日,中共中央政治局决定在中央政治局和中央书记处之下成立中共中央军事委员会,领导中国人民解放军和其他武装力量。决议指出:“中央政治局认为,必须同过去一样在中央政治局和书记处之下成立一个党的军事委员会,来担负整个军事工作的领导”,根据七大党章第三十四条,“由中央委员会全体会议选举中央政治局与中央书记处,并选举中央委员会主席一人。中央政治局,在中央委员会前后两届全体会议期间,党的中央指导机关,指导党的一切工作。中央书记处在中央政治局决议之下处理中央日常工作。中央委员会主席即为中央政治局主席与中央书记处主席。中央委员会依工作需要,设组织、宣传等部与军事、党报等委员会及其他工作机关,分别办理中央各项工作,受中央政治局、中央书记处及中央主席之指导监督”,政治局主席毛泽东自然担任中共中央军事委员会主席。

中华人民共和国主席和中国共产党中央委员会主席职位集于一身,使党政军“三位一体”体制正式建立。从此,毛泽东大量重要的政治活动就通过国家主席和最高国务会议进行的,在从一九五四年到一九五八年,毛泽东召开过大约十六次最高国务会议,相反,中共中央的会议远远少于国务会议。

八大党章也确保了中央委员会主席即为政治局主席,自然也是政治局下设的中国共产党军事委员会主席。一九五六年九月二十六日,八大通过的党章第三十七条规定:“党的中央委员会全体会议选举中央政治局、中央政治局的常务委员会和中央书记处,并且选举中央委员会主席一人、副主席若干人和总书记一人。中央政治局和它的常务委员会在中央委员会全体会议闭会期间,行使中央委员会的职权。中央书记处在中央政治局和它的常务委员会领导之下,处理中央日常工作。中央委员会的主席和副主席同时是中央政治局的主席和副主席。中央委员会认为有必要的时候,可以设立中央委员会名誉主席一人。”

然而,这个“三位一体”的体制由于一九五九年毛泽东辞去国家主席,而选举刘少奇出任国家主席第一次发生了分裂,即按照八大党章,毛泽东是党主席和中共中央军事委员会主席,拥有最高的政治主权,而按照宪法,刘少奇是国家主席和国防委员会主席,拥有最高的法律主权。如果二者在党内合作顺利,那么,两个主权合一就能维持三位一体体制,否则,中国的宪政体制就面临着分裂。不幸的是,由于毛泽东和刘少奇在政治路线上的分歧,导致中国宪法体制的危机。而事实也证明了这一点,名不正则言不顺,毛泽东在退居二线不再主持中央日常工作后,不在一线,对大量事情很难直接插手,出现了大量违背毛泽东意愿的做法,如大跃进期间的浮夸等“五风”错误,毛泽东虽然一而再,再而三地强烈批评一线领导,甚至以党内通信的方式要求“根本不要管上级规定的那一套指标。不管这些,只管现实可能性”《毛泽东致六级干部的公开信一九五九年四月二十九日》,但是效果甚微;一线领导在庐山会议上想彻底打倒彭德怀;三年困难时期,一线领导又想下马原子弹这样的战略项目;六二年刘邓等一线领导又想搞包产到户,走小农经济道路;教育、文艺、医疗等长期不为人民服务而为官老爷服务;在“四清”运动中不是针对干部(见《前十条》)而是把整肃的矛头指向群众(见《后十条》);文化大革命初期派工作组镇压学生运动等等。一线二线不统一,毛泽东深受其苦,文革开始后就果断结束了一线二线。

文革中,毛泽东逐渐形成了这样的理念,“中国共产党是全中国人民的领导核心。没有这样一个核心,社会主义事业就不能胜利。党、政、军、民、学,东、西、南、北、中,党是领导一切的”,因此必须实现党的一元化领导,从比较宪法的角度看,实权元首在国家政治生活中居于最高地位,甚至是唯一的最高地位,当属天经地义。

九大后,毛泽东考虑改变国家宪政体制,党主席和国家主席的二元化,国家主席和国家总理的二元化。存在带来国家不稳定的内在隐患。虽然党主席可以兼任国家主席,但是国家主席和国务院总理的二元化问题依然存在,如果党主席不是最高国家元首,就会面临宪政分裂,如毛泽东决定退居二线后,刘少奇任国家主席,但毛泽东仍然是党的军委主席,由于没有修改宪法,这样就在理论上产生了二元化的问题,与“党指挥枪”的原则产生矛盾,尽管实际上这段时间还是坚持党对军队的绝对领导,毛泽东仍然是武装力量的最高领导,但终究是一个宪政弊端。如果党主席不担任国家主席,那么,任何人担任国家主席都有可能让宪法体制陷入分裂,因此毛泽东经过长期思考后,提出在“改变国家体制,不设国家主席”。这个思想最后体现到一九七五年的宪法了,七五年宪法规定:

“第十五条中国人民解放军和民兵是中国共产党领导的工农子弟兵,是各族人民的武装力量。中国共产党中央委员会主席统率全国武装力量。

第十六条全国人民代表大会是在中国共产党领导下的最高国家权力机关。

第十七条全国人民代表大会的职权是:修改宪法,制定法律,根据中国共产党中央委员会的提议任免国务院总理和国务院的组成人员,批准国民经济计划、国家的预算和决算,以及全国人民代表大会认为应当由它行使的其他职权。

第十八条全国人民代表大会常务委员会是全国人民代表大会的常设机关。它的职权是:召集全国人民代表大会会议,解释法律,制定法令,派遣和召回驻外全权代表,接受外国使节,批准和废除同外国缔结的条约,以及全国人民代表大会授予的其他职权。

第二十条国务院的职权是:根据宪法、法律和法令,规定行政措施,发布决议和命令;统一领导各部、各委员会和全国地方各级国家机关的工作;制定和执行国民经济计划和国家预算;管理国家行政事务;全国人民代表大会和它的常务委员会授予的其他职权。”

依据七五宪法,中共中央委员会主席统帅全国武装力量,中共中央委员会主席提名国务院总理,彻底实现了党的一元化领导,彻底解决了宪政分裂的隐患。

应当指出,文革结束后,翻案后的走资派在否定了文革以后,邓小平为了非毛和垂帘听政的需要,废除了党章中的“中共中央委员会主席制”,实行“中共中央委员会总书记”制(与中共中央军委主席分离),恢复了宪法中的“国家主席”职位,同时设“中华人民共和国中央军事委员会主席”职位,终于造出了和原来同名但不同权的“国家主席”和“总书记”。

根据一九八二年九月六日十二大通过的党章:

“第二十一条党的中央政治局、中央政治局常务委员会、中央书记处和中央委员会总书记,由中央委员会全体会议选举。中央委员会总书记必须从中央政治局常务委员会委员中产生。中央政治局和它的常务委员会在中央委员会全体会议闭会期间,行使中央委员会的职权。中央书记处在中央政治局和它的常务委员会领导下,处理中央日常工作。中央委员会总书记负责召集中央政治局会议和中央政治局常务委员会会议,并主持中央书记处的工作。党的中央军事委员会组成人员由中央委员会决定。中央军事委员会主席,必须从中央政治局常务委员会委员中产生。”

这样,就废除了党主席制,恢复了党总书记(邓小平在五六十年代曾担任过十年书记处总书记,但当时的总书记只是书记处的总书记,不是中央委员会的总书记)制,同时实行中共中央军事委员会主席与党的总书记分离制。由此形成党权与军权的分离。一九八二年九月十三日,在中共十二届一中全会上,邓小平任中央政治局常委兼任中共中央军委主席。

另外,邓小平又搞了一部一九八二年宪法,又重新设立了国家主席,但此“国家主席”已非彼“国家主席”,由一九五四年的实权元首,变成了虚权元首,宪法中又设立了独立的中央军事委员会。一九八二年十二月四日宪法规定:

“第八十条中华人民共和国主席根据全国人民代表大会的决定和全国人民代表大会常务委员会的决定,公布法律,任免国务院总理、副总理、国务委员、各部部长、各委员会主任、审计长、秘书长,授予国家的勋章和荣誉称号,发布特赦令,发布戒严令,宣布战争状态,发布动员令。

第八十一条中华人民共和国主席代表中华人民共和国,接受外国使节;根据全国人民代表大会常务委员会的决定,派遣和召回驻外全权代表,批准和废除同外国缔结的条约和重要协定。

第九十三条中华人民共和国中央军事委员会领导全国武装力量。”

这样,此时的国家主席毫无实权,既无军权又无行政权,完全是一个虚位元首。国家军委则领导全国武装力量。完全使国家元首与行政权、军权分离。

更令人气愤地是,邓小平为了继续掌控军权,在十三大时通过的党章修正案中规定:

第二十一条第五段“党的中央军事委员会组成人员由中央委员会决定。中央军事委员会主席,必须从中央政治局常务委员会委员中产生。”改为:“党的中央军事委员会组成人员由中央委员会决定。”就是说,普通党员也可以担任中共中央军事委员会主席。

经过党章和宪法的修正后,党权、行政权、军权全面分离,新设立的中华人民共和国中央军事委员会和中国共产党中央军事委员严重冲突,不得不采用“两个牌子,一套人马”的办法,但即使这样,因为两个委员会的任期不一样,也会出现两套人马的情况。这一阶段国家中央军事委员会主席独立于国家主席,直接由全国人民代表大会选举产生,任期每届五年,需要注意的是国家主席、副主席、国务院总理、副总理及人大常委会委员长等都有连续任职不得超过两届的规定,而国家中央军事委员会主席则没有限定。就宪法文本来看,现行宪法规定国家主席有连任不得超过两届,而国家军事委员会主席没有连任的限制,就产生了国家主席与国家军事委员会主席非一元化的可能。由于军事领导机关强调权力和责任的高度集中,同时又实行首长负责制,当出现非一元化的情形时,从中国以往的历史经验看,可能会出现危及国家安定的风险。从春秋战国时期诸侯割据,到南北朝时期的诸国林立,再到民国时期的军阀混战,中华民族承受了太多战乱,所以军权的统一关系到国家的长治久安。事实上在当代,各国武装力量的统帅权大多由国家元首享有,比如,在法国,总统是军队的最高军事统帅;在美国,总统除享有国家武装力量的统率外作为三军总司令,还享有较大的战争权。

后来的历史事实也证明了,邓小平采用的这种分权模式,属于典型的因人设事,除了加强了他个人的权力外,没有给国家带来任何好处,两届总书记胡耀邦和赵紫阳都被逼下台,且酿成了“六四”镇压学生事件。

“六四”后,迫于党内外强大的压力,一九九〇年邓小平辞去中央军委主席,党的总书记江泽民担任中央军委主席,一九九三年,江泽民又顺利地出任国家主席,实现了党权、军权和国家元首的统一。二〇〇二年,胡锦涛担任党的总书记,二〇〇三年胡锦涛又担任国家主席,二〇〇四年,胡锦涛又担任中共中央军委主席,二〇〇四年九月二十日,江泽民和胡锦涛在出席军委扩大会议时,江泽东同志指出“锦涛同志是党的总书记、国家主席,接任军委主席的职务顺理成章。党的总书记、国家主席、军委主席三位一体这样的领导体制和领导形式,对我们这样一个大党、大国来说,不仅是必要的,而且是最妥当的办法”,至此“三位一体”的体制又建立起来了。

但这却依靠领导人的自觉,没有制度的保证。像这种多个职务集于一身“多位一体”才能保证实权元首的体制,尤其是两个军委主席(国家军委和中共军委)的职位还是有些问题,如二〇〇四年九月,中国共产党中央军事委员会主席胡锦涛授予张定发及靖志远上将军衔。由于当时中华人民共和国中央军事委员会主席为江泽民,胡锦涛以中共中央军委主席授衔是违宪的,直到二〇〇五年三月的全国人大上,胡锦涛才接替江泽民担任国家军委主席。

所以,总体来看,既然宪法规定,“中国共产党是全中国人民的领导核心。工人阶级经过自己的先锋队中国共产党实现对国家的领导。”,党对国家机构的一元化领导这个问题,七五年宪法是解决得最彻底的。
\mnitem{2}一九七〇年四月十二日,毛泽东在林彪的建议上写的批示,林彪建议如下:

一、关于这次“人大”国家主席的问题,林彪同志仍然建议由毛主席兼任。这样做,对党内、党外、国内、国外人民的心理状态适合。否则,不适合人民的心理状态;

二、关于副主席问题,林彪同志认为,可设可不设。可多设,可少设。关系都不大。

三、林彪同志认为,他自己不宜担任副主席的职务。
\mnitem{3}一九七〇年七月十八日,在中央修改宪法的起草委员会开会期间,针对有人提出“如果毛主席当国家主席就设国家主席”的意见的批示。
\end{maonote}
