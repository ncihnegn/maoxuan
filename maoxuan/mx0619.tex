
\title{在第二次郑州会议上的讲话}
\date{一九五九年二月二十七日、二十八日、三月一日、三月五日}
\thanks{这是毛泽东同志在一九五九年二月二十七日至三月五日在郑州召开的中共中央政治局扩大会议(即第二次郑州会议)上的多次讲话节选。}
\maketitle


\date{一九五九年二月二十七日}
\section{一、劳动人民的劳动成果怎么能无偿占有呢}

一九五八年,我们在各个战线上取得了伟大的成绩,不论在思想政治战线上,工业战线上,农业战线上,交通运输战线上,商业战线上,文教战线上,国防战线上,以及其他方面。都是如此。特别显着的。是工业和农业生产方面有了一个伟大的跃进。一九五八年,在全国农村中普遍建立了人民公社。

人民公社的建立使农村中原来的生产资料集体所有制扩大和提高了,并且开始带有若干全民所有制的成份,人民公社的规模比农业生产合作社大得多,并且实行了工农商学兵,农林牧副渔的结合,这就有力的促进了农业生产和整个农村经济的发展,广大的农民。尤其是贫农和下中农。对于人民公社表现了热烈的欢迎。广大干部在人民公社运动中做了大量的有益的工作。他们表现了作为一个共产主义者的极大的积极性,这是非常宝贵的,没有他们这种积极性。要取得这样伟大的成绩是不可能的。当然,我们的工作中,不但有伟大的成绩。也有一些缺点。在一个新的、像人民公社这样缺乏经验的前无古人的几亿人民的社会运动中。人民和他们的领导者们都只能从他们的实践中逐步取得经验,对事物的本质逐步加深他们的认识,揭露事物的矛盾,解决这些矛盾,肯定工作中的成绩,克服工作中的缺点,谁要说一个广大的社会运动能够完全没有缺点,那他不过就是一个空想家,或者是一个观潮派算账派,或者简直是敌对分子。我们的成绩和缺点的关系,正是我们所常说的,只是十个指头中九个指头和一个指头的关系。有些人怀疑或者否认一九五八年的大跃进。怀疑或者否认人民公社的优越性,这种观点显然是完全错误的。

人民公社现在正在进行整顿巩固工作,就是说整社,已经或者正在辩论一九五八年有无大跃进和人民公社有无优越性两个问题。各级党委正在整社工作中,按着六中全会的方针,采取了首先肯定大跃进的成绩,肯定人民公社的优越性,然后才能指出工作中的缺点错误这种次序,这种作法是完全恰当的。这样作,可以保护广大干部和群众的积极性。就干部来说,百分之九十几都是好的。都是应当加以坚决保护的。

现在我来说一点人民公社的问题。我认为人民公社现在有一个矛盾,一个可以说相当严重的矛盾,还没有被许多同志所认识,它的性质还没有被揭露,因而还没有被解决。而这个矛盾我认为必须迅速地加以解决,才有利于调动广大人民群众更高的积极性,才有利于改善我们和基层干部的关系,这主要是县委、公社党委和基层干部之间的关系。

究竟是什么样的一种矛盾呢?大家看到,目前我们跟农民的关系在一些事情上存在着一种相当紧张的状态,突出的现象是在一九五八年农业大丰收以后,粮食、棉花、油料等等农产品的收购至今还有一部分没有完成任务。再则全国,除少数灾区外,几乎普遍地发生瞒产私分,大闹粮食、油料、猪肉、蔬菜“不足”的风潮,其规模之大,较之一九五三年和一九五五年那两次粮食风潮都有过之无不及。同志们,请你们想一想,这究竟是什么一回事呢?我认为,我们应当透过这种现象看出问题的本质即主要矛盾在什么地方。这里面有几方面的原因,但是我以为主要地应当从我们对农村人民公社所有制的认识和我们所采取的政策方面去寻找答案。

农村人民公社所有制要不要有一个发展过程?是不是公社一成立,马上就有了完全的公社所有制,马上就可以消灭生产队的所有制呢?我这是说的生产队,有些地方是生产大队即管理区,总之大体上相当于原来的农业生产合作社。现在有许多人还不认识公社所有制必须有一个发展过程,在公社内,由队的小集体所有制到社的大集体所有制,需要一个过程,这个过程要有几年时间才能完成。他们误认人民公社一成立,各生产队的生产资料、人力、产品,就都可以由公社领导机关直接支配。他们误认社会主义为共产主义,误认按劳分配为按需分配,误认集体所有制为全民所有制。他们在许多地方否认价值法则,否认等价交换。因此,他们在公社范围内,实行贫富拉平,平均分配,对生产队的某些财产无代价地上调,银行方面也把许多农村中的贷款一律收回。一平、二调、三收款,引起广大农民的很大恐慌。这就是我们目前同农民关系中的一个最根本的问题。

六中全会的决议\mnote{1}写明了集体所有制过渡到全民所有制和社会主义过渡到共产主义所必须经过的发展阶段。但是没有写明公社的集体所有制也需要有一个发展过程,这是一个缺点。因为那时我们还不认识这个问题。这样,下面的同志也就把公社、生产大队、生产队三级所有制之间的区别模糊了,实际上否认了目前还存在于公社中并且具有极大重要性的生产队(或者生产大队,大体上相当于原来的高级社)的所有制,而这就不可避免要引起广大农民的坚决抵抗。从一九五八年秋收以后全国性的粮食、油料、猪肉、蔬菜“不足”的风潮,就是这种反抗的一个集中表现。一方面,中央、省、地、县、社五级(如果加上管理区就是六级)党委大批评生产队、生产小队的本位主义,瞒产私分;另一方面,生产队、生产小队却几乎普遍地瞒产私分,甚至深藏密窖,站岗放哨,以保卫他们的产品。我以为,产品本来有余,应该向国家交售而不交售的这种本位主义确实是有的,犯本位主义的党员干部是应该受到批评的,但是有很多情况并不能称之为本位主义。即令本位主义属实,应该加以批评,在实行这种批评之前,我们也必须首先检查和纠正自己的两种倾向,即平均主义倾向和过分集中倾向。所谓平均主义倾向,即是否认各个生产队和各个个人的收入应当有所差别。而否认这种差别,就是否认按劳分配、多劳多得的社会主义原则。所谓过分集中倾向,即否认生产队的所有制,否认生产队应有的权利,任意把生产队的财产上调到公社来。同时,许多公社和县从生产队抽取的积累太多,公社的管理费又包括很大的浪费,例如有一些大社竟有成千工作人员不劳而食或半劳而食,甚至还有脱产文工团。上述两种倾向,都包含有否认价值法则、否认等价交换的思想在内,这当然是不对的。凡此一切,都不能不引起各生产队和广大社员的不满。

目前我们的任务,就是要向广大干部讲清道理,经过充分的酝酿和讨论,使他们得到真正的了解,然后我们和他们一起,共同妥善地坚决地纠正这些倾向,克服平均主义,改变权力、财力、人力过分集中于公社一级的状态。公社在统一决定分配的时候,要承认队和队、社员和社员的收入有合理的差别,穷队和富队的伙食和工资应当有所不同。工资应当实行死级活评。公社应当实行权力下放,三级管理,三级核算,并且以队的核算为基础。在社与队、队与队之间要实行等价交换。公社的积累应当适合情况,不要太高。必须坚决克服公社管理中的浪费现象。只有这样,我们才能够有效地去克服那种确实存在于一部分人中的本位主义,巩固公社制度。这样做了以后,公社一级的权力并不是很小,仍然是相当大的;公社一级领导机关并不是没有事做,仍然有很多事做,并且要用很大的努力才能把事情做好。

公社在一九五八年秋季成立之后,刮起了一阵“共产风”。主要内容有三条:一是穷富拉平。二是积累太多,义务劳动太多。三是“共”各种“产”。所谓“共”各种“产”,其中有各种不同情况。有些是应当归社的,如大部分自留地。有些是不得不借用的,如公社公共事业所需要的部分房屋、桌椅板凳和食堂所需要的刀锅碗筷等。有些是不应当归社而归了社的,如鸡鸭和部分的猪归社而未作价。这样一来,“共产风”就刮起来了。即是说,在某种范围内,实际上造成了一部分无偿占有别人劳动成果的情况。当然,这里面不包括公共积累、集体福利、经全体社员同意和上级党组织批准的某些统一分配办法,如粮食供给制等,这些都不属于无偿占有性质。无偿占有别人劳动的情况,是我们所不许可的。看看我们的历史吧。我们只是无偿剥夺了日德意帝国主义的、封建主义的、官僚资本主义的生产资料,和地主的一部分房屋、粮食等生活资料。所有这些都不是侵占别人劳动成果,因为这些被剥夺的人都是不劳而获的。对于民族资产阶级的生产资料,我们没有采取无偿剥夺的办法,而是实行赎买政策。因为他们虽然是剥削者,但是他们曾经是民主革命的同盟者,现在又不反对社会主义改造。我们采取赎买政策,就使我们在政治上获得主动,经济上也有利。同志们,我们对于剥削阶级的政策尚且是如此,那末,我们对于劳动人民的劳动成果,又怎么可以无偿占有呢?

\date{一九五九年二月二十八日}
\section{二、公社不允许有脱产的文工团}

劳动分配,现在极为不合理,农业(包括农、林、牧、副、渔)分配太少,而工业,行政人员和服务行业的人员太多(有的多到百分之三十到百分之四十),必须坚决的减下来。中国从张之洞办工业以来产业工人只有四百万,解放以来平均每年增长一百万,即八百万,共一千二百万,而去年一年增了二千六百万,再加上各行各业转过来转过去的四百万,共为三千万,突然增加三千万,一则一喜,一则一忧。上面这三部分人,都有大批浪费,必须坚决减下来,从事农林牧副渔,否则有危险。据说工业浪费百分之二十,要回农村,服务行业要大减,行政人员只许有千分之几。公社不允许有脱产的文工团。

\date{一九五九年三月一日}
\section{三、凡是劳动总要等价交换}

要提高农民的生产积极性,改善政府与农民的关系,必须从改变所有制着手。现在一平、二调、三提款,否定按劳分配,否定等价交换。价值法则,等价交换不仅存在公社内部,也存在于集体所有制与全民所有制之间,实际上生产资料各部门之间也有价值法则起作用。现在就是一平、二调、三提款,提起就走,一张条子要啥调啥,不给钱是起破坏作用。现在银行不投资农业,我建议每年增加十亿,十年搞一百亿无利长期贷款,主要支援贫队,一部购买大型农具,十年之后国有化了,就变为国家投资了,忽然一股风,一平、二调、三提款,破坏经济秩序,许多产品归社不归队。六中全会公社决议的一套制度,二个半月来根本没有实行。问题不这样提,共产风会继续发展。为什么六中全会的决议没有阻止这股风的发展?是不是只有冀、鲁、豫三省?是不是南方各省道德特别高尚,马克思主义多?我就不相信。

我提议请你们开一个六级干部会,找一批算账派参加。共产党就是反反复复。

十二句话应再加两句价值法则,等价交换。统一领导,队为基础,分级管理,权力下放;三级核算,各计盈亏,适当积累,合理调剂,收入分配,由社决定,多劳多得,承认差别,价值法则、等价交换。不解决这个问题,大跃进就没有了。我这篇话不讲,就不足以掀起议论,这几个月许多地方实际上破坏了价值法则。去年郑州会议,就吵这个问题,拉死人来压活人。凡是瞒产私分者,一定都是一平、二调、三提款(引起的)。农民从十月以来,发生大恐慌,怕共产,从桌、椅、板凳开始,还有个工业抗旱,破钢烂铁,无代价献宝。这在战时是可以的,无代价或者很少代价。战时只给饭吃,不给代价。这也不是长期的,否则也会破坏生产。

今年你们要节制,尽最少放“卫星”,如体育卫星、诗歌卫星、银行卫星等。

共产主义是不是推迟了?早已推迟了,六中全会决议讲了十年到二十年,还有五个条件没有完成。现在有些同志在这个问题上还是想早一点,我看越想搞越搞不成,越慢一点,越可以快。用“无偿”来搞共产主义不行。猪只有一条,你有他就没有了。凡是劳动,总要等价交换的。

\date{一九五九年三月五日}
\section{四、我代表一千万队长级干部五亿农民说话}

放一大炮是否灵,放对了没有?

要拿王国藩穷棒子社对穷户、穷队、穷社,解决穷社、穷队、穷户问题。一是贷款,二是公共积累。国家每年拿出十亿解决这一问题,社工业少办,主要是解决这问题。共产主义没有饭吃,天天搞共产,实际是“抢产”,向富队共产。旧社会谓之贼,红帮为抢,青帮叫偷,对下面不要去讲抢,抢和偷的科学名词叫做无偿占有别人的劳动。地主叫超经济剥削,资本家叫剩余劳动,也就是剩余价值。我们不是要推翻地主、资本家吗?富队里有富人,吃饭不要钱就侵占了一部分,这个问题要想办法解决,一平、二调、三收款,就是根本否定价值法则和等价交换,是不能持久的。过去汉族同少数民族是不等价交换,剥削他们,那时不等价还出了一点价,现在一点价也不给,有一点就拿走,这是个大事,民心不安,军心也就不安,甚至征购粮款也被公社拿走,国家出了钱,公社拦腰就抢。这些人为什么这样不聪明呢?他们的政治水平那里去了。问题是省、地、县委没有教育他们。整社三个月没有整到痛处,隔靴抓痒,在武昌会议时,不感到这个问题,回到北京感到了,睡不着觉,九月就充分暴露了,大丰收!国家征购粮(却)完不成,城市油吃不到了。赵紫阳的报告\mnote{2}和内部参考中的材料你们看到没有?我就不相信长江、珠江流域马克思主义就那样多?我抓住赵紫阳把陶铸\mnote{3}的辫子抓到了。瞒产私分很久了,开始在襄阳发现,刘子厚\mnote{4}谈话对我有很大启发,河北一月开党代会,开始搞共产主义,倾向于一曰大、二曰公,二月十三日就感到有问题,决心改变主意,但还没有接触到所有制问题。到山东谈了吕洪宾合作社\mnote{5},开条子调东西调不动,就让许多人拿秤去秤粮食,群众普遍抵制,于是翻箱倒柜;进而进行神经战,一顶帽子“本位主义”一框,你框农民就看出你没有办法了,他也不在乎,这一着神经战也不灵,一张条子,一把秤,一顶帽子三不灵后才受到了教育,才用一把钥匙,解决思想问题,但也没有接触到所有制,河南说“虽有本位主义情有可原,不予处分,不再上调”,安徽说“错是错了,但不算错”。什么叫情?情者情况也,等价交换也,不是人家本位主义,而是我们上级犯了冒险主义,翻箱倒柜,“一平、二调、三收款”,一张条子,一把秤,一顶帽子,这是什么主义?人往高处走,水往低处流,“老弱转乎沟壑,壮者散而四方”,哪里有钱就往哪里跑。你不等价交换,人家人财两空,吕鸿宾改变主意,一张安民布告,一个楼梯下楼,要下楼,首先要下楼的是我们,就是解决所有制问题。

土地属谁所有,劳动力属谁所有,产品就属谁所有。农民历来知道土地是搬不走的,不怕,但劳动力、产品是可以搬得走的,这就怕了。拿共产主义的招牌,实际实行抢产,如不愿不等价交换,就叫没有共产主义风格,什么叫共产主义,还不是公开抢?没有钱嘛!不是抢是什么?什么叫一曰大、二曰公?一曰大是指地多,二曰公是指自留地归公。现在什么公?猪、鸭、鸡、萝卜、白菜都归公了,这样调人都跑了。河北定县一个公社有七、八万人,二、三万个劳动力,跑掉一万多。这样的共产主义政策,人都走光了。劳动力走掉根本原因是什么,要研究。

整了三个月社,只做了一些改良主义工作,修修补补,办好公共食堂,睡好觉,一个楼梯,一张布告之类,但未搞出根本性办法。要承认三级所有制,重点是生产队所有制,“有人斯有土,有土斯有财”,所有人、土、财都在生产队,五亿农民都在生产队,上面只有几个工作人员。如不承认所有制,就立即破坏。我是事后诸葛亮,以前还未看到这个问题。在批转赵紫阳的报告\mnote{6}(时),就有此思想。六中全会有好处,农民不怕中央了,认为中央好讲价钱,中央雇工是拿钱的,购粮油是拿钱的,征购不多,注意生活福利,八小时工作等。仇恨集中在公社,第二在县,县也调了些人,调了些东西,县、社办那么多事干啥?所以,要对公社同志讲清楚,公社不要搞太多,十大任务\mnote{7}做不完。你们有经验,你们过去不是骂中央统死统多吗?现在你们当了婆婆就打媳妇,就忘记了。现在中央已经改了。去年权力下放,说了不算,拿出一张表来你们才放心。现在你们领导之下的公社,就实行“一平、二调,三收款”,调,一曰物、二曰人。当然出卖劳动力,不是出卖给资本家,而是出卖给中央、省、县、公社,但也要等价交换。过去长沙建筑工人罢工,我们叫增加工资,他们叫涨价,那是一九二一年的事,到现在三十八年了,我们还不懂涨价这个道理吗?劳动力到处流动,磨洋工,对这点我甚为欣赏,放一炮,瞒产私分,劳动力外逃,磨洋工,这是在座诸公政策错误的结果。上千万队长级的干部很坚决,几万万社员拥护他们的领袖,所以立即下决心瞒产私分。我们许多政策引起他们下决心这样做,这是合法的。我们领导是没有群众支持的。当然也包括桌椅板凳,刀锅碗筷,去年工业抗旱,大闹钢铁,献工献料,什么代价也没有。此外,还要拿人工,专业队都要青年,还有文工团都是青年,队长实在痛心,生产队稀稀拉拉。这样下去一定垮台,垮了也好,垮了再建。无非是天下大笑。我代表一千万队长级干部,五亿农民说话,坚持搞右倾机会主义,贯彻到底,你们不跟我来贯彻,我一人贯彻,直到开除党籍,也要到马克思那里告状。

严格按照价值法则,等价交换办事。三级所有制,改变为基本公社所有制部分队所有制,要有一个过程,还要三、五、七年。要穷队赶上来,穷队变富队,穷变富每个省都可以找到例子,像王国藩那样,最大的希望是穷队,不能把苏联的钢砍给我们二千万吨,如果这样,苏联也好造反,世界上的事没有不交换的,人同自然界作斗争,也有交换,如人吃东西,吸空气,新陈代谢。重工业各部门之间也要等价交换。夏热冬寒,一切都等价交换。国家给钱,就是公社不给钱。犯了个大错误。六中全会的东西现在有许多没有执行,就是否定价值法则,所谓拥护中央是句空话,起码暂时还难说,其实是不通。无代价的上调是违反中央的,要搞工业,不搞农业,未到期的贷款都收回了,是不是中央不两条腿走路?相反,今年要增加十亿,一部分是可以收的,贫农贷款是四年,六十年才到期,现在就收回了。我看这可以给人民银行行长戴一顶帽子,叫做破坏农业生产,破坏人民公社,也不撤职。全部退回,到期不到期的都退,你们可以打个折扣,到期的可以不退。我为了对付你的全部收回,我就来个全部退回,你要左倾,我要右倾,就是到期还可以延长。

你们认为怎样才能巩固人民公社?一平、二调、三收款,还是改变?我看这样下去公社非垮台不可。斯大林为什么改变公社的办法?他们觉得浪费太多,义务交售制,余粮征集制不能刺激生产,才改为粮食税。斯大林三十年之久实际没有实行集体所有制,还是地主超经济剥削,拿走农民的百分之七十,因此,三十年还是只能进行单纯的再生产。俄皇时代,无机械化和集体所有制。斯大林搞了这两点,粮食产量和沙皇时代相等。那时可能是为了搞重工业,留的只够农民吃,无力扩大再生产。当然不是斯大林一个人的问题,而是有一批人热心于搞重工业、搞共产主义。我们是办公社工业,如果这样搞下去,非搞翻农民不可。任何大跃进、中跃进、小跃进也不可能,生产就会停滞。

搞三、五、七年,来个过程,基本上以原来的高级社为基础,等价交换,不能乱开条子。队与队是买卖关系,若干调剂要协商。灾队、穷队没有饭吃由省解决。

一个是瞒产私分,一个是劳动力外逃,一个是磨洋工,一个是粮食伸手向上要,白天吃萝卜,晚上吃好的,我很赞成,这样做非常正确。你不等价交换,我就坚决抵制,河南分配给农民百分之三十,瞒产私分百分之十五,共百分之四十五,否则就过不了生活,这是保卫他们的神圣权利,极为正确。还反对人家本位主义?相反应该批评我们的冒险主义。真正本位主义,只有一部分,主要是冒险主义。钱交给公社不交队,他们抵制,这不叫本位主义。给他钱他不缴,才是本位主义。

大问题是把六级干部会开好,公社党委来一个书记,管理区来二人,生产队来二人,都要一穷一富。河南简报要看两遍,这是现场会议。对穷队要讲王国藩。河北省遵化县鸡鸣村区,穷棒子王国藩社现在是一个大社,很富了。开始只有二十三人,三条驴腿,无车无粮。他的章程就是不要国家贷款,不要救济,砍柴卖,从此出了名,变为几十户,几百户。现在多少户了?各省都可以找出这样例子来。自力更生为主,外援为辅,由贫到富的社,各省都有。国家投资,第一是扶助工业,第二是扶助穷队。四六开或三七开。穷队占六到七。十亿人民币,三亿交公社,七亿交穷队。一是靠本身,二是靠公社,三是靠国家。

一盘棋要三照顾。生产队有五亿人口,千万干部(队长、会计),得罪他们不得了。过去七十万个小社,一社五十个干部,则是三千万干部。瞒产私分为什么有那么大的劲,决心那么大,因为有五亿农民支持他们,我们则脱离了群众。认识这个问题,时间有五个月之久,相当迟,客现实际反映到主观,有个过程。

工人寄钱问题,中心是说服公社,不能拦路劫抢。军官寄钱回去,公社扣了,军官有很大反映。财产权利必须神圣不可侵犯,这样反而建设得快。要说服公社,懂得发展过程,懂得等价交换。

城市办公社,我就想不通。天津人说,要办就办一个,人民代表大会就是人民公社嘛。企业学校都是全民所有制,至于要办食堂随你办,至于家属就业要怎么办就怎么办,已经是国有制还办人民公社干什么。小城市和县城还可以办。

有些东西,不要什么民族风格,如火车、飞机、大炮,政治、艺术可以有民族风格。干部下放,军官当兵,五项并举,蚂蚁啃骨头,是中国香肠,不输出,自己吃,这是马列主义,没有修正主义。公社倒是有修正主义,拦路劫抢、不等价交换。一平二调三提,不是马列主义,违反客观规律,是向“左”的修正主义。误认社会主义为共产主义,误认按劳分配为按需分配,误认集体所有制为全民所有制,想快反慢。武昌会议时,价值法则,等价交换,已弄清,但根本未执行,等于放屁。

城市公社问题,1、小城市可以搞;2、中等城市没有搞的不搞,已成立了的不要一下解散,可以试办;3、大城市不搞。

\begin{maonote}
\mnitem{1}指一九五八年十一月二十八日至十二月十日在武昌召开的中共八届六中全会通过的《关于人民公社若干问题的决议》。
\mnitem{2}赵紫阳一九五九年一月二十七日的这一报告说,自去年十二月中旬以来,粮食问题已经成为农村舆论的中心。雷南县去年晚造生产有很大跃进,年底却出现了粮食紧张的不正常现象。为此全县召开了一系列干部会议,结果查出瞒产私分的粮食七千万斤。雷南县的经验证明,目前农村有大量粮食,粮食紧张完全是假象,是生产队和分队进行瞒产私分造成的。召开以县为单位的生产队长、分队长以上的干部大会是解决粮食问题最主要、最好的形式。在干部大会上,一定要根据群众思想发展规律来进行工作,要把普遍系统深入的思想发动、阶级教育同个别突破、个别交待粮食情况相结合。必须反复交待两条政策:一、粮食政策。明确宣布一九五九年夏收之前粮食消费以生产队为单位进行包干,以解除大家对粮食问题的顾虑。二、对待瞒产干部的政策。应明确宣布瞒产是错误的,但只要坦白交待,可既往不咎;拒不交待的,要给予处分,甚至法办。会议后期要在查错漏、查平衡的基础上迅速安排社员的生活,总结思想,整顿组织,纯洁队伍。
\mnitem{3}陶铸,时任华南经济协作区主任委员,中南局第一书记。
\mnitem{4}刘子厚,时任共河北省委书记处书记,河北省省长。
\mnitem{5}吕洪宾合作社,毛泽东曾视察过这个合作社,这个社的带头人就是吕洪宾。
\mnitem{6}指毛泽东在一九五九年二月二十二日批转的一个重要文件,批示如下:

各省、市、区党委:

赵紫阳同志给广东省委关于解决粮食问题的信件及广东省委的批语\mnote{8},极为重要,现在转发你们。公社大队长小队长瞒产私分粮食一事,情况严重,造成人心不安,影响广大基层干部的共产主义品德,影响春耕和一九五九年大跃进的积极性,影响人民公社的巩固,在全国是一个普遍存在的问题,必须立即解决。各地各县凡是对于这个问题尚未正确解决的,必须立即动手照赵紫阳同志在雷南县所采用的政策和方法,迅速予以解决。瞒产私分是公社成立后,广大基层干部和农民惧怕集体所有制马上变为国有制,“拿走他们的粮食”,所造成的一种不正常的现象。六中全会关于人民公社的决议,肯定了公社在现阶段仍为社会主义的集体所有制,这一点使群众放了心。但公社很大,各大队小队仍怕公社拿走队上的粮食,并且在秋收后已经瞒产私分了,故必须照雷南县那样宣布粮食和干部两条正确的政策,并举行一个坚决的教育运动,才能解决问题。只要政策和方法正确,解决问题的时间只需要十天或者半个月就够了。此件可登党刊,并可转发各地、县。

中央

一九五九年二月二十二日
\mnitem{7}十大任务,是指一九五九年二月确定的一九五九年十大国庆建筑,最终确定的十大国庆工程项目是:人民大会堂、中国革命历史博物馆、中国人民革命军事博物馆、全国农业展览馆、北京火车站、工人体育场、民族文化宫、民族饭店、迎宾馆(钓鱼台国宾馆)、华侨大厦。
\mnitem{8}中共广东省委一九五九年一月三十一日转发赵紫阳报告的批语说:粮食问题必须解决,这是关系到今年生产跃进和整顿公社的最重要问题。许多地方的事实证明,去年粮食大丰收、大跃进是完全肯定的,粮食是有的。必须坚决领导和进行好反瞒产、反本位主义的斗争,才能保证完成粮食外调任务和安排好群众生活。
\end{maonote}
