
\title{关于西藏平叛\mnote{1}}
\date{一九五九年四月十五日}
\thanks{这是毛泽东同志在第十六次最高国务会议上的讲话中关于西藏问题的部分。}
\maketitle


有些人对于西藏寄予同情,但是他们只同情少数人,不同情多数人,一百个人里头,同情几个人,就是那些叛乱分子,而不同情百分之九十几的人。在外国,有那么一些人,他们对西藏就是只同情一两万人,顶多三四万人。西藏本部(只讲昌都、前藏、后藏这三个区域)大概是一百二十万人。一百二十万人,用减法去掉几万人,还有一百一十几万人,世界上有些人对他们不同情。我们则相反,我们同情这一百一十几万人,而不同情那少数人。

那少数人是一些什么人呢?就是剥削、压迫分子。讲贵族,班禅\mnote{2}和阿沛\mnote{3}两位也算贵族,但是贵族有两种,一种是进步的贵族,一种是反动的贵族,他们两位属于进步的贵族。进步分子主张改革,旧制度不要了,舍掉它算了。旧制度不好,对西藏人民不利,一不人兴,二不财旺。西藏地方大,现在人口太少了,要发展起来。这个事情,我跟达赖\mnote{4}讲过。我说,你们要发展人口。我还说,你们的佛教,就是喇嘛教,我是不信的,我赞成你们信。但是,有些规矩可不可以稍微改一下子?你们一百二十万人里头,有八万喇嘛,这八万喇嘛是不生产的,一不生产物质,二不生产人。你看,就神职人员来说,基督教是允许结婚的,回教是允许结婚的,天主教是不允许结婚的。西藏的喇嘛也不能结婚,不生产人。同时,喇嘛要从事生产,搞农业,搞工业,这样才可以维持长久。你们不是要天长地久、永远信佛教吗?我是不赞成永远信佛教,但是你们要信,那有什么办法!我们是毫无办法的,信不信宗教,只能各人自己决定。

至于贵族,对那些站在进步方面主张改革的革命的贵族,以及还不那么革命、站在中间动动摇摇但不站在反革命方面的中间派,我们采取什么态度呢?我个人的意见是:对于他们的土地、他们的庄园,是不是可以用我们对待民族资产阶级的办法,即实行赎买政策,使他们不吃亏。比如我们中央人民政府把他们的生活包下来,你横直剥削农奴也是得到那么一点,中央政府也给你那么一点,你为什么一定要剥削农奴才舒服呢?

我看,西藏的农奴制度,就像我们春秋战国时代那个庄园制度,说奴隶不是奴隶,说自由农民不是自由农民,是介乎这两者之间的一种农奴制度。贵族坐在农奴制度的火山上是不稳固的,每天都觉得要地震,何不舍掉算了,不要那个农奴制度了,不要那个庄园制度了,那一点土地不要了,送给农民。但是吃什么呢?我看,对革命的贵族,革命的庄园主,还有中间派的贵族,中间派的庄园主,只要他不站在反革命那方面,就用赎买政策。我跟大家商量一下,看是不是可以。现在是平叛,还谈不上改革,将来改革的时候,凡是革命的贵族,以及中间派动动摇摇的,总而言之,只要是不站在反革命那边的,我们不使他吃亏,就是照我们现在对待资本家的办法。并且,他这一辈子我们都包到底。资本家也是一辈子包到底。几年定息\mnote{5}过后,你得包下去,你得给他工作,你得给他薪水,你得给他就业,一辈子都包下去。这样一来,农民(占人口的百分之九十五以上)得到了土地,农民就不恨这些贵族了,仇恨就逐渐解开了。

日本有个报纸哇哇叫,讲了一篇,它说,共产党在西藏问题上打了一个大败仗,全世界都反对共产党。说我们打了大败仗,谁人打了大胜仗呢?总有一个打了大胜仗的吧。只有人打了大败仗,又没有人打了大胜仗,哪有那种事?你们讲,究竟胜负如何?假定我们中国人在西藏问题上打了大败仗,那末,谁人打了大胜仗呢?是不是可以说印度干涉者打了大胜仗?我看也很难说。他打了大胜仗,为什么那么痛哭流涕,如丧考妣呢?你们看我这个话有一点道理没有?

还有个美国人,名字叫艾尔索普,写专栏文章的。他隔那么远,认真地写一篇文章,说西藏这个地方没有二十万军队是平定不了的,而这二十万军队,每天要一万吨物资,不可能运这么多去,西藏那个山高得不得了,共产党的军队难得去。因此,他断定叛乱分子灭不了。叛乱分子灭得了灭不了呀?我看大家都有这个疑问。因为究竟灭得了灭不了,没有亲临其境,没有打过游击战争的人,是不会知道的。我这里回答:平叛不要二十万军队,只要五万军队,二十万的四分之一。一九五六年以前我们就五万人(包括干部)在那里,一九五六年那一年我们撤了三万多,剩下一万多。那个时候我们确实认真地宣布六年不改革,六年以后,如果还不赞成,我们还可以推迟,是这样讲的\mnote{6}。你们晓得,整个藏族不是一百二十万人,而是三百万人。刚才讲的西藏本部(昌都、前藏、后藏)是一百二十万人,其他在哪里呢?主要是在四川西部,就是原来西康\mnote{7}区域,以及川西北就是毛儿盖、松潘、阿坝那些地方。这些地方藏族最多。第二是青海,有五十万人。第三是甘肃南部。第四是云南西北部。这四个区域合计一百八十万人。四川省人民代表大会开会,商量在藏族地区搞点民主改革,听了一点风,立即就传到原西康这个区域,一些人就举行武装叛乱。现在青海、甘肃、四川、云南的藏族地区都改革了,人民武装起来了。藏人扛起枪来,组织自卫武装,非常勇敢。这四个区域能够把叛乱分子肃清,为什么西藏不能肃清呢?你讲复杂,原西康这个区域是非常复杂的。原西康的叛乱分子打败了,跑到西藏去了。他们跑到那里,奸淫虏掠,抢得一塌糊涂。他要吃饭,就得抢,于是同藏人就发生矛盾。原西康跑去的,青海跑去的,有一万多人。一万多人要不要吃呢?要吃,从哪里来呢?就在一百二十万人中间吃过来吃过去,从去年七月算起,差不多已经吃了一年了。这回我们把叛乱分子打下来,把他们那些枪收缴了。比如在日喀则,把那个地方政府武装的枪收缴了,江孜也收缴了,亚东也收缴了。收缴了枪的地方,群众非常高兴。老百姓怕他们三个东西:第一是怕他那个印,就是怕那个图章;第二是怕他那个枪;第三,还有一条法鞭,老百姓很怕。把这三者一收,群众皆大欢喜,非常高兴,帮助我们搬枪枝弹药。西藏的老百姓痛苦得不得了。那里的反动农奴主对老百姓硬是挖眼,硬是抽筋,甚至把十几岁女孩子的脚骨拿来作乐器,还有拿人的头骨作饮器喝酒。这样野蛮透顶的叛乱分子完全能够灭掉,不需要二十万军队,只需要五万军队,可以灭得干干净净。灭掉是不是都杀掉呢?不是。所谓灭掉,并不是把他们杀掉,而是把他们捉起来教育改造,包括反动派,比如索康\mnote{8}那种人。这样的人,跑出去的,如果他回来,悔过自新,我们不杀他。

再讲一个中国人的议论。此人在台湾,名为胡适\mnote{9}。他讲,据他看,这个“革命军”(就是叛乱分子)灭不了。他说,他是徽州人,日本人打中国的时候,占领了安徽,但是没有去徽州。什么道理呢?徽州山太多了,地形复杂。日本人连徽州的山都不敢去,西藏那个山共产党敢去?我说,胡适这个方法论就不对,他那个“大胆假设”是危险的。他大胆假设,他推理,说徽州山小,日本人尚且不敢去,那末西藏的山大得多、高得多,共产党难道敢去吗?因此结论:共产党一定不敢去,共产党灭不了那个地方的叛乱武装。现在要批评胡适这个方法论,我看他是要输的,他并不“小心求证”,只有“大胆假设”。

有些人,像印度资产阶级中的一些人,又不同一点,他们有两面性。他们一方面非常不高兴,非常反对我们三月二十日以后开始的坚决镇压叛乱,非常反对我们这种政策,他们同情叛乱分子。另一方面,又不愿意跟我们闹翻,他们想到过去几千年中国跟印度都没有闹翻过,没有战争,同时,他们看到无可奈何花落去,花已经落去了。一九五四年中印两国订了条约\mnote{10},就是声明五项原则的那个条约,他们承认西藏是中国的一部分,是中国的领土。他们留了一手,不做绝。英国人最鬼,英国外交大臣劳埃德,工党议员这个一问,那个一问,他总是一问三不知,说:没有消息,我们英国跟西藏没有接触,在那里没有人员,因此我无可奉告。老是这么讲。他还说,要等西藏那个人出来以后,看他怎么样,我们才说话。他的意思就是达赖出来后,看他说什么话。中国共产党并没有关死门,说达赖是被挟持走的,又发表了他的三封信\mnote{11}。这次人民代表大会,周总理的报告\mnote{12}里头要讲这件事。我们希望达赖回来,还建议这次选举不仅选班禅,而且要选达赖。他是个年轻人,现在还只有二十五岁。假如他活到八十五岁,从现在算起还有六十年,那个时候二十一世纪了,世界会怎么样呀?要变的。那个时候,我相信他会回来的。他五十九年不回来,第六十年他有可能回来。那时候世界都变了。这里是他的父母之邦,生于斯,长于斯,现在到外国,仰人鼻息,几根枪都缴了。我们采取这个态度比较主动,不做绝了。

总理的报告里头要讲希望达赖回国。如果他愿意回国,能够摆脱那些反动分子,我们希望他回国。但是,事实上看来他现在难于回国。他脱离不了那一堆人。同时,他本人那个情绪,上一次到印度他就不想回来的,而班禅是要回来的。那时,总理劝解,可能还有尼赫鲁\mnote{13}劝解,与其不回不如回。那个时候就跟他这么讲:你到印度有什么作用?不过是当一个寓公,就在那里吃饭,脱离群众,脱离祖国的土地和人民。现在,还看不见他有改革的决心。说他要改革,站在人民这方面,站在劳动人民这方面,看来不是的。他那个世界观是不是能改变?六十年以后也许能改,也许不要六十年。而现在看来,一下子要他回来也难。他如果是想回来,明天回来都可以,但是他得进行改革,得平息叛乱,就是要完全站在我们这方面来。看来,他事实上一下子也很难。但是,我们文章不做绝了。

\begin{maonote}
\mnitem{1}一九五九年三月十日,西藏上层反动集团在外国势力支持下,蓄意破坏《关于和平解放西藏办法的协议》的实行,公开宣布“西藏独立”。十七日,达赖喇嘛逃往印度。十九日,叛乱分子发动对人民解放军驻拉萨部队和中央代表机关的全面进攻。中国人民解放军驻藏部队于二十日对拉萨叛乱武装实施反击,并相继平息了其他地区的武装叛乱,维护了国家统一和民族团结。
\mnitem{2}班禅,即班禅额尔德尼·确吉坚赞(一九三八——一九八九),青海循化人。原西藏地方宗教和政治领袖之一。时任政协全国委员会副主席、西藏自治区筹备委员会代理主任委员。
\mnitem{3}阿沛,即阿沛·阿旺晋美,一九一一年生,西藏拉萨人。时任西藏自治区筹备委员会副主任委员兼秘书长。
\mnitem{4}达赖,即达赖喇嘛·丹增嘉措,一九三五年生,青海湟中人。原西藏地方宗教和政治领袖之一。曾任全国人大常委会副委员长、西藏自治区筹备委员会主任委员、中国佛教协会名誉会长。一九五九年西藏上层反动集团发动武装叛乱时逃往印度。
\mnitem{5}定息,是我国在资本主义工商业实行全行业公私合营后,对民族资本家的生产资料进行赎买的一种形式,即不论企业盈亏,统一由国家每年按照合营时清产核资确定的私股股额,发给资本家固定的利息(一般是年息百分之五)。从一九五六年起支付定息。一九六六年九月停止支付。
\mnitem{6}一九五六年十二月三十日,周恩来在同达赖喇嘛的谈话中说,毛泽东主席让我告诉你,可以肯定,在第二个五年计划以内根本不谈改革;过六年之后,如可以改的话,仍然由西藏地方政府根据那时的情况和条件决定。
\mnitem{7}西康,即西康省,一九五五年撤销。撤销时,原辖区划归四川省。
\mnitem{8}索康,即索康·旺清格勒,曾任西藏自治区筹备委员会委员。一九五九年西藏上层反动集团武装叛乱的策动者之一。
\mnitem{9}胡适(一八九一——一九六二),字适之,安徽绩溪人。一九一九年发表《多研究些问题,少谈些“主义”》一文,反对马克思主义的传播,还提出“大胆假设,小心求证”的研究方法。一九四八年去美国,后到台湾。
\mnitem{10}一九五四年四月二十九日,中国和印度两国政府在北京签订了《中印关于中国西藏地方和印度之间的通商和交通协定》。协定明确以和平共处五项原则为指导两国关系的准则,并以此确定了促进中国西藏地方和印度之间的通商贸易及便利两国人民互相朝圣往来的各项具体办法。主要内容是:双方互设商务代理处;双方商人、香客在指定地点进行贸易和按惯例朝圣,并经一定山口、道路往来;关于两国外交、公务人员及国民过境事宜的规定等。协定一九五四年六月三日生效,有效期八年,一九六二年六月期满失效。
\mnitem{11}指《人民日报》一九五九年三月三十日发表的达赖喇嘛三月十一日、十二日、十六日先后写给中央驻西藏代理代表、西藏军区政治委员谭冠三的三封信。三月十一日的信中说:“昨天我决定去军区看戏,但由于少数坏人的煽动,而僧俗人民不解真相追随其后,进行阻拦,确实无法去访,使我害羞难言,忧虑交加,而处于莫知所措的境地。”又说:“反动的坏分子们正在借口保护我的安全而进行着危害我的活动。”三月十二日的信中说:“对于昨天、前天发生的、以保护我的安全为名而制造的严重离间中央与地方关系的事件,我正尽一切可能设法处理。”三月十六日的信中说:“现在此间内外的情况虽然仍很难处置,但我正在用巧妙的办法,在政府官员中从内部划分进步与反对革命的两种人的界线。一旦几天之后,有了一定数量的足以信赖的力量之后,将采取秘密的方式前往军区。”
\mnitem{12}指周恩来总理即将向第二届全国人民代表大会第一次会议作的《政府工作报告》。
\mnitem{13}尼赫鲁(一八八九——一九六四),时任印度总理。
\end{maonote}
