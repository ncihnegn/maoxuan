
\title{中国共产党在民族战争中的地位}
\date{一九三八年十月十四日}
\thanks{这是毛泽东在中国共产党第六届中央委员会扩大的第六次全体会议上的政治报告《论新阶段》的一部分。这个报告是在一九三八年十月十二日至十四日作的,这一部分是十四日讲的。这次会议批准了以毛泽东为首的党中央政治局的路线,是一次很重要的会议。毛泽东在报告中提出“中国共产党在民族战争中的地位”这一问题,便是为的使全党同志明确地知道并认真地负起中国共产党领导抗日战争的重大历史责任。全会确定了坚持抗日民族统一战线的方针,同时指出了在统一战线中有团结又有斗争,“一切经过统一战线”的提法对于中国情况是不适合的,这样就批判了关于统一战线问题上的迁就主义的错误;毛泽东在这次会议的结论中所讲的“统一战线中的独立自主问题”,就是关于这方面的问题。全会同时又确定了全党从事组织人民的抗日武装斗争的极端重要性,决定党的主要工作方面是战区和敌后,而批判了那种把战胜日本帝国主义的希望寄托于国民党军队以及把人民的命运寄托于国民党反动派统治下的合法运动等项错误思想;毛泽东在结论中所讲的“战争和战略问题”,就是关于这一方面的问题。}
\maketitle


同志们!我们有一个光明的前途;我们必须战胜日本帝国主义,必须建设新中国,也一定能够达到这些目的。但是由现在到这个光明前途的中间,存在着一段艰难的路程。为着一个光明的中国而斗争的中国共产党和全国人民,必须有步骤地同日寇作斗争;而要打败它,只有经过长期的战争。关于这个战争的各方面问题,我们已经说得很多。抗战以来的经验,我们也总结了;当前的形势,我们也估计了;全民族的紧急任务,我们也提出了;用长期的抗日民族统一战线支持长期的战争的理由和方法,我们也说明了;国际形势,我们也分析了。那末,还有什么问题呢?同志们,还有一个问题,这就是中国共产党在民族战争中处于何种地位的问题,这就是共产党员应该怎样认识自己、加强自己、团结自己,才能领导这次战争达到胜利而不致失败的问题。

\section{爱国主义和国际主义}

国际主义者的共产党员,是否可以同时又是一个爱国主义者呢?我们认为不但是可以的,而且是应该的。爱国主义的具体内容,看在什么样的历史条件之下来决定。有日本侵略者和希特勒的“爱国主义”,有我们的爱国主义。对于日本侵略者和希特勒的所谓“爱国主义”,共产党员是必须坚决地反对的。日本共产党人和德国共产党人都是他们国家的战争的失败主义者。用一切方法使日本侵略者和希特勒的战争归于失败,就是日本人民和德国人民的利益;失败得越彻底,就越好。日本共产党人和德国共产党人都应该这样做,他们也正在这样做。这是因为日本侵略者和希特勒的战争,不但是损害世界人民的,也是损害其本国人民的。中国的情况则不同,中国是被侵略的国家。因此,中国共产党人必须将爱国主义和国际主义结合起来。我们是国际主义者,我们又是爱国主义者,我们的口号是为保卫祖国反对侵略者而战。对于我们,失败主义是罪恶,争取抗日胜利是责无旁贷的。因为只有为着保卫祖国而战才能打败侵略者,使民族得到解放。只有民族得到解放,才有使无产阶级和劳动人民得到解放的可能。中国胜利了,侵略中国的帝国主义者被打倒了,同时也就是帮助了外国的人民。因此,爱国主义就是国际主义在民族解放战争中的实施。为此理由,每一个共产党员必须发挥其全部的积极性,英勇坚决地走上民族解放战争的战场,拿枪口瞄准日本侵略者。为此理由,我们的党从九一八事变\mnote{1}开始,就提出了用民族自卫战争反抗日本侵略者的号召;后来又提出了抗日民族统一战线的主张,命令红军改编为抗日的国民革命军开赴前线作战,命令自己的党员站在抗日战争的最前线,为保卫祖国流最后一滴血。这些爱国主义的行动,都是正当的,都正是国际主义在中国的实现,一点也没有违背国际主义。只有政治上糊涂的人,或者别有用心的人,才会瞎说我们做得不对,瞎说我们抛弃了国际主义。

\section{共产党员在民族战争中的模范作用}

根据上述理由,共产党员应在民族战争中表现其高度的积极性;而这种积极性,应使之具体地表现于各方面,即应在各方面起其先锋的模范的作用。我们的战争,是在困难环境之中进行的。广大人民群众的民族觉悟、民族自尊心和自信心的不足,大多数民众的无组织,军力的不坚强,经济的落后,政治的不民主,腐败现象和悲观情绪的存在,统一战线内部的不团结、不巩固等等,形成了这种困难环境。因此,共产党员不能不自觉地担负起团结全国人民克服各种不良现象的重大的责任。在这里,共产党员的先锋作用和模范作用是十分重要的。共产党员在八路军和新四军中,应该成为英勇作战的模范,执行命令的模范,遵守纪律的模范,政治工作的模范和内部团结统一的模范。共产党员在和友党友军发生关系的时候,应该坚持团结抗日的立场,坚持统一战线的纲领,成为实行抗战任务的模范;应该言必信,行必果,不傲慢,诚心诚意地和友党友军商量问题,协同工作,成为统一战线中各党相互关系的模范。共产党员在政府工作中,应该是十分廉洁、不用私人、多做工作、少取报酬的模范。共产党员在民众运动中,应该是民众的朋友,而不是民众的上司,是诲人不倦的教师,而不是官僚主义的政客。共产党员无论何时何地都不应以个人利益放在第一位,而应以个人利益服从于民族的和人民群众的利益。因此,自私自利,消极怠工,贪污腐化,风头主义等等,是最可鄙的;而大公无私,积极努力,克己奉公,埋头苦干的精神,才是可尊敬的。共产党员应和党外一切先进分子协同一致,为着团结全国人民克服各种不良现象而努力。必须懂得,共产党员不过是全民族中的一小部分,党外存在着广大的先进分子和积极分子,我们必须和他们协同工作。那种以为只有自己好、别人都不行的想法,是完全不对的。共产党员对于落后的人们的态度,不是轻视他们,看不起他们,而是亲近他们,团结他们,说服他们,鼓励他们前进。共产党员对于在工作中犯过错误的人们,除了不可救药者外,不是采取排斥态度,而是采取规劝态度,使之翻然改进,弃旧图新。共产党员应是实事求是的模范,又是具有远见卓识的模范。因为只有实事求是,才能完成确定的任务;只有远见卓识,才能不失前进的方向。因此,共产党员又应成为学习的模范,他们每天都是民众的教师,但又每天都是民众的学生。只有向民众学习,向环境学习,向友党友军学习,了解了他们,才能对于工作实事求是,对于前途有远见卓识。在长期战争和艰难环境中,只有共产党员协同友党友军和人民大众中的一切先进分子,高度地发挥其先锋的模范的作用,才能动员全民族一切生动力量,为克服困难、战胜敌人、建设新中国而奋斗。

\section{团结全民族和反对民族中的奸细分子}

为要克服困难,战胜敌人,建设新中国,只有巩固和扩大抗日民族统一战线,发动全民族中的一切生动力量,这是唯一无二的方针。但是我们的民族统一战线中已经存在着起破坏作用的奸细分子,这就是那些汉奸、托派\mnote{2}、亲日派分子。共产党员应该随时注意那些奸细分子,用真凭实据揭发他们的罪恶,劝告人民不要上他们的当。共产党员必须提高对于民族奸细分子的政治警觉性。共产党员必须明白,揭发和清除奸细,是和扩大和巩固民族统一战线不能分离的。只顾一方面,忘记另一方面,是完全错误的。

\section{扩大共产党和防止奸细混入}

为了克服困难,战胜敌人,建设新中国,共产党必须扩大自己的组织,向着真诚革命、信仰党的主义、拥护党的政策、并愿意服从纪律、努力工作的广大工人、农民和青年积极分子开门,使党成为一个伟大的群众性的党。在这里,关门主义倾向是不能容许的。但是在同时,对于奸细混入的警觉性也决不可少。日本帝国主义的特务机关,时刻企图破坏我们的党,时刻企图利用暗藏的汉奸、托派、亲日派、腐化分子、投机分子,装扮积极面目,混入我们的党里来。对于这些分子的警惕和严防,一刻也不应该放松。不可因为怕奸细而把自己的党关起门来,大胆地发展党是我们已经确定了的方针。但是在同时,又不可因为大胆发展而疏忽对于奸细分子和投机分子乘机侵入的警戒。只顾一方面,忘记另一方面,就会犯错误。“大胆发展而又不让一个坏分子侵入”,这才是正确的方针。

\section{坚持统一战线和坚持党的独立性}

坚持民族统一战线才能克服困难,战胜敌人,建设新中国,这是毫无疑义的。但是在同时,必须保持加入统一战线中的任何党派在思想上、政治上和组织上的独立性,不论是国民党也好,共产党也好,其它党派也好,都是这样。三民主义\mnote{3}中的民权主义,在党派问题上说来,就是容许各党派互相联合,又容许各党派独立存在。如果只谈统一性,否认独立性,就是背弃民权主义,不但我们共产党不能同意,任何党派也是不能同意的。没有问题,统一战线中的独立性,只能是相对的,而不是绝对的;如果认为它是绝对的,就会破坏团结对敌的总方针。但是决不能抹杀这种相对的独立性,无论在思想上也好,在政治上也好,在组织上也好,各党必须有相对的独立性,即是说有相对的自由权。如果被人抹杀或自己抛弃这种相对的自由权,那就也会破坏团结对敌的总方针。这是每个共产党员,同时也是每个友党党员,应该明白的。

阶级斗争和民族斗争的关系也是这样。在抗日战争中,一切必须服从抗日的利益,这是确定的原则。因此,阶级斗争的利益必须服从于抗日战争的利益,而不能违反抗日战争的利益。但是阶级和阶级斗争的存在是一个事实;有些人否认这种事实,否认阶级斗争的存在,这是错误的。企图否认阶级斗争存在的理论是完全错误的理论。我们不是否认它,而是调节它。我们提倡的互助互让政策,不但适用于党派关系,也适用于阶级关系。为了团结抗日,应实行一种调节各阶级相互关系的恰当的政策,既不应使劳苦大众毫无政治上和生活上的保证,同时也应顾到富有者的利益,这样去适合团结对敌的要求。只顾一方面,不顾另一方面,都将不利于抗日。

\section{照顾全局,照顾多数及和同盟者一道工作}

共产党员在领导群众同敌人作斗争的时候,必须有照顾全局,照顾多数及和同盟者一道工作的观点。共产党员必须懂得以局部需要服从全局需要这一个道理。如果某项意见在局部的情形看来是可行的,而在全局的情形看来是不可行的,就应以局部服从全局。反之也是一样,在局部的情形看来是不可行的,而在全局的情形看来是可行的,也应以局部服从全局。这就是照顾全局的观点。共产党员决不可脱离群众的多数,置多数人的情况于不顾,而率领少数先进队伍单独冒进;必须注意组织先进分子和广大群众之间的密切联系。这就是照顾多数的观点。在一切有愿意和我们合作的民主党派和民主人士存在的地方,共产党员必须采取和他们一道商量问题和一道工作的态度。那种独断专行,把同盟者置之不理的态度,是不对的。一个好的共产党员,必须善于照顾全局,善于照顾多数,并善于和同盟者一道工作。我们过去在这些方面存在着很大的缺点,必须注意改进。

\section{干部政策}

中国共产党是在一个几万万人的大民族中领导伟大革命斗争的党,没有多数才德兼备的领导干部,是不能完成其历史任务的。十七年来,我们党已经培养了不少的领导人材,军事、政治、文化、党务、民运各方面,都有了我们的骨干,这是党的光荣,也是全民族的光荣。但是,现有的骨干还不足以支撑斗争的大厦,还须广大地培养人材。在中国人民的伟大的斗争中,已经涌出并正在继续涌出很多的积极分子,我们的责任,就在于组织他们,培养他们,爱护他们,并善于使用他们。政治路线确定之后,干部就是决定的因素\mnote{4}。因此,有计划地培养大批的新干部,就是我们的战斗任务。

不但要关心党的干部,还要关心非党的干部。党外存在着很多的人材,共产党不能把他们置之度外。去掉孤傲习气,善于和非党干部共事,真心诚意地帮助他们,用热烈的同志的态度对待他们,把他们的积极性组织到抗日和建国的伟大事业中去,这是每一个共产党员的责任。

必须善于识别干部。不但要看干部的一时一事,而且要看干部的全部历史和全部工作,这是识别干部的主要方法。

必须善于使用干部。领导者的责任,归结起来,主要地是出主意、用干部两件事。一切计划、决议、命令、指示等等,都属于“出主意”一类。使这一切主意见之实行,必须团结干部,推动他们去做,属于“用干部”一类。在这个使用干部的问题上,我们民族历史中从来就有两个对立的路线:一个是“任人唯贤”的路线,一个是“任人唯亲”的路线。前者是正派的路线,后者是不正派的路线。共产党的干部政策,应是以能否坚决地执行党的路线,服从党的纪律,和群众有密切的联系,有独立的工作能力,积极肯干,不谋私利为标准,这就是“任人唯贤”的路线。过去张国焘的干部政策与此相反,实行“任人唯亲”,拉拢私党,组织小派别,结果叛党而去,这是一个大教训。鉴于张国焘的和类似张国焘的历史教训,在干部政策问题上坚持正派的公道的作风,反对不正派的不公道的作风,借以巩固党的统一团结,这是中央和各级领导者的重要的责任。

必须善于爱护干部。爱护的办法是:第一,指导他们。这就是让他们放手工作,使他们敢于负责;同时,又适时地给以指示,使他们能在党的政治路线下发挥其创造性。第二,提高他们。这就是给以学习的机会,教育他们,使他们在理论上在工作能力上提高一步。第三,检查他们的工作,帮助他们总结经验,发扬成绩,纠正错误。有委托而无检查,及至犯了严重的错误,方才加以注意,不是爱护干部的办法。第四,对于犯错误的干部,一般地应采取说服的方法,帮助他们改正错误。只有对犯了严重错误而又不接受指导的人们,才应当采取斗争的方法。在这里,耐心是必要的;轻易地给人们戴上“机会主义”的大帽子,轻易地采用“开展斗争”的方法,是不对的。第五,照顾他们的困难。干部有疾病、生活、家庭等项困难问题者,必须在可能限度内用心给以照顾。这些就是爱护干部的方法。

\section{党的纪律}

鉴于张国焘严重地破坏纪律的行为,必须重申党的纪律:(一)个人服从组织;(二)少数服从多数;(三)下级服从上级;(四)全党服从中央。谁破坏了这些纪律,谁就破坏了党的统一。经验证明:有些破坏纪律的人,是由于他们不懂得什么是党的纪律;有些明知故犯的人,例如张国焘,则利用许多党员的无知以售其奸。因此,必须对党员进行有关党的纪律的教育,既使一般党员能遵守纪律,又使一般党员能监督党的领袖人物也一起遵守纪律,避免再发生张国焘事件。为使党内关系走上正轨,除了上述四项最重要的纪律外,还须制定一种较详细的党内法规,以统一各级领导机关的行动。

\section{党的民主}

处在伟大斗争面前的中国共产党,要求整个党的领导机关,全党的党员和干部,高度地发挥其积极性,才能取得胜利。所谓发挥积极性,必须具体地表现在领导机关、干部和党员的创造能力,负责精神,工作的活跃,敢于和善于提出问题、发表意见、批评缺点,以及对于领导机关和领导干部从爱护观点出发的监督作用。没有这些,所谓积极性就是空的。而这些积极性的发挥,有赖于党内生活的民主化。党内缺乏民主生活,发挥积极性的目的就不能达到。大批能干人材的创造,也只有在民主生活中才有可能。由于我们的国家是一个小生产的家长制占优势的国家,又在全国范围内至今还没有民主生活,这种情况反映到我们党内,就产生了民主生活不足的现象。这种现象,妨碍着全党积极性的充分发挥。同时,也就影响到统一战线中、民众运动中民主生活的不足。为此缘故,必须在党内施行有关民主生活的教育,使党员懂得什么是民主生活,什么是民主制和集中制的关系,并如何实行民主集中制。这样才能做到:一方面,确实扩大党内的民主生活;又一方面,不至于走到极端民主化,走到破坏纪律的自由放任主义。

在我们军队中的党组织,也须增加必要的民主生活,以便提高党员的积极性,增强军队的战斗力。但是军队党组织的民主应少于地方党组织的民主。无论在军队或在地方,党内民主都应是为着巩固纪律和增强战斗力,而不是削弱这种纪律和战斗力。

扩大党内民主,应看作是巩固党和发展党的必要的步骤,是使党在伟大斗争中生动活跃,胜任愉快,生长新的力量,突破战争难关的一个重要的武器。

\section{我们的党已经从两条战线斗争中巩固和壮大起来}

十七年来,我们的党,一般地已经学会了使用马克思列宁主义的思想斗争的武器,从两方面反对党内的错误思想,一方面反对右倾机会主义,又一方面反对“左”倾机会主义。

在党的六届五中全会以前\mnote{5},我们党反对了陈独秀的右倾机会主义\mnote{6}和李立三同志的“左”倾机会主义\mnote{7}。由于这两次党内斗争的胜利,使党获得了伟大的进步。五中全会以后,又有过两次有历史意义的党内斗争,这就是在遵义会议\mnote{8}上的斗争和开除张国焘出党的斗争。

遵义会议纠正了在第五次反“围剿”斗争中所犯的“左”倾机会主义性质的严重的原则错误,团结了党和红军,使得党中央和红军主力胜利地完成了长征,转到了抗日的前进阵地,执行了抗日民族统一战线的新政策。由于巴西会议\mnote{9}和延安会议\mnote{10}(反对张国焘路线的斗争是从巴西会议开始而在延安会议完成的)反对了张国焘的右倾机会主义,使得全部红军会合一起,全党更加团结起来,进行了英勇的抗日斗争。这两种机会主义错误都是在国内革命战争中产生的,它们的特点是在战争中的错误。

这两次党内斗争所得的教训在什么地方呢?在于:(一)由于不认识中国革命战争中的特点而产生的、表现于第五次反“围剿”斗争中的严重的原则错误,包含着不顾主客观条件的“左”的急性病倾向,这种倾向极端地不利于革命战争,同时也不利于任何革命运动。(二)张国焘的机会主义,则是革命战争中的右倾机会主义,其内容是他的退却路线、军阀主义和反党行为的综合。只有克服了它,才使得本质很好而且作了长期英勇斗争的红军第四方面军的广大的干部和党员,从张国焘的机会主义统制之下获得解放,转到中央的正确路线方面来。(三)十年土地革命战争时期的伟大的组织工作,不论是军事建设工作也好,政府工作也好,民众工作也好,党的建设工作也好,是有大的成绩的,没有这种组织工作和前线的英勇战斗相配合,要支持当时的残酷的反对蒋介石的斗争是不可能的。然而在后一个时期内,党的干部政策和组织政策方面,是犯了严重的原则性的错误的,这表现在宗派倾向、惩办主义和思想斗争中的过火政策。这是过去立三路线的残余未能肃清的结果,也是当时政治上的原则错误的结果。这些错误,也因遵义会议得到了纠正,使党转到了正确的干部政策和正确的组织原则方面来了。至于张国焘的组织路线,则是完全离开了共产党的一切原则,破坏了党的纪律,从小组织活动一直发展到反党反中央反国际的行为。中央对于张国焘的罪恶的路线错误和反党行为,曾经尽了一切可能的努力去克服它,并企图挽救张国焘本人。但是到了张国焘不但坚持地不肯改正他的错误,采取了两面派的行为,而且在后来实行叛党,投入国民党的怀抱的时候,党就不得不坚决地开除他的党籍。这一处分,不但获得了全党的拥护,而且获得了一切忠实于民族解放事业的人们的拥护。共产国际也批准了这一处分,并指出:张国焘是一个逃兵和叛徒。

以上这些教训和成功,给了我们今后团结全党,巩固思想上、政治上和组织上的一致,胜利地进行抗日战争的必要的前提。我们的党已经从两条战线斗争中巩固和壮大起来了。

\section{当前的两条战线斗争}

在今后的抗日形势中,从政治上反对右的悲观主义,将是头等重要的;但是在同时,反对“左”的急性病,也仍然要注意。在统一战线问题上,在党的组织问题上和在民众的组织问题上,则须继续反对“左”的关门主义倾向,以便实现和各抗日党派的合作,发展共产党和发展民众运动;但是在同时,无条件的合作,无条件的发展,这种右倾机会主义的倾向也要注意反对,否则也就会妨碍合作,妨碍发展,而变为投降主义的合作和无原则的发展了。

两条战线的思想斗争必须切合于具体对象的情况,决不应主观地看问题,决不应使过去那种“乱戴帽子”的坏习惯继续存在。

在反倾向的斗争中,反对两面派的行为,是值得严重地注意的。因为两面派行为的最大的危险性,在于它可能发展到小组织行动;张国焘的历史就是证据。阳奉阴违,口是心非,当面说得好听,背后又在捣鬼,这就是两面派行为的表现。必须提高干部和党员对于两面派行为的注意力,才能巩固党的纪律。

\section{学习}

一般地说,一切有相当研究能力的共产党员,都要研究马克思、恩格斯、列宁、斯大林的理论,都要研究我们民族的历史,都要研究当前运动的情况和趋势;并经过他们去教育那些文化水准较低的党员。特殊地说,干部应当着重地研究这些,中央委员和高级干部尤其应当加紧研究。指导一个伟大的革命运动的政党,如果没有革命理论,没有历史知识,没有对于实际运动的深刻的了解,要取得胜利是不可能的。

马克思、恩格斯、列宁、斯大林的理论,是“放之四海而皆准”的理论。不应当把他们的理论当作教条看待,而应当看作行动的指南。不应当只是学习马克思列宁主义的词句,而应当把它当成革命的科学来学习。不但应当了解马克思、恩格斯、列宁、斯大林他们研究广泛的真实生活和革命经验所得出的关于一般规律的结论,而且应当学习他们观察问题和解决问题的立场和方法。我们党的马克思列宁主义的修养,现在已较过去有了一些进步,但是还很不普遍,很不深入。我们的任务,是领导一个几万万人口的大民族,进行空前的伟大的斗争。所以,普遍地深入地研究马克思列宁主义的理论的任务,对于我们,是一个亟待解决并须着重地致力才能解决的大问题。我希望从我们这次中央全会之后,来一个全党的学习竞赛,看谁真正地学到了一点东西,看谁学的更多一点,更好一点。在担负主要领导责任的观点上说,如果我们党有一百个至二百个系统地而不是零碎地、实际地而不是空洞地学会了马克思列宁主义的同志,就会大大地提高我们党的战斗力量,并加速我们战胜日本帝国主义的工作。

学习我们的历史遗产,用马克思主义的方法给以批判的总结,是我们学习的另一任务。我们这个民族有数千年的历史,有它的特点,有它的许多珍贵品。对于这些,我们还是小学生。今天的中国是历史的中国的一个发展;我们是马克思主义的历史主义者,我们不应当割断历史。从孔夫子到孙中山,我们应当给以总结,承继这一份珍贵的遗产。这对于指导当前的伟大的运动,是有重要的帮助的。共产党员是国际主义的马克思主义者,但是马克思主义必须和我国的具体特点相结合并通过一定的民族形式才能实现。马克思列宁主义的伟大力量,就在于它是和各个国家具体的革命实践相联系的。对于中国共产党说来,就是要学会把马克思列宁主义的理论应用于中国的具体的环境。成为伟大中华民族的一部分而和这个民族血肉相联的共产党员,离开中国特点来谈马克思主义,只是抽象的空洞的马克思主义。因此,使马克思主义在中国具体化,使之在其每一表现中带着必须有的中国的特性,即是说,按照中国的特点去应用它,成为全党亟待了解并亟须解决的问题。洋八股\mnote{11}必须废止,空洞抽象的调头必须少唱,教条主义必须休息,而代之以新鲜活泼的、为中国老百姓所喜闻乐见的中国作风和中国气派。把国际主义的内容和民族形式分离起来,是一点也不懂国际主义的人们的做法,我们则要把二者紧密地结合起来。在这个问题上,我们队伍中存在着的一些严重的错误,是应该认真地克服的。

当前的运动的特点是什么?它有什么规律性?如何指导这个运动?这些都是实际的问题。直到今天,我们还没有懂得日本帝国主义的全部,也还没有懂得中国的全部。运动在发展中,又有新的东西在前头,新东西是层出不穷的。研究这个运动的全面及其发展,是我们要时刻注意的大课题。如果有人拒绝对于这些作认真的过细的研究,那他就不是一个马克思主义者。

学习的敌人是自己的满足,要认真学习一点东西,必须从不自满开始。对自己,“学而不厌”,对人家,“诲人不倦”,我们应取这种态度。

\section{团结和胜利}

中国共产党内部的团结,是团结全国人民争取抗日胜利和建设新中国的最基本的条件。经过了十七年锻炼的中国共产党,已经学到了如何团结自己的许多方法,已经老练得多了。这样,我们就能在全国人民中形成一个坚强的核心,争取抗日的胜利和建设一个新中国。同志们,只要我们能团结,这个目的就一定能够达到。


\begin{maonote}
\mnitem{1}见本书第一卷\mxnote{论反对日本帝国主义的策略}{4}。
\mnitem{2}见本卷\mxnote{论持久战}{9}。
\mnitem{3}见本书第一卷\mxnote{湖南农民运动考察报告}{8}。
\mnitem{4}一九三四年一月斯大林《在党的第十七次代表大会上关于联共(布)中央工作的总结报告》中说:“在正确的政治路线提出以后,组织工作就决定一切,其中也决定政治路线本身的命运,即决定它的实现或失败。”斯大林在这里说到了正确挑选人才的问题。一九三五年五月斯大林《在克里姆林宫举行的红军学院学员毕业典礼上的讲话》中,提出和说明了“干部决定一切”的口号。(《斯大林选集》下卷,人民出版社1979年版,第343、371页)
\mnitem{5}指从一九二七年八月七日中共中央紧急会议到一九三四年一月中共六届五中全会以前这一段时间。
\mnitem{6}见本书第一卷\mxnote{中国革命战争的战略问题}{4}。
\mnitem{7}见本书第一卷\mxnote{中国革命战争的战略问题}{5}。
\mnitem{8}见本书第一卷\mxnote{中国革命战争的战略问题}{7}。
\mnitem{9}巴西会议是一九三五年九月九日由毛泽东、周恩来、张闻天、秦邦宪和王稼祥在巴西(今属四川省若尔盖县)召开的紧急会议。当时,红军一、四方面军正在长征途中,张国焘拒绝执行中央的北上方针,并企图危害中央。毛泽东等在这次会议上决定脱离危险区域,率领一部分红军先行北上。张国焘则率领另一部分被他欺骗的红军从阿坝地区南下天全、芦山等地,另立“中央”,揭出反党分裂的旗帜。
\mnitem{10}延安会议是一九三七年三月在延安召开的中共中央政治局扩大会议。这次会议讨论了当时国内政治形势和党的任务,并着重批评了张国焘的错误。在会议作出的《关于张国焘同志错误的决定》中,对于张国焘的机会主义、军阀主义和反党行为,进行了系统的批判和总结。张国焘本人参加了这次会议,表面上表示接受对他的批判,实际上准备最后叛党。参见本书第一卷\mxnote{论反对日本帝国主义的策略}{24}。
\mnitem{11}参见本书第三卷\mxart{反对党八股}一文中关于洋八股的说明。
\end{maonote}
