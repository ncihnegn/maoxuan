
\title{抗日根据地的政权问题}
\date{一九四〇年三月六日}
\thanks{这是毛泽东为中共中央起草的对党内的指示。}
\maketitle


(一)目前是国民党反共顽固派极力反对我们在华北、华中等地建立抗日民主政权,而我们则必须建立这种政权,并已经可能在各主要的抗日根据地内建立这种政权的时候。我们和反共顽固派为政权问题在华北、华中和西北的斗争,带着推动全国建立统一战线政权的性质,为全国观感之所系,因此,必须谨慎地处理这个问题。

(二)在抗日时期,我们所建立的政权的性质,是民族统一战线的。这种政权,是一切赞成抗日又赞成民主的人们的政权,是几个革命阶级联合起来对于汉奸和反动派的民主专政。它是和地主资产阶级的反革命专政区别的,也和土地革命时期的工农民主专政有区别。对于这种政权性质的明确了解和认真执行,将大有助于全国民主化的推动。过左和过右,均将给予全国人民以极坏的影响。

(三)目前正在开始的召集河北参议会和选举河北行政委员会,是一件具有严重意义的事。同样,在晋西北,在山东,在淮河以北,在绥德、富县、陇东等地建立新的政权,也具有严重的意义。必须依照上述原则进行,力避过右和过左的倾向。目前更严重的是忽视争取中等资产阶级和开明绅士的“左”的倾向。

(四)根据抗日民族统一战线政权的原则,在人员分配上,应规定为共产党员占三分之一,非党的左派进步分子占三分之一,不左不右的中间派占三分之一。

(五)必须保证共产党员在政权中占领导地位,因此,必须使占三分之一的共产党员在质量上具有优越的条件。只要有了这个条件,就可以保证党的领导权,不必有更多的人数。所谓领导权,不是要一天到晚当作口号去高喊,也不是盛气凌人地要人家服从我们,而是以党的正确政策和自己的模范工作,说服和教育党外人士,使他们愿意接受我们的建议。

(六)必须使党外进步分子占三分之一,因为他们联系着广大的小资产阶级群众。我们这样做,对于争取小资产阶级将有很大的影响。

(七)给中间派以三分之一的位置,目的在于争取中等资产阶级和开明绅士。这些阶层的争取,是孤立顽固派的一个重要的步骤。目前我们决不能不顾到这些阶层的力量,我们必须谨慎地对待他们。

(八)对于共产党以外的人员,不问他们是否有党派关系和属于何种党派,只要是抗日的并且是愿意和共产党合作的,我们便应以合作的态度对待他们。

(九)上述人员的分配是党的真实的政策,不能敷衍塞责。为着执行这个政策,必须教育担任政权工作的党员,克服他们不愿和不惯同党外人士合作的狭隘性,提倡民主作风,遇事先和党外人士商量,取得多数同意,然后去做。同时,尽量地鼓励党外人士对各种问题提出意见,并倾听他们的意见。绝不能以为我们有军队和政权在手,一切都要无条件地照我们的决定去做,因而不注意去努力说服非党人士同意我们的意见,并心悦诚服地执行。

(十)上述人员数目的分配是一种大体上的规定,各地须依当地的实际情况施行,不是要机械地凑足数目字。最下层政权的成分可以酌量变通,防止地主豪绅钻进政权机关。政权建立已久的晋察冀边区、冀中区、太行山区和冀南区,应照此原则重新审查自己的方针。在建立新的政权时,一概照此原则。

(十一)抗日统一战线政权的选举政策,应是凡满十八岁的赞成抗日和民主的中国人,不分阶级、民族、男女、信仰、党派、文化程度,均有选举权和被选举权。抗日统一战线政权的产生,应经过人民选举。其组织形式,应是民主集中制。

(十二)抗日统一战线政权的施政方针,应以反对日本帝国主义,保护抗日的人民,调节各抗日阶层的利益,改良工农的生活和镇压汉奸、反动派为基本出发点。

(十三)对参加我们政权的党外人士的生活习惯和言论行动,不能要求他们和共产党员一样,否则将使他们感到不满和不安。

(十四)责成各中央局、各中央分局、各区党委、各军队首长,对党内作明确的说明,使此指示充分地实现于政权工作中。
