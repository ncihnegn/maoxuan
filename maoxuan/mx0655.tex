
\title{不搞科学技术,生产力无法提高}
\date{一九六三年十二月十六日}
\thanks{这是毛泽东同志在听取国务院副总理兼国家科学技术委员会主任聂荣臻汇报十年科学技术规划时插话的节录。}
\maketitle


社会科学也要有一个十年规划。社会科学落后了,这回没有搞规划。社会科学也要投一点资。

有一本杂志《自然辩证法研究通讯》,曾停了很久,现在复刊了。复刊了就好。现在第二期已经出了。

要有革命精神和严格的科学态度。

科学技术这一仗,一定要打,而且必须打好。过去我们打的是上层建筑的仗,是建立人民政权、人民军队。建立这些上层建筑干什么呢?就是要搞生产。搞上层建筑、搞生产关系的目的就是解放生产力。现在生产关系是改变了,就要提高生产力。不搞科学技术,生产力无法提高。

科学研究有实用的,还有理论的。要加强理论研究,要有专人搞,不搞理论是不行的。要培养一批懂得理论的人才,也可以从工人农民中间来培养。我们这些人要懂得些自然科学理论,如医学方面、生物学方面。

死光\mnote{2},要组织一批人专门去研究它。要有一小批人吃了饭不做别的事,专门研究它。没有成绩不要紧。军事上除进攻武器外,要注意防御问题的研究,也许我们将来在作战中主要是防御。进攻武器,比原子弹的数量我们比不赢人家。战争历来都需要攻防两手,筑城、挖山洞都是防嘛。秦始皇\mnote{2}的万里长城没有多大用处。我们准备做一些蠢事,要搞地下工厂、地下铁道,逐年地搞。

三大革命运动中的科学实验,主要是指自然科学。社会科学的研究不能完全采用实验的方法。例如研究政治经济学不能用实验方法,要用抽象法,这是马克思在《资本论》里说的\mnote{3}。商品、战争、辩证法等,是观察了千百次现象才能得出理论概括的。

\begin{maonote}
\mnitem{1}死光,即激光。
\mnitem{2}秦始皇,即赢政(公元前二五九——前二一〇),秦王朝的建立者。
\mnitem{3}见马克思《资本论》一八六七年第一版序言。原文是:“分析经济形式,既不能用显微镜,也不能用化学试剂。二者都必须用抽象力来代替。”(《马克思恩格斯选集》第2卷,人民出版社1995年版,第99页)
\end{maonote}
