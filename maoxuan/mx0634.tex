
\title{同蒙哥马利的谈话}
\date{一九六〇年五月二十七日}
\maketitle


\mxsay{蒙哥马利(简称蒙):}我来到中国发现西方对中国的看法完全错误,他们以为中国人是受压抑的,很不愉快,饿着肚皮。事实上,大家都很愉快,满面笑容,看起来都吃得很饱。今天我访问了一个人民公社。社长才三十岁,他是一个很聪明、很能干的人。他的公社办得很好。

\mxsay{毛泽东主席(简称毛):}西方对我们的看法可以说是基本上错误的。我们的粮食还不够。按照平均人口计算,每人每年平均只有四百公斤粮食。

\mxsay{蒙:}可是没有人饿着肚皮。

\mxsay{毛:}这四百公斤包括口粮、种子、饲料和储备粮。当然比过去有很大的好转。比十年前好,比蒋介石统治时期好,就是比前几年也好。所以西方的观点基本上是错误的。

\mxsay{蒙:}可是大家还是有足够吃的。

\mxsay{毛:}相对来说是够的。

\mxsay{蒙:}孩子们看起来吃得很饱。

\mxsay{毛:}这是对的。

\mxsay{蒙:}所有的人看起来都很健康。

\mxsay{毛:}他们是很高兴的。人们都组织起来了,为建设自己的国家和改善生活而努力。

\mxsay{蒙:}我去天津近郊看了你们的士兵。他们的身体都很健康。

\mxsay{毛:}我们现在的日子还不能算是富足。还要等十年或者两个十年,那个时候我们每人每年可能有七百五十公斤到一千公斤粮食。

\mxsay{蒙:}再过十年就增加了一亿五千万人口。

\mxsay{毛:}一亿左右,这不要紧。

\mxsay{蒙:}你们粮食的增长可以满足你们人口增长的需要。

\mxsay{毛:}粮食增长快于人口增长,而且我们也在控制人口的增长。

\mxsay{蒙:}你们每年人口的增长率是不是百分之二?

\mxsay{毛:}百分之二左右。我们的死亡率减少了,平均年龄提高了。过去平均寿命只有三十岁。就是死得多死得早,现在的平均寿命已提到五十岁。

\mxsay{蒙:}这是因为你们有了各种医药、卫生设备和抗生素等。

\mxsay{毛:}人民的生活改善了,我们也进行了防疫工作。在蒋介石统治时期,我们生产的钢铁每年很少。今年可能多点。但是,这还是不够的。你们平均每人每年有半吨钢。要是我们按照六亿五千万人口计算,每人每年只有一点点,还差得远呢。

\mxsay{蒙:}我们是一个高度组织起来的工业国家。

\mxsay{毛:}你们是一个高度工业化的拍国家。

\mxsay{蒙:}而且还在更加工业化。我们国家面积小,但是人口多。

\mxsay{毛:}你们人口密度比中国大。

\mxsay{蒙:}你是否去过英国?

\mxsay{毛:}没有,可是去过香港,所以也可以说是去过英国。我去过二十多次。

\mxsay{蒙:}最近的一次是在什么时候?

\mxsay{毛:}最后的一次是一九二四年。

\mxsay{蒙:}那是三十六年以前。三十年以前我曾经到过上海。当时上海是一个欧洲城市。现在仍然是欧洲的建筑物,但是欧洲人不在了。

\mxsay{毛:}英国还有一些侨民,还有一些商业和企业在上海,例如英国还有一个毛线厂在上海。

\mxsay{蒙:}那很好。请你给我讲一讲你对今天的世界局势有什么看法?

\mxsay{毛:}国际局势很好,没有什么坏,无非是全世界反苏反华。

\mxsay{蒙:}这是很坏的。

\mxsay{毛:}这是美国制造的,不坏。

\mxsay{蒙:}但这是很坏的。

\mxsay{毛:}不坏,是好的。他们如果不反对我们,我们就同艾森豪威尔、杜勒斯\mnote{2}一样了,所以照理应该反。他们这样做,是有间歇性的。去年一年反华,今年反苏。

\mxsay{蒙:}那是美国做的,不是英国。

\mxsay{毛:}主要是美国,它也策动在各国的走狗这样做。

\mxsay{蒙:}因此我认为局势是坏的。

\mxsay{毛:}现在的局势我看不是热战破裂,也不是和平共处,而是第三种:冷战共处。

\mxsay{蒙:}困难就在这里。在冷战中相处是困难的。

\mxsay{毛:}我们就要解决这个问题。

\mxsay{蒙:}我们必须找到一个解决办法。

\mxsay{毛:}但是我们要有两个方面的准备。一个是继续冷战,另一个是把冷战转为和平共处。所以你做转化工作,我们欢迎。

\mxsay{蒙:}是的。我认为我们不能在这种紧张局势中生活下去。我们的孩子们是在冷战中长大的,这对孩子们是坏的。所以我们必须把这种情况转为和平共处。我不希望看到我的孩子长大以后认为世界必须一直存在着紧张。

\mxsay{毛:}这要有分析。冷战有好的一面,也有坏的一面。坏的一面是它有可能转为热战。

\mxsay{蒙:}有可能。

\mxsay{毛:}好的一面是有可能转为和平共处。

\mxsay{蒙:}这不能够称为是冷战的好处。

\mxsay{毛:}我们说有好处,因为美国制造紧张局势,就制造更多反对它的人,例如在南朝鲜、日本、土耳其以及拉丁美洲,很多国家都反对美国人的控制。这是美国人自己造成的。

\mxsay{蒙:}我不能肯定美国在西方国家集团中制造了它的反对者,在西方集团中没有发生这种情况,虽然我希望发生这种情况。

\mxsay{毛:}我不是指欧洲,欧洲是比较平静的。我是指南朝鲜、南越、日本、土耳其、古巴以及其他拉丁美洲国家和非洲。非洲不能光责备美国,首先是要责备欧洲的殖民主义者。但是,美国要在那里取欧洲殖民主义者的地位而代之。因此,我说好的一面就在于使这些国家反对美帝国主义。这正在动摇整个资本主义世界的基础。

\mxsay{蒙:}西方世界的领袖是美国,现在西方国家怕被这个领袖领到战争中去,这是个很奇怪的现象。因为在上两次世界大战中,美国都等到战争打到一半才参加进来。可是现在西方国家却怕美国把它们带入战争。我们必须把这样一种情况改过来:现在的情况是,西方集团的领袖跟东方集团两个最大的国家根本谈不拢。由于这个原因,美国在西方的领导受到怀疑。

\mxsay{毛:}只要美国的领导不削弱,由英国、法国来加强,就不可能改变局势。

\mxsay{蒙:}我相信必然产生这样的一种情况。

\mxsay{毛:}你是英国人,你到法国跑过,你去过两次苏联,现在你来到了中国。有没有这种可能,英、法、苏、中在某些重大国际问题上取得一致意见?

\mxsay{蒙:}是的,我想是可能的。但是,由于美国的领导,英、法会害怕这样做。

\mxsay{毛:}慢慢来。我们希望你们的国家强大一些,希望法国强大一些,希望你们的发言权大一些,那样事情就好办了,使美国、西德、日本有所约束。

威胁你们和法国的是美国和西德,还有在远东的日本。威胁我们的也是这三个国家。我们不感到英国对我们是个威胁,也不认为法国对我们是个威胁。对我们的威胁主要来自美国和日本。

\mxsay{蒙:}我觉得很重要的是,在这个非常复杂的局势中,我们应首先采取哪一个步骤?我觉得首先应该从别国领土上撤走一切外国军队,这是需要时间的。

\mxsay{毛:}主要是美国的势力,一部分在欧洲,一部分在亚洲。英国在德国只有四个师。

\mxsay{蒙:}只有三个。

\mxsay{毛:}而美国在国外有一百五十万军队,二百五十个军事基地,包括在西德、英国、土耳其,还有在摩洛哥。在东方,美国在日本、南朝鲜、台湾、菲律宾有军事基地;美国还在南越有军事人员,在泰国和巴基斯坦有空军基地。

\mxsay{蒙:}主要的问题是大家应该回到本国去。如果我们能做两件事,我们就有可能和缓紧张局势:第一,停止对欧洲的军事占领;第二,解决台湾问题。问题只能一个一个来。

\mxsay{毛:}但是人民在做。南朝鲜人民、日本人民,还有土耳其人民,都在进行示威游行。土耳其刚刚发生了政变\mnote{3},这总不能说是共产党搞的吧。

\mxsay{蒙:}要同时做一切事情是没有好处的。我是个军人,我了解这一点。你也是个军人,你也应该了解这一点。

\mxsay{毛:}你有三十五年军龄,你比我长,我只有二十五年。

\mxsay{蒙:}我有五十二年了。

\mxsay{毛:}可是我还是共产党军事委员会主席。

\mxsay{蒙:}那很好。我读过你关于军事的著作,写得很好。

\mxsay{毛:}我不觉得有什么好。我是从你们那儿学来的。你学过克劳塞维茨\mnote{4},我也学过。他说战争是政治的另一种形式的继续。

\mxsay{蒙:}我也学过成吉思汗\mnote{5},他强调机动性。

\mxsay{毛:}你没有看过两千年以前我国的《孙子兵法》吧?里面很有些好东西。

\mxsay{蒙:}是不是提到了更多的军事原则?

\mxsay{毛:}一些很好的原则,一共有十三篇。

\mxsay{蒙:}我们应当从两千年以前回到现在了。

你同意不同意,我回到伦敦以后,在结束欧洲的军事占领和解决台湾这两个大问题上动员世界的舆论?你是否同意先从这两个问题开始?

\mxsay{毛:}好,我赞成。

\mxsay{蒙:}我可以使美国非常为难。

\mxsay{毛:}这里也有两条:一条就是你这样做;另一条就是美国人非常自高自大,他们是寸土不让的。

\mxsay{蒙:}我可以使美国非常为难。

\mxsay{毛:}有可能。

\mxsay{蒙:}我跟美国人很熟,在美国有很多朋友,他们的看法跟我一样。

\mxsay{毛:}我们的政策也是使美国为难。

\mxsay{蒙:}在美国,我有很多朋友会同意我的意见的。很多强大的报界人士也会同意我的。我过去从来也没有设法使美国为难,我想现在就要使它为难了。

\mxsay{毛:}美国现在很被动。有几百条绞索把美国捆起来,它在国外有二百五十个军事基地。

\mxsay{蒙:}我想应该对美国人讲一些不客气的老实话。

\mxsay{毛:}美国有一半的军队都捆在基地上。它有三百万军队,其中一百五十万在海外,包括在你们的英国和中国的台湾。我们在国外没有一个军事基地,没有一个兵。

\mxsay{蒙:}主席同意不同意我跟周恩来谈的关于美国应该遵守的那几条原则?那就是:第一,美国应该承认台湾是中国的一部分;第二,美国应该从台湾撤走;第三,台湾问题应该由中国和蒋介石谈判。

\mxsay{毛:}我知道,我也同意。我们不要同美国用战争解决问题。同蒋介石就不同了,但是如果他不用武力,我们也不用武力。

\mxsay{蒙:}这点我是同意的。

\mxsay{毛:}美国声明愿意通过和平谈判解决国际问题,而不使用武力或不以武力威胁。它这个话是否可靠还是个假定,还要等着看。可是蒋介石没有发表这样的声明,他反对同中国共产党谈判,而我们早就表示我们愿意同蒋介石通过谈判解决问题。

\mxsay{蒙:}你认不认识蒋介石?

\mxsay{毛:}他是我的老朋友,我怎能不认识?蒋介石就是经过我们的帮助才掌权的。在他没有掌权以前,我们同孙中山打交道。

\mxsay{蒙:}毛主席同蒋介石是否在抗日的时候合作过?

\mxsay{毛:}抗日合作了八年。后来他又同美国合作来打我们。

过去你们英国同日本有一个同盟,对付沙皇俄国。那时候,远东是你们的天下,中国主要是你们的势力范围。这种情况是什么时候改变的呢?第一次世界大战时开始变了。第二次世界大战后,日本你们就管不了啦,由美国管了。英国还同美国订了一项君子协定,把中国让给美国。这是克里浦斯夫人到延安时告诉我的。她说,在中国问题上,英国没有发言权了。从此以后,中国人民对英国的仇恨就消除了,中国人民的仇恨转向美国。日本投降以后,在中国的美国军队有九万人。

\mxsay{蒙:}可是过去的仇恨是针对英国的。

\mxsay{毛:}过去是对着英国,同时也是对着日本。

\mxsay{蒙:}我们曾经是最坏的洋鬼子。

\mxsay{毛:}过去也有日本,后来就成为日本和美国。

\mxsay{蒙:}你们反对美国,是不是因为美国派了马歇尔\mnote{6}将军来中国干涉中国内政?

\mxsay{毛:}日本就是在美国的帮助下才占了大半个中国。日本没有铁,没有石油,煤也很少。这三样东西都是美国源源不断地给日本送去的。但是,美国扶植了一个力量,却造成了一个珍珠港事件\mnote{7}。

\mxsay{蒙:}你们今天不怕日本了吧?

\mxsay{毛:}还有点怕,因为美国扶植日本的军国主义。

\mxsay{蒙:}日本是一个高度组织起来的工业国家。

\mxsay{毛:}美国在东方的主要基地是日本。本月十九日,日本在国会中强行通过了同美国的军事同盟条约\mnote{8}。

\mxsay{蒙:}日本对中国有没有什么坏的意图?

\mxsay{毛:}我看是有。

\mxsay{蒙:}什么样的意图?

\mxsay{毛:}当然主要是美国。日美条约把中国沿海地区,也包括在日本所解释的远东范围之内。

我读过艾登\mnote{9}的回忆录。他讲到苏伊士问题、埃及问题和伊朗问题,也谈到东南亚条约组织\mnote{10}问题。他说,美国在组织东南亚条约组织的时候,英国希望印度参加,美国坚决反对。美国说如果英国要印度参加,美国就要蒋介石和日本参加。

\mxsay{蒙:}印度是不会参加的。

\mxsay{毛:}那个时候,艾登想让印度参加来对付美国。艾登在回忆录中说,他想不通蒋介石怎么能同尼赫鲁\mnote{11}相提并论。

\mxsay{蒙:}我有一个有趣的问题想问一下主席:中国大概需要五十年,一切事情就办得差不多了,人民生活会有大大的改善,房屋问题、教育问题和建设问题都解决了,到那时候,你看中国的前途将会怎样?

\mxsay{毛:}你的看法是,那时候我们会侵略,是不是?

\mxsay{蒙:}不,至少我希望你们不会。

\mxsay{毛:}你怕我们会侵略。

\mxsay{蒙:}我觉得,当一个国家强大起来以后,它应该很小心,不进行侵略。看看美国就知道了。

\mxsay{毛:}对,很对,也可以看一看英国。第一次世界大战以前,世界上最强大的国家就是英帝国。一百八十年前的美国呢,只是英国的殖民地。

\mxsay{蒙:}历史的教训是,当一个国家非常强大的时候,就倾向于侵略。

\mxsay{毛:}要向外侵略,就会被打回来。到底是华盛顿\mnote{12}的北美强大,还是英帝国强大?但是,华盛顿用几枝烂枪,打了八年,把英帝国赶回去了。

\mxsay{蒙:}美国革命是件好事。革命往往是件好事。如果不是美国革命,加拿大就不是今天的加拿大。中国的革命也是好的。所以革命可以是好的。

\mxsay{毛:}你很开明!

\mxsay{蒙:}我是个军人。

\mxsay{毛:}外国是外国人住的地方,别人不能去,没有权利也没有理由硬挤进去。

\mxsay{蒙:}我同意。

\mxsay{毛:}如果去,就要被赶走,这是历史教训。

\mxsay{蒙:}五十年以后中国的命运怎么样?那时中国会是世界上最强大的国家了。

\mxsay{毛:}那不一定。五十年以后,中国的命运还是九百六十万平方公里。中国没有上帝,有个玉皇大帝。五十年以后,玉皇大帝管的范围还是九百六十万平方公里。如果我们占人家一寸土地,我们就是侵略者。实际上,我们是被侵略者,美国还占着我们的台湾。可是联合国却给我们一个封号,叫我们是“侵略者”。你在同一个“侵略者”说话,你知道不知道?在你对面坐着一个“侵略者”,你怕不怕?

\mxsay{蒙:}革命前,你们曾遭受过我们的侵略。

\mxsay{毛:}过去有过,现在那种仇恨没有了,只留了一点尾巴了。你们的政府只要改善一点态度,我们就可以同你们建立正式外交关系,互派大使。

\mxsay{蒙:}我希望如此。

\mxsay{毛:}如果英、法、苏、中四国能够比较接近,事情就会好些。

\mxsay{蒙:}我希望看到这种情况。

\mxsay{毛:}你们为什么不稍稍改善一点你们的态度呢?基本问题已经解决了,你们同台湾没有正式外交关系,同意北京政府代表中国,基本事情你们已经做了。只剩下个别问题,这就是:

一、在联合国讨论蒋介石代表权问题的时候,同美国站在一起;

二、在台湾你们还有领事;

三、你们的政府比较亲台湾而对中国疏远,有很多蒋介石的人从台湾到伦敦,受到你们外交部的接待。

此外,在西藏问题上,你们也同美国站在一起。西藏的一名叛乱分子到伦敦,受到你们外交部的负责人接见。

\mxsay{蒙:}这我不知道。西藏是在中国之内的。

\mxsay{毛:}你们外交部做的很多事情,你是不知道的。所以我看来,我们不能轻易地把正式代表权给英国,不能同英国正式互换大使。

\mxsay{蒙:}这是需要时间和等待的。

\mxsay{毛:}你们只要少许改善一下态度,我们的关系就会改善。

\mxsay{蒙:}我觉得你提到的关于英、法、俄、中这一个问题是很有趣的。我同麦克米伦\mnote{13}和戴高乐\mnote{14}是很熟的。戴高乐曾要我下个月到巴黎去同他会见,我将把这一点告诉他。戴高乐是一个很好的人。

\mxsay{毛:}我们对戴高乐有两方面的感觉:第一,他还不错;第二,他有缺点。

\mxsay{蒙:}人人都有缺点。

\mxsay{毛:}说他还不错是因为他有勇气同美国闹独立性。他不完全听美国的指挥棒,他不准美国在法国建立空军基地,他的陆军也由他指挥而不是由美国指挥。

\mxsay{蒙:}海军也是这样。

\mxsay{毛:}法国在地中海的舰队原来由美国指挥,现在他也把指挥权收回了。这几点我们都很欣赏。

另一方面他的缺点很大。他把他的军队的一半放在阿尔及利亚进行战争,使他的手脚被捆住了。

\mxsay{蒙:}戴高乐会说,阿尔及利亚是法国的一个省份,而在法律上戴高乐这样说是对的。

\mxsay{毛:}阿尔及利亚人可不同意,他们要求独立。

\mxsay{蒙:}麻烦就在这里,所以必须解决。但是法律上阿尔及利亚是法国的一个省份。这个问题必须解决。

\mxsay{毛:}阿尔及利亚问题应该解决。阿尔及利亚人告诉我,法国在阿尔及利亚有九十万军队,我觉得没有这么多,大概有五六十万。每天、每月、每年,法国都在阿尔及利亚消耗大量军费,这对法国很不利。

\mxsay{蒙:}这个问题必须解决。

\mxsay{毛:}是必须解决。法国军队不能打仗,在越南他们也打不过胡志明\mnote{15}部队。

\mxsay{蒙:}这个问题必须解决。

\mxsay{毛:}他们在阿尔及利亚打了六年。开头阿尔及利亚只有三千名游击队,现在已经发展到十万人的军队了。

\mxsay{蒙:}这个问题必须解决。戴高乐的地位在很大程度上取决于他能否解决这个问题,如果他解决不了,他可能被迫下台。

\mxsay{毛:}也会决定他是否能够同英国和美国一道在欧洲有平等的权利。

\mxsay{蒙:}他已经得到了。他曾经坚持这一点。

\mxsay{毛:}不完全如此,美国人不干。我们看到麦克米伦到法国访问、戴高乐到伦敦访问时受到隆重接待,我们感到很高兴。我们希望你们两个国家能够合作。

\mxsay{蒙:}麦克米伦可能是西方世界最好的政治领袖。

\mxsay{毛:}可能。至少他比艾森豪威尔好。

\mxsay{蒙:}谁会比他更好呢?我是指在西方世界里。

\mxsay{毛:}我们希望英国能够更加强大。

\mxsay{蒙:}他在西方集团是最聪明、最老实的人了。

\mxsay{毛:}人们可以看出,他比较有章法。

\mxsay{蒙:}我衡量一个政治领袖的标准是看他是否会为了地位而牺牲他的原则。你同意不同意这样一种标准?如果一个领袖为了取得很高的地位而牺牲他的原则,他就不是一个好人。

\mxsay{毛:}我的意见是这样的,一个领袖应该是绝大多数人的代言人。

\mxsay{蒙:}但是他也不能牺牲他的原则啊!

\mxsay{毛:}这就是原则,他应该代表人民的愿望。

\mxsay{蒙:}他必须带领人民去做最有利的事。

\mxsay{毛:}他必须是为了人民的利益。

\mxsay{蒙:}但是人民并不经常知道什么对他们最有利,领袖必须带领他们去做对他们有利的事情。

\mxsay{毛:}人民是懂事情的。终究还是人民决定问题。正因为克伦威尔\mnote{16}代表人民,所以国王才被迫让步。

\mxsay{蒙:}克伦威尔只代表少数人。

\mxsay{毛:}他是代表资产阶级反对封建主。

\mxsay{蒙:}但是他失败了。克伦威尔去世并且埋葬以后,过了几年,人家又把他的尸体挖出来,砍掉他的脑袋,并且把他的头在议会大厦屋顶上挂了好几年。

\mxsay{毛:}但是在历史上克伦威尔是有威信的。

\mxsay{蒙:}如果不是克伦威尔的话,英国就不是今天的英国了。

\mxsay{毛:}耶稣是在十字架上被钉死的,但是耶稣有威信。

\mxsay{蒙:}那是在他死以后,在他活着的时候,他没有很多的跟随者。

\mxsay{毛:}华盛顿是代表美国人民的。

\mxsay{蒙:}可是他被暗杀\mnote{17}了。

\mxsay{毛:}印度的甘地\mnote{18}也是被暗杀的,但是他是代表印度人民的。

\begin{maonote}
\mnitem{1}蒙哥马利(一八八七——一九七六),英国陆军元帅。第二次世界大战期间是盟军指挥官之一。后曾任英军总参谋长、北大西洋公约组织盟军最高副总司令。
\mnitem{2}艾森豪威尔,时任美国总统。杜勒斯,一九五三年至一九五九年任美国国务卿。
\mnitem{3}一九五〇年,在土耳其自由选举中,曼德列斯领导的民主党获胜,取得政权。曼德列斯执政后,背弃诺言,抛弃民主和宪法,压制反对派。一九六〇年五月二十七日,土耳其武装部队一批下级军官指挥伊斯坦布尔和安卡拉的部队发动军事政变,推翻了曼德列斯政府。曼德列斯被处死刑。
\mnitem{4}克劳塞维茨(一七八〇——一八三一),德国军事理论家,主要著作有《战争论》。
\mnitem{5}成吉思汗(一一六二——一二二七),元太祖,名铁木真,军事家和政治家。
\mnitem{6}马歇尔(一八八〇——一九五九),美国民主党人,前国务卿和国防部长。一九四五年十二月曾被美国总统派为驻华特使,以“调处”为名,参与国共谈判,支持国民党政府发动内战。一九四六年八月宣布“调处”失败,不久返回美国。
\mnitem{7}一九四一年十二月八日,日本未经宣战,突然袭击美国在太平洋地区的最大海空军基地珍珠港,击毁、击伤美国军舰十九艘,飞机二百余架,美国太平洋舰队遭到惨重损失。当日,美国对日本宣战,太平洋战争爆发。
\mnitem{8}指《日美共同合作和安全条约》,是一九五一年九月八日在旧金山签订的军事同盟条约。条约以维持日本“安全”为由,规定美国有权在日本驻扎军队和建立军事基地。一九六〇年一月,日美修改了该条约,重新签订了《日美共同合作和安全条约》。主要内容是:发展日本“抵抗武装进攻的能力”;在“应付共同危险”时,日本负有保护驻日美军的义务;美国继续有权在日本驻军和使用军事基地;“鼓励两国经济合作”等。
\mnitem{9}艾登(一八九七——一九七七),英国保守党人,英国前首相、外交家。
\mnitem{10}一九五四年九月八日,在美国策动下,由美国、英国、法国、澳大利亚、新西兰、菲律宾、泰国和巴基斯坦在菲律宾首都马尼拉签订了《东南亚集体防务条约》,又称《马尼拉条约》。这是一个军事同盟条约,条约声明要用“自助和互助的办法”“抵抗武装进攻”。条约附有美国提出的“谅解”,对“侵略和武装进攻的意义”解释为“只适用于共产党的侵略”。条约还以议定书的形式,把柬埔寨、老挝和南越划为它的“保护地区”。一九五五年二月十九日条约生效时成立了东南亚条约组织。一九六二年七月日内瓦会议通过的《关于老挝中立的宣言》,不承认它对老挝的所谓保护。一九六七年起法国拒绝派正式代表团参加该组织的部长级理事会。一九七二年十一月八日巴基斯坦宣布退出。一九七七年六月该组织宣布解散。
\mnitem{11}尼赫鲁,时任印度总理。
\mnitem{12}华盛顿(一七三二——一七九九),美国第一任总统。一七七五年美国独立战争爆发后,任大陆军总司令,将武装落后、组织松散的地方民军整编训练成为能与英军正面抗衡的正规军,领导美国取得独立战争的胜利。
\mnitem{13}麦克米伦(一八九四——一九八六),英国保守党人,时任英国首相。
\mnitem{14}戴高乐(一八九〇——一九七〇),时任法国总统。
\mnitem{15}胡志明,时任越南劳动党中央委员会主席、越南民主共和国主席。曾领导越南人民进行抗法救国战争。
\mnitem{16}克伦威尔(一五九九——一六五八),十七世纪英国资产阶级革命时期的主要军事、政治领导人,独立派领袖,资产阶级新贵族集团的代表人物。他率军战胜国王军队,一六四九年宣布成立共和国。一六五三年自任“护国主”,建立军事独裁政权。
\mnitem{17}这里蒙哥马利的记忆有误,华盛顿是一七九九年十二月十四日在家乡芒特弗农病逝的。他似乎指的是林肯。
\mnitem{18}甘地(一八六九——一九四八),印度民族独立运动领袖,曾当选为印度国大党主席。长期领导反对英国殖民统治、争取印度独立的斗争,一九四八年一月三十日遇刺身亡。
\end{maonote}
