
\title{马列主义基本原理至今未变,个别结论可以改变}
\date{一九五九年二月十四日}
\thanks{这是毛泽东同志同智利《最后一点钟》报社社长阿图罗·马特·阿历山德里谈话的节录。}
\maketitle


\mxsay{马特:}我有些理论问题不清楚,想在这里提一提。主席先生是世界上第一流的理论家,不知有无时间解答?

\mxsay{毛泽东:}你这个说法不对。

\mxsay{马:}理论著作我看过很多。我认为只有列宁和您的著作最好。

\mxsay{毛:}忙于工作,没有充分的时间研究理论问题。

\mxsay{马:}我认为,所谓马列主义是指马克思和列宁两人的著作。其中属于认识论的唯物主义是普遍的真理,还有一部分是属于具体实用的而必须在实践中加以考察的学说。就此而言,马列主义中属于后一部分的学说是否需要不断改变?

\mxsay{毛:}你这个讲法不合适,马列主义至今未变。唯物主义并不等于马列主义,在马克思主义产生以前就已经有唯物主义,资产阶级曾经发挥了唯物主义,例如法国的唯物主义。辩证法也不是马克思发现的,例如德国过去有唯心辩证法。马克思是改造了这两种东西。他把唯物主义改造成为辩证唯物主义,认为世界是联系的、发展的。为什么会有发展呢?因为有矛盾存在。他把辩证法改造成为唯物辩证法。唯物辩证法是正确反映客观世界的辩证法,这与德国黑格尔的唯心辩证法不同。至于马克思、列宁关于个别问题的结论做得不合适,这种情况是可能的,因为受当时条件的限制,例如马克思关于无产阶级革命首先在西方几个国家同时取得胜利的结论。

\mxsay{马:}这正是我要提的问题。

\mxsay{毛:}列宁已经解决了这个问题。

\mxsay{马:}列宁改变了一些马克思主义学说,说无产阶级革命可以在单独一个国家内取得胜利。

\mxsay{毛:}马克思、恩格斯说过无产阶级革命将在几个经济文化比较发达的国家同时发生,现在改变了这个结论。例如,俄国经济比西欧落后,却首先取得了革命的胜利。这就证明,无产阶级革命是可以首先在一个国家取得胜利的。

\mxsay{马:}这本来是没有预料到的,即使是列宁,也是事后才看出来的。

\mxsay{毛:}列宁在十月革命以前就已经发现。

\mxsay{马:}但是在二月革命之后。

\mxsay{毛:}列宁在从事研究工作的后期,指出了资本主义发展的帝国主义阶段,看出了资本主义发展不平衡的规律,有可能引起世界大战,有可能首先在一国或几国爆发革命,在东方国家中也有这种可能。

\mxsay{马:}列宁是在以后才看出的。

\mxsay{毛:}列宁在第一次世界大战期间研究了这个问题,并且得出结论:无产阶级革命可以首先在一个国家取得胜利\mnote{1}。当然,他没有预料到在什么国家首先取得胜利。

\mxsay{马:}列宁说专政只能由一个政党执行,而主席先生说无产阶级专政可以由几个政党联合执行,这是不是离开了马列主义?

\mxsay{毛:}不能说离开了马列主义。中国的民主革命,可以说是几个政党联合进行的,但是以共产党为首。中国有两次革命:一九四九年以前是资产阶级民主革命,解决反帝、反封建、反国民党统治的问题,那时我们没有触动民族资产阶级的所有制;后一次革命,想必你也知道,就触及了民族资产阶级的所有制以及个体手工业与个体农业的所有制。对这些所有制进行社会主义改造经过了几个步骤。后一次革命经过几年,已基本完成。这两次革命,都是以共产党为领导进行的。

\mxsay{马:}中国解放的第二天我就知道了。中国进行社会主义改造是解放以后的事,中国走社会主义的道路,却早已知道了。

\mxsay{毛:}我们先动买办资本,而对民族资本我们好几年未动。至于个体农民,我们首先分配土地为农民所有,第二步才将他们组织起来,搞合作化。对于民族资本,我们分几个步骤进行改造,最后一个步骤就是公私合营。国家采取了和平赎买政策,分好几年赎买。现在,原属民族资本的那些企业都是由国家管理。

中国过去的资本主义,主要是官僚资本和帝国主义资本,占百分之八十,民族资本只占百分之二十。官僚资本和帝国主义资本归国家所有以后,资本的主要部分就已归国家所有了。对民族资本我们采取和缓的政策。

\mxsay{马:}很同意中国的做法。

\mxsay{毛:}资本家现在同我们合作。

\mxsay{马:}还有一个理论问题,为什么马克思主义先在东方经济落后的国家取得胜利?

\mxsay{毛:}这个道理很简单。东方国家的统治者不能解决他们国家的问题,东方国家人民首要的任务是反帝、反封建。这些国家被西方国家剥削得很贫困。这里讲的东方,包括亚洲、非洲和拉丁美洲,即所谓帝国主义的后方。

\mxsay{马:}马列主义有关实用的部分,现在有些已不适用了。比如西方国家,社会经济是垄断的,但政治上并不尽然,因为还有自由,这个自由并不影响社会的发展。所以,社会经济制度和政治上的专制这两个东西,是否可以分开来?

\mxsay{毛:}这首先要对专政作分析,西方国家的所谓自由,实际上是资本家有剥削的自由。

\mxsay{马:}同意。中国是把人们从经济上解放出来。

\mxsay{毛:}西方国家的政权不可能由多数人掌握,而只由资本家统治,这就是独裁。

\mxsay{马:}那些国家的共产党为什么不上台呢?

\mxsay{毛:}因为资本家不允许。

\mxsay{马:}在智利,共产党上台是有可能的。

\mxsay{毛:}今天还是投资产阶级的票的人多。

\mxsay{马:}为什么呢?

\mxsay{毛:}中国过去也是如此,赞成我们的人开始也很少。经过二十八年,自一九二一年至一九四九年,我们逐渐取得了人民的拥护。

\mxsay{马:}西方国家已经经历了好几百年了,再过一百年也还会是那样。

\mxsay{毛:}但是今后的情况是会改变的。

\mxsay{马:}当然会这样,这就需要一个新的方案。

\mxsay{毛:}马克思活着的时候,不能将后来出现的所有的问题都看到,也就不能在那时把所有的这些问题都加以解决。俄国的问题只能由列宁解决,中国的问题只能由中国人解决。

马西方国家和拉丁美洲国家问题的解决将要受十月革命和中国革命的影响,但是你们说这将是马列主义的胜利,而不说是运用辩证唯物主义的胜利。如果你们对马列主义采取这种态度,拉丁美洲人民就会怀疑马列主义。

\mxsay{毛:}世界观是辩证唯物主义,这是共产党的理论基础。无产阶级专政与阶级斗争的学说是革命的理论,即运用这个世界观来观察与解决革命问题的理论。

马列主义应包含三部分:一、马列主义的哲学,这是理论基础;二、马列主义的经济学,这是用马列主义的观点来考察经济现象的学说;三、马列主义的革命学说,比如关于阶级斗争、政党、无产阶级专政等的学说。这三部分不能分割,而应视为马列主义的三个有机联系的组成部分。

\mxsay{马:}世界观不变,是不是革命理论部分是可以改变的?

\mxsay{毛:}要看是什么理论。比如革命将首先在一国或几个国家内取胜的理论,是改变了马、恩的结论的。

\mxsay{马:}如果把今天谈话的内容发表出去,会不会有人认为主席先生背叛了马列主义?

\mxsay{毛:}我不认为中国背叛了马列主义。中国的党一贯遵守马列主义的原则,因为它是普遍的真理。这是普遍真理与中国具体情况的统一的问题。

\begin{maonote}
\mnitem{1}见列宁《论欧洲联邦口号》(一九一五年八月):“社会主义可能首先在少数甚至在单独一个资本主义国家内获得胜利。”又见列宁《无产阶级革命的军事纲领》(一九一六年九月):“社会主义不能在所有国家内同时获得胜利。它将首先在一个或者几个国家内获得胜利。而其余的国家在一段时间内将仍然是资产阶级的或资产阶级以前的国家。”(《列宁选集》第2卷,人民出版社1995年版,第554、722页)
\end{maonote}
