
\title{会见美国总统尼克松时的谈话}
\date{一九七二年二月二十一日}
\maketitle


\mxsay{尼克松:}你读过很多书。总理说你读的书比他多。

\mxsay{毛泽东:}昨天你在飞机上给我们出了一个难题。说是我们几个要吹的问题,限于哲学\mnote{1}方面。(众笑)

\mxsay{尼克松:}我之所以那么说,是因为我读过主席的诗词和讲话,我知道主席是一位思想深刻的哲学家。

\mxsay{毛泽东:}(手指基辛格博士\mnote{2})他是个哲学博士?

\mxsay{尼克松:}他是一位思想博士。

\mxsay{毛泽东:}(手指基辛格博士)今天主讲要请他,博士,philosopher(哲学家),哲学博士。

\mxsay{尼克松:}他是一位哲学专家。

\mxsay{基辛格:}我在哈佛大学教书时,指定我的学生要阅读主席的全集。

\mxsay{毛泽东:}我的那些东西没什么。我写的东西里面没什么教育意义。(望着摄影师们)现在他们想干扰我们的会谈,我们这儿的秩序。

\mxsay{尼克松:}主席的著作推动了一个国家,改变了这个世界。

\mxsay{毛泽东:}我没能力改变世界。我顶多改变北京附近的几个地方。

我们共同的老朋友,就是说蒋介石委员长,他不赞成。他说我们是共匪。他最近还发表了一篇讲话\mnote{3}。你见到啦?

\mxsay{周恩来:}就是在他们最近召开的“国会”上。

\mxsay{尼克松:}蒋介石把主席称为共匪。主席怎么称呼蒋介石?

\mxsay{周恩来:}我们一般称他们为蒋介石集团。在报纸上有时也称他为匪。他们也回敬我们为匪。不管如何,我们是互相对骂罢了。

\mxsay{毛泽东:}那还不是匪?彼此叫匪,互相对骂。其实我们跟他作朋友的时间比你跟他做朋友的时间长得多。

\mxsay{周恩来:}从一九二四年开始。

\mxsay{尼克松:}是的,我知道。

\mxsay{毛泽东:}我们两个不要垄断整个谈话。不让基辛格博士说说不行。你(指基辛格)跑中国跑出了名嘛,头一次来\mnote{4},公告发表以后,全世界都震动了。

\mxsay{基辛格:}是总统确定方向,决定方案。

\mxsay{尼克松:}他这么讲,说明他是个绝顶聪明的助手。(毛和周笑)

\mxsay{毛泽东:}他在夸你,说你这样做很聪明。

\mxsay{尼克松:}他看起来不像一个特工人员。他是唯一有本事在不自由的情况下去巴黎十二次,来北京一次,却没人知道的人,可能有两三个漂亮姑娘除外。(周恩来大笑)

\mxsay{基辛格:}她们可不知道。我是用来做掩护的。

\mxsay{尼克松:}利用漂亮姑娘做掩护的人,一定是有史以来最伟大的外交家。

\mxsay{毛泽东:}你们的姑娘常被人利用啊?

\mxsay{尼克松:}他的姑娘,不是我的。如果我用姑娘做掩护,我可就有大麻烦啦。

\mxsay{周恩来:}(大笑)特别是大选的时候。(基辛格大笑)基辛格博士不竞选总统,因为他不是美国出生的公民。

\mxsay{基辛格:}唐小姐有资格当美国总统。

\mxsay{尼克松:}那她将是第一位女总统。我们有了候选人了。

\mxsay{毛泽东:}如果有这样的候选人,可就很危险了。讲老实话,这个民主党如果再上台,我们也不能不同它打交道。

\mxsay{尼克松:}我们理解。我们希望我们不使你们面对那样的问题。

\mxsay{毛泽东:}这些问题不是在我这里讨论,而是应该和总理讨论。我的问题是哲学问题,就是说,你竞选时我投了你一票。这里有个美国人叫弗兰柯,在贵国尚处混战,就是你上次竞选的时候,他写了篇文章,说你会当选。我很欣赏那篇文章。但现在他反对你的访问。

\mxsay{尼克松:}我想主席投我一票,是在两个坏东西中间选择好一点的一个。

\mxsay{毛泽东:}我喜欢右派\mnote{5}。人家说你是右派,说你们共和党是右派。

\mxsay{尼克松:}是的。

\mxsay{毛泽东:}那位首相希思\mnote{6}也是右派。

\mxsay{尼克松:}戴高乐将军\mnote{7}也是。

\mxsay{毛泽东:}戴高乐是另一回事。他们还说西德的基督教民主党也是右派。我喜欢右派,比较高兴这些右派当政。。

\mxsay{尼克松:}我想重要的是,在美国,至少现在,像我这样的右派可以做那些左派只能口头上说说的事情。(毛泽东点头)

\mxsay{基辛格:}总统先生,我觉得那些左派的人是亲苏的,他们不鼓励我们向人民共和国这边靠拢,而且批评你这样做。

\mxsay{毛泽东:}就是嘛。有些人在反对你。我们国内有一派也反对我们跟你们往来,结果坐一架飞机跑到外国去了。\mnote{8}

\mxsay{周恩来:}也许你们知道这件事。

\mxsay{毛泽东:}全世界的侦察就只有美国的比较准确,其次就是日本。至于苏联,他们就在那里挖尸,但什么都不说。

\mxsay{周恩来:}在外蒙古。

\mxsay{尼克松:}最近在印度—巴基斯坦危机中,我们也碰到了同样的问题。美国的左派非常严厉地批评我不站在印度一边。这有两个原因:一个是他们亲印度,另一个他们亲苏联。我认为,着眼于更大情势是重要的。不能让一个国家,无论它多么强大,去吞并邻国。这让我付出了政治代价——我并不后悔,因为我做得对——历史会证明我这样做是对的。

\mxsay{毛泽东:}提个建议,只是建议,你少发点简报好不好?(总统指着基辛格博士和周大笑)如果你把我们谈的这些,我们讨论的哲学,向其他人通报,你认为好吗?

\mxsay{尼克松:}主席可以放心,我们所讨论的一切,或者我与总理所讨论的一切,统统不会泄露。这是进行最高层会谈的唯一办法。

\mxsay{毛泽东:}那就好。

\mxsay{尼克松:}如果可能,我希望跟总理,以及以后跟主席除了讨论眼前的问题:台湾问题、越南问题、朝鲜问题而外,……

\mxsay{毛泽东:}这些问题我不感兴趣,那是他(指周总理)跟你谈的事。我看你的题目更好——哲学问题。

\mxsay{尼克松:}世界上大多数国家都是赞成我这次访问的,苏联不赞成;日本是怀疑的,它已经表示了这种怀疑;印度不赞成。所以,我们要研究为什么会这样,并决定我们的政策,看就全世界来说,我们应如何发展,而不是看眼前的问题。当然,朝鲜、越南、台湾这些问题也要讨论。

\mxsay{毛泽东:}对,赞成。

\mxsay{尼克松:}比如说,我们必须问自己,苏联为什么向贵国边界部署的兵力比向西欧边界部署的兵力多?日本的未来是什么?我们对此存在分歧,那么,是让日本中立和全无防卫更好?还是让它在一个时期内同美国具有一些关系更好?我现在讲的属于哲学范畴,问题在于,在国际关系领域没有好的选择。有一点可以肯定:我们不能留下真空,因为真空会被填补。例如,总理指出,他感到美国在伸手,苏联在伸手,那么问题是,人民共和国面临的危险,是来自美国的侵略,还是来自苏联的侵略?这是一个困难的问题,但是我们要讨论这个问题。

\mxsay{毛泽东:}来自美国侵略的问题,或者来自中国侵略的问题,比较小。也可以说不是个大问题。现在不存在我们两个国家打仗的问题。你们想撤一些兵回国,我们的兵也不出国。因此,我们两国间的状态很奇怪,过去二十二年总是谈不拢。现在从我们开始打乒乓球\mnote{9}不到十个月。如果从你们在华沙提出建议时算起,也不到两年。我们这边办事也有官僚主义。比如,你们想搞人员往来、贸易。我们就是死不肯,坚持解决不了大问题,小问题就不干。我自己也这么坚持过。后来我看还是你们对,我们就打起了乒乓球。总理说这也是尼克松总统上台后的事。

巴基斯坦前总统把尼克松总统介绍给我们。那时,我们驻巴基斯坦大使不同意我们同你们交往。他说要比较一下约翰逊总统\mnote{10}或尼克松总统哪个更好。但是叶海亚\mnote{11}说这两个人没法比。他说一个像土匪——指约翰逊总统。我不知他从哪儿得到那么个印象。我们也不太喜欢他。从杜鲁门\mnote{12}到约翰逊,我们也都不那么高兴。

这个中间有八年的共和党\mnote{13},那个时候,你们也没有想通。

\mxsay{周恩来:}主要是约翰·福斯特·杜勒斯\mnote{14}的政策。

\mxsay{毛泽东:}之前他(周恩来)和基辛格博士讨论过这事。

\mxsay{尼克松:}但是他们(朝着周总理和基辛格博士)握了手。(周恩来笑)

\mxsay{毛泽东:}你有什么话要说的,博士?

\mxsay{基辛格:}主席先生,世界形势在那个时期也已发生急剧变化。我们学到了很多东西。我们原以为所有社会主义、共产主义国家都千篇一律。直到总统当政,我们才理解中国革命的不同性质,和其他社会主义国家发展的革命道路。

\mxsay{尼克松:}主席先生,我知道,我多年来对人民共和国的立场,是主席和总理所完全不同意的。我们现在走在一起来了,是因为我们承认存在着一个新的世界形势。我们承认重要的不是一个国家的对内政策和它的哲学,重要的是它对世界上其他国家的政策以及对于我们的政策。这就是为什么(这一点我认为可以实话实说)我们存在分歧的原因。总理和基辛格博士讨论过这些分歧。

\mxsay{毛泽东:}就是嘛。

\mxsay{尼克松:}我还想说的是,审视两个大国,美国与中国,我们知道中国并不威胁美国的领土。

\mxsay{毛泽东:}也不威胁日本和南朝鲜\mnote{15}。

\mxsay{周恩来:}任何国家都不威胁。

\mxsay{尼克松:}我们也不威胁别人,我想你们也知道美国对于中国也没有领土要求。我们知道中国不想统治美国,我们认为你们也懂得美国不想统治中国。同时,我相信,当然你们可能不会相信,美国和中国都是伟大的国家,它们都不想统治世界。正因为我们这两个国家在这些重大问题上态度相同,所以我们相互并不构成威胁。因此,我们虽然有分歧,但是可以找到共同点来建立一个世界结构,一个我们都可以在其中安全地发展自己、各走各的路的结构。对世界上另外一些国家谈不上这一点。

\mxsay{毛泽东:}你们下午还有事情?现在几点了?

\mxsay{周恩来:}四点半开全体会,现在是三点三刻。

\mxsay{毛泽东:}吹到这里差不多了吧?

\mxsay{尼克松:}是的。我想在结束时说,主席先生,我们知道你和总理冒了很大风险邀请我们到这里来。这对我们也是一个很困难的决定。但是,我读了主席的一些讲话,知道主席是一个一旦机会来临就能看到的人,也知道你一定要“只争朝夕”。

\mxsay{毛泽东:}(指着基辛格博士)“只争朝夕”的是他。

\mxsay{尼克松:}从个人的意义上说,你和总理对我都是不了解的,因此你们不应该信任我。但是你们会发现,我不能做的就决不说,但我做的比说的要多。我就是想在这样的基础上,同主席和总理坦率地交换意见。

\mxsay{毛泽东:}大概我这种人放大炮的时候多。无非是“全世界团结起来,打倒帝、修、反”这一套,建立社会主义。

\mxsay{尼克松:}(微笑)就是像我这样的人,还有匪徒。

\mxsay{毛泽东:}你可能就个人来说,不在打倒之列。可能他(指基辛格)也不在内。都打倒了,我们就没有朋友了嘛。

\mxsay{尼克松:}(笑)就没有靶子了。

\mxsay{尼克松:}主席的一生我们都是熟悉的。你出身于一个贫穷的家庭,现在到达世界上人口最多的国家、一个伟大的国家的顶峰。我的背景,不那么被人所知,我也是出身于贫穷家庭,现在达到了一个大国的顶峰。我感到,是历史把我们带到一起来的。问题是,我们的哲学是不同的,但我们都脚踏实地,都来自于人民,我们可以实现一个突破。这种突破不仅将有益于中美两国,而且在今后的岁月中会有益于全世界。我就是为此而来的。

\mxsay{毛泽东:}你的《六次危机》\mnote{16}写得不错。

\mxsay{尼克松:}(对着基辛格)他(指毛泽东)读的书太多。

\mxsay{毛泽东:}太少。我对美国了解不多。我要请你派一些老师来,主要是历史和地理老师。

\mxsay{尼克松:}好哇,太好了。

\mxsay{毛泽东:}所以我跟早几天去世的记者斯诺\mnote{17}说过,我们谈得成也行,谈不成也行,何必那么僵着呢?一定要谈成?

\mxsay{尼克松:}他的死很令人悲伤。

\mxsay{毛泽东:}人们会说话的。一次没有谈成,无非是我们的路子走错了。那我们第二次又谈成了,你怎么办啊?(双方站起来)

\mxsay{尼克松:}(握着毛泽东的手)我们在一起可以改变世界。

\mxsay{毛泽东:}我就不送你了。

\begin{maonote}
\mnitem{1}[合众国际社关岛阿加尼亚二月二十日电]尼克松总统今天说,他准备同中国领导人进行马拉松式的会谈,如果这些会谈证明在缓和中美紧张局势方面有成果的话。

尼克松在他的蓝、白、银三色的“七六年精神号”喷气式飞机上对记者们说:“我们的主人想参加会谈多久,我就准备参加会谈多久。”

总统说,他期望他同共产党主席毛泽东和周恩来总理的谈话从哲学的角度来进行,而不是只集中讨论眼前的问题。……

尼克松说,毛和周都是“有哲学头脑的人物,他们不是仅仅讲究实际的、注意日常问题的领导人”。

他说:“他们是一些眼光看得很远的人。”

他说:“我自己对世界上的长期的和双边的问题的态度不是策略性的。美国领导人的眼光必须看得很远——我们的政策辩论必须根据一项妥善地制订、并且为人们充分了解的哲学,这是我们国际关系的基础。”

这里的“哲学”实际上是“战略”的同义语。
\mnitem{2}基辛格,时任国家安全事务助理。他出生在德国,后移民美国。
\mnitem{3}指蒋介石二月二十日“在国民大会第五次会议开会典礼中的致词”。其中讲了“我们和共匪”之间的“区别”,诬蔑“大陆的精神文化”被“摧残”,“大陆七亿人民的生活”已是“血干泪尽”。接着说:“试问这是一个什么‘政权’?这如何能说它对大陆有了‘有效的控制’?……所以今天国际间任何与恶势力谋求政治权力均衡的姑息举动,绝不会有助于世界和平,而适以延长我七亿人民的苦难,增大全世界的灾祸!我们对任何有损于中华民国主权利益的行动,保有高度的警惕!”在这一“致词”中,蒋介石未提美国,未提尼克松。他在尼克松抵达北京的前一天讲了这番话,矛头所指,不言自明。
\mnitem{4}“头一次来”,是指基辛格一九七一年七月九日至十一日秘密访华;当时周恩来同他商定了“尼克松总统于一九七二年五月以前的适当时间访问中国”的公告,由双方以各自不同的方式于七月十五日同时发表。
\mnitem{5}“右派”,是作为“哲学问题”讲的。从“哲学”即“战略”上讲,当时西方一些国家中,在“苏攻美守”的形势下,右派主张对苏强硬,或可称之为抗苏派,左派主张对苏妥协,或可称之为亲苏派。为了抗苏,也就主张改善对华关系。由于对苏妥协,往往不愿或不敢接近中国。正是在这个意义上,毛泽东把尼克松、基辛格说成右派。
\mnitem{6}指爱德华·希思,一九七零年至一九七四年担任英国保守党政府首相。
\mnitem{7}戴高乐将军,曾任法国第五共和国总统,与一九七零年十一月九日去世。
\mnitem{8}一九七一年九月十三日夜,林彪一家驾机逃往苏联,最后坠毁于蒙古温都尔汗,史称“九一三”事件。
\mnitem{9}打乒乓球,指一九七一年中国邀请美国乒乓球队访华事件。

事件起于同年三、四月在日本名古屋举行的第三十一届世界乒乓球锦标赛。开赛前夕,周恩来召集有关人士开会时要求这次参赛“接触许多国家的代表队”,“我们也可以请他们来比赛”。同时他要在座的人“动动脑筋”。比赛开始第一天,中国队乘巴士从住地去体育馆时,美国运动员科恩上来搭车,于是中国运动员庄则栋主动和他握手、寒喧,并送他一块中国杭州织锦留作纪念。这个细节被在场记者抓住,成为爆炸性新闻。四月三日中国外交部以及国家体委就是否邀请美国乒乓球队访华问题向中央请示。经过三天的反复考虑,毛泽东在比赛闭幕前夕决定邀请美国队访华。次日,美国国务院接到驻日本大使馆《关于中国邀请美国乒乓球队访华的报告》,立即向白宫报告。尼克松在深夜得知这个消息后,立即发电报给美国驻日大使,同意中方的邀请。事后尼克松说:“我从未料到对中国的主动行动会以乒乓球队访问北京的形式得到实现。”

一九七一年四月十日早晨,美国乒乓球队到达深圳。在十三日下午北京西郊举行了一场中美乒乓球友谊赛。十四日下午,周恩来总理在人民大会堂会见了美国乒乓球代表团全体成员,在与美代表团团长谈话时说:“你们作国前来中华人民共和国访问的第一个美国代表团,打开了两国人民友好往来的大门。尽管中国和美国目前还没有外交关系,我相信中美两国人民的友好往来,将会得到两国大多数人民的赞成与支持。”后美方领队说,也希望中国乒乓球在最近的将来访问美国。在周恩来接见美国乒乓球的同时,美国白宫也在密切注视着这一重要活动。

周恩来讲话后不到几小时,白宫就宣布了旨在缩小两国间鸿沟的一系列开禁措施:放松美国对中国实行了二十一年的禁运,对愿意访问美国的中国人可以加快发给签证,放宽货物管制等等。尼克松还高兴地宣布:“美国的对华政策已经打开了坚冰,现在就要测水有多深了!我希望,其实我是期待着,有一天我将以某种身份访问大陆中国。”

小球推动大球获得成功。

三个月后的七月九日清晨,尼克松总统的特使——国家安全事务助理基辛格博士秘密抵达北京,同周恩来进行了高级会谈。

通过乒乓球来实现中美的外交谈判,可以说是小球转变大球的成功之处,是全世界为之叹服的创举。

这就是乒乓外交。
\mnitem{10}约翰逊,一九六三年十一月二十二日,美国总统肯尼迪遇刺身亡,时任副总统的约翰逊就任总统,并在一九六五年正式当选美国总统,任期到一九六九年。
\mnitem{11}叶海亚,叶海亚·汗,原任巴基斯坦总统,他曾利用与中国和美国都保持的良好关系,为中美秘密传话,建立了著名的“叶海亚汗通道”,并成功安排基辛格秘密访华。
\mnitem{12}杜鲁门,一九四五年,美国总统罗斯福在任期内逝世,时任副总统的杜鲁门就任总统,并在一九四八年正式当选美国总统,任期到一九五三年。杜鲁们对新中国采取敌视态度,一九五零年六月二十五日朝鲜战争爆发。六月二十七日杜鲁门命令第七舰队开进台湾海峡,阻止我解放台湾,并宣称台湾地位未定。不久,美国在台湾设立军事基地。一九五四年十二月,美蒋签订“共同防御条约”。
\mnitem{13}一九五三年至一九六一年,美国由共和党执政,艾森豪威尔任总统,尼克松任副总统。
\mnitem{14}杜勒斯,美国共和党人,一九五三年至一九五九年任美国国务卿。他在国际活动中,鼓吹冷战,推行“战争边缘”“大规模核报复”以及对社会主义国家进行“和平演变”等战略。一九五零年他参与策划美国政府利用朝鲜战争武装侵占中国领土台湾。一九五四年他又策划美国政府同台湾当局签订《美台共同防御条约》,企图使霸占台湾的行为合法化,将台湾长期作为美国的军事基地。他一贯敌视社会主义国家和民族解放运动,坚持不承认中国、非法排斥中国在联合国的合法地位、对中国实行封锁禁运,明目张胆地进行制造“两个中国”的阴谋活动。
\mnitem{15}南朝鲜,今大韩民国,简称韩国。
\mnitem{16}尼克松一九六二年写了《六次危机》一书,记叙他自己的生活经历,自道短长,自言甘苦。
\mnitem{17}斯诺,(一九〇五——一九七二年)美国新闻记者、作家,埃德加·斯诺,美国进步作家、记者。一九二八年第一次到中国。一九三六年到陕北革命根据地访问,见到了毛泽东等中共和红军的领导人,后写了《西行漫记》等书。新中国成立后,在一九六〇年、一九六四年、一九七〇年访问中国。一九七二年二月十五日因病在瑞士日内瓦逝世。

毛泽东高度评价斯诺先生,“斯诺先生是中国人民的朋友。他一生为增进中美两国人民之间的相互了解和友谊进行了不懈的努力,作出了重要的贡献。他将永远活在中国人民心中。”

遵照其遗愿,其一部分骨灰葬在中国,地点在北京大学未名湖畔。一九七三年十月十九日,在斯诺骨灰的安葬仪式上,斯诺夫人洛伊斯牵着女儿茜安的手说:“我的丈夫在遗言中表达了对中国的热爱,并表示他生前一部分身心常系中国,希望死后也将自己一部分遗体安放在新中国的古老土地下,安放在中国的新人中间。在这里,对人类的尊重达到了新的高度;在这里,世界的希望发射着新的光芒。”
\end{maonote}
