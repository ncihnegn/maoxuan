
\title{批判离开总路线的右倾观点}
\date{一九五三年六月十五日}
\thanks{这是毛泽东同志在中共中央政治局会议上的讲话的一部分。在讲话中,毛泽东同志批判了刘少奇等人提出的“确立新民主主义社会秩序”等右倾机会主义观点。}
\maketitle


党在过渡时期\mnote{1}总路线和总任务,是要在十年到十五年或者更多一些时间内,基本上完成国家工业化和对农业、手工业、资本主义工商业的社会主义改造。这条总路线是照耀我们各项工作的灯塔。不要脱离这条总路线,脱离了就要发生“左”倾或右倾的错误。

有人认为过渡时期太长了,发生急躁情绪。这就要犯“左”倾的错误。有人在民主革命成功以后,仍然停留在原来的地方。他们没有懂得革命性质的转变,还在继续搞他们的“新民主主义”,不去搞社会主义改造。这就要犯右倾的错误。就农业来说,社会主义道路是我国农业唯一的道路。发展互助合作运动,不断地提高农业生产力,这是党在农村中工作的中心。

右倾的表现有这样三句话:

“确立新民主主义社会秩序”。这种提法是有害的。过渡时期每天都在变动,每天都在发生社会主义因素。所谓“新民主主义社会秩序”,怎样“确立”?要“确立”是很难的哩!比如私营工商业,正在改造,今年下半年要“立”一种秩序,明年就不“确”了。农业互助合作也年年在变。过渡时期充满着矛盾和斗争。我们现在的革命斗争,甚至比过去的武装革命斗争还要深刻。这是要把资本主义制度和一切剥削制度彻底埋葬的一场革命。“确立新民主主义社会秩序”的想法,是不符合实际斗争情况的,是妨碍社会主义事业的发展的。

“由新民主主义走向社会主义”。这种提法不明确。走向而已,年年走向,一直到十五年还叫走向?走向就是没有达到。这种提法,看起来可以,过细分析,是不妥当的。

“确保私有财产”。因为中农怕“冒尖”,怕“共产”,就有人提出这一口号去安定他们。其实,这是不对的。

我们提出逐步过渡到社会主义,这比较好。所谓逐步者,共分十五年,一年又有十二个月。走得太快,“左”了;不走,太右了。要反“左”反右,逐步过渡,最后全部过渡完。


\begin{maonote}
\mnitem{1}这里所说的“过渡时期”,是指从中华人民共和国成立,到社会主义改基本完成这一时期。党在这个过渡时期的总路线和总任务,是要在一个相当的时期内,基本上完成国家工业化和对农业、手工业、资本主义工商业的社会义改造。这个过渡时期和毛泽东同志在一九六二年九月党的八届十中全会及以后所说的过渡时期,含义不同,后者是指由资本主义过渡到共产主义的整个史时期。
\end{maonote}
