
\title{西藏要准备对付那里的可能的全局叛乱}
\date{一九五八年六月二十四日}
\thanks{这是毛泽东同志转发青海省委关于镇压叛乱问题的报告的批语。}
\maketitle


此件\mnote{1}请克诚\mnote{2}同志印发军委会议各同志。同时请小平印发各省、市、自治区党委,先发川滇黔藏。使大家知道这件事。青海反动派叛乱,极好,劳动人民解放的机会就到来了。青海省委的方针是完全正确的。西藏要准备对付那里的可能的全局叛乱。乱子越大越好。只要西藏反动派敢于发动全局叛乱,那里的劳动人民就可以早日获得解放,毫无疑义。

\begin{maonote}
\mnitem{1}指中共青海省委一九五八年六月十八日关于镇压叛乱问题给中央、军委并兰州军区、各州、县委、柴达木工委、西宁市委的报告。报告说,青海地区的反革命武装叛乱已经蔓延成为全局性的问题,全省牧业区的六个自治州都或多或少地先后发生了叛乱。从平息叛乱、捕捉反革命分子等方面获得的材料,证明青海地区的反革命武装叛乱与西藏拉萨方面的反动集团在帝国主义唆使下的阴谋分裂祖国的活动密切相联,是帝国主义和拉萨反动集团策动的。其实质是帝国主义与拉萨反动派阴谋分裂活动和社会主义改造与反改造的反映,是一场尖锐剧烈的你死我活的阶级斗争。要取得这场斗争的胜利,必须争取群众,坚决地实行社会主义改造,真正使劳动人民从政治上翻身,彻底铲除叛乱的根源。同时对于反革命的武装叛乱,必须以革命的武装予以严厉打击。一九八一年三月二十三日,中共中央批复了青海省委关于解决一九五八年平叛斗争扩大化遗留问题的请示报告。中央批语指出:“应当肯定,当时平息局部地区的武装叛乱是正确的,但是,由于‘左’的指导思想影响,犯了扩大化的错误,使一批干部、群众在政治上、经济上遭到了很大的、甚至是不可弥补的损失。责任主要在领导。处理这个问题,要着重从政治上解决,要做好深入细致的思想政治工作,消除群众之间、干部之间和民族之间的隔阂,引导各族人民顾全大局,团结起来向前看;在经济上,也要给予适当的抚恤、救济和补助,但目前国家经济还有困难,不可能完全解决,最根本的办法,是发展生产,把经济搞上去。”
\mnitem{2}克诚,即黄克诚,时任中共中央军委秘书长、国防部副部长。
\mnitem{3}刘,指刘少奇。邓,指邓小平,时任中共中央总书记、国务院副总理。彭,指彭德怀,时任国务院副总理兼国防部部长。黄,指黄克诚。
\end{maonote}
