
\title{大兴调查研究之风}
\date{一九六一年一月十三日}
\thanks{这是毛泽东同志在中共中央工作会议\mnote{1}上讲话的一部分。}
\maketitle


这一次中央工作会议,开得比过去几次都要好一些,大家的头脑比较清醒一些。比如关于冷热结合这个问题,过去总是冷得不够,热得多了一点,这一次结合得比过去有进步,对问题有分析,情况比较摸底。当然,现在有许多情况,就中央和省一级来说,还是不摸底。我们是向着摸清底的方向去做,这就进了一步。省委的书记、常委、委员,包括地委第一书记,他们就摸底吗?如果摸底就不成问题了。但是应该说,也比过去进了一步,在动,在用试点的方法去了解情况。

我希望同志们回去之后,要搞调查研究,把小事撇开,用一部分时间,带几个助手,去调查研究一两个生产队、一两个公社。在城市要彻底调查一两个工厂、一两个城市人民公社。一个省委第一书记,又要调查农村又要调查城市,这就要好好部署一下。去做调查,就是要使自己心里有底,没有底是不能行动的。了解情况,要用眼睛看,要用口问,要用手记。谈话的时候还要会谈,不然就会受骗。要看群众是不是面有菜色,群众的粮食究竟是很缺,还是够,还是很够,这是看得出来的。

这些年来,我们的同志调查研究工作不做了。要是不做调查研究工作,只凭想像和估计办事,我们的工作就没有基础。所以,请同志们回去后大兴调查研究之风,一切从实际出发,没有把握就不要下决心。调查研究工作,并不那么困难,时间并不要那么多,调查的单位也不要那么多。比如,在农村搞一两个生产队、一两个公社,在城市搞一两个工厂、一两个商店、一两个学校,加在一起也只有十个左右。这些调查并不都要自己亲身去搞。自己亲身搞的,农村有一两个、城市有一两个就够了。要组织调查研究的班子,指导他们去搞。比如宝坻县江石窝的调查,就不是我们中央去搞的,是中央农村工作部搞的。我看,这是他们的一大功劳。沔阳县通海口的材料,是湖北省的同志搞的;信阳的调查,是信阳地委搞的;灵宝县的调查,是河南省委的同志搞的。调查研究极为重要,要教会所有的省委书记加上省委常委、省一级和省的各个部门的负责同志、地委书记、县委书记、公社党委书记做调查研究。他们不做调查,情况就不清楚。公社内部平调的情况,公社的党委书记不一定都知道。一个公社平均有三十个生产队左右,他怎么会知道那么多呢?不可能嘛!但是,有一个办法,三十个生产队他调查三个就行了,一个最坏的,一个中等的,一个最好的。

我们讲情况明,决心大,方法对,要有这三条。

第一条情况明。这是一切工作的基础,情况不明,一切无从着手。因此要摸清情况,要做调查研究。

第二条决心大。这次会议我们开了二十几天,情况逐步明了了,决心逐步增大了,但是决心还是参差不齐。有些同志讲到要破产还债。这个话讲得不那么好听就是了,实际上就是要破产还债。县、社两级,如果为了还债,破产了,那就再白手起家。无产阶级不能剥夺劳动者,不能剥夺农民和城市小生产者,只能剥夺剥夺者,这是一条基本原则。资产阶级、地主阶级是剥夺劳动人民的,他们使那些小生产者破产,才有了无产阶级。他们剥夺的方法比我们一些人“高明”一点。他们是逐步逐步地使小生产者欠账、破产,而我们一些人是一下子就把人家的东西拿过来。用这种方法来建立社有经济、国有经济是不行的。比如收购农副产品压级压价,非常不等价,搞得太凶,脱离群众。执行原则,这个话好讲,我们许多同志也是懂得的,但实行起来就比较困难了。

现在,是不是所有的省委第一书记都有那么大的决心破产还债呢?我看还要看。将来会出现各省参差不齐的情况,这也是不平衡规律。要看究竟哪些省破产还债,彻底退赔,能将农民群众团结在自己周围。一个省内几十个县或者一百多个县里头,也会出现参差不齐的情况的。

第三条方法对。抗日战争时期,解放战争时期,我们做调查研究比较认真一些,注意从实际出发,实事求是。通过调查研究,情况明了来下决心,决心就大,方法也就对。方法就是措施、办法,实现方针、政策要有一套方法。这三条里头没有提方针、政策,因为我们已经有了方针、政策。有了好的方针、政策,你情况不明,决心不大,方法不对,还是等于没有。郑州会议\mnote{2}的方针政策是对的,只有一条不对,就是不要算账。郑州会议是三月初开完的。有浙江的经验\mnote{3},有麻城的经验\mnote{4},到了四月上海会议就搞了十八条\mnote{5},决定旧账坚决要算。我写的那个批语\mnote{6}还早嘛。那两个月我动笔批的文字有两万字以上。打笔墨官司没有多少用,两万字也好,再加一倍也好,不起作用。底下县、社、队封锁,你有什么办法。我们的省、地两级没有像现在这么一致的认识。有个同志讲,郑州会议是压服的,不是说服的。那也是了,那个时候认识上有个距离。

关于农业,我只讲这么一点。

工业呢?我们已经摸了一些底,还要继续摸底。大体上今年的盘子,要缩短基本建设战线,缩短工业战线,缩短重工业战线。除了煤炭、木材、矿山、铁道等,别的都要缩短。特别是要缩短基本建设战线,今年不搞新的基建,只搞正在建的或急需建的项目;正在建的和急需建的项目也不是今年统统都搞,其中有一部分今年也不搞,癞痢头就让他癞痢头。

至于长远计划,现在还作不出来。我们还要再搞十年才能作。

现在看来,搞社会主义建设不要那么十分急。十分急了办不成事,越急就越办不成,不如缓一点,波浪式地向前发展。这同人走路一样,走一阵要休息一下。军队行军有大休息、小休息,劳逸结合,有劳有逸。两个战役之间也要休息整顿。我们这次会开了二十多天,当然相当累了,但也不是每天上午、下午、晚上都开会。这次会不算“神仙会”,是相当紧张的会,但也不那么十分紧张,文件也比较少。过去我们开一次会议,决议很多,以为这些决议会灵,其实并不那么灵。会议的决议,多不一定灵,少也不一定灵,关键还是在于情况明不明,决心大不大,方法对不对。

今天看了一条消息,说西德去年搞了三千四百万吨钢,英国去年钢产量是两千四百万吨,法国前年搞了一千六百万吨,去年是一千七百万吨。他们都是搞了很多年才达到的。我看我们搞几年慢腾腾的,然后再说。今年、明年、后年搞扎实一点。不要图虚名而招实祸。我们要做巩固工作,提高产品质量,增加品种、规格,提高管理水平,提高劳动生产率。

总之,我们对国内情况还是不太明,决心也不大,方法也不那么对。我们要分批摸各省、市、自治区的底,二十七个地方分开来摸。每一个省,每一个市,每一个自治区又按地、县、公社分头去摸。今年搞一个实事求是年好不好?河北省有个河间县,汉朝封了一个王叫河间献王。班固在《汉书·河间献王刘德》中说他“实事求是”,这句话一直流传到现在。提出今年搞个实事求是年,当然不是讲我们过去根本一点也不实事求是。我们党是有实事求是传统的,就是把马列主义的普遍真理同中国的实际相结合。但是建国以来,特别是最近几年,我们对实际情况不大摸底了,大概是官做大了。我这个人就是官做大了,我从前在江西那样的调查研究,现在就做得很少了。今年要做一点,这个会开完,我想去一个地方,做点调查研究工作。不然,对实际情况就不摸底。不摸清一个农村公社,不摸清一个城市公社,不摸清一个工厂,不摸清一个学校,不摸清一个商店,不摸清一个连队,就不行。其实,摸清这么几个单位的情况就差不多了。

现在我们看出了一个方向,就是同志们要把实事求是的精神恢复起来了。

\begin{maonote}
\mnitem{1}这次会议于一九六〇年十二月二十四日至一九六一年一月十三日在北京召开。
\mnitem{2}郑州会议,指一九五九年二月二十七日至三月五日在郑州召开的中共中央政治局扩大会议。
\mnitem{3}指中共浙江省委一九五九年三月二十二日关于整顿人民公社的十项规定(草案)。其中第十项中关于“账目问题”规定:“在处理公社与生产队之间的经济关系中,所有旧账,都应该按照等价交换原则结算清楚。”
\mnitem{4}指王延春等一九五九年三月三十日关于麻城县万人大会情况给王任重并湖北省委的第二次报告。报告说,为了发动群众,算清生产队的账,县委决定采取开动员会、小组会、个别谈话、搞试点和开广播大会等方式,交代政策,动员社员参加算账。在算账过程中,发现生产队许多干部贪污挪用现象严重。县委要求有问题的干部在会上交代,承认错误。回去以后,通过开社员代表大会或社员大会彻底算清。另注:王任重一九五九年三月三十一日报送毛泽东四份材料:王延春等三月二十七日关于麻城县万人大会情况和关于棉花技术措施问题给王任重并湖北省委的两个报告,王延春同日关于穷队赶富队问题给王任重的报告,湖北省委书记许道琦三月二十三日关于与基层干部座谈“吃饭不要钱”问题给王任重、王延春并省委的报告。其中关于麻城县万人大会情况的报告说,麻城县的万人大会上,县、公社、管理区三级党委层层检讨,承认错误,带动基层干部也纷纷检讨,密切了上下级关系。检讨之后强调算账,强调兑现,认真解决一平二调三收款的问题,公社调生产队的钱和物资,立即退回;缺口粮的,立即供应;该支援的穷队,立即予以贷款。
\mnitem{5}指一九五九年三月二十五日至四月一日在上海召开的中共中央政治局扩大会议形成的会议纪要《关于人民公社的十八个问题》。这个纪要于同年四月二日至五日在上海举行的中共八届七中全会上原则通过。纪要对人民公社管理体制问题作了若干原则规定,进一步明确人民公社的所有制基本上是生产队所有制;并且将《郑州会议记录》中关于人民公社化过程中平调财物的旧账一般不算的规定,改为旧账一般要算,凡是县社调用生产队的劳力、资财,或者社队调用社员的私人财物,都要进行清理,如数归还,或者折价补偿。
\mnitem{6}指毛泽东一九五九年三月三十日在陶鲁笳关于山西各县五级干部大会情况的报告上的批注。批注内容:旧账一般不算这句话,是写到了郑州讲话(见本文\mnote{2})里面去了的,不对,应改为旧账一般要算。算账才能实行那个客观存在的价值法则。这个法则是一个伟大的学校,只有利用它,才有可能教会我们的几千万干部和几万万人民,才有可能建设我们的社会主义和共产主义。否则一切都不可能。对群众不能解怨气。对干部,他们将被我们毁坏掉。有百害而无一利。一个公社竟可以将原高级社的现金收入四百多万元退还原主,为什么别的社不可以退还呢?不要“善财难舍”。须知这是劫财,不是善财。无偿占有别人劳动是不许可的。
\end{maonote}
