
\title{关于平津战役\mnote{1}的作战方针}
\date{一九四八年十二月十一日}
\thanks{这是毛泽东为中共中央军事委员会起草的给林彪、罗荣桓等的电报。平津战役是中国人民解放战争中三个有决定意义的最大战役的最后一个。这个战役歼灭和改编了五十二万多国民党军,解放了北平、天津、张家口等重要城市,基本上结束了解放华北的战争。毛泽东在这里提出的战役方针,得到了完全的实现。}
\maketitle


一、张家口、新保安、怀来和整个北平、天津、塘沽、唐山诸敌,除某几个部队例如三十五军、六十二军、九十四军中的若干个别的师,在依靠工事保守时尚有较强的战斗力外,攻击精神都是很差的,都已成惊弓之鸟,尤其你们入关后是如此。切不可过分估计敌人的战斗力。我们有些同志过去都吃了过分估计敌人战斗力的亏,经过批评后他们也已懂得了。现在张家口、新保安两敌确已被围,大体上很难突围逃走。十六军约有一半迅速被歼。怀来敌一〇四军慌忙南逃,估计今日或明日可能被歼。该敌被歼后,你们准备以四纵由西南\mnote{2}向东北切断南口和北平间联系。估计此着不易实现,不是九十四军和十六军残部迅速撤回北平,就是九十四军、十六军和九十二军一起集中南口、昌平、沙河镇区域集团防守。但四纵此举直接威胁北平西北郊和北郊,可以钳制这些敌人不敢动。若这些敌人再敢西进接援三十五军,则可以直接切断其后路或直接攻北平,因此,这些敌人大约不敢再西进。我华北杨罗耿\mnote{3}兵团以九个师包围三十五军三个师,是绝对优势。他们提出早日歼灭该敌,我们拟要他们暂时不要打,以便吸引平津之敌不好下从海上逃走的决心。他们此次以两个纵队围住三十五军,以一个纵队阻住一〇四军,两敌都被击退。

二、我们现在同意你们以五纵立即去南口附近,从东北面威胁北平、南口、怀柔诸敌。将来该纵即位于该地,以便将来(大约在十天或十五天之后,即在华北杨罗耿兵团歼灭三十五军之后)腾出四纵使用于东面。如此,请令五纵本日仍继续西进。

三、三纵决不要去南口,该纵可按我们九日电开至北平以东、通县以南地区,从东面威胁北平,同四纵、十一纵、五纵形成对北平的包围。

四、但我们的真正目的不是首先包围北平,而是首先包围天津、塘沽、芦台、唐山诸点。

五、据我们估计,大约十二月十五日左右你们的十纵、九纵、六纵、八纵、炮纵、七纵就可集中于玉田为中心的地区。我们提议,十二月二十日至十二月二十五日数日内即取神速动作,以三纵(由北平东郊东调)、六纵、七纵、八纵、九纵、十纵等六个纵队包围天津、塘沽、芦台、唐山诸点之敌,如果诸点之敌那时大体仍如现时状态的话。其办法是以两个纵队位于以武清为中心的地区,即廊房、河西务、杨村诸点,以五个纵队插入天津、塘沽、芦台、唐山、古冶诸点之间,隔断诸敌之联系。各纵均须构筑两面阻击阵地,务使敌人不能跑掉,然后休整部队,恢复疲劳,然后攻歼几部分较小之敌。此时,四纵应由平西北移至平东。我华北杨罗耿兵团应于四纵移动之前歼灭新保安之敌。东面则应依情况,力争先歼塘沽之敌,控制海口。只要塘沽(最重要)、新保安两点攻克,就全局皆活了。以上部署,实际上是将张家口、新保安、南口、北平、怀柔、顺义、通县、宛平(涿县、良乡已被我占领)、丰台、天津、塘沽、芦台、唐山、开平诸点之敌一概包围了。

六、此项办法,大体上即是你们在义县、锦州、锦西、兴城、绥中、榆关、滦县线上作战时期用过的办法\mnote{4}。

七、从本日起的两星期内(十二月十一日至十二月二十五日)基本原则是围而不打(例如对张家口、新保安),有些则是隔而不围(即只作战略包围,隔断诸敌联系,而不作战役包围,例如对平、津、通州),以待部署完成之后各个歼敌。尤其不可将张家口、新保安、南口诸敌都打掉,这将迫使南口以东诸敌迅速决策狂跑,此点务求你们体会。

八、为着不使蒋介石迅速决策海运平津诸敌南下,我们准备令刘伯承、邓小平、陈毅、粟裕于歼灭黄维兵团之后,留下杜聿明指挥之邱清泉、李弥、孙元良诸兵团(已歼约一半左右)之余部,两星期内不作最后歼灭之部署。

九、为着不使敌人向青岛逃跑,我们准备令山东方面集中若干兵力控制济南附近一段黄河,并在胶济线上预作准备。

十、敌向徐州、郑州、西安、绥远\mnote{5}诸路逃跑,是没有可能或很少可能的。

十一、唯一的或主要的是怕敌人从海上逃跑。因此,在目前两星期内一般应采围而不打或隔而不围的办法。

十二、此种计划出敌意外,在你们最后完成部署以前,敌人是很难觉察出来的。敌人现时可能估计你们要打北平。

十三、敌人对于我军的积极性总是估计不足的,对于自己力量总是估计过高,虽然他们同时又是惊弓之鸟。平津之敌决不料你们在十二月二十五日以前能够完成上列部署。

十四、为着在十二月二十五日以前完成上列部署,你们应该鼓励部队在此两星期内不惜疲劳,不怕减员,不怕受冻受饥,在完成上列部署以后,再行休整,然后从容攻击。

十五、攻击次序大约是:第一塘芦区,第二新保安,第三唐山区,第四天津、张家口两区,最后北平区。

十六、你们对上述计划意见如何?这个计划有何缺点?执行有何困难?统望考虑电告。


\begin{maonote}
\mnitem{1}平津战役,是一九四八年十一月二十九日至一九四九年一月三十一日由中国人民解放军东北野战军和华北军区第二、第三兵团及地方部队,在西起张家口,东至塘沽、唐山,包括北平、天津在内的地区同国民党军进行的一次具有决定意义的战役。中共中央决定由林彪、罗荣桓、聂荣臻组成总前委,领导平津前线人民解放军的一切行动。东北野战军在胜利地完成了解放东北全境的任务后,根据中央军委和毛泽东的指示,迅即挥师入关,和华北的人民解放军连同东北、华北参战的地方部队,总兵力约一百万人,合力围歼国民党华北“剿总”总司令傅作义指挥下的五十多万国民党军队。一九四八年十一月二十五日,华北军区第三兵团根据中央军委指示,自集宁地区东进,二十九日向张家口外围守敌发起进攻,形成对张家口包围之势,切断敌人西逃的道路。十二月二日,华北军区第二兵团向涿鹿地区急进,割断怀来、宣化间敌军的联系;东北野战军先遣兵团向南口、怀来急进,切断北平、怀来敌军的联系。毛泽东十二月十一日为中央军委起草的这个电报下达后,十二月中旬,人民解放军将敌分割包围于北平、天津、张家口、新保安、塘沽五个据点,截断了敌军南逃西撤的通路。十二月二十二日,围歼了新保安傅作义集团主力三十五军军部和二个师。二十四日,解放了张家口,全歼守敌五万四千余人。一九四九年一月十四日,包围天津的人民解放军发起总攻,经二十九小时激战,全歼守敌十三万余人,俘虏天津警备司令陈长捷,解放天津。至此,北平二十余万守敌,在人民解放军严密包围下完全陷于绝境。由于中国共产党的努力争取,经过谈判,北平守敌在傅作义率领下接受和平改编。一月三十一日人民解放军进入北平,北平宣告和平解放。在整个战役中,除塘沽守敌五万余人由海上逃跑外,人民解放军共歼灭和改编了国民党军队五十二万余人。绥远国民党军也于一九四九年九月通电起义,接受改编。
\mnitem{2}这里说的“西南”,是指南口西南地区。
\mnitem{3}杨罗耿,指杨得志、罗瑞卿、耿飚。
\mnitem{4}东北野战军一九四八年九月在北宁线作战时,为不使义县、锦州、锦西、兴城、绥中、榆关(山海关)、滦县、昌黎之敌收缩集中,采用了先以一部分兵力分别包围和切断上述各点之敌,然后逐一歼灭的办法。
\mnitem{5}绥远,原来是一个省,一九五四年撤销,原辖地区划归内蒙古自治区。
\end{maonote}
