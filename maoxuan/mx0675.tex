
\title{上面要靠党的领导,下面要靠广大群众}
\date{一九六五年十二月二十一日}
\thanks{这是毛泽东同志在杭州会议上与陈伯达、艾思奇\mnote{1}等同志的谈话。}
\maketitle


今后的几十年对祖国的前途和人类的命运是多宝贵而重要的时期!现在二十岁的青年,再过二、三十年是四、五十岁的人,我们这一代青年人,将亲手把我们一穷二白的祖国建设成为伟大的社会主义强国,将亲手参加埋葬帝国主义的战斗,任重而道远。有志气有抱负的中国青年,一定要为完成我们伟大历史使命而奋斗终身,为完成我们伟大的历史使命,我们这一代要下决心一辈子艰苦奋斗!

政治工作要走群众路线,单靠首长不行,你能管这么多吗?许多事你们看不到的,你只能看到一部分。所以要发动人人负责,人人开口,人人鼓动、人人批评。每个人都长着眼睛和嘴,就应该让他们去看,让他们去说。群众的事情由群众来办理就是民主,这里有两条路线,一条是单靠个人来办,一条是发动群众来办。我们的政治是群众的政治、民主的政治要靠大家来治,而不是靠少数人来治,一定要发动人人开口。每个人既然长了嘴巴,就要负担两个责任,一个是吃饭,一个是说话。在坏事情坏作风面前,就要说话,就要负起斗争的责任来。

没有党的领导,单靠首长个人来领导,事情一定办不好,一定要靠党和同志们来办事,而不是靠个人来办,群众要发动,要形成群众动手动口的风气。上面要靠党的领导,下面要靠广大群众,这样才能把事情办好。

\begin{maonote}
\mnitem{1}艾思奇(一九一〇年——一九六六年),我国著名的马克思主义哲学家、教育家和革命家,原名李生萱,哲学家。云南腾冲人,蒙古族后裔。“艾思奇”的名字是从英文SH(其英文转写Sheng Hsuen)得到灵感,并成为自己的笔名,时任中国科学院哲学社会科学部委员,著作有《大众哲学》、《哲学与生活》、《艾思奇文集》,主编有《辩证唯物主义与历史唯物主义》等,毛泽东对他所做的贡献给予了极高的评价,称他是“党的理论战线上的忠诚战士”。
\end{maonote}
