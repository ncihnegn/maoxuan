
\title{四分五裂的反动派为什么还要空喊“全面和平”?}
\date{一九四九年二月十五日}
\maketitle


国民党反动统治崩溃的速度,比人们预料的要快。现在距离解放军攻克济南只有四个多月,距离攻克沈阳只有三个多月,但是国民党在军事上、政治上、经济上、文化宣传上的一切残余力量,却已经陷于不可挽救的四分五裂、土崩瓦解的状态。国民党统治的总崩溃开始于北线的辽沈战役\mnote{1}、平津战役\mnote{2}和南线的淮海战役\mnote{3}期间,这三个战役使国民党在去年十月初至今年一月底的不足四个月中丧失约一百五十四万多人,包括国民党正规军一百四十四个整师。国民党统治总崩溃是中国人民解放战争和中国人民革命运动伟大胜利的必然结果,但是国民党及其美国主人的“和平”叫嚣,对于促进国民党崩溃一事,也起了相当的作用。国民党反动派从今年一月一日开始搬起的一块名叫“和平攻势”的石头,原想用来打击中国人民的,现在是打在他们自己的脚上了。或者说得正确些,是把国民党自己从头到脚都打烂了。除了傅作义将军协助人民解放军已经和平地解决了北平问题以外,各地希望和平解决的还大有人在。美国人站在一旁发干急,深恨其儿子们不争气。其实,和平攻势这个法宝出产于美国工厂,还在大半年以前就由美国人送给了国民党。司徒雷登\mnote{4}本人曾经泄露了这个秘密。他在蒋介石发出所谓元旦文告以后,曾告中央社记者说,这是“我过去所一直亲自努力以求的东西”。据美国通讯社称,该记者因发表了这段“不得发表”的话而丢了饭碗。蒋介石集团长期地不敢接受美国人的这个命令,其理由,在国民党中央宣传部去年十二月二十七日的一项指示中说得很明白:“我如不能战,即亦不能和。我如能战,则言和又徒使士气人心解体。故无论我能战与否,言和皆有百害而无一利。”国民党当时发出这个指示,是因为国民党的其它派别已经在主张言和了。去年十二月二十五日,白崇禧及其指导下的湖北省参议会向蒋介石提出了“和平解决”的问题\mnote{5},迫使蒋介石不得不在今年一月一日发布在五个条件\mnote{6}下进行和谈的声明。蒋介石希望从白崇禧手里夺回和平攻势的发明权,并在新的商标下继续其旧的统治。蒋介石于一月八日派张群到汉口去要求白崇禧的支持,同日向美英法苏四国政府要求干涉中国的内战\mnote{7}。但是这些步骤全都失败了。中国共产党毛泽东主席在一月十四日的声明,致命地击破了蒋介石的假和平阴谋,使蒋介石在一个星期以后不得不“引退”到幕后去。虽然蒋介石、李宗仁和美国人对于这一手曾经作过各种布置,希望合演一出比较可看的双簧,但是结果却和他们的预期相反,不但台下的观众愈走愈稀,连台上的演员也陆续失踪。蒋介石在奉化仍然以“在野地位”继续指挥他的残余力量,但是他已丧失了合法地位,相信他的人已愈来愈少。孙科的“行政院”自动宣布“迁政府于广州”,它一面脱离了它的“总统”“代总统”,另一面也脱离了它的“立法院”“监察院”。孙科的“行政院”号召战争\mnote{8},但是进行战争的“国防部”却既不在广州,也不在南京,人们只知道它的发言人在上海。这样,李宗仁在石头城上所能看见的东西,就只剩下了“天低吴楚,眼空无物”\mnote{9}。李宗仁自上月二十一日登台到现在下过的命令,没有一项是实行了的。虽然国民党已经没有一个“全面”的“政府”,虽然许多地方都在进行着局部和平的活动,但是国民党死硬派却在反对局部和平而要求所谓“全面和平”,其目的就是取消和平,妄想再战;他们深怕局部和平的活动蔓延起来,至于不可收拾。以一个四分五裂土崩瓦解的国民党而要求所谓“全面和平”的滑稽剧,在本月九日上海伪国防部政工局长战争罪犯邓文仪的一篇声明中,达到了高峰。邓文仪和孙科一样,推翻了上月二十二日李宗仁关于以中共的八项和平条件\mnote{10}为谈判基础的声明,而要求所谓“平等的和平,全面的和平”,否则“不惜牺牲一切,与共党周旋到底”。但是邓文仪没有说出在今天他的对方究竟应和什么人去谈判“平等的”“全面的”和平。似乎找邓文仪是不能解决问题的,似乎不找邓文仪或者其它张三李四也不能解决问题,这就未免叫人为难了。据中央社上海九日电称:“新闻记者问邓文仪:李代总统是否已同意邓局长所发表之四项意见\mnote{11}?答:本人系在国防部立场发言,本日所发表之四项意见,事前并未呈经李代总统过目。”邓文仪在这里不但创造了一个伪国防部的局部立场以区别于伪国民党政府的全面立场,而且事实上还创造了一个伪国防部政工局的小局部立场以区别于伪国防部的大局部立场。因为邓文仪公开反对并污蔑北平的和平解决,而伪国防部则在一月二十七日称赞北平的和平解决,是“为了缩短战争,获致和平,借以保全北平故都基础与文物古迹”,并称其它地方例如大同绥远等处\mnote{12}亦将依同样方法“实施休战”。由此可见,叫喊“全面和平”最起劲的反动派,原来就是最缺乏全面立场的反动派。一个国防部政工局可以和国防部互相矛盾,又可以和它的代总统互相矛盾。这些反动派是今天中国实现和平的最大障碍。他们梦想在“全面和平”的口号下鼓吹全面战争,即所谓“战要全面战,和要全面和”。但是,事实上他们既没有什么力量实行全面和平,也没有什么力量实行全面战争。全面的力量是在中国人民、中国人民解放军、中国共产党和其它民主党派这一方面,不在四分五裂土崩瓦解的国民党方面。一方面,握有全面的力量,另一方面,陷于四分五裂土崩瓦解的绝境,这种局面,是中国人民长期奋斗和国民党长期作孽的结果。任何郑重的人,都不能忽视今天中国政治形势中这个基本的事实。


\begin{maonote}
\mnitem{1}见本卷\mxnote{关于辽沈战役的作战方针}{1}。
\mnitem{2}见本卷\mxnote{关于平津战役的作战方针}{1}。
\mnitem{3}见本卷\mxnote{关于淮海战役的作战方针}{1}。
\mnitem{4}见本卷\mxnote{别了,司徒雷登}{1}。
\mnitem{5}一九四八年十二月二十四日,国民党华中“剿总”总司令白崇禧,利用当时对蒋介石极为不利的形势,发出致蒋介石电,提出“恢复和平谈判”以解决时局的主张,目的是逼蒋下台,抬高桂系地位。二十五日,在白崇禧指导下,湖北省参议会通过致蒋介石电,指出“如战祸继续蔓延,不立谋改弦更张之道,则国将不国,民将不民”,要蒋介石“循政治解决之常轨,觅取途径,恢复和谈”。
\mnitem{6}见本卷\mxnote{中共发言人关于命令国民党反动政府重新逮捕前日本侵华军总司令冈村宁次和逮捕国民党内战罪犯的谈话}{6}。
\mnitem{7}国民党政府于一九四九年一月八日向美、英、法、苏四国政府要求干涉中国内战,遭到四国政府拒绝。美国政府在其一月十二日答复国民党政府的“备忘录”中,说明美国之所以拒绝国民党政府的要求,是因“殊难相信”“能达到任何有益的效果”。这就是说,美国当时已感到再也无力挽救它所扶植的蒋介石反动政权的灭亡。
\mnitem{8}国民党政府行政院长孙科于一九四九年二月六七两日在广州两次发表谈话,反对李宗仁关于以中共所提八项和平条件作为谈判基础的声明,说是“现政府已迁穗办公,吾人应对过去重新检讨”,又说“共党所提出之惩治战犯一节,即系绝对不能接受者”。
\mnitem{9}这是借用公元十四世纪元朝人萨都剌所作《登石头城》调寄《念奴娇》词中的话。这首词的上阕是:“石头城上,望天低吴楚,眼空无物。指点六朝形胜地,惟有青山如壁。蔽日旌旗,连云樯橹,白骨纷如雪。一江南北,消磨多少豪杰。”南京古称石头城。吴楚泛指长江的中下游。
\mnitem{10}见本卷\mxart{中共中央毛泽东主席关于时局的声明}。
\mnitem{11}国民党政府国防部政工局长邓文仪,于一九四九年二月九日在上海发表《和平与战争的发展》的书面谈话,提出所谓“四项意见”:一、“政府要和”,二、“中共要战”,三、“北平局部和平成了骗局”,四、“备战以言和,能战方能和”,“不惜牺牲一切,与共党周旋到底”。
\mnitem{12}在天津、北平解放后,华北国民党军队只剩下几个残余的孤立据点,其中包括太原、大同、新乡、安阳、归绥(今呼和浩特市)等地。太原国民党军于一九四九年四月二十四日被人民解放军完全歼灭。大同国民党军于四月二十九日接受和平改编。新乡国民党军于五月五日向人民解放军投降。五月六日,安阳国民党军被人民解放军歼灭,归绥于九月十九日和平解放。
\end{maonote}
