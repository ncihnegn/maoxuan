
\title{对在押国民党战犯、党政军特人员一律释放}
\date{一九七五年二月二十七日}
\thanks{这是毛泽东同志在对在押国民党战犯、党政军特人员处理意见的批语。}
\maketitle


\date{一九七五年二月二十七日}
\section*{(一)\mnote{1}}

锦州、大虎山、沈阳、长春,还有战犯,为什么没有?\mnote{2}放战犯的时候要开欢送会,请他们吃顿饭,多吃点鱼、肉,每人发一百元零用钱,每人都有公民权。

不要强迫改造。都放了算了,强迫人家改造也不好。

土改的时候我们杀恶霸地主,不杀,老百姓怕。这些人老百姓都不知道,你杀他干什么,所以一个不杀。

气魄太小了:十五元太少,十三人不放,也不开欢送会。\mnote{3}有些人有能力可以做工作。年老有病的要给治病,跟我们的干部一样治。人家放下武器二十五年了。

\date{一九七五年九月九日}
\section*{(二)\mnote{4}}

建议一律释放。本地不能就业的,转别地就业。如何,请酌定。

\begin{maonote}
\mnitem{1}本篇曾由中共中央办公厅一九七五年二月二十八日印发,分送在京中央政治局委员。一九七五年三月十七日,第四届全国人大常委会举行第二次会议,讨论周恩来根据中共中央和毛泽东意见提出的关于特赦释放全部在押战犯的建议,并听取国务院副总理兼公安部部长华国锋所作的说明。会议决定,对全部在押战犯实行特赦释放,并给予公民权。三月十八日,《全国人民代表大会常务委员会关于特赦释放全部在押战争罪犯的决定》在《人民日报》公布。十九日,最高人民法院在战犯管理所召开大会,宣布特赦释放的在押战犯二百九十三人名单,并发放特赦释放通知书。至此,在押的战犯全部处理完毕。最后一批战犯被特赦后,有十名原国民党高级将领申请去台湾与家人团聚。他们的申请很快就得到了批准,政府有关部门还帮助他们办理赴港手续,发给他们适合香港情况的服装和足够的费用,并指定香港中国旅行社负责照料他们的生活。但是,台湾当局却认为这是中共的“统战阴谋”,回台的十人是共党派去的“间谍”和“统战分子”,因而拒不接纳,并声明“绝不上当”。由于台湾当局的极端恐惧和百般阻挠,申请回台的这十个人终于没能与家人团聚,最后有四人去了美国,两人留在了香港,三人返回大陆,一人自杀身亡。
\mnitem{2}这句话指解放战争中锦州战役、大虎山战役、沈阳战役、长春战役中还有一些国民党高级将领战犯没有列入当时送审的一份特赦释放在押战犯的名单。
\mnitem{3}当时送审的一份有关释放在押战犯的报告中讲到,拟给释放的每位战犯发十五元零用钱,但不开欢送会。另有十三名罪大恶极的战犯拟继续关押,不予释放。
\mnitem{4}中共中央副主席、国务院副总理邓小平一九七五年九月七日报送的公安部党的核心小组的《对清理在押国民党省、将级党政军特人员的请示报告》说,我们对各地在押的国民党省、将级党政军特人员进行了调查摸底,并于最近召开了清理工作会议。现将情况和处理意见报告如下:

(一)全国现押的这类人员有三百四十一名,刑满就业的有六百一十二名。

(二)在现押的这类人员中,拟对其中以历史罪或主要以历史罪判刑的二百三十二名予以清理释放。因现行罪逮捕判刑的一百零九人,这次不予清理。

(三)对清理释放人员,要给予妥善安置。

1、发给释放裁定书,同时宣布摘掉帽子,给公民权。

2、有家的回家,无家可归的、自愿留下的由原劳改单位安置。愿回台湾的可准予回去,并提供方便。

3、释放时发给衣被和零用钱,安置回家的发给路费,回家后生活有困难的酌情予以补助。

4、生活和教育问题由当地民政部门和基层组织负责,其中按起义投诚人员对待的,由统战部门负责,并酌情安置工作。

(四)对刑满就业的人员,拟同时清理。除表现很坏的以外,均可摘掉帽子,给公民权。清理后的去路同对待在押犯的原则一样。

(五)对因历史罪判刑在押的三千三百多名和已刑满就业的一万名相当于县团级以上人员的清理,将经过调查研究,参照上述精神予以考虑。

毛泽东的批语,写在邓小平的送审报告上。一九七五年十二月十五日,司法机关决定对在押的原国民党县团以上党政军特人员一律宽大释放。十五日至十八日,各地司法机关先后召开了宽大释放大会。
\end{maonote}
