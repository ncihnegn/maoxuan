
\title{打退资产阶级右派的进攻}
\date{一九五七年七月九日}
\thanks{这是毛泽东同志在上海干部会议上的讲话。}
\maketitle


三月间,我在这个地方同党内的一些干部讲过一次话。从那个时候到现在,一百天了。这一百天,时局有很大的变化。我们同资产阶级右派打了一仗,人民的觉悟有所提高,而且是相当大的提高。当时我们就料到这些事情了。比如,我在这里说过,人家批评起来,就是说火一烧起来,岂不是疼吗?要硬着头皮顶住。人这个地方叫头,头有一张皮,叫头皮。硬着头皮顶住,就是你批评我,我就硬着头皮听,听一个时期,然后加以分析,加以答复,说得对的就接受,说得不对的加以批评。

我们总要相信,全世界也好,我们中国也好,多数人是好人。所谓多数人,不是百分之五十一的人,而是百分之九十以上的人。在我国六亿人口中,工人、农民是我们的基本群众。共产党里,青年团里,民主党派里,学生和知识分子里,多数人总是好人。他们的心总是善良的,是诚实的,不是狡猾的,不是别有用心的。应当承认这一点。这是每一次运动都证明了的。比如这一次,拿学生来说,北京大学有七千多人,右派只有百分之一、二、三。什么叫百分之一、二、三呢?就是坚决的骨干分子,经常闹的,闹得天翻地覆的,始终只有五十几个人,不到百分之一。另外百分之一、二的人,是为他们拍掌的,拥护他们的。

放火烧身可不容易。现在听说你们这个地方有些同志后悔了,感到没有放得厉害。我看上海放得差不多了,就是有点不够,有点不过痛,早知这么妙,何不大放特放呢?让那些毒草长嘛,让那些牛鬼蛇神出台嘛,你怕他干什么呢?三月份那个时候我就讲不要怕。我们党里头有些同志,就是怕天下大乱。我说,这些同志是忠心耿耿,为党为国,就是没有看见大局面,就是没有估计到大多数人,即百分之九十几的人是好人。不要怕群众,他们是跟我们一块的。他们可以骂我们,但是他们不用拳头打我们。右派只有极少数,象刚才讲的北京大学,只有百分之一、二、三。这是讲学生。讲到教授、副教授,那就不同一些,大概有百分之十左右的右派。左派也有百分之十左右。这两方面旗鼓相当。中间派占百分之八十左右。有什么可怕呢?我们有些同志,就有那么一些怕,又怕房子塌下来,又怕天塌下来。从古以来,只有“杞人忧天”\mnote{1},就是那个河南人怕天塌下来。除了他以外,从来就没有人怕天塌下来的。至于房子,我看这个房子不会塌下来,刚刚砌了不好久嘛,怎么那么容易塌下来呢?

总而言之,无论什么地方,百分之九十几的人是我们的朋友、同志,不要怕。怎么怕群众呢?怕群众是没有道理的。什么叫领导人物呢?小组长、班长、支部书记、学校里头的校长、党委书记,都是领导人物,还有柯庆施同志,我也算一个。我们这些人总是有那么一点政治资本,就是替人民多少做了一点事。现在把火放起来烧,百分之九十以上的人,是希望把我们的同志烧好。我们每一个同志,都有一点毛病,那有没有毛病的呢?“人非圣贤,孰能无过”,总要讲错一点话,办错一点事,就是什么官僚主义之类。这些东西往往是不自觉的。

要定期“放火”。以后怎么搞呀?你们觉得以后是一年烧一次,还是三年烧一次?我看至少是象闰年、闰月一样,三年一闯,五年再闯,一个五年计划里头至少搞两次。孙悟空在太上老君的八卦炉里头一锻炼就更好了。孙悟空不是很厉害的人物吗?人家说是“齐天大圣”呀,还要在八卦炉里头烧一烧。不是讲锻炼吗?锻者就是锤打,炼就是在高炉里头炼铁,平炉里头炼钢。钢炼出来要锻。现在锻要拿汽锤锻。那个锻可厉害哩!我们人也要锻炼。有些同志,你问他赞成不赞成锻炼,那他是很赞成的,“啊,我有缺点,很想去锻炼一下”。人人都说要锻炼一番。平常讲锻炼,那舒服得很,真正要锻炼了,真正要拿汽锤给他一锻,他就不干了,吓倒了。这一回就是一次锻炼。一个时期天昏地黑,日月无光。就是两股风来吹:一是大多数好人,他们贴大字报,讲共产党有缺点,要改;另外是极少数右派,他们是攻击我们的。两方面进攻的是一个方向。但是多数人的进攻是应当的,攻得对。这对我们是一种锻炼。右派的进攻,对我们也是一种锻炼。真正讲锻炼,这一回还是要感谢右派。对我们党,对广大群众,对工人阶级、农民、青年学生以及民主党派,右派给的教育最大。每一个城市都有一些右派,这些右派是要打倒我们的。对这些右派,现在我们正在围剿。

我们的革命是人民的革命,是无产阶级领导的六亿人民的革命,是人民的事业。民主革命是人民的事业,社会主义革命是人民的事业,社会主义建设是人民的事业。那末,社会主义革命和社会主义建设好不好?有没有成绩?成绩是主要的,还是错误是主要的?右派否定人民事业的成绩。这是第一条。第二条,走那一个方向呢?走这边就是社会主义,走那边就是资本主义。右派就是要倒转这个方向,走资本主义道路。第三条,要搞社会主义,谁人来领导?是无产阶级领导,还是资产阶级领导?是共产党领导,还是那些资产阶级右派来领导?右派说不要共产党领导。我看这一回是一次大辩论,就是在这三个问题上的大辩论。辩论一次好。这些问题没有辩论过。

民主革命是经过长期辩论的。从清朝末年起,到辛亥革命、反袁世凯、北伐战争、抗日战争,都是经过辩论的。要不要抗日?一派人是唯武器论,说中国的枪不够,不能抗;另外一派人说不怕,还是人为主,武器不如人,还是可以打。后头接着来的解放战争,也是经过辩论的。重庆的谈判,重庆的旧政协,南京的谈判,都是辩论。蒋介石对我们的意见,对人民的意见,一概都不听,他要打仗。打的结果,他打输了。所以,那一场民主革命是经过辩论的,经过长时期精神准备的。

社会主义革命来得急促。在六、七年之内,资本主义所有制和小生产者个体所有制的社会主义改造,就基本上完成了。但是人的改造,虽然也改造了一些,那就还差。社会主义改造有两方面:一方面是制度的改造,一方面是人的改造。制度不单是所有制,而且有上层建筑,主要是政权机关、意识形态。比如报纸,这是属于意识形态范围的。有人说,报纸没有阶级性,报纸不是阶级斗争的工具。这种话就讲得不对了。至少在帝国主义消灭以前,报纸,各种意识形态的东西,都是要反映阶级关系的。学校教育,文学艺术,都是意识形态,都是上层建筑,都是有阶级性的。自然科学分两个方面,就自然科学本身来说,是没有阶级性的,但是谁人去研究和利用自然科学,是有阶级性的。大学里,一个中文系,一个历史系,唯心论最多。办报纸的,唯心论最多。你们不要以为只是社会科学方面唯心论多,自然科学方面也有许多唯心论。搞自然科学的许多人,世界观是唯心论的。你要说水是什么东西构成的,那他们是唯物论,水是两种元素构成的,他们是照那个实际情形办事的。你要讲社会怎么改造,那他们是唯心论。我们说整风是要整好共产党,他们中间一些人说要消灭共产党。这一回暴露了这么一些情形。

右派进攻的时候,我们的政策是这样,就是只听不说。有那么几个星期,硬着头皮,把耳朵扯长一点,就听,话是一句不说。而且不通知团员,不通知党员,也不通知支部书记,不通知支部委员会,让他们混战一场,各人自己打主意。学校的党委、总支里头混进来一些敌人,清华大学党委的委员里就有敌人。你这里一开会,他就告诉敌人了,这叫做“起义分子”。不是有起义将军吗?这是“起义文人”。这一件事,敌人和我们两方面都高兴。在敌人方面,看见共产党员“起义”了,共产党要“崩溃”了,他们很高兴。这一回崩溃了多少呀?上海不晓得,北京学校的党员大概是崩溃了百分之五,团员崩溃得多一点,也许百分之十,或者还多一点。这种崩溃,我说是天公地道。百分之十也好,百分之二十也好,百分之三十也好,百分之四十也好,总而言之,崩溃了我就高兴得很。你满脑子资产阶级思想,满脑子唯心论,你钻进共产党、青年团里头,名为共产主义,实际上是反共产主义,或者是动摇分子。所以,在我们方面,看见“起义”的,我们也高兴。那一年清党、清团清得这么干净呀?他自己跑出去了,不要我们清理。但是,现在的情况变了,反过来了。我们把右派一包围,许多跟右派有联系但并非右派的人起来一揭露,他不“起义”了。现在右派不好混了,有一些右派起义了。我三月在这里讲话以后,一百天工夫,时局起了这样大的变化。

这次反右派斗争的性质,主要是政治斗争。阶级斗争有各种形式,这次主要是政治斗争,不是军事斗争,不是经济斗争。思想斗争的成份有没有呢?有,但是我看政治斗争占主要成份。思想斗争主要还在下一阶段,那要和风细雨。共产党整风,青年团整风,是思想斗争。要提高一步,真正学点马克思主义。要真正互相帮助。有什么缺点,主观主义有一点没有呀?官僚主义有一点没有呀?我们要真正用脑筋想一想,写一点笔记,搞那么几个月,把马克思主义水平、政治水平和思想水平提高一步。

反击右派也许还要几十天,还要个把月。右派言论尽这样在报纸上登,今年登一年,明年登一年,后年登一年,那也不好办事。右派就那么多,右派言论登的差不多了,也没有那么多东西登嘛。以后就阴登一点,阳登一点,有就登一点,没有就不登。我看七月还是反击右派紧张的一个月。右派最喜欢急风暴雨,最不喜欢和风细雨。我们不是提倡和风细雨吗?他们说,和风细雨,黄梅雨天天下,秧烂掉,就要闹饥荒,不如急风暴雨。你们上海不是有那么一个人写了一篇文章叫《乌“昼”啼》吗?那个“乌鸦”他提此一议。他们还说,你们共产党就不公道,你们从前整我们就是急风暴雨,现在你们整自己就和风细雨了。其实,我们从前搞思想改造,包括批评胡适、梁漱溟,我们党内下的指示都是要和风细雨的。世界上的事情总是曲折的。比如走路,总是这么弯弯曲曲走的。莫干山你们去过没有呀?上下都是一十八盘。社会的运动总是采取螺旋形前进的。现在,右派还要挖,不能松劲,还是急风暴雨。因为他们来了个急风暴雨,这好象是我们报复他们。这个时候,右派才晓得和风细雨的好处。他看见那里有一根草就想抓,因为他要沉下去了。好比黄浦江里将要淹死的人一样,那怕是一根稻草,他都想抓。我看,那个“乌鸦”现在是很欢迎和风细雨了。现在是暴雨天,过了七月,到了八月那个时候就可以和风细雨了,因为没有多少东西挖了嘛。

右派是很好的反面教员。我们中国历来如此,有正面的教员,有反面的教员。人需要正反两方面的教育。日本帝国主义是我们第一个大好的反面教员。从前还有清政府,有袁世凯,有北洋军阀,后头有蒋介石,都是我们很好的反面教员。没有他们,中国人民教育不过来,单是共产党来当正面教员还不够。现在也是一样。我们有许多话他们不听。所谓不听,是什么人不听呢?是许多中间人士不听,特别是右派不听。中间人士将信将疑。右派根本不听,许多话我们都跟他们讲了的,但是他们不听,另外搞一套。比如我们主张“团结——批评——团结”,他们就不听。我们说肃反成绩是主要的,他们又不听。我们讲要民主集中制,要无产阶级领导的人民民主专政,他们又不听。我们讲要联合社会主义各国和全世界爱好和平的人民,他们也不听。总而言之,这些东西都讲过的,他们都不听。还有一条他们特别不听的,就是说毒草要锄掉。牛鬼蛇神让它出来,然后展览,展览之后,大家说牛鬼蛇神不好,要打倒。毒草让它出来,然后锄掉,锄倒可以作肥料。这些话讲过没有呢?还不是讲过吗?毒草还是要出来。农民每年都跟那些草讲,就是每年都要锄它几次,那个草根本不听,它还要长。锄了一万年,草还要长,一万万年,还是要长。右派不怕锄,因为我讲话那个时候,不过是讲要锄草,并没有动手锄;而且他们认为自己并非毒草,是香花,我们这些人是毒草,他们并不是应当锄掉的,而要把我们锄掉。他们就没有想到,他们正是那些应当锄掉的东西。

现在就是辩论我上边讲过的那三个问题。社会主义革命来得急促,党在过渡时期的总路线没有经过充分辩论,党内没有充分辩论,社会上也没有充分辩论。象牛吃草一样,先是呼噜呼噜吞下去,有个袋子装起来,然后又回过头来慢慢嚼。我们在制度方面,首先是生产资料所有制,第二是上层建筑,政治制度也好,意识形态也好,进行了社会主义革命,但是没有展开充分辩论。这回经过报纸,经过座谈会,经过大会,经过大字报,就是展开辩论。

大字报是个好东西,我看要传下去。孔夫子的《论语》传下来了,“五经”、“十三经”传下来了,“二十四史”都传下来了。这个大字报不传下去呀?我看一定要传下去。比如将来工厂里头整风要不要大字报呀?我看用大字报好,越多越好。大字报是没有阶级性的,等于语言没有阶级性一样。白话没有阶级性,我们这些人演说讲白话,蒋介石也讲白话。现在都不讲文言了,不是讲“学而时习之,不亦说乎”,“有朋自远方来,不亦乐乎”。无产阶级讲白话,资产阶级也讲白话。无产阶级可以用大字报,资产阶级也可以用大字报。我们相信,多数人是站在无产阶级这一边的。因此,大字报这个工具有利于无产阶级,不利于资产阶级。一个时候,两三个星期,天昏地黑,日月无光,好象是利于资产阶级。我们讲硬着头皮顶住,也就是那两三个星期,睡不着觉,吃不下饭。你们不是讲锻炼吗?有几个星期睡不着觉,吃不下饭,这就是锻炼,并非要把你塞到高炉里头去烧。

有许多中间人士动摇一下,这也很好。动摇一下,他们得到了经验。中间派的特点就是动摇,不然为什么叫中间派?这一头是无产阶级,那一头是资产阶级,还有许多中间派,两头小中间大。但是,中间派归根结底是好人,他们是无产阶级的同盟军。资产阶级也想争取他们作同盟军,一个时候他们也有点象。因为中间派也批评我们,但他们是好心的批评。右派看见中间派批评我们,就来捣乱了。在你们上海,就是什么王造时,陆治,陈仁炳,彭文应,还有一个吴茵,这么一些右派人物出来捣乱。右派一捣乱,中间派就搞糊涂了。右派的老祖宗就是章伯钧、罗隆基、章乃器,发源地都是在北京。北京那个地方越乱越好,乱得越透越好。这是一条经验。

刚才讲大字报,这是个方式的问题,是取一种什么形式作战的问题。大字报是作战的武器之一,象步枪、短枪、机关枪这类轻武器。至于飞机、大炮,那大概是《文汇报》之类吧,还有《光明日报》,也还有一些别的报纸。有一个时期,共产党的报纸也登右派言论。我们下了命令,所有右派言论,要照原样注销来。我们运用这种方式,以及其它各种方式,使广大群众从正反两面受到了教育。比如《光明日报》、《文汇报》的工作人员,这次得了很深刻的教育。他们过去分不清什么叫无产阶级报纸,什么叫资产阶级报纸,什么叫社会主义报纸,什么叫资本主义报纸。一个时候,他们的右派领导人把报纸办成资产阶级报纸。这些右派领导人仇恨无产阶级,仇恨社会主义。他们不是把学校引到无产阶级方向,而是要引到资产阶级方向。

资产阶级和旧社会过来的知识分子,要不要改造?他们非常之怕改造,说改造就出那么一种感,叫做“自卑感”,越改就越卑。这是一种错误的说法。应当是越改造越自尊,应当是自尊感,因为是自己觉悟到需要改造。那些人的“阶级觉悟”很高,他们认为他们本身不要改造,相反要改造无产阶级。他们要按照资产阶级的面貌来改造世界,而无产阶级要按照无产阶级的面貌来改造世界。我看,多数人,百分之九十以上的人,经过踌躇、考虑、不大愿意、摇摆这么一些过程,总归是要走到愿意改造。越改造就越觉得需要改造。共产党还在改造,整风就是改造,将来还要整风。你说整了这次风就不整了?整了这次风就没有官僚主义了?只要过两三年,他都忘记了,那个官僚主义又来了。人就有那么一条,他容易忘记。所以,过一个时候就要整整风。共产党还要整风,资产阶级和旧社会过来的知识分子就不要整风?不要改造?那就更需要整风,更需要改造。

现在各民主党派不是在整风吗?整个社会要整一整风。把风整一整,有什么不好?又不是整那些鸡毛蒜皮,而是整大事,整路线问题。现在民主党派整风的重点是整路线问题,整资产阶级右派的反革命路线。我看整得对。现在共产党整风的重点不是整路线问题,是整作风问题。而民主党派现在作风问题在其次,主要是走那条路线的问题。是走章伯钧、罗隆基、章乃器、陈仁炳、彭文应、陆治、孙大雨那种反革命路线,还是走什么路线?首先要把这个问题搞清楚,要把我讲的这三个问题搞清楚:社会主义革命和社会主义建设的成绩,几亿人民作的事情究竟好不好?是走社会主义道路,还是走资本主义道路?要搞社会主义的话,要那一个党来领导?是要章罗同盟领导,还是要共产党来领导?来它一个大辩论,把路线问题搞清楚。

共产党也有个路线问题,对于那些“起义分子”,共产党、青年团里头的右派,是个路线问题。教条主义现在不是个路线问题,因为它没有形成。我们党的历史上有几次教条主义路线问题,因为它形成制度,形成政策,形成纲领。现在的教条主义没有形成制度、政策、纲领,它是有那么一些硬性的东西,现在这么一锤子,火这样烧一下,也软了一点。各个机关、学校、工厂的领导人,不是在“下楼”吗?他们不要那个国民党作风和老爷习气了,不做官当老爷了。合作社主任跟群众一起耕田,工厂厂长、党委书记到车间里头去,同工人一起劳动,官僚主义大为减少。这个风将来还要整。要出大字报,开座谈会,把应当改正的,应当批评的问题都分类解决。再就是要提高一步,学一点马克思主义。

我相信,我们中国人多数是好人,我们中华民族是个好民族。我们这个民族是很讲道理的,很热情的,很聪明的,很勇敢的。我希望造成这么一种局面:就是又集中统一,又生动活泼,就是又有集中,又有民主,又有纪律,又有自由。两方面都有,不只是一方面,不是只有纪律,只有集中,把人家的嘴巴都封住,不准人家讲话,本来不对的也不准批评。应当提倡讲话,应当是生动活泼的。凡是善意提出批评意见的,言者无罪,不管你怎么尖锐,怎么痛骂一顿,没有罪,不受整,不给你小鞋穿。小鞋于那个东西穿了不舒服。现在要给什么人小鞋子穿呢?现在我们给右派穿。给右派一点小鞋穿是必要的。

不要怕群众,要跟群众在一起。有些同志怕群众跟怕水一样。你们游水不游水呀?我就到处提倡游水。水是个好东西。你只要每天学一小时,不间断,今天也去,明天也去,有一百天,我保管你学会游水。第一不要请先生,第二不要拿橡皮圈,你搞那个橡皮圈就学不会。“但是我这条命要紧呀,我不会呀!”你先在那个浅水的地方游嘛。如果说学一百天,你在那个浅水的地方搞三十天,你就学会了。只要学会了,那末你到长江也好,到太平洋也好,一样的,就是一种水,就是那么一个东西。有人说在游泳池淹下去还可以马上把你提起来,死不了,在长江里头游水可不得了,水流得那么急,沉下去了到那里去找呀?拿这一条理由来吓人。我说这是外行人讲的话。我们的游泳英雄,游泳池里头的教员、教授,原先不敢下长江,现在都敢了。你们黄浦江现在不是也有人游吗?黄浦江、长江是一个钱不花的游泳池。打个比喻,人民就象水一样,各级领导者,就象游水的一样,你不要离开水,你要顺那个水,不要逆那个水。不要骂群众,群众是不能骂的呀!工人群众,农民群众,学生群众,民主党派的多数成员,知识分子的多数,你不能骂他们,不能跟群众对立,总要跟群众一道。群众也可能犯错误。他犯错误的时候,我们要好好讲道理,好好讲他不听,就等一下,有机会又讲。但是不要脱离他,等于我们游水一样不要脱离水。刘备得了孔明,说是“如鱼得水”,确有其事,不仅小说上那么写,历史上也那么写,也象鱼跟水的关系一样。群众就是孔明,领导者就是刘备。一个领导,一个被领导。

智慧都是从群众那里来的。我历来讲,知识分子是最无知识的。这是讲得透底。知识分子把尾巴一翘,比孙行者的尾巴还长。孙行者七十二变,最后把尾巴变成个旗杆,那么长。知识分子翘起尾巴来可不得了呀!“老子就是不算天下第一,也算天下第二”。“工人、农民算什么呀?你们就是‘阿斗’,又不认得几个字”。但是,大局问题,不是知识分子决定的,最后是劳动者决定的,而且是劳动者中最先进的部分,就是无产阶级决定的。

无产阶级领导资产阶级,还是资产阶级领导无产阶级?无产阶级领导知识分子,还是知识分子领导无产阶级?知识分子应当成为无产阶级的知识分子,没有别的出路。“皮之不存,毛将焉附”\mnote{2},过去知识分子这个“毛”是附在五张“皮”上,就是吃五张皮的饭。第一张皮,是帝国主义所有制。第二张皮,是封建主义所有制。第三张皮,是官僚资本主义所有制。民主革命不是要推翻三座大山吗?就是打倒帝国主义、封建主义、官僚资本主义。第四张皮,是民族资本主义所有制。第五张皮,是小生产所有制,就是农民和手工业者的个体所有制。过去的知识分子是附在前三张皮上,或者附在后两张皮上,附在这些皮上吃饭。现在这五张皮还有没有?“皮之不存”了。帝国主义跑了,他们的产业都拿过来了。封建主义所有制消灭了,土地都归农民,现在又合作化了。官僚资本主义企业收归国有了。民族资本主义工商业实行公私合营了,基本上(还没有完全)变成社会主义的了。农民和手工业者的个体所有制变为集体所有制了,尽管这个制度现在还不巩固,还要几年才能巩固下来。这五张皮都没有了,但是它还影响“毛”,影响这些资本家,影响这些知识分子。他们脑筋里头老是记得那几张皮,做梦也记得。从旧社会、旧轨道过来的人,总是留恋那种旧生活、旧习惯。所以,人的改造,时间就要更长些。

现在,知识分子附在什么皮上呢?是附在公有制的皮上,附在无产阶级身上。谁给他饭吃?就是工人、农民。知识分子是工人阶级、劳动者请的先生,你给他们的子弟教书,又不听主人的话,你要教你那一套,要教八股文,教孔夫子,或者教资本主义那一套,教出一些反革命,工人阶级是不干的,就要辞退你,明年就不下聘书了。

一百天以前我在这个地方讲过,从旧社会来的知识分子,现在没有基础了,他丧失了原来的社会经济基础,就是那五张皮没有了,他除非落在新皮上。有些知识分子现在是十五个吊桶打水,七上八下。他在空中飞,上不着天,下不着地。我说,这些人叫“梁上君子”。他在那个梁上飞,他要回去,那边空了,那几张皮没有了,老家回不去了。老家没有了,他又不甘心情愿附在无产阶级身上。你要附在无产阶级身上,就要研究一下无产阶级的思想,要跟无产阶级有点感情,要跟工人、农民交朋友。他不,他也晓得那边空了,但是还是想那个东西。我们现在就是劝他们觉悟过来。经过这一场大批判,我看他们多少会觉悟的。

那些中间状态的知识分子应当觉悟,尾巴不要翘得太高,你那个知识是有限的。我说,这种人是知识分子,又不是知识分子,叫半知识分子比较妥当。因为你的知识只有那么多,讲起大道理来就犯错误。现在不去讲那些右派知识分子,那是反动派。中间派知识分子犯的错误就是动摇,看不清楚方向,一个时候迷失方向。你那么多的知识,为什么犯错误呀?你那么厉害,尾巴翘得那么高,为什么动摇呀?墙上一克草,风吹两边倒。可见你知识不太多。在这个方面,知识多的是工人,是农民里头的半无产阶级。什么孙大雨那一套,他们一看就知道不对。你看谁人知识高呀?还是那些不大识字的人,他们知识高。决定大局,决定大方向,要请无产阶级。我就是这么一个人,要办什么事,要决定什么大计,就非问问工农群众不可,跟他们谈一谈,跟他们商量,跟接近他们的干部商量,看能行不能行。这就要到各地方跑一跑。蹲在北京可不得了,北京是什么东西都不出的呀!那里没有原料。原料都是从工人、农民那里拿来的,都是从地方拿来的。中共中央好比是个加工厂,它拿这些原料加以制造,而且要制作得好,制作得不好就犯错误。知识来源于群众。什么叫正确解决人民内部矛盾?就是实事求是,群众路线。归根到底就是群众路线四个字。不要脱离群众,我们跟群众的关系,就象鱼跟水的关系,游泳者跟水的关系一样。

对右派是不是要一棍子打死?打他几棍子是很有必要的。你不打他几棍子他就装死。对这种人,你不攻一下,不追一下?攻是必要的。但是我们的目的是攻得他回头。我们用各种方法切实攻,使他们完全孤立,那就有可能争取他们,不说全部,总是可以争取一些人变过来。他们是知识分子,有些是大知识分子,争取过来是有用的。争取过来,让他们多少做一点事。而且这一回他们帮了大忙,当了反面教员,从反面教育了人民。我们并不准备把他们抛到黄浦江里头去,还是用治病救人这样的态度。也许有一些人是不愿意过来的。象孙大雨这种人,如果他顽固得很,不愿意改,也就算了。我们现在有许多事情要办,如果天天攻,攻他五十年,那怎么得了呀!有那么一些人不肯改,那你就带到棺材里头去见阎王。你对阎王说,我是五张皮的维护者,我很有“骨气”,共产党、人民群众斗争我,我都不屈服,我都抵抗过来了。但是你晓得,现在的阎王也换了。这个阎王,第一是马克思,第二是恩格斯,第三是列宁。现在分两个地狱,资本主义世界的阎王大概还是老的,社会主义世界就是这些人当阎王。我看顽固不化的右派,一百年以后也是要受整的。


\begin{maonote}
\mnitem{1}见《列子·天瑞》。
\mnitem{2}见《左传·信公十四年》。
\end{maonote}
