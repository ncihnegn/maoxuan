
\title{读苏联《政治经济学教科书》的谈话\mnote{1}}
\date{一九五九年十二月——一九六〇年二月}
\maketitle


\section{一、关于世界观和方法论}

人们的主观运动的规律和外界的客观运动的规律是同一的。辩证法的规律,是客观所固有的,是客观运动的规律,这种客观运动的规律,反映在人们的头脑中,就成为主观辩证法。这个客观辩证法和主观辩证法是同一的。这是恩格斯多次阐明的论点。说思维和存在没有同一性,势必走到反对恩格斯的论点。当然,思维和存在不能划等号。说二者同一,不是说二者等同,不是说思维等同于存在。思维是一种特殊物质的运动形态,它能够反映客观的性质,能够反映客观的运动,并且由此产生科学的预见,而这种预见经过实践又能够转化成为事物。恩格斯举了这方面的例子。也可以拿我们的人民大会堂作一个例子。人民大会堂现在是事物,但是在它没有开始建设以前,只是一个设计的蓝图,而蓝图则是思维。这种思维又是设计工程师们集中了过去成千成万建筑物的经验,并且经过多次修改而制定出来的。许多建筑物转化成人民大会堂的蓝图——思维,然后蓝图——思维交付施工,经过建设,又转化为事物——人民大会堂。这就说明蓝图能够反映客观世界,又能够转化为客观世界;说明客观世界可以被认识,人们的主观世界可以同客观世界相符合,预见可以变为事实。

存在是第一性的,思维是第二性的,只要肯定了这一条,我们就同唯心主义划清界限了。然后还要进一步解决客观存在能否认识、如何认识的问题。还是马克思说的那些话对,思维是“移入人的头脑并在人的头脑中改造过的物质的东西”\mnote{2}。说思维和存在不能等同,是对的,但是因此就说思维和存在没有同一性,则是错误的。

教科书说,随着生产资料社会主义公有化,“人们成为自己社会经济关系的主人”,“能够完全自觉地掌握和利用规律”。\mnote{3}把事情说得太容易了。这要有一个过程。规律,开始总是少数人认识,后来才是多数人认识。就是对少数人说来,也是从不认识到认识,也要经过实践和学习的过程。任何人开始总是不懂的,从来也没有什么先知先觉。斯大林自己还不是对有些东西认识不清楚?他曾经说过,搞得不好,社会主义社会的矛盾可以发展到冲突的程度;搞得好,就可以不致发生冲突。\mnote{4}斯大林的这些话,讲得好。教科书比斯大林退了一步。认识规律,必须经过实践,取得成绩,发生问题,遇到失败,在这样的过程中,才能使认识逐步推进。要认识事物发展的客观规律,必须进行实践,在实践中必须采取马克思主义的态度来进行研究,而且必须经过胜利和失败的比较。反复实践,反复学习,经过多次胜利和失败,并且认真进行研究,才能逐步使自己的认识合乎规律。只看见胜利,没有看见失败,要认识规律是不行的。

教科书不承认现象和本质的矛盾。本质总是藏在现象的后面,只有通过现象才能揭露本质。教科书没有讲人们认识规律要有一个过程,先锋队也不例外。

看来,这本书没有系统,还没有形成体系。这也是有客观原因的,因为社会主义经济本身还没有成熟,还在发展中。一种意识形态成为系统,总是在事物运动的后面。因为思想、认识是物质运动的反映。规律是在事物的运动中反复出现的东西,不是偶然出现的东西。规律既然反复出现,因此就能够被认识。例如资本主义的经济危机,过去是八年到十年出现一次,经过多次的反复,就有可能使我们认识到资本主义社会中经济危机的规律。在土地改革中要实行平分土地的政策,也是经过反复多次以后才能认识清楚的。第二次国内战争的后期,当时的中央曾经主张按劳力分配土地,不赞成按人口平分土地。当时“左”倾冒险路线的同志认为按人口平分土地是阶级观点不明确,群众观点不充分,对发展生产不利。实践证明错的不是按人口平分土地,而是按劳动力分配土地。因为按劳动力分配土地,对富裕中农最有利。当时,他们还主张地主不分田。既然地主不杀掉,却不给谋生之道,地主有劳动力,却不分给他们土地,这种政策,是破坏社会、破坏社会生产力的政策。富农分坏田,也是这种性质的政策。中国的农民是寸土必争的。土地改革中贫农总是打富裕中农的主意,他们的办法是给富裕中农戴上富农的帽子,把富裕中农多余的土地拿出来。这个问题经过反复争论和实践,结果证明,按人口平分土地是符合我国民主革命阶段中彻底解决土地问题的客观规律的。我们在土地改革中实际上消灭了富农经济,在这点上带有社会主义革命的性质。

规律自身不能说明自身。规律存在于历史发展的过程中。应当从历史发展过程的分析中来发现和证明规律。不从历史发展过程的分析下手,规律是说不清楚的。

很有必要写出一部中国资本主义发展史来。研究通史的人,如果不研究个别社会、个别时代的历史,是不能写出好的通史来的。研究个别社会,就是要找出个别社会的特殊规律。把个别社会的特殊规律研究清楚了,那末整个社会的普遍规律就容易认识了。要从研究特殊中间,看出一般来。特殊规律搞不清楚,一般规律是搞不清楚的。例如要研究动物的一般规律,就必须分别研究脊椎动物、非脊椎动物等等的特殊规律。

绝对真理包括在相对真理里面。相对真理的积累,就使人们逐步地接近于绝对真理。不能认为相对真理只是相对真理,不包含任何绝对真理的成分,而到了一天人们忽然找到了绝对真理。

世界上没有不能分析的事物,只是:一、情况不同;二、性质不同。许多基本范畴,特别是对立统一的法则,对各种事物都是适用的。这样来研究问题、看问题,就有了一贯的完整的世界观和方法论。这本教科书就没有运用这样一贯的、完整的世界观和方法论来分析事物。

两重性,任何事物都有,而且永远有,当然总是以不同的具体的形式表现出来,性质也各不相同。例如,保守和进步,稳定和变革,都是对立的统一,这也是两重性。生物的代代相传,就有而且必须有保守和进步的两重性。稻种改良,新种比旧种好,这是进步,是变革。人生儿子,儿子比父母更聪明粗壮,这也是进步,是变革。但是,如果只有进步的一面,只有变革的一面,那就没有一定相对稳定形态的具体的动物和植物,下一代就和上一代完全不同,稻子就不成其为稻子,人就不成其为人了。保守的一面,也有积极作用,可以使不断变革中的植物、动物,在一定时期内相对固定起来,或者说相对地稳定起来,所以稻子改良了还是稻子,儿子比父亲粗壮聪明了还是人。但是如果只有保守和稳定,没有进步和变革一方面,植物和动物就没有进化,就永远停顿下来,不能发展了。

量变和质变是对立的统一。量变中有部分的质变,不能说量变的时候没有质变;质变是通过量变完成的,不能说质变中没有量变。质变是飞跃,在这个时候,旧的量变中断了,让位于新的量变。在新的量变中,又有新的部分质变。

在一个长过程中,在进入最后的质变以前,一定经过不断的量变和许多的部分质变。这里有个主观能动性的问题。如果我们在工作中,不促进大量的量变,不促进许多的部分质变,最后的质变就不能来到。

打垮蒋介石,这是一个质变。这个质变是通过量变完成的。例如,要有三年半的时间,要一部分一部分地消灭蒋介石军队和政权。而这个量变中,同样有若干的部分质变。在解放战争期间,战争经过几个不同的阶段,每个新的阶段同旧的阶段比较,都有若干性质的区别。

社会主义一定要向共产主义过渡。过渡到了共产主义的时候,社会主义阶段的一些东西必然是要灭亡的。就是到了共产主义阶段,也还是要发展的。它可能要经过几万个阶段。能够说到了共产主义,就什么都不变了,就一切都“彻底巩固”下去吗?难道那个时候只有量变而没有不断的部分质变吗?

一切事物总是有“边”的。事物的发展是一个阶段接着一个阶段不断地进行的,每一个阶段也是有“边”的。不承认“边”,就是否认质变或部分质变。

这一段\mnote{5}很有问题,不如斯大林讲得好。教科书说,社会主义制度下的矛盾不是不可调和的矛盾,这个说法不合乎辩证法。一切矛盾都是不可调和的,哪里有什么可以调和的矛盾?只能说有对抗性的和非对抗性的矛盾,不能说有不可以调和的矛盾和可以调和的矛盾。资本主义制度是没落的,社会主义制度不是没落的,所以社会主义制度的矛盾,是前进道路上的矛盾,这点教科书是说得对的。

社会主义制度下,虽然没有一个阶级推翻另一个阶级的革命,但是还有革命,技术革命,文化革命,也是革命。从社会主义过渡到共产主义是革命,从共产主义的这一个阶段过渡到另一个阶段,也是革命。共产主义一定会有很多的阶段,因此也一定会有很多的革命。

我们党里有人说,学哲学只要读《反杜林论》、《唯物主义和经验批判主义》就够了,其他的书可以不必读。这种观点是错的。马克思这些老祖宗的书,必须读,他们的基本原理必须遵守,这是第一。但是,任何国家的共产党,任何国家的思想界,都要创造新的理论,写出新的著作,产生自己的理论家,来为当前的政治服务,单靠老祖宗是不行的。只有马克思和恩格斯,没有列宁,不写出《两个策略》\mnote{6}等著作,就不能解决一九〇五年和以后出现的新问题。单有一九〇八年的《唯物主义和经验批判主义》,还不足以对付十月革命前后发生的新问题。适应这个时期革命的需要,列宁就写了《帝国主义论》\mnote{7}、《国家与革命》等著作。列宁死了,又需要斯大林写出《论列宁主义基础》和《论列宁主义的几个问题》这样的著作,来对付反对派,保卫列宁主义。我们在第二次国内战争末期和抗战初期写了《实践论》、《矛盾论》,这些都是适应于当时的需要而不能不写的。现在,我们已经进入社会主义时代,出现了一系列的新问题,如果单有《实践论》、《矛盾论》,不适应新的需要,写出新的著作,形成新的理论,也是不行的。

无产阶级哲学的发展是这样,资产阶级哲学的发展也是这样。资产阶级哲学家都是为他们当前的政治服务的,而且每个国家,每个时期,都有新的理论家,提出新的理论。英国曾经出现了培根和霍布斯这样的资产阶级唯物论者;法国曾经出现了百科全书派\mnote{8}这样的唯物论者;德国和俄国的资产阶级也有他们的唯物论者。他们都是资产阶级唯物论者,各有特点,但都是为当时的资产阶级政治服务的。所以,有了英国的,还要有法国的;有了法国的,还要有德国的和俄国的。

\section{二、关于民主革命和社会主义革命}

中国和俄国的历史经验证明:要取得革命的胜利,就要有一个成熟的党,这是一个很重要的条件。俄国布尔什维克党\mnote{9}积极地参加了俄国的民主革命,在一九〇五年提出了同资产阶级相区别的民主革命纲领,这个纲领不只是要解决推翻沙皇的问题,而且要解决在推翻沙皇的革命斗争中如何同立宪民主党\mnote{10}争夺领导权的问题。我们常说中国共产党在一九二七年的时候是幼年的党,从主要意义上来说,就是指我们党在同资产阶级联盟的时候,没有看到资产阶级会叛变革命,而且也没有做好应付这种叛变的准备。

我们为什么能够坚持长期战争而又取得了胜利呢?主要是我们对农民采取了正确的政策,例如征收公粮和收购粮食的经济政策,在不同时期实行不同的土地改革政策,在战争中紧紧依靠了农民。

这里说,中国无产阶级和资产阶级的联盟,是“在地主阶级和买办资产阶级被粉碎的条件下产生的”,\mnote{11}这个说法不对。在第一次大革命的时候,我们就和孙中山建立了这种联盟。大革命失败后,大资产阶级背叛了这个联盟。但是,我们同民族资产阶级联盟的因素还存在,如宋庆龄、何香凝\mnote{12}坚持同我们合作。九一八事变\mnote{13}后,杨杏佛、史量才\mnote{14}也转过来靠近我们了。抗战时,我们同民族资产阶级又建立了抗日的联盟;三年解放战争,我们同他们是反蒋反美联盟。

中国的资产阶级和俄国的资产阶级不同。我们历来把中国资产阶级分为两部分,一部分是官僚资产阶级,一部分是民族资产阶级。我们把官僚资产阶级这个大头吃掉了,民族资产阶级这个小头,想反抗也没有力量。他们看到中国无产阶级力量强大,同时我们又采取适当的政策对待他们,所以在民主革命胜利后,他们就有可能接受社会主义改造。

第三国际\mnote{15}在中国第一次国内革命战争失败以后的一个决议中说,在反帝反封建的同时要反对资产阶级。\mnote{16}这个决议没有区别中国资产阶级的两个部分,甚至认为中间派比蒋介石更危险。当时的“左”倾冒险分子执行了这条错误路线,结果把自己完全孤立起来。这个决议也没有区别民主革命和社会主义革命,所以立三路线\mnote{17}就提出,一省或数省胜利之日,就是社会主义革命的开始。他们不懂得,民主革命在全国胜利之日,才是社会主义革命的开始。

这一段有问题。这里讲“某些资本主义国家和过去的殖民地国家中,工人阶级通过议会和平地取得政权是有现实的可能性的。”\mnote{18}这里说“某些”究竟是哪一些呢?欧洲的主要国家,北美洲的国家,现在都武装到了牙齿,他们能让你和平地取得政权吗?我们认为,每一个国家的共产党和革命力量都要有两手准备:一手是和平方法取得胜利,一手是暴力斗争取得政权,缺一不可。而且要看到,总的趋势来说,资产阶级不愿意让步,不愿意放弃政权,他们要挣扎。资产阶级在要命的时候,他们为什么不用武力?十月革命,是准备了两手的。俄国一九一七年七月以前,列宁也曾经想用和平的方法取得胜利。七月事件\mnote{19}说明,把政权和平地转到无产阶级手里已经不可能,布尔什维克转过来进行了三个月的武装准备,举行武装起义,才取得了十月革命的胜利。十月革命以后,列宁还想用和平的方法,用赎买的方法,实行社会主义改造,消灭资本主义。但是,资产阶级勾结十四个国家,发动了反革命的武装暴动和武装干涉。在俄国党的领导下,进行了三年的武装斗争,才巩固了十月革命的胜利。至于中国革命,我们是用了革命的两手政策来对付反动派的反革命两手政策的。

说中国的阶级斗争不尖锐,这不合乎实际。中国革命可尖锐呢。我们连续打了二十二年的仗。我们用战争推翻了国民党的统治,接着没收了在整个资本主义经济中占百分之八十的官僚资本,这样才使我们有可能对占百分之二十的民族资本,采取和平的方式,逐步地进行社会主义改造,并且利用他们的经济、文化来为社会主义建设服务。在改造过程中,还经过了“三反”、“五反”\mnote{20}那样激烈的斗争。

列宁指出的那句话很对。一直到现在,社会主义革命成功的国家,资本主义发展水平比较高的,只有东德和捷克;其他的国家,资本主义发展水平都比较低。西方资本主义发展水平很高的国家,革命都没有革起来。列宁曾经说过,革命首先从帝国主义世界的薄弱环节突破\mnote{21}。十月革命时的俄国是这样的薄弱环节,十月革命后的中国也是这样的薄弱环节。俄国和中国的共同点是:都有相当数量的无产阶级,都有大量的农民群众,都是大国。

这一段\mnote{22}值得研究,对民主革命转变到社会主义革命没有讲清楚。十月革命是社会主义革命,它附带地完成了民主革命遗留下来的任务。十月革命一开始,就宣布了土地国有令,但是完全解决农民土地问题,在革命胜利以后还用了一段时间。我国资本主义发展水平,同十月革命以前的俄国差不多,而封建经济则是更大量地存在。我们经过解放战争,赢得了民主革命的胜利。一九四九年中华人民共和国建立,标志着新民主主义革命阶段的基本结束和社会主义革命阶段的开始。我们立即没收了占全国工业、运输业固定资产百分之八十的官僚资本,转为全民所有。同时,用了三年的时间,完成全国的土地改革。如果因此说全国解放以后,“革命在最初阶段主要是资产阶级民主革命性质的,只是后来才逐渐地发展成为社会主义革命”,这是不对的。

中国新民主主义革命的任务,长时期内是反帝反封建。在解放战争时期,我们又提出了反对官僚资本主义。反对官僚资本主义的斗争,包含着两重性:一方面,反官僚资本就是反买办资本,是民主革命的性质;另一方面,反官僚资本就是反对大资产阶级,又带有社会主义革命的性质。过去有一种说法,民主革命和社会主义革命可以毕其功于一役。这种说法,混淆了两个革命阶段,是不对的;但只就反对官僚资本来说,是可以的。官僚资本和民族资本的比例,是八比二。我们在解放后没收了全部官僚资本,就把中国资本主义的主要部分消灭了。

解放以后,民族资产阶级走上社会主义改造的道路,这是逼出来的。我们打倒了蒋介石,没收了官僚资本,完成了土地改革,进行了“三反”、“五反”,实现了合作化,从一开始就控制了市场。这一系列的变化,一步一步地逼着民族资产阶级不能不走上接受改造的道路。另一方面,《共同纲领》\mnote{23}规定了各种经济成分各得其所,使资本家有利可图的政策;宪法又给了他们一张选票、一个饭碗的保证,这些又使他们感到接受改造就能保持一定的地位,并且能够在经济上、文化上发挥一定的作用。

现在,在公私合营企业中,资本家实际上已经成了国家的雇员,对企业没有实际上的管理权。我们对民族资产阶级是拉住它,又整住它。中国民族资本家从来没有统一过,解放前有什么上海帮、广东帮、天津帮之类行会性的组织,解放以后我们帮助他们成立全国工商业联合会,把他们统一起来,又对资本家区别不同情况,分而治之。这几年每年还给他们一亿二千万元定息\mnote{24},实行大规模的收买,收买整个阶级,收买他们整个阶级的几百万人,包括家属在内。

教科书关于中国的资本主义所有制转变为全民所有制的问题,说得不对。它只说了我们对民族资本的改造政策,没有说我们对官僚资本的没收政策。对于民族资本,也没有说我们是经过了三个步骤,即加工定货、统购包销、公私合营,来实现对它的社会主义改造。就每个步骤来讲,如加工定货,也是逐步前进的。公私合营也经过了从单个企业的公私合营到全行业公私合营的过程。由于我们的国家一方面掌握了原料,另一方面又控制着市场,同时又对资本家贷给流动资金,这样就使民族资本家不能不接受改造。实行这样的改造政策,不仅生产没有受到破坏,而且有些私营工厂在过去几年中还进行了部分的扩建。资本家由于在过去几年中有利可图,有些人也还自愿地向工厂进行投资。我们在处理资产阶级的问题上,有很丰富的经验,创造了许多新的经验。例如,公私合营以后给资本家定息,就是一个新经验。

教科书的这个提法\mnote{25}不妥当。中国民主革命胜利以后,能够走上社会主义的道路,主要是由于我们推翻了帝国主义、封建主义、官僚资本主义的统治。国内的因素是主要的。已经胜利了的社会主义国家对我们的帮助,是一个重要条件。但是,它不能决定我们能不能够走社会主义道路的问题,只能影响我们在走上社会主义道路以后是前进得快一点还是慢一点的问题,有帮助可以快一些,没有帮助会慢一些。所谓帮助,包括他们经济上的援助,同时也包括我们对他们成功和失败的、正面和反面的经验的学习。

教科书承认我们搞国家资本主义是对的,但是,它没有写清楚我国的国家资本主义的发展过程和阶段,它也没有吸收我们所说的公私合营是四分之三的社会主义这个意思。现在来说,已经不是四分之三,而是十分之九,甚至更多了。

我们是联合农民来反对资本家。而列宁在一个时期曾经说过,宁愿同资本家打交道,想把资本主义变成国家资本主义,来对付小资产阶级的自发势力\mnote{26}。这种不同的政策,是由不同的历史条件所决定的。

\section{三、关于社会主义建设}

社会主义这个阶段,又可能分为两个阶段,第一个阶段是不发达的社会主义,第二个阶段是比较发达的社会主义。后一阶段可能比前一阶段需要更长的时间。经过后一阶段,到了物质产品、精神财富都极为丰富和人们的共产主义觉悟极大提高的时候,就可以进入共产主义社会了。

建设社会主义,原来要求是工业现代化,农业现代化,科学文化现代化,现在要加上国防现代化。在我们这样的国家,完成社会主义建设是一个艰巨任务,建成社会主义不要讲得过早了。

“每一个”国家都“具有自己特别的具体的社会主义建设的形式和方法”,\mnote{27}这个提法好。一八四八年有一个《共产党宣言》,在一百一十年以后,又有一个“共产党宣言”,这就是一九五七年各国共产党的莫斯科宣言\mnote{28}。在这个宣言中,就讲到了普遍规律和具体特点相结合的问题。

我们的情况和他们不同,一条是有苏联的存在和帮助,这是一个很大的因素。但是,主要是国内的因素。我们搞了二十二年的根据地政权工作,积累了根据地管理经济的经验,培养了一批管理经济的干部,同农民建立了联盟,从他们那里得到了粮食和原料。所以,在全国解放以后,很快地进行和完成了经济的恢复工作。接着,我们就提出了过渡时期的总路线\mnote{29},把主要力量放在社会主义革命方面,同时开始了第一个五年计划\mnote{30}的建设。由于我们没有管理全国经济的经验,所以第一个五年计划的建设,不能不基本上照抄苏联的办法。到生产资料所有制的社会主义改造基本完成以后,我们就提出了建设社会主义的两种方法的问题,在一九五八年正式形成了社会主义建设的总路线\mnote{31}。

解放后,三年恢复时期,对搞建设,我们是懵懵懂懂的。接着搞第一个五年计划,对建设还是懵懵懂懂的,只能基本上照抄苏联的办法,但总觉得不满意,心情不舒畅。一九五六年,基本完成生产资料所有制的三大社会主义改造。一九五六年春,同三十几个部长谈话,一个问题一个问题凑,提出了《论十大关系》\mnote{32}。当时还看了斯大林一九四六年选举演说\mnote{33},苏联在一九二一年产钢四百多万吨,一九四〇年增加到一千八百万吨,二十年中增加了一千四百万吨。当时就想,苏联和中国都是社会主义国家,我们是不是可以搞得快点多点,是不是可以用一种更多更快更好更省的办法建设社会主义。后来提出了建设社会主义的两种方法的问题,提出了多快好省,提出了“促进委员会”,要当社会主义的促进派,不当促退派。还搞了一个农业发展纲要四十条\mnote{34}。此外没提其他的具体措施。

恩格斯说,在社会主义制度下,“按照预定计划进行社会生产就成为可能”,\mnote{35}这是对的。资本主义社会里,国民经济的平衡是通过危机达到的。社会主义社会里,有可能经过计划来实现平衡。但是也不能因此就否认我们对必要比例的认识要有一个过程。教科书说“自发性和自流性同生产资料公有制的存在是不相容的”\mnote{36},可以这样说。但是不能认为社会主义社会里就没有自发性和自流性。我们对规律的认识,不是一开始就是完善的。实际工作告诉我们,在一个时期内,可以有这样的计划,也可以有那样的计划;可以有这些人的计划,也可以有那些人的计划。不能说这些计划都是完全合乎规律的。实际上是,有些计划合乎规律,或者基本上合乎规律,有些计划不合乎规律,或者基本上不合乎规律。

认为对比例关系的认识,不要有个过程,不要经过成功和失败的比较,不要经过曲折的发展,这都是形而上学的看法。自由是对必然的认识并根据对必然的认识成功地改造客观世界。这个必然不是一眼就能看穿看透的。世界上没有天生的圣人。到了社会主义社会,也还是没有什么“先知先觉”。为什么教科书过去没有出版,为什么出版了以后要一次又一次地修改,还不是因为过去认识不清楚,现在也还认识不完善吗?拿我们自己的经验来说,开始我们也不懂得搞社会主义,以后在实践中逐步有了认识。认识了一些,也不能说认识够了。如果认识够了,那就没有事做了。

计划是意识形态。意识是实际的反映,又对实际起反作用。过去我们计划规定沿海省份不建设新的工业,一九五七年以前没有进行什么新建设,整整耽误了七年的时间。一九五八年以后,才开始在这些省份进行大的建设,两年中得到很快的发展。这就说明,像计划这类意识形态的东西,对经济的发展和不发展,对经济发展的快慢,有着多么大的作用。

这段\mnote{37}讲得好。“社会主义计划化建立在严格的科学基础上”,这个当作任务来提,是对的。问题在于能否掌握有计划发展的规律,掌握到什么程度;在于是否善于利用这个规律,能利用到什么程度。

社会主义国家的经济能够有计划按比例地发展,使不平衡得到调节,但是不平衡并不消失。“物之不齐,物之情也。”\mnote{38}因为消灭了私有制,可以有计划地组织经济,所以就有可能自觉地掌握和利用不平衡是绝对的、平衡是相对的这个客观规律,以造成许多相对的平衡。

不以规律为计划的依据,就不能使有计划按比例发展的规律的作用发挥出来。

要经常保持比例,就是由于经常出现不平衡。因为不成比例了,才提出按比例的任务。平衡了又不平衡,按比例了又不按比例,这种矛盾是经常的、永远存在的,教科书不讲这个观点。

社会主义经济发展过程中,经常出现不按比例、不平衡的情况,要求我们按比例和综合平衡。例如,经济发展了,到处感到技术人员不够,干部太少,于是就出现干部的需要和干部的分配的矛盾,这就促进我们多办学校,多培养干部,来解决这个矛盾。

这段\mnote{39}写得不对,既否认了资本主义制度下的某种平衡,也否认了社会主义制度下的某种不平衡。资本主义技术的发展,有不平衡的方面,也有平衡的方面。问题是这种平衡和不平衡,同社会主义制度下的平衡和不平衡,在性质上不同。在社会主义制度下,技术发展有平衡,也有不平衡。例如解放初期,我们的地质工作人员只有二百来人,地质勘探情况同国民经济发展的需要极不平衡,经过几年来努力加强工作,这种不平衡已经走向平衡。但是,技术发展的新的不平衡又出现了。我国目前手工劳动还占很大比重,同发展生产、提高劳动生产率的需要不平衡,因此有必要广泛开展技术革新和技术革命,来解决这个不平衡。特别值得注意的,新的技术部门出现以后,技术发展不平衡的状况更加显著,例如,我们现在要搞尖端技术,就感到许多东西不相适应。

技术的发展是这样,经济的发展也是这样。教科书没有接触到社会主义生产发展的波浪式前进。说社会主义经济的发展一点波浪也没有,这是不可能设想的。任何事物的发展都不是直线的,而是螺旋式地上升,也就是波浪式发展。我们读书也是波浪式的,读书之前要做别的事情,读了几个钟头以后,要休息,不能无日无夜地读下去。今天读得多,明天读得少;而且每天读的时候,有时议论多,有时议论少。这些都是波浪,都是起伏。平衡是对不平衡来说的,没有了不平衡,还有什么平衡?事物的发展总是不平衡的,因此有平衡的要求。平衡和不平衡的矛盾,在各方面、各部门、各个部门的各个环节都存在,不断地产生,不断地解决。有了头年的计划,又要有第二年的计划;有了年度的计划,又要有季度的计划;有了季度的计划,还要有月计划。一年十二个月,月月要解决平衡和不平衡的矛盾。计划常常要修改,就是因为新的不平衡的情况又出来了。

生产资料优先增长的规律,是一切社会扩大再生产的共同规律。资本主义社会如果不是生产资料优先增长,它的社会生产也不能不断增长。斯大林把这个规律具体化为优先发展重工业。斯大林的缺点是过分强调了重工业的优先增长,结果在计划中把农业忽略了。前几年东欧各国也有这个问题。我们把这个规律具体化为:在优先发展重工业的条件下,工农业同时并举。我们实行的几个同时并举,以工农业同时并举为最重要。统计局的材料,说我国日用品销于农村的占百分之六十三左右。不实行工农业并举,这怎么能行?我们在一九五六年提出工农业并举,到现在已经四年了,真正实行是在一九六0年。

我国人民现在还要像苏联那个时候一样,忍受一点牺牲,但是只要我们能够使农业、轻工业、重工业都同时高速度地向前发展,我们就可以保证在迅速发展重工业的同时,适当改善人民的生活。苏联和我们的经验都证明,农业不发展,轻工业不发展,对重工业的发展是不利的。

关于工农业的关系问题,要说工业向农业要求扩大市场,也要说农业向工业要求增加各种工业品的供应。要保证农民得到更多的工业品,保证农民提高自己的文化水平。

多发展农业和轻工业,多为重工业创造一些积累,从长远来看,对人民是有利的。只要农民和全国人民了解到,国家在买卖农产品和轻工业品方面赚的钱是用来干什么的,他们就会赞成,不会反对。农民自己已经提出了农业支援工业的口号,就是证明。当然,赚钱不能过分,工农业产品的交换不能够完全等价,但要相当地等价。

这里说到一九二五年到一九五七年苏联的生产资料生产增长了九十三倍,消费资料生产增长了十七点五倍,问题是,九十三同十七点五的比例,是否对发展重工业有利。这么多年来,消费品生产只增长了那么一些,为什么在这个问题上又不讲“物质刺激”呢?要使重工业迅速发展,就要大家都有积极性,大家都高兴。而要这样,就必须使工业和农业同时并举,轻重工业同时并举。

在农业区,我们也要搞工业。

这一段\mnote{40}的说法,原则上对。工业的发展当然要快于农业。但是,提法要适当,不能把工业强调到不适当的地位,否则一定会发生问题。拿我们的辽宁来说,这个省的工业很多,城市人口已经占全省人口的三分之一。过去因为总是把工业放在第一位,没有同时注意大力发展农业,结果本省的农业不能给城市保证粮食、肉类、蔬菜的供应,总是要中央从外省往那里运粮,运肉类,运蔬菜。主要的问题是农业劳动力紧张,没有必要的农业机械,使农业生产的发展受到限制,增长较慢。过去我们因为看到这里的工业多,在给他们任务的时候,总是要他们多抓工业。没有了解到,恰恰是因为那里的工业比重大,更应该注意好好地抓农业、发展农业,不能只强调抓工业。工业发展了,城市人口增加了,对农业的要求也就更多了。因此就必须使农业能够和工业得到相适应的发展。在农村劳动力减少的情况下,必须对农业进行技术改造,提高劳动生产率,更多地增加农产品的生产。

我们的提法是在优先发展重工业的条件下,发展工业和发展农业同时并举。所谓并举,并不否认重工业优先增长,不否认工业发展快于农业;同时,并举也并不是要平均使用力量。例如,一九六〇年估计可生产钢材一千三四百万吨,拿出十分之一的钢材来搞农业技术改造和水利建设,其余十分之九的钢材,主要还是用于重工业和交通运输的建设,在目前的条件下,这就是工农业并举了。这样做怎么会妨碍优先发展重工业和加快发展工业呢?

这里说,“列宁的社会主义建设计划是以尽力发展国营工业和农民经济之间的经济联系为前提的”\mnote{41},说得好。我们在长期战争中曾经打断了城乡的旧的经济联系,在解放初期,全国普遍召开物资交流会,在新的基础上恢复城乡的经济联系,包括恢复过去的牙行、经纪等。

这段话\mnote{42}说得对。资本主义长期着重发展轻工业。我们把生产资料优先增长的公式具体化为:在优先发展重工业的条件下,实行几个同时并举;每一个并举中间,又有主导的方面。例如,中央和地方,以中央为主导;工业与农业,以工业为主导。农业上不去,许多问题得不到解决。东欧各国过去几年都是这样的。从一九六〇年起,我们要增加农业所需要的钢材。

现在我国工业化速度也是一个很尖锐的问题。原来的工业越落后,速度问题也越尖锐,不但国与国之间比较起来是这样,就是一个国家内部,这个地区和那个地区比较起来也是这样。例如,我国的东北和上海,因为那里的工业基础比较好,国家对这些地区的投资增长速度相对地较慢一些。而另外一些工业基础薄弱,而又迫切需要发展的地区,国家在这些地区的投资增长速度却快得多。上海解放后十年共投资二十二亿元,其中包括资本家投资二亿元。上海原有工人五十多万,现在全市工人除了已调出几十万人外,还有一百多万,只比过去增加一倍。这同一些职工大量增加的新城市相比较,就可以明显地看到工业基础差的地区的速度问题更加尖锐。

苏联的工农业劳动生产率,现在还没有超过美国,我们则差得更远。人口虽多,但是劳动生产率远远比不上人家,还要继续紧张地努力若干年,分几个阶段,把我们的国家搞强大起来,使我们的人民进步起来。

提高劳动生产率,一靠物质技术,二靠文化教育,三靠政治思想工作。后两者都是精神作用。

社会主义竞赛这一节,一般写得不错。引用的斯大林的话也好。斯大林讲了先进者给予落后者以帮助,求得普遍的提高。普遍提高之后,仍然有先进和落后的矛盾,又要求进一步的普遍提高。

苏联在第一个五年计划完成以后,大工业总产值占工农业总产值的百分之七十,就宣布实现了工业化。根据统计,我国一九五八年工业总产值占工农业总产值的百分之六十六点六;一九五九年计划完成后,估计一定会超过百分之七十。即使这样,我们还可以不宣布实现了工业化。我们还有五亿多农民从事农业生产。如果现在就宣布实现了工业化,不仅不能确切地反映我国国民经济的实际状况,而且可能由此产生松劲情绪。

我们现在还不一般地提自动化。机械化要讲,但也不要讲得过头。机械化、自动化讲得过多了,会使人们看不起半机械化和土法生产。过去就曾经有过这样的偏向,大家都片面追求新技术、新机器,追求大规模、高标准,看不起土的、半洋半土的,看不起中小的。提出洋土并举、大中小并举后,这个偏向才克服。

我们要实现全盘机械化,第二个十年还不行,恐怕要第三个十年以至更长的时间。在一个时期内因为机器不够,要提倡半机械化和改良农具。最近苏北发明一种挖泥的新技术,大大提高了劳动生产率。这样的办法,应该大大提倡。

资本主义各国,苏联,都是靠采用最先进的技术,来赶上最先进的国家,我国也要这样。拿汽车来说,我们这样的大国,最少应该有三四个像长春汽车厂那样的制造厂。就是在搞大的、洋的方面,我们也不能指靠人家。一九五八年提破除迷信,自己动手。经过一九五九年春夏的一段反复,证明自己来搞,是可以做好的。

反对分散建设资金,如果是说建设单位搞得过多,因而都不能按期竣工,这当然是要反对的。如果因此就反对建设中小型企业,那就不对。我国新的工业基地,主要是在一九五八年大量发展中小型企业的基础上建立起来的。今后钢铁工业在建设一些大型基地的同时,还要建设一批中型和小型的钢铁基地。过去的中小型企业对钢铁工业的发展起了很大的作用,拿一九五九年来说,全国全年生产的生铁是二千多万吨,其中一半是由中小型企业生产的。今后中小型钢铁企业对钢铁工业的发展还要起很大的作用。许多小的会变成中的,许多中的会变成大的,同落后的会变成先进的、土法的会变成洋法的一样,这是客观发展的规律。

都是全民所有制的企业,实行不实行中央和地方分权,哪些企业由谁去管,这些都是有关建设的重大问题。中央不能只靠自己的积极性,还必须同时依靠地方的积极性。过去中央有些部门,把地方办的事业不当作自己的,只把直属的企业看成自己的,这种看法妨碍了充分发挥地方的积极性。中央和地方都要注意发挥企业的积极性。去年有些基本建设单位实行了投资包干制,就大大发挥了这些单位的积极性。

我们在《关于农业合作化问题》\mnote{43}中曾经说到,要用四个五年计划到五个五年计划来实现农业机械化。一九五九年以前,我们的农业生产,主要靠兴修水利。一九五九年我国七个省遇到很大的旱灾,如果没有过去几年的水利建设,要不减产而能增产,是不能设想的。

一九五九年冬,全国参加搞水利的人有七千七百多万。我们要继续搞这样大规模的运动,使我们的水利问题基本上得到解决。从一年、二年或者三年来看,花这么多的劳动,粮食单位产品的价值当然很高,单用价值规律来衡量,好像是不合算的。但是,从长远来看,粮食可以增加得更多更快,农业生产可以稳定增产。那末,每个单位产品的价值也就更便宜,人民对粮食的需要也就更能够得到满足。

级差地租不完全是由客观条件形成的。“事在人为”,在土地改良里是很重要的。自然条件相同,经济条件相同,一个地方“人为”了,结果就好;一个地方“人不为”,结果就不好。例如,在河北省内,京汉路沿线的机井很多,津浦路沿线的机井却很少,同样是河北平原,同样是交通方便,但是土地的改良却各不相同。这里可能有土地利于或不利于改良的原因,也可能有不同的历史原因,但是,最主要的原因是“事在人为”。同在上海,有的养猪养得好,有的却养不好。崇明县,原来说那里芦苇多,不利于养猪,现在却看到芦苇多的条件下不但不妨碍养猪,反而有利于养猪。这些说明养猪多少、好坏这件事,同世界观是密切相关的,同“事在人为”是密切相关的。北京昌平县过去常闹水旱灾害,修了十三陵水库,情况改善了,还不是“事在人为”吗?河南省计划在一九五九、一九六〇年以后再用几年,治理黄河,完成几个大型水利工程的建设,也都是“事在人为”。实际上,精耕细作,机械化,集约化,都是“事在人为”。

这最后一句话\mnote{44}讲得不对。拿我国来说,粮食不能说已经建立了必要的后备,苏联也同样有这个问题。应该改成社会主义国家必须建立必要的后备。这是一个任务,不能说是所有的社会主义国家都已经解决了。

在社会主义工业化过程中,随着农业机械化的发展,农业人口会减少。如果让减少下来的农业人口,都拥到城市里来,使城市人口过分膨胀,那就不好。从现在起,我们就要注意这个问题。要防止这一点,就要使农村的生活水平和城市的生活水平大致一样,或者还好一些。

这里把厉行节约,积累大量的物力和财力,当成只是在极为困难的情况下要做的事情,这是不对的。难道困难少了,就不需要厉行节约了吗?

在国与国的关系上,我们主张,各国尽量多搞,以自力更生、不依赖外援为原则。自己尽可能独立地搞,凡是自己能办的,必须尽量地多搞。只有自己实在不能办的才不办。特别是农业,更应当搞好。吃饭靠外国,危险得很,打起仗来,更加危险。他们和我们相反,不提倡各国尽量搞,而提倡“可以不必生产能靠其他国家供应来满足需要的产品”\mnote{45}。似乎想用经济力量来控制别的国家。他们不懂得,这样“管”起来,对他们自己也不见得有利。

列宁这句话,“社会主义是生气勃勃的,创造性的,是人民群众本身的创造”\mnote{46},讲得好。我们的群众路线,就是这样的。是不是合乎列宁主义呢?教科书在引用这句话以后,讲要吸收广大劳动群众“直接地和积极地参加生产管理,参加国家机关的工作,参加国家社会生活的一切部门的领导”,也讲得好。但是,讲是讲,做是做,做起来并不容易。

这里讲到苏联劳动者享受的各种权利时,没有讲劳动者管理国家、管理军队、管理各种企业、管理文化教育的权利。实际上,这是社会主义制度下劳动者最大的权利,最根本的权利。没有这种权利,劳动者的工作权、休息权、受教育权等等权利,就没有保证。

这段\mnote{47}的最后一句话讲得好。要达到这个目的,就要做工作。我们的经验,如果干部不放下架子,不同工人打成一片,工人就往往不把工厂看成自己的,而看成干部的。干部的老爷态度使工人不愿意自觉地遵守劳动纪律,而且破坏劳动纪律的往往首先是那些老爷们。不能以为,在社会主义制度下,不用做工作,就自然会出现劳动者和企业领导人员的创造性合作。

我很担心我们的干部子弟,他们没有生活经验和社会经验,可是架子很大,有很大的优越感。要教育他们不要靠父母,不要靠先烈,要完全靠自己。

反对平均主义,是正确的;反过头了,会发生个人主义。过分悬殊也是不对的。我们的提法是既反对平均主义,也反对过分悬殊。

历史的规律是,只有经过革命战争才能消灭阶级,只有消灭了阶级才能永远消灭战争。不进行革命战争,要消灭阶级,我们不相信。没有消灭阶级,要消灭战争武器,这不可能。世界上从有历史以来,没有不搞实力地位的事情。任何阶级、任何国家,都是要搞实力地位的。搞实力地位,这是历史的必然趋势。国家是阶级统治的机关,军队是阶级的实力。只要有阶级,就不能不搞军队。当然我们是希望不打世界大战的,我们是希望和平的。我们赞成用极大的努力来禁止原子战争,并且争取两个阵营签订互不侵犯协定。争取十年、二十年的和平,是我们最早提出的主张。如果能够实现这个主张,对整个社会主义阵营,对我国的社会主义建设,都是很有利的。

\section{四、关于政治经济学的一些问题}

我们要以生产力和生产关系的平衡和不平衡,生产关系和上层建筑的平衡和不平衡,作为纲,来研究社会主义社会的经济问题。政治经济学研究的对象主要是生产关系,但是要研究清楚生产关系,就必须一方面联系研究生产力,另一方面联系研究上层建筑对生产关系的积极作用和消极作用。这本书提到了国家,但没有加以研究,这是这本书的缺点之一。当然,在政治经济学的研究中,生产力和上层建筑这两方面的研究不能太发展了。生产力的研究太发展了,就成为自然科学、技术科学了;上层建筑的研究太发展了,就成为阶级斗争论、国家论了。马克思主义三个组成部分中的科学社会主义部分所研究的,是阶级斗争学说、国家论、党论、战略策略,等等。

生产力和生产关系之间、生产关系和上层建筑之间的矛盾和不平衡是绝对的。上层建筑适应生产关系,生产关系适应生产力,或者说它们之间达到平衡,总是相对的。平衡和不平衡这个矛盾的两个侧面,不平衡是绝对的,平衡是相对的。如果只有平衡,没有不平衡,生产力、生产关系、上层建筑就不能发展了,就固定了。矛盾、斗争、分解是绝对的,统一、一致、团结是相对的,有条件的。有了这样的观点,就能够正确认识我们的社会和其他事物;没有这样的观点,认识就会停滞、僵化。

从世界的历史来看,资产阶级工业革命\mnote{48},不是在资产阶级建立自己的国家以前,而是在这以后;资本主义的生产关系的大发展,也不是在上层建筑革命以前,而是在这以后。都是先把上层建筑改变了,生产关系搞好了,上了轨道了,才为生产力的大发展开辟了道路,为物质基础的增强准备了条件。当然,生产关系的革命,是生产力的一定发展所引起的。但是,生产力的大发展,总是在生产关系改变以后。拿资本主义发展的历史来说,正如马克思所说的,简单的协作就创造了一种生产力\mnote{49}。手工工场就是这样一种简单协作,在这种协作的基础上,就产生了资本主义发展第一阶段的生产关系。手工工场是非机器生产的资本主义。这种资本主义生产关系产生了一种改进技术的需要,为采用机器开辟了道路。在英国,是资产阶级革命(十七世纪)以后,才进行工业革命(十八世纪末到十九世纪初)。法国、德国、美国、日本,都是经过不同的形式,改变了上层建筑、生产关系之后,资本主义工业才大大发展起来。

首先制造舆论,夺取政权,然后解决所有制问题,再大大发展生产力,这是一般规律。在无产阶级革命夺取政权以前,不存在社会主义的生产关系,而资本主义的生产关系,在封建社会中已经初步成长起来。在这点上,无产阶级革命和资产阶级革命有所不同。但是,这个一般规律,对无产阶级革命和资产阶级革命都是适用的,基本上是一致的。

这里讲发展大工业是对经济进行社会主义改造的基础,说得不完全。一切革命的历史都证明,并不是先有充分发展的新生产力,然后才改造落后的生产关系,而是要首先造成舆论,进行革命,夺取政权,才有可能消灭旧的生产关系。消灭了旧的生产关系,确立了新的生产关系,这样就为新的生产力的发展开辟了道路。

教科书在这里承认社会主义社会中生产关系和生产力的矛盾的存在,也讲要克服这个矛盾,但是不承认矛盾是动力。

这一段\mnote{50}说批评和自我批评是社会主义社会发展的强大动力,这个说法不妥当。矛盾才是动力,批评和自我批评是解决矛盾的方法。

这一段\mnote{51}只说社会主义社会的特点是“团结一致,十分稳定”,不说社会主义社会内部的矛盾;说精神上政治上的一致,是社会主义国家强大的社会发展动力,不说社会矛盾是社会发展的动力。这样一来,矛盾的普遍性这个规律,在他们那里被否定了,辩证法在他们那里就中断了。没有矛盾就没有运动。社会总是运动发展的。在社会主义时代,矛盾仍然是社会运动发展的动力。因为不一致,才有团结的任务,才需要为团结而斗争。如果总是十分一致,那还有什么必要不断进行团结的工作呢?

“决定因素之一”、“根本方法之一”\mnote{52},这个提法可以赞成。但是当作决定性动力,就不对了。要保证人们吃饱饭,然后人们才能继续生产。没有这一条是不行的。物质利益是一个重要原则,但总不是唯一的原则,总还有另外的原则,教科书中不也是常说“精神鼓励”原则吗?同时,物质利益也不能单讲个人利益、暂时利益、局部利益,还应当讲集体利益、长远利益、全局利益,应当讲个人利益服从集体利益,暂时利益服从长远利益,局部利益服从全局利益。他们现在强调的是个人、暂时、局部的利益,不强调集体、长远和全局的利益。

我们要教育人民,不是为了个人,而是为了集体,为了后代,为了社会前途而努力奋斗。要使人民有这样的觉悟。教科书对于为前途、为后代总不强调,只强调个人物质利益。常常把物质利益的原则,一下子变成个人物质利益的原则,有一点偷天换日的味道。他们不讲全体人民的利益解决了,个人的利益也就解决了;他们所强调的个人物质利益,实际上是最近视的个人主义。这种倾向,是资本主义时期无产阶级队伍中的经济主义、工团主义\mnote{53}在社会主义时期的表现。历史上许多资产阶级革命家英勇牺牲,他们也不是为个人的眼前利益,而是为他们这个阶级的利益,为这个阶级的后代的利益。

公是对私来说的,私是对公来说的。公和私是对立的统一,不能有公无私,也不能有私无公。我们历来讲公私兼顾,早就说过没有什么大公无私,又说过先公后私。个人是集体的一分子,集体利益增加了,个人利益也随着改善了。

关于劳动生产中人与人的关系问题,苏联教科书只有一句空洞的话,即社会主义制度下人与人的关系是“同志式的互助合作的关系”\mnote{54}。这句话是对的,但是没有展开,没有分析,没有接触到实质问题。教科书没有写这方面的文章。所有制问题基本解决以后,最重要的问题是管理问题,即全民所有的企业如何管理的问题,集体所有的企业如何管理的问题,这也就是人与人的关系问题。这方面是大有文章可做的。当然全民所有制的企业,集体所有制的企业,在所有制方面还要有它的变化,有它的发展。但是所有制的变革,在一定时期内总是有底的,总是不能没有限度的。例如,集体所有制过渡到全民所有制之后,在相当长的时期内,它的性质总还是社会主义全民所有制。当然将来还会从社会主义全民所有制过渡到共产主义全民所有制,达到了这一步,它的性质,在一定时期内又会没有多大变化。可是在一定时期内,即所有制性质相对稳定的时期内,在劳动生产中人与人的关系,却不能不是不断变革的。例如我们的国营企业,解放以后,一直是社会主义全民所有制的性质,而在这十年中间,人与人在劳动生产中的关系,变化却是很大的。在这方面,我们做了很多文章。要领导者采取平等态度待人;一年、两年整一次风;进行大协作;对企业的管理,采取集中领导和群众运动相结合,工人群众、领导干部和技术人员三结合,干部参加劳动,工人参加管理,不断改革不合理的规章制度,等等。这些方面都是属于劳动生产中人与人的关系。这种关系是改变还是不改变,对于推进还是阻碍生产力的发展,都有直接的影响。

在劳动生产中人与人的关系,也是一种生产关系。在这里,例如领导人员以普通劳动者姿态出现,以平等态度待人,改进规章制度,干部参加劳动,工人参加管理,领导人员、工人和技术人员三结合,等等,有很多文章可做。生产关系包括生产资料所有制、劳动生产中人与人的关系、分配制度这三个方面。所有制方面的革命,在一定时期内是有底的,例如集体所有制过渡到全民所有制,整个国民经济变成了单一的全民所有制以后,在相当长的时期内,它还是全民所有制。但是,人们在劳动生产和分配中的相互关系,总要不断地改进,这方面很难说有什么底。原始社会的公有制度,时间很长,多少万年都是同样性质的,但是人们在劳动生产中的相互关系却有很多变化。可以设想,将来全世界实现共产主义以后,人们在劳动生产和分配中的相互关系,还会有无穷的变化,但是所有制方面不会有多大变化。

关于产品分配,苏联教科书写得最不好,要重新另写,换一种写法。应当强调艰苦奋斗,强调扩大再生产,强调共产主义前途、远景,要用共产主义理想教育人民。要强调个人利益服从集体利益,局部利益服从整体利益,眼前利益服从长远利益。要讲兼顾国家、集体和个人,把国家利益、集体利益放在第一位,不能把个人利益放在第一位。不能像他们那样强调个人物质利益,不能把人引向“一个爱人,一座别墅,一辆汽车,一架钢琴,一台电视机”那样为个人不为社会的道路上去。“千里之行,始于足下”,如果只看到足下,不想到前途,不想到远景,那还有什么千里旅行的兴趣和热情呢?

人民的需要是逐步满足的。

共产主义社会,实行按需分配了,也不能一下子完全满足需要,因为需要是不断被创造的。拿过去来说,没有文字,人们就没有对文具的需要,文字产生了,人们对文具的需要也随着创造出来了。拿现在来说,因为发明了电视机,所以人们对于它的需要也随着提出来了。

人们生活的需要,是不断增长的。需要刺激生产的不断发展,生产也不断创造新的需要。人们对粮食的需要,在数量方面总不能是无限制的,但是在品种方面也会变化。

社会主义社会里面的按劳分配、商品生产、价值规律等等,现在是适合于生产力发展的要求的,但是,发展下去,总有一天要不适合生产力的发展,总有一天要被生产力的发展所突破,总有一天它们要完结自己的命运。能说社会主义社会里面的经济范畴都是永久存在的吗?能说按劳分配这些范畴是永久不变的,而不是像其他范畴一样都是历史范畴吗?

写出一本社会主义共产主义政治经济学教科书,现在说来,还是一件困难的事情。有英国这样一个资本主义发展成熟的典型,马克思才能写出《资本论》。社会主义社会的历史,至今还不过四十多年,社会主义社会的发展还不成熟,离共产主义的高级阶段还很远。现在就要写出一本成熟的社会主义共产主义政治经济学教科书,还受到社会实践的一定限制。

社会主义政治经济学教科书,究竟怎样写才好?从什么地方开始写起?这个问题很值得研究。

如果我们写社会主义政治经济学,也可以从所有制出发。先写生产资料私有制变革为生产资料公有制,把官僚资本主义私有制和民族资本主义私有制变为社会主义公有制;把地主土地私有制变为个体农民私有制,再变为社会主义集体所有制;把个体的手工业变为社会主义集体所有制。然后,再写两种社会主义公有制的矛盾,以及这个矛盾发展的趋势和解决的办法,社会主义集体所有制如何过渡到社会主义全民所有制。集体所有制本身有个变化、变革的过程,全民所有制本身也有变化、变革的过程,如体制下放、分级管理、企业自治权等。在我们这里,同是全民所有制的企业,但是有的由中央部门直接管,有的由省、市、自治区管,有的由地区管,有的由县管。都是全民所有制,归谁管,归哪级管,只要一个积极性还是要两个积极性,这是个很大的问题,是整个社会主义时期进行社会主义建设过程中要经常注意解决的很关重要的问题。人民公社管的企业,有的具有半全民半集体的性质。中央部门管的和地方各级管的企业,都在统一领导和统一计划下,具有一定的自治权。有没有这种自治权,对促进生产的发展,还是阻碍生产的发展,关系很大。

不能说这本书完全没有马克思主义,因为书中有许多观点是马克思主义的;也不能说完全是马克思主义的,因为书中有许多观点是离开马克思主义的。特别是写法不好,不从生产力和生产关系的矛盾、经济基础和上层建筑的矛盾出发,来研究问题,不从历史的叙述和分析开始自然得出结论,而是从规律出发,进行演绎。

这本教科书,只讲物质前提,很少涉及上层建筑,即:阶级的国家,阶级的哲学,阶级的科学。政治经济学研究的对象主要是生产关系,但是,政治经济学和唯物史观难得分家。不涉及上层建筑方面的问题,经济基础即生产关系的问题不容易说得清楚。

这本书的另一个缺点,是先下定义,不讲道理。定义是分析的结果,不是分析的出发点。研究问题应该从历史的分析开始。但是,搞出了一本社会主义政治经济学,总是一个大功劳,不管里面有多少问题。

这本书的写法很不好,总是从概念入手。研究问题,要从人们看得见、摸得到的现象出发,来研究隐藏在现象后面的本质,从而揭露客观事物的本质的矛盾。《资本论》对资本主义经济的分析,就是用这种方法,总是从现象出发,找出本质,然后又用本质解释现象,因此,能够提纲挚领。教科书对问题不是从分析入手,总是从规律、原则、定义出发,这是马克思主义从来反对的方法。原理、原则是结果,这是要进行分析,经过研究才能得出的。人的认识总是先接触现象,通过现象找出原理、原则来。而教科书与此相反,它所用的方法,不是分析法,而是演绎法。形式逻辑说,人都要死,张三是人,所以张三要死。这里,人都要死是大前提。教科书对每个问题总是先下定义,然后把这个定义作为大前提,来进行演绎,证明他们所要说的道理。他们不懂得,大前提也应当是研究的结果,必须经过具体分析,才能够证明是正确的。

教科书的写法,不是高屋建瓴,势如破竹,没有说服力,没有吸引力,读起来没有兴趣,一看就可以知道是一些只写文章、没有实际经验的书生写的。这本书说的是书生的话,不是革命家的话。他们做实际工作的人没有概括能力,不善于运用概念、逻辑这一套东西;而做理论工作的人又没有实际经验,不懂得经济实践。两种人,两方面——理论和实践没有结合起来。同时作者们没有辩证法。没有哲学家头脑的作家,要写出好的经济学来是不可能的。马克思能够写出《资本论》,列宁能够写出《帝国主义论》,因为他们同时是哲学家,有哲学家的头脑,有辩证法这个武器。

\begin{maonote}
\mnitem{1}一九五八年十一月第一次郑州会议期间,为了使各级领导干部更多地了解马克思主义基本经济理论,以更好地认识与纠正当时出现的一些错误倾向,毛泽东给中央、省区、地、县四级党委委员写信,建议读斯大林《苏联社会主义经济问题》和《马克思、恩格斯、列宁、斯大林论共产主义社会》,同时提出也可以读苏联《政治经济学教科书》。随后在中共八届六中全会上,毛泽东又重申了这一要求。一九五九年庐山会议召开时,毛泽东拟定讨论的十八个问题,第一个问题“读书”,就包括要求各级领导干部读苏联《政治经济学教科书》。一九五九年冬至一九六〇年春,毛泽东、刘少奇、周恩来分别组织了读书小组。毛泽东组织的读书小组有陈伯达、胡绳、邓力群、田家英等参加。这个读书小组从一九五九年十二月十日至一九六〇年二月九日,先后在杭州、上海和广州,采取边读边议的方法,通读苏联科学院经济研究所编的《政治经济学教科书》修订第三版下册(人民出版社一九五九年版),毛泽东发表了许多谈话。
\mnitem{2}见马克思《资本论》第一卷第二版跋。原文是:“观念的东西不外是移入人的头脑并在人的头脑中改造过的物质的东西而已。”(《马克思恩格斯选集》第2卷,人民出版社1995年版,第112页)
\mnitem{3}《教科书》中包含这个引文的那段话是:“随着资本主义的消灭和生产资料的社会主义公有化,人们成为自己社会经济关系的主人。人们认识了客观规律以后,能够完全自觉地掌握和利用这些规律来为整个社会谋福利。”(苏联《政治经济学教科书》修订第三版下册,人民出版社一九五九年版,第446页)
\mnitem{4}见斯大林《苏联社会主义经济问题》。原文是:“在社会主义制度下,通常不会弄到生产关系和生产力发生冲突,社会有可能及时使落后了的生产关系去适合生产力的性质。社会主义社会有可能做到这点,是因为在这个社会中没有那些能够组织反抗的衰朽的阶级。当然,就是在社会主义制度下,也会有落后的惰性的力量,它们不了解生产关系有改变的必要,但是这种力量,当然不难克服,不致把事情弄到冲突的地步。”(《斯大林选集》下卷,人民出版社一九七九年版,第577页)
\mnitem{5}指《教科书》中以下一段话:“然而,同以剥削为基础的社会形态不同,在社会主义制度下,这种矛盾不是对抗性的、不可调和的矛盾。因此,事情不会弄到发生经济危机、阶级斗争和社会革命这类冲突的地步。这些矛盾是发展中的矛盾,是社会在社会主义向共产主义逐步过渡的前进道路上的矛盾。充分认识社会经济发展规律的社会主义国家,依靠群众——共产主义建设者的积极活动,能够及时克服产生的矛盾,因为在社会主义制度下,没有力图保存腐朽的经济关系的阶级,对社会发展进行着有意识的有计划的领导。”(苏联《政治经济学教科书》修订第三版下册,人民出版社一九五九年版,第444页)
\mnitem{6}即《社会民主党在民主革命中的两种策略》。
\mnitem{7}即《帝国主义是资本主义的最高阶段》。中国出版的中译本,曾有将书名译为《帝国主义论》的。
\mnitem{8}百科全书派,是十八世纪法国一部分启蒙思想家在编纂《百科全书》的过程中形成的派别。其核心是以主编狄德罗为首的唯物主义者,他们的基本政治倾向是反对封建特权制度和天主教会,主张一切制度和观念要在理性的审判庭上受到批评和衡量。
\mnitem{9}布尔什维克党,原苏联共产党的前身,一九一二年正式建立。
\mnitem{10}立宪民主党,又称人民自由党,俄国自由资产阶级政党,一九〇五年成立。主张君主立宪,支持沙皇政权。
\mnitem{11}《教科书》中包含这个引文的那两段话是:“中国共产党创造性地运用马克思列宁主义的原则,巧妙地运用了国内现有的通过广泛利用国家资本主义作为过渡措施,并同民族资产阶级结成联盟,来实行和平的社会主义经济改造的客观可能性。”“这个联盟是由中国革命的特点、中国革命的反封建反帝国主义的性质以及关心反对国际帝国主义及其在中国的走狗的统治的斗争的中国民族资产阶级的地位决定的。这个联盟是在地主阶级和买办资产阶级被粉碎的条件下产生的。”(苏联《政治经济学教科书》修订第三版下册,人民出版社一九五九年版,第419——420页)
\mnitem{12}宋庆龄(一八九三——一九八一),广东文昌(今属海南)人,国民党左派政治活动家、孙中山夫人。一九二五年孙中山逝世后,继续坚持联俄、联共、扶助农工的三大政策。一九二七年蒋介石叛变革命后,发表声明严厉谴责蒋介石的叛变行为。何香凝(一八七八——一九七二),广东南海人,国民党左派政治活动家,廖仲恺夫人。一九二四年任国民党中央执行委员兼妇女部长。一九二七年蒋介石叛变革命后,辞去国民党政府的一切职务,同国民党反动派进行了不懈的斗争。
\mnitem{13}一九三一年九月十八日,日本驻在中国东北境内的所谓“关东军”进攻沈阳,中国人民习惯上称日本这次侵略行动为九一八事变。事变发生后,驻沈阳及东北各地的中国军队执行蒋介石的不准抵抗的命令,使日军得以迅速地占领辽宁、吉林、黑龙江三省。
\mnitem{14}杨杏佛(一八九三——一九三三),江西清江(今樟树)人。一九二五年随孙中山北上,任秘书。后任东南大学工学院院长、国民党政府中央研究院总干事等职。九一八事变以后积极参加抗日救亡活动。一九三二年同宋庆龄、蔡元培、鲁迅等在上海发起组织中国民权保障同盟,任总干事,进行反蒋抗日的进步活动。一九三三年六月十八日,被国民党特务暗杀。史量才(一八八〇——一九三四),江苏青浦(今属上海市)人。一九一三年起任《申报》总经理,在蒋介石统治初期采取拥蒋立场,九一八事变以后政治态度逐步改变。上海一二八抗战时,曾捐款支持抗日,并任抗日群众组织上海地方协会会长,后来又积极支持中国民权保障同盟。一九三四年十一月十三日,被国民党特务暗杀。
\mnitem{15}第三国际,即共产国际,一九一九年三月在列宁领导下成立。一九二二年中国共产党参加共产国际,成为它的一个支部。一九四三年五月,共产国际执行委员会主席团通过决定,提议解散共产国际,六月共产国际正式宣布解散。
\mnitem{16}见《共产国际执行委员会第八次全体会议关于中国问题决议案》(一九二七年五月)。原文是:“中国革命的危机与社会阶级力量目前的组合说明并证实,反对封建制度的资产阶级民主革命(农民革命属之,只有这样,才可说到反帝国主义的斗争)是只有在反对已经反革命的资产阶级的斗争中,才可完成。彻底的民族自由斗争不仅与发展工农群众运动或土地革命的要求毫无冲突,而且要直接以下层广大民众奋起推翻帝国主义之革命运动的扩大为前提。”
\mnitem{17}立三路线,指第二次国内革命战争时期以李立三为代表的“左”倾冒险主义错误。一九三〇年六月十一日,中共中央政治局在李立三主持下通过了《新的革命高潮与一省或几省首先胜利》的决议,形成了以他为代表的“左”倾冒险主义错误。不久,李立三又制订了组织以武汉为中心的全国中心城市武装起义和集中全国红军进攻中心城市的冒险计划,随后又将党、青年团、工会的各级领导机关,合并为准备武装起义的各级行动委员会,使一切经常工作陷于停顿。同年九月,中国共产党召开六届三中全会,纠正了李立三的“左”倾冒险主义错误。
\mnitem{18}《教科书》中包含这个引文的那段话是:“在现代条件下,在强大的社会主义阵营已经形成,资本主义总危机进一步深刻化,殖民体系进一步瓦解,社会主义、民主和和平的思想对于全体劳动人类具有极大的吸引力的情况下,在某些资本主义国家和过去的殖民地国家中,工人阶级通过议会和平地取得政权是有现实的可能性的。在这些国家中,工人阶级如果把绝大多数的人民——劳动农民、小资产阶级、知识分子的广大阶层、国内的一切爱国力量——在自己的领导下团结起来,并且给机会主义分子以坚决的回击,就可能击败反动的反人民的势力,在议会中取得稳定的多数,把议会从资产阶级政权的机关变成人民的、工人的政权的机关,变成劳动者的民主的工具。这样依靠着无产阶级和劳动群众的革命运动的真正人民的议会,将能够顺利地解决社会主义革命的根本任务,其中包括把基本生产资料变成人民的财产的任务。”(苏联《政治经济学教科书》修订第三版下册,人民出版社一九五九年版,第330页)
\mnitem{19}一九一七年俄国二月革命后,出现了资产阶级临时政府和工兵代表苏维埃两个政权并存的局面。临时政府奉行反革命、反人民的政策。七月十六日,革命群众举行示威游行,要求工兵代表苏维埃正式夺取政权。布尔什维克党认为武装夺取政权的时机还不成熟,经极力劝阻无效,决定参加示威。七月十七日,彼得格勒五十万工人、士兵和水兵举行示威游行,要求全部政权归苏维埃。临时政府出动军队,屠杀示威群众,死伤约四百多人,同时逮捕和杀害布尔什维克党人,封闭《真理报》,解除赤卫队武装。这一事件称七月事件。此后,两个政权并存的局面结束。
\mnitem{20}“三反”,指一九五一年十二月至一九五二年十月在国家机关、部队和国营企业等单位中开展的反对贪污、反对浪费、反对官僚主义的斗争。“五反”,指一九五二年在全国资本主义工商业中开展的反对行贿、反对偷税漏税、反对盗骗国家财产、反对偷工减料、反对盗窃经济情报的斗争。
\mnitem{21}参见列宁《第三国际及其在历史上的地位》(《列宁选集》第3卷,人民出版社1995年版,第789—797页)。
\mnitem{22}指《教科书》中以下一段话:“在某些国家中,随着推翻剥削阶级的政治统治,革命立刻具有社会主义的性质。1917年10月俄国的情况就是如此,当时在无产阶级革命的过程中同时彻底解决了资产阶级民主革命的任务。在另外一些脱离了资本主义体系的国家中,最初主要是解决一般民主主义的任务,同时,在许多情况下,革命在最初阶段主要是资产阶级民主性质的,只是后来才逐渐地发展成为社会主义革命。这要由每个国家资本主义发展的水平、有没有资本主义前的形式存在、阶级力量的对比和政治情况等等来决定。”(苏联《政治经济学教科书》修订第三版下册,人民出版社一九五九年版,第330——331页)
\mnitem{23}《共同纲领》,即《中国人民政治协商会议共同纲领》,一九四九年九月二十九日由中国人民政治协商会议第一届全体会议通过。一九五四年《中华人民共和国宪法》颁布以前,它起了临时宪法的作用。
\mnitem{24}定息,是我国在资本主义工商业实行全行业公私合营后,对民族资本家的生产资料进行赎买的一种形式,即不论企业盈亏,统一由国家每年按照合营时清产核资确定的私股股额,发给资本家固定的利息(一般是年息百分之五)。从一九五六年开始支付定息。一九六六年九月停止支付。
\mnitem{25}指《教科书》中以下一段话:“社会主义革命要在俄国取得胜利,至少要有中等的资本主义发展水平,这是因为俄国是无产阶级专政的唯一的国家。现在,当社会主义已经在苏联取得了胜利,世界社会主义经济体系已经形成的时候,在资本主义不发达的、资本主义前经济形式占优势的国家里,由于先进的社会主义国家的帮助,也可以胜利地解决社会主义革命的任务。”(苏联《政治经济学教科书》修订第三版下册,人民出版社一九五九年版,第331页)
\mnitem{26}参见列宁《论粮食税》(《列宁选集》第4卷,人民出版社1995年版,第488—525页)。
\mnitem{27}《教科书》中包含这个引文的那段话是:“同时必须注意到,各国的社会主义革命虽然在主要方面和基本方面是一致的,但是它在每一个脱离了帝国主义体系的国家中必然具有自己特别的具体的社会主义建设的形式和方法,这些形式和方法是由每一个国家发展的历史、民族、经济、政治和文化条件,人民的传统,以及某一个时期的国际环境产生的。”(苏联《政治经济学教科书》修订第三版下册,人民出版社一九五九年版,第329页)
\mnitem{28}《莫斯科宣言》,指一九五七年十一月十四日至十六日在莫斯科召开的社会主义国家共产党和工人党代表会议通过的《社会主义国家共产党和工人党代表会议宣言》。在这个《宣言》中,提出了普遍适用于各个走上社会主义道路的国家的九条共同的规律,即:以马克思列宁主义政党为核心的工人阶级,领导劳动群众进行这种形式或那种形式的无产阶级革命,建立这种形式或那种形式的无产阶级专政;建立无产阶级同农民基本群众和其他劳动阶层的联盟;消灭基本生产资料资本主义所有制和建立基本生产资料的公有制;逐步实现农业的社会主义改造;有计划地发展国民经济,以便建成社会主义和共产主义,提高劳动人民的生活水平;进行思想文化领域的社会主义革命,造成忠于工人阶级、劳动人民和社会主义事业的强大的知识分子队伍;消灭民族压迫,建立各民族间的平等和兄弟友谊;保卫社会主义果实,不让它受到国内外敌人的侵犯;实行无产阶级的国际主义,同各国工人阶级团结一致。
\mnitem{29}过渡时期的总路线是中共中央按照毛泽东的建议于一九五二年提出的。一九五三年十二月中共中央批准的《关于党在过渡时期总路线的学习和宣传提纲》对总路线作了以下的表述:“从中华人民共和国成立,到社会主义改造基本完成,这是一个过渡时期。党在这个过渡时期的总路线和总任务,是要在一个相当长的时期内,逐步实现国家的社会主义工业化,并逐步实现国家对农业、对手工业和对资本主义工商业的社会主义改造。这条总路线是照耀我们各项工作的灯塔,各项工作离开它,就要犯右倾或‘左’倾的错误。”
\mnitem{30}第一个五年计划,是一九五三年至一九五七年中华人民共和国发展国民经济的第一个五年计划的简称。这个计划的编制,从一九五一年开始进行,历时四年,五易其稿,于一九五五年七月由第一届全国人民代表大会第二次会议正式通过。执行的结果是,到一九五六年,基本上完成了对农业、手工业和资本主义工商业的社会主义改造,建立了社会主义的基本经济制度;到一九五七年,计划规定的各项建设任务也胜利实现,许多指标超额完成,这就为国家工业化打下了初步基础,同时人民生活也得到了很大改善。
\mnitem{31}指一九五八年五月五日至二十三日在北京召开的中国共产党第八次全国代表大会第二次会议通过的“鼓足干劲、力争上游、多快好省地建设社会主义”的总路线。
\mnitem{32}《论十大关系》,见毛选第五卷。
\mnitem{33}指斯大林《在莫斯科市斯大林选区选举前的选民大会上的演说》(《斯大林选集》下卷,人民出版社一九七九年版,第488页)。
\mnitem{34}指《一九五六年到一九六七年全国农业发展纲要(草案)》。这个草案是中共中央提出的,一九五六年一月公布。一九五七年十月公布修正草案。后又经修改,于一九六〇年四月第二届全国人民代表大会第二次会议通过后作为正式文件颁发。纲要全文共四十条,提出了我国农业、畜牧业、林业、渔业、副业以及农村商业、信贷、交通、邮电、广播、科学、文化、教育、卫生等方面的发展规划。
\mnitem{35}见恩格斯《社会主义从空想到科学的发展》。原文是:“无产阶级革命,矛盾的解决:无产阶级将取得公共权力,并且利用这个权力把脱离资产阶级掌握的社会生产资料变为公共财产。通过这个行动,无产阶级使生产资料摆脱了它们迄今具有的资本属性,使它们的社会性有充分的自由得以实现。从此按照预定计划进行的社会生产就成为可能的了。”(《马克思恩格斯选集》第3卷,人民出版社1995年版,第759页)
\mnitem{36}《教科书》中包含这个引文的那段话是:“由于生产资料的公有化,社会主义国民经济各部门间生产资料和劳动力分配的必要比例是有计划地实现的。自发性和自流性是同生产资料公有制的存在不相容的。”(苏联《政治经济学教科书》修订第三版下册,人民出版社一九五九年版,第465页)
\mnitem{37}指《教科书》中以下一段话:“社会主义计划化建立在严格的科学基础上,它要求经常总结共产主义建设的实践,利用科学技术的一切成就。用计划来指导国民经济,这就是预见。科学的预见基于对客观的经济规律的认识,并根据社会物质生活发展的业已成熟的需要。”(苏联《政治经济学教科书》修订第三版下册,人民出版社一九五九年版,第472页)
\mnitem{38}见《孟子·膝文公上》。原文是:“夫物之不齐,物之情也。”
\mnitem{39}指《教科书》中以下一段话:“在资本主义制度下,技术的发展极不平衡,必然加深生产中的比例失调现象,与此相反,社会主义计划经济保证根据国民经济的需要有计划地发展科学和技术。”(苏联《政治经济学教科书》修订第三版下册,人民出版社一九五九年版,第482页)
\mnitem{40}指《教科书》中以下一段话:“生产资料生产的优先增长意味着工业的发展快于农业。在社会主义制度下,工业和农业间的比例,要在更迅速地发展工业的基础上,保证农业生产的不断增长。”(苏联《政治经济学教科书》修订第三版下册,人民出版社一九五九年版,第623页)
\mnitem{41}《教科书》中包含这个引文的那段话是:“列宁的社会主义建设计划是以尽力发展国营工业和农民经济之间的经济联系为前提的。从小农经济的性质看来,农民迫切需要的同城市进行经济联系的形式是通过买卖的交换。在过渡时期,国营工业和小农经济之间的商业结合是经济的必然性。”(苏联《政治经济学教科书》修订第三版下册,人民出版社一九五九年版,第346页)
\mnitem{42}指《教科书》中以下一段话:“国家的工业化是在生产资料的生产比消费品的生产优先增长的规律作用的基础上实现的。生产资料生产的发展为制造消费品部门的发展、为提高人民的福利创造前提。”(苏联《政治经济学教科书》修订第三版下册,人民出版社一九五九年版,第365页)
\mnitem{43}《关于农业合作化问题》,见本书第六卷第418—442页。
\mnitem{44}指《教科书》中以下一句话:“为了防止和消除国民经济中的个别比例失调现象,社会主义国家建立了必要的后备。”(苏联《政治经济学教科书》修订第三版下册,人民出版社一九五九年版,第627页)
\mnitem{45}《教科书》中包含这个引文的那段话是:“同时,社会主义的分工可以使各个国家作为世界社会主义体系的平等成员,彼此取长补短,因而有可能节约财力和人力,消除国民经济中个别部门的不必要的平行发展,以加快各国经济发展的速度。每个国家都可以集中自己的人力财力来发展在本国有最有利的自然条件和经济条件、有生产经验和干部的部门。而且个别国家可以不必生产能靠其他国家供应来满足需要的产品。这样就可以在工业中达到合理的生产专业化和协作化,在粮食和原料生产上达到最适当的分工。”(苏联《政治经济学教科书》修订第三版下册,人民出版社一九五九年版,第659页)
\mnitem{46}见列宁《全俄中央执行委员会会议文献》中的《答左派社会革命党人的质问》。新的译文是:“生气勃勃的创造性的社会主义是由人民群众自己创立的。”(《列宁全集》第33卷,人民出版社1985年版,第53页)
\mnitem{47}指《教科书》中以下一段话:“由于剥削的消灭,脑力劳动者和体力劳动者之间的关系根本改变了。资本主义所特有的工人和企业领导人员间的利益对立消失了。在社会主义制度下,体力劳动者和企业领导人员是统一的生产集体的成员,他们都极其关心生产的发展和改进。由此就产生出体力劳动者和脑力劳动者旨在不断改进生产的创造性的合作。”(苏联《政治经济学教科书》修订第三版下册,人民出版社一九五九年版,第500页)
\mnitem{48}资产阶级工业革命,即产业革命。指十七至十八世纪英国资本主义从以手工技术为基础的手工业工场过渡到采用机器的工厂制度的过程。
\mnitem{49}见马克思《资本论》。原文是:“通过协作提高了个人生产力,而且是创造了一种生产力,这种生产力本身必然是集体力。”(《马克思恩格斯全集》第23卷,人民出版社一九七二年版,第362页)
\mnitem{50}指《教科书》中以下一段话:“批评和自我批评是社会主义制度下新旧斗争的基本形式之一,是社会主义社会发展的强大动力。批评和自我批评可以在发挥人民群众的积极性的基础上,揭露和消灭工作中的缺点和困难,铲除一切官僚主义现象,发现加速经济发展的新的潜力,从而克服社会主义社会的矛盾。”(苏联《政治经济学教科书》修订第三版下册,人民出版社一九五九年版,第453页)
\mnitem{51}指《教科书》中以下一段话:“随着社会主义的胜利和人剥削人的现象的消灭,苏联没有对抗性的阶级了,没有不可调和的阶级矛盾了。社会主义社会的阶级关系是以工人阶级、农民、知识分子的牢不可破的友谊和同志式的合作为特征的。工人阶级和农民间的阶级差别,同这两个阶级和知识分子之间的差别一样,日渐消失。可是资本主义社会却被阶级对抗和民族对抗弄得四分五裂,很不稳定;而社会主义社会根本就没有阶级对抗和民族对抗,它的特点是团结一致,十分稳定。公有制和社会主义的经济体系在苏联的统治地位,是社会主义社会的精神上政治上的一致、各族人民的友谊、苏维埃爱国主义这些强大的社会发展动力赖以发挥的经济基础。而这些动力又反过来给予经济很大的影响,加速经济的发展。”(苏联《政治经济学教科书》修订第三版下册,人民出版社一九五九年版,第413页)
\mnitem{52}《教科书》中包含这个引文的那段话是:“在社会主义阶段,使工作者从个人的物质利益上关心劳动结果是刺激生产发展的决定因素之一。保证这种关心的,是工作者的报酬以他的劳动数量和质量为转移。利用每个工作者从物质利益上对劳动结果的关心是社会主义经营的根本方法之一。”(苏联《政治经济学教科书》修订第三版下册,人民出版社一九五九年版,第487页)
\mnitem{53}经济主义,是十九世纪末二十世纪初俄国社会民主工党内的一种机会主义思潮。主要代表人物有普罗柯波维奇、库斯柯娃。经济主义反对工人阶级提出自己的政治要求,认为工人阶级只应当以罢工为手段,去进行争取改善经济状况和切身利益的斗争,否认无产阶级政党及其领导作用。工团主义,是国际工人运动中的一种小资产阶级机会主义思潮。二十世纪初在法国、意大利、西班牙、瑞士等地流传很广。主要代表人物有法国索烈尔、拉加德尔。工团主义反对政治斗争,否定无产阶级政党的领导作用,否定无产阶级革命和无产阶级专政,宣传工会高于一切和管理一切,幻想以工会代替国家机构。
\mnitem{54}《教科书》中这句话的原文是:“社会主义企业生产者之间的社会关系,是同志式的合作和社会主义的互助关系。”(苏联《政治经济学教科书》修订第三版下册,人民出版社一九五九年版,第442页)
\end{maonote}
