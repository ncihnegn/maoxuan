
\title{论联合政府}
\date{一九四五年四月二十四日}
\thanks{这是毛泽东在中国共产党第七次全国代表大会上的政治报告。}
\maketitle


\section{一 中国人民的基本要求}

我们的大会是在这种情况之下开会的:中国人民在其对于日本侵略者作了将近八年的坚决的英勇的不屈不挠的奋斗,经历了无数的艰难困苦和自我牺牲之后,出现了这样的新局面——整个世界上反对法西斯侵略者的神圣的正义的战争,已经取得了有决定意义的胜利,中国人民配合同盟国打败日本侵略者的时机,已经迫近了。但是中国现在仍然不团结,中国仍然存在着严重的危机。在这种情况下,我们应该怎样做呢?毫无疑义,中国急需把各党各派和无党无派的代表人物团结在一起,成立民主的临时的联合政府,以便实行民主的改革,克服目前的危机,动员和统一全中国的抗日力量,有力地和同盟国配合作战,打败日本侵略者,使中国人民从日本侵略者手中解放出来。然后,需要在广泛的民主基础之上,召开国民代表大会,成立包括更广大范围的各党各派和无党无派代表人物在内的同样是联合性质的民主的正式的政府,领导解放后的全国人民,将中国建设成为一个独立、自由、民主、统一和富强的新国家。一句话,走团结和民主的路线,打败侵略者,建设新中国。

我们认为只有这样做,才是反映了中国人民的基本要求。因此,我的报告,主要地就是讨论这些要求。中国应否成立民主的联合政府,已成了中国人民和同盟国民主舆论界十分关心的问题。因此,我的报告将着重地说明这个问题。

中国共产党在八年抗日战争中的工作,已经克服了很多的困难,获得了巨大的成绩;但是在目前形势下,在我们党和人民面前,还有严重的困难。目前的时局,要求我们党进一步地从事紧急的和更加切实的工作,继续地克服困难,为完成中国人民的基本要求而奋斗。

\section{二 国际形势和国内形势}

中国人民能不能实现我们在上面提出的那些基本要求呢?这要依靠中国人民的觉悟、团结和努力的程度来决定。但是目前的国际国内形势,对中国人民提供了极其有利的条件。中国人民如果能很好地利用这些条件,积极地坚决地再接再厉地向前奋斗,战胜侵略者和建设新中国,是毫无疑义的。中国人民应当加倍努力,为完成自己的神圣任务而奋斗。

目前的国际形势是怎样的呢?

目前的军事形势是苏军已经攻击柏林,英美法联军也正在配合打击希特勒残军,意大利人民又已经发动了起义。这一切,将最后地消灭希特勒。希特勒被消灭以后,打败日本侵略者就为时不远了。和中外反动派的预料相反,法西斯侵略势力是一定要被打倒的,人民民主势力是一定要胜利的。世界将走向进步,决不是走向反动。当然应该提起充分的警觉,估计到历史的若干暂时的甚至是严重的曲折,可能还会发生;许多国家中不愿看见本国人民和外国人民获得团结、进步和解放的反动势力,还是强大的。谁要是忽视了这些,谁就将在政治上犯错误。但是,历史的总趋向已经确定,不能改变了。这种情况,仅仅不利于法西斯和实际上帮助法西斯的各国反动派,而对于一切国家的人民及其有组织的民主势力,则都是福音。人民,只有人民,才是创造世界历史的动力。苏联人民创造了强大力量,充当了打倒法西斯的主力军。苏联人民加上其它反法西斯同盟国的人民的努力,使打倒法西斯成为可能。战争教育了人民,人民将赢得战争,赢得和平,又赢得进步。

这一新形势,和第一次世界大战时代的形势大不相同。在那时,还没有苏联,也没有现在许多国家的人民的觉悟程度。两次世界大战是两个完全不同的时代。

法西斯侵略国家被打败、第二次世界大战结束、国际和平实现以后,并不是说就没有了斗争。广泛地散布着的法西斯残余势力,一定还要捣乱;反法西斯侵略战争的阵营中存在着反民主的和压迫其它民族的势力,他们仍然要压迫各国人民和各殖民地半殖民地。所以,国际和平实现以后,反法西斯的人民大众和法西斯残余势力之争,民主和反民主之争,民族解放和民族压迫之争仍将充满世界的大部分地方。只有经过长期的努力,克服了法西斯残余势力、反民主势力和一切帝国主义势力,才能有最广泛的人民的胜利。到达这一天,决不是很快和很容易的,但是必然要到达这一天。反法西斯的第二次世界大战的胜利,给这个战后人民斗争的胜利开辟了道路。也只有这后一种斗争胜利了,巩固的和持久的和平才有保障。

目前的国内形势是怎样的呢?

中国的长期战争,使中国人民付出了并且还将再付出重大的牺牲;但是同时,正是这个战争,锻炼了中国人民。这个战争促进中国人民的觉悟和团结的程度,是近百年来中国人民的一切伟大的斗争没有一次比得上的。在中国人民面前,不但存在着强大的民族敌人,而且存在着强大的实际上帮助民族敌人的国内反动势力,这是一方面。但是另一方面,中国人民不但已经有了比过去任何时候都高的觉悟程度,而且有了强大的中国解放区和日益高涨着的全国性的民主运动。这是国内的有利条件。如果说,中国近百年来一切人民斗争都遭到了失败或挫折,而这是因为缺乏国际的和国内的若干必要的条件,那末,这一次就不同了,比较以往历次,一切必要的条件都具备了。避免失败和取得胜利的可能性充分地存在着。如果我们能够团结全国人民,努力奋斗,并给以适当的指导,我们就能够胜利。

中国人民团结起来打败侵略者和建设新中国的信心,现在是极大地增强了。中国人民克服一切困难,实现其具有伟大历史意义的基本要求的时机,已经到来了。这一点还有疑义吗?我以为没有疑义了。

这些,就是目前国际和国内的一般形势。

\section{三 抗日战争中的两条路线}

\subsection{中国问题的关键}

谈到国内形势,我们还应对中国抗日战争加以具体的分析。

中国是全世界参加反法西斯战争的五个最大的国家之一,是在亚洲大陆上反对日本侵略者的主要国家。中国人民不但在抗日战争中起了极大的作用,而且在保障战后世界和平上将起极大的作用,在保障东方和平上则将起决定的作用。中国在八年抗日战争中,为了自己的解放,为了帮助各同盟国,曾经作了伟大的努力。这种努力,主要地是属于中国人民方面的。中国军队的广大官兵,在前线流血战斗,中国的工人、农民、知识界、产业界,在后方努力工作,海外华侨输财助战,一切抗日政党,除了那些反人民分子外,都对战争有所尽力。总之,中国人民以自己的血和汗同日本侵略者英勇地奋战了八年之久。但是多年以来,中国反动分子造作谣言,蒙蔽舆论,不使中国人民在抗日战争中所起作用的真相为世人所知。同时,对于中国八年抗日战争的各项经验,也还没有人作出全面的总结来。因此,我们的大会,应当对这些经验作出适当的总结,借以教育人民,并为我党决定政策的根据。

提到总结经验,那末,大家可以很清楚地看到,在中国有两条不同的指导路线,一条是能够打败日本侵略者的,一条是不但不能打败日本侵略者,而且在某些方面说来它是在实际上帮助日本侵略者危害抗日战争的。

国民党政府所采取的对日消极作战的政策和对内积极摧残人民的反动政策,招致了战争的挫折,大部国土的沦陷,财政经济的危机,人民的被压迫,人民生活的痛苦,民族团结的破坏。这种反动政策妨碍了动员和统一一切中国人民的抗日力量进行有效的战争,妨碍了中国人民的觉醒和团结。但是,中国人民的觉醒和团结的运动并没有停止,它是在日本侵略者和国民党政府的双重压迫之下曲折地发展着。两条路线:国民党政府压迫中国人民实行消极抗战的路线和中国人民觉醒起来团结起来实行人民战争的路线,很久以来,就明显地在中国存在着。这就是一切中国问题的关键所在。

\subsection{走着曲折道路的历史}

为了使大家明了何以这个两条路线问题是一切中国问题的关键所在,必须回溯一下我们抗日战争的历史。

中国人民的抗日战争,是在曲折的道路上发展起来的。这个战争,还是在一九三一年就开始了。一九三一年九月十八日,日本侵略者占领沈阳\mnote{1},几个月内,就把东三省占领了。国民党政府采取了不抵抗政策。但是东三省的人民,东三省的一部分爱国军队,在中国共产党领导或协助之下,违反国民党政府的意志,组织了东三省的抗日义勇军和抗日联军,从事英勇的游击战争。这个英勇的游击战争,曾经发展到很大的规模,中间经过许多困难挫折,始终没有被敌人消灭。一九三二年,日本侵略者进攻上海,国民党内的一派爱国分子,又一次违反国民党政府的意志,率领十九路军,抵抗了日本侵略者的进攻。一九三三年,日本侵略者进攻热河、察哈尔\mnote{2},国民党内的又一派爱国分子,第三次违反国民党政府的意志,并和共产党合作,组织了抗日同盟军,从事抵抗。但是一切这些抗日战争,除了中国人民、中国共产党、其它民主派别和海外爱国华侨给了援助之外,国民党政府根据其不抵抗政策,是没有给任何援助的。相反地,上海、察哈尔两次抗日行动,都被国民党政府一手破坏了。一九三三年,十九路军在福建成立的人民政府,也被国民党政府破坏了。

那时的国民党政府为什么采取不抵抗政策呢?主要的原因,在于国民党在一九二七年破坏了国共两党的合作,破坏了中国人民的团结。

一九二四年,孙中山先生接受了中国共产党的建议,召集了有共产党人参加的国民党第一次全国代表大会,订出了联俄、联共、扶助农工的三大政策\mnote{3},建立了黄埔军校\mnote{4},实现了国共两党和各界人民的民族统一战线,因而在一九二四年至一九二五年,扫荡了广东的反动势力\mnote{5},在一九二六年至一九二七年,举行了胜利的北伐战争,占领了长江流域和黄河流域的大部,打败了北洋军阀政府,发动了中国历史上空前广大的人民解放斗争。但是到了一九二七年春夏之交,正当北伐战争向前发展的紧要关头,这个代表中国人民解放事业的国共两党和各界人民的民族统一战线及其一切革命政策,就被国民党当局的叛卖性的反人民的“清党”政策和屠杀政策所破坏了。昨天的同盟者——中国共产党和中国人民,被看成了仇敌,昨天的敌人——帝国主义者和封建主义者,被看成了同盟者。就是这样,背信弃义地向着中国共产党和中国人民来一个突然的袭击;生气蓬勃的中国大革命就被葬送了。从此以后,内战代替了团结,独裁代替了民主,黑暗的中国代替了光明的中国。但是,中国共产党和中国人民并没有被吓倒,被征服,被杀绝。他们从地下爬起来,揩干净身上的血迹,掩埋好同伴的尸首,他们又继续战斗了。他们高举起革命的大旗,举行了武装的抵抗,在中国的广大区域内,组织了人民的政府,实行了土地制度的改革,创造了人民的军队——中国红军,保存了和发展了中国人民的革命力量。被国民党反动分子所抛弃的孙中山先生的革命的三民主义,由中国人民、中国共产党和其它民主分子继承下来了。

到了日本侵略者打入东三省以后,中国共产党就在一九三三年,向一切进攻革命根据地和红军的国民党军队提议:在停止进攻、给予人民以自由权利和武装人民这样三个条件之下,订立停战协定,以便一致抗日。但是国民党当局拒绝了这个提议。

从此以后,一方面,是国民党政府的内战政策越发猖狂;另一方面,是中国人民要求停止内战一致抗日的呼声越发高涨。各种人民爱国组织,在上海和其它许多地方建立起来。一九三四年至一九三六年,长江南北各地的红军主力,在我们党中央领导之下,经历了千辛万苦,移到了西北,并和西北红军汇合在一起。就在这两年,中国共产党适应新的情况,决定并执行了抗日民族统一战线的新的完整的政治路线,以团结抗日和建立新民主主义共和国为奋斗目标。一九三五年十二月九日,北平学生群众,在我们党领导之下,发动了英勇的爱国运动,成立了中华民族解放先锋队\mnote{6},并把这种爱国运动推广到了全国各大城市。一九三六年十二月十二日,国民党内部主张抗日的两派爱国分子——东北军和十七路军,联合起来,勇敢地反对国民党当局的对日妥协和对内屠杀的反动政策,举行了有名的西安事变。同时,国民党内的其它爱国分子,也不满意国民党当局的当时政策。在这种形势下,国民党当局被迫地放弃了内战政策,承认了人民的要求。西安事变的和平解决成了时局转换的枢纽:在新形势下的国内的合作形成了,全国的抗日战争发动了。在卢沟桥事变的前夜,即一九三七年五月,我们党召集了一个具有历史意义的全国代表会议,这个会议批准了党中央自一九三五年以来的新的政治路线。

从一九三七年七月七日卢沟桥事变到一九三八年十月武汉失守这一个时期内,国民党政府的对日作战是比较努力的。在这个时期内,日本侵略者的大举进攻和全国人民民族义愤的高涨,使得国民党政府政策的重点还放在反对日本侵略者身上,这样就比较顺利地形成了全国军民抗日战争的高潮,一时出现了生气蓬勃的新气象。当时全国人民,我们共产党人,其它民主党派,都对国民党政府寄予极大的希望,就是说,希望它乘此民族艰危、人心振奋的时机,厉行民主改革,将孙中山先生的革命三民主义付诸实施。可是,这个希望是落空了。就在这两年,一方面,有比较积极的抗战;另一方面,国民党当局仍旧反对发动广大民众参加的人民战争,仍旧限制人民自动团结起来进行抗日和民主的活动。一方面,国民党政府对待中国共产党及其它抗日党派的态度比较过去有了一些改变;另一方面,仍旧不给各党派以平等地位,并多方限制它们的活动。许多爱国政治犯并没有释放。最主要的是国民党政府仍旧保持其自一九二七年发动内战以来的寡头专政制度,未能建立举国一致的民主的联合政府。

还在这一时期的开始,我们共产党人就指出中国抗日战争的两条路线:或者是人民的全面的战争,这样就会胜利;或者是压迫人民的片面的战争,这样就会失败。我们又指出:战争将是长期的,必然要遇到许多艰难困苦;但是由于中国人民的努力,最后胜利必归于中国人民。

\subsection{人民战争}

这一时期内,中国共产党领导的移到了西北的中国红军主力,改编为中国国民革命军第八路军,留在长江南北各地的中国红军游击部队,则改编为中国国民革命军新编第四军,相继开赴华北华中作战。内战时期的中国红军,保存了并发展了北伐时期黄埔军校和国民革命军的民主传统,曾经扩大到几十万人。由于国民党政府在南方各根据地内的残酷的摧毁、万里长征的消耗和其它原因,到抗日战争开始时,数量减少到只剩几万人。于是有些人就看不起这支军队,以为抗日主要地应当依靠国民党。但是人民是最好的鉴定人,他们知道八路军新四军这时数量虽小,质量却很高,只有它才能进行真正的人民战争,它一旦开到抗日的前线,和那里的广大人民相结合,其前途是无限的。人民是正确的,当我在这里做报告的时候,我们的军队已发展到了九十一万人,乡村中不脱离生产的民兵发展到了二百二十万人以上。不管现在我们的正式军队比起国民党现存的军队来(包括中央系和地方系)在数量上要少得多,但是按其所抗击的日军和伪军的数量及其所担负的战场的广大说来,按其战斗力说来,按其有广大的人民配合作战说来,按其政治质量及其内部统一团结等项情况说来,它已经成了中国抗日战争的主力军。

这个军队之所以有力量,是因为所有参加这个军队的人,都具有自觉的纪律;他们不是为着少数人的或狭隘集团的私利,而是为着广大人民群众的利益,为着全民族的利益,而结合,而战斗的。紧紧地和中国人民站在一起,全心全意地为中国人民服务,就是这个军队的唯一的宗旨。

在这个宗旨下面,这个军队具有一往无前的精神,它要压倒一切敌人,而决不被敌人所屈服。不论在任何艰难困苦的场合,只要还有一个人,这个人就要继续战斗下去。

在这个宗旨下面,这个军队有一个很好的内部和外部的团结。在内部——官兵之间,上下级之间,军事工作、政治工作和后勤工作之间;在外部——军民之间,军政之间,我友之间,都是团结一致的。一切妨害团结的现象,都在必须克服之列。

在这个宗旨下面,这个军队有一个正确的争取敌军官兵和处理俘虏的政策。对于敌方投诚的、反正的、或在放下武器后愿意参加反对共同敌人的人,一概表示欢迎,并给予适当的教育。对于一切俘虏,不许杀害、虐待和侮辱。

在这个宗旨下面,这个军队形成了为人民战争所必需的一系列的战略战术。它善于按照变化着的具体条件从事机动灵活的游击战争,也善于作运动战。

在这个宗旨下面,这个军队形成了为人民战争所必需的一系列的政治工作,其任务是为团结我军,团结友军,团结人民,瓦解敌军和保证战斗胜利而斗争。

在这个宗旨下面,在游击战争的条件下,全军都可以并且已经是这样做了:利用战斗和训练的间隙,从事粮食和日用必需品的生产,达到军队自给、半自给或部分自给之目的,借以克服经济困难,改善军队生活和减轻人民负担。在各个军事根据地上,也利用了一切可能性,建立了许多小规模的军事工业。

这个军队之所以有力量,还由于有人民自卫军和民兵这样广大的群众武装组织,和它一道配合作战。在中国解放区内,一切青年、壮年的男人和女人,都在自愿的民主的和不脱离生产的原则下,组织在抗日人民自卫军之中。自卫军中的精干分子,除加入军队和游击队者外,则组织在民兵的队伍中。没有这些群众武装力量的配合,要战胜敌人是不可能的。

这个军队之所以有力量,还由于它将自己划分为主力兵团和地方兵团两部分,前者可以随时执行超地方的作战任务,后者的任务则固定在协同民兵、自卫军保卫地方和进攻当地敌人方面。这种划分,取得了人民的真心拥护。如果没有这种正确的划分,例如说,如果只注意主力兵团的作用,忽视地方兵团的作用,那末,在中国解放区的条件下,要战胜敌人也是不可能的。在地方兵团方面,组织了许多经过良好训练,在军事、政治、民运各项工作上说来都是比较地更健全的武装工作队,深入敌后之敌后,打击敌人,发动民众的抗日斗争,借以配合各个解放区正面战线的作战,收到了很大的成效。

在中国解放区,在民主政府领导之下,号召一切抗日人民组织在工人的、农民的、青年的、妇女的、文化的和其它职业和工作的团体之中,热烈地从事援助军队的各项工作。这些工作不但包括动员人民参加军队,替军队运输粮食,优待抗日军人家属,帮助军队解决物质困难,而且包括动员游击队、民兵和自卫军,展开袭击运动和爆炸运动,侦察敌情,清除奸细,运送伤兵和保护伤兵,直接帮助军队的作战。同时,全解放区人民又热烈地从事政治、经济、文化、卫生各项建设工作。在这方面,最重要的是动员全体人民从事粮食和日用品的生产,并使一切机关、学校,除有特殊情形者外,一律于工作或学习之暇,从事生产自给,以配合人民和军队的生产自给,造成伟大的生产热潮,借以支持长期的抗日战争。在中国解放区,敌人的摧残是异常严重的;水、旱、虫灾,也时常发生。但是,解放区民主政府领导全体人民,有组织地克服了和正在克服着各种困难,灭蝗、治水、救灾的伟大群众运动,收到了史无前例的效果,使抗日战争能够长期地坚持下去。总之,一切为着前线,一切为着打倒日本侵略者和解放中国人民,这就是中国解放区全体军民的总口号、总方针。

这就是真正的人民战争。只有这种人民战争,才能战胜民族敌人。国民党之所以失败,就是因为它拚命地反对人民战争。

中国解放区的军队一旦得到新式武器的装备,它就会更加强大,就能够最后地打败日本侵略者了。

\subsection{两个战场}

中国的抗日战争,一开始就分为两个战场:国民党战场和解放区战场。

一九三八年十月武汉失守后,日本侵略者停止了向国民党战场的战略性的进攻,逐渐地将其主要军事力量移到了解放区战场;同时,针对着国民党政府的失败情绪,声言愿意和它谋取妥协的和平,并将卖国贼汪精卫诱出重庆,在南京成立伪政府,实施民族的欺骗政策。从这时起,国民党政府开始了它的政策上的变化,将其重点由抗日逐渐转移到反共反人民。这首先表现在军事方面。它采取了对日消极作战的政策,保存军事实力,而把作战的重担放在解放区战场上,让日寇大举进攻解放区,它自己则“坐山观虎斗”。

一九三九年,国民党政府采取了反动的所谓《限制异党活动办法》\mnote{7},将抗战初期人民和各抗日党派争得的某些权利,一概取消。从此时起,在国民党统治区内,国民党政府将一切民主党派,首先和主要地是将中国共产党,打入地下。在国民党统治区各个省份的监狱和集中营内,充满了共产党人、爱国青年及其它民主战士。从一九三九年起直至一九四三年秋季为止的五年之内,国民党政府发动了三次大规模的“反共高潮”\mnote{8},分裂国内的团结,造成严重的内战危险。震动中外的“解散”新四军和歼灭皖南新四军部队九千余人的事变,就是发生在这个时期内。直到现时为止,国民党军队向解放区军队进攻的事件还未停止,并且看不出任何准备停止的征象。在这种情况下,一切污蔑和谩骂,都从国民党反动分子的嘴里喷出来。什么“奸党”、“奸军”、“奸区”,什么“破坏抗战、危害国家”等等污蔑共产党、八路军、新四军和解放区的称号和断语,都是这些反动分子制造出来的。一九三九年七月七日,中国共产党中央委员会发表宣言,针对着当时的危机,提出了这样的口号:“坚持抗战,反对投降;坚持团结,反对分裂;坚持进步,反对倒退。”按照这些适合时宜的口号,我们党在五年之内,有力地打退了三次反动的反人民的“反共高潮”,克服了当时的危机。

在这几年内,国民党战场实际上没有严重的战争。日本侵略者的刀锋,主要地向着解放区。到一九四三年,侵华日军的百分之六十四和伪军的百分之九十五,为解放区军民所抗击;国民党战场所担负的,不过日军的百分之三十六和伪军的百分之五而已。

一九四四年,日本侵略者举行打通大陆交通线的作战了,国民党军队表现了手足无措,毫无抵抗能力。几个月内,就将河南、湖南、广西、广东等省广大区域沦于敌手。仅在此时,两个战场分担的抗敌的比例,才起了一些变化。然而就在我做这个报告的时候,在侵华日军(满洲的未计在内)四十个师团,五十八万人中,解放区战场抗击的是二十二个半师团,三十二万人,占了百分之五十六;国民党战场抗击的,不过十七个半师团,二十六万人,仅占百分之四十四。抗击伪军的情况则完全无变化。

还应指出,数达八十万以上的伪军(包括伪正规军和伪地方武装在内),大部分是国民党将领率部投敌,或由国民党投敌军官所组成的。国民党反动分子事先即供给这些伪军以所谓“曲线救国”\mnote{9}的叛国谬论,事后又在精神上和组织上支持他们,使他们配合日本侵略者反对中国人民的解放区。此外,则动员大批军队封锁和进攻陕甘宁边区及各解放区,其数量达到了七十九万七千人之多。这种严重情形,在国民党政府的新闻封锁政策下,很多的中国人外国人都无法知道。

\subsection{中国解放区}

中国共产党领导的中国解放区,现在有九千五百五十万人口。其地域,北起内蒙,南至海南岛,大部分敌人所到之处,都有八路军、新四军或其它人民军队的活动。这个广大的中国解放区,包括十九个大的解放区,其地域包括辽宁、热河、察哈尔、绥远\mnote{10}、陕西、甘肃、宁夏、山西、河北、河南、山东、江苏、浙江、安徽、江西、湖北、湖南、广东、福建等省的大部分或小部分。延安是所有解放区的指导中心。在这个广大的解放区内,黄河以西的陕甘宁边区,只有人口一百五十万,是十九个解放区中的一个;而且除了浙东、琼崖两区之外,按其人口说来,它是一个最小的。有些人不明了这种情形,以为所谓中国解放区,主要就是陕甘宁边区。这是国民党政府的封锁政策造成的一个误会。在所有这些解放区内,实行了抗日民族统一战线的全部必要的政策,建立了或正在建立民选的共产党人和各抗日党派及无党无派的代表人物合作的政府,亦即地方性的联合政府。解放区内全体人民的力量都动员起来了。所有这一切,使得中国解放区在强敌压迫之下,在国民党军队的封锁和进攻的情况之下,在毫无外援的情况之下,能够屹立不摇,并且一天一天发展,缩小敌占区,扩大自己的区域,成为民主中国的模型,成为配合同盟国作战、驱逐日本侵略者、解放中国人民的主要力量。中国解放区的军队——八路军、新四军和其它人民军队,不但在对日战争的作战上,起了英勇的模范的作用,在执行抗日民族统一战线的各项民主政策上,也是起了模范作用的。一九三七年九月二十二日,中国共产党中央委员会发表宣言,承认“孙中山先生的三民主义为中国今日之必需,本党愿为其彻底实现而奋斗”,这一宣言,在中国解放区是完全实践了。

\subsection{国民党统治区}

国民党内的主要统治集团,坚持独裁统治,实行了消极的抗日政策和反人民的国内政策。这样,就使得它的军队缩小了一半以上,并且大部分几乎丧失了战斗力;使得它自己和广大人民之间发生了深刻的裂痕,造成了民生凋敝、民怨沸腾、民变蜂起的严重危机;使得它在抗日战争中的作用,不但是极大地减少了,并且变成了动员和统一中国人民一切抗日力量的障碍物。

为什么在国民党主要统治集团领导下会产生这种严重情况呢?因为这个集团所代表的利益是中国的大地主、大银行家、大买办阶层的利益。这些极少数人所形成的反动阶层,垄断着国民党政府管辖之下的军事、政治、经济、文化的一切重要的机构。他们将保全自己少数人的利益放在第一位,而把抗日放在第二位。他们也说什么“民族至上”,但是他们的行为却不符合于民族中大多数人民的要求。他们也说什么“国家至上”,但是他们所指的国家,就是大地主、大银行家、大买办阶层的封建法西斯的独裁国家,并不是人民大众的民主国家。因此,他们惧怕人民起来,惧怕民主运动,惧怕认真地动员全民的抗日战争。这就是他们对日消极作战的政策,对内的反人民、反民主、反共的反动政策的总根源。他们在各方面都采取这样的两面政策。例如:一面虽在抗日,一面又采取消极的作战政策,并且还被日本侵略者经常选择为诱降的对象。一面在口头上宣称要发展中国经济,一面又在实际上积累官僚资本,亦即大地主、大银行家、大买办的资本,垄断中国的主要经济命脉,而残酷地压迫农民,压迫工人,压迫小资产阶级和自由资产阶级。一面在口头上宣称实行“民主”,“还政于民”,一面又在实际上残酷地压迫人民的民主运动,不愿实行丝毫的民主改革。一面在口头上宣称“共党问题为一政治问题,应用政治方法解决”,一面又在军事上、政治上、经济上残酷地压迫中国共产党,把共产党看成他们的所谓“第一个敌人”,而把日本侵略者看成“第二个敌人”,并且每天都在积极地准备内战,处心积虑地要消灭共产党。一面在口头上宣称要建立一个“近代国家”,一面又在实际上拚死命保持那个大地主、大银行家、大买办的封建法西斯独裁统治。一面和苏联在形式上保持外交关系,一面又在实际上采取仇视苏联的态度。一面同美国孤立派合唱“先亚后欧论”,借以延长法西斯德国也就是延长一切法西斯的寿命,延长自己对于中国人民的法西斯统治的寿命,一面又在外交上投机取巧,把自己打扮成为反法西斯的英雄。要问如此种种的自相矛盾的两面政策从何而来,就是来自大地主、大银行家、大买办社会阶层这一个总根源。

但是,国民党是一个复杂的政党。它虽被这个代表大地主、大银行家、大买办阶层的反动集团所统治,所领导,却并不整个儿等于这个反动集团。它有一部分领袖人物不属于这个集团,而且被这个集团所打击、排斥或轻视。它有不少的干部、党员群众和三民主义青年团的团员群众并不满意这个集团的领导,而且有些甚至是反对它的领导的。在被这个反动集团所统制的国民党的军队、国民党的政府机关、国民党的经济机关和国民党的文化机关中,都存在着这种情形。在这些军队和机关里,包藏着不少的民主分子。这个反动集团,其中又分为几派,互相斗争,并不是一个严密的统一体。把国民党看成清一色的反动派,无疑是很不适当的。

\subsection{比较}

中国人民从中国解放区和国民党统治区,获得了明显的比较。

难道还不明显吗?两条路线,人民战争的路线和反对人民战争的消极抗日的路线,其结果:一条是胜利的,即使处在中国解放区这种环境恶劣和毫无外援的地位;另一条是失败的,即使处在国民党统治区这种极端有利和取得外国接济的地位。

国民党政府把自己的失败归咎于缺乏武器。但是试问:缺乏武器的是国民党的军队呢,还是解放区的军队?中国解放区的军队是中国军队中武器最缺乏的军队,他们只能从敌人手里夺取武器和在最恶劣条件下自己制造武器。

国民党中央系军队的武器,不是比起地方系军队来要好得多吗?但是比起战斗力来,中央系却多数劣于地方系。

国民党拥有广大的人力资源,但是在它的错误的兵役政策下,人力补充却极端困难。中国解放区处在被敌人分割和战斗频繁的情况之下,因为普遍实施了适合人民需要的民兵和自卫军制度,又防止了对于人力资源的滥用和浪费,人力动员却可以源源不竭。

国民党拥有粮食丰富的广大地区,人民每年供给它七千万至一万万市担的粮食,但是大部分被经手人员中饱了,致使国民党的军队经常缺乏粮食,士兵饿得面黄肌瘦。中国解放区的主要部分隔在敌后,遭受敌人烧杀抢“三光”政策的摧残,其中有些是像陕北这样贫瘠的区域,但是却能用自己动手、发展农业生产的方法,很好地解决了粮食问题。

国民党区域经济危机极端严重,工业大部分破产了,连布匹这样的日用品也要从美国运来。中国解放区却能用发展工业的方法,自己解决布匹和其它日用品的需要。

在国民党区域,工人、农民、店员、公务人员、知识分子以及文化工作者,生活痛苦,达于极点。中国解放区的全体人民都有饭吃,有衣穿,有事做。

利用抗战发国难财,官吏即商人,贪污成风,廉耻扫地,这是国民党区域的特色之一。艰苦奋斗,以身作则,工作之外,还要生产,奖励廉洁,禁绝贪污,这是中国解放区的特色之一。

国民党区域剥夺人民的一切自由。中国解放区则给予人民以充分的自由。

国民党统治者面前摆着这些反常的状况,怪谁呢?怪别人,还是怪他们自己呢?怪外国缺少援助,还是怪国民党政府的独裁统治和腐败无能呢?这难道还不明白吗?

\subsection{“破坏抗战、危害国家”的是谁?}

真凭实据地破坏了中国人民的抗战和危害了中国人民的国家的,难道不正是国民党政府吗?这个政府一心一意地打了整十年的内战,将刀锋向着同胞,置一切国防事业于不顾,又用不抵抗政策送掉了东北四省\mnote{11}。日本侵略者打进关内来了,仓皇应战,从卢沟桥退到了贵州省。但是国民党人却说:“共产党破坏抗战,危害国家。”(见一九四三年九月国民党十一中全会的决议案)唯一的证据,就是共产党联合了各界人民创造了英勇抗日的中国解放区。这些国民党人的逻辑,和中国人民的逻辑是这样的不相同,无怪乎很多问题都讲不通了。

两个问题:

第一个,究竟什么原因使得国民党政府抛弃了从黑龙江到卢沟桥,又从卢沟桥到贵州省这样广大的国土和这样众多的人民?难道不是由于国民党政府所采取的不抵抗政策、消极的抗日政策和反人民的国内政策吗?

第二个,究竟什么原因使得中国解放区战胜了敌伪军长期的残酷的进攻,从民族敌人手里恢复了这样广大的国土,解放了这样众多的人民?难道不是由于人民战争的正确路线吗?

\subsection{所谓“不服从政令、军令”}

国民党政府还经常以“不服从政令、军令”责备中国共产党。但是我们只能这样说:幸喜中国共产党人还保存了中国人民的普通常识,没有服从那些实际上是把中国人民艰难困苦地从日本侵略者手里夺回来的中国解放区再送交日本侵略者的这种所谓“政令、军令”,例如,一九三九年的所谓《限制异党活动办法》,一九四一年的所谓“解散新四军”和“退至旧黄河以北”,一九四三年的所谓“解散中国共产党”,一九四四年的所谓“限期取消十个师以外的全部军队”,以及在最近谈判中提出来的所谓将军队和地方政府移交给国民党,其交换条件是不许成立联合政府,只许收容几个共产党员到国民党独裁政府里去做官,并将这种办法称之为国民党政府的“让步”等等。幸喜我们没有服从这些东西,替中国人民保存了一片干净土,保存了一支英勇抗日的军队。难道中国人民不应该庆贺这一个“不服从”吗?难道国民党政府自己用自己的法西斯主义的政令和失败主义的军令,将黑龙江至贵州省的广大的土地、人民送交日本侵略者,还觉得不够吗?除了日本侵略者和反动派欢迎这些“政令、军令”之外,难道还有什么爱国的有良心的中国人欢迎这些东西吗?没有一个不是形式的而是实际的、不是法西斯独裁的而是民主的联合政府,能够设想中国人民会允许中国共产党人,擅自将这个获得了解放的中国解放区和抗日有功的人民军队,交给失败主义和法西斯主义的国民党法西斯独裁政府吗?假如没有中国解放区及其军队,中国人民的抗日事业还有今日吗?我们民族的前途还能设想吗?

\subsection{内战危险}

迄今为止,国民党内的主要统治集团,坚持着独裁和内战的反动方针。有很多迹象表明,他们早已准备,尤其现在正在准备这样的行动:等候某一个同盟国的军队在中国大陆上驱逐日本侵略者到了某一程度时,他们就要发动内战。他们并且希望某些同盟国的将领们在中国境内执行英国斯科比\mnote{12}将军在希腊所执行的职务。他们对于斯科比和希腊反动政府的屠杀事业,表示欢呼。他们企图把中国抛回到一九二七年至一九三七年的国内战争的大海里去。国民党主要统治集团现在正在所谓“召开国民大会”和“政治解决”的烟幕之下,偷偷摸摸地进行其内战的准备工作。如果国人不加注意,不去揭露它的阴谋,阻止它的准备,那末,会有一个早上,要听到内战的炮声的。

谈判

为着打败日本侵略者和建设新中国,为着防止内战,中国共产党在取得了其它民主派别的同意之后,于一九四四年九月间的国民参政会上,提出了立即废止国民党一党专政、成立民主的联合政府一项要求。无疑地,这项要求是适合时宜的,几个月内,获得了广大人民的响应。

关于如何废止一党专政、成立联合政府以及实行必要的民主改革等项问题,我们和国民党政府之间曾经有过多次谈判,但是我们的一切建议都遭到了国民党政府的拒绝。国民党不但对一党专政不愿废止,对联合政府不愿成立,即对任何迫切需要的民主改革,例如,取消特务机关,取消镇压人民自由的反动法令,释放政治犯,承认各党派的合法地位,承认解放区,撤退封锁和进攻解放区的军队等等,也一项不愿实行。就是这样,使得中国的政治关系处在非常严重的局面之下。

\subsection{两个前途}

从整个形势看来,从上述一切国际国内的实际情况的分析看来,我请大家注意,不要以为我们的事业,一切都将是顺利的,美妙的。不,不是这样,事实是好坏两个可能性、好坏两个前途都存在着。继续法西斯独裁统治,不许民主改革;不是将重点放在反对日本侵略者方面,而是放在反对人民方面;即使日本侵略者被打败了,中国仍然可能发生内战,将中国拖回到痛苦重重的不独立、不自由、不民主、不统一、不富强的老状态里去。这是一个可能性,这是一个前途。这个可能性,这个前途,依然存在,并不因为国际形势好,国内人民觉悟程度增长和有组织的人民力量发展了,它就似乎没有了,或自然地消失了。希望中国实现这个可能性、实现这个前途的,在中国是国民党内的反人民集团,在外国是那些怀抱帝国主义思想的反动分子。这是一方面,这是必须注意的一方面。

但是,另一方面,同样是从整个形势看来,从上述一切内外情况的分析看来,使我们更有信心地更有勇气地去争取第二个可能性,第二个前途。这就是克服一切困难,团结全国人民,废止国民党的法西斯独裁统治,实行民主改革,巩固和扩大抗日力量,彻底打败日本侵略者,将中国建设成为一个独立、自由、民主、统一和富强的新国家。希望中国实现这个可能性、实现这个前途的,在中国是广大的人民,中国共产党及其它民主派别,在外国是一切以平等地位待我的民族,外国的进步分子,外国的人民大众。

我们清楚地懂得,在我们和中国人民面前,还有很大的困难,还有很多的障碍物,还要走很多的迂回路程。但是我们同样地懂得,任何困难和障碍物,我们和全国人民一道一定能够加以克服,而使中国的历史任务获得完成。竭尽全力地去反对第一个可能性,争取第二个可能性,反对第一个前途,争取第二个前途,是我们和全国人民的伟大任务。国际国内形势的主要方面,是有利于我们和全国人民的。这些,我在前面已经说得很清楚了。我们希望国民党当局,鉴于世界大势之所趋,中国人心之所向,毅然改变其错误的现行政策,使抗日战争获得胜利,使中国人民少受痛苦,使新中国早日诞生。须知不论怎样迂回曲折,中国人民独立解放的任务总是要完成的,而且这种时机已经到来了。一百多年来无数先烈所怀抱的宏大志愿,一定要由我们这一代人去实现,谁要阻止,到底是阻止不了的。

\section{四 中国共产党的政策}

上面,我已将中国抗日战争中的两条路线,给了一个分析。这样的一个分析是完全必要的。因为在广大的中国人中间,至今还有很多人不明白中国抗日战争中的具体情况。在国民党统治区,在国外,由于国民党政府的封锁政策,很多人被蒙住了眼睛。在一九四四年中外新闻记者参观团来到中国解放区以前,那里的许多人对于解放区几乎是什么也不知道的。国民党政府非常害怕解放区的真实情况泄露出去,所以在一九四四年的一次新闻记者团回去之后,立即将大门堵上,不许一个新闻记者再来解放区。对于国民党区域的真相,国民党政府也是同样地加以封锁。因此,我感到我们有责任将“两个区域”的真相尽可能使人们弄清楚。只有在弄清中国的全部情况之后,才有可能了解中国的两个最大政党——中国共产党和中国国民党的政策何以有这样的不同,何以有这样的两条路线之争。只有这样,才会使人们了解,两党的争论,不是如有些人们所说不过是一些不必要的,不重要的,或甚至是意气用事的争论,而是关系着几万万人民生死问题的原则的争论。

在目前中国时局的严重形势下,中国人民,中国一切民主党派和民主分子,一切关心中国时局的外国人民,都希望中国的分裂局面重趋于团结,都希望中国能实行民主改革,都愿意知道中国共产党对于解决当前许多重大问题上所持的政策。我们的党员对于这些,当然更加关心。

我们的抗日民族统一战线的政策历来是明确的,八年的战争考验了这些政策。我们的大会应该对此作出结论,作为今后奋斗的指针。

下面,我就来说明我们党在为解决中国问题而得出的关于重要政策方面的若干确定的结论。

\subsection{我们的一般纲领}

为着动员和统一中国人民一切抗日力量,彻底消灭日本侵略者,并建立独立、自由、民主、统一和富强的新中国,中国人民,中国共产党和一切抗日的民主党派,迫切地需要一个互相同意的共同纲领。

这种共同纲领,可以分为一般性的和具体性的两部分。我们先来说一般性的纲领,然后再说到具体性的纲领。

在彻底消灭日本侵略者和建设新中国的大前提之下,在中国的现阶段,我们共产党人在这样一个基本点上是和中国人口中的最大多数相一致的。这就是说:第一,中国的国家制度不应该是一个由大地主大资产阶级专政的、封建的、法西斯的、反人民的国家制度,因为这种反人民的制度,已由国民党主要统治集团的十八年统治证明为完全破产了。第二,中国也不可能、因此就不应该企图建立一个纯粹民族资产阶级的旧式民主专政的国家,因为在中国,一方面,民族资产阶级在经济上和政治上都表现得很软弱;另一方面,中国早已产生了一个觉悟了的,在中国政治舞台上表现了强大能力的,领导了广大的农民阶级、城市小资产阶级、知识分子以及其它民主分子的中国无产阶级及其领袖——中国共产党这样的新条件。第三,在中国的现阶段,在中国人民的任务还是反对民族压迫和封建压迫,在中国社会经济的必要条件还不具备时,中国人民也不可能实现社会主义的国家制度。

那末,我们的主张是什么呢?我们主张在彻底地打败日本侵略者之后,建立一个以全国绝对大多数人民为基础而在工人阶级领导之下的统一战线的民主联盟的国家制度,我们把这样的国家制度称之为新民主主义的国家制度。

这是一个真正适合中国人口中最大多数的要求的国家制度,因为,第一,它取得了和可能取得数百万产业工人,数千万手工业工人和雇佣农民的同意;其次,也取得了和可能取得占中国人口百分之八十,即在四亿五千万人口中占了三亿六千万的农民阶级的同意;又其次,也取得了和可能取得广大的城市小资产阶级、民族资产阶级、开明士绅及其它爱国分子的同意。

自然,这些阶级之间仍然是有矛盾的,例如劳资之间的矛盾,就是显着的一种;因此,这些阶级各有一些不同的要求。抹杀这种矛盾,抹杀这种不同要求,是虚伪的和错误的。但是,这种矛盾,这种不同的要求,在整个新民主主义的阶段上,不会也不应该使之发展到超过共同要求之上。这种矛盾和这种不同的要求,可以获得调节。在这种调节下,这些阶级可以共同完成新民主主义国家的政治、经济和文化的各项建设。

我们主张的新民主主义的政治,就是推翻外来的民族压迫,废止国内的封建主义的和法西斯主义的压迫,并且主张在推翻和废止这些之后不是建立一个旧民主主义的政治制度,而是建立一个联合一切民主阶级的统一战线的政治制度。我们的这种主张,是和孙中山先生的革命主张完全一致的。孙先生在其所著《中国国民党第一次全国代表大会宣言》里说:“近世各国所谓民权制度,往往为资产阶级所专有,适成为压迫平民之工具。若国民党之民权主义,则为一般平民所共有,非少数人所得而私也。”这是孙先生的伟大的政治指示。中国人民,中国共产党及其它一切民主分子,必须尊重这个指示而坚决地实行之,并同一切违背和反对这个指示的任何人们和任何集团作坚决的斗争,借以保护和发扬这个完全正确的新民主主义的政治原则。

新民主主义的政权组织,应该采取民主集中制,由各级人民代表大会决定大政方针,选举政府。它是民主的,又是集中的,就是说,在民主基础上的集中,在集中指导下的民主。只有这个制度,才既能表现广泛的民主,使各级人民代表大会有高度的权力;又能集中处理国事,使各级政府能集中地处理被各级人民代表大会所委托的一切事务,并保障人民的一切必要的民主活动。

军队和其它武装力量,是新民主主义的国家权力机关的重要部分,没有它们,就不能保卫国家。新民主主义国家的一切武装力量,如同其它权力机关一样,是属于人民和保护人民的,它们和一切属于少数人、压迫人民的旧式军队、旧式警察等等,完全不同。

我们主张的新民主主义的经济,也是符合于孙先生的原则的。在土地问题上,孙先生主张“耕者有其田”。在工商业问题上,孙先生在上述宣言里这样说:“凡本国人及外国人之企业,或有独占的性质,或规模过大为私人之力所不能办者,如银行、铁道、航路之属,由国家经营管理之,使私有资本制度不能操纵国民之生计,此则节制资本之要旨也。”在现阶段上,对于经济问题,我们完全同意孙先生的这些主张。

有些人怀疑中国共产党人不赞成发展个性,不赞成发展私人资本主义,不赞成保护私有财产,其实是不对的。民族压迫和封建压迫残酷地束缚着中国人民的个性发展,束缚着私人资本主义的发展和破坏着广大人民的财产。我们主张的新民主主义制度的任务,则正是解除这些束缚和停止这种破坏,保障广大人民能够自由发展其在共同生活中的个性,能够自由发展那些不是“操纵国民生计”而是有益于国民生计的私人资本主义经济,保障一切正当的私有财产。

按照孙先生的原则和中国革命的经验,在现阶段上,中国的经济,必须是由国家经营、私人经营和合作社经营三者组成的。而这个国家经营的所谓国家,一定要不是“少数人所得而私”的国家,一定要是在无产阶级领导下而“为一般平民所共有”的新民主主义的国家。

新民主主义的文化,同样应该是“为一般平民所共有”的,即是说,民族的、科学的、大众的文化,决不应该是“少数人所得而私”的文化。

上述一切,就是我们共产党人在现阶段上,在整个资产阶级民主革命的阶段上所主张的一般纲领,或基本纲领。对于我们的社会主义和共产主义制度的将来纲领或最高纲领说来,这是我们的最低纲领。实行这个纲领,可以把中国从现在的国家状况和社会状况向前推进一步,即是说,从殖民地、半殖民地和半封建的国家和社会状况,推进到新民主主义的国家和社会。

这个纲领所规定的无产阶级在政治上的领导权,无产阶级领导下的国营经济和合作社经济,是社会主义的因素。但是这个纲领的实行,还没有使中国成为社会主义社会。

我们共产党人从来不隐瞒自己的政治主张。我们的将来纲领或最高纲领,是要将中国推进到社会主义社会和共产主义社会去的,这是确定的和毫无疑义的。我们的党的名称和我们的马克思主义的宇宙观,明确地指明了这个将来的、无限光明的、无限美妙的最高理想。每个共产党员入党的时候,心目中就悬着为现在的新民主主义革命而奋斗和为将来的社会主义和共产主义而奋斗这样两个明确的目标,而不顾那些共产主义敌人的无知的和卑劣的敌视、污蔑、谩骂或讥笑;对于这些,我们必须给以坚决的排击。对于那些善意的怀疑者,则不是给以排击而是给以善意的和耐心的解释。所有这些,都是异常清楚、异常确定和毫不含糊的。

但是,一切中国共产党人,一切中国共产主义的同情者,必须为着现阶段的目标而奋斗,为着反对民族压迫和封建压迫,为着使中国人民脱离殖民地、半殖民地、半封建的悲惨命运,和建立一个在无产阶级领导下的以农民解放为主要内容的新民主主义性质的,亦即孙中山先生革命三民主义性质的独立、自由、民主、统一和富强的中国而奋斗。我们果然是这样做了,我们共产党人,协同广大的中国人民,曾为此而英勇奋斗了二十四年。

对于任何一个共产党人及其同情者,如果不为这个目标奋斗,如果看不起这个资产阶级民主革命而对它稍许放松,稍许怠工,稍许表现不忠诚、不热情,不准备付出自己的鲜血和生命,而空谈什么社会主义和共产主义,那就是有意无意地、或多或少地背叛了社会主义和共产主义,就不是一个自觉的和忠诚的共产主义者。只有经过民主主义,才能到达社会主义,这是马克思主义的天经地义。而在中国,为民主主义奋斗的时间还是长期的。没有一个新民主主义的联合统一的国家,没有新民主主义的国家经济的发展,没有私人资本主义经济和合作社经济的发展,没有民族的科学的大众的文化即新民主主义文化的发展,没有几万万人民的个性的解放和个性的发展,一句话,没有一个由共产党领导的新式的资产阶级性质的彻底的民主革命,要想在殖民地半殖民地半封建的废墟上建立起社会主义社会来,那只是完全的空想。

有些人不了解共产党人为什么不但不怕资本主义,反而在一定的条件下提倡它的发展。我们的回答是这样简单:拿资本主义的某种发展去代替外国帝国主义和本国封建主义的压迫,不但是一个进步,而且是一个不可避免的过程。它不但有利于资产阶级,同时也有利于无产阶级,或者说更有利于无产阶级。现在的中国是多了一个外国的帝国主义和一个本国的封建主义,而不是多了一个本国的资本主义,相反地,我们的资本主义是太少了。说也奇怪,有些中国资产阶级代言人不敢正面地提出发展资本主义的主张,而要转弯抹角地来说这个问题。另外有些人,则甚至一口否认中国应该让资本主义有一个必要的发展,而说什么一下就可以到达社会主义社会,什么要将三民主义和社会主义“毕其功于一役”。很明显地,这类现象,有些是反映着中国民族资产阶级的软弱性,有些则是大地主大资产阶级对于民众的欺骗手段。我们共产党人根据自己对于马克思主义的社会发展规律的认识,明确地知道,在中国的条件下,在新民主主义的国家制度下,除了国家自己的经济、劳动人民的个体经济和合作社经济之外,一定要让私人资本主义经济在不能操纵国民生计的范围内获得发展的便利,才能有益于社会的向前发展。对于中国共产党人,任何的空谈和欺骗,是不会让它迷惑我们的清醒头脑的。

有些人怀疑共产党人承认“三民主义为中国今日之必需,本党愿为其彻底实现而奋斗”,似乎不是忠诚的。这是由于不了解我们所承认的孙中山先生在一九二四年《中国国民党第一次全国代表大会宣言》里所解释的三民主义的基本原则,同我党在现阶段的纲领即最低纲领里的若干基本原则,是互相一致的。应当指出,孙先生的这种三民主义,和我党在现阶段上的纲领,只是在若干基本原则上是一致的东西,并不是完全一致的东西。我党的新民主主义纲领,比之孙先生的,当然要完备得多;特别是孙先生死后这二十年中中国革命的发展,使我党新民主主义的理论、纲领及其实践,有了一个极大的发展,今后还将有更大的发展。但是,孙先生的这种三民主义,按其基本性质说来,是一个和在此以前的旧三民主义相区别的新民主主义的纲领,当然这是“中国今日之必需”,当然“本党愿为其彻底实现而奋斗”。对于中国共产党人,为本党的最低纲领而奋斗和为孙先生的革命三民主义即新三民主义而奋斗,在基本上(不是在一切方面)是一件事情,并不是两件事情。因此,不但在过去和现在已经证明,而且在将来还要证明:中国共产党人是革命三民主义的最忠诚最彻底的实现者。

有些人怀疑共产党得势之后,是否会学俄国那样,来一个无产阶级专政和一党制度。我们的答复是:几个民主阶级联盟的新民主主义国家,和无产阶级专政的社会主义国家,是有原则上的不同的。毫无疑义,我们这个新民主主义制度是在无产阶级的领导之下,在共产党的领导之下建立起来的,但是中国在整个新民主主义制度期间,不可能、因此就不应该是一个阶级专政和一党独占政府机构的制度。只要共产党以外的其它任何政党,任何社会集团或个人,对于共产党是采取合作的而不是采取敌对的态度,我们是没有理由不和他们合作的。俄国的历史形成了俄国的制度,在那里,废除了人剥削人的社会制度,实现了最新式的民主主义即社会主义的政治、经济、文化制度,一切反对社会主义的政党都被人民抛弃了,人民仅仅拥护布尔什维克党,因此形成了俄国的局面,这在他们是完全必要和完全合理的。但是在俄国的政权机关中,即使是处在除了布尔什维克党以外没有其它政党的条件下,实行的还是工人、农民和知识分子联盟,或党和非党联盟的制度,也不是只有工人阶级或只有布尔什维克党人才可以在政权机关中工作。中国现阶段的历史将形成中国现阶段的制度,在一个长时期中,将产生一个对于我们是完全必要和完全合理同时又区别于俄国制度的特殊形态,即几个民主阶级联盟的新民主主义的国家形态和政权形态。

\subsection{我们的具体纲领}

根据上述一般纲领,我们党在各个时期中还应当有具体的纲领。在整个资产阶级民主革命阶段中,在几十年中,我们的新民主主义的一般纲领是不变的。但是在这个大阶段的各个小的阶段中,情形是变化了和变化着的,我们的具体纲领便不能不有所改变,这是当然的事情。例如,在北伐战争时期,在土地革命战争时期和在抗日战争时期,我们的新民主主义的一般纲领并没有变化,但其具体纲领,三个时期中是有了变化的,这是因为我们的敌军和友军在三个时期中发生了变化的缘故。

目前中国人民是处在这样的情况中:(一)日本侵略者还未被打败;(二)中国人民迫切地需要团结起来,实现一个民主的改革,以便造成民族团结,迅速地动员和统一一切抗日力量,配合同盟国打败日本侵略者;(三)国民党政府分裂民族团结,阻碍这种民主的改革。在这些情况下,我们的具体纲领即中国人民的现时要求是什么呢?

我们认为下面这些要求是适当的,并且是最低限度的。

动员一切力量,配合同盟国,彻底打败日本侵略者,并建立国际和平;要求废止国民党一党专政,建立民主的联合政府和联合统帅部;要求惩办那些分裂民族团结和反对人民的亲日分子、法西斯主义分子和失败主义分子,造成民族团结;要求惩办那些制造内战危机的反动分子,保障国内和平;要求惩办汉奸,讨伐降敌军官,惩办日本间谍;要求取消一切镇压人民的反动的特务机关和特务活动,取消集中营;要求取消一切镇压人民的言论、出版、集会、结社、思想、信仰和身体等项自由的反动法令,使人民获得充分的自由权利;要求承认一切民主党派的合法地位;要求释放一切爱国政治犯;要求撤退一切包围和进攻中国解放区的军队,并将这些军队使用于抗日前线;要求承认中国解放区的一切抗日军队和民选政府;要求巩固和扩大解放区及其军队,收复一切失地;要求帮助沦陷区人民组织地下军,准备武装起义;要求允许中国人民自动武装起来,保乡卫国;要求从政治上军事上改造那些由国民党统帅部直接领导的经常打败仗、经常压迫人民和经常排斥异己的军队,惩办那些应对溃败负责的将领;要求改善兵役制度和改善官兵生活;要求优待抗日军人家属,使前线官兵安心作战;要求优待殉国战士的遗族,优待残废军人,帮助退伍军人解决生活和就业问题;要求发展军事工业,以利作战;要求将同盟国的武器和财政援助公平地分配给抗战各军;要求惩办贪官污吏,实现廉洁政治;要求改善中下级公务员的待遇;要求给予中国人民以民主的权利;要求取消压迫人民的保甲制度\mnote{13};要求救济难民和救济灾荒;要求设立大量的救济基金,在国土收复后,广泛地救济沦陷区的受难人民;要求取消苛捐杂税,实行统一的累进税;要求实行农村改革,减租减息,适当地保证佃权,对贫苦农民给予低利贷款,并使农民组织起来,以利于发展农业生产;要求取缔官僚资本;要求废止现行的经济统制政策;要求制止无限制的通货膨胀和无限制的物价高涨;要求扶助民间工业,给予民间工业以借贷资本、购买原料和推销产品的便利;要求改善工人生活,救济失业工人,并使工人组织起来,以利于发展工业生产;要求取消国民党的党化教育\mnote{14},发展民族的科学的大众的文化教育;要求保障教职员生活和学术自由;要求保护青年、妇女、儿童的利益,救济失学青年,并使青年、妇女组织起来,以平等地位参加有益于抗日战争和社会进步的各项工作,实现婚姻自由,男女平等,使青年和儿童得到有益的学习;要求改善国内少数民族的待遇,允许各少数民族有民族自治的权利;要求保护华侨利益,扶助回国的华侨;要求保护因被日本侵略者压迫而逃来中国的外国人民,并扶助其反对日本侵略者的斗争;要求改善中苏邦交,等等。而要做到这一切,最重要的是要求立即取消国民党一党专政,建立一个包括一切抗日党派和无党派的代表人物在内的举国一致的民主的联合的临时的中央政府。没有这个前提条件,要想在全国范围内,就是说,在国民党统治区域进行稍为认真的改革,是不可能的。

这些都是中国广大人民的呼声,也是各同盟国广大民主舆论界的呼声。

一个为各个抗日民主党派互相同意的最低限度的具体纲领,是完全必要的,我们准备以上述纲领为基础和他们进行协商。各党可以有不同的要求,但是各党之间应该协定一个共同的纲领。

这样的纲领,对于国民党统治区,暂时还是一个要求的纲领;对于沦陷区,除组织地下军准备武装起义一项外,是一个要等到收复后才能实施的纲领;对于解放区,则是一个早已实施并应当继续实施的纲领。

在上述中国人民的目前要求或具体纲领中,包含着许多战时和战后的重大问题,需要在下面加以说明。在说明这些问题时,我们将批评国民党主要统治集团的一些错误观点,同时也将回答其它人们的一些疑问。

\subsubsection{第一 彻底消灭日本侵略者,不许中途妥协}

开罗会议\mnote{15}决定应使日本侵略者无条件投降,这是正确的。但是,现在日本侵略者正在暗地里进行活动,企图获得妥协的和平;国民党政府中的亲日分子,经过南京傀儡政府,也正在和日本密使勾勾搭搭,并未遇到制止。因此,中途妥协的危险并未完全过去。开罗会议又决定将东北四省、台湾、澎湖列岛归还中国,这是很好的。但是根据国民党政府的现行政策,要想依靠它打到鸭绿江边,收复一切失地,是不可能的。在这种情形下,中国人民应该怎么办呢?中国人民应该要求国民党政府彻底消灭日本侵略者,不许中途妥协。一切妥协的阴谋活动,必须立刻制止。中国人民应该要求国民党政府改变现在的消极的抗日政策,将其一切军事力量用于积极对日作战。中国人民应该扩大自己的军队——八路军、新四军及其它人民军队,并在一切敌人所到之处,广泛地自动地发展抗日武装,准备直接配合同盟国作战,收复一切失地,决不要单纯地依靠国民党。打败日本侵略者,是中国人民的神圣的权利。如果反动分子要想剥夺中国人民的这种神圣的权利,要想压制中国人民的抗日活动,要想破坏中国人民的抗日力量,那末,中国人民在向他们劝说无效之后,应该站在自卫立场上给以坚决的回击。因为中国反动分子的这种背叛民族利益的反动行为,完全是帮助日本侵略者的。

\subsubsection{第二 废止国民党一党专政,建立民主的联合政府}

为着彻底消灭日本侵略者,必须在全国范围内实行民主改革。而要这样做,不废止国民党的一党专政,建立民主的联合政府,是不可能的。

所谓国民党的一党专政,实际上是国民党内反人民集团的专政,它是中国民族团结的破坏者,是国民党战场抗日失败的负责者,是动员和统一中国人民抗日力量的根本障碍物。从八年抗日战争的惨痛经验中,中国人民已经深刻地认识了它的罪恶,很自然地要求立即废止它。这个反人民的专政,又是内战的祸胎,如不立即废止,内战惨祸又将降临。

中国人民要求废止这个反人民专政的呼声是如此普遍而响亮了,使得国民党当局自己也不能不公开承认“提早结束训政”,可见这个所谓“训政”或“一党专政”的丧失人心,威信扫地,到了何种地步了。在中国,已经没有一个人还敢说“训政”或“一党专政”有什么好处,不应该废止或“结束”了,这是当前时局的一大变化。

应该“结束”是确定的了,毫无疑义的了。但是如何结束呢,可就意见分歧了。一个说:立即结束,成立民主的临时的联合政府。一个说:等一会再结束,召开“国民大会”,“还政于民”,却不能还政于联合政府。

这是什么意思呢?

这是两种做法的表现:真做和假做。

第一种,真做。这就是立即宣布废止国民党一党专政,成立一个由国民党、共产党、民主同盟\mnote{16}和无党无派分子的代表人物联合组成的临时的中央政府,发布一个民主的施政纲领,如同我们在前面提出的那些中国人民的现时要求,以便恢复民族团结,打败日本侵略者。为着讨论这些事情,召集一个各党派和无党派的代表人物的圆桌会议,成立协议,动手去做。这是一个团结的方针,中国人民是坚决拥护这个方针的。

第二种,假做。不顾广大人民和一切民主党派的要求,一意孤行地召开一个由国民党反人民集团一手包办的所谓“国民大会”,在这个会上通过一个实际上维持独裁反对民主的所谓“宪法”,使那个仅仅由几十个国民党人私自委任的、完全没有民意基础的、强安在人民头上的、不合法的所谓国民政府,披上合法的外衣,装模作样地“还政于民”,实际上,依然是“还政”于国民党内的反人民集团。谁要不赞成,就说他是破坏“民主”,破坏“统一”,就有“理由”向他宣布讨伐令。这是一个分裂的方针,中国人民是坚决反对这个方针的。

我们的反人民的英雄们根据这种分裂方针所准备采取的步骤,有把他们自己推到绝路上去的危险。他们准备把一条绳索套在自己的脖子上,并且让它永远也解不开,这条绳索的名称就叫做“国民大会”。他们的原意是想把所谓“国民大会”当作法宝,祭起来,一则抵制联合政府,二则维持独裁统治,三则准备内战理由。可是,历史的逻辑将向他们所设想的反面走去,“搬起石头打自己的脚”。因为现在谁也明白,在国民党统治区域,人民没有自由,在日寇占领区域,人民不能参加选举,有了自由的中国解放区,国民党政府又不承认它,在这种情况下,哪里来的国民代表?哪里来的“国民大会”?现在叫着要开的,是那个还在内战时期,还在八年以前,由国民党独裁政府一手伪造的所谓国民大会。如果这个会开成了,势必闹到全国人民群起反对,请问我们的反人民的英雄们如何下台?归根结底,伪造国民大会如果开成了,不过将他们自己推到绝路上。

我们共产党人提出结束国民党一党专政的两个步骤:第一个步骤,目前时期,经过各党各派和无党无派代表人物的协议,成立临时的联合政府;第二个步骤,将来时期,经过自由的无拘束的选举,召开国民大会,成立正式的联合政府。总之,都是联合政府,团结一切愿意参加的阶级和政党的代表在一起,在一个民主的共同纲领之下,为现在的抗日和将来的建国而奋斗。

不管国民党人或任何其它党派、集团和个人如何设想,愿意或不愿意,自觉或不自觉,中国只能走这条路。这是一个历史法则,是一个必然的、不可避免的趋势,任何力量,都是扭转不过来的。

在这个问题和其它任何有关民主改革的问题上,我们共产党人声明:不管国民党当局现在还是怎样坚持其错误政策和怎样借谈判为拖延时间、搪塞舆论的手段,只要他们一旦愿意放弃其错误的现行政策,同意民主改革,我们是愿意和他们恢复谈判的。但是谈判的基础必须放在抗日、团结和民主的总方针上,一切离开这个总方针的所谓办法、方案,或其它空话,不管它怎样说得好听,我们是不能赞成的。

\subsubsection{第三 人民的自由}

目前中国人民争自由的目标,首先地和主要地是向着日本侵略者。但是国民党政府剥夺人民的自由,捆起人民的手足,使他们不能反对日本侵略者。不解决这个问题,就不能在全国范围内动员和统一一切抗日的力量。我们在纲领中提出了废止一党专政,成立联合政府,取消特务,取消镇压自由的法令,惩办汉奸、间谍、亲日分子、法西斯分子和贪官污吏,释放政治犯,承认各民主党派的合法地位,撤退包围和进攻解放区的军队,承认解放区,废止保甲制度,以及其它许多经济的文化的和民众运动的要求,就是为着解开套在人民身上的绳索,使人民获得抗日、团结和民主的自由。

自由是人民争来的,不是什么人恩赐的。中国解放区的人民已经争得了自由,其它地方的人民也可能和应该争得这种自由。中国人民争得的自由越多,有组织的民主力量越大,一个统一的临时的联合政府便越有成立的可能。这种联合政府一经成立,它将转过来给予人民以充分的自由,巩固联合政府的基础。然后才有可能,在日本侵略者被打倒之后,在全部国土上进行自由的无拘束的选举,产生民主的国民大会,成立统一的正式的联合政府。没有人民的自由,就没有真正民选的国民大会,就没有真正民选的政府。难道还不清楚吗?

人民的言论、出版、集会、结社、思想、信仰和身体这几项自由,是最重要的自由。在中国境内,只有解放区是彻底地实现了。

一九二五年,孙中山先生在其临终的遗嘱上说:“余致力国民革命凡四十年,其目的在求中国之自由平等。积四十年之经验,深知欲达到此目的,必须唤起民众及联合世界上以平等待我之民族共同奋斗。”背叛孙先生的不肖子孙,不是唤起民众,而是压迫民众,将民众的言论、出版、集会、结社、思想、信仰和身体等项自由权利剥夺得干干净净;对于认真唤起民众、认真保护民众自由权利的共产党、八路军、新四军和解放区,则称之为“奸党”、“奸军”、“奸区”。我们希望这种颠倒是非的时代快些过去。如果要延长这种颠倒是非的时间,中国人民将不能忍耐了。

\subsubsection{第四 人民的统一}

为着消灭日本侵略者,为着防止内战,为着建设新中国,必须将分裂的中国变为统一的中国,这是中国人民的历史任务。

但是如何统一呢?独裁者的专制的统一,还是人民的民主的统一呢?从袁世凯\mnote{17}以来,北洋军阀强调专制的统一。但是结果怎么样呢?和这些军阀的志愿相反,所得的不是统一而是分裂,最后是他们自己从台上滚下去。国民党反人民集团抄袭袁世凯的老路,追求专制的统一,打了整整十年的内战,结果把一个日本侵略者打了进来,自己也缩上了峨眉山\mnote{18}。现在又在山上大叫其专制统一论,这是叫给谁听呢?难道还有什么爱国的有良心的中国人愿意听它吗?经过了十六年的北洋军阀的统治,又经过了十八年的国民党的独裁统治,人民已经有了充分的经验,有了明亮的眼睛。他们要一个人民大众的民主的统一,不要独裁者的专制的统一。我们共产党人还在一九三五年就提出了抗日民族统一战线的方针,没有一天不为此而奋斗。一九三九年国民党推行其反动的《限制异党活动办法》,造成投降、分裂、倒退的危机,国民党人大叫其专制统一论的时候,我们又说:非统一于投降而统一于抗战,非统一于分裂而统一于团结,非统一于倒退而统一于进步。只有这后一种统一才是真统一,其它一切都是假统一\mnote{19}。又过了六年了,问题还是一样。

没有人民的自由,没有人民的民主政治,能够统一吗?有了这些,立刻就统一了。中国人民争自由、争民主、争联合政府的运动,同时就是争统一的运动。我们在具体纲领中提出了许多争自由争民主的要求,提出了联合政府的要求,同时就是为了这个目的。不废止国民党内反人民集团的专政,成立民主的联合政府,不但在国民党统治区不能实行任何民主的改革,不能动员那里的全体军民打倒日本侵略者,而且还将发展为内战的惨祸,这是很多人都明白的常识了。为什么如此众多的有党有派无党无派的民主分子,包括国民党内的许多民主分子在内,一致要求成立联合政府?就因为他们看清楚了时局的危机,非如此不能克服这种危机,不能达到团结对敌和团结建国的目的。

\subsubsection{第五 人民的军队}

中国人民要自由,要统一,要联合政府,要彻底地打倒日本侵略者和建设新中国,没有一支站在人民立场上的军队,那是不行的。彻底地站在人民立场的军队,现在还只有解放区的不很大的八路军和新四军,还很不够。可是,国民党内的反人民集团却处心积虑地要破坏和消灭解放区的军队。一九四四年,国民党政府提出了一个所谓“提示案”,叫共产党“限期取消”解放区军队的五分之四。一九四五年,即最近的一次谈判,又叫共产党将解放区军队全部交给它,然后它给共产党以“合法地位”。

这些人们向共产党人说:你交出军队,我给你自由。根据这个学说,没有军队的党派该有自由了。但是一九二四年至一九二七年,中国共产党只有很少一点军队,国民党政府的“清党”政策和屠杀政策一来,自由也光了。现在的中国民主同盟和中国国民党的民主分子并没有军队,同时也没有自由。十八年中,在国民党政府统治下的工人、农民、学生以及一切要求进步的文化界、教育界、产业界,他们一概没有军队,同时也一概没有自由。难道是由于上述这些民主党派和人民组织了什么军队,实行了什么“封建割据”,成立了什么“奸区”,违反了什么“政令军令”,因此才不给自由的吗?完全不是。恰恰相反,正是因为他们没有这样做。

“军队是国家的”,非常之正确,世界上没有一个军队不是属于国家的。但是什么国家呢?大地主、大银行家、大买办的封建法西斯独裁的国家,还是人民大众的新民主主义的国家?中国只应该建立新民主主义的国家,并在这个基础之上建立新民主主义的联合政府;中国的一切军队都应该属于这个国家的这个政府,借以保障人民的自由,有效地反对外国侵略者。什么时候中国有一个新民主主义的联合政府出现了,中国解放区的军队将立即交给它。但是一切国民党的军队也必须同时交给它。

一九二四年,孙中山先生说:“今日以后,当划一国民革命之新时代。……第一步使武力与国民相结合;第二步使武力为国民之武力。”\mnote{20}八路军、新四军正是因为实行了这种方针,成了“国民之武力”,就是说,成了人民的军队,所以能打胜仗。国民党军队在北伐战争的前期,做到了孙先生所说的“第一步”,所以打了胜仗。从北伐战争后期直至现在,连“第一步”也丢了,站在反人民的立场上,所以一天一天腐败堕落,除了“内战内行”之外,对于“外战”,就不能不是一个“外行”。国民党军队中一切爱国的有良心的军官们,应该起来恢复孙先生的精神,改造自己的军队。

在改造旧军队的工作中,对于一切可以教育的军官,应当给予适当的教育,帮助他们学得正确观点,清除陈旧观点,为人民的军队而继续服务。

为创造中国人民的军队而奋斗,是全国人民的责任。没有一个人民的军队,便没有人民的一切。对于这个问题,切不可只发空论。

我们共产党人愿意赞助改革中国军队的事业。八路军、新四军对于一切愿意团结人民、反对日本侵略者而不反对中国解放区的军队,都应该看作自己的友军,给以适当的协助。

\subsubsection{第六 土地问题}

为着消灭日本侵略者和建设新中国,必须实行土地制度的改革,解放农民。孙中山先生的“耕者有其田”的主张,是目前资产阶级民主主义性质的革命时代的正确的主张。

为什么把目前时代的革命叫做“资产阶级民主主义性质的革命”?这就是说,这个革命的对象不是一般的资产阶级,而是民族压迫和封建压迫;这个革命的措施,不是一般地废除私有财产,而是一般地保护私有财产;这个革命的结果,将使工人阶级有可能聚集力量因而引导中国向社会主义方向发展,但在一个相当长的时期内仍将使资本主义获得适当的发展。“耕者有其田”,是把土地从封建剥削者手里转移到农民手里,把封建地主的私有财产变为农民的私有财产,使农民从封建的土地关系中获得解放,从而造成将农业国转变为工业国的可能性。因此,“耕者有其田”的主张,是一种资产阶级民主主义性质的主张,并不是无产阶级社会主义性质的主张,是一切革命民主派的主张,并不单是我们共产党人的主张。所不同的,在中国条件下,只有我们共产党人把这项主张看得特别认真,不但口讲,而且实做。哪些人们是革命民主派呢?除了无产阶级是最彻底的革命民主派之外,农民是最大的革命民主派。农民的绝对大多数,就是说,除开那些带上了封建尾巴的富农之外,无不积极地要求“耕者有其田”。城市小资产阶级也是革命民主派,“耕者有其田”使农业生产力获得发展,对于他们是有利的。民族资产阶级是一个动摇的阶级,他们需要市场,他们也赞成“耕者有其田”;他们又多半和土地联系着,他们中的许多人就又惧怕“耕者有其田”。孙中山是中国最早的革命民主派,他代表民族资产阶级的革命派、城市小资产阶级和乡村农民,实行武装革命,提出了“平均地权”和“耕者有其田”的主张。但是可惜,在他掌握政权的时候并没有主动地实行过土地制度的改革。自国民党反人民集团掌握政权以后,便完全背叛了孙中山的主张。现在坚决地反对“耕者有其田”的,正是这个反人民集团,因为他们是代表大地主、大银行家、大买办阶层的。中国没有单独代表农民的政党,民族资产阶级的政党没有坚决的土地纲领,因此,只有制订和执行了坚决的土地纲领、为农民利益而认真奋斗、因而获得最广大农民群众作为自己伟大同盟军的中国共产党,成了农民和一切革命民主派的领导者。

一九二七年至一九三六年,中国共产党实行了彻底改革土地制度的办法,实现了孙先生的“耕者有其田”的主张。出而张牙舞爪,进行了十年反人民战争,亦即反“耕者有其田”的战争的,就是那个集中了孙中山一切不肖子孙在内的团体——国民党内的反人民集团。

抗日期间,中国共产党让了一大步,将“耕者有其田”的政策,改为减租减息的政策。这个让步是正确的,推动了国民党参加抗日,又使解放区的地主减少其对于我们发动农民抗日的阻力。这个政策,如果没有特殊阻碍,我们准备在战后继续实行下去,首先在全国范围内实现减租减息,然后采取适当方法,有步骤地达到“耕者有其田”。

但是背叛孙先生的人们不但反对“耕者有其田”,连减租减息也反对。国民党政府自己颁布的“二五减租”一类的法令,自己不实行,仅仅我们在解放区实行了,因此也就成立了罪状:名之曰“奸区”。

在抗日期间,出现了所谓民族革命阶段和民主民生革命阶段的两阶段论,这是错误的。

大敌当前,民主民生改革的问题不应该提起,等日本人走了再提好了。——这是国民党反人民集团的谬论,其目的是不愿抗日战争获得彻底胜利。有些人居然随声附和,作了这种谬论的尾巴。

大敌当前,不解决民主民生问题,就不能建立抗日根据地抵抗日本的进攻。——这是中国共产党的主张,并且已经这样作了,收到了很好的效果。

在抗日期间,减租减息及其它一切民主改革是为着抗日的。为了减少地主对于抗日的阻力,只实行减租减息,不取消地主的土地所有权,同时又奖励地主的资财向工业方面转移,并使开明士绅和其它人民的代表一道参加抗日的社会工作和政府工作。对于富农,则鼓励其发展生产。所有这些,是在坚决执行农村民主改革的路线里包含着的,是完全必要的。

两条路线:或者坚决反对中国农民解决民主民生问题,而使自己腐败无能,无力抗日;或者坚决赞助中国农民解决民主民生问题,而使自己获得占全人口百分之八十的最伟大的同盟军,借以组织雄厚的战斗力量。前者就是国民党政府的路线,后者就是中国解放区的路线。

动摇于两者之间,口称赞助农民,但不坚决实行减租减息、武装农民和建立农村民主政权,这是机会主义者的路线。

国民党反人民集团动员一切力量,向着中国共产党放出了一切恶毒的箭:明的和暗的,军事的和政治的,流血的和不流血的。两党的争论,就其社会性质说来,实质上是在农村关系的问题上。我们究竟在哪一点上触怒了国民党反人民集团呢?难道不正是在这个问题上面吗?国民党反人民集团之所以受到日本侵略者的欢迎和鼓励,难道不正是在这个问题上面,给日本侵略者帮了大忙吗?所谓“共产党破坏抗战、危害国家”,所谓“奸党”、“奸军”、“奸区”,所谓“不服从政令、军令”,难道不正是因为中国共产党在这个问题上做了真正符合于民族利益的认真的事业吗?

农民——这是中国工人的前身。将来还要有几千万农民进入城市,进入工厂。如果中国需要建设强大的民族工业,建设很多的近代的大城市,就要有一个变农村人口为城市人口的长过程。

农民——这是中国工业市场的主体。只有他们能够供给最丰富的粮食和原料,并吸收最大量的工业品。

农民——这是中国军队的来源。士兵就是穿起军服的农民,他们是日本侵略者的死敌。

农民——这是现阶段中国民主政治的主要力量。中国的民主主义者如不依靠三亿六千万农民群众的援助,他们就将一事无成。

农民——这是现阶段中国文化运动的主要对象。所谓扫除文盲,所谓普及教育,所谓大众文艺,所谓国民卫生,离开了三亿六千万农民,岂非大半成了空话?

我这样说,当然不是忽视其它约占人口九千万的人民在政治上经济上文化上的重要性,尤其不是忽视在政治上最觉悟因而具有领导整个革命运动的资格的工人阶级,这是不应该发生误会的。

认识这一切,不但中国共产党人,而且一切民主派,都是完全必要的。

土地制度获得改革,甚至仅获得初步的改革,例如减租减息之后,农民的生产兴趣就增加了。然后帮助农民在自愿原则下,逐渐地组织在农业生产合作社及其它合作社之中,生产力就会发展起来。这种农业生产合作社,现时还只能是建立在农民个体经济基础上的(农民私有财产基础上的)集体的互助的劳动组织,例如变工队、互助组、换工班之类,但是劳动生产率的提高和生产量的增加,已属惊人。这种制度,已在中国解放区大大发展起来,今后应当尽量推广。

这里应当指出一点,就是说,变工队一类的合作组织,原来在农民中就有了的,但在那时,不过是农民救济自己悲惨生活的一种方法。现在中国解放区的变工队,其形式和内容都起了变化;它成了农民群众为着发展自己的生产,争取富裕生活的一种方法。

中国一切政党的政策及其实践在中国人民中所表现的作用的好坏、大小,归根到底,看它对于中国人民的生产力的发展是否有帮助及其帮助之大小,看它是束缚生产力的,还是解放生产力的。消灭日本侵略者,实行土地改革,解放农民,发展现代工业,建立独立、自由、民主、统一和富强的新中国,只有这一切,才能使中国社会生产力获得解放,才是中国人民所欢迎的。

这里还要指出一点,就是说,从城市到农村工作的知识分子,不容易了解农村现时还是以分散的落后的个体经济为基础的这种特点;在解放区,则还要加上暂时还是被敌人分割的和游击战争的环境的特点。因为不了解这些特点,他们就往往不适当地带着他们在城市里生活或工作的观点去观察农村问题,去处理农村工作,因而脱离农村的实际情况,不能和农民打成一片。这种现象,必须用教育的方法加以克服。

中国广大的革命知识分子应该觉悟到将自己和农民结合起来的必要。农民正需要他们,等待他们的援助。他们应该热情地跑到农村中去,脱下学生装,穿起粗布衣,不惜从任何小事情做起,在那里了解农民的要求,帮助农民觉悟起来,组织起来,为着完成中国民主革命中一项极其重要的工作,即农村民主革命而奋斗。

在日本侵略者被消灭以后,对于日本侵略者和重要汉奸分子的土地应当没收,并分配给无地和少地的农民。

\subsubsection{第七 工业问题}

为着打败日本侵略者和建设新中国,必须发展工业。但是,在国民党政府统治之下,一切依赖外国,它的财政经济政策是破坏人民的一切经济生活的。国民党统治区内仅有的一点小型工业,也不能不处于大部分破产的状态中。政治不改革,一切生产力都遭到破坏的命运,农业如此,工业也是如此。

就整个来说,没有一个独立、自由、民主和统一的中国,不可能发展工业。消灭日本侵略者,这是谋独立。废止国民党一党专政,成立民主的统一的联合政府,使全国军队成为人民的武力,实现土地改革,解放农民,这是谋自由、民主和统一。没有独立、自由、民主和统一,不可能建设真正大规模的工业。没有工业,便没有巩固的国防,便没有人民的福利,便没有国家的富强。一八四〇年鸦片战争\mnote{21}以来的一百零五年的历史,特别是国民党当政以来的十八年的历史,清楚地把这个要点告诉了中国人民。一个不是贫弱的而是富强的中国,是和一个不是殖民地半殖民地的而是独立的,不是半封建的而是自由的、民主的,不是分裂的而是统一的中国,相联结的。在一个半殖民地的、半封建的、分裂的中国里,要想发展工业,建设国防,福利人民,求得国家的富强,多少年来多少人做过这种梦,但是一概幻灭了。许多好心的教育家、科学家和学生们,他们埋头于自己的工作或学习,不问政治,自以为可以所学为国家服务,结果也化成了梦,一概幻灭了。这是好消息,这种幼稚的梦的幻灭,正是中国富强的起点。中国人民在抗日战争中学得了许多东西,知道在日本侵略者被打败以后,有建立一个新民主主义的独立、自由、民主、统一、富强的中国之必要,而这些条件是互相关联的,不可缺一的。果然如此,中国就有希望了。解放中国人民的生产力,使之获得充分发展的可能性,有待于新民主主义的政治条件在全中国境内的实现。这一点,懂得的人已一天一天地多起来了。

在新民主主义的政治条件获得之后,中国人民及其政府必须采取切实的步骤,在若干年内逐步地建立重工业和轻工业,使中国由农业国变为工业国。新民主主义的国家,如无巩固的经济做它的基础,如无进步的比较现时发达得多的农业,如无大规模的在全国经济比重上占极大优势的工业以及与此相适应的交通、贸易、金融等事业做它的基础,是不能巩固的。

我们共产党人愿意协同全国各民主党派,各部分产业界,为上述目标而奋斗。中国工人阶级在这个任务中将起伟大的作用。

中国工人阶级,自第一次世界大战以来,就开始以自觉的姿态,为中国的独立、解放而斗争。一九二一年,产生了它的先锋队——中国共产党,从此以后,使中国的解放斗争进入了新阶段。在北伐战争、土地革命战争和抗日战争三个时期中,中国工人阶级和中国共产党,对于中国人民的解放事业,作了极大的努力和极有价值的贡献。在最后打败日本侵略者的斗争中,特别是在收复大城市和交通要道的斗争中,中国工人阶级将起着极大的作用。在抗日结束以后,可以预断,中国工人阶级的努力和贡献将会是更大的。中国工人阶级的任务,不但是为着建立新民主主义的国家而斗争,而且是为着中国的工业化和农业近代化而斗争。

在新民主主义的国家制度下,将采取调节劳资间利害关系的政策。一方面,保护工人利益,根据情况的不同,实行八小时到十小时的工作制以及适当的失业救济和社会保险,保障工会的权利;另一方面,保证国家企业、私人企业和合作社企业在合理经营下的正当的赢利;使公私、劳资双方共同为发展工业生产而努力。

日本侵略者被打败以后,日本侵略者和重要汉奸分子的企业和财产,应当没收,归政府处理。

\subsubsection{第八 文化、教育、知识分子问题}

民族压迫和封建压迫所给予中国人民的灾难中,包括着民族文化的灾难。特别是具有进步意义的文化事业和教育事业,进步的文化人和教育家,所受灾难,更为深重。为着扫除民族压迫和封建压迫,为着建立新民主主义的国家,需要大批的人民的教育家和教师,人民的科学家、工程师、技师、医生、新闻工作者、著作家、文学家、艺术家和普通文化工作者。他们必须具有为人民服务的精神,从事艰苦的工作。一切知识分子,只要是在为人民服务的工作中着有成绩的,应受到尊重,把他们看作国家和社会的宝贵的财富。中国是一个被民族压迫和封建压迫所造成的文化落后的国家,中国的人民解放斗争迫切地需要知识分子,因而知识分子问题就特别显得重要。而在过去半世纪的人民解放斗争,特别是五四运动\mnote{22}以来的斗争中,在八年抗日战争中,广大革命知识分子对于中国人民解放事业所起的作用,是很大的。在今后的斗争中,他们将起更大的作用。因此,今后人民的政府应有计划地从广大人民中培养各类知识分子干部,并注意团结和教育现有一切有用的知识分子。

从百分之八十的人口中扫除文盲,是新中国的一项重要工作。

一切奴化的、封建主义的和法西斯主义的文化和教育,应当采取适当的坚决的步骤,加以扫除。

应当积极地预防和医治人民的疾病,推广人民的医药卫生事业。

对于旧文化工作者、旧教育工作者和旧医生们的态度,是采取适当的方法教育他们,使他们获得新观点、新方法,为人民服务。

中国国民文化和国民教育的宗旨,应当是新民主主义的;就是说,中国应当建立自己的民族的、科学的、人民大众的新文化和新教育。

对于外国文化,排外主义的方针是错误的,应当尽量吸收进步的外国文化,以为发展中国新文化的借镜;盲目搬用的方针也是错误的,应当以中国人民的实际需要为基础,批判地吸收外国文化。苏联所创造的新文化,应当成为我们建设人民文化的范例。对于中国古代文化,同样,既不是一概排斥,也不是盲目搬用,而是批判地接收它,以利于推进中国的新文化。

\subsubsection{第九 少数民族问题}

国民党反人民集团否认中国有多民族存在,而把汉族以外的各少数民族称之为“宗族”\mnote{23}。他们对于各少数民族,完全继承清朝政府和北洋军阀政府的反动政策,压迫剥削,无所不至。一九四三年对于伊克昭盟蒙族人民的屠杀事件\mnote{24},一九四四年直至现在对于新疆少数民族的武力镇压事件\mnote{25},以及近几年对于甘肃回民的屠杀事件\mnote{26},就是明证。这是大汉族主义的错误的民族思想和错误的民族政策。

一九二四年,孙中山先生在其所著的《中国国民党第一次全国代表大会宣言》里说:“国民党之民族主义,有两方面之意义:一则中国民族自求解放;二则中国境内各民族一律平等。”“国民党敢郑重宣言,承认中国以内各民族之自决权,于反对帝国主义及军阀之革命获得胜利以后,当组织自由统一的(各民族自由联合的)中华民国。”

中国共产党完全同意上述孙先生的民族政策。共产党人必须积极地帮助各少数民族的广大人民群众为实现这个政策而奋斗;必须帮助各少数民族的广大人民群众,包括一切联系群众的领袖人物在内,争取他们在政治上、经济上、文化上的解放和发展,并成立维护群众利益的少数民族自己的军队。他们的言语、文字、风俗、习惯和宗教信仰,应被尊重。

多年以来,陕甘宁边区和华北各解放区对待蒙回两民族的态度是正确的,其工作是有成绩的。

\subsubsection{第十 外交问题}

中国共产党同意大西洋宪章和莫斯科、开罗、德黑兰、克里米亚各次国际会议\mnote{27}的决议,因为这些国际会议的决议都是有利于打败法西斯侵略者和维持世界和平的。

中国共产党的外交政策的基本原则,是在彻底打倒日本侵略者,保持世界和平,互相尊重国家的独立和平等地位,互相增进国家和人民的利益及友谊这些基础之上,同各国建立并巩固邦交,解决一切相互关系问题,例如配合作战、和平会议、通商、投资等等。

中国共产党对于保障战后国际和平安全的机构之建立,完全同意敦巴顿橡树林会议所作的建议和克里米亚会议对这个问题所作的决定。中国共产党欢迎旧金山联合国代表大会。中国共产党已经派遣自己的代表加入中国代表团出席旧金山会议,借以表达中国人民的意志\mnote{28}。

我们认为国民党政府必须停止对于苏联的仇视态度,迅速地改善中苏邦交。苏联是第一个废除不平等条约并和中国订立平等新约的国家。在一九二四年孙中山先生召集的国民党第一次全国代表大会时和在其后进行北伐战争时,苏联是当时唯一援助中国解放战争的国家。在一九三七年抗日战争开始以后,苏联又是第一个援助中国反对日本侵略者的国家。中国人民对于苏联政府和苏联人民的这些援助,表示感谢。我们认为太平洋问题的最后的彻底的解决,没有苏联参加是不可能的。

我们要求各同盟国政府,首先是美英两国政府,对于中国最广大人民的呼声,加以严重的注意,不要使他们自己的外交政策违反中国人民的意志,因而损害同中国人民之间的友谊。我们认为任何外国政府,如果援助中国反动分子而反对中国人民的民主事业,那就将要犯下绝大的错误。

中国人民欢迎许多外国政府宣布废除对于中国的不平等条约,并和中国订立平等新约的措施。但是,我们认为平等条约的订立,并不就表示中国在实际上已经取得真正的平等地位。这种实际上的真正的平等地位,决不能单靠外国政府的给予,主要地应靠中国人民自己努力争取,而努力之道就是把中国在政治上经济上文化上建设成为一个新民主主义的国家,否则便只会有形式上的独立、平等,在实际上是不会有的。就是说,依据国民党政府的现行政策,决不会使中国获得真正的独立和平等。

我们认为在日本侵略者被打败并无条件投降之后,为着彻底消灭日本的法西斯主义、军国主义及其所由产生的政治、经济、社会的原因,必须帮助一切日本人民的民主力量建立日本人民的民主制度。没有日本人民的民主制度,便不能彻底地消灭日本法西斯主义和军国主义,便不能保证太平洋的和平。

我们认为开罗会议关于朝鲜独立的决定是正确的,中国人民应当帮助朝鲜人民获得解放。

我们希望印度独立。因为一个独立的民主的印度,不但是印度人民的需要,也是世界和平的需要。

对于南洋各国——缅甸、马来亚、印度尼西亚、越南、菲律宾,我们希望这些国家的人民在日本侵略者被打败以后,能够得到建立独立的民主的国家制度的权利。对于泰国,应当仿照对待欧洲法西斯附属国的方法去处理。

关于具体纲领的说明,主要的就是这样。

再说一遍,一切这些具体纲领,如果没有一个举国一致的民主的联合政府,就不可能顺利地在全中国实现。

中国共产党在其为中国人民的解放事业而奋斗的二十四年中,创造了这样的地位,就是说,不论什么政党或社会集团,也不论是中国人或外国人,在有关中国的问题上,如果采取不尊重中国共产党的意见的态度,那是极其错误而且必然要失败的。过去和现在都有这样的人,企图孤行己见,不尊重我们的意见,但是结果都行不通。这是什么缘故呢?不是别的,就是因为我们的意见,符合于最广大的中国人民的利益。中国共产党是中国人民的最忠实的代言人,谁要是不尊重中国共产党,谁就是在实际上不尊重最广大的中国人民,谁就一定要失败。

\subsection{中国国民党统治区的任务}

关于我党的一般纲领和具体纲领,我已在上面作了充分的说明。无疑地,这些纲领是要在全中国实行的;整个国际国内的形势,给中国人民展开了这种想望。但是,目前在国民党统治区,在沦陷区,在解放区,这三种地方互不相同的情势,不能不使我们在实行时要有所区别。不同的情形,产生不同的任务。这些任务,有些我已经在前面说到了,有些还须在下面加以补充。

在国民党统治区,人民没有爱国活动的自由,民主运动被认为非法,但是包括许多阶层、许多民主党派和民主分子的积极活动是在发展中。中国民主同盟,在今年一月发表了要求结束国民党一党专政和成立联合政府的宣言。社会各界发表同类性质的宣言的,还有许多。国民党内也有许多人,对于他们自己的领导机关的政策,日益表示怀疑和不满,日益感觉他们的党在广大人民中孤立起来的危险性,而要求有一种适合时宜的民主的改革。重庆等地的工人、农民、文化界、学生界、教育界、妇女界、工商界、公务人员乃至一部分军人的民主运动,正在发展。所有这些,预示着一切受压迫阶层的民主运动正在逐渐地向着同一的目标而汇合起来。目前运动的弱点,在于社会的基层分子还没有广泛地参加,地位非常重要而生活痛苦不堪的农民、工人、士兵和下层公教人员,还没有组织起来。目前运动的另一弱点,是参加运动的民主分子中,还有许多人对于根据民主原则发动斗争以求转变时局这一个基本方针,还缺乏明确的和坚决的精神。但是客观形势,正在迫着一切受压迫的阶层、党派和社会集团,逐渐地觉悟和团结起来。不管国民党政府如何镇压,也不能阻止这一运动的发展。

国民党统治区内被压迫的一切阶层、党派和集团的民主运动,应当有一个广大的发展,并把分散的力量逐渐统一起来,为着实现民族团结,建立联合政府,打败日本侵略者和建设新中国而斗争。中国共产党和解放区人民,应当给予他们以一切可能的援助。

在国民党统治区,共产党人应当继续执行广泛的抗日民族统一战线政策。不管什么人,哪怕昨天还是反对我们的,只要他今天不反对了,就应该同他合作,为共同的目标而奋斗。

\subsection{中国沦陷区的任务}

在沦陷区,共产党人应当号召一切抗日人民,学习法国和意大利的榜样,将自己组织于各种团体中,组织地下军,准备武装起义,一俟时机成熟,配合从外部进攻的军队,里应外合地消灭日本侵略者。日本侵略者及其忠实走狗,对于我沦陷区内的兄弟姊妹们的摧残、掠夺、奸淫和侮辱,激起了一切中国人的火一样的愤怒,报仇雪耻的时机快要到来了。沦陷区的人民,在欧洲战场的胜利和八路军新四军的胜利的鼓舞之下,极大地增高了他们的抗日情绪。他们迫切地需要组织起来,以便尽可能迅速地获得解放。因此,我们必须将沦陷区的工作提到和解放区的工作同等重要的地位上。必须有大批工作人员到沦陷区去工作。必须就沦陷区人民中训练和提拔大批的积极分子,参加当地的工作。在沦陷区中,东北四省沦陷最久,又是日本侵略者的产业中心和屯兵要地,我们应当加紧那里的地下工作。对于流亡到关内的东北人民,应当加紧团结他们,准备收复失地。

在一切沦陷区,共产党人应当执行最广泛的抗日民族统一战线政策。不管什么人,只要是反对日本侵略者及其忠实走狗的,就要联合起来,为打倒共同敌人而斗争。

应当向一切帮助敌人反对同胞的伪军伪警及其它人员提出警告:他们必须赶快认识自己的罪恶行为,及时回头,帮助同胞反对敌人,借以赎回自己的罪恶。否则,敌人崩溃之日,民族纪律是不会宽容他们的。

共产党人应当向一切有群众的伪组织进行争取说服工作,使被欺骗的群众站到反对民族敌人的战线上来。同时,对于那些罪大恶极不愿改悔的汉奸分子进行调查工作,以便在国土收复之后,依法惩治他们。

对于国民党内组织汉奸反对中国人民、中国共产党、八路军、新四军和其它人民军队的背叛民族的反动分子,必须向他们提出警告,叫他们早日悔罪。否则,在国土收复之后,必然要将他们和汉奸一体治罪,决不宽饶。

\subsection{中国解放区的任务}

我党的全部新民主主义的纲领已经在解放区实行了并且有了显着的成绩,聚集了巨大的抗日力量,今后应当从各方面发展和巩固这种力量。

在目前条件下,解放区的军队应向一切被敌伪占领而又可能攻克的地方,发动广泛的进攻,借以扩大解放区,缩小沦陷区。

但是同时应当注意,敌人在目前还是有力量的,它还可能向解放区发动进攻。解放区军民必须随时准备粉碎敌人的进攻,并注意解放区的各项巩固工作。

应当扩大解放区的军队、游击队、民兵和自卫军,并加紧整训,增强战斗力,为最后打败侵略者准备充分的力量。

在解放区,一方面,军队应实行拥政爱民的工作,另一方面,民主政府应领导人民实行拥军优抗的工作,更大地改善军民关系。

共产党人在各个地方性的联合政府的工作中,在社会工作中,应当继续同一切抗日民主分子,在新民主主义纲领的基础上,进行很好的合作。

同样,在军事工作中,共产党人应当同一切愿意和我们合作的抗日民主分子,在解放区军队的内部和外部,很好地合作。

为了提高工农劳动群众在抗日和生产中的积极性,减租减息和改善工人、职员待遇的政策,必须充分地执行。解放区的工作人员,必须努力学会做经济工作。必须动员一切可能的力量,大规模地发展解放区的农业、工业和贸易,改善军民生活。为此目的,必须实行劳动竞赛,奖励劳动英雄和模范工作者。在城市驱逐日本侵略者以后,我们的工作人员,必须迅速学会做城市的经济工作。

为着提高解放区人民大众首先是广大的工人、农民、士兵群众的觉悟程度和培养大批工作干部,必须发展解放区的文化教育事业。解放区的文化工作者和教育工作者在推进他们的工作时,应当根据目前的农村特点,根据农村人民的需要和自愿的原则,采用适宜的内容和形式。

在推进解放区的各项工作时,必须十分爱惜当地的人力物力,任何地方都要作长期打算,避免滥用和浪费。这不但是为着打败日本侵略者,而且是为着建设新中国。

在推进解放区的各项工作时,必须十分注意扶助本地人管理本地的事业,必须十分注意从本地人民优秀分子中大批地培养本地的工作干部。一切从外地去的人,如果不和本地人打成一片,如果不是满腔热情地勤勤恳恳地并适合情况地去帮助本地干部,爱惜他们,如同爱惜自己的兄弟姊妹一样,那就不能完成农村民主革命这个伟大的任务。

八路军、新四军及其它人民军队,每到一地,就应立即帮助本地人民,不但要组织以本地人民的干部为领导的民兵和自卫军,而且要组织以本地人民的干部为领导的地方部队和地方兵团。然后,就可以产生有本地人领导的主力部队和主力兵团。这是一项非常重要的任务。如果不能完成此项任务,就不能建立巩固的抗日根据地,也不能发展人民的军队。

当然,一切本地人,应当热烈地欢迎和帮助从外地去的革命工作人员和人民军队。

关于对待暗藏的民族破坏分子的问题,必须提起大家的注意。因为公开的敌人,公开的民族破坏分子,容易识别,也容易处置;暗藏的敌人,暗藏的民族破坏分子,就不容易识别,也就不容易处置。因此,对于这后一种人必须采取严肃态度,而在处理时又要采取谨慎态度。

根据信教自由的原则,中国解放区容许各派宗教存在。不论是基督教、天主教、回教、佛教及其它宗教,只要教徒们遵守人民政府法律,人民政府就给以保护。信教的和不信教的各有他们的自由,不许加以强迫或歧视。

我们的大会应向各解放区人民提议,尽可能迅速地在延安召开中国解放区人民代表会议\mnote{29},以便讨论统一各解放区的行动,加强各解放区的抗日工作,援助国民党统治区人民的抗日民主运动,援助沦陷区人民的地下军运动,促进全国人民的团结和联合政府的成立。中国解放区现在已经成了全国广大人民抗日救国的重心,全国广大人民的希望寄托在我们身上,我们有责任不要使他们失望。中国解放区人民代表会议的召集,将对中国人民的民族解放事业起一个巨大的推进作用。

\section{五 全党团结起来,为实现党的任务而斗争}

同志们,我们已经了解了我们的任务和我们为完成这些任务所采取的政策,那末,我们应该用怎样的工作态度去执行这些政策和完成这些任务呢?

目前国际国内的形势,在我们和中国人民面前显示了光明的前途,具备了前所未有的有利条件,这是显然的,毫无疑义的。但是同时,依然存在着严重的困难条件。谁要是只看见光明一面,不看见困难一面,谁就会不能很好地为实现党的任务而斗争。

我们的党和中国人民一道,不论在整个党的二十四年历史中,在八年抗日战争中,为中国人民创造了巨大的力量,我们的工作成绩是很显然的,毫无疑义的。但是同时,我们的工作中依然存在着缺点。谁要是只看见成绩一面,不看见缺点一面,谁也就不会很好地为实现党的任务而斗争。

中国共产党自从一九二一年诞生以来,在其二十四年的历史中,经历了三次的伟大斗争,这就是北伐战争、土地革命战争和现在还在进行中的抗日战争。我们的党从它一开始,就是一个以马克思列宁主义的理论为基础的党,这是因为这个主义是全世界无产阶级的最正确最革命的科学思想的结晶。马克思列宁主义的普遍真理一经和中国革命的具体实践相结合,就使中国革命的面目为之一新,产生了新民主主义的整个历史阶段。以马克思列宁主义的理论思想武装起来的中国共产党,在中国人民中产生了新的工作作风,这主要的就是理论和实践相结合的作风,和人民群众紧密地联系在一起的作风以及自我批评的作风。

反映了全世界无产阶级实践斗争的马克思列宁主义的普遍真理,在它同中国无产阶级和广大人民群众的革命斗争的具体实践相结合的时候,就成为中国人民百战百胜的武器。中国共产党正是这样做了。我们党的发展和进步,是从同一切违反这个真理的教条主义和经验主义作坚决斗争的过程中发展和进步起来的。教条主义脱离具体的实践,经验主义把局部经验误认为普遍真理,这两种机会主义的思想都是违背马克思主义的。我们党在自己的二十四年奋斗中,克服了和正在克服着这些错误思想,使得我们的党在思想上极大地巩固了。我们党现在已有了一百二十一万党员。其中绝大多数是在抗日时期入党的,在他们之中存在着各种不纯正的思想。在抗日以前入党的党员中,也有这种情形。几年来的整风工作收到了巨大的成效,使这些不纯正的思想受到了很多的纠正。今后应当继续这种工作,以“惩前毖后、治病救人”的精神,更大地展开党内的思想教育。必须使各级党的领导骨干都懂得,理论和实践这样密切地相结合,是我们共产党人区别于其它任何政党的显着标志之一。因此,掌握思想教育,是团结全党进行伟大政治斗争的中心环节。如果这个任务不解决,党的一切政治任务是不能完成的。

我们共产党人区别于其它任何政党的又一个显着的标志,就是和最广大的人民群众取得最密切的联系。全心全意地为人民服务,一刻也不脱离群众;一切从人民的利益出发,而不是从个人或小集团的利益出发;向人民负责和向党的领导机关负责的一致性;这些就是我们的出发点。共产党人必须随时准备坚持真理,因为任何真理都是符合于人民利益的;共产党人必须随时准备修正错误,因为任何错误都是不符合于人民利益的。二十四年的经验告诉我们,凡属正确的任务、政策和工作作风,都是和当时当地的群众要求相适合,都是联系群众的;凡属错误的任务、政策和工作作风,都是和当时当地的群众要求不相适合,都是脱离群众的。教条主义、经验主义、命令主义、尾巴主义、宗派主义、官僚主义、骄傲自大的工作态度等项弊病之所以一定不好,一定要不得,如果什么人有了这类弊病一定要改正,就是因为它们脱离群众。我们的代表大会应该号召全党提起警觉,注意每一个工作环节上的每一个同志,不要让他脱离群众。教育每一个同志热爱人民群众,细心地倾听群众的呼声;每到一地,就和那里的群众打成一片,不是高踞于群众之上,而是深入于群众之中;根据群众的觉悟程度,去启发和提高群众的觉悟,在群众出于内心自愿的原则之下,帮助群众逐步地组织起来,逐步地展开为当时当地内外环境所许可的一切必要的斗争。在一切工作中,命令主义是错误的,因为它超过群众的觉悟程度,违反了群众的自愿原则,害了急性病。我们的同志不要以为自己了解了的东西,广大群众也和自己一样都了解了。群众是否已经了解并且是否愿意行动起来,要到群众中去考察才会知道。如果我们这样做,就可以避免命令主义。在一切工作中,尾巴主义也是错误的,因为它落后于群众的觉悟程度,违反了领导群众前进一步的原则,害了慢性病。我们的同志不要以为自己还不了解的东西,群众也一概不了解。许多时候,广大群众跑到我们的前头去了,迫切地需要前进一步了,我们的同志不能做广大群众的领导者,却反映了一部分落后分子的意见,并且将这种落后分子的意见误认为广大群众的意见,做了落后分子的尾巴。总之,应该使每个同志明了,共产党人的一切言论行动,必须以合乎最广大人民群众的最大利益,为最广大人民群众所拥护为最高标准。应该使每一个同志懂得,只要我们依靠人民,坚决地相信人民群众的创造力是无穷无尽的,因而信任人民,和人民打成一片,那就任何困难也能克服,任何敌人也不能压倒我们,而只会被我们所压倒。

有无认真的自我批评,也是我们和其它政党互相区别的显着的标志之一。我们曾经说过,房子是应该经常打扫的,不打扫就会积满了灰尘;脸是应该经常洗的,不洗也就会灰尘满面。我们同志的思想,我们党的工作,也会沾染灰尘的,也应该打扫和洗涤。“流水不腐,户枢不蠹”,是说它们在不停的运动中抵抗了微生物或其它生物的侵蚀。对于我们,经常地检讨工作,在检讨中推广民主作风,不惧怕批评和自我批评,实行“知无不言,言无不尽”,“言者无罪,闻者足戒”,“有则改之,无则加勉”这些中国人民的有益的格言,正是抵抗各种政治灰尘和政治微生物侵蚀我们同志的思想和我们党的肌体的唯一有效的方法。以“惩前毖后,治病救人”为宗旨的整风运动之所以发生了很大的效力,就是因为我们在这个运动中展开了正确的而不是歪曲的、认真的而不是敷衍的批评和自我批评。以中国最广大人民的最大利益为出发点的中国共产党人,相信自己的事业是完全合乎正义的,不惜牺牲自己个人的一切,随时准备拿出自己的生命去殉我们的事业,难道还有什么不适合人民需要的思想、观点、意见、办法,舍不得丢掉的吗?难道我们还欢迎任何政治的灰尘、政治的微生物来玷污我们的清洁的面貌和侵蚀我们的健全的肌体吗?无数革命先烈为了人民的利益牺牲了他们的生命,使我们每个活着的人想起他们就心里难过,难道我们还有什么个人利益不能牺牲,还有什么错误不能抛弃吗?

同志们,我们的大会闭幕之后,我们就要上战场去,根据大会的决议,为着最后地打败日本侵略者和建设新中国而奋斗。为达此目的,我们要和全国人民团结起来。我重说一遍,不管什么阶级,什么政党,什么社会集团或个人,只要是赞成打败日本侵略者和建设新中国的,我们就要加以联合。为达此目的,我们要把我们党的一切力量在民主集中制的组织和纪律的原则之下,坚强地团结起来。不论什么同志,只要他是愿意服从党纲、党章和党的决议的,我们就要和他团结。我们的党,在北伐战争时期,不超过六万党员,后来大部分被当时的敌人打散了;在土地革命战争时期,不超过三十万党员,后来大部分也被当时的敌人打散了。现在我们有了一百二十多万党员,这一回无论如何不要被敌人打散。只要我们能吸取三个时期的经验,采取谦虚态度,防止骄傲态度,在党内,和全体同志更好地团结起来,在党外,和全国人民更好地团结起来,就可以保证,不但不会被敌人打散,相反地,一定要把日本侵略者及其忠实走狗坚决、彻底、干净、全部地消灭掉,并且在消灭他们之后,把一个新民主主义的中国建设起来。

三次革命的经验,尤其是抗日战争的经验,给了我们和中国人民这样一种信心:没有中国共产党的努力,没有中国共产党人做中国人民的中流砥柱,中国的独立和解放是不可能的,中国的工业化和农业近代化也是不可能的。

同志们,有了三次革命经验的中国共产党,我坚决相信,我们是能够完成我们的伟大政治任务的。

成千成万的先烈,为着人民的利益,在我们的前头英勇地牺牲了,让我们高举起他们的旗帜,踏着他们的血迹前进吧!

一个新民主主义的中国不久就要诞生了,让我们迎接这个伟大的日子吧!


\begin{maonote}
\mnitem{1}一九三一年九月十八日,日本驻在中国东北境内的“关东军”进攻沈阳,九月十九日晨占领了沈阳城。
\mnitem{2}热河,原来是一个省,一九五五年撤销,原辖地区划归河北、辽宁两省和内蒙古自治区。察哈尔,原来也是一个省,一九五二年撤销,原辖地区划归河北、山西两省。
\mnitem{3}参见本书第一卷\mxnote{中国社会各阶级的分析}{4}。
\mnitem{4}见本书第二卷\mxnote{战争和战略问题}{11}。
\mnitem{5}见本书第二卷\mxnote{战争和战略问题}{2}。
\mnitem{6}中华民族解放先锋队,简称“民先队”,是一二九运动中的先进青年在中国共产党领导下所组织的革命青年团体,成立于一九三六年二月。抗日战争爆发后,许多民先队员参加了战争和建立抗日根据地的工作。国民党统治地区的民先队组织,一九三八年被国民党政府强迫解散。在抗日根据地的民先队组织,后来并入更广泛的青年团体青年救国会。
\mnitem{7}见本书第二卷\mxnote{必须制裁反动派}{5}。
\mnitem{8}参见本卷\mxart{评国民党十一中全会和三届二次国民参政会}一文中关于国民党发动三次反共高潮的叙述。
\mnitem{9}“曲线救国”,是抗日战争时期国民党内一些顽固派分子为实行降日反共而制造的一种叛国谬论。他们指使或支持一部分国民党军队和官员投降日本侵略者,变成伪军、伪官,和日军一起进攻抗日根据地,并将这种叛国投敌行为诡称为“曲线救国”。
\mnitem{10}绥远,原来是一个省,一九五四年撤销,原辖地区划归内蒙古自治区。
\mnitem{11}见本书第一卷\mxnote{论反对日本帝国主义的策略}{5}。
\mnitem{12}斯科比是英国派驻希腊的英军司令。一九四四年十月,德国侵略军在希腊败退。斯科比率领英军,带着在伦敦的希腊流亡政府,进入希腊。同年十二月,斯科比指挥英军并协助希腊政府进攻长期英勇抵抗德军的希腊人民解放军,屠杀希腊爱国人民。
\mnitem{13}保甲制度是国民党政府实行法西斯统治的基层政治制度。一九三二年八月,蒋介石在河南、湖北、安徽三省颁布《各县编查保甲户口条例》,其中规定“保甲之编组以户为单位,户设户长,十户为甲,甲设甲长,十甲为保,保设保长”,实行各户互相监视和互相告发的联保连坐法,以及各项反革命的强迫劳役办法。一九三四年十一月七日,国民党政府正式决定在它所统治的各省市一律推行这种保甲制度。
\mnitem{14}这里是指国民党政府所实行的封建的买办的法西斯教育。
\mnitem{15}开罗会议是一九四三年十一月中、美、英三国首脑在埃及首都开罗举行的一次国际会议。这次会议发表了中、美、英三国开罗宣言,明确规定日本必须无条件投降,并将侵占的中国领土台湾等地归还中国。
\mnitem{16}民主同盟成立于一九四一年,当时名中国民主政团同盟,一九四四年改组为中国民主同盟。
\mnitem{17}见本书第一卷\mxnote{论反对日本帝国主义的策略}{1}。
\mnitem{18}峨眉山是四川省西南部的名山。毛泽东的这句话是说国民党统治集团在抗日战争中最后撤退到四川山地。
\mnitem{19}参见本书第二卷\mxart{必须制裁反动派}、\mxart{团结一切抗日力量,反对反共顽固派}、\mxart{向国民党的十点要求}等文。
\mnitem{20}见一九二四年十一月十日孙中山的《北上宣言》(《孙中山全集》第11卷,中华书局1986年版,第296—297页)。
\mnitem{21}见本书第一卷\mxnote{论反对日本帝国主义的策略}{35}。
\mnitem{22}见本书第一卷\mxnote{实践论}{6}。
\mnitem{23}这是指蒋介石在《中国之命运》中的一种错误说法。
\mnitem{24}一九四二年冬至一九四三年春,驻内蒙伊克昭盟的国民党反动军队,强行霸占蒙族人民的牧地,并且向当地人民勒索大量粮食、牲畜。一九四三年三月二十六日,伊克昭盟的蒙族保安队和人民群众被迫发动了武装反抗。四月,国民党军队前往镇压,对当地的蒙族人民进行了血腥的屠杀。
\mnitem{25}一九四四年九月,新疆北部伊犁地区的少数民族人民,为了反对国民党反动派的民族压迫和经济掠夺,发动了声势浩大的武装起义。这次起义先后同阿山、塔城一带的起义武装联合,占领了新疆北部的大部分地区,所以又被称为“三区革命”。国民党反动派从甘肃和新疆各地调集了大批军队,对起义军实行长期的大规模的武力镇压。起义军在新疆各族广大人民的积极支持下,进行了英勇的抵抗,一直坚持到一九四九年新疆解放。
\mnitem{26}这里是指一九四二年、一九四三年国民党反动派对甘肃省南部回、汉、藏等族起义农民的屠杀事件。一九四二年冬,甘肃省南部临洮、康乐一带的农民,为了反对国民党反动派的横征暴敛、抓兵抓夫等反动措施,在回民马福善等率领下,发动了大规模的武装起义。到一九四三年四月以后,起义地区发展到二十多县,参加人数最多时达到十万多人。国民党反动派先后调动了七个师以上的军队,甚至出动飞机,配合地方武装,对起义的群众进行残酷的屠杀。
\mnitem{27}大西洋宪章是一九四一年八月美英大西洋会议结束时联合发表的一个文件。莫斯科会议是一九四三年十月苏、美、英三国外长在莫斯科举行的会议。德黑兰会议是一九四三年十一月至十二月苏、美、英三国首脑在伊朗首都德黑兰举行的会议。克里米亚会议是一九四五年二月苏、美、英三国首脑在苏联南部克里米亚半岛雅尔塔举行的会议。当时所有这些国际会议都决定以联合的力量击败法西斯德国和日本,并在战后防止侵略势力和法西斯残余的再起,维护世界和平,赞助各国人民的独立民主的愿望。
\mnitem{28}一九四四年八月至十月,苏、美、英、中四国代表按照莫斯科会议和德黑兰会议的决定,在美国首都华盛顿郊区的敦巴顿橡树园举行会议,草拟了联合国机构的组织草案。一九四五年四月至六月,在美国旧金山召开了有五十个国家代表参加的联合国成立大会。当时中国解放区派遣董必武为代表加入中国代表团,参加了这次会议。
\mnitem{29}中国共产党第七次全国代表大会以后,一九四五年七月十三日,各解放区、各人民团体以及八路军、新四军等各方面的代表,曾在延安开会,成立“中国解放区人民代表会议筹备委员会”。日本投降以后,因为时局变化,中国解放区人民代表会议没有召开。
\end{maonote}
