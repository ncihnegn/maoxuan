
\title{关于“三反”、“五反”的斗争}
\date{一九五一年十一月——一九五二年三月}
\thanks{这是毛泽东同志为中共中央起草的一些重要指示。}
\maketitle


\date{一九五一年十一月三十日}
\section*{一}

反贪污、反浪费一事,实是全党一件大事,我们已告诉你们严重地注意此事。我们认为需要来一次全党的大清理,彻底揭露一切大、中、小贪污事件,而着重打击大贪污犯,对中小贪污犯则取教育改造不使重犯的方针,才能停止很多党员被资产阶级所腐蚀的极大危险现象,才能克服二中全会所早已料到的这种情况,并实现二中全会防止腐蚀的方针,务请你们加以注意。

\date{一九五一年十一月三十日}
\section*{二}

必须严重地注意干部被资产阶级腐蚀发生严重贪污行为这一事实,注意发现、揭露和惩处,并须当作一场大斗争来处理。

\date{一九五一年十二月八日}
\section*{三}

应把反贪污、反浪费、反官僚主义的斗争看作如同镇压反革命的斗争一样的重要,一样的发动广大群众包括民主党派及社会各界人士去进行,一样的大张旗鼓去进行,一样的首长负责,亲自动手,号召坦白和检举,轻者批评教育,重者撤职、惩办、判处徒刑(劳动改造),直至枪毙一批最严重的贪污犯,才能解决问题。

\date{一九五二年一月二十六日}
\section*{四}

在全国一切城市,首先在大城市和中等城市中,依靠工人阶级,团结守法的资产阶级及其它市民,向着违法的资产阶级开展一个大规模的坚决的彻底的反对行贿、反对偷税漏税、反对盗骗国家财产、反对偷工减料和反对盗窃经济情报的斗争,以配合党政军民内部的反对贪污、反对浪费、反对官僚主义的斗争,现在是极为必要和极为适时的。在这个斗争中,各城市的党组织对于阶级和群众的力量必须作精密的部署,必须注意利用矛盾、实行分化、团结多数、孤立少数的策略,在斗争中迅速形成“五反”的统一战线。这种统一战线,在一个大城市中,在猛烈展开“五反”之后,大约有三个星期就可以形成。只要形成了这个统一战线,那些罪大恶极的反动资本家就会陷于孤立,国家就能很有理由地和顺利地给他们以各种必要的惩处,例如逮捕、徒刑、枪决、没收、罚款等等。全国各大城市(包括各省城)在二月上旬均应进入“五反”战斗,请你们速作部署。

\date{一九五二年三月五日}
\section*{五}

一、在“五反”运动中对工商户处理的基本原则是:过去从宽,今后从严(例如补税一般只补一九五一年的);多数从宽,少数从严;坦白从宽,抗拒从严;工业从宽,商业从严;普通商业从宽,投机商业从严。望各级党委在“五反”中掌握这几条原则。

二、在“五反”目标下划分私人工商户的类型,应分为守法的、基本守法的、半守法半违法的、严重违法的和完全违法的五类。就大城市说,前三类约占百分之九十五左右,后二类约占百分之五左右。各个大城市略有出入,大体相差不远。中等城市则和这个比例数字相差较大。

三、这五类包括资产阶级和非资产阶级的独立手工业户及家庭商业户,不包括摊贩。各大城市可以暂时不去处理摊贩,但对独立手工业户和家庭商业户最好给以处理。各中等城市在“五反”中最好对独立工商户及摊贩均给以处理。不雇工人、店员(但有些人家带了学徒)的独立工商户在我国各大中城市数目很大,他们中许多是守法的,也有许多是基本守法部分违法的(即有小额偷漏税,即所谓有小问题的),也有少数是属于半守法半违法即偷漏税较大的。我们既要在此次“五反”中处理一大批小资本家,给他们做出结论,也应尽可能努力将和小资本家数目大略相等的独立工商户加以处理,给他们做出结论,这对于目前的“五反”和今后的经济建设都是有利的。这两种工商户一般都无大问题,给他们做结论是不困难的。做了结论以后,我们就获得了广大群众的拥护。但个别城市如认为先给其它工商户做结论,而将独立工商户的结论放在后面去做较为方便,也是可以的。

四、根据城市的实际情况,我们决定将过去所定的四类工商户改为五类,即将守法户一类改为守法户和基本守法户两类,其它三类不变。在北京五万工商户(包括独立工商户,不包括摊贩)中,守法户约占百分之十左右,基本守法户约占百分之六十左右,半守法半违法户约占百分之二十五左右,严重违法户约占百分之四左右,完全违法户约占百分之一左右。将完全守法户和有小问题的基本守法户分开,又将基本守法户中偷漏税较少的和偷漏税稍多的分别对待,这样做,可能发生很大的教育作用。

五、各大中城市中,有些市委,对于各类工商户的情况极不明了,如何分别对待这些工商户的策略观点又不明确,工会和政府工作队(或检查组)的组织和训练甚为潦草,便仓卒发动“五反”,引起了一些混乱,望这些市委提起注意,迅速加以克服。此外,检查违法工商户,必须由市委市政府予以严密控制,各机关不得自由派人检查,更不得随便捉资本家到机关来审讯。又无论“三反”、“五反”,均不得采用肉刑逼供方法,严防自杀现象发生,已发生者立即订出防止办法,务使“三反”、“五反”均按正轨健全发展,争取完满胜利。

六、县、区、乡现在一律不进行“三反”、“五反”,将来何时进行及如何进行,中央另有通知。个别已在县城试做“五反”、在区试做“三反”者务须严格控制,不得妨碍春耕和经济活动。中等城市也不要同时一律进行“五反”,而要分批进行,并须在严格控制下进行。


\date{一九五二年三月二十三日}
\section*{六}

在此次“五反”斗争中及其以后,我们必须达到下述目的:

(一)彻底查明私人工商业的情况,以利团结和控制资产阶级,进行国家的计划经济。情况不明,是无法进行计划经济的。

(二)明确划分工人阶级和资产阶级的界限,肃清工会中的贪污现象和脱离群众的官僚主义现象,清除资产阶级在工会中的走狗。各地工会均发生此种走狗及动摇于劳资之间的中间分子,我们必须在斗争中教育并争取中间分子,对于有严重罪行的资本家走狗则予以开除。

(三)改组同业公会和工商联合会,开除那些“五毒”\mnote{1}俱全的人们及其它业已完全丧失威信的人们出这些团体的领导机关,吸引那些在“五反”中表现较好的人们进来。除完全违法者外,各类工商业者均应有代表。

(四)帮助民主建国会的负责人整顿民主建国会,开除那些“五毒”俱全的人及大失人望的人,增加一批较好的人,使之成为一个能够代表资产阶级主要是工业资产阶级的合法利益,并以共同纲领和“五反”的原则教育资产阶级的政治团体。各部分资本家的秘密结社,例如“星四聚餐会”\mnote{2}等,则应设法予以解散。

(五)清除“五毒”,消灭投机商业,使整个资产阶级服从国家法令,经营有益于国计民生的工商业;在国家划定的范围内,尽量发展私人工业(只要资本家愿意和合乎《共同纲领》),逐步缩小私人商业;国家逐年增加对私营产品的包销订货计划,逐年增加对私营工商业的计划性;重新划定私资利润额,既要使私资感觉有利可图,又要使私资无法夺取暴利。

(六)废除后账,经济公开,逐步建立工人、店员监督生产和经营的制度。

(七)从补、退、罚、没中追回国家及人民的大部分经济损失。

(八)在一切大的和中等的私营企业的工人、店员中建立党的支部,加强党的工作。


\begin{maonote}
\mnitem{1}“五毒”,指资本家的行贿、偷税漏税、盗骗国家财产、偷工减料和盗窃经济情报五种违法行为。
\mnitem{2}“星四聚餐会”,是重庆一些资本家的秘密结社,它进行了一系列严重违法的地下活动,在“五反”运动中被揭发和取缔。
\end{maonote}
