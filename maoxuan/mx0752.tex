
\title{与黎笋的谈话}
\date{一九七〇年五月十一日}
\thanks{这是毛泽东同志在会见越共第一书记黎笋\mnote{1}时的主要谈话。}
\maketitle


\section*{(一)}

\mxsay{毛泽东:}我上一次是什么时候见到你的?

\mxsay{黎笋:}一九六四年。我们看到毛主席身体健康,感到很高兴。目前,越南和印度支那的形势比较复杂,也存在一些困难。

\mxsay{毛泽东:}困难哪一国都有,苏联也有,美国也有。

\mxsay{黎笋:}我们很需要得到毛主席的指示。如果我们中央和政治局知道毛主席给我们的工作提意见的话,一定会非常高兴。

\mxsay{毛泽东:}你们的工作做得很好,而且你们是越做越好。

\mxsay{黎笋:}我们尽最大的努力做工作。我们所以能把工作做好,也是按照毛主席以前对我们讲到的三点去做:第一,不怕,不怕敌人;第二,要各个击破;第三,要长期地打。

\mxsay{毛泽东:}对,是持久战。你们要准备打持久战,但是如果打短一些不是更好吗?

究竟谁怕谁呢?是你们越南人、柬埔寨人以及东南亚人怕美帝国主义,还是美帝国主义怕你们呢?这是个问题,值得考虑,值得研究。还是大国怕小国,风吹草动他都惊慌失措。一九六四年北部湾事件\mnote{2}你们是整了他一下,但也不是有意想打美国海军。实际上,你们并不是真的打他,但他们自己紧张起来了,说越南的鱼雷艇来了,开炮就打。后来美国人自己也讲不出来究竟是真的还是假的。美国各地的新闻记者肯定那是假的,是场虚惊。既然打起来了,那就只好打了。对军火商是有利的。(从那时起)美国的总统每天晚上睡觉很少。尼克松\mnote{3}自己讲,他的主要的精力是对付越南。

现在还有另外一个人,西哈努克\mnote{4}亲王,他也不好惹,你惹他,他就要跳出来骂娘。

我们有些大使馆,我看要整顿一下。我们中国一些大使馆有大国沙文主义,尽是把人家的缺点看得多,不顾大局。(中国)驻越南的上任大使是谁?

\mxsay{周恩来:}朱其文。

\mxsay{毛泽东:}朱其文跟你们搞得很僵。事实上朱其文是个国民党,他要跑到外国去。我们不知道他是国民党,你们跟国民党打交道,他为什么不捣你们的乱呀?那时我们也不知道,但我们看到(他发回的)那些电报是不高兴的。

\mxsay{黎笋:}我们越南人民心中牢记着毛主席的恩情。在九年的抗法战争中,如果没有中国共产党和毛主席的支持,我们是不可能取得胜利的。我们为什么能坚持打持久战,尤其是在南方坚持长期抗战?我们为什么敢于长期打下去?这主要因为我们依靠了毛主席的著作。

\mxsay{毛泽东:}不一定。

\mxsay{黎笋:}当然,这是事实。我们还要善于在越南的实际环境中运用。

\mxsay{毛泽东:}你们有你们自己的创造。怎么能说你们没有创造,没有经验呢?吴庭艳一杀十六万,我是听的报告,不晓得准不准确,但我知道杀死了十几万人就是了。

\mxsay{黎笋:}是的,十六万人被杀害了,而且许多人被投入监狱。

\mxsay{毛泽东:}我看这就好了。你来杀我,我不可以杀你呀?

\mxsay{黎笋:}正是这样。单是在一九六九年我们就打死打伤敌人六十一万人,其中二十三万是美国人。

\mxsay{毛泽东:}美国人的人力分配不够,因为他们在全世界铺的太广了。所以,当他们的人被杀死时,他们的心都碎了。死个几万人,对他是一件大事。你们越南人,不论北越、南越,我看死人是要死的。

\mxsay{黎笋:}我们现在的打法死伤很少。不然,我们就不能长期坚持下去。

\mxsay{毛泽东:}是啊。恐怕老挝困难一点。老挝族在中国有没有啊?

\mxsay{周恩来:}有一些。

\mxsay{毛泽东:}他们住在哪儿?

\mxsay{周恩来:}在云南省,同老挝交界的地方。

\mxsay{毛泽东:}在西双版纳吗?

\mxsay{黄永胜:}西双版纳也有一些。

\mxsay{周恩来:}我们的壮族跟他们非常相像。

\mxsay{毛泽东:}将来老挝决胜的时候,可以到广西招一些壮族人,到云南招一些傣族人。壮族人很能打。过去军阀白崇禧和李宗仁就是靠壮族人。壮族人现在有多少?八百万?

\mxsay{周恩来:}现在多了,有一千多万人了。

\mxsay{毛泽东:}这是韦国清\mnote{5}那一族,他自己不承认。我曾问过他,他是哪一族的?是少数民族吧?他说他是汉族。后来他才承认他是壮族人。

\mxsay{周恩来:}太平天国的战士都很善战,他们之中的一些人就是壮族人。

\mxsay{毛泽东:}太平天国也有一部分军队是广西的。

\mxsay{黎笋:}越南的侬族也很能打仗,他们和广西的壮族人都属于同一个民族。

\mxsay{毛泽东:}东南亚是一个马蜂窝。东南亚的人民一天天在觉醒。有些和平主义者认为无非是公鸡好斗。哪有那么多公鸡呀!现在母鸡也好斗了嘛!

\mxsay{黎笋:}不斗就没有出路。

\mxsay{毛泽东:}是啊,不斗不行了。你逼得人家没有路走了嘛,欺负人嘛!

\mxsay{黎笋:}柬埔寨人和老挝人本来都信佛教,不好斗。现在他们也好斗起来了。

\mxsay{毛泽东:}就是呀,你也不要说信佛教的人就不好斗,中国也是信佛教的,辛亥革命打了十七年,后来是内部两派打,这样教育了人民。然后是北伐战争,然后就有红军了,然后日本人侵略中国,日本人投降以后就是蒋介石同我们打,打了不到四年,他就不干了,跑到台湾去了。他现在在联合国里说他代表整个中国。他跟我们几个人的关系可密切了。我跟蒋介石算是见过几次面,国民党在广州开中央全会的时候见过。我是一个国民党,是个跨党分子,我又是共产党的中央委员,又是国民党的候补中央委员。那个时候,我们几个人都参加了,我们的总理是蒋介石黄埔军校的政治部主任,蒋介石第一军的代理党代表。林彪同志就不用说了,他是蒋介石的学生,他在黄埔读了九个月。在中国,老一辈的人没有跟他打过交道的很少。

\section*{(二)}

\mxsay{黎笋:}最近,尼克松宣称说,在过去一百九十年里美国从来没有被打败过。意思是他这次也不愿意被越南打败。

\mxsay{毛泽东:}没有被打败过?

\mxsay{黎笋:}实际上他被打败好几次了。在中国,在朝鲜和印度支那的抗法战争中,他给法国人开支百分之八十的军费,结果他仍然被打败了。

\mxsay{毛泽东:}就是呀!你刚才讲的第一点是不要怕帝国主义。究竟谁真怕谁?小国,部分小民族存在这个问题。他慢慢试,试几年就理解了。

\section*{(三)}

\mxsay{毛泽东:}……那时我还告诉你,如果美国人不到中国边境来,你们不请我们去,我们是不出兵的。

\mxsay{黎笋:}我们也这样想。当我们还有能力打的时候,我们希望我们的“大后方”更稳固一些。在我们越南人民打美国人的时候,中国是我们的“大后方”。因此,我们曾经发出过这样的命令,即使我们的飞机受到攻击,也不要到中国机场降落。

\mxsay{毛泽东:}可以嘛,我们不怕。如果美国空军要来打越南空军的“庇护所”,就让他来嘛。

\mxsay{黎笋:}虽然我们下了这样的命令,但我们仍然需要依靠你们的支持。那时,你们派来的几个师在越南同样打美国飞机。

\mxsay{毛泽东:}就是嘛。美国人就是怕打,他们没勇气。你们可以谈判,我不是说你们不能谈判,但是你们的主要精力应该放在打上。两次日内瓦会议是谁破坏的?你们和我们都是老老实实遵守(会议协议),但是他们不干嘛!不干更好。

因此,甚至苏联总理柯西金在他发表公开讲话时,也不得不说只要召开国际会议,就必须同越南、老挝和柬埔寨商量。他们现在的领导人好多我不熟悉,不认识。柯西金我认识,并且同他谈过话。西方报纸总是造他们的谣,说他们的领导层怎样不和。我也闹不清楚。听说老百姓对柯西金这位领导人比较感兴趣。

\mxsay{黎笋:}我们也听说了。

\mxsay{毛泽东:}你们也听说了?我看斯大林又活起来了。当今世界主要的倾向是革命,包括整个世界。大国打世界大战的可能性是有,只是因为多了几颗原子弹,大家都不敢下手。主要是两个大国,现在都说三大国,中国不算。中国的原子武器还刚开始研究,还在研究阶段,这有什么可怕呀?中国人多,因此他们怕中国。但是我们也怕,因为人多要吃饭,要穿衣,那个问题可多了。所以我们现在也研究节制生育,使人口少一点。

\section*{(四)}

\mxsay{黎笋:}我们能够继续我们的战斗,是因为毛主席讲了,七亿中国人民是越南人民的坚强后盾,美国害怕了。这是非常重要的。

\mxsay{毛泽东:}有什么好害怕的?你侵略人家,我做个后方都做不得呀?你去几十万海、陆、空军欺负越南人民,中国做个后方都做不得呀?哪个法律上有这条规定?

\mxsay{黎笋:}美国人说他可以动员一千二百万军队,可是他只能派五十万军队到越南,如果越过了这个界限,他们就害怕了。

\mxsay{周恩来:}中国人多他也怕。

\mxsay{毛泽东:}人多还有一个不怕,打就打。横竖我跟你没有关系。你占了我的台湾,我没有占领你的长岛\mnote{6}。

\begin{maonote}
\mnitem{1}黎笋(一九〇七年四月七日——一九八六年七月十日),越南无产阶级革命家,忠诚的无产阶级革命战士,越南现代史上胡黎政权的不二接班人,胡志明衣钵最忠诚的继承者。在胡志明逝世后,黎笋取得了北越的领导权并随后统一了越南。时任越共第一书记。
\mnitem{2}北部湾事件,也作东京湾事件,一九六四年七月底,美国军舰协同西贡海军执行“34A”行动计划,对越南北方进行海上袭击。八月一日,美第七舰队驱逐舰“马多克斯”号为收集情报,侵入越南民主共和国领海,次日与越南海军交火,击沉越南鱼雷艇。美国政府迅即发表声明,宣称美海军遭到挑衅。四日,美国宣称美军舰只再次遭到越南民主共和国鱼雷艇袭击,即所谓“北部湾事件”,并以此为借口于五日出动空军轰炸越南北方义安、鸿基、清化等地区。七日,美国国会通过《东京湾决议案》,授权总统在东南亚使用武装力量。这一事件是美国在侵越战争中推行逐步升级战略,把战火扩大到越南北方的重要标志。

二〇〇五年美国国家安全局发表报告,承认一九六四年八月四日当夜越南人民军当夜对美军“沒有攻击”。北部湾事件是美国为入侵越南捏造的。
\mnitem{3}尼克松,时任美国总统。
\mnitem{4}西哈努克,柬埔寨国王,抗美战争期间,中国政府开辟的通过柬埔寨到达南越的运输线,日夜抢运军援物资。越南抵抗部队并深入柬境内十余公里,从那里出发,频繁袭击南越军队和美军。美国对西哈努克亲王坚持和平、中立政策,反对美军入侵越南的态度深为不满,不断指使南越当局对柬埔寨实施战争威胁,向西哈努克施加压力,最终于一九七〇年三月十八日乘西哈努克亲王出国访问之机,策动柬政府中的右翼势力朗诺集团发动政变,推翻西哈努克亲王领导的王国政府,宣布废黜国家元首西哈努克,建立了以朗诺为首的政权。政变后,他长期寓居北京,一直受到元首待遇,并于一九七〇年三月二十三日宣布成立柬埔寨民族统一阵线并任主席,发表了告高棉同胞书和声明,领导民族解放斗争,五月在北京组成柬埔寨民族统一阵线和王国民族团结政府,任阵线主席和国家元首。一九七五年四月十七日金边解放后回国,担任民柬国家元首。
\mnitem{5}韦国清,壮族,开国上将,时任广西壮族自治区革命委员会主任。
\mnitem{6}长岛(Long Island)是一位于北美洲大西洋岸的岛屿,行政上属于美国纽约州的一部份。长岛长一百九十公里,宽约二十到三十公里,它从纽约港伸入北大西洋。
\end{maonote}
