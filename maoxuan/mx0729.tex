
\title{论革命的“三结合”}
\date{一九六七年三月、四月、十月}
\maketitle


\date{一九六七年三月七日}
\section*{(一)论革命的“三结合”\mnote{1}}

在需要夺权的那些地方和单位,必须实行革命的“三结合”的方针,建立一个革命的、有代表性的、有无产阶级权威的临时权力机构。这个权力机构的名称,叫革命委员会好。

从上至下,凡要夺权的单位,都要有军队代表或民兵代表参加,组成“三结合”,不论工厂、农村、财贸、文教(大、中、小学)、党政机关及民众团体都要这样做。县以上都派军队代表,公社以下都派民兵代表,这是非常之好的。军队代表不足,可以暂缺,将来再派。

\date{一九六七年三月二十日}
\section*{(二)关于“三相信、三依靠”的指示\mnote{2}}

这是一个很好的报告。

一、依靠群众。这一条是主要的。我们都是从群众中来的嘛!群众就是工、农、兵、学、商。中央办公厅所属机关的干部都是群众嘛!工作主要是群众作的,靠少数领导人是不行的,也是不够的。哪一样也不能离开群众,要有群众观点。

二、依靠军队。我们的军队不仅会打仗,而且会做群众工作、宣传工作、生产工作等。军队内的很多干部,从小参加军队,很少读书,文化是在部队慢慢提高的,思想比较单纯。军队与地方不同,没有地权,没有财权,说走就走了。省里有地盘,军队没有地盘。军队还有一条,组织性纪律性强,动作快。如沈阳军区的支左、支工、支农经验,中央批转以后,全军二十一天内就行动起来了。如果是地方,传来传去,动作很慢。

三、要依靠干部。干部绝大多数是好的。很多事要让干部去办,政策靠他们去执行。有些省委书记要赶快解放出来,要他们好好检讨。有的省过去犯了错误,就是因为他们害怕群众,动员一些人去“保护”他们,结果害了自己。

\date{一九六七年五一前后}
\section*{(三)关于干部问题的指示\mnote{3}}

一、我们这些干部是经过长期锻炼的,还是要有些人站出来。现在的情况是:站出来就打。我们的事情总是要有人挂帅的。红卫兵能挂帅吗?今天上台,明天就会被打倒,原因是政治上不成熟。现在我们要做很多工作,使那些主要干部站出来,哪怕是黎元洪\mnote{4}式的人物也是好的,出来两年也好。红卫兵不行,没有经过锻炼,这样大的事情,信不过他们。

二、要依靠干部,干部绝大多数是好的,很多事是要让干部去办,政策靠他们去执行。有些省委书记要赶快解放出来,要让他们好好检讨。有的省委书记过去犯了错误,就是因为他们害怕群众,动员一些人去“保护”他们,结果害了自己。

三、对于犯有错误的干部,要正确对待,不能一概打倒。只要不是反党反社会主义分子而又坚持不改和累教不改的,就要允许他们改过,鼓励他们将功赎罪。惩前毖后,治病救人,这是党的传统政策。只有这样才能使犯错误的本人心悦诚服,也才能使无产阶级革命派取得大多数人的衷心拥护,使自己立于不败之地,否则是很危险的。

要爱护干部,要支持革命领导干部出来,革命小将现在叫他们挂帅还不行,有个培养的过程,但小将是可爱的,是无产阶级革命事业的接班人。

我们现在要搞三结合,要使青年参加各方面的领导工作。不要看不起青年人,二十几岁,三十几岁都可以接受他们作事情,不把新一代搞上来怎么使他们受到锻炼?这个三结合就是老、中、小,就是二十岁以上就行了。

我们提倡青年人上台,有人说青年人没有经验,上台就有经验了。过去也提培养无产阶级革命事业接班人,那是从形式上讲的,现在要落实在组织上。

三结合,老、中、小要三结合,不主张把老干部都打倒,老干部一天天见上帝了。

全国应该有更多的第一把手站出来。

\date{一九六七年十月十七日}
\section*{(四)\mnote{5}}

各工厂、各学校、各部门、各企业单位,都必须在革命原则下,按照系统,按照行业,按照班级,实现革命的大联合,以利于促进革命三结合的建立,以利于大批判和各单位斗、批、改的进行,以利于抓革命、促生产、促工作、促战备。

\date{一九六七年十二月十八日}
\section*{(五)\mnote{6}}

这次文化大革命是一个“大审干”,用群众性方法审查干部。有可能要冤枉一部分好人,但横竖不杀,搞错了将来平反。

\begin{maonote}
\mnitem{1}这是毛泽东在一九六起年三月七日《论革命的“三结合”》一文中所写的两段话(一九六七年第五期《红旗》杂志社论)。
\mnitem{2}这是毛泽东在听了“林彪三月二十日在军级以上干部会议上的讲话”后的指示。
\mnitem{3}这是毛泽东在一九六七年四月二十九日至五月一日的中央常委、中央文革小组和政治局会议上的讲话节录。
\mnitem{4}黎元洪,一九一一年,武昌起义爆发,黎元洪被革命党人强迫推举为湖北都督。
\mnitem{5}中共中央、国务院、中央军委、中央文革于一九六七年十月十七日发出了《关于按照系统实行革命大联合的通知》,毛泽东起草了本段。
\mnitem{6}这是毛泽东在一九六七年十二月十八日会见以谢·佩奇为首的阿中友协代表团谈话节录。
\end{maonote}
