
\title{视察华北、中南和华东地区时的重要指示}
\date{一九六七年十月}
\thanks{这是毛泽东同志在视察华北、中南和华东地区的谈话纪要。}
\maketitle


\section{(一)关于形势}

七、八、九三个月,形势发展很快。全国的无产阶级文化大革命形势大好,不是小好。整个形势比以往任何时候都好。

形势大好的重要标志是人民群众充分发动起来了。从来的群众运动都没有像这次发动得这么广泛,这么深入。全国的工厂、农村、机关、学校、部队,到处都在讨论无产阶级文化大革命的问题,大家都在关心国家大事。过去一家人碰到一块,说闲话的时候多。现在不是,到一块就是辩论无产阶级文化大革命的问题。父子之间、兄弟姐妹之间、夫妻之间,连十几岁娃娃和老太大,都参加了辩论。

有些地方前一段好像很乱,其实那是乱了敌人,锻炼了群众。

再有几个月的时间,整个形势将会变得更好。

\section{(二)关于大联合}

在工人阶级内部,没有根本的利害冲突。在无产阶级专政下的工人阶级内部,更没有理由一定要分裂成为势不两立的两大派组织。一个工厂,分成两派,主要是走资本主义道路的当权派为了保自己,蒙蔽群众,挑动群众斗群众。群众组织里头,混进了坏人,这是极少数。有些群众组织受无政府主义的影响,也是一个原因。有些人当了保守派,犯了错误,是认识问题。有人说是立场问题,立场问题也可以变的嘛。站队站错了,站过来就是了。极少数人的立场是难变的,大多数人是可以变的。革命的红卫兵和革命的学生组织要实现革命的大联合。只要两派都是革命的群众组织,就要在革命的原则下实现革命的大联合。两派要互相少讲别人的缺点、错误,别人的缺点、错误,让人家自己讲,各自多做自我批评,求大同,存小异。这样才有利于革命的大联合。

\section{(三)关于革命大联合以谁为核心}

什么“以我为核心”,这个问题要解决。核心是在斗争中实践中群众公认的,不是自封的。自己提“以我为核心”是最蠢的。王明、博古、张闻天\mnote{1},他要做核心,要人家承认他是核心,结果垮台了。什么是农民,什么是工人,什么打仗,什么打土豪分田地,他们都不懂。要正确地对待受蒙蔽的群众。对受蒙蔽的群众,不能压,主要是做好思想政治工作。

\section{(四)关于向坏人专政的问题}

政府和左派都不要捉人,发动革命群众组织自己处理。例如,北京大体就是这样做的。专政是群众的专政,靠政府捉人不是好办法。政府只宜根据群众的要求和协助,捉极少数的人。

一个组织里的坏头头,要靠那个组织自己发动群众去处理。

\section{(五)关于干部问题}

绝大多数的干部都是好的,不好的只是极少数。对党内走资本主义道路的当权派,是要整的,但是,他们是一小撮。我们的干部中,除了投敌、叛变、自首的以外,绝大多数在过去十几年、几十年里总做过一些好事!要团结干部的大多数。犯了错误的干部,包括犯了严重错误的干部,只要不是坚持不改,屡教不改的,都要团结教育他们。要扩大教育面,缩小打击面,运用“团结—批评和自我批评—团结”这个公式来解决我们内部的矛盾。在进行批判斗争时,要用文斗,不要搞武斗,也不要搞变相的武斗。有一些犯错误的同志一时想不通,还应该给他时间,让他多想一个时候。要允许他们思想有反复,一时想通了,遇到一些事又想不通,还可以等待。要允许干部犯错误,允许干部改正错误。不要一犯错误就打倒。犯了错误有什么要紧?改了就好。要解放一批干部,让干部站出来。

正确地对待干部,是实行革命三结合,巩固革命大联合,搞好本单位斗、批改的关键问题,一定要解决好。我们党,经过延安整风,教育了广大干部,团结了全党,保证了抗日战争和解放战争的胜利。这个传统,我们一定要发扬。

\section{(六)关于上下级关系问题}

有些干部为什么会受到群众的批判斗争呢?一个是执行了资产阶级反动路线,群众有气。一个是官做大了,薪水多了,自以为了不起,就摆架子,有事不跟群众商量,不平等待人,不民主,喜欢骂人、训人。严重脱离群众。这样,群众就有意见。平时没有机会讲,无产阶级文化大革命中爆发了,一爆发,就不得了,弄得他们很狼狈。今后要吸取教训,很好地解决上下级关系问题,搞好干部和群众的关系。以后干部要分别到下面去走一走,看一看,遇事多和群众商量,做群众的小学生。在某种意义上说,最聪明、最有才能的,是最有实践经验的战士。

要讲团结。干部有错误,有问题,不要背后说,找他个别谈,或者会议上讲。

我们现在有的严肃、紧张有余,团结、活泼不足。

\section{(七)关于教育干部的问题}

干部问题,要从教育着手,扩大教育面。不仅武的(军队),还要文的(党、政),都要进行教育,加强学习。中央、各大区、各省、市都要办学习斑,分期分批地轮训。每省都要开县人武部以上各级干部会,一个省二、三百人,多则四、五百人,大省应到千人左右。半年之内争取办好此事,否则,一年也可。

今后,争取每年搞一次,每一次的时间不要太长,大体上两个月左右。

\section{(八)关于红卫兵和造反派}

对红卫兵要进行教育,加强学习。要告诉革命造反派的头头和红卫兵小将们,现在正是他们有可能犯错误的时候。要用我们自己犯错误的经验教训,教育他们。对他们做思想政治工作,主要是同他们讲道理。

广大工农群众、人民解放军指战员、红卫兵小将、革命干部和革命知识分子,在一年多来的无产阶级文化大革命中建立了功勋。

要斗私、批修,要拥军爱民,要抓革命促生产、促工作、促战备,把各方面的工作做得更好,把无产阶级文化大革命进行到底。

\begin{maonote}
\mnitem{1}张闻天,笔名洛甫,张闻天在遵义会议后一度被历史大潮推到党的总负责的职位上。一九五九年庐山会议期间,一九五九年七月二日至八月一日在庐山召开的中共中央政治局扩大会议和八月二日至十六日召开的中共八届八中全会,会议通过的《关于以彭德怀同志为首的反党集团的错误的决议》,揭发和批判了彭德怀、张闻天、黄克诚等同志的错误,他们把一些暂时的、局部的、早已克服了或者正在迅速克服中的缺点收集起来,并且加以极端夸大,把形势描写成为漆黑一团,企图向党要权。
\end{maonote}
