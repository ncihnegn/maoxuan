
\title{发展自己的工业和农业不要依靠外国}
\date{一九七〇年六月二十四日}
\thanks{这是毛泽东同志同苏丹政府友好代表团团长马哈古卜\mnote{1}的谈话纪要。}
\maketitle


(马哈古卜说,在英国还没有完全从苏丹撤走前,美国又进来了。但是由于人民的团结和斗争,英国的星辰已经落下去了,美国的星辰也要落下去。)

这个世界上怕英国人、怕美国人的很多,这几年慢慢地都不大怕了,并敢于说你们刚才所说的那些话。你们苏丹人民了不起啊!

对美国人民和美国的统治者是要分开来看待的。我对美国人民是寄予希望的,就是美国的工人以及其他穷苦的人。他们总有一天要站在你们和我们这个战线的。

美国独立时,只有三百多万人口。那时英国有近两千万人口,是全盛时代,可是打了败仗。打了八年啊。还不就是一些比你们现在还要差的武器。你们苏丹有一千四百多万人口。

(马哈古卜说,人民已经起来,毛主席的教导在全世界已经传播开了。人们学习了毛主席的教导,理解了毛主席的话,懂得真理应当属于自己,所以人民一定会起来。)

中国的经验可以作参考,要结合各国自己的经验。我们也是找外国的经验,结合中国的经验,譬如巴黎公社的经验,十月革命的经验,等等。

要发展自己的工业和农业。不要依靠外国。可以进口外国技术,利用外国的技术人员帮助你们训练科学家、工程技术人员。但是要自己搞。如果没有这个决心,帝国主义还是要整你们的。

(马哈古卜感谢中国向苏丹提供友好援助,特别是在谈判中贯穿的友好精神。)

物质援助,要到了手,见了效,那才算数。不到手,工厂没有建立起来,那不能算数。你们自己的工程技术人员、科学家没有培养出来,中国人不撤走,也还不能算数。那末,什么叫帝国主义呢?它就是不帮助人家建设的,它赖着不肯走,那没有办法,只好革命了。

非洲可沸腾起来了,闹起来了,大闹起来了!半个多世纪了,五十多年,经过两次世界大战,帝国主义又培养了反对它的人,另外培养一批走狗拥护它自己。

美帝国主义、世界人民不喜欢它,美国人民也不喜欢它。这个美帝国主义专门团结那些各国人民认为是不好的人。它喜欢那类人作朋友,我们就喜欢我们这类人。

帝国主义是要对非洲各国人民的领袖和著名政治家进行暗害阴谋活动的,要提高警惕。

\begin{maonote}
\mnitem{1}曼苏尔·马哈古卜,时任苏丹财政部长。
\end{maonote}
