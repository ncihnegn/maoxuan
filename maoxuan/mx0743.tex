
\title{接见首都红代会“五大领袖”\mnote{1}时的谈话}
\date{一九六八年七月二十八日}
\thanks{这是毛泽东同志在接见首都红代会负责人时谈话节选。}
\maketitle


\section*{(一)}

\mxsay{毛泽东:}(聂元梓、谭厚兰、韩爱晶、王大宾走进接见会场,毛主席站起来同他们一一握手)都是一些年轻人!

\mxsay{江青:}好久不见了。你们又不贴大字标语。

\mxsay{毛泽东:}还不是天安门上见过,又没有谈话,不行嘛!你们是无事不登三宝殿,但是你们的小报我都看过,你们的情况,我都了解。蒯大富怎么没来,是出不来,还是不愿来?

\mxsay{谢富治:}恐怕是不肯来。

\mxsay{韩爱晶:}不会的。这个时候,他要知道中央文革接见会不来?他见不到主席会哭的,肯定是出不来。

\mxsay{毛泽东:}蒯大富要抓黑手\mnote{2},这么多工人“镇压”“压迫”红卫兵,黑手是什么?现在抓不出来。黑手就是我嘛!他又不来,抓我就好,来抓我嘛!本来新华印刷厂、针织总厂、中央警卫团是我派去的。我说大学武斗怎么解决?你们去做工作看看,结果去了三万人。其实他们恨北大,不恨清华。(对聂)工人学生这么搞几万人游行,听说你们那里搞招待还好,是你们,还是“(北大)井冈山”?\mnote{3}

\mxsay{聂元梓:}我们在门口摆开水供给……

\mxsay{温玉成、黄作珍:}不是他们。北大和那个单位冲突了。

\mxsay{聂元梓:}是与农科院。我们还在门口摆了茶水,他们骂我们二流派,老保。

\mxsay{毛泽东:}你们没跟他们打?

\mxsay{聂元梓:}互相打了。

\mxsay{毛泽东:}北大抓黑手,这黑手不是我,是谢富治。我也没有这么大的野心。我说你们去那么一点人,跟他们商量商量。蒯大富说有十万。

\mxsay{谢富治:}不到三万人。

\mxsay{毛泽东:}你们看大学武斗怎么办?一个是统统撤出去,学生也不要管,谁想打就打,过去革委会、卫戍区对大学的武斗不怕乱,不管、不急、不压,这看来还是对的。另一个是帮助一下,这个问题深得工人的赞成,深得农民的赞成,深得多数学生的赞成。大专院校五十几个,打得凶的,也就大概五、六个,试试你们的能力。至于如何解决,你们一个住南方,一个住北方,都叫新北大,打个括弧,新北大(井冈山)、新北大(公社),就像苏联共产党(布)。要么军管,请林彪同志挂帅,还有黄永胜,问题总要解决嘛!你们搞了两年文化大革命了。斗批改,现在是一不斗、二不批、三不改。斗是斗,你们是搞武斗,人民不高兴,工人不高兴,农民不高兴,居民不高兴,多数学校学生不高兴,你们学校多数学生也不高兴。就连拥护你的一派,也有人不高兴。就这样统一天下?你“新北大公社”,老佛爷(聂元梓的外号)是多数,是哲学家。“新北大公社”、校文革里就没有反对你的人了?我才不信呢!当面不说,背后还是会说怪话。王大宾,你的事情好办一些吧?

\mxsay{王大宾:}那几个反对谢富治的跑了。

\mxsay{谢富治:}他的二把手聂树人\mnote{4}要夺权,说他右了。

\mxsay{毛泽东:}他就那么左,马克思?

\mxsay{王大宾:}那是他们挑拨关系。聂树人是一个好同志,出身又好,苦大仇深。这个人很正直,革命干劲也大,革命性强,就是急一些,不大会团结人,工作方法生硬一些。

\mxsay{毛泽东:}你能团结他吗?一个左,一个右,很好团结嘛!你坐过来,到我这里来。

\mxsay{林彪:}来嘛!

\mxsay{谢富治:}去!去!(王大宾坐到毛身旁)

\mxsay{毛泽东:}坐下,坐下。事情要留有余地,都是学生。他们也没有搞黑帮,最近有些学校斗了黑帮,画了像。“新北大公社”搞了几十个,就那么点黑帮?我看不止那么点黑帮。就是关键在于两派,忙于武斗,心都到武斗上去了。现在不搞斗、批、改,就搞斗批走。学生不讲了嘛,斗批走,斗批散。现在逍遥派那么多。现在社会上说聂元梓、蒯大富的坏话多起来了。聂元梓的炮灰不多,蒯大富的炮灰也不多,有时三百人,有时候一百五十人,哪像林彪、黄永胜那么多。这回我一出就是三万多。

\mxsay{林彪:}哪有黄永胜那么多。

\mxsay{毛泽东:}天下大事嘛,分久必合,合久必分。把武斗工事统统拆掉,什么热武器、冷武器,要刀枪入库。聂元梓,他们叫你“老佛爷”、“佛爷老巢”。还有你这个谭厚兰同志,梳两个小辫子,你要求下放,在学校里读了十几年书,大家都同意你下放,我怕你走不开,你走了,谁代替你呢?

\mxsay{谭厚兰:}都安排好了。

\mxsay{毛泽东:}你们这五大将,我们都是护你们的,包括抓“黑手”的蒯大富。我们有偏向。“(北大)井冈山”(聂元梓“新北大公社”的反对派“井冈山”)、“四·一四”(蒯大富清华“井冈山兵团”的反对派“四·一四”组织)、“兵团”(北京师范大学谭厚兰“井冈山公社”的反对派“造反兵团”)就会对我们有意见。我不怕别人打倒。清华“四·一四”说《“四·一四”思潮必胜》,我就不高兴,说打江山的人不能坐江山,无产阶级打天下不能坐天下。“四·一四”有个理论家叫周泉缨\mnote{5},理论家何必抓他?人家是一派的理论家,写篇文章你抓他干什么?把他放出来。人家有意见,让他再写嘛!不然会说没有言论自由了嘛。我说你老佛爷,也大方一点,你北大“井冈山”几千人,那一河水放出来,大水冲了龙王庙,你受得了受不了?你这个老佛爷,不然就实行军管。第三个办法,按着辩证法,不要住在一个城里,就一分为二,或者你搬到南方,或者(北大)“井冈山”搬到南方,一个南,一个北,根本不见面,打不起来,各人清理自己内部,一统天下。不然你也害怕,把你老佛爷老巢一捅,捅得你睡不着觉。你怕,他也怕。稍微留一手,是必要的,何必那么紧张呢?怕人家打,你不留点后手,人家一冲怎么得了啊?听人家说了,不是有个凶手要戳你吗?知道了凶手是谁也不一定要抓,算了,明明知道是他也不要说了。不过你以后要注意一点,不要一个人到处乱跑。

\mxsay{江青:}她有人保镖。

\mxsay{聂元梓:}没有。

\mxsay{毛泽东:}你哥哥也不好,姐姐也不好。你那个娘家就是不好嘛。哥哥不好是哥哥嘛,姐姐不好是姐姐嘛,为什么一定要牵连妹妹呢?(有人报告找不到蒯大富)蒯大富是不愿来,还是来不了?

\mxsay{谢富治:}广播了,点名中说中央文革要找,请蒯大富开会。他就是不肯来。

\mxsay{江青:}是他不肯来,还是出不来?

\mxsay{谢富治:}我估计有人控制他。

\mxsay{姚文元:}有可能。

\mxsay{毛泽东:}蒯大富这个人我看是好人,出面多。操纵他的人是坏人,蒯大富以及出面的人是好人,这个经验很多。王大宾,你那里没有打架?

\mxsay{王大宾:}没有。六六年九月二十三日与保守派干了一架,是伯达同志派人救了我们,以后我们取得了胜利。

\mxsay{毛泽东:}那就好,以后一个你,一个韩爱晶不要打架。韩爱晶,你是韩信的后代,很会打主意,是个谋士啊!

\mxsay{韩爱晶:}我们学校工农子弟多,比较朴实,有不同意见,但没有分裂为两派。

\mxsay{姚元文:}我才不信呢?你们那里就没有不同意见,纯之又纯?

\mxsay{康生:}韩爱晶不是你说的那样吧?

\mxsay{毛泽东:}你们不要把韩爱晶说得那么坏,人家很难受。

\mxsay{韩爱晶:}蒯大富周围有一批人,里边比较复杂。运动初期靠写大字报起家的人数极少了,武斗这批人多了,要求改组总部,蒯大富控制不了。

\mxsay{毛泽东:}谭厚兰,文化革命两年了,你那个一、二百人的兵团也弄得睡不着觉。你暂时还不能走,你是一个女皇。今天到会四个,有两个女的,真了不起!我看你暂时不能走。你要给他们粮吃,出入无阻,那些人也够惨的。北师大“造反兵团”是个湖南‘省无联’\mnote{6}式的大杂烩,因为他夺权\mnote{7}嘛!别的有些学校都参加了,你(指韩爱晶)、蒯大富都有份。

\mxsay{韩爱晶:}我也参加了。

\mxsay{江青:}那是韩爱晶去颠覆人家谭厚兰。

\mxsay{毛泽东:}你有份,我们的蒯司令也有份。年青人就是要作些好事,也会作些坏事。你们说中央文革没打招呼,林彪同志、周总理在三月二十四日、三月二十七日讲了话,又开了十万人大会。这次黄作珍同志、温玉成同志都讲了话,可是下面还打,好像专门和我们作对。我们这个道理,第一条要文斗,不要武斗。如果你们要打,也可以。越打越大,两方都有土炮,你们算什么打嘛!你们的打法算不了啥,把卡宾枪、大炮都使出来,像四川一样把高射炮对天上打。

\mxsay{江青:}败家子。

\mxsay{毛泽东:}你这个老佛爷那么大神通,调起兵来也只有那么两三百,你的兵跑那里去了?还得依靠工人、复员军人作主力,没有那个你还不行,林彪同志兵多,给你几千几万,可把“(北大)井冈山”通通消灭嘛。这问题也不要在这里答复,商量商量,也可以开会讨论讨论,但是首先还是要联合。

\mxsay{林彪:}首先还是要联合。主席讲了四个方案:第一是军管;第二,一分为二;第三,斗批走;第四,要打就大打。

\mxsay{毛泽东:}一分为二,因为结了仇,双方紧张得很,两方都睡不着觉。搬家可是个问题,找地点,在北京就会争起来。我看,这个大会堂很空的,中南海地方很大,可容四、五万红卫兵,办个学校还不行?或者聂元梓来,或者候汉清来。你们不是叫“杀牛宰猴炖羊肉”嘛?牛是牛辉林,猴是侯汉清,羊是杨克明(这三人都是聂的反对派北大“井冈山”负责人)。这三个人我只知道杨克明,杨克明也是个年青人嘛!还到过十一中全会,那张大字报(指《宋硕、陆平、彭珮云在文化革命中究竟干些什么?》的大字报)杨克明还帮了忙。你们这张大字报分成两家,这种社会现象是不以人们的意志为转移的,谁会料到这么打起来?原来打算停课半年,登了报,后来不行,延长一年,不行再延长一年、二年、三年。我说,如果不行,要多长时候给多长时候,反正人是会长的,你一年级现在就是三年级了,再搞两年、四年、八年的,你还不是在哪里过一天长大一天。斗批走也是个办法,谭厚兰不是想走吗?走光,扫地出门。大学要不要办呢?要不要招生呢?不招也不行。我那个讲话(“七二一指示”)是留有余地的,大学还要办,讲了理工科,并没有说文科都不办,但旧的制度、旧的办法不行,学制要缩短,教育要革命。搞不出名堂就拉倒。我看高中、高小、初中基础课跟大学差不多,上六年、十年顶多了。高中重复初中,大学重复高中,基础课都是重复。这专业课,先生都不懂专业,哲学家讲不出哲学,还学什么?你聂元梓不是哲学家吗?

\mxsay{聂元梓:}我不是哲学家。

\mxsay{毛泽东:}这个哲学有什么学头呢?这个哲学是能够在大学里学出来的吗?又没有做过工人、农民,就去哲学,那个哲学叫什么哲学?

\mxsay{林彪:}越学越窄,是“窄学”。

\mxsay{毛泽东:}如果学文学呢?就不要搞文学史,但要学写小说,每周给我写一篇稿,写不出来就到工厂去当学徒,当学徒就写当学徒的过程。现在学文学的,写不出小说。上海的胡万春\mnote{8}原来还写了很多东西,以后就没有看到什么了。

\mxsay{周恩来:}还有高玉宝\mnote{9},都进了大学,后来头脑就僵化了。

\mxsay{毛泽东:}我跟你们讲讲马恩列斯,除了马克思、列宁读完了大学,其他人都没有读完。列宁读法律读了一年,恩格斯只上了一年半,中学还没有读完,父亲叫他到工厂当会计。后来工厂搬到英国,在工厂里接触了工人。恩格斯的自然科学怎么学的?他是在伦敦图书馆里学的,在那里呆了八年,根本没有进过大学。斯大林没有进过大学,教会中学毕业的。高尔基只读了两年小学,比江青还差。江青是高小毕业,读了六年,高尔基只读了两年。

\mxsay{叶群:}江青同志自学很刻苦。

\mxsay{毛泽东:}你不要吹她。学问才不是靠学校里学来的。从前我在学校里是不守规矩的,只是以不开除为原则的。考试嘛,五、六十分以上,八十分以下,七十分为准。好几门学科我是不搞的,要搞有时没办法,有的考试我交白卷,考几何我就画一个鸡蛋,这不是几何吗?因为是一笔,交卷最快。

\mxsay{林彪:}我读中学读了四年,没毕业就走了,自动退学,没有中学文凭,就去当小学教员,喜欢自学。

\mxsay{毛泽东:}现在办的军事学校害死人,黄埔军校你们知道多长?三个月、六个月!

\mxsay{林彪:}第一、二、三期只有三个月,第四期起加长了。

\mxsay{毛泽东:}就是训练一下,改变一下观点。至于有什么学问,不那么多,实际学习一些军事教练。

\mxsay{林彪:}有一点,学了就忘了。学几个礼拜的东西到军队几天就一目了然,百闻不如一见。

\mxsay{毛泽东:}我就没有上过什么军事学校,我就没有读过什么兵法。人家说我打仗靠“三国演义”、“孙子兵法”,我说,“孙子兵法”我没看过,“三国演义”是看过的。

\section*{(二)}

\mxsay{毛泽东:}我为啥子不找你们的反对派呢?今天找你们来谈这事,使你们有准备啊!我是历来不搞录音的,今天录了,不然你们回去各取所需。如果你们各取所需,我就放我这录音。你们先去讨论讨论。这么一搞多人都被动。搞了这么多天不算数。开了这么多天会,开始黄作珍讲话不算数,找蒯大富也不算数,一定要让中央直接表态。除了开始管一下,后来事多,也就管不上了。北京有谢富治来管嘛。过去召集你们开会,我也不到的,林彪同志也不到的,当官僚了。这次怕你们把我开除党籍。官僚主义就开除,我早就不大想当了,我又是黑手镇压红卫兵。

\mxsay{林彪:}昨天我开车子,我说去看看大字报,我问怎么没有北大、清华的大字报啊?人家说:他们武斗。我说,你们脱离群众,群众要求制止武斗的呼声很高。

\mxsay{毛泽东:}群众就是不要打内战。

\mxsay{林彪:}你们脱离了工农兵。

\mxsay{毛泽东:}有人讲,广西布告\mnote{10}只适用于广西,陕西布告\mnote{11}只适用于陕西,在我这里不适用。那现在再发一个全国性的布告,谁如果还继续造反,打解放军,破坏交通、杀人、放火,就是犯罪。如果有少数人不听劝阻,坚持不改,就是土匪,就是国民党,就要包围起来,还继续顽抗,就要实行歼灭。

\mxsay{林彪:}现在有的是真正的造反派,有的是土匪、国民党,打着我们的旗号造反。广西烧了一千间房子。

\mxsay{毛泽东:}在布告上写清楚,给学生讲清楚,如果坚持不改,就抓起来,这是轻的。重的实行围剿。

\section*{(三)}

(黄作珍报告蒯大富来了。蒯大富进来就大哭。毛站起来上前握手,江青笑了。)

\mxsay{蒯大富:}(一边哭一边告状)主席救我,主席救我!“杨余傅”\mnote{12}黑后台调几万工人突然把清华包围。我们跟工人讲理,他们也不讲。我们学生一出去,他们就把学生抓到卡车上拉走。我们打不过工人,我们的人现在都在大街上……。

\mxsay{韩爱晶:}(流泪)不要胡说!工人、解放军是毛主席派去的。

\mxsay{蒯大富:}不可能!主席每次派解放军制止武斗,都是不带枪、不打人、不骂人,把人隔开。这次怎么抓我们的人!

\mxsay{毛泽东:}(对着谢富治、温玉成问道)是不是抓人了?谁让你们抓人!统统放了!

\mxsay{蒯大富:}我们二把手鲍长康也被抓了。

\mxsay{毛主席:}(对谢富治说)把所有的人都放了!把鲍长康放到人民大会堂门口。

(蒯大富嗯嗯地哭。整个气氛被蒯大富的情绪所影响,毛主席是极重感情的人。毛主席流着眼泪,江青也哭了。)

\mxsay{江青:}(重复着说)蒯大富,安静点,不要激动。蒯大富,你不要激动。你坐下来。

\mxsay{毛泽东:}(对黄作珍)你叫黄作珍,那里人?

\mxsay{黄作珍:}江西宁都人。

\mxsay{毛泽东:}老表嘛,久闻大名。黄作珍同志讲话不算数,谢富治同志讲话也不算数,市革委会开会也不算数,不晓得我们中央开会算不算数?我变成了黑手,把我抓到卫戍区去吧!

\mxsay{姚文元:}伸出红手,宣传毛泽东思想,我们都紧跟。

\mxsay{毛泽东:}四个办法,是哪四种?

\mxsay{姚文元:}军管,一分为二,斗批走,要打大打。

\mxsay{毛泽东:}一是军管,二是一分为二,三是斗批走,你们一不斗,二不批,三不改,一直打了几个月?

\mxsay{周恩来:}去年开始。

\mxsay{毛泽东:}第四再大打,打它一万人。工人撤出来,把枪还给你们大打,像四川一样。

\mxsay{江青:}败家子。

\mxsay{毛泽东:}我才不怕打哩,一听打仗我就高兴。北京算什么打,无非冷兵器,开了几枪。四川才算打,双方都有几万人,有枪有炮,听说还有无线电。以后布告出来要广泛宣传。再不听的,个别的抓起来,个别的包围消灭,反革命嘛!

\mxsay{江青:}广西围了快两个月了。

\mxsay{周恩来:}你们也不想一想,广西布告为什么是毛主席的伟大战略部署?说关心国家大事,你们五个也不发表联合声明表示态度,做做工作。

\mxsay{毛泽东:}他们忙啊!

\mxsay{周恩来:}这就是国家大事嘛!

\mxsay{毛泽东:}不要分派了。

\mxsay{江青:}希望你们团结起来,不要分“天派”“地派”\mnote{13},什么张家派、李家派,都是毛泽东思想派!

\mxsay{毛泽东:}不要搞两派,搞成一派算了,搞什么两派?困难是有的。

\mxsay{陈伯达:}教育革命,教改搞不上去。

\mxsay{毛泽东:}教育革命搞不上去,我们也搞不上去,何况你们。这是旧制度害了你们,为什么搞不上去呢?我们的陈伯达同志在中央的会议上着急,我说不要着急,过几年,人家走了,就算了么?我看无非是这么几条,搞什么教育革命,搞不成还不就散了。这是学生讲的,我还不是从逍遥派那里得点消息!现在我们来管这些事情,我看不公道,打一点内战无关紧要嘛,所以四条中有一条要打就大打。

\mxsay{姚文元:}我倾向有些学校斗批散,斗批走。

\mxsay{毛泽东:}地球一转一年,十转十年。两派这样下去,我看不走也得走,要打就让他们大打,空出地盘来。让人家写小说的去自修,学文学的你要写诗,写剧本。学哲学的你给我搞家史、历史,写革命的过程。学政治经济学的不能学北大教授,北大有什么出名的教授?这些东西不要先生教。先生教,这是个害人的办法。组织个小组,自己读书,自修大学。来来去去,半年一年,二年三年均可。不要考试,考试不是办法。一本书考十题,一本书一百个观点,不只是十分之一吗?就考对了,对其它百分之九十怎么办呢?谁考马克思?谁考恩格斯?谁考列宁?谁考林彪同志?谁考黄作珍同志?群众需要,蒋介石当教员,我们都是这样,中学要教师,小学要教师,教材要删繁就简。

\mxsay{姚文元:}办好几个图书馆。

\mxsay{毛泽东:}让工农兵都有时间去,到图书馆读书是个好办法。我在湖南图书馆读了半年,在北大图书馆读了半年。自己选择图书,谁教啊?我只上了一门新闻学。大学不要办得那么死,这个大学应该比较自由一些。

\mxsay{江青:}现在是搞武斗。

\mxsay{毛泽东:}武斗有两个好处,第一是打了仗有作战经验,第二个好处是暴露了坏人。对武斗要作全面分析,社会现象不以人的意志为转移。现在工人去干涉,如果不行,把工人撤出来,再斗十年,地球照样转动,天也不会掉下来。

\mxsay{江青:}我们真痛心你们,瞎说什么不要大学生啦,我们是要你们的。你们有的有时还听我们一些,有的听,有的当面一套,背后一套。你们后头的东西我们也搞不清。

\mxsay{毛泽东:}背后不听,我们这里有个办法,工人伸出“黑手”,用工人去干涉,无产阶级去干涉。

(聂元梓要求派解放军去北大)

\mxsay{毛泽东:}你要合你胃口的,一定要六十三军,别的你又不要。三十八军可以不可以?如果“(北大)井冈山”三十八军真支持,我就给你派六十三军。你应该去作三十八军的工作。

\mxsay{江青:}聂元梓应该去作三十八军的工作,你们欢迎三十八军行不行?

\mxsay{毛泽东:}去一半三十八军,一半六十三军。三十八军不像你们讲得那么坏,根子在杨成武和北京军区。北京军区开了两个会,第一个会不太好,第二个会就比较好,郑维山\mnote{14}作了检讨。谭厚兰,其实你的炮位一直是在聂元梓身上,你谭厚兰这么女将轰了一炮,郑维山够紧张的。郑维山正好不在北京,到保定、山西解决问题去了,我们不是没见他吗?各军都不知道这个军长是好是坏,把将军们都吓坏了。他找你(指谭厚兰)的麻烦没有?

\mxsay{谭厚兰:}没有,同学对他有意见。

\mxsay{毛泽东:}过去是有历史原因,有点历史,这些事情不是偶然的,不是突如其来的。

\mxsay{陈伯达:}六六年上半年比较好,北京大专院校在全国煽风点火,搞革命风暴是对的。现在脑子膨胀了,自以为了不得,想要统一天下。蒯大富、韩爱晶到处伸手,又没有知识学问。

\mxsay{毛泽东:}二十几岁嘛,不能轻视年青人。周瑜出身起兵,才十六岁,你们不要摆老资格。

\mxsay{江青:}我们十几岁参加革命。

\mxsay{毛泽东:}不要膨胀起来,全身膨胀,全闹浮肿病。

\mxsay{陈伯达:}你韩爱晶对毛主席的思想,对中央意见没有好好考虑、思索,凭小道消息开秘密会议,个人第一,要走到危险道路上去。

\mxsay{毛泽东:}第一条是我官僚主义,一次未见你们。人家不要抓黑手,我还不会找你们哩。让蒯大富猛醒过来。

\mxsay{陈伯达:}蒯大富,你应该猛醒过来,悬崖勒马,道路是危险的。

\mxsay{林彪:}悬崖勒马,承认错误。

\mxsay{毛泽东:}不要说承认错误。

\mxsay{陈伯达:}蒯大富不尊重工人群众,再不听就是不尊重中央,不尊重毛主席,这是危险的道路。

\mxsay{毛泽东:}是相当危险,现在是轮到小将犯错误的时候了。

\mxsay{周恩来:}主席早就讲,现在是轮到小将犯错误的时候了。

\mxsay{林彪:}蒯大富,我们对你的态度是通过卫戍区和市革委会的,你说你不了解中央的态度,今天是毛主席亲自关心你们,作了最重要的、最正确的、最明确的、最及时的教导,这次还置若罔闻,要犯很大的错误。你们红卫兵在文化大革命中起了很大作用,现在全国很多学校实现了革命大联合。大联合问题,你们有些学校落后了,要赶上去,你们没有看到运动每个时期需要干什么。

\mxsay{毛泽东:}谭厚兰那里对立面只有二百多人,一年还不能压服。其它学校对立面更大了,怎么能征服呢?曹操用武力征服孙权,赤壁打了败仗。刘备要用武力征服孙权,也打了败仗。孔明想征服司马懿也不行,失了街亭,司马懿要征服诸葛亮也不成,头一仗打得很长,张邰只剩了个马。

\mxsay{林彪:}打走资派是好事情。文艺界的牛鬼蛇神也必须斗。现在有些人不是搞这个,而是搞学生斗学生,群众斗群众,他们大多是工农子弟,被坏人利用。有的是反革命,有的是开始革命,慢慢革命性少了,走向反面。有的主观上要革命,但客观上行动是相反的。有一小撮人主观客观都是反革命。你们脱离群众。

\mxsay{毛泽东:}工农占全国总人口百分之九十几。你们学校百分之九十以上是比较好的,打内战的比较少,北京只有六所。

\mxsay{谢富治:}清华二万人,参加武斗的不到五千人。

\mxsay{林彪:}那些不参战的人就是不同意。

\mxsay{毛泽东:}他们也是上了老虎的背,想下也没个好办法下。蒯大富可以下来嘛,下来照样当官作老百姓。蒯大富应该欢迎工人。

\mxsay{谢富治:}工人手无寸铁,只带三件武器:一是毛主席语录,二是毛主席最新指示,三是“七·三”布告。

\mxsay{康生:}清华的枪是北航给的,支持清华两汽车枪。蒯大富是司令,韩爱晶是政委。

\mxsay{韩爱晶:}没那回事,根本没那回事。卫戍区到我们那里去检查了好几次,枪一支不少。

\mxsay{谢富治:}就都是你正确,又全是你对。我批评了你几次你都不接受,你毫无自我批评。

\mxsay{陈伯达:}是否把他的枪给收回来?

\mxsay{韩爱晶:}主席,我有一个要求,给我派一个解放军监督我。很多事情不是那么回事。我是很爱蒯大富的,我也知道,跟他很多事情要受牵连。但我觉得,要努力保他,不让他倒台。他的命运与全国红卫兵的命运是有联系的。给我派了解放军,这样什么事情都清楚了。

\mxsay{陈伯达:}没有自我批评精神。

\mxsay{江青:}我有错误,宠了你。谢富治,你比我还宠,宠坏了,现在下点毛毛雨,还是主席这个办法好。

\mxsay{毛泽东:}不要老是批评。杨成武搞多中心,国防科委搞多中心论。全国可以搞几千个几万个多中心。都是中心就没有中心。各人皆以为自己天下第一,还有什么中心?

\mxsay{江青:}韩爱晶,我批评了你好几次,你就一直没给很好表个态。

\mxsay{毛泽东:}不要说他。你们专门责备人家,不责备自己。

\mxsay{江青:}我是说他太没有自我批评精神。

\mxsay{毛泽东:}年青人听不得批评,他的性格有点像我年青的时候。孩子们就是主观主义强,厉害得很,只能批评别人。

\mxsay{江青:}蒯大富有点笑容了。轻松一下,别那么紧张。(蒯报告,“(清华)井冈山”总部陈育延是女同学,被工人抓了。)陈育延出来没有?陈育延是个女孩子,要保护。

\mxsay{蒯大富:}陈育延在北航睡觉呢!

\mxsay{毛泽东:}你们要抓黑手,黑手就是我。对你毫无办法。我们倾向你们这一派。“四·一四”必胜思想我不能接受。但要争取他们中间群众,包括领袖中一些人。周家缨的主要观点是打天下的人不能坐天下,说蒯大富只能把权交给“四·一四”。我们叫工人去作宣传,你们拒绝,黄作珍、谢富治讲了话,毫无办法。工人是徒手,你们拒绝,打死打伤工人。正像北大一样,我们倾向聂元梓一样,偏向你们五大领袖,你不知道几万人到清华去干什么事情?没有中央决定他们敢?你们很被动,“四·一四”反而欢迎,“(清华)井冈山”反而不欢迎,你们搞得不对头。今天来的就没有“四·一四”,“(北大)井冈山”、“四·一四”思想不对嘛,“(北大)井冈山”、“红旗飘”中坏人多一些,聂元梓一派好人多一些。

\mxsay{聂元梓:}王、关、戚\mnote{15}插了一手。

\mxsay{毛泽东:}你们反王、关、戚好嘛。你们搞串连,我也禁止不了。韩爱晶、蒯大富你们不是好朋友吗?你们两个以后还要作好朋友。韩爱晶以后要帮助他,政策上作得好一些。现在 “四·一四”高兴,认为“(清华)井冈山”要垮了。我就不信,我看井冈山还是井冈山。前年我就上了井冈山,我不是说你的老佛爷的“(北大)井冈山”。

\mxsay{姚文元、谢富治:}革命的井冈山!

\mxsay{江青:}不要搞得我们爱莫能助。

\mxsay{毛泽东:}有很多打工人的,不是你们,听说是外地来的。

\mxsay{周恩来:}你们那里还有没有人呢?

\mxsay{蒯大富:}有。

\mxsay{毛泽东:}今天晚上睡觉。你们都还没有睡觉呢,蒯大富你没有地方睡觉到韩爱晶那里去睡,韩爱晶好好招待。韩爱晶,你要好好招待他。你们几个人找到一起,都到韩爱晶那里去,休息一下,然后开个会。

\mxsay{周恩来:}韩爱晶,你能帮他想点办法。

\mxsay{毛泽东:}蒯大富,你真蠢哪,我们搭梯子让你下来,你不下来。你们这样和中央的政策对抗,黄作珍讲话不听,谢富治讲话不听,市委开会不算数,中央才出来,伸出‘黑手’,调动革命,制止武斗。宣传多大,敲锣打鼓,你们又不理,你们脱离群众,脱离工农兵,脱离绝大部分学生,甚至脱离自己领导下的部分群众,你领导下的学生,说你的坏话的不少。清华直接没打招呼,间接是打了招呼的。

\mxsay{吴德:}昨天我找蒯大富谈过,他不听。

\mxsay{毛泽东:}“四·一四”欢迎工人,你们“(清华)井冈山”很蠢,很被动。我才不高兴那个“四·一四”。

\mxsay{江青:}“四·一四”是骂我的。

\mxsay{毛泽东:}他们抬尸游行,他们搞砸电缆。在这个时候,“四·一四”也没有知道,为什么他们欢迎?这一次你们很蠢,让“四·一四”欢迎工人。

\mxsay{江青:}就是“四·一四”的群众,他们也说蒯大富偏左,沈如槐(“四·一四”的负责人)偏右的。清华搞大联合,没有蒯大富还是不行的。

\mxsay{毛泽东:}蒯大富,你能不能当校长?“(清华)井冈山”二人,“四·一四”一人,沈如槐当副校长。

\mxsay{蒯大富:}我不能当了,当不了。

\mxsay{毛泽东:}还是要联合,是要蒯大富,没有蒯大富是不行的,蒯大富是偏左的,“(清华)井冈山”两个。“四·一四”是右倾的。

\mxsay{江青:}现在你们五个先做起来,反正先不要打。

\mxsay{毛泽东:}第一军管。第二一分为二。“四·一四”分一个,你蒯大富分一个。第三斗批走,这就是提出来的,他们不愿干了。你们一不斗,二不批,三不改,集中精力打内战,当然打内战是几个月。第四把工人撤出来,把枪都还给你们,无非是大打,要打就大打。文科要不要办呢?文科还是要办的。至于如何办法,研究出另一个办法,过去的办法是培养修正主义的。

\mxsay{谭爱晶:}师范大学要不要办?

\mxsay{毛泽东:}不办,谁教高中?谁教中专?外语学院不办怎么行?一风吹不行,吹那么几年也可以,天塌不下来。欧洲大战一打几年,不但大学没有办,其实中学、小学也都没办。鸡飞狗跳的。

\mxsay{江青:}改是个艰苦的工作,你们屁股坐不下来。

\mxsay{毛泽东:}学问不是在学校里学出来的,林彪刚才不讲了吗?他们学文,哪里学来的,难道是黄埔大学学来的?黄永胜学了一年半,温玉成你是幸运的,你上了三年了,你是黄岗的?也就认识几个字,社会是个最大的大学嘛,坐在那个搂里怎么能行。整个社会是个最大的大学,列宁大学读了一年半,恩格斯中学没读完。我们两个比高尔基高明得多,高尔基只上过两年学。华罗庚数学家就是个中学生,自学的。苏联卫星上天,祖宗是中学教员。发明蒸气机的人是工人,不是什么大学教师,是工人。我看我们的一些孩子,读书十几年把人毁了,睡不着觉,一个孩子读历史,不懂阶段斗争,历史就是阶段斗争的历史,可是读了好几年,就是不懂阶级斗争。

\mxsay{江青:}读那些厚本本,几十种。而马克思、恩格斯和毛主席的书都成了参考资料,辅助材料,他们老师的书才是正式教材。

\mxsay{毛泽东:}小学六年太长,中学六年太长,荒废无度。不要考试,考试干什么?一样不考才好哩!对于考试一概废除,搞个绝对化,谁考马、恩、列、斯,谁考林彪同志,谁考我?谢富治同志,把他们(学生)统统都招回来,统统回学校,可能有些生了气,不勉强,把“四·一四”留在学校里,“(清华)井冈山”反而在外面,这样不好,“(清华)井冈山”统统到人大会堂来。对“四·一四”的头头要有所区别,分别对待。

\mxsay{韩爱晶:}主席,我问一个问题。如果几十年以后,一百年以后中国打起内战来,你也说是毛泽东思想,我也说是毛泽东思想,出现了割剧混战的局面,怎么办?

\mxsay{毛泽东:}这个问题问得好,韩爱晶你还小,不过你问我,我可以告诉你,出了也没啥大事嘛!一百多年来,中国清朝打二十年,跟蒋介石打了几十年,中国党内出了陈独秀、李立三、王明、博古、张国焘,什么高岗,什么刘少奇,多了。有了这些经验比马克思还好。

\mxsay{林彪:}有毛泽东思想。

\mxsay{毛泽东:}文化大革命的经验比没文化大革命好,但我们保证要好些,你们要跟人民在一起,跟生产者在一起,把他们消灭干净,有人民就行,就是把林彪以及在座都消灭,全国人民是灭不掉的,不能把中国人民都灭掉,只要有人民就行,最怕脱离工人、农民、战士,脱离生产者,脱离实际,对修正主义警惕性不够,不修也得修。你看朱成昭\mnote{16}刚当了几天司令,就往外国跑或者“保爹保妈”\mnote{17},就不干了。聂元梓攻她哥哥姐姐的不好来攻她。你那个姐姐也不那么坏嘛,聂元梓,哥哥、姐姐为什么一定和她联系起来呢?

\mxsay{周恩来:}我的弟弟周永爱跟王、关、戚混在一起,我把他抓送到卫戍区去了。

\mxsay{毛泽东:}我那个父亲也不高明,要是现在也得坐喷气式。

\mxsay{林彪:}鲁迅的弟弟是个大汉奸嘛。

\mxsay{毛泽东:}我自己也不高明,读了哪个就信哪个。以后又读了七年,包括在中学读半年资本主义,至于马克思主义,一窍不通。不知道世界上还有马克思,只知道拿破仑、华盛顿。在图书馆读书实在比上课好,一个烧饼就行了,图书馆的老头都跟我熟了。

\mxsay{陈伯达:}韩爱晶过去就是提过这个问题,有林彪同志这个毛主席的好接班人,有毛泽东思想,我不怕出修正主义。

\mxsay{毛泽东:}不能保证这次文化大革命以后就不搞文化革命了,还是会有波折的。不要讲什么新阶段,好几个新阶段,我讲上海机床厂,又是什么新阶段。一次文化革命可能不够。

\mxsay{姚文元:}这个问题,主席已经讲过了。

\mxsay{周恩来:}林彪同志主席著作学得好,包括苏联在内对马列著作都没掌握好,林副主席掌握了。

\mxsay{毛泽东:}党内出了陈独秀,党就没有啦?党犯错误,党还是有的,还是要革命的,军队还是要前进的。第四次王明路线那么长还不是纠正了,张闻天搞了十年也不高明。灾难多了,解放后又是多少次?我们这个党是伟大的党,光荣的党,不要因为出了刘少奇、王明、张国焘,我们党就不伟大了。你们年轻人就是没有经验,上帝原谅你们。韩爱晶你问起我,我答复你了,不要以为我们这些人有什么了不起,不要以为有我们这些人在就行,没有我们这些人天就掉下来了。

\mxsay{江青:}韩爱晶给我写几次信,讲这个问题,韩爱晶为什么提出这个问题?一是脱离工农,二是脱离实际。一到我跟前就想将来,总说几十年以后的事。还问我第三次世界大战什么时候打?

\mxsay{毛泽东:}想得远好。这个人好啊!这个人好啊!我们有几种死法,一是炸弹炸死,二是病死,被细菌钻死,三是被火车、飞机砸死,四是我又爱游泳,被水淹死,无非如此。最后一种是寿终正寝,还是细菌么!薄一波差点死了,听说刘少奇也救活了,一种肺炎,一种心脏病,还有肾感染,四个医生和两个护士抢救,可以说脱离危险期了,你们听说了吗?\mnote{18}

\mxsay{大家说:}没听说。

\mxsay{姚文元:}历史发展规律总是前进的,曲折的,前途是光明的。相信毛泽东思想,相信群众。韩爱晶,你是个悲观主义者,对共产主义没有信心。

\mxsay{韩爱晶:}我相信共产主义一定会胜利,如果我对共产主义没有信心,我就不会献身共产主义事业,可是我认为,历史的发展是波浪式的,不可能是条直线,难道中国革命,由民主革命到社会主义革命到共产主义就是一条直线走向胜利吗?不会出现反复吗?不是波浪式吗?按照辩证法肯定有曲折。

\mxsay{毛泽东:}一次前进是没有的,历史前进总是曲折的。一九二七年受挫折,二三次受挫折,胜了以后,又出现高饶反党联盟,庐山会议以后,出了彭德怀。现在有走资派,像蒯大富那个“彻底砸烂旧清华”,“四·一四”就不赞成,“四·一四”就说,教员也有好的,可你们说的彻底砸烂,不是砸烂好人,而是一小撮坏人,你把含义讲清楚,他就驳不倒了,赶快把六七个领导找来,集中起来,你们今天晚上睡个觉,明天再开会,散会算了,以后再来。

\mxsay{韩爱晶:}(握着主席手)主席,我一定为您的革命路线奋斗终生。

\mxsay{蒯大富:}(握着主席手)主席,谢谢您,祝您万寿无疆。

\mxsay{毛泽东:}(走了又返回来对中央领导)我走了,又不放心,怕你们又反过来整蒯大富,所以又回来了。不要又反过来整蒯大富啦,不要又整他们。

\begin{maonote}
\mnitem{1}接见红代会五大领袖,一九六八年七月二十八日凌晨三点,在人大会堂湖南厅,毛泽东接见了首都红卫兵代表大会的“五大领袖”,指清华的蒯大富、北大的聂元梓、北京航空学院的韩爱晶、北京师范大学的谭厚兰、北京地质学院的王大宾,历时五个多小时。陪同接见的有中共中央副主席、国防部长林彪元帅、国务院总理周恩来、中央文革小组组长陈伯达、文革小组顾问康生、毛主席夫人文革小组副组长江青、文革小组成员姚文元、林彪夫人叶群、中央办公厅主任中央警卫团负责人汪东兴、国务院副总理、公安部长、北京市革委员会主任、北京军区政委谢富治、解放军总参谋长黄永胜、空军司令员吴法宪、副总参谋长北京卫戍区司令温玉成、北京卫戍区政委黄作珍、北京市革命委员副主任吴德。
\mnitem{2}一九六八年夏季,文化大革命在政治上全面夺权的任务基本完成,北京几所大学的两派武斗却战火不断。为了解决这个“老大难”问题,在毛泽东的亲自指示下,北京数万名工人组成了“首都工人毛泽东思想宣传队”,开始进驻首都各大专院校;然而进驻遇到阻力。七月二十七日上午,当数千名工人组成的工宣队进驻北清大学收缴武器、拆除工事、制止武斗时,在北清大学遭到了顽强的武装抵抗。特别是在北清东校,蒯大富领导的井冈山兵团用长矛枪支和手榴弹袭击了工宣队,死五人,伤数百人,他们在高音喇叭里呼喊的口号是:“打倒镇压学生运动的黑后台!”“镇压学生运动绝无好下场!”
\mnitem{3}北大当时有两个相互对立的造反派红卫兵组织,一个是指聂元梓领导的“新北大公社”,另一个是“井冈山兵团”,负责人是牛辉林、侯汉清、杨克明。
\mnitem{4}北京地质学院的造反派红卫兵组织“东方红公社”负责人是王大宾、聂树人等。
\mnitem{5}周泉缨,清华大学当时有两个相互对立的造反派红卫兵组织,蒯大富、鲍长康领导的“井冈山兵团”,简称“团派”,另一个就是沈如槐等领导的“四·一四”组织,简称“四派”,周泉缨是“四派”的理论家,曾写出《“四·一四”思潮必胜》的大字报。井岗山兵团的天安门纵队在清华《井岗山》报上发表《谁敢否定无产阶级文化大革命我们就和他拼命——评〈四一四思潮必胜〉》。
\mnitem{6}“省无联”正确全称是“湖南省会无产阶级革命派大联合委员会”,“省无联”是其简称,只是个省会组织,不是全省组织,它只是部分群众造反组织的一个派系联席会议式的松散集合,而其对参入者并无任何约束力。
\mnitem{7}一九六七年九月七日,在蒯大富、韩爱晶的支持和参与下,北师大谭厚兰的反对派“井冈山造反兵团”的头头们趁开大会之机,搞突然袭击,把谭厚兰反剪双手押上了主席台,批斗了谭厚兰,并宣布夺权。这就是颠覆新生的红色政权北师大革委会事件,又称“九·七事件”。事后,蒯大富、韩爱晶等人受到了毛主席和中央文革的严厉批评,并被迫向谭厚兰道了歉。
\mnitem{8}胡万春,一九二九年生于上海一个贫苦工人家庭。从小失学,十三岁当童工,十七岁进上海钢铁厂当工人。建国后,曾先后担任过工厂工会委员、副主席,党宣传部长,上海市第三、四届人民代表大会代表,中国作家协会和作家协会上海分会理事等。一九五二年开始创作,一九五五年发表短篇小说《青春》,描述青年男女的新型恋爱关系。一九五六年写了自传体小说《骨肉》,描述旧社会一个工人家庭家破人亡,骨肉分离的悲惨遭遇,受到好评,并在一九五七年世界青年联欢节举办的国际文艺竞赛中获荣誉奖状。在当时成为全国知名的作家。他的作品洋溢着强烈的阶级感情,具有浓厚的生活气息。
\mnitem{9}高玉宝,一九二七年生于辽宁瓦房店。八岁上小学一个月,九岁当童工,十五岁当劳工,十七岁学木匠,一九四七年参加解放军,一九五四年他才上学读书,一九六二年毕业于中国人民大学新闻系。历任文艺干事,师职创作员,辽宁省民间文学协会理事,深阳军区创作室名誉主任,短篇小说《我要读书》和《半夜鸡叫》曾被选入小学语文课本,著有长篇小说《高玉宝》,一九五五年四月中国青年出版社初版。长篇小说《高玉宝》在国内用七种民族文字出版,并被十多个国家和地区用十五种外文翻译出版,仅汉文出版的就达四百五十多万册,并被改编为二十四种连环画,被周恩来总理称为“战士作家”。
\mnitem{10}广西布告,即“七·三”布告,指中共中央、国务院、中央军委、中央文革一九六八年七月三日布告。布告说,最近两个月来,在广西柳州、桂林、南宁地区,连续发生了一系列反革命事件。中央认为,这是一小撮阶级敌人破坏无产阶级专政、破坏抗美援越斗争、破坏无产阶级文化大革命的反革命罪行。为了迅速予以制止,中央号召广西无产阶级革命派和广大革命群众,在广西革筹小组的领导下,努力实现以下各点:一,立即停止武斗,拆除工事,撤离据点。首先撤离铁路交通线上的各据点。二,无条件地迅速恢复柳州铁路局全线的铁路交通运输,停止一切干扰和串连,保证运输畅通。三,无条件地交回抢去的援越物资。四,无条件地交回抢去的人民解放军武器装备。五,一切外地人员和倒流城市的下乡上山青年,应立即返回本地区、本单位。六,对于确有证据的杀人放火、破坏交通运输、冲击监狱、盗窃国家机密、私设电台等现行反革命分子,必须依法惩办。
\mnitem{11}陕西布告,即“七·二四”布告,指中共中央、国务院、中央军委、中央文革一九六八年七月二十四日的布告。主要是严厉制止日益蔓延的破坏交通、抢劫军用列车、冲击解放军机关、杀伤解放军指战员和大规模武斗。

布告说,最近以来,在陕西省的一些地方,连续发生了一些极其严重的反革命事件。中央认为,这是属于一小撮阶级敌人破坏无产阶级专政、破坏无产阶级文化大革命、破坏国家社会主义建设的反革命罪行。为了迅速予以制止,中央特再重申:

(一)任何群众组织、团体和个人,都必须坚决、彻底、认真地执行伟大领袖毛主席亲自批准的“七·三”布告,不得违抗。

(二)立即停止武斗,解散一切专业武斗队,教育那些受蒙蔽的人回去生产。拆除工事、据点、关卡。

(三)抢去的现金、物资,必须迅速交回。

(四)中断的车船、交通、邮电,必须立即恢复。

(五)抢去人民解放军的武器装备,必须立即交回。

(六)对于确有证据的杀人放火,抢劫、破坏国家财物,中断交通通讯,私设电台,冲击监狱、劳改农场,私放劳改犯的现行反革命分子以及幕后操纵者,必须坚决实行无产阶级专政,依法惩办。
\mnitem{12}杨余傅,杨,指杨成武,原任中共中央军委常委兼副秘书长、全军文革小组副组长、中国人民解放军代总参谋长。余,指余立金,原任中国人民解放军空军政委。傅,指傅崇碧,原任中国人民解放军北京卫戍区司令员。一九六八年的三月二十四日,林彪及中央“文革小组”成员在人民大会堂接见了人民解放军各总部、国防科委、国防工办、各军兵种、北京军区等单位团以上干部一万多人。林彪代表中央宣布重要决定,“最近从空军中发生了杨成武同余立金勾结,要篡夺空军的领导权,要打倒吴法宪;杨成武同傅崇碧勾结,要打倒谢富治。杨成武的个人野心,还想排挤许世友,排挤韩先楚,排挤黄永胜以及与他们地位不相上下的人。中央在主席那里最近接连开会,开了四次会,主席亲自主持。会议决定,撤销杨成武的代总长职务,要把余立金逮捕起来,法办!撤销北京的卫戍司令傅崇碧的职务。”
\mnitem{13}“天派”“地派”,文革时有名的红卫兵的组织是北京“天派”和“地派”。与其他一些有两大派对立组织的地区的情况不同,它的所谓“派”,并不存在一个实在的、具体的常设组织,严格说来,它只是一个观点相同或相近的一些组织不很紧密的联合体。“天派”和“地派”的形成不是一朝一夕所完成的,它是在一系列派性活动中逐渐形成的。一九六七年七月底、八月初,北京高校“天派”和“地派”的说法逐渐开始流行。所谓“天派”,是因为这一派的主要组织之一是北京航空学院“红旗”,主要领导人为韩爱晶。是和天有关的,所以称为“天派”;所谓“地派”,是因为这一派的主要组织之一是北京地质学院“东方红”,就取它的“地”字,称为“地派”,天地相对,用“天派”、“地派”来称呼北京对立的两大派很形象,也很通俗,于是很快就流行开了。

聂元梓的“新北大公社”是“天派”的,对立派“北大井冈山”是“地派”的;蒯大富的“清华井冈山”是“天派”的,对立派沈如槐“四·一四兵团”是“地派”的;北师大谭厚兰(女)的“井冈山公社”是“地派”的,对立派王颂平(女)的“造反兵团”是“天派”的。

“五大领袖中”聂元梓(女)、蒯大富、韩爱晶是“天派”的,谭厚兰(女)、王大宾是“地派”的。
\mnitem{14}郑维山,时任北京军区司令员。
\mnitem{15}王、关、戚,指的是王力、关锋和戚本禹三人,一九六六年后,王力、关锋和戚本禹相继成为《红旗》副总编辑、“中央文革小组”的成员,曾为中央文革成员,王力被任命为中央宣传组组长,关锋成为总政副主任、“军委文革小组”副组长,并受林彪委托兼管《解放军报》,而戚本禹则是中央办公厅秘书局副局长、中共中央办公厅代主任。一九六七年八月七日,王力对外交系统造反派发表讲话(即著名的“王八七讲话”),一些极左分子受到王力讲话的蛊惑,一度夺取了外交大权。八月二十二日,在外交部造反派的策划组织下,火烧了英国驻华代办处,导致了文革期间最严重的外交事件。另外,王力、关锋在当年当期的《红旗》杂志上,组织发表了《红旗》杂志的“八·一”社论,提出了“揪军内一小撮走资派”与“有带枪的刘邓路线”口号,十二日,毛泽东看后写了“大毒草”三个字!批示:“还我长城!”,二十四日,毛泽东在王力“八七”讲话稿上写上了:“大大大毒草”,二十六日,毛泽东指示,“王、关、戚是破坏文化大革命的,不是好人”,随后,王力、关锋被隔离审查,一九六八年一月,对戚本禹也做了同样处理。
\mnitem{16}朱成昭,北京地质学院“东方红公社”的早期负责人,其女朋友是叶剑英之女叶向真,曾在一九六六年十二月组织人将彭真从家中抓走批斗,把彭德怀从成都抓到北京,六七年初,朱成昭当副局长的父亲在上海也被打成了黑帮,思想开始保皇,反对中央文革和文化大革命,二月四日,公开炮打中央文革,由王大宾接任“东方红公社”的一把手。一九六七年七月,朱成昭和叶向真准备偷渡香港,被周总理派人抓回,八月二十日,朱成昭和叶向真被公安部以组织“反革命集团”罪名正式逮捕。
\mnitem{17}“保爹保妈”,文革中的红卫兵分为“老红卫兵”和“造反派红卫兵”,“老红卫兵”活跃在文革初期,即一九六六年八月到一九六七年初,不到半年时间,以“红五类”子弟(即家庭出身为工人、贫农、下中农、革命干部、革命军人的学生)为基本队伍,以高干军干子女为头目的第一批红卫兵组织,如北京的“红卫兵联合行动委员会”(由原北京东城、西城红卫兵纠察队等组成,简称“联动”),宣扬“血统论”,公开声明“老子英雄儿好汉,老子反动儿混蛋”,他们的目标就是保住自己父母的权力,保省委保市委,充当了文革期间首批保守派。一九六七年“一月风暴”中,“造反派红卫兵”登上历史舞台,其组织构成已不注重家庭成份,其领导层也大多是平民子弟,开始“造”省委市委及学校党委的“反”,击败并瓦解了“老红卫兵”,并讥讽“老红卫兵”是“保爹保妈”。
\mnitem{18}抢救刘少奇,指一九六八年七月中旬,刘少奇病危,经过十多天抢救,脱离了生命危险。四名医生是指陶桓乐、黄宛、董长城、顾英奇,两名护士指李留壮和马小先。

二〇〇九年十一月十五日《重庆晚报》刊登了《文革中抢救刘少奇纪实》一文,讲述者是亲历者顾英奇,历任中共中央办公厅警卫局保健处医生,北京医院主治医师,中央警卫团中南海门诊部副主任,总参警卫局保健处主任军医,中国康复医学会会长,卫生部副部长等职。该文全文如下:

从一九六八年二月到一九六九年十月,中南海门诊部的医务人员和全国知名的各科专家,为刘少奇做了大量的治疗护理工作,在他病重、病危期间进行了卓有成效的抢救工作,多次把他从死亡边缘抢救回来。

一、门诊部一开张就遇到给刘少奇看病的问题

少奇同志于一九六三年九月搬到中南海福禄居。一九六七年一月,我从下乡医疗队回到北京医院总值班室上班。当时医院虽然混乱,但对刘少奇看病的问题,周总理和中央办公厅曾给北京医院下达一条原则:刘少奇需要看病时,经他的警卫人员与北京医院总值班室联系,由医院的医生、护士出诊;药品还是从保健药房(文革中编入中央警卫团后勤部)发给。据刘少奇病历记载,北京医院曾有四位医生到他家出诊过,为解决疑难问题也曾在他家多次组织过会诊。

一九六八年二月,奉国务院和中央军委命令,我被调回中南海并编入部队,任新成立的中央警卫团(八三四一部队)中南海门诊部(现警卫局保健处前身)两名负责人之一,主持日常工作。之后,根据上级指示我们从北京医院接过刘少奇的医疗任务和病历。领导给我们的指示是:“刘少奇如果生病,叫你们去看病,你们就去,还要认真给他治病。”

当时,我们先指派助理军医李留壮和护士马小先负责平时到刘少奇家的巡诊工作,门诊部的其他医生、护士也到他家出诊。医护每次出诊都做认真的检查、治疗并详细记录于病历。

那时,少奇同志的活动范围仅限于庭院和室内,他的身体已经十分虚弱。他原来患有糖尿病、冠心病、慢性支气管炎、肺气肿。他情绪很低落,食欲下降,血压很高,糖尿病加重,身体消瘦,体力大为减弱。

一九六八年四月,少奇同志开始语言减少,有时糊涂,尿失禁,手抖,步子变小(碎步)。为了查明病因,我们即请北京医院神经内科王新德主任会诊,王主任虽未肯定脑部有局灶性病变,但肯定是脑供血不足引起的病态。这段时间,少奇同志还是照常下地活动,在卫士或护士的搀扶下散步,一直到他重病卧床不起时为止。

在这期间,我们不但经常去给他看病,还对他的生活照顾及时提出指导意见:一九六八年五月三日,发现给刘少奇做饭的师傅马文全患痢疾,便立即进行了隔离治疗,另换一名师傅;同时把他的厨房、餐厅、餐具都做了消毒,以保护他的健康。

六月十八日,针对他血压增高,血糖波动等问题组织了会诊,调整了治疗药物;因胆固醇摄入过高对他不利,把他原先每天吃六个鸡蛋进行了调整,改为每天两个;并提出肉类和蔬菜要适当搭配,少吃些猪肉,多吃些牛羊肉、豆制品、蔬菜等。采购人员和厨师都很配合,都认真做到了。

二、抢救刘少奇

在工作中,我们虽然不能像以往那样和他有思想感情交流,但我们确实严格按照医患之间的关系来处理他的健康和医疗问题,没有任何轻视和懈怠。

当时,少奇同志身体已经相当虚弱,免疫力较低,易病。一九六八年六月初他受凉感冒,虽是小病,但疗效却较差。七月六日起病情又渐重,七月九日发烧、咳嗽加重,肺部罗音增多,我们看后初步诊断为肺炎。当时即派护士马小先住在他家进行护理。门诊部医生会诊研究了治疗措施,并当即将病情上报。

当时,毛主席、周总理都明确指示说:“要千方百计地给他治病。”根据这一指示,我们即请北京、上海的知名专家会诊,并请陶桓乐、黄宛两教授和中南海门诊部医生董长城和我住在刘少奇家。因没有足够的床铺,我和董长城就在地板上搭地铺。同时安排四名护士参加护理工作,日夜值班。就此组成医疗组。

当时,是汪东兴向我传达毛主席、周总理的指示。在刘少奇病重时,周总理派他的保健医生卞志强(也是中南海门诊部负责人之一)几乎每天都来了解情况,只是要尽力救治。

三、七次从凶险的肺炎中把他抢救回来

经X线胸片及痰培养证实,他患的是“肺炎杆菌性肺炎”。肺炎杆菌毒力较强,耐药,较难控制,这是一种十分凶险的疾病,对老年人常常是致命的。

七月十二日,少奇同志病势渐沉重,高烧不退,神志不清,谵妄,痰咳不出,有阻塞气道的危险。

专家会诊提出,必要时需要做气管切开。这需要先请示得到批准,以便关键时刻立即施行。报告后,很快汪东兴即传达毛主席、周总理的指示:“如果病情需要,即同意医生的决定。”我们当即请耳鼻喉科、麻醉科专家(负责呼吸管理)住到刘少奇家,准备需要时及时做气管切开手术。

此次肺炎至七月二十二日基本控制,但少奇同志的意识没有恢复。他虽然也是夜里睡觉,白天醒来,睁着眼睛,头左右转动,但没有思维活动,不能说话,认知力丧失,熟人也不认识(医学上叫醒觉昏迷)。从此一直卧床不起,吃饭靠护士喂,大小便不能自主,靠护士照顾。

由于自身免疫力太低,所以肺炎反复发作七次(一九六八年五次,一九六九年二次)。在这期间曾多次出现病危,均经抢救、治疗得以转危为安。

四、脑软化日益严重,意识一直没有恢复

一九六八年十月九日,他突然不能进食,头向左转、眼向左凝视,诊断为脑供血不足,为脑干有弥漫性小软化灶所致。十月十一日开始鼻饲,由于炊事员与医护人员的密切配合,使每日总热量保持在一千五百千卡——一千七百千卡。因此,少奇同志到一九六九年,虽然意识、认知力、语言、记忆功能丧失,但体重增加,面色红润,枕部的头发变黑;虽然生活不能自理,长期卧床,大小便失禁,但没有发生过褥疮,这都是由于护士按护理规程,按时按摩、翻身、擦澡、被褥清洁才得到这样好的结果。

按照周总理的指示,我们留下了他在“家庭病房”里的照片。照片上刘少奇虽然已经不能认人和说话,但他头部自由转动,眼睛明亮,注视着景物。“家庭病房”窗明几净,床单雪白,器物整齐。

在一九六八年七月至一九六九年八月六日期间,请专家会诊共四十次,仅一九六八年七月会诊即达二十三次。一般上午、晚上各一次,有时一日会诊三次。参加会诊的专家有:上海的董承琅、北京的吴洁、陶桓乐、王叔咸、李邦琦、黄宛、王新德、薛善一、吴家瑞、姜世杰等教授专家。中南海门诊部参加医疗工作的有:卞志强、董长城、张林、牛福康、李留壮、马小先。参加特护的有:北京医院的曹兵(瑞英)、纪秀云,警卫团的韩世泉(男)、三〇一医院的董洁秋、卿喜珍等。

为刘少奇治疗所用的药品,都是由保健药房供应的。绝大部分为进口药,尤其是所用的抗生素,大部分是当时国内临床医院所没有的。如果没有这些抗生素,是无法多次控制肺部感染的。

从一九六八年七月至一九六九年十月,医护人员天天守护在刘少奇床边,从未发生过差错。至于严重的精神创伤,免疫力低下,肺炎反复发作,脑组织软化,意识严重障碍等病情发展,实非当时以至目前医疗技术所能挽回的。

五、中南海门诊部派医护,把刘少奇护送到洛阳

少奇同志在我们的照顾下生活了一年零八个月后,一九六九年十月,上级指示要把刘少奇疏散到河南去。中南海门诊部派董长城医生和曹兵、纪秀云两位护士携带医疗抢救用品一直护送到洛阳,并向当地接班医生做了详细交班。

一九七九年十一月二十七日,中纪委的王绍棠把刘少奇的病历、治疗和抢救工作的全部资料取走,其中包括数张刘少奇在家庭病房中生活和治疗的照片。

党的十一届三中全会以后,党中央对两案的审查已有结论。其中对中南海门诊部给刘少奇治病和抢救是满意的。中办警卫局领导向我们传达了上述情况说:“门诊部为刘少奇治疗和抢救,以及所写的病历,受到两案审查组的表扬。认为记录清楚、内容翔实、字迹工整;反映了治疗、抢救、护理、专家会诊、治疗处置和医嘱执行等各项工作的每一个细节,真实可信,无懈可击。审察组认为门诊部的工作是认真负责的。”

刘少奇的这段病历大部分是董长城医生执笔的。
\end{maonote}
