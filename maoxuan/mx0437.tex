
\title{再克洛阳后给洛阳前线指挥部的电报}
\date{一九四八年四月八日}
\thanks{这是毛泽东为中共中央起草的电报。因为它的内容不但适用于洛阳,也基本上适用于一切新解放的城市,所以这个电报同时发给了其它前线和其它地区的领导同志。}
\maketitle


此次再克洛阳\mnote{1},可能巩固。关于城市政策,应注意下列各点。

一、极谨慎地清理国民党统治机构,只逮捕其中主要反动分子,不要牵连太广。

二、对于官僚资本要有明确界限,不要将国民党人经营的工商业都叫作官僚资本而加以没收。对于那些查明确实是由国民党中央政府、省政府、县市政府经营的,即完全官办的工商业,应该确定归民主政府接管营业的原则。但如民主政府一时来不及接管或一时尚无能力接管,则应该暂时委托原管理人负责管理,照常开业,直至民主政府派人接管时为止。对于这些工商业,应该组织工人和技师参加管理,并且信任他们的管理能力。如国民党人已逃跑,企业处于停歇状态,则应该由工人和技师选出代表,组织管理委员会管理,然后由民主政府委任经理和厂长,同工人一起加以管理。对于著名的国民党大官僚所经营的企业,应该按照上述原则和办法处理。对于小官僚和地主所办的工商业,则不在没收之列。一切民族资产阶级经营的企业,严禁侵犯。

三、禁止农民团体进城捉拿和斗争地主。对于土地在乡村家在城里的地主,由民主市政府依法处理。其罪大恶极者,可根据乡村农民团体的请求送到乡村处理。

四、入城之初,不要轻易提出增加工资减少工时的口号。在战争时期,能够继续生产,能够不减工时,维持原有工资水平,就是好事。将来是否酌量减少工时增加工资,要依据经济情况即企业是否向上发展来决定。

五、不要忙于组织城市人民进行民主改革和生活改善的斗争。要等市政管理有了头绪,人心已经安定,经过周密调查,弄清情况和筹有妥善解决办法的时候,才可以按情况酌量处理。

六、大城市目前的中心问题是粮食和燃料问题,必须有计划地加以处理。城市一经由我们管理,就必须有计划地逐步解决贫民的生活问题。不要提“开仓济贫”的口号。不要使他们养成依赖政府救济的心理。

七、国民党员和三青团员,必须妥善地予以清理和登记。

八、一切作长期打算。严禁破坏任何公私生产资料和浪费生活资料,禁止大吃大喝,注意节约。

九、市委书记和市长必须委派懂政策有能力的人担任。市委书记和市长应该对所属一切工作人员加以训练,讲明各项城市政策和策略。城市已经属于人民,一切应该以城市由人民自己负责管理的精神为出发点。如果应用对待国民党管理的城市的政策和策略,来对待人民自己管理的城市,那就是完全错误的。


\begin{maonote}
\mnitem{1}洛阳是当时国民党军队在河南西部的一个重要据点。人民解放军于一九四八年三月十四日首次攻克洛阳,以后为便于歼灭敌人有生力量,又主动撤出。同年四月五日,人民解放军再度攻克该城。
\end{maonote}
