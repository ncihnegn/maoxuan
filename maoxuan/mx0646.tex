
\title{在八届十中全会上的讲话}
\date{一九六二年九月二十四日}
\thanks{这是毛泽东同志在八届十中全会上讲话的主要部分。}
\maketitle


现在是十点钟,开会。

这次中央全会解决了几个重大问题。一是农业问题,二是商业问题,这是两个重要问题;还有工业问题,计划问题,这是第二位的问题;第三个是党的团结问题。有几位同志讲话,农业问题由陈伯达同志说明;商业问题由李先念同志说明;工业计划问题由李富春、薄一波同志说明。另外还有监察委员会扩大名额问题,干部上下左右交流问题。

会议不是今天开始的。这个会开了两个多月了,在北戴河开了一个月,到北京差不多也是一个月。实际问题在八、九两个月。各个小组(在座的人都参加了),经过小组,实际上是大组,都讨论清楚了。现在开大会,不需要很多时间了,大约三天到五天就够了。二十七号不够就开到二十八号,至迟二十八号要结束。

我在北戴河提出三个问题,阶级、形势、矛盾。阶级问题,提出这个问题,因为阶级问题没有解决。国内不要讲了。国际形势,有帝国主义、民族主义、修正主义存在,那是资产阶级国家,是没有解决阶级问题的。所以我们有反帝任务,有支持民族解放运动的任务,就是要支持亚、非、拉三大洲广大的人民群众,包括工人、农民、革命的民族资产阶级和革命的知识分子,我们要团结这么多的人,但不包括反动的资产阶级,如尼赫鲁,也不包括反动的资产阶级知识分子,如日共叛徒春日藏次郎,主张结构改革论\mnote{1},有七、八个人。

那么,社会主义国家有没有阶级存在?有没有阶级斗争?现在可以肯定,社会主义国家有阶级存在,阶级斗争肯定是存在的。列宁曾经说,革命胜利后,本国被推翻的阶级,因为国际上有资产阶级存在,国内还有资产阶级残余,小资产阶级的存在不断产生资产阶级,因此被推翻了的阶级还是长期存在的,甚至要复辟的。欧洲资产阶级革命,如英国法国等都曾几次反复。封建主义被推翻以后,都经过了几次反复辟,经过了几次反复。社会主义国家也可能出现这种反复。如南斯拉夫就变质了,是修正主义了,由工人农民的变成一个反动的民族主义分子统治的国家。我们这个国家,要好好掌握,好好认识,好好研究这个问题,要承认阶级长期存在,承认阶级与阶级斗争,反动阶级可能复辟,要提高警惕,要好好教育青年人,教育干部,教育群众,教育中层和基层干部,老干部也要研究、教育。不然,我们这样的国家,还会走向反面。走向反面也没有什么要紧,还要来一个否定之否定,以后又会走向反面。如果我们的儿子一代搞修正主义,走向反面,虽然名为社会主义,实际是资本主义,我们的孙子肯定会起来暴动的,推翻他们的老子,因为群众不满意。所以我们从现在起,就必须年年讲,月月讲,天天讲,开大会讲,开党代会讲,开全会讲,开一次会就讲,使我们对这个问题有一条比较清醒的马克思列宁主义的路线。

国内形势,过去几年不大好,现在已经开始好转。一九五九年、一九六〇年,因为办错了一些事情,主要由于认识问题,多数人没有经验。主要是高征购,没有那么多粮食硬说有,瞎指挥,农业、工业都有瞎指挥。还有几个大办的错误,一九六〇年下半年开始纠正,说起来就早了,一九五八年十月第一次郑州会议\mnote{2}就开始了,然后十一月武昌会议\mnote{3},五九年二、三月第二次郑州会议\mnote{4},然后四月上海会议\mnote{5}就注意纠正。这中间,一九六零年有一段时间对这个问题讲得不够,因为修正主义来了,压我们,注意反对赫鲁晓夫了。从一九五八年下半年开始,他就想封锁中国海岸,要在我国搞共同舰队,控制沿海,要封锁我们,赫来我们就是为了这个问题。然后是五九年九月中印边界问题,赫支持尼\mnote{6}攻我们,塔斯社发表声明。以后赫压我国,十月在我国国庆十周年宴会上,在我们讲台上,攻击我们,六〇年布加勒斯特会议围攻我们,然后两党会议,二十六国起草委员会,八十一国莫斯科会议,还有一个华沙会议,都是马列主义与修正主义的争论。一九六〇年一月与赫打仗。你看社会主义国家,马列主义中出现这样的问题,其实根子很远,事情很早就发生了,就是不许中国革命。那是一九四五年,斯大林就阻止中国革命,说不能打内战,要与蒋介石合作,否则中华民族就要灭亡。当时我们没有执行,革命胜利了。革命胜利后,又怕中国是南斯拉夫,我就变成铁托。以后到莫斯科签订了中苏同盟互助条约,也是经过一场斗争的。他不愿意签,经过两个月的谈判签了。斯大林相信我们是从什么时候起的呢?是从抗美援朝起,一九五〇年冬季,相信我们不是铁托、不是南斯拉夫了。但是,现在我们又变成“左倾冒险主义”、“民族主义”、“教条主义”、“宗派主义”者了。而南斯拉夫倒成为“马列主义”者了。现在南斯拉夫很行呵,他可吃得开了,听说变成了“社会主义”。所以社会主义阵营内部也是复杂的,其实也是简单的。道理就是一条,就是阶级斗争问题。无产阶级与资产阶级斗争问题,马列主义与反马列主义斗争的问题,马列主义与修正主义之间的斗争问题。

至于形势,无论国际内外都是好的,开国初期,包括我在内,还有少奇同志曾经有这个看法,认为亚洲的党和工会、非洲党,恐怕会受摧残。后来证明,这个看法是不正确的,不是我们所想的。第二次世界大战后,蓬蓬勃勃的民族解放斗争,无论亚洲、非洲、拉丁美洲,都是一年比一年发展的。出现了古巴革命;出现了阿尔及利亚独立;出现了印尼亚洲运动会;几万人示威,打烂印度领事馆,印度孤立;西伊里安、荷兰交出来;出现了越南南部的武装斗争,那是很好的武装斗争;出现了武装斗争胜利的阿尔及利亚;出现了老挝这个胜利的斗争;出现了苏伊士运河事件;埃及独立,阿联偏右;出现了伊拉克,两个都是中间偏右的,但它反帝。阿尔及利亚不到一千万人口,法国八十万军队,打了七、八年之久,结果阿尔及利亚胜利了。所以国际形势很好。陈毅同志作了一个很好的报告。

所谓矛盾,是我们同帝国主义的矛盾,全世界人民同帝国主义的矛盾是主要的。各国人民反对反动资产阶级,各国人民反对反动的民族主义。各国人民同修正主义的矛盾;帝国主义国家之间的矛盾;民族主义国家同帝国主义国家之间的矛盾;帝国主义国家内部的矛盾;社会主义与帝国主义之间的矛盾。

中国的右倾机会主义,看来改个名字好,叫做中国的修正主义。从北戴河到北京两个多月的会议,是两种性质的问题,一种是工作问题;一种是阶级斗争问题,就是马克思主义与修正主义的斗争。工作问题也是与资产阶级思想斗争的问题,也是马列主义与修正主义斗争的问题。

工作问题有几个文件,有工业的、农业的、商业的等,有几个同志讲话。

关于党如何对待国内、党内的修正主义问题、资产阶级的问题,我看还是照我们的历来方针不变。不论犯了什么错误的同志,还是一九四二年到一九四五年整风运动时的那个路线,只要认真改正,都表示欢迎,就要团结他,治病救人,惩前毖后,团结——批评——团结。但是,是非要搞清楚,不能吞吞吐吐,敲一下吐一点,不能采取这样的态度。为什么和尚念经要敲木鱼?西游记里讲,取回来的经被鲤鱼精吃了,敲一下吐一个字,就是这么来的。不要采取这种态度和鲤鱼精一样,要好好想一想。犯了错误的同志,只要认识错误,回到马克思主义立场方面来,我们就与你团结。在座的几位同志,我欢迎,不要犯了错误见不得人。我们允许犯错误,你已经犯了嘛,也允许改正错误。不要不允许犯错误,不允许改正错误。有许多同志改得好,改好了就好了嘛!李维汉\mnote{7}同志的发言就是现身的说法,李维汉同志的错误改了,我们信任嘛,一看二帮,我们坚决这样做。还有许多同志。我也犯过错误,去年我就讲了,你们也要容许我犯错误,容许我改正错误,改了,你们也欢迎。去年我讲,对人是要分析的,人是不能不犯错误的,所谓圣人,说圣人没有缺点,是形而上学的观点,而不是马克思主义、辩证唯物主义的观点,任何事物都是可以分析的,我劝同志们,无论是里通外国也好,搞什么秘密反党小集团的也好,只要把那一套统统倒出来,真正实事求是讲出来,我们就欢迎,还给工作做,绝不采取不理他们的态度,更不能采取杀头的办法。杀戒不可开。许多反革命都没有杀,潘汉年是一个反革命嘛!胡风、饶漱石也是反革命嘛!我们都没有杀嘛!宣统皇帝是不是反革命?还有王耀武、康泽、杜聿明、杨虎等战犯,也有一大批没杀。多少人改正了错误,就赦免他嘛,我们也没有杀。右派改了的,摘了帽子嘛!近来平反之风不对,真正错了再平反,搞对了不能平反,真错了的平反,全错全平反,部分错了部分平反,没有错的不平反,不能一律都平反。

工作问题,还请同志们注意,阶级斗争不要影响我们的工作,一九五九年第一次庐山会议本来是搞工作的,后来出了彭德怀,说:“你操了我四十天娘,我操你二十天娘不行!”这一操,就被扰乱了,工作受影响,二十天还不够,我们把工作丢了。这次不可能,这次传达要注意,各地、各部门要把工作放在第一位,工作与阶级斗争要平行。阶级斗争不要放在很突出的地位,现在已经组成一个专案调查委员会,把问题搞清楚。不要因阶级斗争干扰我们的工作,等下次或再下次全会来搞清楚,把问题说清,要说服人。阶级斗争要搞,但要有专门的人搞这个工作,公安部门是搞阶级斗争的。它的主要任务是对付敌人的破坏。有人搞破坏工作,我们开杀戒,只是对那些破坏工厂,破坏桥梁,在广州边界搞破坏案、杀人放火的人。保卫工作要保卫我们的事业,保卫工厂、企业、公社、生产队、学校、政府、军队、党、群众团体,还有文化机关,包括报馆、刊物、新闻社,保卫上层建筑。

现在不是写小说盛行吗?利用写小说\mnote{8}搞反动活动,是一大发明。凡是要推翻一个政权,总要先造成舆论,总要先做意识形态方面的工作。革命的阶级是这样,反革命的阶级也是这样。我们的意识形态是搞点革命的马克思的学说、列宁的学说,马列主义普遍真理与中国革命具体实践相结合。结合得好,问题就解决得好些。结合的不好,就会失败受挫折。讲社会主义建设时,也是普遍真理与建设相结合,现在是结合好了还是没有结合好?我们仍然正在解决这个问题。军事建设也是如此。如前几年的军事路线与这几年的军事路线就不同。叶剑英同志搞了部著作,很尖锐,大关节是不糊涂的,我一向批评你说不尖锐,这次可尖锐了,我送你两句话:“诸葛一生惟谨慎,呂端大事不糊涂。”

\begin{maonote}
\mnitem{1}结构改革论,第二次世界大战后意大利共产党总书记陶里亚蒂提出的关于和平过渡社会主义的理论。一九五六年12月意共召开第八次代表大会,陶里亚蒂正式系统地提出“结构改革论”的理论和路线。他指出:结构改革是意大利共产党争取实现的一个积极目标,而这个目标在当前的政治斗争条件下是可以实现的。“结构改革论”的基本内容是:主张不诉诸使用暴力打碎旧的国家机器,通过议会斗争和群众斗争相结合的途径,争取群众大多数的支持,逐步改变国家内部社会力量的对比,使工人阶级及其同盟者进入国家领导机构,建立“新型民主制”,实现工人阶级的领导;主张在宪法规定的范围内使经济关键部门有计划、有步骤地实行国有化,使经济管理部门民主化、经济规划化,实行土地改革,实现耕者有其田和技术进步。通过国家的税收和财政改革,达到限制和打击大垄断资本和大庄园主,为向社会主义过渡准备条件。意共和陶里亚蒂认为,“结构改革”本身并不是社会主义,它只是为向社会主义前进开辟道路。在结构改革过程中,当反对派使用暴力时,无产阶级也要用暴力来对付。“要实现社会主义的完全结构改革,从而解决我国社会内部的根本矛盾,只有在工人阶级及其盟友夺得了政权后才能达到。”“结构改革”不仅是意共的基本路线,也为当时西欧一些国家的共产党所接受。“结构改革论”也是70年代兴起的“欧洲共产主义”的理论来源之一。春日藏次郎:曾任日共领导人。
\mnitem{2}第一次郑州会议,指指一九五八年十一月二日至十日毛泽东在郑州召集的有部分中央领导人和部分地方负责人参加的会议,也称第一次郑州会议。毛泽东在会上批评了急于想使人民公社由集体所有制过渡到全民所有制、由社会主义过渡到共产主义,以及企图废除商品生产等错误主张。
\mnitem{3}武昌会议指一九五八年十一月二十一日至二十七日,中共中央在武昌召开的政治局扩大会议,参加会议的有部分中央领导人和各省、市、自治区党委第一书记。
\mnitem{4}第二次郑州会议指一九五九年二月二十七日至三月五日在郑州召开的中共中央政治局扩大会议。
\mnitem{5}上海会议指一九五九年三月二十五日至四月一日在上海召开的中共中央政治局扩大会议。
\mnitem{6}尼指尼赫鲁,时任印度总理。
\mnitem{7}李维汉(一八九六——一九八四),湖南长沙人。时任中共中央统战部部长。
\mnitem{8}这里的小说是指《刘志丹》,作者为刘志丹的弟媳李建彤,小说主要反映刘志丹革命生涯,但其中内容牵涉到西北根据地历史上几个重要事件及对此的评价问题,如三嘉原事件、肃反运动、西北局高干会议,情况极为复杂,牵涉面极大,一九六二年中央认定,小说《刘志丹》涉嫌反党,被禁止出版,主要错误是歪曲西北革命历史,拔高刘志丹比毛主席更英明,把毛主席关于中国革命运动中的农民运动、武装斗争、统一战线、土地政策、工商业政策、根据地问题、农村包围城市的战略思想等重要理论,都搬到刘志丹身上,而且提出的时间,往往在毛主席之前,同时任意贬低同刘志丹有不同意见的其他西北老同志。

值得一提的是,走资派控制中央后,为了反毛和夺权的需要,一风吹的翻案,一九七九年八月,中央认为这部小说“是一部比较好的歌颂老一辈无产阶级革命家、描写革命斗争历史的小说”,并认为“是康生制造的一起大错案”,该案被描绘成证明当时中央“极左”路线的佐证。

小说《刘志丹》于一九八四年十二月重新组织出版,但这部小说一面世,仍然引起许多老干部的不满,纷纷给中央写信,要求按照党的纪律严肃处理,一九八六年中央再次认定,这部小说确实存在严重的错误,违背中央文件精神,违背党的原则,于是中央决定,该书立即停止发行,并对作者进行严肃的批评帮助和适当的处理。

二〇〇九年小说《刘志丹》再次出版公开发售,但很快就又被中央制止了,并作了严肃处理。

静火有言:小说《刘志丹》最终永远被禁,被禁的理由和一九六二年的理由几乎一致,到底是错案呢还是翻案呢?
\end{maonote}
