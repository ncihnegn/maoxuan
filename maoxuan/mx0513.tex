
\title{镇压反革命必须实行党的群众路线}
\date{一九五一年五月十五日}
\thanks{这是毛泽东同志修改第三次全国公安会议决议时写的几段指示。}
\maketitle


\section*{一}

目前在全国进行的镇压反革命的运动,是一场伟大的激烈的和复杂的斗争。全国各地已经实行的有效的工作路线,是党的群众路线。这就是:党委领导,全党动员,群众动员,吸收各民主党派及各界人士参加,统一计划,统一行动,严格地审查捕人和杀人的名单,注意各个时期的斗争策略,广泛地进行宣传教育工作(各种代表会、干部会、座谈会、群众会,在会上举行苦主控诉,展览罪证,利用电影、幻灯、戏曲、报纸、小册子和传单作宣传,做到家喻户晓,人人明白),打破关门主义和神秘主义,坚决地反对草率从事的偏向。凡是完全遵照这个路线去做的,就是完全正确的。凡是没有遵照这个路线去做的,就是错误的。凡是大体上遵照了这个路线,但没有完全遵照这个路线去做的,就是大体上正确但不完全正确的。我们认为这个工作路线是继续深入镇压反革命工作和取得完满胜利的保证。在今后镇反工作中必须完全遵守这个工作路线。其中最重要者为严格地审查逮捕和判处死刑名单,和广泛地做好宣传教育。做到了这两点,就可以避免犯错误。

\section*{二}

关于杀反革命的数字,必须控制在一定比例以内。这里的原则是:对于有血债或其它最严重的罪行非杀不足以平民愤者和最严重地损害国家利益者,必须坚决地判处死刑,并迅即执行。对于没有血债、民愤不大和虽然严重地损害国家利益但尚未达到最严重的程度,而又罪该处死者,应当采取判处死刑、缓期二年执行、强迫劳动、以观后效的政策。此外还应明确地规定:凡介在可捕可不捕之间的人一定不要捕,如果捕了就是犯错误;凡介在可杀可不杀之间的人一定不要杀,如果杀了就是犯错误。

\section*{三}

为了防止在镇压反革命运动的高潮中发生“左”的偏向,决定从六月一日起,全国一切地方,包括那些至今仍然杀人甚少的地方在内,将捕人批准权一律收回到地委专署一级,将杀人批准权一律收回到省一级,离省远者由省级派代表前往处理。任何地方不得要求改变此项决定。

\section*{四}

对于“中层”和“内层”的反革命分子,必须从现在开始,有计划地加以清查。决定遵照中央指示在今年夏秋两季,采用整风方式,对留用人员和新吸收的知识分子普遍地初步地清查一次。其目的是弄清情况和处理一些最突出的问题。其方法是学习镇反文件,向着留用人员和新吸收的知识分子,号召他们中间有问题的人(不是一切人)用真诚老实的态度,交清历史,坦白其隐藏的问题。这种坦白运动必须由首长负责主持,采取自愿原则,不得施行强迫。每一单位时间要短,不要拖长。其策略是争取多数,孤立少数,以待冬季的进一步清查。对于首脑机关、公安机关及其它要害部门,则须首先加以清查,并取得经验,以资推广。在政府系统、学校和工厂中进行此种清查工作时,必须有党外人士参加此种清查工作的委员会,避免由共产党员孤立地去做。

\section*{五}

全国各地,必须在此次镇压反革命的伟大斗争中普遍地组织群众的治安保卫委员会。此项委员会,农村以乡为单位,城市以机关、学校、工厂、街道为单位,经过人民选举组织之。委员人数少者三人,多者十一人,必须吸收可靠的党外爱国分子参加,成为统一战线的保卫治安的组织。此项委员会受基层政府和公安机关的领导,担负协助人民政府肃清反革命,防奸、防谍,保卫国家和公众治安的责任。此项委员会,乡村须在土改完成之后,城市须在镇反工作开展之后,有领导地进行组织,以免坏人乘机侵入。
