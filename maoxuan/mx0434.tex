
\title{关于情况的通报}
\date{一九四八年三月二十日}
\thanks{这是毛泽东为中共中央写的对党内的通报。在这以后,中共中央就离开陕甘宁边区,经晋绥解放区进入晋察冀解放区,在一九四八年五月到达河北省西部平山县的西柏坡村。}
\maketitle


一、最近几个月,中央集中全力解决在新形势下面关于土地改革方面、关于工商业方面、关于统一战线方面、关于整党方面、关于新区工作方面的各项具体的政策和策略的问题,反对党内右的和“左”的偏向,而主要是“左”的偏向。我们党的历史情况表明,在我党和国民党结成统一战线时期,党内容易发生右的偏向,而在我党和国民党分裂时期,党内容易发生“左”的偏向。现在的“左”的偏向,主要的是侵犯中农,侵犯民族资产阶级,职工运动中片面强调工人眼前福利,对待地主和对待富农没有区别,对待地主的大中小、恶霸非恶霸没有区别,不按平分原则给地主留下必要的生活出路,在镇压反革命斗争中越出了某些政策界限,以及不要代表民族资产阶级的党派,不要开明绅士,在新解放区忽视缩小打击面(即忽视中立富农和小地主)在策略上的重要性,工作步骤上的急性病等。这些“左”的偏向,在过去大约两年的时间内,各解放区都或多或少地发生过,有时成了严重的冒险主义倾向。好在纠正这类偏向并不甚困难,几个月内已经大体上纠正过来了,或者正在纠正着。但须各级领导者着重用力才能彻底纠正此类偏向。右的偏向主要是过高地估计敌人的力量,惧怕美国大量援蒋,对长期战争有些厌倦,对国际民主力量的强大的程度有些怀疑,不敢放手发动群众消灭封建制度,对党内成分不纯和作风不纯熟视无睹等。但这类偏向现在不是主要的,改正亦不困难。最近几个月,我党在战争、土地改革、整党整军、发展新区和争取民主党派等方面均有成绩,在这些工作中所发生的偏向有了着重的纠正,或正在纠正中,这样就可以使整个中国革命运动走上健全发展的轨道。只有党的政策和策略全部走上正轨,中国革命才有胜利的可能。政策和策略是党的生命,各级领导同志务必充分注意,万万不可粗心大意。

二、由于对美国和蒋介石存着某种幻想,对我党和人民具有足以战胜一切内外敌人的力量表示怀疑,并因此认为所谓第三条道路\mnote{1}尚有存在可能、将自己处于国共两党之间的中间地位的某些民主人士,在国民党的突然的攻势之下,使自己处于被动地位,最后终于在一九四八年一月间采用我党的口号,声明反蒋反美,联共联苏\mnote{2}。对于这些人,我们应当对他们采取团结的政策,对他们的某些错误观点则作适当的批评。在将来成立中央人民政府时,邀请他们一部分人参加政府工作是必要的和有益的。这些人的特点是从来不愿意接近劳动群众,又习惯于大城市的生活,不愿轻易到解放区来。虽然如此,他们所代表的社会基础,即民族资产阶级,却有其重要性,不可忽视。因此,应当争取他们。估计要待我们有更大的胜利,夺取几个例如沈阳、北平、天津那样的城市,共产党胜、国民党败的形势业已完全判明以后,邀请他们参加中央人民政府,他们可能愿意来解放区和我们共事。

三、本年内,我们不准备成立中央人民政府,因为时机还未成熟。在本年蒋介石的伪国大开会选举蒋介石当了总统\mnote{3},他的威信更加破产之后,在我们取得更大胜利,扩大更多地方,并且最好在取得一二个头等大城市之后,在东北、华北、山东、苏北、河南、湖北、安徽等区连成一片之后,便有完全的必要成立中央人民政府。其时机大约在一九四九年。目前我们正将晋察冀区、晋冀鲁豫区和山东的渤海区统一在一个党委(华北局)、一个政府、一个军事机构的指挥之下(渤海区也许迟一点合并),这三区包括陇海路以北、津浦路和渤海以西、同蒲路以东、平绥路以南的广大地区\mnote{4}。这三区业已连成一片,共有人口五千万,大约短期内即可完成合并任务。这样做,可以有力地支持南线作战,可以抽出许多干部输往新解放区。该区的领导中心设在石家庄。中央亦准备移至华北,同中央工作委员会\mnote{5}合并。

四、我南线各军,即山东兵团九个旅,苏北兵团七个旅,河淮间兵团二十一个旅,豫鄂陕兵团十个旅,江淮汉水间兵团十九个旅,西北兵团十二个旅,晋南豫北兵团十二个旅,除江淮汉水间刘邓兵团的主力因白崇禧集中兵力向大别山进攻\mnote{6},未获休整,到二月底才抽出一部到淮河以北休整外,其余各兵团均在十二月至二月间作了休整。这是过去二十个月作战中的第一次大休整。这次休整,采取群众诉苦(诉旧社会和反动派所给予劳动人民之苦)、三查(查阶级成分,查工作,查斗志)和群众性练兵(官教兵,兵教官,兵教兵)的方法,发动了全军指挥员战斗员的高度的革命积极性,教好了或清除了一部分军队中的地主、富农分子或坏分子,提高了纪律,讲明了土地改革中的各项政策、对待工商业和知识分子的政策,发扬了军队中的民主作风,提高了军事技术和战术。这样就使得我军极大地增长了战斗力。现在除刘邓兵团的一部尚在休整外,各兵团均已于二月底三月初先后开始新的作战行动,并在两星期内歼敌九个旅。北线各军,即东北兵团四十六个旅、晋察冀兵团十八个旅、晋绥兵团两个旅,在冬季则大部作战,一部休整。东北兵团,利用辽河结冰,举行了三个月作战,歼敌八个旅,争取敌一个旅起义,攻占彰武、法库、新立屯、辽阳、鞍山、营口和四平街,并收复吉林。该兵团现已开始休整。俟休整完毕,或打长春,或打北宁路上之敌。晋察冀兵团休整一个多月,现已向平绥线行动。晋绥兵团数量较小,其主要任务是对阎锡山起钳制作用。总计我军现有南北两线大小十个兵团,正规兵力已达五十个纵队(等于国民党的整编师),一百五十六个旅(等于国民党的整编旅),一百三十二万二千余人,平均每旅(三个团)人数八千左右。此外,尚有非正规军,包括地方兵团、部队、游击队、后方军事机关、军事学校等在内,一百一十六万八千余人(其中作战部队占八十万人),全军总计为二百四十九万一千余人。而在一九四六年七月以前,我们只有正规军二十八个纵队,一百一十八个旅,六十一万二千余人,平均每旅(三个团)人数不足五千;加上非正规军六十六万五千余人,总计一百二十七万八千余人。可以看出,我们的军队现在是壮大了。旅的数目增加不多,每旅的人数却大为增加。经过二十个月作战,战斗力亦大为增加。

五、国民党的正规军,从一九四六年七月至去年夏季,是九十三个师,二百四十八个旅,现在则有一百零四个师,二百七十九个旅的番号。其分布是:北线二十九个师,九十三个旅(沈阳卫立煌十三个师,四十五个旅;北平傅作义十一个师,三十三个旅;太原阎锡山五个师,十五个旅),约五十五万人。南线六十六个师,一百五十八个旅(郑州顾祝同三十八个师,八十六个旅;九江白崇禧十四个师,三十三个旅;西安胡宗南十四个师,三十九个旅),约一百零六万人。第二线九个师,二十八个旅(西北区,包括兰州以西地区,四个师,八个旅;西南区,包括川、康、滇、黔,四个师,十个旅;东南区,包括长江以南诸省,八个旅;台湾,一个师,两个旅),约十九万六千人。国民党正规军番号增加的原因,是因为国民党军大量被我军歼灭,并由战略攻势转入战略守势之后,甚感兵力不足,因此将大量地方部队和伪军升级或编组为正规军,计北线卫立煌系统增加三个师,十四个旅;傅作义系统增加两个师,六个旅;南线顾祝同系统,增加六个师,九个旅;胡宗南系统,增加两个旅;共计增加十一个师,三十一个旅。因此,国民党军现在不是九十三个师,而是一百零四个师,不是二百四十八个旅,而是二百七十九个旅。但是第一,最近几个月(至三月二十日为止)被我歼灭的六个师,二十九个旅,只有空番号,尚未来得及重建或补充,也许有一部分永远无法重建或补充了,因此,国民党军在实际上现在只有九十八个师,二百五十个旅,比之去年夏季以前只多了五个师的番号和二个实际的旅。第二,现在实有的二百五十个旅中,只有一百十八个旅未受过我军歼灭性的打击,其余一百三十二个旅,或者被我军歼灭过一次、二次,甚至三次,然后补充起来的;或者是受过我军一次、二次,甚至三次歼灭性打击的(以旅为单位,全体被消灭,或大部被消灭者,称为被歼灭;一个团以上被消灭,但其主力未受损失者,称为受歼灭性打击),其士气和战斗力甚为低落。在未受歼灭性打击的一百十八个旅中,有一部分是在第二线训练的新兵,有一部分是从地方部队和伪军升级或编组的,战斗力很弱。第三,国民党军队的数量也减少了。一九四六年七月以前,其正规部队二百万人,非正规部队七十三万八千人,特种部队三十六万七千人,海空部队十九万人,后勤机关、学校一百零一万人,总计四百三十万五千人。而在一九四八年二月,他的正规部队一百八十一万人,非正规部队五十六万人,特种部队二十八万人,海空部队十九万人,后勤机关、学校八十一万人,总计三百六十五万人,即是说,减少了六十五万五千人。一九四六年七月至一九四八年一月的十九个月中,我军共消灭国民党军队一百九十七万七千人(二月和三月上半月尚未统计好,大约有十八万人左右),即是说,国民党不但将其在过去作战期间所动员参军的一百余万新兵消耗了,而且大量消耗了它原有的兵力。在此种形势下,国民党采取和我们相反的方针,不是充实各旅人员的数目,而是减少旅的人员,增加旅的番号。国民党军在一九四六年平均每旅差不多有八千人,而在现在则平均每旅只有六千五百人左右。今后我军占地日广,国民党军兵源粮源日益缩小,估计再打一个整年,即至明年春季的时候,敌我两军在数量上可能达到大体上平衡的程度。我们的方针是稳扎稳打,不求速效,只求平均每个月消灭国民党正规军八个旅左右,每年消灭敌军约一百个旅左右。事实上,从去年秋季以后,超过了这个数目;今后可能有更大的超过。五年左右(一九四六年七月算起)消灭国民党全军的可能性是存在的\mnote{7}。

六、目前南北两线敌军在两个地区尚有较大的机动兵力,可以举行战役性的进攻,使那里的我军暂时处于困难地位。其一,即大别山,有约十四个机动旅。其二,淮河以北地区,有约十二个机动旅。在这两区,国民党军还有主动权(淮河以北地区,由于我抽出九个主力旅开至黄河以北休整,准备使用于其它方面,故国民党军有了主动权)。其余一切战场的敌军,全是被动挨打。具有对我特别有利形势的战场是东北、山东、西北、苏北、晋察冀、晋冀鲁豫和郑汉路\mnote{8}以西、长江以北、黄河以南的广大地区。


\begin{maonote}
\mnitem{1}见本卷\mxnote{目前形势和我们的任务}{12}。
\mnitem{2}一九四七年十月,国民党反动政府宣布中国民主同盟为“非法团体”,强制民盟总部宣布解散。当时,其它民主党派也都遭到国民党反动派的迫害,不能在国民党统治区公开活动。一九四八年一月一日,李济深等在香港成立中国国民党革命委员会;一月五日,沈钧儒等在香港召开中国民主同盟一届三中全会,宣布恢复民盟总部,继续进行政治斗争。这两个党派当时都接受了中国共产党关于时局的主张,发表宣言,主张联合中共和其它民主党派,推翻蒋介石独裁政权,反对美国武装干涉中国内政。
\mnitem{3}一九四八年三月二十九日至五月一日,国民党反动派在南京召集“国民大会”,选举蒋介石为“总统”,李宗仁为“副总统”。
\mnitem{4}一九四八年五月,晋察冀解放区和晋冀鲁豫解放区实行合并,改称华北解放区,同时成立了中共华北中央局、华北军区。晋察冀边区行政委员会和晋冀鲁豫边区政府也联合办公。同年八月,晋察冀和晋冀鲁豫两边区政府正式合并,成立华北人民政府。山东的渤海区仍属中共华东中央局管辖。
\mnitem{5}见本卷\mxnote{中共中央关于暂时放弃延安和保卫陕甘宁边区的两个文件}{4}。
\mnitem{6}白崇禧开始进攻大别山地区的时间是一九四七年十二月,进攻的兵力共三十三个旅。
\mnitem{7}五年左右消灭国民党全军,这是当时的预计。后来,这个时间缩短为三年半左右。参见本卷\mxart{中国军事形势的重大变化}。
\mnitem{8}指平汉路郑州至武汉段。
\end{maonote}
