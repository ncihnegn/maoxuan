
\title{关于建立报告制度}
\date{一九四八年一月七日}
\thanks{这是毛泽东为中共中央起草的对党内的指示。这个指示中所规定的报告制度,是中共中央坚持民主集中制、反对无纪律无政府倾向的长期斗争在新条件下的一个发展。这个问题在这时之所以特别重要,是因为革命形势已经有了极大的进展,许多解放区已经连成一片,许多城市已经解放或者即将解放,人民解放军和人民解放战争的正规性程度大为提高,全国胜利已经在望。这种情况,要求党迅速克服存在于党内和军队内的任何无纪律无政府状态,把一切必须和可能集中的权力集中于中央。建立严格的报告制度,就是中共中央为此目的而采取的一个重要步骤。关于这个问题,参看本卷\mxart{一九四八年的土地改革工作和整党工作}的第六部分和\mxart{中共中央关于九月会议的通知}的第四点。}
\maketitle


为了及时反映情况,使中央有可能在事先或事后帮助各地不犯或少犯错误,争取革命战争更加伟大的胜利起见,从今年起,规定如下的报告制度。

(一)各中央局和分局,由书记负责(自己动手,不要秘书代劳),每两个月,向中央和中央主席作一次综合报告。报告内容包括该区军事、政治、土地改革、整党、经济、宣传和文化等各项活动的动态,活动中发生的问题和倾向,对于这些问题和倾向的解决方法。报告文字每次一千字左右为限,除特殊情况外,至多不要超过两千字。一次不能写完全部问题时,分两次写。或一次着重写几个问题,对其余问题则不着重写,只略带几笔;另一次,则着重写其余问题,而对上次着重写过的只略带几笔。综合报告内容要扼要,文字要简练,要指出问题或争论之所在。写发综合报告的日期是单月的上旬,报告用电报发来。这是各中央局、分局书记个人负责向中央和中央主席作的经常性的报告和请示。书记在前线指挥作战时,除自己报告外,指定代理书记或副书记作后方活动的报告。此外,各中央局和分局向中央所作的临时性的报告和请示,照过去一样,不在此内。

我们所以规定这项政策性的经常的综合的报告和请示的制度,是因为党的第七次全国代表大会以后,仍然有一些(不是一切)中央局和分局的同志,不认识事先或事后向中央作报告并请求指示的必要和重要性,或仅仅作了一些技术性的报告和请示,以致中央不明了或者不充分明了他们重要的(不是次要的或技术性的)活动和政策的内容,因而发生了某些不可挽救的、或难以挽救的、或能够挽救但已受了损失的事情。而那些事前请示、事后报告的中央局或分局,则避免了或减少了这样的损失。从今年起,全党各级领导机关,必须改正对上级事前不请示、事后不报告的不良习惯。各中央局和分局是受中央委任、代表中央执行其所委托的任务的机关,必须同中央发生最密切的联系。各省委或区党委,同各中央局和分局也须密切联系。当此革命已进入新的高潮时期,加强此种联系,极为必要。

(二)各野战军首长和军区首长,除作战方针必须随时报告和请示,并且照过去规定,每月作一次战绩报告、损耗报告和实力报告外,从今年起,每两个月要作一次政策性的综合报告和请示。此项报告和请示的内容是:关于该军纪律,物质生活,指战员情绪,指战员中发生的偏向,克服偏向的方法,技术、战术进步或退步的情况,敌军的长处、短处和士气高低,我军政治工作的情况,我军对土地政策、城市政策、俘虏政策的执行情况和克服偏向的方法,军民关系和各阶层人民的动向等。此项报告的字数、写作方法以及发报时间,和各中央局、分局报告的办法相同。如规定的写报告时间(逢单月的上旬)恰在作战紧张的时候,则可提前或推迟若干天,但须申明原因。其中关于政治工作部分,由该军政治部主任起草,经司令员、政治委员审查修改,并且共同署名。报告用电报发给军委主席。我们规定此项政策性综合报告的理由,和上述中央局、分局应作综合报告的理由相同。
