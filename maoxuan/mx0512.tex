
\title{中共中央政治局扩大会议决议要点}
\date{一九五一年二月十八日}
\thanks{这是毛泽东同志为中共中央起草的党内通报。}
\maketitle


中央于二月中旬召开有各中央局负责同志参加的政治局会议,讨论了各项重要问题,兹将决议要点通报如下。

\section{一 二十二个月的准备工作}

“三年准备、十年计划经济建设”的思想,要使省市级以上干部都明白。准备时间,现在起,还有二十二个月,必须从各方面加紧进行工作。

\section{二 抗美援朝的宣传教育运动}

必须在全国范围内继续推行这个运动,已推行者深入之,未推行者普及之,务使全国每处每人都受到这种教育。

\section{三 土改}

1、农忙时一律停一下,总结经验。

2、争取今年丰收。

3、依靠县农民代表会及训练班。

4、积极造成条件。凡条件不成熟者,无论何时何地不要勉强去做。

5、土改完成,立即转入生产、教育两大工作。

6、同意华东分期退押的办法。

7、劝告农民以不采非刑拷打为有利。

8、土改后,增划区乡,缩小区乡行政范围。

\section{四 镇压反革命}

1、判处死刑一般须经过群众,并使民主人士与闻。

2、严密控制,不要乱,不要错。

3、注意“中层”\mnote{1},谨慎地清理旧人员及新知识分子中暗藏的反革命分子。

4、注意“内层”,谨慎地清理侵入党内的反革命分子,十分加强保密工作。

5、还要向干部做教育,并给干部撑腰。

\section{五 城市工作}

1、各中央局、分局、省市区党委今年必须召开两次城市工作会议,议程如中央所通知,向中央作两次专题报告。

2、加强党委对城市工作的领导,实行七届二中全会决议。

3、向干部做教育,明确依靠工人阶级的思想。

4、工厂内,以实现生产计划为中心,实行党、政、工、团的统一领导。

5、力争在增加生产的基础上逐步改善工人生活。

6、在城市建设计划中,应贯彻为生产、为工人服务的观点。

7、全国总工会及各上级工会应着重解决下面的具体问题。

8、党委及工会应着重典型经验的创造,迅速推及各处。

\section{六 整党及建党}

1、我们的党是伟大的,光荣的,正确的,这是主要方面,必须加以肯定,并向各级干部讲明白。但是存在着问题,必须加以整理,并对新区建党采取慎重的态度,这方面也要讲明白。

2、整党、建党,均须由中央及各中央局实行严格的控制,下面不得自由行动。

3、整党,应以三年时间实现之。其步骤,应是以一年时间(一九五一年)普遍进行关于怎样做一个共产党员的教育,使所有党员明白做一个共产党员的标准,并训练组织工作人员。同时,进行典型试验。然后,根据经验进行整党,但城市可以在一九五一年进行整党。整党时,首先将“第四部分人”\mnote{2}清洗出去。然后对“第二部分人”、“第三部分人”加以区别,对其中经过教育而仍确实不合党员条件者劝其退党,务使这些退党者自愿地退出,不要伤感情,不要重复一九四八年“搬石头”\mnote{3}的经验。

4、城市及新区建党必须采取慎重的方针。城市着重在产业工人中建立党的组织。乡村须在土改完毕始能吸收经过教育合于党员条件者建立党的支部,在头两年内乡村支部一般不要超过十个党员。无论在城市和乡村,均应对于愿意接受党的教育的积极分子,进行关于怎样做一个共产党员的教育,经过这种教育然后将其中确实合于党员条件者吸收入党。

\section{七 统一战线工作}

1、要求各中央局、分局、省市区党委一九五一年召集两次会议,讨论统一战线工作,并向中央作两次关于这方面的专题报告。

2、必须向干部讲清楚为什么要加强统一战线工作的理由。

3、知识分子,工商业界,宗教界,民主党派,民主人士,必须在反帝反封建的基础上将他们团结起来,并加以教育。

4、认真在各少数民族中进行工作,推行区域自治和训练少数民族自己的干部是两项中心工作。

\section{八 整风}

一年一次,冬季进行,时间要短,任务是检查工作,总结工作经验,发扬成绩,纠正缺点错误,借以教育干部。


\begin{maonote}
\mnitem{1}镇压反革命的工作,分为外层、中层、内层。清理“外层”,是指清查隐藏在社会上的反革命分子;清理“中层”,是指清查隐藏在我军政机关内部的反革命分子;清理“内层”,是指清查隐藏在我党内的反革命分子。
\mnitem{2}一九五一年整党时,把党员划分为四种人:一、具备党员条件的;二、不完全具备党员条件,或者有较严重的毛病,必须加以改造提高的;三、不够党员条件的消极落后分子;四、混入党内的阶级异己分子、叛变分子、投机分子、蜕化变质分子等。
\mnitem{3}“搬石头”是刘少奇在一九四八年解放区土改、整党时提出的。他污蔑广大农村干部是压在农民头上的“石头”,要把他们撤职、清洗。
\end{maonote}
