
\title{十年总结}
\date{一九六〇年六月十八日}
\thanks{这是毛泽东同志上海会议\mnote{1}期间写的一篇文稿,题目是毛泽东拟的。}
\maketitle


前八年照抄外国的经验。但从一九五六年提出十大关系起,开始找到自己的一条适合中国的路线。

一九五七年反右整风斗争,是在社会主义革命过程中反映了客观规律,而前者则是开始反映了中国客观经济规律。

一九五八年五月党大会制定了一个较为完整的总路线\mnote{2},并且提出了打破迷信、敢想敢说敢做的思想。这就开始了一九五八年的大跃进。是年八月发现人民公社是可行的。赫然挂在河南新乡县七里营的墙上的是这样几个字:“七里营人民公社”。

我到襄城县、长葛县看了大规模的生产合作社。河南省委史向生同志,中央《红旗》编辑部李友九同志,同遂平县委、嵖岈山乡党委会同在一起,起草了一个嵖岈山卫星人民公社章程\mnote{3}。这个章程是基本正确的。

八月在北戴河,中央起草了一个人民公社决议\mnote{4},九月发表。几个月内公社的架子就搭起来了,但是乱子出得不少,与秋冬大办钢铁同时并举,乱子就更多了。于是乎有十一月的郑州会议\mnote{5},提出了一系列的问题,主要谈到价值法则、等价交换、自给生产、交换生产。又规定了劳逸结合,睡眠、休息、工作,一定要实行生产、生活两样抓。

十二月武昌会议\mnote{6},作出了人民公社的长篇决议,基本正确,但只解决了集体、国营两种所有制的界线问题,社会主义与共产主义的界线问题,一共解决两个外部的界线问题,还不认识公社内部的三级所有制问题。一九五八年八月北戴河会议提出了三千万吨钢在一九五九年一年完成的问题,一九五八年十二月武昌会议降至二千万吨。

一九五九年一月北京会议\mnote{7}是为了想再减一批而召开的,我和陈云\mnote{8}同志对此都感到不安,但会议仍有很大的压力,不肯改。我也提不出一个恰当的指标来。

一九五九年四月上海会议\mnote{9}规定一个一六五〇万吨的指标,仍然不合实际。我在会上作了批评。这个批评之所以作,是在会议开会之前两日,还没有一个成文的盘子交出来,不但各省不晓得,连我也不晓得,不和我商量,独断专行,我生气了,提出了批评。我说:我要挂帅。这是大家都记得的。

下月(五月)北京中央会议\mnote{10}规定指标为一三〇〇万吨,这才完全反映了客观实际的可能性。五、六、七月出现了一个小小马鞍形。七八两月在庐山基本上取得了主动。但在农业方面仍然被动,直至于今。

管农业的同志,和管工业的同志,管商业的同志,在这一段时间内,思想方法有一些不对头,忘记了实事求是的原则,有一些片面思想(形而上学思想)。

一九五九年夏季庐山会议\mnote{11},右倾机会主义猖狂进攻。他们教育了我们,使我们基本上清醒了。找们举行反击,获得胜利。一九六〇年六月上海会议\mnote{12}规定后三年\mnote{13}的指标,仍然存在一个极大的危险,就是对于留余地,对于藏一手,对于实际可能性还要打一个人大的折扣,当事人还不懂得。一九五六年周恩来同志主持制定的第二个五年计划\mnote{14},大部分指标,如钢等,替我们留了三年余地,多么好啊!农业方面则犯了错误,指标高了,以至不可能完成。要下决心改。在今年七月的党大会\mnote{15}上一定要改过来。从此就完全主动了。

同志们,主动权是一个极端重要的事情。主动权,就是“高屋建瓴”、“势如破竹”。这件事来自实事求是,来自客观情况对于人们头脑的真实的反映,即人们对于客观外界的辩证法的认识过程。我们过去十年的社会主义革命和社会主义建设,就是这样一个过程。中间经过许多错误的认识,逐步改正这些错误,以归于正确。

现在就全党同志来说,他们的思想并不都是正确的,有许多人并不懂得马列主义的立场、观点和方法。我们有责任帮助他们,特别是县、社、队的同志们。

我本人也有过许多错误。有些是和当事人一同犯了的。例如,我在北戴河同意一九五九年完成三千万吨钢。十二月又在武昌同意了可以完成二千万吨。又在上海会议同意了一六五〇万吨。例如,一九五九年三月在第二次郑州会议上\mnote{16},主张对一平二调问题的账可以不算;到四月,因受浙江同志和湖北同志的启发,才坚决主张一定要算账\mnote{17}。如此等类。看来,错误不可能不犯。

如列宁所说,不犯错候的人从来没有。郑重的党在于重视错误,找出错误的原因,分析所以犯错误的主观和客观的原因,公开改正。我党的总路线是正确的,实际工作也是基本上做得好的。有一部分错误大概也是难于避免的。哪里有完全不犯错误、一次就完成了真理的所谓圣人呢?真理不是一次完成的,而是逐步完成的。我们是辩证唯物论的认识论者,不是形而上学的认识论者。自由是必然的认识和世界的改造。由必然王国到自由王国的飞跃,是在一个长期认识过程中逐步地完成的。对于我国的社会主义革命和建设,我们已经有了十年的经验了,已经懂得了不少的东西了。但是我们对于社会主义时期的革命和建设,还有一个很大的盲目性,还有一个很大的未被认识的必然王国。我们还不深刻地认识它。我们要以第二个十年时间去调查它,去研究它,从其中找出它的固有的规律,以便利用这些规律为社会主义的革命和建设服务。对中国如此,对整个世界也应当如此。

我试图做出一个十年经验的总结。上述这些话,只是一个轮廓,而且是粗浅的,许多问题没有写进去,因为是两个钟头内写出的,以便在今天下午讲一下。

\begin{maonote}
\mnitem{1}上海会议,在一九六〇年六月十四日至十八日于上海召开中共中央政治局扩大会议。
\mnitem{2}指一九五八年五月五日至二十三日在北京举行的党的八大二次会议上通过的“鼓足干劲、力争上游、多快好省地建设社会主义”的总路线。
\mnitem{3}指河南省遂平县《嵖岈山卫星人民公社试行简章(草案)》。
\mnitem{4}指一九五八年八月十七日至三十日在北戴河举行的中共中央政治局扩大会议作出的《关于农村建立人民公社的决议》。
\mnitem{5}指毛泽东一九五八年十一月在郑州主持召开的有部分中央领导人、大区负责人和省市委书记参加的工作会议,又称第一次郑州会议。
\mnitem{6}指一九五八年十一月二十八日至十二月十日在武昌举行的中共八届六中全会通过的《关于人民公社若干问题的决议》。
\mnitem{7}指一九五九年一月下旬在北京召开的各省、市、自治区党委第一书记会议。
\mnitem{8}陈云,时任中共中央副主席、国务院副总理。
\mnitem{9}指一九五九年四月二日至五日在上海举行的中共八届七中全会。
\mnitem{10}指一九五九年五月在北京举行的中共中央政治局会议。
\mnitem{11}指一九五九年七月二日至八月一日在庐山举行的中共中央政治局扩大会议和八月二日至十六日在庐山举行的中共八届八中全会。
\mnitem{12}指当时正在上海召开的中共中央政治局扩大会议。
\mnitem{13}指第二个五年计划的后三年。
\mnitem{14}指周恩来一九五六年九月在中共第八次全国代表大会上作的《关于发展国民经济第二个五年计划的建议的报告》。
\mnitem{15}指原准备一九六O年七月召开的中共八大三次会议,后因故未召开。
\mnitem{16}指一九五九年二月二十七日至三月五日在郑州举行的中共中央政治局扩大会议。
\mnitem{17}指王任重一九五九年三月三十一日送毛泽东审阅的四份材料:王延春等三月二十七日关于麻城县万人大会情况和关于棉花技术措施问题给王任重并湖北省委的两个报告、王延春同日晚关于穷队赶富队问题给王任重的报告和湖北省委书记处书记许道琦三月二十三日关于与基层干部座谈“吃饭不要钱”问题给王任重、王延春并省委的报告。其中关于麻城县万人大会情况的报告说,麻城县的万人大会采取县委的会议和公社的会议一起开的办法,把会内会外联系起来,把开会和领导结合起来,把干部开会和发动群众结合起来,内外呼应,连成一气。会议开始,县、公社和管理区三级党委,层层检讨,承认错误,基层干部大受感动,也纷纷检讨,上下级关系更加密切了。检讨之后,最重要的是强调兑现,认真解决“一平二调三收款”的问题。兑现的内容主要有三:一是公社调生产队的钱和物资,立即退回;二是缺口粮的,立即供应;三是该支援的穷队,立即予以贷款。

毛泽东同志在这些材料二上写的批语是:此件极好,每一个县、社都应这样做。算帐才能团结;算帐才能帮助干部从贪污浪费的海洋中拔出身来,一身清净;算帐才能教会干部学会经营管理方法;算帐才能教会五亿农民自己管理自己的公社,监督公社的各级干部只许办好事,不许办坏事,实现群众的监督,实现真正的民主集中制。
\end{maonote}
