
\title{哪里有压迫哪里就有反抗}
\date{一九六四年七月九日}
\thanks{这是毛泽东同志同在朝鲜平壤参加第二次亚洲经济讨论会后访华的亚洲、非洲、大洋洲一些国家和地区的代表谈话的主要部分。}
\maketitle


亚非拉人民斗争的前途,这是大家关心的问题。如果要看前途,一定要看历史。从亚洲、非洲、拉丁美洲在第二次世界大战以后十几年的历史来看,就知道亚非拉人民将来的前途。比如中国吧,在十九年以前,日本军国主义霸占了我们大半个国家,我们同它打了八年仗。抗战胜利后美国人来了,他们支持蒋介石发动内战。我们在解放前要对付的敌人,有日本军国主义和美帝国主义,还有它们的走狗汪精卫\mnote{1}、“满洲国”的康德皇帝\mnote{2}、蒋介石。我们解放后,有一位日本资本家叫南乡三郎\mnote{3},和我谈过一次话,他说:“很对不起你们,日本侵略了你们。”我说:“不,如果没有日本帝国主义发动大规模侵略,霸占了大半个中国,全中国人民就不可能团结起来反对帝国主义,中国共产党也就不可能胜利。”事实上,日本帝国主义当了我们的好教员。第一,它削弱了蒋介石;第二,我们发展了共产党领导的根据地和军队。在抗战前,我们的军队曾达到过三十万,由于我们自己犯了错误,减少到两万多。在八年抗战中间,我们军队发展到了一百二十万人。你看,日本不是帮了我们的大忙?这个忙不是日本共产党帮的,是日本军国主义帮的。因为日本共产党没有侵略我们,而是日本垄断资本和它的军国主义政府侵略我们。我们的第二个教员,帮了我们忙的是美帝国主义。第三个帮了我们忙的教员是蒋介石。当时蒋介石有四百多万军队向我们进攻,我们的军队同他打了四年仗,从过去一百二十万发展到五百多万。蒋介石的百分之九十五以上的军队统统被我们消灭,只剩下百分之五不到的军队跑到台湾去了。中国得到的教训是这样:有压迫,就有反抗;有剥削,就有反抗。帝国主义,不管是日本帝国主义、美帝国主义或其他帝国主义,都是可以打倒的。国内反动派,如蒋介石,不管多么强大,也都是可以打倒的。这就是中国的历史情况。

现在日本人民有很大的觉悟,发动了很大的反对美帝国主义、反对本国垄断资本的运动。是谁使他们起来的呢?是美帝国主义和日本垄断资本的压迫和剥削把他们教育出来的,而不是中国共产党教他们这样做的。我说,日本垄断资本也不是完全赞成美帝国主义占领日本的,有一部分日本垄断资本家不满意这种占领,因为在美帝占领下,日本不仅丧失了殖民地,而且自己也受美国控制。现在不仅日本人民,而且一部分日本垄断资本家也开始反对美帝国主义。

讲到亚洲、非洲和拉丁美洲的历史,近十几年也是发生了很大的变化。单是非洲,就有三十几个国家独立了。一九五八年前,我很少见到非洲人。从一九五八年到一九六四年,每年经常看到非洲朋友。非洲有一个很大的反对帝国主义和殖民主义的风暴。比如埃及,一九五六年发生了苏伊士运河事件\mnote{4},是英、法联军强大些,还是埃及军队强大些?英国、法国那样强大,为什么一打就跑了呢?现在苏伊士运河究竟在埃及人手里,还是在英、法帝国主义手里呢?再看阿尔及利亚,民族解放军打了八年仗,法国在战争后期出动了八十万军队,民族解放军只有三四万军队,究竟是阿尔及利亚人民强大些,还是法国帝国主义强大些?再讲讲古巴,是美帝国主义和它的走狗巴蒂斯塔\mnote{5}强大些,还是卡斯特罗\mnote{6}强大些?卡斯特罗军队八十多人从国外乘船回古巴登岸,激战后剩下的十二人,躲在农民家里,后来又起来搞游击战争,他们经过两年多,就取得了胜利。至于对越南反对法国的侵略,对阿尔及利亚反对法国的侵略,我们都是公开支持的,这样岂不要得罪法国政府吗?不,胡志明\mnote{7}胜利了,本·贝拉\mnote{8}胜利了,法国却承认了中国。所以说,世界上的事情在发生变化。现在法国人在教训美国人,叫美国接受法国的教训,不要在越南南方打仗了:“我们法国人失败了,你们美国人要打,也要像我们一样失败的。”美国大概会接受法国的教训,已经打了三年打不赢,再打下去也不行,它不走是不行的。你们看吧!三年也好,再长一些时间也好,美国总是要从越南走的。美国从泰国、老挝、菲律宾、南朝鲜、日本等地也都要走的,还有从台湾也是要走的。它走的时间算不准,但一定要走的。所以凡是压迫亚洲、非洲和拉丁美洲的帝国主义、殖民主义总有一天要走的,只要人民团结起来,加强斗争。它走,也可以文明一点走。请它走它不走,怎么办?那就学卡斯特罗的办法,学本·贝拉的办法,学胡志明的办法,也可以学中国的办法。所以,我们看历史,就会看到前途。

讲到人民,不是讲反动派,是没有一个国家的人民压迫、剥削另一个国家的人民的。比如你们在座的亚非各国人民的代表,你们压迫中国人民了吗?你们剥削中国人民了吗?我们没有感觉到。中国人民能压迫你们吗?能剥削你们吗?如果中国政府这么干,那末中国政府就是帝国主义,而不是社会主义了。如果有的中国人不尊重你们,不讲平等,在你们国家捣鬼,那末你们可以把这样的中国人赶走。这就是亚非拉人民团结反对帝国主义的最根本原则。我们之间的相互关系是兄弟关系,不是老子对儿子的关系。要巩固团结,要建立广泛的统一战线。不管什么人,不管是黑人、白人或是黄种人,不管他信什么宗教,是天主教、基督教、伊斯兰教或佛教,也包括一部分民族资产阶级,只要是反对帝国主义的,反对帝国主义走狗的,都应该团结,只不包括帝国主义在这些国家内的走狗。

至于如何打败国内反动派的问题,我认为或者用文的办法,或者用武的办法。有些国家要号召广大人民起来用武力反对反动派,因为反动派手里有武器。这就要按照各国情况,利用适当时机,他要打,我就打。这个方法是从反动派那里学来的。我们就是从蒋介石那里学来的,蒋介石打我,我就打他。他可以打我,难道我就不能打他呀?

有人说,武器是第一,人是第二。我们反过来说,人是第一,武器是第二。武器同机器差不多,都是人手的延长而已。是人拿在武器手里,还是武器拿在人手里?当然是后者,因为武器没有手,哪个武器有手?我打了二十五年仗,包括朝鲜战争三年。我原来是不会打仗的,不知道怎样打,是通过二十五年的战争过程学会打的。我从没有看见过武器有手,只看见人有手,而人用手掌握武器。

我们的“名誉”很不好,美帝国主义者说我们是侵略者。他们说我们侵略了中国,这确实是“侵略”了蒋介石,但那是蒋介石先侵略了我们嘛!又说我们侵略了朝鲜,那是因为美帝国主义打到了鸭绿江边,我们才不得不出兵抗美援朝。还说我们侵略了印度,那是因为印度打进了我国几十公里,它打了好几年,我们才自卫还击。一打就打回了老国境。在几千公里的老国境线,印度人跑光了,没有军队,那我们就撤回来了,撤到帝国主义规定的所谓新国境线\mnote{9}这条线我们是不承认的。我们从这里后退二十公里,设立了缓冲地区。帝国主义者还说我们是好战分子,原因是我们过去帮助胡志明打法国人,现在又支援越南南方打美国人,我们也支持过本·贝拉打法国人。哪个地方需要支持,我们就支持,因此就“名誉”不好,当了“好战分子”

安哥拉朋友问,建立独立的民族经济要防止哪些幻想和危险?由于安哥拉现在还没有解放,仍要搞武装斗争,你们现在只能搞革命,经济建设只能在根据地搞一些。葡萄牙是不会帮助你们的。美国的帮助是别有用心的。如果说要防止幻想,我想要防止对美国的幻想。至于建设过程中会出现哪些危险,现在很难说。如果要说防止危险,就是防止从帝国主义方面来的危险。至于实际工作犯些错误,那是难免的。哪个政党都要犯错误的,中国共产党就犯过很多错误,犯过重大错误。犯了错误,改正就是了。错误能帮助人头脑清醒。

\begin{maonote}
\mnitem{1}汪精卫(一八八三——一九四四),浙江山阴(今绍兴)人。一九二五年在广州任国民党政府主席。一九二七年七月在武汉发动反革命政变。一九三七年抗日战争爆发后,任国民党国防最高会议副主席,主张对日妥协,是国民党内亲日派首领。一九三八年三月任国民党副总裁,同年十二月公开投降日本帝国主义,后任日本帝国主义扶植的南京傀儡政府主席。
\mnitem{2}康德皇帝,即爱新觉罗·溥仪(一九〇六——一九六七),北京人,清朝末代皇帝。一九一二年中华民国建立后被迫退位。一九三二年在日本帝国主义一手策划下出任伪满洲国“执政”。一九三四年改称“满洲帝国皇帝”,年号康德。一九四五年日本投降后被苏军俘虏,一九五〇年八月被移交给中华人民共和国政府。一九五九年十二月被特赦释放。一九六四年后任政协全国委员会委员。
\mnitem{3}南乡三郎,一九五五年出任日中输出入组合理事长。一九五六年曾两次访问中国。一九五八年作为日本通商使节团代表来华参加签订第四次中日贸易协定。
\mnitem{4}苏伊士运河位于埃及的东北部,是连接地中海和红海的国际通航运河。它处于欧、亚、非三洲交界地带的要冲,战略地位十分重要。一八六九年正式通航后,英、法两国垄断了苏伊士运河公司的绝大部分股份,每年从中获得巨额利润,英国还在运河地区建立了海外最大的军事基地。第二次世界大战后,埃及人民为收回苏伊士运河的主权进行了不懈的斗争。一九五六年七月二十六日,埃及政府宣布将苏伊士运河公司收归国有。中国政府及世界许多国家的政府和领导人发表声明,支持埃及的正义行动。同年十月,英、法和以色列发动侵略埃及的战争,妄图重新夺取运河,结果遭到失败。
\mnitem{5}巴蒂斯塔(一九〇一——一九七三),古巴前总统。执政期间实行独裁统治,使古巴完全从属于美国。一九五九年一月一日,其政权被卡斯特罗领导的起义军推翻。
\mnitem{6}菲德尔·卡斯特罗,一九二六年生,一九五九年起任古巴总理。
\mnitem{7}胡志明(一八九〇——一九六九),时任越南劳动党中央委员会主席、越南民主共和国主席。
\mnitem{8}本·贝拉,指艾哈迈德·本·贝拉,一九一八年生,阿尔及利亚民族解放阵线领导人之一。一九五六年因积极参与组织发动全国反法武装起义,被法国殖民当局监禁。一九五八年阿尔及利亚临时政府成立时,被缺席推选为第一副总理。一九六二年获释回国,同年九月阿尔及利亚民主人民共和国成立,任政府总理。一九六三年九月当选第一任总统兼武装部队最高统帅。一九六四年四月任民族解放阵线总书记。
\mnitem{9}指麦克马洪线。它是一九一四年三月英国殖民主义者背着中国中央政府代表,同西藏地方当局以秘密换文方式制造的一条非法边界线。该线将位于中印边界东段历来属于中国的九万平方公里土地划归当时英国殖民统治下的印度。中国政府从未批准或承认这条边界线。一九五三年,印度基本上侵占了该线以南的中国领土。
\end{maonote}
