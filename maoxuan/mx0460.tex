
\title{南京政府向何处去?}
\date{一九四九年四月四日}
\maketitle


两条路摆在南京国民党政府及其军政人员的面前:一条是向蒋介石战犯集团及其主人美国帝国主义靠拢,这就是继续与人民为敌,而在人民解放战争中和蒋介石战犯集团同归于尽;一条是向人民靠拢,这就是与蒋介石战犯集团和美国帝国主义决裂,而在人民解放战争中立功赎罪,以求得人民的宽恕和谅解。第三条路是没有的。

在南京的李宗仁何应钦政府\mnote{1}中,存在着三部分人。一部分人坚持地走第一条路。无论他们在口头上怎样说得好听,在行动上他们是继续备战,继续卖国,继续压迫和屠杀要求真和平的人民。他们是蒋介石的死党。一部分人愿意走第二条路,但是他们还不能作出有决定性的行动。第三部分是一些徘徊歧路、动向不明的人们。他们既不想得罪蒋介石和美国政府,又想得到人民民主阵营的谅解和容纳。但这是幻想,是不可能的。

南京的李宗仁何应钦政府,基本上是第一部分人和第三部分人的混合物,第二部分人为数甚少。这个政府到今天为止,仍然是蒋介石和美国政府的工具。

四月一日发生于南京的惨案\mnote{2},不是什么偶然的事件。这是李宗仁何应钦政府保护蒋介石、保护蒋介石死党、保护美国侵略势力的必然结果。这是李宗仁何应钦政府和蒋介石死党一同荒谬地鼓吹所谓“平等的光荣的和平”,借以抵抗中共八项和平条件\mnote{3},特别是抵抗惩办战争罪犯的结果。李宗仁何应钦政府既然派出和谈代表团前来北平同中国共产党谈判和平,并表示愿意接受中国共产党的八项条件以为谈判的基础,那末,如果这个政府是有最低限度的诚意,就应当以处理南京惨案为起点,逮捕并严惩主凶蒋介石、汤恩伯、张耀明,逮捕并严惩在南京上海的特务暴徒,逮捕并严惩那些坚决反对和平、积极破坏和谈、积极准备抵抗人民解放军向长江以南推进的反革命首要。庆父不死,鲁难未已\mnote{4}。战犯不除,国无宁日。这个真理,难道现在还不明白吗?

我们愿意正告南京政府:如果你们没有能力办这件事,那末,你们也应协助即将渡江南进的人民解放军去办这件事。时至今日,一切空话不必说了,还是做件切实的工作,借以立功自赎为好。免得逃难,免得再受蒋介石死党的气,免得永远被人民所唾弃。只有这一次机会了,不要失掉这个机会。人民解放军就要向江南进军了。这不是拿空话吓你们,无论你们签订接受八项条件的协定也好,不签这个协定也好,人民解放军总是要前进的。签一个协定而后前进,对几方面都有利——对人民有利,对人民解放军有利,对国民党政府系统中一切愿意立功自赎的人们有利,对国民党军队的广大官兵有利,只对蒋介石,对蒋介石死党,对帝国主义者不利。不签这个协定,情况也差不多,可以用局部谈判的方法去解决。可能还有些战斗,但是不会有很多的战斗了。从新疆到台湾这样广大的地区内和漫长的战线上,国民党只有一百一十万左右的作战部队了,没有很多的仗可打了。无论签订一个全面性的协定也好,不签这个协定而签许多局部性的协定也好,对于蒋介石,对于蒋介石死党,对于美国帝国主义,一句话,对于一切至死不变的反动派,情况都是一样的,他们将决定地要灭亡。也许签订一个全面性协定对于南京方面和我们方面,都比较不签这个协定,来得稍微有利一些,所以我们还是争取签订这个协定。但是签订这个全面性协定,我们须得准备应付许多拖泥带水的事情。不签这个协定而去签订许多局部协定,对于我们要爽快得多。虽然如此,我们还是准备签订这个协定。南京政府及其代表团如果也愿意这样做,那末,就得在这几天下决心,一切幻想和一切空话都应当抛弃了。我们并不强迫你们下这个决心。南京政府及其代表团是否下这个决心,有你们自己的自由。就是说,你们或者听蒋介石和司徒雷登\mnote{5}的话,并和他们永远站在一起,或者听我们的话,和我们站在一起,对于这二者的选择,有你们自己的自由。但是选择的时间没有很多了,人民解放军就要进军了,一点游移的余地也没有了。


\begin{maonote}
\mnitem{1}一九四九年三月十二日,李宗仁在孙科辞职后,任命何应钦继任行政院长。
\mnitem{2}一九四九年四月一日,南京十一个专科以上学校的学生六千余人举行游行示威,要求国民党反动政府接受中国共产党的八项和平谈判条件。国民党南京卫戍总司令张耀明在蒋介石授意下,指使军警特务凶殴示威学生,死学生二人,伤一百余人。
\mnitem{3}见本卷\mxart{中共中央毛泽东主席关于时局的声明}。
\mnitem{4}事见《左传·闵公元年》。庆父是春秋时鲁国的公子,曾经一再制造鲁国的内乱,先后杀死两个国君。当时的人有“不去庆父,鲁难未已”的说法。后人常常把制造内乱的人比之为庆父。
\mnitem{5}见本卷\mxnote{别了,司徒雷登}{1}。
\end{maonote}
