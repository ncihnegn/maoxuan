
\title{加强相互学习,克服固步自封、骄傲自满}
\date{一九六三年十二月十三日}
\thanks{这是毛泽东同志为中共中央起草的党内指示。}
\maketitle


\mxname{各中央局,各省、市、区党委,军委,中央各人民团体党委,各部委会党委、党组:}

现将湖南省委李瑞山、华国锋两同志一九六三年十一月六日写的一个参观广东农业生产情况的报告\mnote{1}以及附在上面的湖南省委一九六三年十二月七日写的一个指示\mnote{2},发给你们研究。中央认为,这种虚心学习外省、外市、外区优良经验的态度和办法,是很好的,是发展我国经济、政治、思想、文化、军事、党务的重要方法之一。固步自封,骄傲自满,对于自己所管区域的工作,不采取马克思主义的辩证分析方法(一分为二,既有成绩,也有缺点错误),只研究成绩一方面,不研究缺点错误一方面。只爱听赞扬的话,不爱听批评的话。对于外省、外市、外区、别的单位的工作,很少有兴趣组织得力高级中级干部去虚心地认真地加以考察,以便和本省、本市、本区、本单位的情况结合起来,加以改进。永远限于本地区本单位这个狭隘世界,不能打开自己的眼界,不知还有别的新天地,这叫做夜郎自大。对外国人、外地人以及中央派下去的人,只让看好的,不让看坏的。只向他们谈成绩,不向他们谈缺点及错误,要谈也谈得不深刻,敷衍几句了事。中央多次对同志们提出这个问题,认为一个共产党人必须具备对于成绩与缺点、真理与错误这个两分法的马克思主义辩证思想。事物(经济、政治、思想、文化、军事、党务等等)总是作为过程而向前发展的。而任何一个过程,都是由矛盾着的两个侧面互相联系又互相斗争而得到发展的。这应当是马克思主义者的普通常识。但是,中央和各地同志中,有许多人却很少认真地用这种观点去思索去工作。他们的头脑,长期存在着形而上学的思想方法而不能解脱。所谓形而上学,就是否认事物的对立统一、对立斗争(两分法)、矛盾着对立着的事物在一定条件下互相转化走向它们的反面,这样一个真理;就是人们固步自封、骄傲自满,只见成绩,不见缺点,只愿听好话,不愿听批评话,自己不愿意批评(对自己的两分法),更怕别人批评。中央有几十个部,明明有几个工作成绩工作作风较好的部,例如石油部,别的部却视若无睹,永远不去那里考察研究请教一番。一个部所管企业事业,明明有许多厂矿、企业、事业、科学研究处所及其人员,工作做得较好,上面却不知道,因而也不能提倡人们向那些单位学习。同志们,中央在这里所说的犯有形而上学错误的同志是指一部分同志,不是指全部同志。但是,应当指出,有大量的好同志却被那些高官厚禄、养尊处优、骄傲自满、固步自封、爱好资产阶级形而上学的同志们,亦即官僚主义者,所压住了,现在必须加以改革。凡不虚心地认真地对本地本单位本人作分析,对别地别单位别人作分析,拒绝马克思主义辩证分析方法的同志,要进行同志式的劝告和批评,以便把不良情况改变过来。把向别部、别省、别市、别区、别单位的好经验、好作风、好方法学过来这样一种方法,定为制度。这个问题是一个大问题,请你们加以讨论。以后还要在中央工作会议及中央全会上加以讨论。湖南省委过去在一个时期内,不作调查研究,主观主义地下达许多指示,往下灌的东西多,由下面反映上来的真实情况少,因而脱离群众,产生很大困难。从一九六一年起,他们开始改变了,以至情况大好起来。但是他们认为还是远不如广东和上海,所以他们派遣大批省、地、县三级干部,还有省和市的干部,组成两个考察团,分别到广东上海去学习。这一点,请你们注意研究,是否也可以这样办。中央认为,不但可以而且应当这样办。如有不同意见,请你们提出。

\begin{maonote}
\mnitem{1}中共湖南省委书记处书记李瑞山、华国锋给中共湖南省委的报告说:参观广东省的农业生产,感到广东省在一直坚持大办水利、推广良种、合理密植、办劳动大学方面,在贯彻以粮为纲、全面发展、多种经营的方针方面,在提高单位面积产量方面,在抓水、抓肥、抓种子方面,在抓经济政策方面,在农业生产上走群众路线、大搞群众运动方面,都非常突出。通过参观,大家感到扩大了眼界,解放了思想,开阔了胸襟增强了信心,鼓舞了干劲。同时也学习了许多宝贵的经验。应当把广东的经验同湖南的具体情况结合起来,因地制宜地加以运用和推广。
\mnitem{2}中共湖南省委转发李瑞山、华国锋的报告给湖南各地、市、县委,省委各部门,各厅局党组的指示说:省委认为,这个报告很重要,必须一直发到公社党委和厂矿党委。让大家照照镜子,展开讨论,把从先进地区、先进单位学来的经验,很好地运用到我们自己的工作中去。学习外地经验要和总结自己的经验相结合。同时,领导亲自搞试点是特别重要的。只有树立出了样板,才能带动全面,取得更显著的成效。
\end{maonote}
