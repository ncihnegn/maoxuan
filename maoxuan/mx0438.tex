
\title{新解放区农村工作的策略问题}
\date{一九四八年五月二十四日}
\thanks{这是毛泽东给邓小平的电报。}
\maketitle


新解放区农村工作的策略问题有全盘考虑之必要。新解放区必须充分利用抗日时期的经验,在解放后的相当时期内,实行减租减息和酌量调剂种子口粮的社会政策和合理负担的财政政策,把主要的打击对象限于政治上站在国民党方面坚决反对我党我军的重要反革命分子,如同抗日时期只逮捕汉奸分子和没收他们的财产一样,而不是立即实行分浮财、分土地的社会改革政策。因为过早地分浮财,只是少数勇敢分子欢迎,基本群众并未分得,因而会表示不满。而且,社会财富迅速分散,于军队亦不利。过早地分土地,使军需负担过早地全部落在农民身上,不是落在地主富农身上。不如不分浮财,不分土地,在社会改革上普遍实行减租减息,使农民得到实益;在财政政策上实行合理负担,使地主富农多出钱。这样,社会财富不分散,社会秩序较稳定,利于集中一切力量消灭国民党反动派。在一两年甚至三年以后,在大块根据地上,国民党反动派已被消灭,环境已经安定,群众已经觉悟和组织起来,战争已经向遥远地方推进,那时就可进入像华北那样的分浮财、分土地的土地改革阶段。这一个减租减息阶段是任何新解放地区所不能缺少的,缺少了这个阶段,我们就要犯错误。就是在华北、东北、西北各大解放区的接敌地区,亦须实行上述同样的策略。
