
\title{给清华大学附属中学红卫兵的信}
\date{一九六六年八月一日}
\thanks{这是毛泽东同志准备答复红卫兵的信没有发出但作为八届十一中全会文件印发了。}
\maketitle


\mxname{清华大学附属中学红卫兵同志们:}

你们在七月二十八日寄给我的两张大字报\mnote{1}以及转给我要我回答的信,都收到了。你们在六月二十四日和七月四日的两张大字报,说明对一切剥削压迫工人、农民、革命知识分子和革命党派的地主阶级、资产阶级、帝国主义、修正主义和他们的走狗,表示愤怒和申讨,说明对反动派造反有理,我向你们表示热烈的支持。同时我对北京大学附属中学红旗战斗小组说明对反动派造反有理的大字报和由彭小蒙同志\mnote{2}于七月二十五日在北京大学全体师生员工大会上,代表她们红旗战斗小组所作的很好的革命演说,表示热烈的支持。在这里,我要说,我和我的革命战友,都是采取同样态度的。不论在北京,在全国,在文化大革命运动中,凡是同你们采取同样革命态度的人们,我们一律给予热烈的支持。还有,我们支持你们,我们又要求你们注意争取团结一切可以团结的人们。对于犯有严重错误的人们,在指出他们的错误以后,也要给以工作和改正错误重新作人的出路。马克思说,无产阶级不但要解放自己,而且要解放全人类。如果不能解放全人类,无产阶级自己就不能最后地得到解放。这个道理,也请同志们予以注意。

\begin{maonote}
\mnitem{1}指清华大学附属中学的红卫兵组织写的六月二十四日的《无产阶级的革命造反精神万岁》和七月四日的《再论无产阶级的革命造反精神万岁》,表示造反有理,要一反到底。七月二十八日,清华附中红卫兵举行海淀区中学革命师生代表大会,把他们写的“论革命造反精神”的大字报和给毛泽东的信交给出席大会的江青,请她转交,并请毛泽东对他们的观点及与工作组的争论作指示。
\mnitem{2}彭小蒙,北京大学附属中学红卫兵组织“红旗战斗小组”的负责人之一,她在一九六六年七月二十五日的北京大学全体师生大会上,发表了批判工作组的演讲。
\end{maonote}
