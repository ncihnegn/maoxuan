
\title{关于《红楼梦》研究问题的信}
\date{一九五四年十月十六日}
\thanks{这是毛泽东同志写给中共中央政治局的同志和其它有关同志的一封信。}
\maketitle


驳俞平伯的两篇文章付上,请一阅。这是三十多年以来向所谓《红楼梦》研究权威作家的错误观点的第一次认真的开火。作者是两个青年团员。他们起初写信给《文艺报》请问可不可以批评俞平伯,被置之不理。他们不得已写信给他们的母校——山东大学的老师,获得了支持,并在该校刊物《文史哲》上注销了他们的文章驳《红楼梦简论》。问题又回到北京,有人要求将此文在《人民日报》上转载,以期引起争论,展开批评,又被某些人以种种理由(主要是“小人物的文章”,“党报不是自由辩论的场所”)给以反对,不能实现;结果成立妥协,被允许在《文艺报》转载此文。嗣后,《光明日报》的《文学遗产》栏又发表了这两个青年的驳俞平伯《红楼梦研究》一书的文章。看样子,这个反对在古典文学领域毒害青年三十余年的胡适派资产阶级唯心论的斗争,也许可以开展起来了。事情是两个“小人物”做起来的,而“大人物”往往不注意,并往往加以拦阻,他们同资产阶级作家在唯心论方面讲统一战线,甘心作资产阶级的俘虏,这同影片《清宫秘史》\mnote{1}和《武训传》放映时候的情形几乎是相同的。被人称为爱国主义影片而实际是卖国主义影片的《清宫秘史》,在全国放映之后,至今没有被批判。《武训传》虽然批判了,却至今没有引出教训,又出现了容忍俞平伯唯心论和阻拦“小人物”的很有生气的批判文章的奇怪事情,这是值得我们注意的。

俞平伯这一类资产阶级知识分子,当然是应当对他们采取团结态度的,但应当批判他们的毒害青年的错误思想,不应当对他们投降。


\begin{maonote}
\mnitem{1}《清宫秘史》是一部污蔑义和团爱国运动,鼓吹投降帝国主义的反动影片。刘少奇把这部卖国主义影片吹捧为“爱国主义”影片。
\end{maonote}
