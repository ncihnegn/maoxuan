
\title{关于知识青年到农村去的号召}
\date{一九六八年十二月十一日}
\thanks{这是毛泽东同志对上山下乡运动\mnote{1}的指示。}
\maketitle


知识青年到农村去,接受贫下中农的再教育,很有必要。要说服城里干部和其他人,把自己初中、高中、大学毕业的子女,送到乡下去,来一个动员。各地农村的同志应当欢迎他们去。

\begin{maonote}
\mnitem{1}“上山下乡”一词,源于一九五六年一月中共中央政治局拟定的《一九五六年到一九五七年全国农业发展纲要(草案)》。纲要提到,“城市的中小学毕业的青年,除了能够在城市升学就业的以外,应当积极响应国家的号召,下乡上山去参加农业生产,参加社会主义农村建设的伟大事业”。

上山下乡运动最早可以追溯到一九五五年。当年,河南省郏县有一批中学生回乡参加农业合作化运动,毛泽东为此写道:

“组织中学生和高小毕业生参加合作化的工作,值得特别注意。一切可以到农村去工作的知识分子,应当高兴地到那里去。农村是一个广阔的天地,在那里是可以大有作为的。”

一九六四年,中共中央、国务院发布了《关于城镇青年参加农村社会主义建设的决定》草案,这是上山下乡运动一个纲领性文件。一九六五年中央办公厅为此发布了通知。从此,上山下乡被列入党和国家重要的日常工作范围。

一九六八年十二月二十二日,《人民日报》发表毛泽东的号召后,全国知识青年上山下乡的高潮出现。至一九七八年,共有1400万知识青年上山下乡,改变了城市和农村的面貌。

知识青年在农村办学讲课,担任低薪金的教员,使得农村的中小学入学率及识字率大大提高。上山下乡运动对我国的教育普及起到了极大的推动作用,这一时期我国的人均教育水平获得极大幅度的提高,识字率和入学率大规模暴增,小学入学率即由一九六三年的57\%,大幅提升至一九七六年的96\%(同期印度小学入学率为一九六一年40\%,至一九七八年上升仅为58\%)。在增加基础教育的同时,还取消职业中学,大幅度提升初中及高中普通中学学额,普通初中招生数从一九六三年263.5万大升至一九七六年2344.3万,普通高中招生数从一九六三年43.3万大升至一九七六年861.1万。

同样由于知青下乡运动,合作医疗制度得以建立,大批知青从事赤脚医生的职业,建立了覆盖全国范围的提供保障的医疗保健制度。一九六九年九月二十五日,《人民日报》宣布“在全国范围内实现药品全面大幅度降价”,文章指出:

“遵照伟大领袖毛主席“把医疗卫生工作的重点放到农村去”的光辉指示,这次降价幅度最大的是劳动人民特别是农村广大劳动人民常用的普通药品。如一支二十万单位青霉素的价格,比降价前下降了46\%,比一九五二年下降了90\%;一支一百万单位的双氢链霉素的价格,比降价前下降了63\%,比一九五二年下降了94\%;消炎片的价格,比降价前降低了13.3\%,比一九五二年下降了67\%;磺胺嘧啶片的价格比降价前下降了50\%,比一九五二年下降了16.5\%。最常用的、原来价格就很低的解热止痛药去痛片的价格,这一次又下降了20\%。

在药品全面降价的同时,还实行了全国统一药价的革命措施,取消了地区差价,改变了过去由于实行地区差价造成的越是边远地区药价越高的现象,减轻了山区、农村和边疆各族广大劳动人民的药费负担。药品价格的大幅度下降和地区差价的取消,对发展和巩固农村合作医疗制度,促进农村医疗卫生事业的发展,有着重要的意义。”

知青下乡运动为缩小“三大差别”(即工农差别、城乡差别、和体力与脑力劳动差别)做出了巨大贡献,对于农村的教育普及、合作医疗制度的建立乡镇企业的建立(乡镇企业成立多是知青推动,初期其业务骨干也大多是知青担任)都起到了决定性作用,大幅度改变了农民面貌。广大知识青年在农村的艰苦生活磨练了他们的意志品质,了解了生活,也为今后成就事业奠定了坚实的基础。
\end{maonote}
