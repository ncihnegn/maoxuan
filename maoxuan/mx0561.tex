
\title{事情正在起变化}
\date{一九五七年五月十五日}
\thanks{这是毛泽东同志写的一篇文章,曾发给党内干部阅读。}
\maketitle


对立面的统一和斗争,是社会生活中普遍存在的。斗争的结果,走向自己的反面,建立新的统一,社会生活就前进了一步。

共产党整风,是一个统一体两种作风之间的斗争。在共产党内部如此,在整个人民的内部也是如此。

在共产党内部,有各种人。有马克思主义者,这是大多数。他们也有缺点,但不严重。有一部分人有教条主义错误思想。这些人大都是忠心耿耿,为党为国的,就是看问题的方法有“左”的片面性。克服了这种片面性,他们就会大进一步。又有一部分人有修正主义或右倾机会主义错误思想。这些人比较危险,因为他们的思想是资产阶级思想在党内的反映,他们向往资产阶级自由主义,否定一切,他们与社会上资产阶级知识分子有千丝万缕的联系。几个月以来,人们都在批判教条主义,却放过了修正主义。教条主义应当受到批判,不批判教条主义,许多错事不能改正。现在应当开始注意批判修正多义。教条主义走向反面,或者是马克思主义,或者是修正主义。就我党的经验说来,前者为多,后者只是个别的,因为他们是无产阶级的一个思想派别,沾染了小资产阶级的狂热观点。有些被攻击的“教条主义”,实际上是一些工作上的错误。有些被攻击的“教条主义”,实际上是马克思主义,被一些人误认作“教条主义”而加以攻击。真正的教条主义分子觉得“左”比右好是有原因的,因为他们要革命。但是对于革命事业的损失说来,“左”比右并没有什么好,因此应当坚决改正。有些错误,是因为执行中央方针而犯的,不应当过多地责备下级。我党有大批的知识分子新党员(青年团员就更多),其中有一部分确实具有相当严重的修正主义思想。他们否认报纸的党性和阶级性,他们混同无产阶级新闻事业与资产阶级新闻事业的原则区别,他们混同反映社会主义国家集体经济的新闻事业与反映资本主义国家无政府状态和集团竞争的经济的新闻事业。他们欣赏资产阶级自由主义,反对党的领导。他们赞成民主,反对集中。他们反对为了实现计划经济所必需的对于文化教育事业(包括新闻事业在内的)必要的但不是过分集中的领导、计划和控制。他们跟社会上的右翼知识分子互相呼应,联成一起,亲如弟兄。批判教条主义的有各种人。有共产党人——一马克思主义者。有括号里面的“共产党人”,即共产党的右派——修正主义者。有社会上的左派、中间派和右派。社会上的中间派是大量的,他们大约占全体党外知识分子的百分之七十左右,而左派大约占百分之二十左右,右派大约占百分之一、百分之三、百分之五到百分之十,依情况而不同。

最近这个时期,在民主党派中和高等学校中,右派表现得最坚决最猖狂。他们以为中间派是他们的人,不会跟共产党走了,其实是做梦。中间派中有一些人是动摇的,是可左可右的,现在在右派猖狂进攻的声势下,不想说话,他们要等一下。现在右派的进攻还没有达到顶点,他们正在兴高采烈。党内党外的右派都不懂辩证法:物极必反。我们还要让他们猖狂一个时期,让他们走到顶点。他们越猖狂,对于我们越有利益。人们说:怕钓鱼,或者说:诱敌深入,聚而歼之。现在大批的鱼自己浮到水面上来了,并不要钓。这种鱼不是普通的鱼,大概是鲨鱼吧,具有利牙,欢喜吃人。人们吃的鱼翅,就是这种鱼的浮游工具。我们和右派的斗争集中在争夺中间派,中间派是可以争取过来的。什么拥护人民民主专政,拥护人民政府,拥护社会主义,拥护共产党的领导,对于右派说来都是假的,切记不要相信。不论是民主党派内的右派,教育界的右派,文学艺术界的右派,新闻界的右派,科技界的右派,工商界的右派,都是如此。有两派最坚决:左派和右派。他们互争中间派,互争对中间派的领导权。右派的企图,先争局部,后争全部。先争新闻界、教育界、文艺界、科技界的领导权。他们知道,共产党在这些方面不如他们,情况也正是如此。他们是“国宝”,是惹不得的。过去的“三反”,肃反,思想改造,岂有此理!太岁头上动土!他们又知道许多大学生属于地主、富农、资产阶级的儿女,认为这些人是可以听右派号召起来的群众。有一部分有右倾思想的学生,有此可能。对大多数学生这样设想,则是做梦。新闻界右派还有号召工农群众反对政府的迹象。

人们反对扣帽子,这只是反对共产党扣他们的帽子。他们扣共产党的帽子,扣民主党派左派中派和社会各界左派中派的帽子则是可以的。几个月以来,报纸上从右派手上飞出了多少帽子呢!中间派反对扣帽子是真实的。我们对中间派过去所扣一切不适当的帽子都要取掉,以后也不要乱扣。在“三反”中,在肃反中,在思想改造中,某些真正做错了的事,都要公开改正,不论对什么人的。只有扣帽子一事,对右派当别论。但是也要扣得对,确是右派才给他扣上右派这顶帽子。除个别例外,不必具体指名,给他们留一个回旋余地,以利在适当条件下妥协下来。所谓百分之一、百分之三、百分之五到百分之十的右派是一种估计,可能多些,可能少些。在各个单位内情况又互相区别,必须确有证据,实事求是,不可过分,过分就是错误。

资产阶级和曾经为旧社会服务过的知识分子的许多人总是要顽强地表现他们自己,总是留恋他们的旧世界,对于新世界总有些格格不入。要改造他们,需要很长的时间,而且不可用粗暴方法。但是必须估计到他们的大多数,较之解放初期是大为进步了,他们对我们提出的批评,大多数是对的,必须接受。只有一部分不对,应当解释。他们要求信任,要求有职有权,是对的,必须信用他们,必须给以权责。右派的批评也有一些是对的,不能一概抹杀。凡对的就应采纳。右派的特征是他们的政治态度右。他们同我们有一种形式上的合作,实际上不合作。有些事合作,有些事不合作。平时合作,一遇有空子可钻,如象现在这样时机,就在实际上不想合作了。他们违背愿意接受共产党领导的诺言,他们企图摆脱这种领导。而只要没有这种领导,社会主义就不能建成,我们民族就要受到绝大的灾难。

全国有几百万资产阶级和曾为旧社会服务的知识分子,我们需要这些人为我们工作,我们必须进一步改善和他们的关系,以便使他们更有效率地为社会主义事业服务,以便进一步改造他们,使他们逐步地工人阶级化,走向现状的反面。大多数人一定可以达到这个目的。改造就是又团结,又斗争,以斗争之手段,达团结之目的。斗争是互相斗争,现在是许多人向我们进行斗争的时候了。多数人的批评合理,或者基本上合理,包括北京大学傅鹰教授那种尖锐的没有在报纸上发表的批评在内。这些人的批评目的,就是希望改善相互关系,他们的批评是善意的。右派的批评往往是恶意的,他们怀着敌对情绪。善意,恶意,不是猜想的,是可以看得出来的。

这一次批评运动和整风运动是共产党发动的。毒草共香花同生,牛鬼蛇神与麟凤龟龙并长,这是我们所料到的,也是我们所希望的。毕竟好的是多数,坏的是少数。人们说钓大鱼,我们说锄毒草,事情一样,说法不同。有反共情绪的右派分子为了达到他们的企图,他们不顾一切,想要在中国这块土地上刮起一阵害禾稼、毁房屋的七级以上的台风。他们越做得不合理,就会越快地把他们抛到过去假装合作、假装接受共产党领导的反面,让人民认识他们不过是一小撮反共反人民的牛鬼蛇神而已。那时他们就会把自己埋葬起来。这有什么不好呢?

右派有两条出路。一条,夹紧尾巴,改邪归正。一条,继续胡闹,自取灭亡。右派先生们,何去何从,主动权(一个短期内)在你们手里。

在我们的国家里,鉴别资产阶级及资产阶级知识分子在政治上的真假善恶,有几个标准。主要是看人们是否真正要社会主义和真正接受共产党的领导。这两条,他们早就承认了,现在有些人想翻案,那不行。只要他们翻这两条案,中华人民共和国就没有他们的位置。那是西方世界(一名自由国家)的理想,还是请你们到那里去吧!

大量的反动的乌烟瘴气的言论为什么允许登在报上?这是为了让人民见识这些毒草、毒气,以便锄掉它,灭掉它。

“你们这一篇话为什么不早讲?”为什么没有早讲?我们不是早已讲了一切毒草必须锄掉吗?

“你们把人们划分为左。中、右,未免不合情况吧?”除了沙漠,凡有人群的地方,都有左、中、右,一万年以后还会是这样。为什么不合情况?划分了,使群众有一个观察人们的方向,便于争取中间,孤立右派。

“为什么不争取右派?”要争取的。只有在他们感到孤立的时候,才有争取的可能。现在,他们的尾巴翘到天上去了,他们妄图灭掉共产党,那肯就范?孤立就会起分化,我们必须分化右派。我们从来就是把人群分为左、中、右,或叫进步、中间、落后,不自今日始,一些人健忘罢了。

是不是要大“整”?要看右派先生们今后行为作决定。毒草是要锄的,这是意识形态上的锄毒草。“整”人是又一件事。不到某人“严重违法乱纪”是不会受“整”的。什么叫“严重违法乱纪”?就是国家的利益和人民的利益受到严重的损害,而这种损害,是在屡戒不听一意孤行的情况下引起的。其它普通犯错误的人,更是治病救人。这是一个恰当的限度,党内党外一律如此。“整”也是治病救人。

要多少时间才可以把党的整风任务完成?现在情况进展甚快,党群关系将迅速改善。看样子,有的几星期,有的几个月,有的一年左右(例如农村),就可完成。至于学习马克思主义,提高思想水平,那就时间要长些。

我们同资产阶级和知识分子的又团结又斗争,将是长期的。共产党整风告一段落之后,我们将建议各民主党派和社会各界实行整风,这样将加速他们的进步,更易孤立少数右翼分子。现在是党外人士帮助我们整风。过一会我们帮助党外人士整风。这就是互相帮助,使歪风整掉,走向反面,变为正风。人民正是这样希望于我们的,我们应当满足人民的希望。
