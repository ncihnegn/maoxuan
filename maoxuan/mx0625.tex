
\title{要政治家办报}
\date{一九五九年六月十三日}
\thanks{这是毛泽东同志同吴冷西的谈话要点。}
\maketitle


现在名气很糟。去年大出风头,今年不好,说我们是纸老虎,不讲信用。

这很好。

不仅敌人,而且朋友\mnote{1}也觉得我们不行。去年名气很大,人人怕我们,不但美英怕我们,很惊慌,朋友也怕,他们也感到了压力,为什么能这么跃进?现在不太怕了。英国说,三十年内中国不是值得重视的力量。朋友还说,中国不是高速度,说话不那么拘谨了。我们自己不那么神气了。

去年三月,我在成都曾说:不要务虚名而得实祸。

去年从九月开始一直被动。大家头脑发胀,要搞四亿吨钢,大谈共产主义。去年十一月到郑州才发现,我狠狠批评了一下。当时大家都在风头上,要把指标从三千万吨落到二千万吨,也难。我在武汉会议上曾提,不要搞一千八百万吨也好,一千五百万吨也成。马克思写《资本论》一百年了,看来经验要自己取得。法则不能违反,要学习政治经济学第三版。过去学了就完了,谁也没有注意价值法则,违反了它就头破血流。

现在失了信用,不要紧。苦战一年,再加一年。那时宣布跃进成绩,现在不要更正,将来再说。

人民日报办得比过去好,老气没有了。但吹的太大,有办成中央日报的危险,新华社也有办成中央社的危险。

人民日报,我只看一些消息,但“参考资料”、“内部参考”,每天必看。“内部参考”是个好刊物,要改进,不要只看现象。大局在“内部参考”。怎样把“内部参考”变成报纸,是你们的工作。

“新闻工作动向”是好刊物。有一期反映了日本专家的意见,这很好。太照顾对象也不好。要吸引人看,要吸引人听。“新闻工作动向”上地方报纸提出的问题,这些问题要多反映。有的读者反对解放日报的编辑,读者是对的。口径不一致,是解放日报对还是毛主席对?\mnote{2}

新闻工作,要看是政治家办,还是书生办。有些人是书生,书生最大的缺点是多谋寡断。刘备、孙权、袁绍,都有这个缺点,曹操就多谋善断。

搞新闻工作,要政治家办报。你到人民日报工作,要有充分思想准备,要准备遇到最坏情况,要有“五不怕”的精神准备。这“五不怕”就是:一不怕撤职,二不怕开除党籍,三不怕离婚。四不怕坐牢,五不怕杀头。有了这五不怕的准备,就敢于实事求是,敢于坚持真理了。

撤职和开除党籍并不罕见,要准备着。杀头在正确路线领导下大概不至于,现在的中央不同于王明左倾路线领导,也不同于张国焘。但对坐牢得有精神准备。共产党内一时受冤屈的事还是有的,不过在正确路线领导下终究会平反纠正的。一个共产党员要经得起受到错误的处分,可能这样对自己反而有益处。

你们要政治家办报,不要书生办报。

\begin{maonote}
\mnitem{1}朋友,指当时苏联领导人赫鲁晓夫。
\mnitem{2}这里所说的材料,都在“新闻工作动向”第四十五期上,本书略。
\end{maonote}
