
\title{整顿党的作风}
\date{一九四二年二月一日}
\thanks{这是毛泽东在中共中央党校开学典礼上的演说。}
\maketitle


党校今天开学,我庆祝这个学校的成功。

今天我想讲一点关于我们的党的作风的问题。

为什么要有革命党?因为世界上有压迫人民的敌人存在,人民要推翻敌人的压迫,所以要有革命党。就资本主义和帝国主义时代说来,就需要一个如共产党这样的革命党。如果没有共产党这样的革命党,人民要想推翻敌人的压迫,简直是不可能的。我们是共产党,我们要领导人民打倒敌人,我们的队伍就要整齐,我们的步调就要一致,兵要精,武器要好。如果不具备这些条件,那末,敌人就不会被我们打倒。

现在我们的党还有什么问题呢?党的总路线是正确的,是没有问题的,党的工作也是有成绩的。党有几十万党员,他们在领导人民,向着敌人作艰苦卓绝的斗争。这是大家看见的,是不能怀疑的。

那末,究竟我们的党还有什么问题没有呢?我讲,还是有问题的,而且就某种意义上讲,问题还相当严重。

什么问题呢?就是有几样东西在一些同志的头脑中还显得不大正确,不大正派。

这就是说,我们的学风还有些不正的地方,我们的党风还有些不正的地方,我们的文风也有些不正的地方。所谓学风有些不正,就是说有主观主义的毛病。所谓党风有些不正,就是说有宗派主义的毛病。所谓文风有些不正,就是说有党八股\mnote{1}的毛病。这些作风不正,并不像冬天刮的北风那样,满天都是。主观主义、宗派主义、党八股,现在已不是占统治地位的作风了,这不过是一股逆风,一股歪风,是从防空洞里跑出来的。(笑声)但是我们党内还有这样的一种风,是不好的。我们要把产生这种歪风的洞塞死。我们全党都要来做这个塞洞工作,我们党校也要做这个工作。主观主义、宗派主义、党八股,这三股歪风,有它们的历史根源,现在虽然不是占全党统治地位的东西,但是它们还在经常作怪,还在袭击我们,因此,有加以抵制之必要,有加以研究分析说明之必要。

反对主观主义以整顿学风,反对宗派主义以整顿党风,反对党八股以整顿文风,这就是我们的任务。

我们要完成打倒敌人的任务,必须完成这个整顿党内作风的任务。学风和文风也都是党的作风,都是党风。只要我们党的作风完全正派了,全国人民就会跟我们学。党外有这种不良风气的人,只要他们是善良的,就会跟我们学,改正他们的错误,这样就会影响全民族。只要我们共产党的队伍是整齐的,步调是一致的,兵是精兵,武器是好武器,那末,任何强大的敌人都是能被我们打倒的。

现在我来讲一讲主观主义。

主观主义是一种不正派的学风,它是反对马克思列宁主义的,它是和共产党不能并存的。我们要的是马克思列宁主义的学风。所谓学风,不但是学校的学风,而且是全党的学风。学风问题是领导机关、全体干部、全体党员的思想方法问题,是我们对待马克思列宁主义的态度问题,是全党同志的工作态度问题。既然是这样,学风问题就是一个非常重要的问题,就是第一个重要的问题。

现在有些糊涂观念,在许多人中间流行着,例如关于什么是理论家,什么是知识分子,什么是理论和实际联系等等问题的糊涂观念。

我们首先要问,我们党的理论水平究竟是高还是低呢?近来马克思列宁主义的书籍翻译的多了,读的人也多了。这是很好的事。但是否就可以说我们党的理论水平已经是提得很高了呢?确实,我们的理论水平是比较过去高了一些。但是按照中国革命运动的丰富内容来说,理论战线就非常之不相称,二者比较起来,理论方面就显得非常之落后。一般地说来,我们的理论还不能够和革命实践相平行,更不去说理论应该跑到实践的前面去。我们还没有把丰富的实际提高到应有的理论程度。我们还没有对革命实践的一切问题,或重大问题,加以考察,使之上升到理论的阶段。你们看,中国的经济、政治、军事、文化,我们究有多少人创造了可以称为理论的理论,算得科学形态的、周密的而不是粗枝大叶的理论呢?特别是在经济理论方面,中国资本主义的发展,从鸦片战争到现在,已经一百年了,但是还没有产生一本合乎中国经济发展的实际的、真正科学的理论书。像在中国经济问题方面,能不能说理论水平已经高了呢?能不能说我党已经有了像样的经济理论家呢?实在不能说。我们读了许多马克思列宁主义的书籍,能不能就算是有了理论家呢?不能这样说。因为马克思列宁主义是马克思、恩格斯、列宁、斯大林他们根据实际创造出来的理论,从历史实际和革命实际中抽出来的总结论。我们如果仅仅读了他们的著作,但是没有进一步地根据他们的理论来研究中国的历史实际和革命实际,没有企图在理论上来思考中国的革命实践,我们就不能妄称为马克思主义的理论家。如果我们身为中国共产党员,却对于中国问题熟视无睹,只能记诵马克思主义书本上的个别的结论和个别的原理,那末,我们在理论战线上的成绩就未免太坏了。如果一个人只知背诵马克思主义的经济学或哲学,从第一章到第十章都背得烂熟了,但是完全不能应用,这样是不是就算得一个马克思主义的理论家呢?这还是不能算理论家的。我们所要的理论家是什么样的人呢?是要这样的理论家,他们能够依据马克思列宁主义的立场、观点和方法,正确地解释历史中和革命中所发生的实际问题,能够在中国的经济、政治、军事、文化种种问题上给予科学的解释,给予理论的说明。我们要的是这样的理论家。假如要作这样的理论家,那就要能够真正领会马克思列宁主义的实质,真正领会马克思列宁主义的立场、观点和方法,真正领会列宁斯大林关于殖民地革命和中国革命的学说,并且应用了它去深刻地、科学地分析中国的实际问题,找出它的发展规律,这样才是我们真正需要的理论家。

现在我们党的中央做了决定\mnote{2},号召我们的同志学会应用马克思列宁主义的立场、观点和方法,认真地研究中国的历史,研究中国的经济、政治、军事和文化,对每一问题要根据详细的材料加以具体的分析,然后引出理论性的结论来。这个责任是担在我们的身上。

我们党校的同志不应当把马克思主义的理论当成死的教条。对于马克思主义的理论,要能够精通它、应用它,精通的目的全在于应用。如果你能应用马克思列宁主义的观点,说明一个两个实际问题,那就要受到称赞,就算有了几分成绩。被你说明的东西越多,越普遍,越深刻,你的成绩就越大。现在我们的党校也要定这个规矩,看一个学生学了马克思列宁主义以后怎样看中国问题,有看得清楚的,有看不清楚的,有会看的,有不会看的,这样来分优劣,分好坏。

其次讲一讲所谓“知识分子”的问题。因为我们中国是一个半殖民地半封建的国家,文化不发达,所以对于知识分子觉得特别宝贵。党中央在两年多以前作过一个关于知识分子问题的决定\mnote{3},要争取广大的知识分子,只要他们是革命的,愿意参加抗日的,一概采取欢迎态度。我们尊重知识分子是完全应该的,没有革命知识分子,革命就不会胜利。但是我们晓得,有许多知识分子,他们自以为很有知识,大摆其知识架子,而不知道这种架子是不好的,是有害的,是阻碍他们前进的。他们应该知道一个真理,就是许多所谓知识分子,其实是比较地最无知识的,工农分子的知识有时倒比他们多一点。于是有人说:“哈!你弄颠倒了,乱说一顿。”(笑声)但是,同志,你别着急,我讲的多少有点道理。

什么是知识?自从有阶级的社会存在以来,世界上的知识只有两门,一门叫做生产斗争知识,一门叫做阶级斗争知识。自然科学、社会科学,就是这两门知识的结晶,哲学则是关于自然知识和社会知识的概括和总结。此外还有什么知识呢?没有了。我们现在看看一些学生,看看那些同社会实际活动完全脱离关系的学校里面出身的学生,他们的状况是怎么样呢?一个人从那样的小学一直读到那样的大学,毕业了,算有知识了。但是他有的只是书本上的知识,还没有参加任何实际活动,还没有把自己学得的知识应用到生活的任何部门里去。像这样的人是否可以算得一个完全的知识分子呢?我以为很难,因为他的知识还不完全。什么是比较完全的知识呢?一切比较完全的知识都是由两个阶段构成的:第一阶段是感性知识,第二阶段是理性知识,理性知识是感性知识的高级发展阶段。学生们的书本知识是什么知识呢?假定他们的知识都是真理,也是他们的前人总结生产斗争和阶级斗争的经验写成的理论,不是他们自己亲身得来的知识。他们接受这种知识是完全必要的,但是必须知道,就一定的情况说来,这种知识对于他们还是片面性的,这种知识是人家证明了,而在他们则还没有证明的。最重要的,是善于将这些知识应用到生活和实际中去。所以我劝那些只有书本知识但还没有接触实际的人,或者实际经验尚少的人,应该明白自己的缺点,将自己的态度放谦虚一些。

有什么办法使这种仅有书本知识的人变为名副其实的知识分子呢?唯一的办法就是使他们参加到实际工作中去,变为实际工作者,使从事理论工作的人去研究重要的实际问题。这样就可以达到目的。

我这样说,难免有些人要发脾气。他们说:“照你这样解释,那末,马克思也算不得知识分子了。”我说:不对。马克思不但参加了革命的实际运动,而且进行了革命的理论创造。他从资本主义最单纯的因素——商品开始,周密地研究了资本主义社会的经济结构。商品这个东西,千百万人,天天看它,用它,但是熟视无睹。只有马克思科学地研究了它,他从商品的实际发展中作了巨大的研究工作,从普遍的存在中找出完全科学的理论来。他研究了自然,研究了历史,研究了无产阶级革命,创造了辩证唯物论、历史唯物论和无产阶级革命的理论。这样,马克思就成了一个代表人类最高智慧的最完全的知识分子,他和那些仅有书本知识的人有根本的区别。马克思在实际斗争中进行了详细的调查研究,概括了各种东西,得到的结论又拿到实际斗争中去加以证明,这样的工作就叫做理论工作。我们党内需要许多同志学做这样的工作。我们党内现在有大批的同志,可以学习从事于这样的理论研究工作,他们大都是聪明有为的人,我们要看重他们。但是他们的方针要对,过去犯过的错误他们不应重复。他们必须抛弃教条主义,必须不停止在现成书本的字句上。

真正的理论在世界上只有一种,就是从客观实际抽出来又在客观实际中得到了证明的理论,没有任何别的东西可以称得起我们所讲的理论。斯大林曾经说过,脱离实际的理论是空洞的理论\mnote{4}。空洞的理论是没有用的,不正确的,应该抛弃的。对于好谈这种空洞理论的人,应该伸出一个指头向他刮脸皮。马克思列宁主义是从客观实际产生出来又在客观实际中获得了证明的最正确最科学最革命的真理;但是许多学习马克思列宁主义的人却把它看成是死的教条,这样就阻碍了理论的发展,害了自己,也害了同志。

另一方面,我们从事实际工作的同志,如果误用了他们的经验,也是要出毛病的。不错,这样的人往往经验很多,这是很可宝贵的;但是如果他们就以自己的经验为满足,那也很危险。他们须知自己的知识是偏于感性的或局部的,缺乏理性的知识和普遍的知识,就是说,缺乏理论,他们的知识也是比较地不完全。而要把革命事业做好,没有比较完全的知识是不行的。

这样看来,有两种不完全的知识,一种是现成书本上的知识,一种是偏于感性和局部的知识,这二者都有片面性。只有使二者互相结合,才会产生好的比较完全的知识。

但是,我们的工农干部要学理论,必须首先学文化。没有文化,马克思列宁主义的理论就学不进去。学好了文化,随时都可学习马克思列宁主义。我幼年没有进过马克思列宁主义的学校,学的是“子曰学而时习之,不亦说乎”\mnote{5}一套,这种学习的内容虽然陈旧了,但是对我也有好处,因为我识字便是从这里学来的。何况现在不是学的孔夫子,学的是新鲜的国语、历史、地理和自然常识,这些文化课学好了,到处有用。我们党中央现在着重要求工农干部学习文化,因为学了文化以后,政治、军事、经济哪一门都可学。否则工农干部虽有丰富经验,却没有学习理论的可能。

由此看来,我们反对主观主义,必须使上述两种人各向自己缺乏的方面发展,必须使两种人互相结合。有书本知识的人向实际方面发展,然后才可以不停止在书本上,才可以不犯教条主义的错误。有工作经验的人,要向理论方面学习,要认真读书,然后才可以使经验带上条理性、综合性,上升成为理论,然后才可以不把局部经验误认为即是普遍真理,才可不犯经验主义的错误。教条主义、经验主义,两者都是主观主义,是从不同的两极发生的东西。

所以,我们党内的主观主义有两种:一种是教条主义,一种是经验主义。他们都是只看到片面,没有看到全面。如果不注意,如果不知道这种片面性的缺点,并且力求改正,那就容易走上错误的道路。

但是在这两种主观主义中,现在在我们党内还是教条主义更为危险。因为教条主义容易装出马克思主义的面孔,吓唬工农干部,把他们俘虏起来,充作自己的用人,而工农干部不易识破他们;也可以吓唬天真烂漫的青年,把他们充当俘虏。我们如果把教条主义克服了,就可以使有书本知识的干部,愿意和有经验的干部相结合,愿意从事实际事物的研究,可以产生许多理论和经验结合的良好的工作者,可以产生一些真正的理论家。我们如果把教条主义克服了,就可以使有经验的同志得着良好的先生,使他们的经验上升成为理论,而避免经验主义的错误。

除了对于“理论家”和“知识分子”存在着糊涂观念而外,还有天天念的一句“理论和实际联系”,在许多同志中间也是一个糊涂观念。他们天天讲“联系”,实际上却是讲“隔离”,因为他们并不去联系。马克思列宁主义理论和中国革命实际,怎样互相联系呢?拿一句通俗的话来讲,就是“有的放矢”。“矢”就是箭,“的”就是靶,放箭要对准靶。马克思列宁主义和中国革命的关系,就是箭和靶的关系。有些同志却在那里“无的放矢”,乱放一通,这样的人就容易把革命弄坏。有些同志则仅仅把箭拿在手里搓来搓去,连声赞曰:“好箭!好箭!”却老是不愿意放出去。这样的人就是古董鉴赏家,几乎和革命不发生关系。马克思列宁主义之箭,必须用了去射中国革命之的。这个问题不讲明白,我们党的理论水平永远不会提高,中国革命也永远不会胜利。

我们的同志必须明白,我们学马克思列宁主义不是为着好看,也不是因为它有什么神秘,只是因为它是领导无产阶级革命事业走向胜利的科学。直到现在,还有不少的人,把马克思列宁主义书本上的某些个别字句看作现成的灵丹圣药,似乎只要得了它,就可以不费气力地包医百病。这是一种幼稚者的蒙昧,我们对这些人应该作启蒙运动。那些将马克思列宁主义当宗教教条看待的人,就是这种蒙昧无知的人。对于这种人,应该老实地对他说,你的教条一点什么用处也没有。马克思、恩格斯、列宁、斯大林曾经反复地讲,我们的学说不是教条而是行动的指南。这些人偏偏忘记这句最重要最重要的话。中国共产党人只有在他们善于应用马克思列宁主义的立场、观点和方法,善于应用列宁斯大林关于中国革命的学说,进一步地从中国的历史实际和革命实际的认真研究中,在各方面作出合乎中国需要的理论性的创造,才叫做理论和实际相联系。如果只是口头上讲联系,行动上又不实行联系,那末,讲一百年也还是无益的。我们反对主观地片面地看问题,必须攻破教条主义的主观性和片面性。

关于反对主观主义以整顿全党的学风的问题,今天讲的就是这些。

现在我来讲一讲宗派主义的问题。

由于二十年的锻炼,现在我们党内并没有占统治地位的宗派主义了。但是宗派主义的残余是还存在的,有对党内的宗派主义残余,也有对党外的宗派主义残余。对内的宗派主义倾向产生排内性,妨碍党内的统一和团结;对外的宗派主义倾向产生排外性,妨碍党团结全国人民的事业。铲除这两方面的祸根,才能使党在团结全党同志和团结全国人民的伟大事业中畅行无阻。

什么是党内宗派主义的残余呢?主要的有下面几种:

首先就是闹独立性。一部分同志,只看见局部利益,不看见全体利益,他们总是不适当地特别强调他们自己所管的局部工作,总希望使全体利益去服从他们的局部利益。他们不懂得党的民主集中制,他们不知道共产党不但要民主,尤其要集中。他们忘记了少数服从多数,下级服从上级,局部服从全体,全党服从中央的民主集中制。张国焘\mnote{6}是向党中央闹独立性的,结果闹到叛党,做特务去了。现在讲的,虽然不是这种极端严重的宗派主义,但是这种现象必须预防,必须将各种不统一的现象完全除去。要提倡顾全大局。每一个党员,每一种局部工作,每一项言论或行动,都必须以全党利益为出发点,绝对不许可违反这个原则。

闹这类独立性的人,常常跟他们的个人第一主义分不开,他们在个人和党的关系问题上,往往是不正确的。他们在口头上虽然也说尊重党,但他们在实际上却把个人放在第一位,把党放在第二位。刘少奇同志曾经说过,有一种人的手特别长,很会替自己个人打算,至于别人的利益和全党的利益,那是不大关心的。“我的就是我的,你的还是我的”。(大笑)这种人闹什么东西呢?闹名誉,闹地位,闹出风头。在他们掌管一部分事业的时候,就要闹独立性。为了这些,就要拉拢一些人,排挤一些人,在同志中吹吹拍拍,拉拉扯扯,把资产阶级政党的庸俗作风也搬进共产党里来了。这种人的吃亏在于不老实。我想,我们应该是老老实实地办事;在世界上要办成几件事,没有老实态度是根本不行的。什么人是老实人?马克思、恩格斯、列宁、斯大林是老实人,科学家是老实人。什么人是不老实的人?托洛茨基、布哈林、陈独秀、张国焘是大不老实的人,为个人利益为局部利益闹独立性的人也是不老实的人。一切狡猾的人,不照科学态度办事的人,自以为得计,自以为很聪明,其实都是最蠢的,都是没有好结果的。我们党校的学生一定要注意这个问题。我们一定要建设一个集中的统一的党,一切无原则的派别斗争,都要清除干净。要使我们全党的步调整齐一致,为一个共同目标而奋斗,我们一定要反对个人主义和宗派主义。

外来干部和本地干部必须团结,必须反对宗派主义倾向。因为许多抗日根据地是八路军新四军到后才创立的,许多地方工作是外来干部去后才发展的,外来干部和本地干部的关系,必须加以很好的注意。我们的同志必须懂得,在这种条件下,只有外来干部和本地干部完全团结一致,只有本地干部大批地生长了,并提拔起来了,根据地才能巩固,我党在根据地内才能生根,否则是不可能的。外来干部和本地干部各有长处,也各有短处,必须互相取长补短,才能有进步。外来干部比较本地干部,对于熟悉情况和联系群众这些方面,总要差些。拿我来说,就是这样。我到陕北已经五六年了,可是对于陕北的情况的了解,对于和陕北人民的联系,和一些陕北同志比较起来就差得多。我们到山西、河北、山东以及其它抗日根据地的同志,一定要注意这个问题。不但如此,即在一个根据地内部,因为根据地内的各个区域有发展先后之不同,干部中也有外来本地之别。比较先进区域的干部到比较落后的区域去,对于当地,也是一种外来干部,也要十分注意扶助本地干部的问题。就一般情形说来,凡属外来干部负领导责任的地方,如果和本地干部的关系弄得不好,那末,这个责任主要地应该放在外来干部的身上。担负主要领导责任的同志,其责任就更大些。现在各地对这个问题的注意还很不够,有些人轻视本地干部,讥笑本地干部,他们说:“本地人懂得什么,土包子!”这种人完全不懂得本地干部的重要性,他们既不了解本地干部的长处,也不了解自己的短处,采取了不正确的宗派主义的态度。一切外来干部一定要爱护本地干部,经常帮助他们,而不许可讥笑他们,打击他们。自然,本地干部也必须学习外来干部的长处,必须去掉那些不适当的狭隘的观点,以求和外来干部完全不分彼此,打成一片,而避免宗派主义倾向。

军队工作干部和地方工作干部的关系也是如此。两者必须完全团结一致,必须反对宗派主义的倾向。军队干部必须帮助地方干部,地方干部也必须帮助军队干部。如有纠纷,应该双方互相原谅,而各对自己作正确的自我批评。在军队干部事实上居于领导地位的地方,在一般的情形之下,如果和地方干部的关系弄不好,那末,主要的责任应该放在军队干部的身上。必须使军队干部首先懂得自己的责任,以谦虚的态度对待地方干部,才能使根据地的战争工作和建设工作得到顺利进行的条件。

几部分军队之间、几个地方之间、几个工作部门之间的关系,也是如此。必须反对只顾自己不顾别人的本位主义的倾向。谁要是对别人的困难不管,别人要调他所属的干部不给,或以坏的送人,“以邻为壑”,全不为别部、别地、别人想一想,这样的人就叫做本位主义者,这就是完全失掉了共产主义的精神。不顾大局,对别部、别地、别人漠不关心,就是这种本位主义者的特点。对于这样的人,必须加重教育,使他们懂得这就是一种宗派主义的倾向,如果发展下去,是很危险的。

还有一个问题,就是老干部和新干部的关系问题。抗战以来,我党有广大的发展,大批新干部产生了,这是很好的现象。斯大林同志在联共十八次代表大会上的报告中说:“老干部通常总是不多,比所需要的数量少,而且由于宇宙自然法则的关系,他们已部分地开始衰老死亡下去。”\mnote{7}他在这里讲了干部状况,又讲了自然科学。我们党如果没有广大的新干部同老干部一致合作,我们的事业就会中断。所以一切老干部应该以极大的热忱欢迎新干部,关心新干部。不错,新干部是有缺点的,他们参加革命还不久,还缺乏经验,他们中的有些人还不免带来旧社会不良思想的尾巴,这就是小资产阶级个人主义思想的残余。但是这些缺点是可以从教育中从革命锻炼中逐渐地去掉的。他们的长处,正如斯大林说过的,是对于新鲜事物有锐敏的感觉,因而有高度的热情和积极性,而在这一点上,有些老干部则正是缺乏的\mnote{8}。新老干部应该是彼此尊重,互相学习,取长补短,以便团结一致,进行共同的事业,而防止宗派主义的倾向。在老干部负主要领导责任的地方,在一般情形之下,如果老干部和新干部的关系弄得不好,那末,老干部就应该负主要的责任。

以上所讲的局部和全体的关系,个人和党的关系,外来干部和本地干部的关系,军队干部和地方干部的关系,军队和军队、地方和地方、这一工作部门和那一工作部门之间的关系,老干部和新干部的关系,都是党内的相互关系。在这种种方面,都应该提高共产主义精神,防止宗派主义倾向,使我们的党达到队伍整齐,步调一致的目的,以利战斗。这是一个很重要的问题,我们整顿党的作风,必须彻底地解决这个问题。宗派主义是主观主义在组织关系上的一种表现;我们如果不要主观主义,要发展马克思列宁主义实事求是的精神,就必须扫除党内宗派主义的残余,以党的利益高于个人和局部的利益为出发点,使党达到完全团结统一的地步。

宗派主义的残余,在党内关系上是应该消灭的,在党外关系上也是应该消灭的。其理由就是:单是团结全党同志还不能战胜敌人,必须团结全国人民才能战胜敌人。中国共产党在团结全国人民的事业上,二十年来做了艰苦的伟大的工作;抗战以来,这个工作的成绩更加伟大。但这并不是说,我们所有的同志对待人民群众都有了正确的作风,都没有了宗派主义的倾向。不是的。在一部分同志中,确实还有宗派主义的倾向,有些人并且很严重。我们的许多同志,喜欢对党外人员妄自尊大,看人家不起,藐视人家,而不愿尊重人家,不愿了解人家的长处。这就是宗派主义的倾向。这些同志,读了几本马克思主义的书籍之后,不是更谦虚,而是更骄傲了,总是说人家不行,而不知自己实在是一知半解。我们的同志必须懂得一条真理:共产党员和党外人员相比较,无论何时都是占少数。假定一百个人中有一个共产党员,全中国四亿五千万人中就有四百五十万共产党员。即使达到这样大的数目,共产党员也还是只占百分之一,百分之九十九都是非党员。我们有什么理由不和非党人员合作呢?对于一切愿意同我们合作以及可能同我们合作的人,我们只有同他们合作的义务,绝无排斥他们的权利。一部分党员却不懂得这个道理,看不起愿意同我们合作的人,甚至排斥他们。这是没有任何根据的。马克思、恩格斯、列宁、斯大林给了我们这样的根据吗?没有。相反地,他们总是谆谆告诫我们,要密切联系群众,而不要脱离群众。中国共产党中央给了我们这个根据吗?没有。中央的一切决议案中,没有一个决议说是我们可以脱离群众使自己孤立起来。相反地,中央总是叫我们密切联系群众,而不要脱离群众。所以,一切脱离群众的行为,并没有任何的根据,只是我们一部分同志自己造出来的宗派主义思想在那里作怪。因为这种宗派主义在一部分同志中还很严重,还在障碍党的路线的实行,所以我们要针对这个问题在党内进行广大的教育。首先要使我们的干部真正懂得这个问题的严重性,使他们懂得共产党员如果不同党外干部、党外人员互相联合,敌人就一定不能打倒,革命的目的就一定不能达到。

一切宗派主义思想都是主观主义的,都和革命的实际需要不相符合,所以反对宗派主义和反对主观主义的斗争,应当同时并进。

关于党八股的问题,今天不能讲了,准备在另外一个会议上来讨论。党八股是藏垢纳污的东西,是主观主义和宗派主义的一种表现形式。它是害人的,不利于革命的,我们必须肃清它。

我们要反对主观主义,就要宣传唯物主义,就要宣传辩证法。但是我们党内还有许多同志,他们并不注重宣传唯物主义,也不注重宣传辩证法。有些同志听凭别人宣传主观主义,也安之若素。这些同志自以为相信马克思主义,但是,他们却不努力宣传唯物主义,听了或看了主观主义的东西也不想一想,也不发议论。这种态度不是共产党员的态度。这使得我们许多同志蒙受了主观主义思想的毒害,发生麻木的现象。所以我们要在党内发动一个启蒙运动,使我们同志的精神从主观主义、教条主义的蒙蔽中间解放出来,号召同志们对于主观主义、宗派主义、党八股加以抵制。这些东西好像日货,因为只有我们的敌人愿意我们保存这些坏东西,使我们继续受蒙蔽,所以我们应该提倡抵制,就像抵制日货\mnote{9}一样。一切主观主义、宗派主义、党八股的货色,我们都要抵制,使它们在市场上销售困难,不要让它们利用党内理论水平低,出卖自己那一套。为此目的,就要同志们提高嗅觉,就要同志们对于任何东西都用鼻子嗅一嗅,鉴别其好坏,然后才决定欢迎它,或者抵制它。共产党员对任何事情都要问一个为什么,都要经过自己头脑的周密思考,想一想它是否合乎实际,是否真有道理,绝对不应盲从,绝对不应提倡奴隶主义。

最后,我们反对主观主义、宗派主义、党八股,有两条宗旨是必须注意的:第一是“惩前毖后”,第二是“治病救人”。对以前的错误一定要揭发,不讲情面,要以科学的态度来分析批判过去的坏东西,以便使后来的工作慎重些,做得好些。这就是“惩前毖后”的意思。但是我们揭发错误、批判缺点的目的,好像医生治病一样,完全是为了救人,而不是为了把人整死。一个人发了阑尾炎,医生把阑尾割了,这个人就救出来了。任何犯错误的人,只要他不讳疾忌医,不固执错误,以至于达到不可救药的地步,而是老老实实,真正愿意医治,愿意改正,我们就要欢迎他,把他的毛病治好,使他变为一个好同志。这个工作决不是痛快一时,乱打一顿,所能奏效的。对待思想上的毛病和政治上的毛病,决不能采用鲁莽的态度,必须采用“治病救人”的态度,才是正确有效的方法。

趁着今天党校开学的机会,我讲了这许多话,希望同志们加以考虑。(热烈的鼓掌)


\begin{maonote}
\mnitem{1}八股文是中国明、清封建皇朝考试制度所规定的一种特殊文体。它内容空洞,专讲形式,玩弄文字。这种文章的每一个段落都要死守在固定的格式里面,连字数都有一定的限制,人们只是按照题目的字义敷衍成文。党八股是指在革命队伍中某些人在写文章、发表演说或者做其它宣传工作的时候,对事物不加分析,只是搬用一些革命的名词和术语,言之无物,空话连篇,也和上述的八股文一样。
\mnitem{2}指一九四一年八月一日《中央关于调查研究的决定》。这个决定要求全党采取具体措施,收集国内外政治、军事、经济、文化及社会阶级关系各方面的材料,加强对于历史,对于环境,对于国内外、省内外、县内外具体情况的调查研究,并将这种调查研究、了解情况的工作,同学习马克思列宁主义理论密切联系起来。
\mnitem{3}指一九三九年十二月一日中共中央关于吸收知识分子的决定,即本书第二卷\mxart{大量吸收知识分子}。
\mnitem{4}见本书第一卷\mxnote{实践论}{10}。
\mnitem{5}这是孔子和他的弟子们的语录《论语》的开头一句话。
\mnitem{6}见本书第一卷\mxnote{论反对日本帝国主义的策略}{24}。
\mnitem{7}这段话的新译文是:“老干部总是少数,不能满足需要,而且由于自然界的天然规律,他们已经部分地开始丧失工作能力。”(《斯大林选集》下卷,人民出版社1979年版,第460页)
\mnitem{8}参见斯大林《在党的第十八次代表大会上关于联共(布)中央工作的总结报告》第三部分第二节(《斯大林选集》下卷,人民出版社1979年版,第460页)。
\mnitem{9}抵制日货是中国人民在二十世纪上半叶所常常采取的反抗日本帝国主义侵略的一种斗争方法。例如,在一九一九年五四爱国运动时期,一九三一年九一八事变之后,中国人民都曾经进行过抵制日货的运动。
\end{maonote}
