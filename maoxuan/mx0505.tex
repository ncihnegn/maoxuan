
\title{征询对待富农策略问题的意见}
\date{一九五〇年三月十二日}
\thanks{这是毛泽东同志给中共中央中南局并华东局、华南分局、西南局、西北局的通知。}
\maketitle


请就你们现正召开的各省负责同志会议中征询关于对待富农策略问题的意见电告我们,即是说在今冬开始的南方几省及西北某些地区的土地改革运动中,不但不动资本主义富农,而且不动半封建富农,待到几年之后再去解决半封建富农问题。请你们考虑这样做是否有利些。这样做的理由:第一是土改规模空前伟大,容易发生过左偏向,如果我们只动地主不动富农,则更能孤立地主,保护中农,并防止乱打乱杀,否则很难防止;第二是过去北方土改是在战争中进行的,战争空气掩盖了土改空气,现在基本上已无战争,土改就显得特别突出,给予社会的震动特别显得重大,地主叫唤的声音将特别显得尖锐,如果我们暂时不动半封建富农,待到几年之后再去动他们,则将显得我们更加有理由,即是说更加有政治上的主动权;第三是我们和民族资产阶级的统一战线,现在已经在政治上、经济上和组织上都形成了,而民族资产阶级是与土地问题密切联系的,为了稳定民族资产阶级起见,暂时不动半封建富农似较妥当的。

关于暂时不动富农的问题,去年十一月政治局会议中,我曾提出过,惟未作详细的分析和未作出决定,现在已到需要作决定的时机了。决定之后,需要修改土地法及其它有关土改的文件,并颁布出去,以利新区各省土改干部的学习,方有利于今年秋后开始土改,否则将错过时机,陷于被动。因此,不但请中南局,而且请华东局,华南分局,西南局,西北局的同志们对此问题加以讨论,并请将此电转发所属各省省委各市市委加以讨论,将赞成和反对的意见收集起来迅速电告中央,以凭考虑决策,是为至要。
