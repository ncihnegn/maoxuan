
\title{国民党反动派由“呼吁和平”变为呼吁战争}
\date{一九四九年二月十六日}
\maketitle


自从一月一日蒋匪介石发动和平攻势以后,曾经连篇累牍地表示自己是愿意“缩短战争时间”,“减轻人民痛苦”,“以拯救人民为前提”的国民党反动派的英雄好汉们,一到二月上旬,和平的调子就突然低落下去,“和共党周旋到底”的老调忽又高弹起来。最近数日,更是如此。二月十三日,国民党中央宣传部发给“各党部各党报”的《特别宣传指示》上说:“叶剑英向我后方宣传中共对和平有诚意,而指责政府军事布置为无诚意谋和。各报对此,必须依据下列各点从正面与侧面力加驳斥。”这个《特别宣传指示》一连列举了好几点应当“驳斥”的理由。“政府与其无条件投降,不如作战到底。”“毛泽东一月十四日声明所提八点为亡国条件,政府原不应接受。”“中共应负破坏和平之责任。今日中共反而提出所谓战犯名单,将政府负责人士尽皆列入,更要求政府先行逮捕,其蛮横无理,显而易见。中共如不改变此种作风,则和平商谈之途径,势难寻觅。”两星期以前那种如丧考妣地急着要谈判的神气,再也不见了。所谓“缩短战争时间”,“减轻人民痛苦”,“以拯救人民为前提”这些传遍人间、沁人心脾的名句,再也不提了。假如中共不愿意改变自己的“作风”,一定要惩办战争罪犯,那就不能谈和平了。究竟是以拯救人民为前提呢,还是以拯救战争罪犯为前提呢?按照国民党英雄好汉的《特别宣传指示》,是选择了后者。战争罪犯的名单,中共方面尚在向各民主党派人民团体征求意见中,现在已经收到了好几方面的意见。根据这些已经收到的意见,都是不赞成去年十二月二十五日中共权威人士所提的那个名单。他们认为那个名单所列战犯只有四十三个,为数太少;他们认为要负发动反革命战争屠杀数百万人民的责任的人决不止四十三个,而应当是一百几十个。现在姑且假定战犯将确定为一百几十个。那末,请问国民党的英雄好汉们,你们为什么要反对惩办战犯呢?你们不是愿意“缩短战争时间”“减轻人民痛苦”的吗?假如因为你们这一反对,使得战争还要打下去,岂非拖延时间,延长战祸?“拖延时间,延长战祸”这八个字的罪名是你们在一九四九年一月二十六日以南京政府发言人的名义发出声明,加在共产党身上的,现在难道你们想收回去,写上招贴,挂在你们自己身上,以为荣耀吗?你们是“以拯救人民为前提”的大慈大悲的人们,为什么一下子又改成以拯救战犯为前提了呢?根据你们政府内政部的统计,中国人民的数目,不是四亿五千万,而是四亿七千五百万,这和一百几十个战犯相比,究竟大小如何呢?英雄们是学过算术的,请你们按照算术教科书好好地算一下再作结论吧。倘若你们不去算清楚就将你们那个原来很好、我们也同意、全国人民也同意的提法——“以拯救人民为前提”,急急忙忙地改成“以拯救一百几十个战犯为前提”,那你们可要仔细,你们就一定站不住脚。这些口口声声“以拯救人民为前提”的人们,在自己“呼吁和平”几个星期之后,又不再是“呼吁和平”,而是呼吁战争了。国民党死硬派就是这样倒霉的,他们坚决地反对人民,站在人民的头上横行霸道,因而把自己孤立在宝塔的尖顶上,而且至死也不悔悟。长江流域和南方的人民大众,包括工人,农民,知识分子,城市小资产阶级,民族资产阶级,开明绅士,有良心的国民党人,都请听着:站在你们头上横行霸道的国民党死硬派,没有几天活命的时间了,我们和你们是站在一个方面的,一小撮死硬派不要几天就会从宝塔尖上跌下去,一个人民的中国就要出现了。
