
\title{支持美国黑人抗暴斗争的声明}
\date{一九六八年四月十六日}
\maketitle


最近,美国黑人牧师马丁·路德·金\mnote{1}突然被美帝国主义者暗杀。马丁·路德·金是一个非暴力主义者,但美帝国主义者并没有因此对他宽容,而是使用反革命的暴力,对他进行血腥的镇压。这一件事,深刻地教训了美国的广大黑人群众,激起了他们抗暴斗争的新风暴,席卷了美国一百几十个城市,是美国历史上前所未有的。它显示了在两千多万美国黑人中,蕴藏着极其强大的革命力量。

这场黑人的斗争风暴发生在美国国内,是美帝国主义当前整个政治危机和经济危机的一个突出表现。它给陷于内外交困的美帝国主义以沉重的打击。

美国黑人的斗争,不仅是被剥削、被压迫的黑人争取自由解放的斗争,而且是整个被剥削、被压迫的美国人民反对垄断资产阶级残暴统治的新号角。它对于全世界人民反对美帝国主义的斗争,对于越南人民反对美帝国主义的斗争,是一个巨大的支援和鼓舞。我代表中国人民,对美国黑人的正义斗争,表示坚决的支持。

美国的种族歧视,是殖民主义、帝国主义制度的产物。美国广大黑人同美国统治集团之间的矛盾,是阶级矛盾。只有推翻美国垄断资产阶级的反动统治,摧毁殖民主义、帝国主义制度,美国黑人才能够取得彻底解放。美国广大黑人同美国白人中的广大劳动人民,有着共同的利益和共同的斗争目标。因此,美国黑人的斗争正在获得越来越多的美国白色人种中的劳动人民和进步人士的同情和支持。美国黑人斗争必将同美国工人运动相结合,最终结束美国垄断资产阶级的罪恶统治。

我在一九六三年《支持美国黑人反对美帝国主义种族歧视的正义斗争的声明》中说过:“万恶的殖民主义、帝国主义制度是随着奴役和贩卖黑人而兴盛起来的,它也必将随着黑色人种的彻底解放而告终。”我现在仍然坚持这个观点。

当前,世界革命进入了一个伟大的新时代。美国黑人争取解放的斗争,是全世界人民反对美帝国主义的总斗争的一个组成部分,是当代世界革命的一个组成部分。我呼吁:世界各国的工人、农民、革命知识分子和一切愿意反对美帝国主义的人们,行动起来,给予美国黑人的斗争以强大的声援!全世界人民更紧密地团结起来,向着我们的共同敌人美帝国主义及其帮凶们发动持久的猛烈的进攻!可以肯定,殖民主义、帝国主义和一切剥削制度的彻底崩溃,世界上一切被压迫人民、被压迫民族的彻底翻身,已经为期不远了。

\begin{maonote}
\mnitem{1}美国著名的黑人民权领袖马丁·路德·金(Martin Luther King, Jr.),一九二九年一月十五日生于美国亚特兰大,一九五四年毕业于波士顿大学。毕业后,他在蒙哥马利城任牧师,并开始参加美国有色人种协进会的活动。

一九五六,马丁·路德·金领导蒙哥马利城黑人成功抵制了当地公共汽车歧视黑人的行为。当时全城五万名黑人都拒绝乘坐公共汽车,斗争坚持了三百八十五天,最终使美国最高法院宣布在交通工具上实施种族隔离为非法。一九五七年,他参与建立了黑人牧师组织——南方基督教领袖大会,并担任首任主席。

一九六三年八月,马丁·路德·金组织了美国历史上影响深远的“自由进军”运动,率领二十多万名黑人向首都华盛顿进军,为全美国的黑人争取人权。八月二十八日,他在林肯纪念堂前发表了著名演说《我有一个梦想》,发出反对种族歧视、争取平等的正义呼声。迫于巨大的压力,美国总统约翰逊于一九六四年签署了民权法案。同年,马丁·路德·金获得诺贝尔和平奖。

一九六八年四月四日,马丁·路德·金在支持田纳西州孟菲斯市清洁工人罢工斗争中被种族主义分子暗杀,年仅三十九岁。他的遇害引发了美国历史上前所未有的黑人抗暴斗争浪潮,席卷全美一百二十五个城市。
\end{maonote}
