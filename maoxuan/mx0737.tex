
\title{接见刚果(布)\mnote{1}总理努马扎莱的谈话}
\date{一九六七年十月三日}
\maketitle


\mxsay{努马:}我看到中国无产阶级文化大革命的伟大胜利,使中国人民政治觉悟大大提高了。

\mxsay{毛泽东:}无政府主义也大大发展了。

\mxsay{努马:}也许是这样,但是我们还没有看到这些。

\mxsay{毛泽东:}有那么个思潮暴露出来好教育。

\mxsay{努马:}你们的干部很谦虚。

\mxsay{毛泽东:}非谦虚不可,否则群众斗他们。

\mxsay{努马:}你们的干部与我们的干部有很大的区别。

\mxsay{毛泽东:}没有多大区别。都是官大了,薪水多了,坐小汽车了。大官还得有人做,大官没人做还得了!薪水多一点,房子好一点,坐汽车也可以,但不要摆架子,和工农群众平等相待。不要动不动就训人、骂人。有的大队书记,薪水不多,房子不好,没坐小汽车,官也不大,就是官架子不小。运动一开始,结果把他们吓了一跳。

\mxsay{努马:}外国讲中国很乱,我们怎么没看到?

\mxsay{毛泽东:}乱一点,你们可以到处走走,乱了以后就不乱了。不闹够就不行。这时候差不多了。我们准备再乱一年。

\mxsay{努马:}什么叫越乱越好呢?

\mxsay{毛泽东:}不乱胜负不分,湖南、湖北、江西、安徽、浙江,除安徽省外都好。到中国得一条经验。湖南煤矿动刀动枪了。生产几万吨下降到几吨,现在已产二万吨。

\mxsay{努马:}这样的矛盾,怎么解决得这么好呢?

\mxsay{毛泽东:}后台揪出来了,群众打够了,这时中央讲几句话就行了。有人吹中国怎么好,不要听那一套,非洲人架子小,所以我们希望你们来。欧洲、亚洲就不行。我只要求你(指干部)一条,要把官僚架子放下,跟老百姓、工人、农民、学生、战士、下级一起,平等待人。不要动不动就训人。有道理为什么要训人,可以解释嘛,有道理为什么要骂人?这样不行,老百姓不同意,也要批评你,当然不会因为这些打倒你。这次一年多的一个大批判运动,可把这些干部吓一跳。

\mxsay{努马:}我们也开始反官架子。

\mxsay{毛泽东:}我不建议你们也搞文化大革命。我们建军四十周年,建国十八年,打了二十二年,拥有打了几十年仗的解放军,所以搞文化大革命。

\mxsay{努马:}我们不搞文化大革命,但我们要研究文化大革命的理论和世界意义。

\mxsay{毛泽东:}这次文化大革命要改变国家部分机构,包括军队。恩克鲁玛\mnote{2}那次来,没有料到推翻他政权的就是他的军队。我看你还是早点回去。

\mxsay{努马:}我们家里还有人,但我尽量早点回去。

\begin{maonote}
\mnitem{1}一八八四年帝国主义瓜分非洲的柏林会议将刚果河以东地区划为比属殖民地,即今扎伊尔,以西地区划为法属殖民地,即现在的刚果,首都布拉柴维尔。一九六〇年八月十五日,刚果获得完全独立,定国名为刚果共和国。一九六四年二月二十二日,中刚两国建交。
\mnitem{2}恩克鲁玛,加纳独立运动领袖,一九五七年三月六日黄金海岸宣布独立,定国名为加纳,他被选为总理。一九六〇年七月一日宣布成立加纳共和国,他当选总统。一九六六年二月恩克鲁玛访问中国时,恰逢国内政变,他领导的政府被推翻,后来流亡并定居几内亚。
\end{maonote}
