
\title{《中国工人》发刊词}
\date{一九四〇年二月七日}
\maketitle


《中国工人》\mnote{1}的出版是必要的。中国工人阶级,二十年来,在自己的政党——中国共产党的领导之下展开了英勇的斗争,成了全国人民中最有觉悟的部分,成了中国革命的领导者。中国工人阶级联合农民和一切革命的人民反对帝国主义和封建主义,为建立新民主主义的中国而斗争,为驱逐日本帝国主义而斗争,这个功劳是非常之大的。但是中国革命尚未成功,还须付出很大的气力,团结自己,团结农民和其它小资产阶级,团结知识分子,团结一切革命的人民。这是极大的政治任务和组织任务。这是中国共产党的责任,这是工人阶级先进分子的责任,这是整个工人阶级的责任。工人阶级和全体人民的最后解放,只能在社会主义实现的时代,中国工人阶级必须为此最后目的而奋斗。但是必须经过反帝反封建的民主革命的阶段,才能进到社会主义的阶段。所以,团结自己和团结人民,反对帝国主义和封建主义,为建立新民主主义的新中国而奋斗,这就是中国工人阶级的当前的任务。《中国工人》的出版,就是为了这一个任务。

《中国工人》将以通俗的言语解释许多道理给工人群众听,报道工人阶级抗日斗争的实际,总结其经验,为完成自己的任务而努力。

《中国工人》应该成为教育工人、训练工人干部的学校,读《中国工人》的人就是这个学校的学生。工人中间应该教育出大批的干部,他们应该有知识,有能力,不务空名,会干实事。没有一大批这样的干部,工人阶级要求得解放是不可能的。

工人阶级应欢迎革命的知识分子帮助自己,决不可拒绝他们的帮助。因为没有他们的帮助,自己就不能进步,革命也不能成功。

我希望这个报纸好好地办下去,多载些生动的文字,切忌死板、老套,令人看不懂,没味道,不起劲。

一个报纸既已办起来,就要当作一件事办,一定要把它办好。这不但是办的人的责任,也是看的人的责任。看的人提出意见,写短信短文寄去,表示欢喜什么,不欢喜什么,这是很重要的,这样才能使这个报办得好。

以上,是我的希望,就当作发刊词。


\begin{maonote}
\mnitem{1}《中国工人》月刊,由中共中央职工运动委员会主办,一九四〇年二月在延安创刊,一九四一年三月终刊。
\end{maonote}
