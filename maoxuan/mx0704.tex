
\title{学校一律要实行半工半读}
\date{一九六六年四月十四日}
\thanks{这是毛泽东同志对《在京艺术院校试行半工(农)半读》\mnote{1}一文的批语。}
\maketitle


一切学校和学科(小学、中学、大学、军事学校、医学院校、文艺院校以及其他学校例如党校、新闻学校、外语学校、外交学校等等,学科包括社会科学、自然科学及二者的常识)都应当这样办。分步骤地有准备地一律下楼出院,到工厂去,到农村去,同工人农民同吃同住同劳动,学工学农,读书。工读比例最好一半对一半,最多是四比六。因此读书的部分要大减。书是要读的,但读多了是害死人的。师生一律平等,放下架子,教学相长。随时总结经验,纠正错误。

许多无用的书,只应束之高阁。就像过去废止读五经四书、读二十四史、读诸子百家、读无穷的文集和选集一样,革命反而胜利了。譬如共产党人和我们的军事干部,一字不识和稍识几字的占了百分之九十几,而多识一些字的,例如读过三几年中学,进过黄埔军校、云南讲武堂、苏联军事院校的,只有极少数,大学毕业生几乎一个也没有。所以有人说,共产党“无学有术”,而他则是“有学无术”。这话从形式上看来是有些对的。但从实质上看,则是完全错误。共产党人曾经进过二十几年的军事大学和革命大学(即二十几年的战争与革命),而那些大学教授和大学生们只会啃书本(这是一项比较最容易的工作),他们一不会打仗,二不会革命,三不会做工,四不会耕田。他们的知识贫乏得很,讲起这些来,一窍不通。他们中的很多人确有一项学问,就是反共反人民反革命,至今还是如此。他们也有“术”,就是反革命的方法。

所以我常说,知识分子和工农分子比较起来是最没有学问的人。他们不自惭形秽,整天从书本到书本,从概念到概念。如此下去,除了干反革命、搞资产阶级复辟、培养修正主义分子以外,其他一样也不会。一些从事过一二次“四清”运动\mnote{2},从工人农民那里取了经回来的人,他们自愧不如,有了革命干劲,这就好了。唐人诗云:“竹帛烟销帝业虚,山河空锁祖龙居。坑灰未烬山东乱,刘项原来不读书。”\mnote{3}有同志说:“学问少的打倒学问多的,年纪小的打倒年纪大的”,这是古今一条规律。经、史、子、集成了汗牛充栋、浩如烟海的状况,就宣告它自己的灭亡。只有几十万分之一的人还去理它,其他的人根本不知道有那回事,这是一大解放,不胜谢天谢地之至。

因此学校一律要搬到工厂和农村去,一律实行半工半读,当然要分步骤,要分批分期,但是一定要去,不去就解散这类学校,以免贻患无穷。

\begin{maonote}
\mnitem{1}这个批语写在中共中央办公厅机要室一九六六年四月十二日编印的《文电摘要》第一六八号上。这期摘要登载的《在京艺术院校试行半工(农)半读》一文,介绍了中国音乐学院抽调一年级学生和部分教师分别到中国汽车工业公司北京分公司试行在工厂办学、到北京市海淀区温泉公社试行半农半读的情况。毛泽东的批语当时没有印发。
\mnitem{2}“四清”运动,指一九六三年至一九六六年先后在部分农村和少数城市工矿企业、学校等单位开展的以清政治、清经济、清组织、清思想为主要内容的社会主义教育运动。
\mnitem{3}这是晚唐诗人章碣写的七绝《焚书坑》。原诗是:“竹帛烟销帝业虚,关河空锁祖龙居。坑灰未冷山东乱,刘项原来不读书。”这里指不读书的刘邦和项羽就能打下天下。
\end{maonote}
