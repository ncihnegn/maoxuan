
\title{关于帝国主义和一切反动派是不是真老虎的问题}
\date{一九五八年十二月一日}
\thanks{这是毛泽东同志在武昌举行的中共八届六中全会期间写的一篇文章。}
\maketitle


这里我想回答帝国主义及一切反动派是不是真老虎的问题。我的回答是,既是真的,又是纸的,这是一个由真变纸的过程的问题。变即转化,真老虎转化为纸老虎,走向反面。一切事物都是如此,不独社会现象而已。我在几年前已经回答了这个问题,战略上藐视它,战术上重视它。不是真老虎,为什么要重视它呢?看来还有一些人不通,我们还得做些解释工作。

同世界上一切事物无不具有两重性(即对立统一规律)一样,帝国主义和一切反动派也有两重性,它们是真老虎又是纸老虎。历史上奴隶主阶级、封建地主阶级和资产阶级,在它们取得统治权力以前和取得统治权力以后的一段时间内,它们是生气勃勃的,是革命者,是先进者,是真老虎。在随后的一段时间,由于它们的对立面,奴隶阶级、农民阶级和无产阶级,逐步壮大,并同它们进行斗争,越来越厉害,它们就逐步向反面转化,化为反动派,化为落后的人们,化为纸老虎,终究被或者将被人民所推翻。反动的、落后的、腐朽的阶级,在面临人民的决死斗争的时候,也还有这样的两重性。一面,真老虎,吃人,成百万人成千万人地吃。人民斗争事业处在艰难困苦的时代,出现许多弯弯曲曲的道路。中国人民为了消灭帝国主义、封建主义和官僚资本主义在中国的统治,花了一百多年时间,死了大概几千万人之多,才取得一九四九年的胜利。你看,这不是活老虎,铁老虎,真老虎吗?但是,它们终究转化成了纸老虎,死老虎,豆腐老虎。这是历史的事实。人们难道没有看见听见过这些吗?真是成千成万!成千成万!所以,从本质上看,从长期上看,从战略上看,必须如实地把帝国主义和一切反动派,都看成纸老虎。从这点上,建立我们的战略思想。另一方面,它们又是活的铁的真的老虎,它们会吃人的。从这点上,建立我们的策略思想和战术思想。向阶级敌人作斗争是如此,向自然界作斗争也是如此。我们在一九五六年发表的十二年农业发展纲要四十条\mnote{1}和十二年科学发展纲要\mnote{2}。这些都是从马克思主义关于宇宙发展的两重性,关于事物发展的两重性,关于事物总是当作过程出现而任何一个过程无不包括两重性,这样一个基本观点,对立统一的观点,出发的。一方面,藐视它,轻而易举,不算数,不在乎,可以完成,能打胜仗。一方面,重视它,并非轻而易举,算数的,千万不可以掉以轻心,不经艰苦奋斗,不苦战,就不能胜利。怕与不怕,是一个对立统一法则。一点不怕,无忧无虑,真正单纯的乐神,从来没有。每一个人都是忧患与生俱来。学生们怕考试,儿童怕父母有偏爱,三灾八难,五痨七伤,发烧四十一度,以及“天有不测风云,人有旦夕祸福”之类,不可胜数。阶级斗争,向自然界的斗争,所遇到的困难,更不可胜数。但是,大多的人类,首先是无产阶级,首先是共产党人,除掉怕死鬼以及机会主义的先生们以外,总是将藐视一切,乐观主义,放在他们心目中的首位的。然后才是重视事物,重视每件工作,重视科学研究,分析事物的每一个矛盾侧面,钻进去,逐步地认识自然运动的法则和社会运动的法则。然后就有可能掌握这些法则,比较自由地运用这些法则,一个一个地解决人们面临的问题,处理矛盾,完成任务,使困难向顺利转化,使真老虎向纸老虎转化,使革命的初级阶段向高级阶段转化,使民主革命向社会主义革命转化,使社会主义的集体所有制向社会主义的全民所有制转化,使社会主义的全民所有制向共产主义的全民所有制转化,使年产几百万吨钢向年产几千万吨钢乃至几万万吨钢转化,使亩产一百多斤或者几百斤粮食向亩产几千斤或者甚至几万斤粮食转化。同志们,我们就是做这些转化工作的。同志们,可能性同现实性是两件东西,是统一性的两个对立面。虚假的可能性同现实的可能性又是两件东西,又是统一性的两个对立面。头脑要冷又要热,又是统一性的两个对立面。冲天干劲是热。科学分析是冷。在我国,在目前,有些人太热了一点。他们不想使自己的头脑有一段冷的时间,不愿意做分析,只爱热。同志们,这种态度是不利于做领导工作的,他们可能跌筋斗,这些人应当注意提醒一下自己的头脑。另有一些人爱冷不爱热。他们对一些事,看不惯,跟不上。对这些人,应当使他们的头脑慢慢热起来。

一九五八年十二月一日,在武昌

\begin{maonote}
\mnitem{1}指《一九五六年到一九六七年全国农业发展纲要(草案)》。
\mnitem{2}指《一九五六——一九六七年科学技术发展远景规划纲要(草案)》。这个草案是国务院根据中共中央关于迅速改变我国在经济上和科学文化上的落后状况的指示精神,从一九五六年四月开始,组织六百多名中国科学技术专家,并邀请二十多位苏联专家,经过半年的研究和讨论制订的。规划提出了国家建设所急需的五十七项重要科学技术任务和六百一十六个中心问题,并指出了各门类科学的发展方向。这个规划的实施,推动了我国科学技术事业的迅速发展。
\end{maonote}
