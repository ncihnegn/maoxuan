
\title{中国社会各阶级的分析}
\date{一九二五年十二月一日}
\thanks{毛泽东此文是为反对当时党内存在着的两种倾向而写的。当时党内的第一种倾向,以陈独秀为代表,只注意同国民党合作,忘记了农民,这是右倾机会主义。第二种倾向,以张国焘为代表,只注意工人运动,同样忘记了农民,这是“左”倾机会主义。这两种机会主义都感觉自己力量不足,而不知道到何处去寻找力量,到何处去取得广大的同盟军。毛泽东指出中国无产阶级的最广大和最忠实的同盟军是农民,这样就解决了中国革命中的最主要的同盟军问题。毛泽东并且预见到当时的民族资产阶级是一个动摇的阶级,他们在革命高涨时将要分化,其右翼将要跑到帝国主义方面去。一九二七年所发生的事变,证明了这一点。}
\maketitle


谁是我们的敌人?谁是我们的朋友?这个问题是革命的首要问题。中国过去一切革命斗争成效甚少,其基本原因就是因为不能团结真正的朋友,以攻击真正的敌人。革命党是群众的向导,在革命中未有革命党领错了路而革命不失败的。我们的革命要有不领错路和一定成功的把握,不可不注意团结我们的真正的朋友,以攻击我们的真正的敌人。我们要分辨真正的敌友,不可不将中国社会各阶级的经济地位及其对于革命的态度,作一个大概的分析。

中国社会各阶级的情况是怎样的呢?

地主阶级和买办阶级。在经济落后的半殖民地的中国,地主阶级和买办阶级完全是国际资产阶级的附庸,其生存和发展,是附属于帝国主义的。这些阶级代表中国最落后的和最反动的生产关系,阻碍中国生产力的发展。他们和中国革命的目的完全不兼容。特别是大地主阶级和大买办阶级,他们始终站在帝国主义一边,是极端的反革命派。其政治代表是国家主义派\mnote{1}和国民党右派。

中产阶级。这个阶级代表中国城乡资本主义的生产关系。中产阶级主要是指民族资产阶级,他们对于中国革命具有矛盾的态度:他们在受外资打击、军阀压迫感觉痛苦时,需要革命,赞成反帝国主义反军阀的革命运动;但是当着革命在国内有本国无产阶级的勇猛参加,在国外有国际无产阶级的积极援助,对于其欲达到大资产阶级地位的阶级的发展感觉到威胁时,他们又怀疑革命。其政治主张为实现民族资产阶级一阶级统治的国家。有一个自称为戴季陶\mnote{2}“真实信徒”的,在北京《晨报》\mnote{3}上发表议论说:“举起你的左手打倒帝国主义,举起你的右手打倒共产党。”这两句话,画出了这个阶级的矛盾惶遽状态。他们反对以阶级斗争学说解释国民党的民生主义,他们反对国民党联俄和容纳共产党\mnote{4}及左派分子。但是这个阶级的企图——实现民族资产阶级统治的国家,是完全行不通的,因为现在世界上的局面,是革命和反革命两大势力作最后斗争的局面。这两大势力竖起了两面大旗:一面是红色的革命的大旗,第三国际\mnote{5}高举着,号召全世界一切被压迫阶级集合于其旗帜之下;一面是白色的反革命的大旗,国际联盟\mnote{6}高举着,号召全世界一切反革命分子集合于其旗帜之下。那些中间阶级,必定很快地分化,或者向左跑入革命派,或者向右跑入反革命派,没有他们“独立”的余地。所以,中国的中产阶级,以其本阶级为主体的“独立”革命思想,仅仅是一个幻想。

小资产阶级。如自耕农\mnote{7},手工业主,小知识阶层——学生界、中小学教员、小员司、小事务员、小律师,小商人等都属于这一类。这一个阶级,在人数上,在阶级性上,都值得大大注意。自耕农和手工业主所经营的,都是小生产的经济。这个小资产阶级内的各阶层虽然同处在小资产阶级经济地位,但有三个不同的部分。第一部分是有余钱剩米的,即用其体力或脑力劳动所得,除自给外,每年有余剩。这种人发财观念极重,对赵公元帅礼拜最勤,虽不妄想发大财,却总想爬上中产阶级地位。他们看见那些受人尊敬的小财东,往往垂着一尺长的涎水。这种人胆子小,他们怕官,也有点怕革命。因为他们的经济地位和中产阶级颇接近,故对于中产阶级的宣传颇相信,对于革命取怀疑的态度。这一部分人在小资产阶级中占少数,是小资产阶级的右翼。第二部分是在经济上大体上可以自给的。这一部分人比较第一部分人大不相同,他们也想发财,但是赵公元帅\mnote{8}总不让他们发财,而且因为近年以来帝国主义、军阀、封建地主、买办大资产阶级的压迫和剥削,他们感觉现在的世界已经不是从前的世界。他们觉得现在如果只使用和从前相等的劳动,就会不能维持生活。必须增加劳动时间,每天起早散晚,对于职业加倍注意,方能维持生活。他们有点骂人了,骂洋人叫“洋鬼子”,骂军阀叫“抢钱司令”,骂土豪劣绅叫“为富不仁”。对于反帝国主义反军阀的运动,仅怀疑其未必成功(理由是:洋人和军阀的来头那么大),不肯贸然参加,取了中立的态度,但是绝不反对革命。这一部分人数甚多,大概占小资产阶级的一半。第三部分是生活下降的。这一部分人好些大概原先是所谓殷实人家,渐渐变得仅仅可以保住,渐渐变得生活下降了。他们每逢年终结账一次,就吃惊一次,说:“咳,又亏了!”这种人因为他们过去过着好日子,后来逐年下降,负债渐多,渐次过着凄凉的日子,“瞻念前途,不寒而栗”。这种人在精神上感觉的痛苦很大,因为他们有一个从前和现在相反的比较。这种人在革命运动中颇要紧,是一个数量不小的群众,是小资产阶级的左翼。以上所说小资产阶级的三部分,对于革命的态度,在平时各不相同;但到战时,即到革命潮流高涨、可以看得见胜利的曙光时,不但小资产阶级的左派参加革命,中派亦可参加革命,即右派分子受了无产阶级和小资产阶级左派的革命大潮所裹挟,也只得附和着革命。我们从一九二五年的五卅运动\mnote{9}和各地农民运动的经验看来,这个断定是不错的。

半无产阶级。此处所谓半无产阶级,包含:(一)绝大部分半自耕农\mnote{10},(二)贫农,(三)小手工业者,(四)店员\mnote{11},(五)小贩等五种。绝大部分半自耕农和贫农是农村中一个数量极大的群众。所谓农民问题,主要就是他们的问题。半自耕农、贫农和小手工业者所经营的,都是更细小的小生产的经济。绝大部分半自耕农和贫农虽同属半无产阶级,但其经济状况仍有上、中、下三个细别。半自耕农,其生活苦于自耕农,因其食粮每年大约有一半不够,须租别人田地,或者出卖一部分劳动力,或经营小商,以资弥补。春夏之间,青黄不接,高利向别人借债,重价向别人籴粮,较之自耕农的无求于人,自然景遇要苦,但是优于贫农。因为贫农无土地,每年耕种只得收获之一半或不足一半;半自耕农则租于别人的部分虽只收获一半或不足一半,然自有的部分却可全得。故半自耕农的革命性优于自耕农而不及贫农。贫农是农村中的佃农,受地主的剥削。其经济地位又分两部分。一部分贫农有比较充足的农具和相当数量的资金。此种农民,每年劳动结果,自己可得一半。不足部分,可以种杂粮、捞鱼虾、饲鸡豕,或出卖一部分劳动力,勉强维持生活,于艰难竭蹶之中,存聊以卒岁之想。故其生活苦于半自耕农,然较另一部分贫农为优。其革命性,则优于半自耕农而不及另一部分贫农。所谓另一部分贫农,则既无充足的农具,又无资金,肥料不足,土地歉收,送租之外,所得无几,更需要出卖一部分劳动力。荒时暴月,向亲友乞哀告怜,借得几斗几升,敷衍三日五日,债务丛集,如牛负重。他们是农民中极艰苦者,极易接受革命的宣传。小手工业者所以称为半无产阶级,是因为他们虽然自有简单的生产手段,且系一种自由职业,但他们也常常被迫出卖一部分劳动力,其经济地位略与农村中的贫农相当。因其家庭负担之重,工资和生活费用之不相称,时有贫困的压迫和失业的恐慌,和贫农亦大致相同。店员是商店的雇员,以微薄的薪资,供家庭的费用,物价年年增长,薪给往往须数年一增,偶与此辈倾谈,便见叫苦不迭。其地位和贫农及小手工业者不相上下,对于革命宣传极易接受。小贩不论肩挑叫卖,或街畔摊售,总之本小利微,吃着不够。其地位和贫农不相上下,其需要一个变更现状的革命,也和贫农相同。

无产阶级。现代工业无产阶级约二百万人。中国因经济落后,故现代工业无产阶级人数不多。二百万左右的产业工人中,主要为铁路、矿山、海运、纺织、造船五种产业的工人,而其中很大一个数量是在外资产业的奴役下。工业无产阶级人数虽不多,却是中国新的生产力的代表者,是近代中国最进步的阶级,做了革命运动的领导力量。我们看四年以来的罢工运动,如海员罢工\mnote{12}、铁路罢工\mnote{13}、开滦和焦作煤矿罢工\mnote{14}、沙面罢工\mnote{15}以及“五卅”后上海香港两处的大罢工\mnote{16}所表现的力量,就可知工业无产阶级在中国革命中所处地位的重要。他们所以能如此,第一个原因是集中。无论哪种人都不如他们的集中。第二个原因是经济地位低下。他们失了生产手段,剩下两手,绝了发财的望,又受着帝国主义、军阀、资产阶级的极残酷的待遇,所以他们特别能战斗。都市苦力工人的力量也很可注意。以码头搬运夫和人力车夫占多数,粪夫清道夫等亦属于这一类。他们除双手外,别无长物,其经济地位和产业工人相似,惟不及产业工人的集中和在生产上的重要。中国尚少新式的资本主义的农业。所谓农村无产阶级,是指长工、月工、零工等雇农而言。此等雇农不仅无土地,无农具,又无丝毫资金,只得营工度日。其劳动时间之长,工资之少,待遇之薄,职业之不安定,超过其它工人。此种人在乡村中是最感困难者,在农民运动中和贫农处于同一紧要的地位。

此外,还有数量不小的游民无产者,为失了土地的农民和失了工作机会的手工业工人。他们是人类生活中最不安定者。他们在各地都有秘密组织,如闽粤的“三合会”,湘鄂黔蜀的“哥老会”,皖豫鲁等省的“大刀会”,直隶及东三省的“在理会”,上海等处的“青帮”\mnote{17},都曾经是他们的政治和经济斗争的互助团体。处置这一批人,是中国的困难的问题之一。这一批人很能勇敢奋斗,但有破坏性,如引导得法,可以变成一种革命力量。

综上所述,可知一切勾结帝国主义的军阀、官僚、买办阶级、大地主阶级以及附属于他们的一部分反动知识界,是我们的敌人。工业无产阶级是我们革命的领导力量。一切半无产阶级、小资产阶级,是我们最接近的朋友。那动摇不定的中产阶级,其右翼可能是我们的敌人,其左翼可能是我们的朋友——但我们要时常提防他们,不要让他们扰乱了我们的阵线。


\begin{maonote}
\mnitem{1}国家主义派指中国青年党,当时以其外围组织“中国国家主义青年团”的名义公开进行活动。组织这个政团的是一些反动政客,他们投靠帝国主义和当权的反动派,把反对中国共产党和苏联当作职业。
\mnitem{2}戴季陶(一八九一——一九四九),又名传贤,原籍浙江湖州,生于四川广汉。早年参加中国同盟会,从事过反对清政府和袁世凯的活动。后曾和蒋介石在上海共同经营交易所的投机事业。一九二五年随着孙中山的逝世和革命高潮的到来,他歪曲孙中山学说的革命内容,散布反对国共合作、反对工农革命运动的谬论,为后来蒋介石的反共活动作了准备。一九二七年南京国民政府成立后,历任国民政府委员、考试院院长等职。一九四九年二月,蒋介石的统治即将崩溃,戴季陶感到绝望而自杀。
\mnitem{3}北京《晨报》,初名《晨钟报》,一九一六年八月创刊于北京,一九一八年十二月改名为《晨报》,一九二八年六月停刊。
\mnitem{4}一九二二年和一九二三年间,孙中山在共产党人的帮助下,决定改组国民党,实行国共合作,容纳共产党人参加国民党,并于一九二四年一月在广州召开国民党第一次全国代表大会,实行联俄、联共、扶助农工的三大政策。李大钊、谭平山、毛泽东、林伯渠、瞿秋白等共产党人参加了这次大会。他们曾经被选为国民党中央执行委员会的委员或候补委员,担任过国民党的许多领导工作,对于帮助国民党走上革命的道路,起了重大的作用。
\mnitem{5}第三国际即共产国际,一九一九年三月在列宁领导下成立。一九二二年中国共产党参加共产国际,成为它的一个支部。一九四三年五月,共产国际执行委员会主席团通过决定,提议解散共产国际,同年六月共产国际正式宣布解散。
\mnitem{6}国际联盟简称国联,一九二〇年一月正式成立。先后参加的有六十多个国家。国际联盟标榜以“促进国际合作,维持国际和平与安全”为目的,实际上日益成为帝国主义国家推行侵略政策的工具。第二次世界大战爆发后无形瓦解,一九四六年四月正式宣布解散。
\mnitem{7}这里是指中农。
\mnitem{8}赵公元帅是中国民间传说的财神,叫赵公明。
\mnitem{9}指一九二五年五月三十日爆发的反帝爱国运动。一九二五年五月间,上海、青岛的日本纱厂先后发生工人罢工的斗争,遭到日本帝国主义和北洋军阀的镇压。上海内外棉第七厂日本资本家在五月十五日枪杀了工人顾正红,并伤工人十余人。二十九日青岛工人被反动政府屠杀八人。五月三十日,上海二千余学生分头在公共租界各马路进行宣传讲演,一百余名遭巡捕(租界内的警察)逮捕,被拘押在南京路老闸巡捕房内,引起了学生和市民的极大愤慨,有近万人聚集在巡捕房门口,要求释放被捕学生。英帝国主义的巡捕向群众开枪,打死打伤许多人。这就是震惊中外的五卅惨案。六月,英日等帝国主义在上海和其它地方继续进行屠杀。这些屠杀事件激起了全国人民的公愤。广大的工人、学生和部分工商业者,在许多城市和县镇举行游行示威和罢工、罢课、罢市,形成了全国规模的反帝爱国运动高潮。
\mnitem{10}这里是指自己有一部分土地,同时租种一部分土地,或出卖一部分劳动力,或兼营小商的贫农。
\mnitem{11}店员有不同的阶层,他们一般不占有生产资料,生活来源的全部或者主要部分是依靠向店主出卖劳动力所取得的工资。毛泽东在这里所指的是店员中的一部分,还有一部分下层店员过着无产阶级的生活。
\mnitem{12}指一九二二年香港和上海的海员罢工。香港海员罢工爆发于一月十二日,坚持了八个星期。最后,香港英帝国主义当局被迫答应增加工资,恢复原工会,释放被捕工人,抚恤在罢工中死难烈士的家属。上海海员罢工于八月五日开始,坚持了三个星期,也得到胜利。
\mnitem{13}指一九二二年和一九二三年中国共产党领导的各主要铁路线的工人罢工。在罢工过程中,工人群众的觉悟迅速提高,要求改善生活的经济斗争迅速发展为反对军阀的政治斗争。一九二三年二月四日,京汉铁路工人为争取组织总工会的自由,举行总罢工。其它许多铁路的工人也纷纷响应。二月七日,英帝国主义支持的北洋军阀吴佩孚、萧耀南等,残酷地屠杀京汉铁路的工人,造成了二七惨案。
\mnitem{14}开滦煤矿是直隶省(今河北省)开平、滦县一带煤矿的总称,当时为英帝国主义者所控制。开滦罢工指一九二二年十月、十一月间矿工三万余人举行的大罢工。英帝国主义者和北洋军阀对这次罢工进行残酷的镇压,工人死伤很多,但是仍然坚持斗争。最后,英帝国主义者不得不答应给工人增加一部分工资。焦作煤矿,在河南省北部,当时也为英帝国主义者所控制。焦作罢工指一九二五年七月爆发的罢工。这次罢工是为响应五卅运动而发动的,前后坚持七个多月。最后,英帝国主义者不得不承认工会有代表工人的权利,并且被迫接受增加工资、不无故开除工人和补偿工人因罢工所受的损失等项条件。
\mnitem{15}沙面当时是英法帝国主义在广州的租界。一九二四年七月,统治沙面的帝国主义者颁布新警律,规定沙面的中国人出入租界必须携带贴有本人相片的执照,在租界内行动必须受各种苛刻的限制,但是外国人却可以自由出入活动。沙面工人于七月十五日宣告罢工,抗议这些无理措施。这次沙面罢工迫使英法帝国主义者取消了新警律。
\mnitem{16}指一九二五年六月一日开始的上海大罢工和六月十九日开始的香港大罢工。这两处罢工是当时全国反帝爱国运动的支柱。前者有二十多万工人参加,坚持了三四个月;后者有二十五万工人参加,坚持了一年零四个月,是截至当时为止的世界工人运动史中时间最长的一次罢工。
\mnitem{17}三合会、哥老会、大刀会、在理会、青帮是旧中国的一些民间秘密团体,参加者主要的是破产农民、失业手工业工人和流氓无产者。这类团体大都用宗教迷信为团聚成员的工具,采取家长制的组织形式,有的还拥有武装。参加这类团体的人,在社会生活中有互相援助的义务,有时还共同反抗压迫他们的地主、官僚和外国侵略者。但是,农民和手工业工人不可能依靠这类团体得到出路。同时,由于这类团体带有严重的封建性和盲目的破坏性,它们又往往容易被反动统治阶级和帝国主义势力所操纵和利用。随着工人阶级力量的壮大和中国共产党的成立,农民和手工业工人在共产党的领导之下逐步地建立了完全新式的群众组织,这类落后的团体就失掉了它们的存在价值。
\end{maonote}
