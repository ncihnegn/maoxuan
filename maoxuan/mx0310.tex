
\title{第二次世界大战的转折点}
\date{一九四二年十月十二日}
\thanks{这是毛泽东为延安《解放日报》写的社论。}
\maketitle


斯大林格勒之战,英美报纸比之为凡尔登战役,“红色凡尔登”之名已传遍于世界。这个比拟并不适当。今天的斯大林格勒之战,比起第一次世界大战时的凡尔登来,有性质的不同。但有一点是相同的,即有许多人在这种时候还被德国的攻势所迷惑,以为德国还有获胜的可能。第一次世界大战结束于一九一八年冬,在一九一六年,德军曾向法国要塞凡尔登举行数度的进攻。当时德军的战役统帅是德国皇太子,投入战斗的力量是德军的最精锐部分。当时的战斗是带决战性的。德军猛攻不克,整个德奥土保阵线再也找不到出路,从此日益困难,众叛亲离,土崩瓦解,走到了最后的崩溃。然而当时英美法阵线方面,还没有看出这种情况,以为德军仍极强大,不知道自己的胜利已经快到面前。在人类历史上,凡属将要灭亡的反动势力,总是要向革命势力进行最后挣扎的,而有些革命的人们也往往在一个期间内被这种外强中干的现象所迷惑,看不出敌人快要消灭,自己快要胜利的实质。整个法西斯势力的兴起及其进行了几年的侵略战争,正是这种最后挣扎的表现;而在战争中,又以攻击斯大林格勒表现它自己的最后挣扎。在这个历史的转折点面前,全世界反法西斯阵线内的人们也有许多被法西斯的凶恶面孔所迷惑,看不出它的实质。自从八月二十三日德军全部渡过顿河河曲,全面地开始攻击斯大林格勒,九月十五日德军一部打入该城西北部工业区,至十月九日苏联情报局宣布红军突破该区德军包围线为止,共计进行了四十八天人类历史上无与伦比的空前苦战。这一战终于胜利了。在这四十八天中,这个城市每天的胜负消息,紧系着无数千万、万万人民的呼吸,使他们忧愁,使他们欢乐。这一战,不但是苏德战争的转折点,甚至也不但是这次世界反法西斯战争的转折点,而且是整个人类历史的转折点。在这四十八天中,世界人民的注视斯大林格勒,和去年十月间世界人民的注视莫斯科,其关心程度,是有过之无不及的。

希特勒在西线胜利以前,他似乎是谨慎的。攻波兰,攻挪威,攻荷、比、法,攻巴尔干,都是注全力于一处,不敢旁骛。西线胜利后,他就冲昏了头脑,企图在三个月内打败苏联。北起摩尔曼斯克,南至克里米亚,向这个庞大坚强的社会主义国家举行了全面的进攻,这样就分散了他的兵力。去年十月向莫斯科进攻的失败,结束了苏德战争的第一阶段;希特勒第一个战略计划破产了。红军制止了德军去年的进攻,并在冬季举行了全线的反攻,是为苏德战争的第二阶段;希特勒转到了退却和防御的地位。在此期间,希特勒撤消了他的前线总司令勃鲁齐区,自己充任总司令,决定放弃全面的进攻计划,搜索欧洲全力,准备向南线作局部的但被认为是打击苏联要害的最后进攻。因为这一进攻带着最后一次的性质,关系法西斯的存亡,希特勒就集中了极大的兵力,连在北非作战中的一部分飞机坦克都抽调过来了。从今年五月进攻刻赤和塞瓦斯托波尔起,进入战争的第三阶段。希特勒调动了一百五十万以上的兵力,附以飞机坦克的主力,向斯大林格勒和高加索作空前剧烈的进攻。他企图迅速攻下两处,达到切断伏尔加和夺取巴库两个目的,然后北攻莫斯科,南出波斯湾,并令日本法西斯集中兵力于满洲,准备在斯大林格勒攻下后进攻西伯利亚。希特勒妄想把苏联力量削弱到足以使德军主力从苏联战场上解脱出来,以便移到西线对付英美的进攻,并可掠取近东资源,打通德日联系,同时,日军主力也可从北面解脱出来,以便西进南进对付我国和英美,而无后顾之忧,这样来争取法西斯阵线的胜利。但是这个阶段的情况是怎样的呢?希特勒遇到了苏联制其死命的策略。苏联采取了先则诱敌深入、继则顽强抵抗的方针。五个月的战争,使德军既没有打进高加索油田,也没有打下斯大林格勒,迫使希特勒顿兵于高山与坚城之下,欲进不能,欲退不得,损失甚大,陷于僵局。现在已是十月,冬季就要到来,战争的第三阶段快要结束,第四阶段快要开始了。希特勒进攻苏联的战略企图没有一个不是失败的。在此期间,希特勒鉴于去夏分兵的失败,集中他的兵力向着南线。然而他尚欲东断伏尔加,南取高加索,一举达成两个目的,仍然分散了他的兵力。他尚未计算到他的实力和他的企图之间的不相称,以致“扁担没扎,两头打塌”,陷入目前的绝路。在相反方面,苏联则是越战越强。斯大林的英明战略指挥,完全站在主动的地位,处处把希特勒引向灭亡。今年冬季开始的第四个阶段,将是希特勒走向死亡的阶段。

拿希特勒在第一阶段上的情况和第三阶段作比较,就可知希特勒是处在最后失败的门口了。目前红军在斯大林格勒和高加索两方面,实际上均已停止了德军的进攻,希特勒已到再衰三竭之时,他对斯大林格勒、高加索两处的进攻已经失败。他在去年十二月至今年五月整个冬季中所整备的一点兵力,已经耗竭了。在苏德战线,距冬季不到一个月了,他须赶快转入防御。整个顿河的以西以南是他的最危险的地带,红军将在这一带转入反攻。今年冬季,希特勒因被死亡所驱迫,将再一次整备他的军队。他或者还可能搜索他的一点残余力量装备出几个新的师团,此外则乞援于意、罗、匈三国法西斯伙伴,向他们勒索一些炮灰,以应付东西两线的危局。但是,他在东线须应付冬季战争的极大消耗,他在西线须准备对付第二条战线,而意、罗、匈等国则将在希特勒大势已去的这种悲观情绪中,一天一天变成离心离德。总之,十月九日以后的希特勒,将只有死路一条好走了。

四十八天中,红军的保卫斯大林城,和去年保卫莫斯科市有某种相同。这就是说,它使得希特勒今年的计划也像他的去年计划一样,归于失败。其不同点,则在莫斯科保卫战之后,虽然接着举行了冬季反攻,可是还要遭到今年德军的一个夏季进攻,这是因为一则德国及其欧洲伙伴尚有余勇可贾,二则英美拖延开辟第二条战线的缘故。而在斯大林格勒保卫战之后,则形势将和去年完全两样。一方面苏联将举行极大规模的第二个冬季反攻,英美对第二条战线的开辟将无可拖延(虽然具体时间仍不能计算),欧洲人民也将准备着起义响应。另一方面,德国及其欧洲伙伴再也无力举行大规模的攻势了,希特勒只好把整个方针转入战略防御。只要迫使希特勒转入了战略防御,法西斯的命运就算完结了。因为像希特勒这样法西斯国家的政治生命和军事生命,从它出生的一天起,就是建立在进攻上面的,进攻一完结,它的生命也就完结了。斯大林格勒一战将停止法西斯的进攻,这一战是带着决定性的。这种决定性,是关系于整个世界战争的。

希特勒面前遇着的,是三个强大敌人:苏联、英美及在其占领区的老百姓。在东线,是屹立不动的红军壁垒和整个第二冬季以及连续下去的红军反攻,这是整个战争和人类命运的决定的力量。在西线,即使英美还采取着观望和拖延的政策,但等到有死老虎可打的时候,第二条战线总是要建立的。希特勒还有一个内部战线,就是德国、法国及欧洲其它部分正在酝酿着的一个伟大的人民起义,只待苏联举行全面反攻和第二条战线炮响,他们将以第三条战线出来响应。这样,三条战线夹击希特勒,就将是斯大林格勒战役以后的伟大历史过程。

拿破仑的政治生命,终结于滑铁卢,而其决定点,则是在莫斯科的失败\mnote{1}。希特勒今天正是走的拿破仑道路,斯大林格勒一役,是他的灭亡的决定点。

这一形势,将直接影响到远东。明年也将不是日本法西斯的吉利年头。它将一天一天感到头痛,直至向它的墓门跨进。

一切对世界形势作悲观观察的人们,应将自己的观点改变过来。


\begin{maonote}
\mnitem{1}一八一五年六月,拿破仑的军队同英普联军激战于比利时的滑铁卢。拿破仑战败,被流放于大西洋南部的圣赫勒拿岛,至一八二一年死于该岛。拿破仑一生征服过欧洲的许多国家,但是在一八一二年进攻俄国的战争中,在莫斯科遭到极大的失败,他的精锐部队几乎全部被消灭。拿破仑受到了这次打击,从此便一蹶不振。关于拿破仑在莫斯科的失败,见本书第二卷\mxnote{论持久战}{41}。
\end{maonote}
