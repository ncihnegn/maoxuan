
\title{关于农业机械化——备战备荒为人民}
\date{一九六六年三月十二日}
\thanks{这是毛泽东同志就农业机械化问题致刘少奇的信。}
\maketitle


\mxname{少奇同志:}

三月十一日信\mnote{1}收到。小计委\mnote{2}派人去湖北,同湖北省委共同研究农业机械化五年、七年、十年的方案,并参观那里自力更生办机械化的试点,这个意见很好。建议各中央局、各省市区党委也各派人去湖北共同研究。有七天至十天时间即可以了。回去后,各做一个五、七、十年计划的初步草案,酝酿几个月,然后在大约今年八九月间召开的工作会议上才有可议。若事前无准备,那时议也恐怕议不好的。此事以各省、市、区自力更生为主,中央只能在原材料等等方面,对原材料等等不足的地区有所帮助,也要由地方出钱购买,也要中央确有原材料储备可以出售的条件,不能一哄而起,大家伸手。否则推迟时间,几年后再说。为此,原材料(钢铁),工作母机,农业机械,凡国家管理、地方制造、超出国家计划远甚者(例如超出一倍以上者),在超过额内,准予留下三成至五成,让地方购买使用。此制不立,地方积极性是调动不起来的。为了农业机械化,多产农林牧副渔等品类,要为地方争一部分机械制造权。所谓一部分机械制造权,就是大超额分成权,小超额不在内。一切统一于中央,卡得死死的,不是好办法。又此事应与备战、备荒、为人民联系起来,否则地方有条件也不会热心去做。

第一是备战,人民和军队总得先有饭吃有衣穿,才能打仗,否则虽有枪炮,无所用之。

第二是备荒,遇了荒年,地方无粮棉油等储蓄,仰赖外省接济,总不是长久之计。一遇战争,困难更大。而局部地区的荒年,无论哪一个省内常常是不可避免的。几个省合起来看,就更加不可避免。

第三是国家积累不可太多,要为一部分人民至今口粮还不够吃、衣被甚少着想;再则要为全体人民分散储备以为备战备荒之用着想;三则更加要为地方积累资金用之于扩大再生产着想。

所以,农业机械化,要同这几方面联系起来,才能动员群众,为较快地但是稳步地实现此种计划而奋斗。苏联的农业政策,历来就有错误,竭泽而渔,脱离群众,以致造成现在的困境,主要是长期陷在单纯再生产坑内,一遇荒年,连单纯再生产也保不住。我们也有过几年竭泽而渔(高征购)和很多地区荒年保不住单纯再生产的经验,总应该引以为戒吧。现在虽然提出了备战、备荒、为人民(这是最好地同时为国家的办法,还是“百姓足,君孰与不足”\mnote{3}的老话)的口号,究竟能否持久地认真地实行,我看还是一个问题,要待将来才能看得出是否能够解决。苏联的农业不是基本上机械化了吗?是何原因至今陷于困境呢?此事很值得想一想。

以上几点意见,是否可行,请予酌定。又小计委何人去湖北,似以余秋里、林乎加\mnote{4}二同志去为宜。如果让各中央局、各省市区党委也派人去的话,似以管农业书记一人计委一人去为宜。总共也只有大约七十人左右去到那里开一个七天至十天的现场会。是否可行,亦请斟酌。

\begin{maonote}
\mnitem{1}指刘少奇一九六六年三月十一日给毛泽东的信。信中说:中共湖北省委关于逐步实现农业机械化设想的文件和主席的批语,已印发给政治局、书记处各同志,并发给包括计委、经委在内的各有关部委及华北局有关同志研究。在有各在京副总理参加的中央常委会上谈了这个问题,大家意见,要小计委就这个问题对有关各方面情况先摸一摸,提出一个方案,中央再来讨论,并要提交下一次中央工作会议加以讨论,以使各地方的努力更加符合实际。周恩来同志已要小计委派人到湖北,同省委共同研究他们提出的方案,先在湖北进行试验。刘少奇当时主持中央一线工作。
\mnitem{2}小计委,是一九六五年初毛泽东决定成立、由周恩来直接领导的一个工作机构,主要任务是研究经济和社会发展战略问题,拟定第三个五年计划的方针任务等。后来在编制第三个五年计划过程中,由小计委实际主持国家计委的工作。
\mnitem{3}见《论语·颜渊》。
\mnitem{4}余秋里(一九一四——一九九九),江西吉安人,时任国家计划委员会第一副主任兼秘书长、小计委负责人。林乎加,一九一六年生,山东长岛人,当时是小计委成员。
\end{maonote}
