
\title{人民解放军应该支持左派广大群众}
\date{一九六七年一月二十一日}
\thanks{这是毛泽东同志在南京军区党委请示报告\mnote{1}上的批语。}
\maketitle


\mxname{林彪\mnote{2}同志:}

应派军队支持左派广大群众\mnote{3}。请酌处。

以后,凡有真正革命派要求军队支持援助,都应该这样做。所谓不介入是假的,早已介入了。此事似应重新发布命令\mnote{4},以前命令作废。请酌。又及。

\begin{maonote}
\mnitem{1}中国人民解放军南京军区党委一九六七年一月二十一日向中共中央军委副主席林彪并中央军委报告说,“顷接安徽军区报告,首都第三造反司令部驻安徽联络站等单位向安徽军区提出,廿二日到廿三日,在合肥召开十五万到二十万人大会”,“要安徽军区派出三百到五百名部队警卫会场。他们提出,如派部队就是支持文化大革命,如不派就是不支持文化大革命,并限安徽军区廿一日十四时前答复。是否派部队,请速指示”。
\mnitem{2}林彪,时任中央政治局常委,中央副主席,主持中央工作。
\mnitem{3}一九六七年一月二十八日,中央军委发布了《军委八条命令》,全文如下:

根据毛主席的指示,无产阶级文化大革命已进入全面阶级斗争的新阶段,军队必须改变过去不介入地方文化大革命的规定。为了适应两个阶级、两条路线斗争发展的新形势,特规定如下:

一、必须坚决支持真正的无产阶级革命派,争取和团结大多数,坚决反对右派,对那些证据确凿的反革命组织和反革命分子,坚决采取专政措施。

二、一切指战员、政治工作人员、勤务、医疗、科研和机要工作人员,必须坚守岗位,不得擅离职守。要抓革命,促战备、促工作、促生产。

三、军队内部开展文化大革命的单位,应该实行大鸣、大放、大字报、大辩论,充分运用摆事实、讲道理的方法。严格区分两类矛盾。不允许用对待敌人的方法来处理人民内部矛盾,不允许无命令自由抓人,不允许任意抄家、封门,不允许体罚和变相体罚,例如:戴高帽、挂黑牌、游街、罚跪,等等。认真提倡文斗,坚决反对武斗。

四、一切外出串连的院校师生、文艺团体、体工队、医院和军事工厂的职工等,应迅速返回本地区、本单位进行斗批改,把本单位被一小撮走资本主义道路当权派篡夺的权夺回来,不要逗留在北京和其他地方。

五、对于冲击军事领导机关问题,要分别对待。过去如果是反革命冲击了,要追究,如果是左派冲击了,可以不予追究。今后则一律不许冲击。

六、军队内战备系统和保密系统,不准冲击,不准串连。凡非文化革命的文件、档案和技术资料一概不得索取的抢劫。有关文化革命的资料,暂时封存,听候处理。

七、军以上机关应按规定分期分批进行文化大革命。军、师、团、营、连和军委指定的特殊单位,坚持采取正面教育的方针,以利于加强战备,保卫国防,保卫无产阶级文化大革命。

八、各级干部,特别是高级干部,要用毛泽东思想严格管教子女,教育他们努力学习毛主席著作,认真与工农相结合,拜工农为师,参加劳动锻炼,改造世界观,争取作无产阶级革命派。干部子女如有违法乱纪行为,应该交给群众教育,严重的,交给公安和司法机关处理。

以上规定,从公布之日起,立即生效,全体指战员、院校师生、文艺团体、体工队、医院和军事工厂的职工同志,必须严格遵守,违者要受纪律处理。

中共中央军事委员会

一九六七年一月二十八日
\mnitem{4}一月二十三日,中共中央、国务院、中央军委、中央文革小组下发了《中共中央、国务院、中央军委、中央文革小组关于人民解放军坚决支持革命左派群众的决定》全文如下:

各中央局,各大军区,各省、市、自治区党委、人委,并转各级党委、人委、军区、军分区:

在毛主席的领导下,无产阶级文化大革命开始了一个新阶段。这个新阶段的主要特点,就是无产阶级革命派大联合,向党内一小撮走资本主义道路的当权派和坚持资产阶级反动路线的顽固分子手里夺权。这场夺权斗争,是无产阶级对资产阶级及其在党内的代理人十七年来猖狂进攻的总反击。这是全国全面的阶级斗争,是一个阶级推翻一个阶级的大革命。

人民解放军是毛主席亲手缔造的无产阶级的革命军队,是无产阶级专政最重要的工具。在这场伟大的无产阶级向资产阶级的夺权斗争中,人民解放军必须坚决站在无产阶级革命派一边,坚决支持和援助无产阶级革命左派。

最近,毛主席指示:人民解放军应该支持左派广大群众。以后,凡有真正革命派要找军队支持、援助,都应当满足他们的要求。所谓“不介入”,是假的,早已介入了。问题不是介入不介入的问题,而是站在哪一边的问题,是支持革命派还是支持保守派甚至右派的问题。人民解放军应当积极支持革命左派。

我军全体指战员必须坚决执行毛主席的指示。

(一)以前关于军队不介入地方文化大革命以及其他违反上述精神的指示,一律作废。

(二)积极支持广大革命左派群众的夺权斗争。凡是真正的无产阶级左派要求军队去援助他们,军队都应当派出部队积极支持他们。

(三)坚决镇压反对无产阶级革命左派的反革命分子、反革命组织,如果他们动武,军队应当坚决还击。

(四)重申军队不得做一小撮党内走资本主义道路当权派和坚持资产阶级反动路线顽固分子防空洞的指示。

(五)在全军深入进行以毛主席为代表的无产阶级革命路线和以刘少奇、邓小平为代表的资产阶级反动路线斗争的教育。

这一指示要原原本本地传达到每个解放军战士。

中共中央

国务院

中央军委

中央文革小组

一九六七年一月二十三日
\end{maonote}
