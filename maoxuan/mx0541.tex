
\title{在中国共产党全国代表会议上的讲话}
\date{一九五五年三月}
\maketitle


\date{一九五五年三月二十一日}
\section{开幕词}

\mxname{中国共产党全国代表会议各位代表同志们:}

我们这次全国代表会议有三个议事日程:第一,关于发展国民经济的第一个五年计划和关于这个计划的报告;第二,关于高岗、饶漱石反党联盟的报告;第三,关于成立中央监察委员会。

中央委员会根据列宁关于过渡时期的学说,总结了中华人民共和国成立以来的经验,在我国国民经济恢复阶段将要结束的时候,即一九五二年,提出了党在过渡时期的总路线。这个总路线就是在大约三个五年计划的期间内,逐步实现国家的社会主义工业化,同时对于农业、手工业和资本主义工商业逐步实现社会主义改造,以求达到在我国建成社会主义社会的目的。党的总路线以及党为着实现这个总路线而采取的各项重要的政策和办法,已经在事实上被证明是正确的。依靠全党同志和全国人民的努力,我们的工作是有很大成绩的。但是我们在工作中也有缺点和错误。我们的许多办法不可能在一切方面都规定得很恰当,这应当在实行中根据新的经验加以补充和修正。

发展国民经济的第一个五年计划是实现党的总路线的一个重大的步骤。这次党的全国代表会议应该根据实际经验,认真地讨论这个计划草案,使它的内容能够比较妥当,而成为切实可行的计划。

在我们这样一个大国里面,情况是复杂的,国民经济原来又很落后,要建成社会主义社会,并不是轻而易举的事。我们可能经过三个五年计划建成社会主义社会,但要建成为一个强大的高度社会主义工业化的国家,就需要有几十年的艰苦努力,比如说,要有五十年的时间,即本世纪的整个下半世纪。我们的任务要求我们必须很好地处理我国人民内部的关系一一特别是工人阶级和农民之间的关系,很好地处理我国各民族之间的关系;同时,必须很好地继续发展同伟大的先进社会主义国家苏联和各人民民主国家的亲密合作,也要发展同资本主义世界一切爱好和平的国家和人民的合作。

我们经常说,不要因为我们的工作有成绩就骄傲自满起来,应该保持谦虚态度,向先进国家学习,向群众学习,在同志间也要互相学习,以求少犯错误。在这次党代表会议上,我感觉仍然需要重复地将这些话说一遍。鉴于高岗、饶漱石的反党事件,骄傲自满情绪在我们党内确实是存在着,在有些同志的身上这种情绪还是严重的,不克服这种情绪,就会妨碍我们建设社会主义社会这个伟大任务的完成。

同志们都知道,高岗、饶漱石反党联盟的出现,不是偶然的现象,它是我国现阶段激烈阶级斗争的一种尖锐的表现。这个反党联盟的罪恶目的,是要分裂我们的党,用阴谋方法夺取党和国家的最高权力,而为反革命的复辟开辟道路。全党在中央委员会团结一致的领导下,已经把这个反党联盟彻底地粉碎了,我们的党因此更加团结起来和巩固起来了。这是我们在为社会主义事业而奋斗中的一个重大的胜利。

对于我们的党说来,高岗、饶漱石事件是一个重要的教训,全党应该引为鉴戒,务必使党内不要重复出现这样的事件。高岗、饶漱石在党内玩弄阴谋,进行秘密活动,在同志背后进行挑拨离间,但在公开场合则把他们的活动伪装起来。他们的这种活动完全是地主阶级和资产阶级在历史上常常采取的那一类丑恶的活动。马克思、恩格斯在《共产党宣言》上说过:“共产党人认为隐秘自己的观点与意图是可耻的事。”我们是共产党人,更不待说是党的高级干部,在政治上都要光明磊落,应该随时公开说出自己的政治见解,对于每一个重大的政治问题表示自己或者赞成或者反对的态度,而绝对不可以学高岗、饶漱石那样玩弄阴谋手段。

为着建成社会主义社会这一个目的,中央委员会认为有必要在这个时候按照党章成立中央监察委员会,代替过去的纪律检查委员会,借以在新的激烈的阶级斗争时期加强党的纪律,加强对各种违法乱纪现象的斗争,特别是防止像高、饶反党联盟这一类严重危害党的利益的事件重复发生。

鉴于种种历史教训,鉴于个人的智慧必须和集体的智慧相结合才能发挥较好的作用和使我们在工作中少犯错误,中央和各级党委必须坚持集体领导的原则,继续反对个人独裁和分散主义两种偏向。必须懂得,集体领导和个人负责这样两个方面,不是互相对立的,而是互相结合的。而个人负责,则和违反集体领导原则的个人独裁,是完全不同的两件事。

目前的国际条件对我们的社会主义建设事业是有利的。以苏联为首的社会主义阵营是强大的,内部是团结的;而帝国主义阵营则是虚弱的,在它们那里有不可克服的重重矛盾和危机。虽然是这样,但是我们应该了解:帝国主义势力还是在包围着我们,我们必须准备应付可能的突然事变。今后帝国主义如果发动战争,很可能像第二次世界大战时期那样,进行突然的袭击。因此,我们在精神上和物质上都要有所准备,当着突然事变发生的时候,才不至于措手不及。这是一方面。另一方面,国内反革命残余势力的活动还很猖獗,我们必须有计划地、有分析地、实事求是地再给他们几个打击,使暗藏的反革命力量更大地削弱下来,借以保证我国社会主义建设事业的安全。如果我们在上述两方面都做了适当的措施,就可能避免敌人给我们的重大危害,否则我们可能要犯错误。

同志们!我们现在是处在新的历史时期。一个六万万人口的东方国家举行社会主义革命,要在这个国家里改变历史方向和国家面貌,要在大约三个五年计划期间内使国家基本上工业化,并且要对农业、手工业和资本主义工商业完成社会主义改造,要在大约几十年内追上或赶过世界上最强大的资本主义国家,这是决不会不遇到困难的,如同我们在民主革命时期所曾经遇到过的许多困难那样,也许还会要遇到比过去更大的困难。但是,同志们,我们共产党人是以不怕困难著名的。我们在战术上必须重视一切困难。对于每一个具体的困难,我们都要采取认真对待的态度,创造必要的条件,讲究对付的方法,一个一个地、一批一批地将它们克服下去。根据我们几十年的经验,我们遇到的每一个困难,果然都被克服下去了。种种困难,遇到共产党人,它们就只好退却,真是“高山也要低头,河水也要让路”。这里就得出一条经验,它叫我们可以藐视困难。这说的是在战略方面,是在总的方面。不管任何巨大的困难,我们一眼就看透了它的底子。所谓困难,无非是社会的敌人和自然界给予我们的。我们知道,帝国主义、国内反革命分子以及他们在我们党内的代理人,等等,都不过是垂死的力量,而我们则是新生的力量,真理是在我们方面。对于他们,我们从来就是不可战胜的。只要想一想我们自己的历史,就会懂得这个道理。我们在一九二一年刚刚建党的时候,只有几十个人,那样渺小,后来发展起来,居然把国内的强大敌人给打倒了。自然界这个敌人也是有办法制服它的。不论在自然界和在社会上,一切新生力量,就其性质来说,从来就是不可战胜的。而一切旧势力,不管它们的数量如何庞大,总是要被消灭的。因此,我们可以藐视而且必须藐视人世遭逢的任何巨大的困难,把它们放在“不在话下”的位置。这就是我们的乐观主义。这种乐观主义是有科学根据的。只要我们更多地懂得马克思列宁主义,更多地懂得自然科学,一句话,更多地懂得客观世界的规律,少犯主观主义错误,我们的革命工作和建设工作,是一定能够达到目的的。

\date{一九五五年三月三十一日}
\section{结论}

\mxname{同志们:}

大家发言已经完了。我就下面的几个问题讲几句话:关于这次会议的评价,五年计划,高饶问题,目前形势,八次大会。

\subsection{一 关于这次代表会议的评价}

绝大多数同志认为,这次会议开得很好,是从延安整风以来的又一次整风会议,发扬了民主,开展了批评与自我批评,使得我们互相了解更多了,思想更加统一了,使得我们有了共同的认识。本来我们是有共同认识的,但是在若干问题上,我们中间还是有不同意见的,经过这一次会议,统一了我们的认识。在这个基础上,在这个思想的、政治的以及许多政策的共同认识的基础上,就可以使我们党更好地团结起来了。正如恩来同志所说,如果说党的第七次代表大会同它以前一个时期全党的思想、政治上的整风,奠定了我们党的统一思想的基础,在这个基础上取得了反对帝国主义、封建主义和官僚资本主义这种民主革命的胜利,那末,这一次会议就会使我们取得社会主义的胜利。

这次会议证明,我们党的水平是大为提高了,不但比十年前的七次大会时期大进了一步,而且比一九四九年的二中全会、一九五〇年的三中全会时期大进了一步。这个情况是好的,这次会议表明我们是进步了的。

我们进入了这样一个时期,就是我们现在所从事的、所思考的、所钻研的,是钻社会主义工业化,钻社会主义改造,钻现代化的国防,并且开始要钻原子能这样的历史的新时期。全党同志钻得有深有浅,在座的同志也是这样。像医生一样,有的能够开刀,有的不行。有的打针能够打静脉,有的就不能,只能打皮下。有一些医生连皮下都不敢动手,就在那个皮上面。虽然有些同志没有钻进去,但大多数同志是在钻,看样子有许多人是钻进去了,就是有一点内行的味道了。在这一次会议上,我们也可以看到这样的情况。这是极大的好事。因为现在我们面临的是新问题:社会主义工业化、社会主义改造、新的国防、其它各方面的新的工作。适合这种新的情况钻进去,成为内行,这是我们的任务。所以必须对那些钻不进去的人、浮在皮面上的人进行教育,使他们都成为内行。

反对高岗、饶漱石反党联盟的斗争,将促使我们党大进一步。

我们要在党内外五百万知识分子和各级干部中,宣传并使他们获得辩证唯物论,反对唯心论,我们将会组成一支强大的理论队伍,而这是我们极为需要的,这又是一件大好事。

我们要作出计划,组成这么一支强大的理论队伍,有几百万人读马克思主义的理论基础,即辩证唯物论和历史唯物论,反对各种唯心论和机械唯物论。我们现在有许多做理论工作的干部,但还没有组成理论队伍,尤其是还没有强大的理论队伍。而没有这支队伍,对我们全党的事业,对我国的社会主义工业化、社会主义改造、现代化国防、原子能的研究,是不行的,是不能解决问题的。因此,我劝同志们要学哲学。有相当多的人,对哲学没有兴趣,他们没有学哲学的习惯。可以先看小册子、短篇文章,从那里引起兴趣,然后再看七八万字的,然后再看那个几十万字一本的书。马克思主义有几门学问:马克思主义的哲学,马克思主义的经济学,马克思主义的社会主义——阶级斗争学说,但基础的东西是马克思主义哲学。这个东西没有学通,我们就没有共同的语言,没有共同的方法,扯了许多皮,还扯不清楚。有了辩证唯物论的思想,就省得许多事,也少犯许多错误。

\subsection{二 关于第一个五年计划}

同志们认为,在讨论五年计划的时候,大多数同志的发言很好,大家是满意的。其中有一部分发言特别好,他们讲透了问题,有点专家的味道了。但是,中央各部门的发言中间,有一部分内容较差,分析和批判不够;地方同志发言中间,也有一部分是较差的,分析和批判是不够的。另有一种情况,就是在有些同志发言中间,对严重的浪费问题以及别的错误,只是揭露了现状,没有说明如何处理。对于这些发言,有些同志不满意。我以为这些不满意是有理由的。

我希望,所有的省委书记、市委书记、地委书记以及中央各部门的负责同志,都要奋发努力,在提高马克思列宁主义水平的基础上,使自己成为精通政治工作和经济工作的专家。一方面要搞好政治思想工作,一方面要搞好经济建设。对于经济建设,我们要真正学懂。

这次会上,地方要求中央解决的许多问题,凡是中央已经有了规定的,应当积极解决。其它的问题,由秘书处会同提议的同志,研究解决办法,报告中央处理。

中央各部门要求地方协作的事也不少。中央部门在各地方办的事业,要请地方党委给以监督和帮助,特别是在政治思想工作方面。地方党委有责任帮助中央在地方所办的事业去完成任务。所以,不仅地方对中央有要求,中央对地方也有要求。只有中央各部门和地方党委齐心协力,分工合作,第一个五年计划才能够完满地实现。

\subsection{三 关于高岗、饶漱石反党联盟}

第一点,有人问:究竟有没有这个联盟?或者不是联盟,而是两个独立国,两个单干户?有的同志说,没有看到文件,他们是联盟总得有一个协定,协定要有文字。文字协定那的确是没有,找不到。我们说,高岗、饶漱石是有一个联盟的。这是从一些什么地方看出来的呢?一、是从财经会议期间高岗、饶漱石的共同活动看出来的。二、是从组织会议期间饶漱石同张秀山配合进行反党活动看出来的。三、是从饶漱石的话里看出来的。饶漱石说,“今后中央组织部要以郭峰为核心”。组织部是饶漱石为部长,高岗的心腹郭峰去作核心。那很好嘛!团结得很密切嘛!四、是从高岗、饶漱石到处散布安子文私拟的一个政治局委员名单这件事看出来的。在这件事上,安子文是受了警告处分的。高岗、饶漱石等人把这个名单散布给所有参加组织会议的人,而且散布到南方各省,到处这么散布,居心何在?五、是从高岗两次向我表示保护饶漱石,饶漱石则到最后还要保护高岗这件事看出来的。高岗说饶漱石现在不得了了,要我来解围。我说,你为什么代表饶漱石说话?我在北京,饶漱石也在北京,他为什么要你代表,不直接来找我呢?在西藏还可以打电报嘛,就在北京嘛,他有脚嘛。第二次是在揭露高岗的前一天,高岗还表示要保护饶漱石。饶漱石直到最后还要保护高岗,他要给高岗申冤。在揭露高岗的中央会议上,我说,北京有两个司令部,一个是以我为首的司令部,就是刮阳风,烧阳火,一个是以别人为司令的司令部,叫做刮阴风,烧阴火,一股地下水。究竟是政出一门,还是政出多门?从上面这许多事看来,他们是有一个反党联盟的,不是两个互不相关的独立国和单干户。

至于说,因为没有明文协定,有的同志就发生疑问,说恐怕不是联盟吧。这是把阴谋分子组成的反党联盟同一般公开的正式的政治联盟和经济联盟等同起来了,看作一样的事情了。他们是搞阴谋嘛!搞阴谋,还要订个文字协定吗?如果说,没有文字协定就不是联盟,那末高岗、饶漱石两个反党集团内部怎么办呢?高岗跟张秀山、张明远、赵德尊、马洪、郭峰之间,也没有订条约嘛!我们也没有看见他们的文字协定嘛!那末连他们这个反党集团也否定了!还有饶漱石跟向明、扬帆之间,也没有看见他们的条约嘛!所以,说没有明文协定就不能认为是联盟,这种意见是不对的。

第二点,受高、饶影响的同志和没有受他们影响的同志,各自应当采取什么态度?受影响的,有浅有深。有些是一般性的,被他们扫了一翅膀;少数几位同志是比较深的,同他们谈了许多问题,在下面有所活动,替他们传播。这两者是有区别的。但是,所有这些人,不管有浅有深,大多数同志在这个会议上都已经表示了态度。有的表示得很好,受到全场的欢迎。有的表示得还好,受到大部分同志的欢迎,但是有缺点。有的表示得不够充分,今天作了补充。有的全文讲得还好,但是有某些部分不妥当。不管怎么样,这几种人总之已经有所表示了,我们应当一律表示欢迎,总算有所表示嘛!还有个别同志要求发言,没有来得及,他们可以用书面向中央写一个报告。还没有讲的人,问题不严重,就是被扫了一翅膀的,知道一些事情,他没有讲。至于已经发了言的人,是不是也还有一些是留了尾巴的?那末现在我们决定,不论是关于五年计划还是关于高饶反党联盟问题,所有的发言、报告,都可以拿回去修改,字斟句酌,在五天之内,把那些没有讲完全的,或讲得不妥的,再加以修改。不要因为在这一次会议上没有讲妥,我们就抓住他的小辫子,将来使他下不去。你还可以修改,以你最后修改的稿子为准。

对这些同志,我们应当采取这样的态度,就是希望他们改正错误,对他们不但要看,而且要帮。就是讲,不但要看他们改不改,而且要帮他们改。人是要有帮助的。荷花虽好,也要绿叶扶持。一个篱笆要打三个桩,一个好汉要有三个帮。单干是不好的,总是要有人帮,在这样的问题上尤其要有人帮。看是要看的,看他们改不改,但单是看是消极的,还要帮助他们。对受了影响的人,不管有深有浅,我们一律欢迎他们改正,不但要看,而且要帮。这就是对待犯错误同志的积极态度。

没有受影响的同志,不要骄傲,谨防害病。这一点极为重要。前面讲的那些同志,可能有些是上当,有些是陷进去比较深,因为犯了错误,他们可能有所警觉,以后不再犯这类错误。害了一次病,取得了免疫力。种了一次牛痘,起预防作用。但是也不能保险,还可能害天花。所以,最好是三年五年再种一次牛痘,就是开我们这种会。其它的同志就不要骄傲,谨防犯错误。高岗、饶漱石为什么对这些人没有惹呢?有几种情形:第一种是他们认为是他们的敌人的人,当然不去传播;第二种是他们看不起的人,认为无足轻重,现在不必去传播,将来“天下大定”了,那些人自然跟着过来的;第三种是他们不敢惹的人,那些人大概是免疫力比较强,他们一看就不对头,虽然这些同志并不被认为是他们的敌人,也不是无足轻重,但是他们不敢去惹;第四种,就是时间来不及。这个瘟疫散播也要时间,再有一年的工夫不揭露,有些人那就难保。所以不要逞英雄:你看,你们不是惹了一点骚吗,而我可干净啦!再有一年不揭露,保管有不少的人是要受他们影响的。

我认为,以上就是受高、饶影响的和没有受影响的两部分同志应当注意的地方。

第三点,在原则性的问题上,在同志之间,对于违反党的原则的言论、行动,应当经常注意保持一个距离。他们那些话,他们那些行动,不符合党的原则,我们又看不惯,在这一部分问题、这一部分情况上,就不要打成一片。对其它的问题,符合党的原则的,比如五年计划,关于高饶反党联盟的决议、报告,以及各种正确的政策,正确的党内法规,这样一些言论、行动,当然要积极支持,打成一片。对不符合党的原则的,就应当保持一个距离,就是说,要划清界限,立即挡回去。不能因为是老朋友,老上司,老部下,老同事,同学,同乡等而废去这个距离。在这次高饶反党事件中,以及在过去党内的路线斗争中,都有过许多这样的经验:只要你以为关系太老了,太深了,不好讲,不保持一个距离,不挡回去,不划清界限,你就越陷越深,他们那个“鬼”就要缠住你。所以,应当表示态度,应当坚持原则。

第四点,有些同志说,“知道高、饶一些坏事情,但是没有看出他们的阴谋”。我说分两种情况:一种是听到高岗、饶漱石讲了许多不符合党的原则的话,甚至高岗、饶漱石还同他们商量过一些反党活动问题,那就应当看出来。一种是普通知道他们一些坏事情,而没有看出阴谋,这是难怪的,那是很难看出的。中央也是到了一九五三年才发现他们的反党阴谋。经过财经会议、组织会议,以及财经会议以前的种种问题,看到他们不正常。财经会议期间,发现了他们的不正常活动,每一次都给他们顶了回去。所以,以后他们就完全转入秘密了。对这个阴谋、阴谋家、阴谋集团,我们是到一九五三年秋冬才发现的。对于高岗、饶漱石,长期没有看出他们是坏人。这种事情过去也有过。井冈山时期有几个叛变分子,我们就从来没有想到他们会叛变。恐怕你们各位都有这种经验。

我们应当从这里得出一条经验,就是不要被假象所迷惑。我们有的同志容易被假象所迷惑。一切事物,它的现象同它的本质之间是有矛盾的。人们必须通过对现象的分析和研究,才能了解到事物的本质,因此需要有科学。不然,用直觉一看就看出本质来,还要科学干什么?还要研究干什么?所以要研究,就是因为现象同本质之间有矛盾。但假象跟一般现象有区别,因为它是假象。所以得出一条经验,就是尽可能不要被假象所迷惑。

第五点,骄傲情绪的危险。不要逞英雄。事业是多数人做的,少数人的作用是有限的。应当承认少数人的作用,就是领导者、干部的作用,但是,没有什么了不起的作用,有了不起的作用的还是群众。干部与群众的正确关系是,没有干部也不行,但是,事情是广大群众做的,干部起一种领导作用,不要夸大干部的这种作用。没有你就不得了吗?历史证明,各种事实证明,没有你也行。比如没有高岗、饶漱石是不是不得了呢?那还不是也行吗!没有托洛茨基,没有张国焘,没有陈独秀,还不是也行吗!这些都是坏人。至于好人呢,没有你也可以。没有你,地球就不转了吗?地球还是照样地转,事业还是照样地进行,也许还要进行得好些。孔夫子早已没有了,我们中国有了共产党,总比孔夫子高明一点吧,可见没有孔夫子事情还做得好一些嘛!

有两种人:一种是老资格,在座的不少,资格很老;一种是新生力量,这是年轻的人。这两种人中间哪一种人更有希望呢?恩来同志今天也讲了这个问题,当然是新生力量更有希望。有些同志,因为自己是老革命,就骄傲起来,这是很不应当的。比较起来,如果说允许骄傲的话,倒是青年人值得骄傲一下。四五十岁以上的人,年纪越大,经验越多,就应当更谦虚。让青年人看到我们确实是有经验的:“这些前辈,确是有点经验,不要看轻他们,你看他们那么谦虚。”四五十岁的人了,因为得了许多经验,反倒骄傲起来,那岂不是不像样子?青年人就要发议论:“你们那些经验就等于没有,还不是跟小孩子一样。”小孩子有点骄傲情绪,那是比较合理的。上了年纪的人,有了这么多经验,还骄傲,把尾巴翘得那样高,可以不必。俗语说:“夹紧尾巴做人。”人本来是没有尾巴的,为什么要夹紧尾巴呢?好比那个狗,有翘尾巴的时候,有夹尾巴的时候。大概是打了几棍子的时候它就夹紧尾巴,大概是有了几批成绩的时候它的尾巴就翘起来了。我希望,我们所有的同志,首先是老同志,不要翘尾巴,而要夹紧尾巴,戒骄戒躁,永远保持谦虚进取的精神。

第六点,戒“左”戒右。有人说,“‘左’比右好”,许多同志都这么说。其实,也有许多人在心里说,“右比‘左’好”,但不讲出来,只有诚实的人才讲出来。有这么两种意见。什么叫“左”?超过时代,超过当前的情况,在方针政策上、在行动上冒进,在斗争的问题上、在发生争论的问题上乱斗,这是“左”,这个不好。落在时代的后面,落在当前情况的后面,缺乏斗争性,这是右,这个也不好。我们党内不但有喜欢“左”的,也有不少喜欢右的,或者中间偏右,都是不好的。我们要进行两条战线的斗争,既反对“左”,也反对右。

关于高岗、饶漱石反党联盟问题,我就讲这么几点意见。

\subsection{四 关于目前形势}

国际形势,国内形势,党内形势,这三种形势怎样呢?是光明面占优势,还是黑暗面占优势呢?应当肯定,不论国际、国内、党内形势,都是光明面占优势,黑暗面占劣势。在我们这个会场上也是如此。不要因为有很多人作了自我批评,以为就黑暗了。这些同志是着重讲了他们的缺点错误,没有讲他们的长处,哪一年参加革命,哪里又打了胜仗,哪里有工作成绩,这些话都没有讲。专看他们这一篇检讨,那末就黑暗了。其实,这是一面,在很多同志身上,这是次要的一面。这跟高岗、饶漱石和张、张、赵、马、郭五虎将不同,他们不能适用这个估计:光明面占优势。高岗有什么光明面占优势呀?他是全部黑暗,天昏地黑,日月无光。至于我们的同志那就不同,略有黑暗,这个东西可以洗干净,用肥皂多洗几回。

为什么要提出准备对付突然事变,准备对付反革命复辟,准备对付高饶事件的重复发生呢?这是说,从最坏的可能性着想,总不吃亏。不论任何工作,我们都要从最坏的可能性来想,来部署。无非是这些坏得不得了的事:帝国主义者发动新的世界大战,蒋介石又来坐北京,高饶反党联盟一类的事件重新发生,而且不只一个,而是十个,一百个。尽管有那么多,我们都先准备好了,就不怕了。你有十个,也只有五双,没有什么了不起,我们都估计到了。帝国主义拿来吓唬我们的原子弹和氢弹,也没有什么可怕。世界上的事情,总是一物降一物,有一个东西进攻,也有一个东西降它。看《封神榜》就知道,哪有一个“法宝”是不能破的呀?那样多的“法宝”都破了。我们相信,只要依靠人民,世界上就没有攻不破的“法宝”。

\subsection{五 为胜利召开党的第八次全国代表大会而斗争}

中央决定一九五六年下半年,召开党的第八次全国代表大会。有三个议事日程:(一)中央委员会的工作报告;(二)修改党章;(三)选举新的中央委员会。明年七月以前要完成代表的选举及文件的准备工作。要求在这一年多的时间内,各方面的工作,经济、文教、军事、党务、政治思想、群众团体、统一战线、少数民族工作,都要大进一步。

我顺便讲一讲少数民族工作的问题。要反对大汉族主义。不要以为只是汉族帮助了少数民族,而少数民族也很大地帮助了汉族。有些同志总是在那里吹,我们可帮助了你们,就没有看到没有少数民族是不行的。我国百分之五十到百分之六十的地方,是什么人住的?是汉族住的,还是什么人住的?百分之五十到百分之六十的地方是少数民族居住的。那里物产丰富,有很多宝贝。现在,我们帮助少数民族很少,有些地方还没有帮助,而少数民族倒是帮助了汉族。有些少数民族,需要我们先去帮助他们,然后他们才能帮助我们。少数民族在政治上很大地帮助了汉族,他们加入了中华民族这个大家庭,就是在政治上帮助了汉族。少数民族和汉族团结在一起了,全国人民都高兴。所以,少数民族在政治上、经济上、国防上,都对整个国家、整个中华民族有很大的帮助。那种以为只有汉族帮助了少数民族,少数民族没有帮助汉族,以及那种帮助了一点少数民族,就自以为了不起的观点,是错误的。

我们讲在这一年中各方面的工作要大进一步,就是要把已经揭露出来的缺点、错误加以改正。不要在这次会上许了愿,到明年召开“八大”时还照样有那么多缺点、错误,原封未动。所以要为召开“八大”而斗争者,就是要把缺点、错误改正,比如铺张浪费、大屋顶这样一些东西,认真地负责地改一下。不要在这里许了愿,一回去大家就把两脚一伸睡起觉来了。

有人建议一年或者两年开一次这样的会议,使同志之间互相监督,我认为可以考虑。谁监督我们这些人呢?互相监督是好办法,可以促进党和国家的事业迅速进步。是迅速进步,不是慢慢地进步。党的代表大会,十年没有开了。当然头五年不应当开,因为兵荒马乱,又开了“七大”,后五年可以开而没有开。没有开也有好处:高饶问题搞清楚再开,不然他们要利用“八大”大做文章。同时,我们的五年计划也上了轨道,过渡时期总路线也提出来了,又经过这次代表会议使大家在思想上更加统一了,为召开党的第八次代表大会准备了条件。在党的第八次代表大会上,不要每个人去检讨一篇,但对我们工作中的缺点和错误,还是要作公开的批评和自我批评。不实行马克思主义的这一条是不行的。

批评要尖锐。这次有些批评,我觉得不那么尖锐,总是怕得罪人的样子。你不那样尖锐,不切实刺一下,他就不痛,他就不注意。要有名有姓,哪一个部门,要指出来。你没有搞好,我是不满意的,得罪了你就得罪了你。怕得罪人,无非是怕丧失选举票,还怕工作上不好相处。你不投我的票,我就吃不了饭?没有那回事。其实,你讲出来了,把问题尖锐地摆在桌面上,倒是好相处了。不要把棱角磨掉。牛为什么要长两只角呢?牛之所以长两只角,是因为要斗争,一为防御,二为进攻。我常跟同志讲,你头上长“角”没有?你们各位同志可以摸一摸。我看有些同志是长了“角”的,有些同志长了“角”但不那样尖锐,还有些同志根本没有长“角”。我看,还是长两只“角”好,因为这是合乎马克思主义的。马克思主义有一条,叫做批评和自我批评。

所以,定期召开会议,进行批评和自我批评,这是一种同志间互相监督,促使党和国家事业迅速进步的好办法。建议各省、市委同志们考虑,你们是不是也可以这样做?你们不是学中央吗?我看这一点是可以学的。

最后,我请同志们注意,也请全党同志注意:

为在一九五六年胜利地召开党的第八次全国代表大会而斗争!

为胜利地完成第一个五年计划而斗争!
