
\title{中共中央关于同国民党进行和平谈判的通知}
\date{一九四五年八月二十六日}
\thanks{这是毛泽东在去重庆同蒋介石进行和平谈判的前两天为中共中央起草的对党内的通知。由于中国共产党和中国广大人民坚定地反对蒋介石的内战阴谋,也由于美国帝国主义当时还顾忌世界民主舆论一致反对蒋介石的内战政策和独裁政策,蒋介石在一九四五年八月十四日、二十日和二十三日三次电邀毛泽东到重庆进行和平谈判,当时美国驻中国大使赫尔利还在八月二十七日为此来到延安。中国共产党为了尽一切可能争取和平,也为了在争取和平的过程中揭露美国帝国主义和蒋介石的真面目,以利于团结和教育广大人民,决定派遣毛泽东、周恩来、王若飞到重庆去同国民党进行和平谈判。毛泽东所起草的这个通知,分析了日本宣布投降以后两个星期内中国形势的发展,说明了中共中央关于和平谈判的方针,在谈判中准备作出的某些让步以及对谈判结果的两种可能情况的对策,分别对华北、华东解放区和华中、华南解放区的斗争作了原则的指示,告诉全党绝对不要因为谈判而放松对蒋介石的警惕和斗争。毛泽东等在八月二十八日到重庆,同国民党进行了四十三天的谈判。这次谈判的结果,虽然只发表了一个国共双方代表会谈纪要(即《双十协定》),但是在政治上却使中国共产党获得了极大的主动,而使国民党陷入被动,因而是成功的。十月十一日,毛泽东回到延安。周恩来、王若飞仍在重庆继续谈判。关于这次谈判的结果,见本卷\mxart{关于重庆谈判}一文。}
\maketitle


日寇迅速投降,改变了整个形势。蒋介石垄断了受降权利,大城要道暂时(一个阶段内)不能属于我们。但是华北方面,我们还要力争,凡能争得者应用全力争之。两星期来,我军收复大小五十九个城市和广大乡村,连以前所有,共有城市一百七十五个,获得了伟大的胜利。华北方面,收复了威海卫、烟台、龙口、益都、淄川、杨柳青、毕克齐、博爱、张家口、集宁、丰镇等处,我军威震华北,配合苏军和蒙古军进抵长城之声势,造成了我党的有利地位。今后一时期内仍应继续攻势,以期尽可能夺取平绥线、同蒲北段、正太路、德石路、白晋路\mnote{1}、道清路,切断北宁、平汉、津浦、胶济、陇海、沪宁各路,凡能控制者均控制之,哪怕暂时也好。同时以必要力量,尽量广占乡村和府城县城小市镇。例如新四军占领了南京、太湖、天目山之间许多县城和江淮间许多县城,山东占领了整个胶东半岛,晋绥占领了平绥路南北许多城市,就造成了极好的形势。再有一时期攻势,我党可能控制江北、淮北、山东、河北、山西、绥远\mnote{2}的绝对大部分,热察\mnote{3}两个全省和辽宁一部。

现在苏美英三国均不赞成中国内战\mnote{4},我党又提出和平、民主、团结三大口号\mnote{5},并派毛泽东、周恩来、王若飞三同志赴渝和蒋介石商量团结建国大计,中国反动派的内战阴谋,可能被挫折下去。国民党在取得沪宁等地、接通海洋和收缴敌械、收编伪军之后,较之过去加强了它的地位,但是仍然百孔千疮,内部矛盾甚多,困难甚大。在内外压力下,可能在谈判后,有条件地承认我党地位,我党亦有条件地承认国民党的地位,造成两党合作(加上民主同盟\mnote{6}等)、和平发展的新阶段。假如此种局面出现之后,我党应当努力学会合法斗争的一切方法,加紧国民党区域城市、农村、军队三大工作(均是我之弱点)。在谈判中,国民党必定要求我方大大缩小解放区的土地和解放军的数量,并不许发纸币,我方亦准备给以必要的不伤害人民根本利益的让步。无此让步,不能击破国民党的内战阴谋,不能取得政治上的主动地位,不能取得国际舆论和国内中间派的同情,不能换得我党的合法地位和和平局面。但是让步是有限度的,以不伤害人民根本利益为原则。

在我党采取上述步骤后,如果国民党还要发动内战,它就在全国全世界面前输了理,我党就有理由采取自卫战争,击破其进攻。同时我党力量强大,有来犯者,只要好打,我党必定站在自卫立场上坚决彻底干净全部消灭之(不要轻易打,打则必胜),绝对不要被反动派的其势汹汹所吓倒。但是不论何时,又团结,又斗争,以斗争之手段,达团结之目的;有理有利有节;利用矛盾,争取多数,反对少数,各个击破等项原则\mnote{7},必须坚持,不可忘记。

在广东、湖南、湖北、河南等省的我党力量比华北、江淮所处地位较为困难,中央对于这些地方的同志们深为关怀。但是国民党空隙甚多,地区甚广,只要同志们对于军事政策(行动和作战)和团结人民的政策,不犯大错误,谦虚谨慎,不骄不躁,是完全有办法的。除中央给予必要的指示外,这些地方的同志必须独立地分析环境,解决问题,冲破困难,获得生存和发展。待到国民党对于你们无可奈何的时候,可能在两党谈判中被迫承认你们的力量,而允许作有利于双方的处置。但是你们绝对不要依靠谈判,绝对不要希望国民党发善心,它是不会发善心的。必须依靠自己手里的力量,行动指导上的正确,党内兄弟一样的团结和对人民有良好的关系。坚决依靠人民,就是你们的出路。

总之,我党面前困难甚多,不可忽视,全党同志必须作充分的精神准备。但是整个国际国内大势有利于我党和人民,只要全党能团结一致,是能逐步地战胜各种困难的。


\begin{maonote}
\mnitem{1}白晋路,指当时山西省东南部由祁县的白圭到晋城的一条未完成的铁路。
\mnitem{2}绥远,原来是一个省,一九五四年撤销,原辖地区划归内蒙古自治区。
\mnitem{3}指热河、察哈尔,见本卷\mxnote{抗日战争胜利后的时局和我们的方针}{11}。
\mnitem{4}在日本投降前后的一个时期内,苏、美、英三国都表示不赞成中国发生内战。但是不久以后的事实证明,美国所谓不赞成中国内战的声明,只不过是它用来作为积极帮助国民党反动政府准备反革命内战的掩护而已。
\mnitem{5}和平、民主、团结三大口号,是一九四五年八月二十五日中国共产党中央委员会《对于目前时局的宣言》中提出的。宣言指出:在日本帝国主义投降以后,“我全民族面前的重大任务是:巩固国内团结,保证国内和平,实现民主,改善民生,以便在和平民主团结的基础上,实现全国的统一,建设独立自由与富强的新中国”。
\mnitem{6}见本书第三卷\mxnote{论联合政府}{16}。
\mnitem{7}参见本书第二卷\mxart{目前抗日统一战线中的策略问题}和\mxart{论政策}。
\end{maonote}
