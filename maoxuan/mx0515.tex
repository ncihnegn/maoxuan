
\title{应当重视电影《武训传》的讨论}
\date{一九五一年五月二十日}
\thanks{这是毛泽东同志为《人民日报》写的社论的节录。}
\maketitle


《武训传》所提出的问题带有根本的性质。像武训那样的人,处在清朝末年中国人民反对外国侵略者和反对国内的反动封建统治者的伟大斗争的时代,根本不去触动封建经济基础及其上层建筑的一根毫毛,反而狂热地宣传封建文化,并为了取得自己所没有的宣传封建文化的地位,就对反动的封建统治者竭尽奴颜婢膝的能事,这种丑恶的行为,难道是我们所应当歌颂的吗?向着人民群众歌颂这种丑恶的行为,甚至打出“为人民服务”的革命旗号来歌颂,甚至用革命的农民斗争的失败作为反衬来歌颂,这难道是我们所能够容忍的吗?承认或者容忍这种歌颂,就是承认或者容忍污蔑农民革命斗争,污蔑中国历史,污蔑中国民族的反动宣传,就是把反动宣传认为正当的宣传。

电影《武训传》的出现,特别是对于武训和电影《武训传》的歌颂竟至如此之多,说明了我国文化界的思想混乱达到了何等的程度!

在许多作者看来,历史的发展不是以新事物代替旧事物,而是以种种努力去保持旧事物使它得免于死亡;不是以阶级斗争去推翻应当推翻的反动的封建统治者,而是像武训那样否定被压迫人民的阶级斗争,向反动的封建统治者投降。我们的作者们不去研究过去历史中压迫中国人民的敌人是些什么人,向这些敌人投降并为他们服务的人是否有值得称赞的地方。我们的作者们也不去研究自从一八四〇年鸦片战争以来的一百多年中,中国发生了一些什么向着旧的社会经济形态及其上层建筑(政治、文化等等)作斗争的新的社会经济形态,新的阶级力量,新的人物和新的思想,而去决定什么东西是应当称赞或歌颂的,什么东西是不应当称赞或歌颂的,什么东西是应当反对的。

特别值得注意的,是一些号称学得了马克思主义的共产党员。他们学得了社会发展史——历史唯物论,但是一遇到具体的历史事件,具体的历史人物(如像武训),具体的反历史的思想(如像电影《武训传》及其它关于武训的著作),就丧失了批判的能力,有些人则竟至向这种反动思想投降。资产阶级的反动思想侵入了战斗的共产党,这难道不是事实吗?一些共产党员自称已经学得的马克思主义,究竟跑到什么地方去了呢?

为了上述种种缘故,应当展开关于电影《武训传》及其它有关武训的著作和论文的讨论,求得彻底地澄清在这个问题上的混乱思想。
