
\title{征询对农业十七条的意见}
\date{一九五五年十二月二十一日}
\thanks{这是毛泽东同志为中共中央起草的给上海局,各省委、自治区党委的通知。}
\maketitle


今年十一月间毛泽东同志在杭州和天津分别同十四个省委书记和内蒙自治区党委书记共同商定的十七条,中央认为应当于一月十日中央召集的有各省委、市委、自治区党委书记参加的会议上,加以确定,以便纳入一九五六年的计划,认真开始实行。为此目的,请你们于接电后即召集所属各地委书记和一部分县委书记详细研究一下:(甲)究竟是否全部可以实现,还是有一部分不能实现,实现的根据是否每条都是充分的;(乙)除了十七条以外,是否还有增加(只要是可行的,可以增加);(丙)你们是否准备立即纳入你们的一九五六年计划内开始实行。以上各点,请你们于一九五六年一月三日以前研究完毕,准备意见。

十七条内容如下:

(一)农业合作化的进度,一九五六年下半年基本上完成初级形式的建社工作,省、市、自治区(除新疆外)一级的指标以要求完成百分之七十五的农户入社为宜,让下面超过一点,达到百分之八十至八十五左右。

合作化的高级形式,争取于一九六〇年基本上完成,是否可以缩短一年,争取于一九五九年基本上完成。为此,需要于一九五六年由县最好由区直接掌握每县或者每区办一个至几个大型(一百户以上的)高级社,再于一九五七年办一批,这两批应占农户百分之二十五左右,以为榜样。这样是否可能。又由小社变大社,规模如何。一乡几社,一乡一社,数乡一社,三者是否都可行。全国总社数三十万个,或者四十万个,或者五十万个,究以何者为宜。苏联是十万个社,我国是否以三十几万个社或者四十万个社为适宜。又先并社后升级为好,还是并社升级同时进行为好,还是先升级后并社为好。以上各点请你们一并加以研究。

(二)地主、富农入社,一九五六年是否即照安徽、山西、黑龙江等省的意见办理,即好的许其入社,不好不坏的许其在社生产,不给社员称号,坏的由社管制生产,凡干部强的老社均可这样做。这样做好处很多,但是有一个缺点就是势必逼使那些目前还不愿入社的上中农勉强入社,并且让他们先入社,然后再让地富入社,才使他们面子上过得去。这样是否有利。或者推迟一年,即到一九五七年才行上述办法。这二者哪一种有利些,请加研究。

(三)合作社领导成分,由现有的贫农和原来是贫农的全部新下中农占三分之二,老下中农和新老两部分上中农占三分之一。

(四)增产的条件:(甲)实行几项基本措施(内容待商,各地可以有些差别);(乙)推广先进经验(每年收集典型例子,每省印成一本)。

(五)一九五六年一切省、地、县、区、乡都要做出一个包括一切必要项目的全面长期计划,着重县、乡计划,于上半年完成初稿,下半年定稿,以后还可修改。计划包括的时间,至少三年,最好七年,可以到十二年。此事必须抓紧去做,你们是否已有布置。因为无经验,有许多可能是很粗糙的,但是必须争取少数县、乡的计划比较接近实际,以利推广。

(六)全面规划保护和繁殖牛、马、骡、驴、猪、羊、鸡、鸭,特别要保护幼畜。繁殖计划待商,请你们准备意见。

(七)同流域规划相结合,大量地兴修小型水利,保证在七年内基本上消灭普通的水灾旱灾。

(八)在七年内,基本上消灭十几种不利于农作物的虫害和病害。

(九)在十二年内,基本上消灭荒地荒山,在一切宅旁、村旁、路旁、水旁,以及荒地上荒山上,即在一切可能的地方,均要按规格种起树来,实行绿化。

(十)在十二年内,大部分地区百分之九十的肥料,一部分地区百分之百的肥料,由地方和合作社自己解决。

(十一)在十二年内,平均每亩粮食产量,在黄河、秦岭、白龙江、黄河(青海境内)以北,要求达到四百斤,黄河以南、淮河以北五百斤,淮河、秦岭、白龙江以南八百斤。棉花、油料、大豆、丝、茶、黄麻、甘蔗、水果等项指标,请你们提出计划数字,待商。

(十二)在七年内,基本上消灭若干种危害人民和牲畜最严重的疾病,例如血吸虫病、血丝虫病、鼠疫、脑炎、牛瘟、猪瘟等。请你们研究各省、区的地方病,哪些是七年内可以基本上消灭的,哪些是要延长时间才能消灭的,哪些是目前无法消灭的。

(十三)除四害,即在七年内基本上消灭老鼠(及其它害兽),麻雀(及其它害鸟,但乌鸦是否宜于消灭,尚待研究),苍蝇,蚊子\mnote{1}。

(十四)在七年内,基本上扫除文盲,每人必须认识一千五百到二千个字。

(十五)在七年内,将省、地、县、区、乡的各种必要的道路按规格修好(其中有些是公路,有些是大路,有些是小路)。

(十六)在七年内,建立有线广播网,使每个乡和每个合作社都能收听有线广播。

(十七)在七年内,完成乡和大型合作社的电话网。

以上各项,请你们和有关同志加以研究,于一月三日以前准备完毕。中央可能于一月四日左右先行邀集若干省委书记开会研究几天,为一月十日的会议准备意见。


\begin{maonote}
\mnitem{1}一九六〇年三月,毛泽东为中共中央起草的关于卫生工作的指示中,对除四害的内容作了改动,将麻雀换为臭虫。
\end{maonote}
