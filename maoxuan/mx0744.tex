
\title{医学教育革命的方向——赤脚医生就是好}
\date{一九六八年九月十四日}
\thanks{这是毛泽东同志对一份调查报告\mnote{1}的修改和批示。}
\maketitle


\section*{(一)}

这个从城里下到农村的医生证明,从旧学校培养的学生,多数或大多数是能够同工农兵结合的,有些人并有所发明、创造,不过要在正确路线领导之下,由工农兵给他们以再教育,彻底改变旧思想。这样的知识分子,工农兵是欢迎的。不信,请看上海川沙县江镇公社的那个医生。

\section*{(二)}

赤脚医生就是好。

\begin{maonote}
\mnitem{1}一九六八年夏天,上海《文汇报》刊载了一篇《从“赤脚医生”的成长看医学教育革命的方向》文章。同年第三期《红旗》杂志和九月十四日出版的《人民日报》都全文转载,也就是在这篇文章中,第一次把农村半医半农的卫生员正式称为“赤脚医生”,毛泽东在当天的《人民日报》上批示“赤脚医生就是好”。从此,“赤脚医生”成为半农半医的乡村医生的特定称谓。

“赤脚医生”这个名字,是在农民中自行叫起来的,因为南方的农村都是水田,种水稻的,只能赤脚下水田,所以赤脚就是劳动的意思,赤脚医生就是既要劳动也要行医。

这篇调查报告最后指出:

从“赤脚医生”的成长道路,可以看到医学院校的教育革命的一些问题。江镇公社的贫下中农体会到:要贯彻执行毛主席关于“把医疗卫生工作的重点放到农村去”的光辉指示,必须搞好医学院校的教育革命。他们学习了毛主席关于教育革命的一系列最新指示,认为:

“(一)医学教育必须为无产阶级政治服务。贫下中农谈起公社里两个同时担任“赤脚医生”的青年:一个后来被保送到嘉定半农半医医专学习(高中毕业才能入学)。这所学校是反革命修正主义分子杨西光抓的重点,向学生灌输了大量修正主义毒素。三年下来,这位“赤脚医生”成了白面书生,平时回家跟贫下中农都不大说话了。她不愿回本公社工作,更不愿回本大队当“赤脚医生”了。另一个只念过两年半书,经过三年的斗争锻炼,提高了政治觉悟。她更加热爱贫下中农,热爱农村医务工作,说:“我药箱里药用少了,比家里米缺了还着急。”在医疗技术上,后者也超过了前者。贫下中农感慨地说:“教育不革命,就算是我伲送去的囡,也要出修正主义!”他们认为,医学教育一定要无产阶级政治挂帅,要把“老三篇”和《实践论》、《矛盾论》作为必修课;要贯彻理论联系实践的原则,学生应该在阶级斗争、生产斗争、科学实验三大革命运动中进行学习。

(二)医学院校的招生对象主要应是“赤脚医生”和卫生员。毛主席最近指出,“要从有实践经验的工人农民中间选拔学生,到学校学几年以后,又回到生产实践中去。”贫下中农认为,“赤脚医生”在农村里滚上二、三年,再进医学院校学习,这个办法好。贫下中农高兴地说:“以后医学院校招生,我伲要把‘赤脚医生’送去,学几年后再回来为我伲服务。”江镇公社在各生产队选拔了一百四十四名不脱产的卫生员,主要由“赤脚医生”带训。带出来以后,“赤脚医生”多了,就可以抽出一部分人去医学院校学习一、二年或二、三年。

(三)坚持在普及的基础上提高。贫下中农认为,当前,医学最主要的是普及,也要提高,但必须沿着工农兵所需要的方向去提高。那些为极少数城市老爷太太服务的所谓“提高”,必须彻底批判。贫下中农还认为:“赤脚医生”中除了一部分进入学校学习外,绝大部分都应当坚持扎根农村,可以采用卫生院医生和“赤脚医生”上下定期对调的办法,加以提高。这样做,既有利于原来的医疗卫生队伍的改造,又有利于在实践中提高‘赤脚医生’。”

赤脚医生的兴起是与当时在全国普遍开展的农村合作医疗密不可分的,赤脚医生是合作医疗的忠实实践者。新中国合作医疗的典型是一个叫覃祥官的人在鄂西长阳土家山寨创造的,一九六六年八月十日,中国历史上第一个农村合作医疗试点——“乐园公社杜家村大队卫生室”挂牌了。覃祥官主动辞去公社卫生所的“铁饭碗”,在大队卫生室当起了记工分、吃农村口粮的“赤脚医生”。农民每人每年交一元合作医疗费,大队再从集体公益金中人均提留五角钱作为合作医疗基金。除个别老痼疾病需要常年吃药的以外,群众每次看病只交五分钱的挂号费,看病吃药就不要钱了。覃祥官首创的看病吃药不花钱的“农村合作医疗制度”,由于毛泽东亲笔批示和《人民日报》头版头条报道,该制度在全国百分之九十以上的农村推广,惠及亿万农民。

农村合作医疗体系在改善中国人民健康水平方面的贡献有目共睹:中国的平均寿命高于很多收入水准比中国高的国家,人民平均寿命的增长幅度在很大程度上超过了其他国家。世界卫生组织有感于这样伟大的成就,在一九七八年召开的阿马阿塔会议上,将中国的医疗卫生体制推崇为世界范围内基层卫生推动计划的模范。此外,中国独特的医疗卫生体系创建,深刻地影响了其他国家的医疗改革,启发那些改革者们发展适合自己的医疗卫生制度,而不是盲目照搬其他国家的制度。

二〇〇七年三月十一日,全国政协十届五次会议在北京人民大会堂举行第三次全体会议。来自中国工程院的院士有钟南山、巴德年、李连达、高润霖、程书钧、王红阳、刘志红,来自中国科学院的院士有杨雄里和强伯勤,他们由巴德年代表做了《加快覆盖城乡居民医疗保健制度的建设及早解决“看病贵、看病难”问题》的联合发言:

“我代表长期工作在医药卫生战线上的中国科学院、中国工程院的九位院士委员作联合发言。

新中国成立之后,我国的卫生事业得到空前发展,许多传染病得到控制,性病被杜绝,人均寿命、婴幼儿死亡率等指标都有了明显改善。世界卫生组织、世界银行等机构赞誉中国只用了世界上1\%的卫生资源,解决了世界22\%人口的卫生保健问题。而到了2000年,世界卫生组织在对191个国家进行评价时,中国的医药卫生总体水平被排在第144位,而卫生公平性竟排在第188位,列倒数第4位。

究其原因,主要是在社会主义市场经济体制下,淡忘了医药卫生事业的公益性质,忽略了“以人为本,健康第一”的理念。90年代我国政府向世界宣布的到2000年“人人享有卫生保健”的承诺,没有兑现。

世界上无论发达国家还是发展中国家,都把卫生投入列入国家财政支出的重要科目,姑且不说发达国家用于医药卫生开支均占GDP的10\%以上,就连巴西也为7.9\%,印度为6.1\%,赞比亚为5.8\%,中国只为2.7\%。而且,中国政府的卫生投入在整个医药卫生总支出的比例,也逐年减少。

多年来,我国某些部门以“中国国情”为由,宣称中国不会走国外全民医疗的老路,要走一条自己的“改革路”,走的结果是走到了第188位,走到了老百姓极不满意、无法承受的地步。”
\end{maonote}
