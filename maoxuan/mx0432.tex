
\title{关于民族资产阶级和开明绅士问题}
\date{一九四八年三月一日}
\thanks{这是毛泽东为中共中央起草的对党内的指示。}
\maketitle


中国现阶段革命的性质,是无产阶级领导的、人民大众的、反对帝国主义、反对封建主义和反对官僚资本主义的革命。所谓人民大众,是指一切被帝国主义、封建主义、官僚资本主义所压迫、损害或限制的人们,也即是一九四七年十月中国人民解放军宣言上明确地指出的工、农、兵、学、商和其它一切爱国人士\mnote{1}。在宣言上所说的“学”,即是指一切受迫害、受限制的知识分子。所说的“商”,即是指一切受迫害、受限制的民族资产阶级,即中小资产阶级。所说的“其它爱国人士”,则主要地是指的开明绅士。现阶段的中国革命,即是由这些人们团结起来,组成反帝、反封建、反官僚资本主义的统一战线,而又以劳动人民为主体的革命。所谓劳动人民,是指一切体力劳动者(如工人、农民、手工业者等)以及和体力劳动者相近的、不剥削人而又受人剥削的脑力劳动者。中国现阶段革命的目的,是在推翻帝国主义、封建主义、官僚资本主义的统治,建立一个以劳动者为主体的、人民大众的新民主主义共和国,不是一般地消灭资本主义。

我们不要抛弃那些过去和我们合作过、现在也还同我们合作、赞成反美蒋和土地改革的开明绅士。例如晋绥边区的刘少白、陕甘宁边区的李鼎铭\mnote{2}等人,在抗日战争和抗日战争以后的困难时期内,曾经给我们以相当的帮助,而在我们实行土地改革的时候,他们又并不妨碍和反对土地改革,因此对他们仍应采取团结的政策。但是团结他们,并不是说将他们当作决定中国革命性质的力量来看。决定革命性质的力量,是主要的敌人和主要的革命者两方面。我们今天的主要敌人是帝国主义、封建主义和官僚资本主义,我们今天同敌人作斗争的主要力量是占全国人口百分之九十的一切从事体力劳动和脑力劳动的人民。这就决定了我们现阶段革命的性质是新民主主义的人民民主革命,而不同于十月革命那样的社会主义革命。

依附帝国主义、封建主义、官僚资本主义,反对人民民主革命的民族资产阶级的少数右翼分子,他们也是革命的敌人;依附劳动人民反对反动派的民族资产阶级左翼分子以及从封建阶级分裂出来的少数开明绅士,他们也是革命者。但是这两者都不是敌人或革命者的主体,两者都不是可以决定革命性质的力量。民族资产阶级是一个在政治上非常软弱的和动摇的阶级。但是他们中间的大多数,由于也受着帝国主义、封建主义、官僚资本主义的迫害和限制,他们又可以参加人民民主革命,或者对革命守中立。他们是人民大众的一部分,但不是人民大众的主体,也不是决定革命性质的力量。但是因为他们在经济上具有重要性,又因为他们可以参加反对美蒋,或者在反对美蒋的斗争中采取中立的态度,因之我们便有可能和必要去团结他们。在中国共产党未产生以前,以孙中山为领导的国民党,曾经代表民族资产阶级,充当过当时中国革命(不彻底的旧民主主义革命)的领导者。但是,中国共产党一经产生,并且表现出自己的能力以后,他们就已经不能是中国革命(新民主主义革命)的领导者了。这个阶级曾经参加了一九二四年到一九二七年的革命运动,而在一九二七年到一九三一年(九一八事变以前),他们中间的不少分子,曾经附和了蒋介石的反动。但是,决不能因为这一点,就认为那个时期我们在政治上不应该争取他们,在经济上不应该保护他们;就认为我们在那个时期内对民族资产阶级的过左的政策不是冒险主义的政策。相反地,那时我们的政策,仍然应当是保护和争取他们,以便我们能够集中力量去反对主要敌人。在抗日时期,民族资产阶级是动摇于国民党和共产党两党之间的抗日的参加者。在现阶段,民族资产阶级的多数是增长了对美蒋的仇恨,他们中间的左翼分子依附于共产党,右翼分子则依附于国民党,其中间派则在国共两党之间采取犹豫和观望的态度。这种情况,使得我们有必要和可能争取其大多数,孤立其少数。为了达到这一目的,对这个阶级的经济地位必须慎重地加以处理,必须在原则上采取一律保护的政策。否则,我们便要在政治上犯错误。

开明绅士是地主和富农阶级中带有民主色彩的个别人士。这些人士,同官僚资本主义和帝国主义有矛盾,同封建的地主、富农也有某种矛盾。我们团结他们,并不是因为他们在政治上有什么大的力量,也不是因为他们在经济上有什么重要性(他们根据封建制度占有的土地,应当在取得他们同意之后交给农民分配),而是因为他们在抗日战争时期,在反美蒋斗争时期,在政治上曾经给我们以相当的帮助。在土地改革时期,如果有少数开明绅士表示赞成我们的土地改革,对于全国土地改革的工作也是有益的。特别是对于争取全国的知识分子(中国的知识分子大部分是地主富农的家庭出身),对于争取全国的民族资产阶级(中国的民族资产阶级大部分同土地有联系),对于争取全国的开明绅士(大约有几十万人),以及对于孤立中国革命的主要敌人蒋介石反动派,都是有益的。正因为开明绅士有这些作用,他们也是反帝反封建反官僚资本主义革命统一战线中的一分子,所以,团结他们也是一个必须注意的问题。我们对于开明绅士的要求,在抗日时期是赞成抗日,赞成民主(不反共),赞成减租减息;在现阶段是赞成反美、反蒋,赞成民主(不反共),赞成土地改革。只要他们能够这样做,我们就应该毫无例外地去团结他们,并且在团结中教育他们。


\begin{maonote}
\mnitem{1}见本卷\mxart{中国人民解放军宣言}中八项政策的第一项。
\mnitem{2}刘少白,一九四二年十一月被选为晋绥边区临时参议会副议长。李鼎铭,一九四一年十一月被选为陕甘宁边区政府副主席。
\end{maonote}
