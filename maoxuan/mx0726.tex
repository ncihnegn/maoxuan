
\title{接见卡博和巴卢库时的谈话}
\date{一九六七年二月三日}
\maketitle


\mxsay{毛泽东:}什么时候谢胡\mnote{1}同志到中国来的呀?

\mxsay{周恩来:}去年五月。

\mxsay{毛泽东:}去年五月我就向他讲这个问题,究竟是修正主义胜利还是马列主义胜利?这是两条路线斗争的问题,我也说,究竟哪一方面胜利现在还看不出来,还不能作结论。有两个可能,第一个可能是资产阶级胜利,修正主义胜利,把我们打倒;第二个可能就是我们把修正主义、资产阶级打倒。我为什么把第一个可能放在我们会失败这一点上呢?我感觉这样看问题比较有利。就是不要轻视敌人。

多少年来,我们党内的斗争没有公开化。比如,一九六二年一月,我们召开了七千人的县委书记以上干部大会,那个时候我讲了一篇话,我说,修正主义要推翻我们,如果我们现在不注意,不进行斗争,少则几年十几年,多则几十年,中国会要变成法西斯专政的。这篇讲演没有公开发表,在内部发表了。以后还要看一看,里面也许有些话还要修改。不过在那个时候已经看出问题来了。一九六二年,六三、六四、六五、六六,五年的时间。为什么说我们有不少工作没有做好?不是跟你们讲客气的,是跟你们讲真话。就是过去我们只抓了一些个别的问题,个别的人物,比如十七年来,就有和高岗、饶漱石\mnote{2}的斗争,他们一个集团,我们把他整下去了,这是一九五三年冬到一九五四年春。然后是一九五九年,把彭德怀、黄克诚、张闻天\mnote{3}这个集团整下去了。过去我们搞了农村的斗争,工厂的斗争,文化界的斗争,进行了社会主义教育运动,但不能解决问题,因为没有找到一种形式,一种方式,公开地、全面地、由下而上地发动广大群众来揭发我们的黑暗面。

这场斗争也准备了一个时期,前年十一月,对一个历史学家吴晗发表一篇批判文章\mnote{4},这篇文章在北京写不行,不能组织班子,只好到上海找姚文元他们搞了一个班子,写出这篇文章。

\mxsay{卡:}受毛泽东同志指示写的?

\mxsay{毛泽东:}开头写我也不知道,是江青他们搞的。搞了交给我看。先告诉我要批评。他们在北京组织不了,到上海去组织,我都不知道。文章写好了交给我看,说这篇文章只给你一个人看,周恩来、康生这些人也不能看,因为要给他们看,就得给刘少奇、邓小平、彭真、陆定一这些人看,而刘、邓这些人是反对发表这篇文章的。文章发表以后,各省都转载,北京不转载。我那个时候在上海,后头我说印小册子。各省都答应发行,就是北京的发行机关不答应,彭真通知出版社,不准翻印。北京市委就是针插不进,水泼不进的市委。现在不是改组了吗?改组了的市委还不行,现在还要改组。当公开发表北京市委改组决定\mnote{5}时,我们增加了两个卫戍师。现在北京有三个陆军师,一个机械化师,一共有四个师。所以,你们才能到处走,我们也才能到处走。原先那两个师是好的,但是,分散的一塌糊涂,到处保卫。

现在的红卫兵当中也有不可靠的,是保皇派,他们白天不活动晚上活动,戴眼镜,戴口罩,手里拿着棍子、刀,到处捣乱,杀了一些好人,杀死了几个人,杀伤了好几百。多数都是一些高级干部的子弟,比如象贺龙、陆定一、罗瑞卿这些人的子弟。所以,我们的军队也不是没有问题的。象贺龙是政治局委员,罗瑞卿是书记处书记、总参谋长。把罗瑞卿的问题处理了,那是前年十二月;把北京市委这些人处理了,是去年五月。发动大字报运动,是去年六月一号。发动红卫兵,是去年八月。你们有人不是见了北京大学聂元梓\mnote{6}吗?谁人去见的呀?

\mxsay{卡:}什图拉同志。

\mxsay{毛泽东:}她在去年五月二十五号写了一张大字报,那个时候我在杭州,到六月一号中午我才看到,我就打电话给康生、陈伯达,我说要广播。这一下大字报就满天飞了!

\mxsay{巴:}大字报就发出信号。

\mxsay{毛泽东:}也不是我写的,是聂元梓她们七个人写的。红卫兵是清华大学附属中学、北京大学附属中学两处搞起来的,他们有一篇材料给我看到了。到了八月一号,我就写了一封信给两个学校的红卫兵,后来就大搞起来了。八月十八号我接见了红卫兵几十万人。接着八月上旬到八月中旬就开了中央八届十一中全会,这个时候我自己才写了一张二百个中国字的大字报,说,从中央到地方某些负责人,站在资产阶级的立场上,反对学生,反对无产阶级,搞白色恐怖。这才揭露了刘少奇、邓小平的问题。现在,两方面的决战还没有完成,大概二、三、四这三个月是决胜负的时候。至于全部解决问题可能要到明年二、三、四月或者还要长。才别相信我们这个党里都是好人。好几年以前我就说要洗刷几百万,那不是讲空话吗?你有什么办法?毫无办法。只有发动群众才有办法,没有群众我们毫无办法。他\mnote{7}不听。一个《人民日报》,就不听我的。《人民日报》是夺了两次权的,第一次是去年六月一号;第二次是最近一月份。过去我公开声明,我说,《人民日报》我不看。当着《人民日报》总编辑也说,我不看你的报纸。讲了好几次,他就是不听。我的这一套在中国是不灵的,所有大中学校都不能进去。因为控制在刘少奇、邓小平、陆定一的宣传部、周扬的文化部这些人手里,还有高等教育部、普通教育部这些人的手里,毫无办法。

我们党内暴露出许多人,大概可以分这么几部分:一部分是搞民主革命的,在民主革命阶段可以合作,他的目的是民主革命,要搞资本主义。打倒帝国主义,打倒封建主义,他是赞成的;打倒官僚资本主义,他也赞成;实际上打倒民族资本主义他就不赞成了。分配土地,他是赞成的,分到农民手里,要组织合作社他就不赞成了。这一部分人,就是一批所谓老干部。

第二部分人就是解放以后才进党的一批人,百分之八十是一九四九年以后进党的。其中有一部分人当了干部,支部书记、党委书记,甚至更高的县委书记、地委书记、省委书记,还有中央委员,这么一批人。

第三部分就是我们收容下来的国民党的这些人,其中有些过去是共产党被国民党抓去,然后叛变了,在报上登报反共。那个时候我们不知道他们反共,不知道他们所谓“履行手续”是一些什么东西。现在一查出来,是拥护国民党,反对共产党。

第四部分人就是资产阶级、地主、富农的子弟,解放以后他们进了学校,甚至进了大学,掌了一部分权。这些人也不是都坏,有许多是站在我们方面的。但是,有一部分是反革命分子。

大概就是这么几部分人。总之,在中国人数并不多,百分之几。他们的阶级基础只有百分之几,比如地主、富农、资本家、国民党等等,顶多有百分之五。那么,七亿人口里面也不过是三千五百万人。他们也分散,分到各个乡村、各个城市、各个街道。如果三千五百万人集中到一起,手里有了武器,那就是一股大军了。

\mxsay{巴:}尽管合在一起也是一个没有思想的军队。

\mxsay{毛泽东:}他们是灭亡的阶级,他们的代表人物,在三千多万人里顶多有几十万人,也分散,分到各城市、各街道、各农村、各学校、各机关。所以,大字报一出来,群众运动一出来,红卫兵一出来,他们吓得要死。

另外,还有一些什么东西也搞得很乱,又给我封了好几个官,什么伟大导师、伟大领袖、伟大统帅、伟大舵手,我就不高兴。但是,有什么办法!他们到处这么搞。有人建议保留一个Teacher,我是个小学教员嘛,就是一个普通教员多好。至于什么Professor(教授)谈不上,我没有进过大学,你们都进过大学吧?

\mxsay{卡:}一个都没有。

\mxsay{毛泽东:}马克思是大学生,列宁是大学生,斯大林读了中学,我也是只读了中学。大学生,有很大一部分人我是怀疑的,特别是读文科、社会科学的。这些人如果不进行教育,不搞文化大革命,很危险,这些人将来就是修正主义。搞文学的不能写小说,不能写诗;学哲学的不能写哲学文章,也不能解释社会现象。还有学政治的、学法律的,都是一些资产阶级的东西,我们没有搞出什么好的教科书。还有学经济的,修正主义分子可多了。但是,现在看来有些希望,斗得厉害。

群众都发动起来了,什么坏东西都可以扔掉。巩固无产阶级专政,我们是乐观的。从去年,我和谢胡同志谈话时,比较乐观些了。

\mxsay{卡:}以毛主席为代表的革命路线取得了巨大胜利。

\mxsay{毛泽东:}取得了相当的胜利,巨大的现在还没有。在明年这个时候也许可以讲。但是,我们还不能断定。也许我们这批人要被打败,我时刻准备着,打败就打败,总有人起来继续战斗。中国这个国家有人吹牛皮说是什么“爱好和平”,才不是那样,爱斗争,动不动就打,我也是一个。好斗,出修正主义就不那么容易了。

\mxsay{卡:}不搞斗争是不行的,不然革命怎么实现呢?

\mxsay{毛泽东:}就是吆!中国搞修正主义不像苏联那么容易,中国是半封建半殖民地国家,受压迫一百多年。我们的国家是军队打的,学校原封未动,党和政府的领导人有的是委派去的,如曹荻秋、陈丕显\mnote{8}不是派去的吗?以后选举的。选举我是不相信的,中国有两千多个县,一个县选举两个就四千多,四个就一万多,哪有那么大的地方开会?那么多人怎么认识?我是北京选的,许多人就没有看见我嘛!见都没见怎么选呢?不过是闻名而已,我和总理都是闻名的。还不如红卫兵,他们的领导人还和他们讲过话呢,不过红卫兵也在不断的分化。在去年夏天左派是极少数,站在我们这边,受压迫,他们被打成“右派”、“反革命”等等。到了冬季起了变化,少数派变成多数派。你们到过清华大学吗?

\mxsay{卡:}去过。

\mxsay{毛泽东:}“井冈山”,过去是少数派,是受打击的;北大的聂元梓也是少数派,受打击的。现在变成多数派。过去受压迫,他们少数派很革命,一到了冬季变成多数派,

去年十二月,今年一月有一些就分化了。有一部分人夏季是革命的,到了冬季就变成反革命的。当然,聂元梓、蒯大富\mnote{9}这两个人,我们是在那里做工作,说服他们。但是,这种人究竟靠得住靠不住,我们还要看。不过,闹起来总会有好人在里头。

现在流行着一种无政府主义思想,口号是“怀疑一切,打倒一切”,结果弄到自己身上。你一切怀疑,你自己呢?你一切打倒,你自己呢?资产阶级要打倒,无产阶级呢?他那个理论就是不行。不过,整个潮流看来,斗来斗去那些错误的人总是最后站不住脚。你们看,北京街上有打倒我的标语,打倒林彪同志的标语。什么打倒周恩来、康生、陈伯达、江青等等的标语都有。至于要打倒李富春、谭震林、李先念、陈毅、叶剑英、聂荣臻、肖华的标语就更多一些。比如杨成武,他是代总参谋长,总参管好几个部,其中一个作战部的部长、副部长写大字报要打倒他。那是贺龙挑起来的。打倒肖华是北京军区司令员\mnote{10}挑起来的,要搞他。昨天挑起打倒肖华,过了一两天他自己被人家打倒了。中国第二大军区呀!一个军区司令,一个政委还有个副政委张南生几个都倒了。但是,有一条真理永远是真理,天不会掉下来的,绝大多数人民,工人、农民、知识分子、干部、党员、团员是好的,我们要坚决相信这条真理。

\mxsay{巴:}他们都是活生生的力量。

\mxsay{毛泽东:}虽然有些人有些错误、缺点。我也不来包庇叶剑英、杨成武、肖华、王树声一点缺点,但是他们基本上是好人。

\mxsay{巴:}工作上可以犯错误,可以改呀。

\mxsay{毛泽东:}可以改嘛!我也犯了一些错误嘛,只有人家犯错误我就不犯?我就犯了一些错误。政治、军事各方面都犯了一些错误。至于犯了一些什么具体的错误,现在没有时间,你们如果多呆几天可以跟你们讲,我不隐瞒自己的错误。有些人吹,说我一点错误也没有,我就不相信,我就不高兴。你那么吹我就不相信,我是一个啥人,自己还不知道?有一点自知之明嘛。

如果中国天黑了,地也黑了,你们也不要怕,要相信一点,全黑也不会的。秦始皇统治十六年就倒了,有两个人首先起义,一个叫陈胜,一个叫吴广,他们都是那个时候的农奴。现在中国贴大字报的红卫兵,在去年夏季被打击,被打成“反革命”的这些人,就是陈胜、吴广\mnote{11}。

\begin{maonote}
\mnitem{1}谢胡,时任阿尔巴尼亚部长会议主席,巴卢库,时任阿尔巴尼亚中央政治局委员、副总理兼国防部长,卡博,时任阿尔巴尼亚中央政治局委员、中央书记,什图拉时任阿尔巴尼亚中央委员。
\mnitem{2}一九五三年全国财经工作会议和全国组织工作会议期间及其前后,时任中共中央政治局委员、国家计划委员会主席的高岗,和时任中共中央委员、中央组织部部长的饶漱石相互勾结,阴谋分裂党、篡夺党和国家的最高权力。一九五四年中共七届四中全会揭露和批判了他们的反党阴谋活动。一九五五年中国共产党全国代表会议通过决议将他们开除出党。
\mnitem{3}一九五九年七月二日至八月一日在庐山召开的中共中央政治局扩大会议和八月二日至十六日召开的中共八届八中全会,会议通过的《关于以彭德怀同志为首的反党集团的错误的决议》,揭发和批判了彭德怀、张闻天、黄克诚等同志的错误,他们把一些暂时的、局部的、早已克服了或者正在迅速克服中的缺点收集起来,并且加以极端夸大,把形势描写成为漆黑一团,企图向党要权。
\mnitem{4}一九六五年十一月十日,姚文元在上海《文汇报》发表《评新编历史剧〈海瑞罢官〉》,批判了吴晗的《海瑞罢官》。
\mnitem{5}一九六六年六月四日,《人民日报》发布了“中共中央决定改组北京市委”,内容如下:

新华社三日讯中共中央决定:由中共中央华北局第一书记李雪峰同志兼任北京市委第一书记,调中共吉林省委第一书记吴德同志任北京市委第二书记,对北京市委进行改组。李雪峰、吴德两同志业已到职工作。北京市的社会主义文化大革命的工作,由新市委直接领导。
\mnitem{6}聂元梓,北京大学学生运动领袖之一。一九六六年五月二十五日下午二时许,北京大学哲学系聂元梓、宋一秀、夏剑豸、杨克明、赵正义、高云鹏、李醒尘七人,在大饭厅东墙上贴出了题为《宋硕、陆平、彭珮云在文化革命中究竟干些什么?》的大字报。
\mnitem{7}指刘少奇,时任国家主席,文革前主持一线工作。
\mnitem{8}曹荻秋文革前原任上海市市长,陈丕显原任上海市委书记,在“一月风暴”中被夺权打倒。一九三二年曹荻秋在上海被捕,五年后履行一定手续后被释放,在文革中被定为叛徒。一九三〇年陈丕显被敌人俘虏,与他同时被俘的涂应达审讯后被敌人杀害,而匪军营长却收陈丕显当了义子,在文革中被隔离审查。
\mnitem{9}蒯大富,清华大学红卫兵组织“井冈山兵团”的领袖。
\mnitem{10}杨勇时任北京军区司令员。
\mnitem{11}陈胜、吴广,领导了中国历史上第一次农民起义战争。
\end{maonote}
