
\title{炮打司令部——我的一张大字报}
\date{一九六六年八月五日}
\thanks{这是毛泽东同志写于八届十一中全会期间的一张大字报并作为全会文件印发。}
\maketitle


全国第一张马列主义大字报和人民日报评论员的评论\mnote{1},写得何等好啊!请同志们重读这一篇大字报和这篇评论。可是在五十多天里\mnote{2},从中央到地方的某些领导同志,却反其道而行之,站在反动的资产阶级立场,实行资产阶级专政,将无产阶级轰轰烈烈的文化大革命运动打下去,颠倒是非,混淆黑白,围剿革命派,压制不同意见,实行白色恐怖,自以为得意,长资产阶级的威风,灭无产阶级的志气,又何其毒也!联系一九六二年的右倾\mnote{3}和一九六四年的形“左”而实右\mnote{4}的错误倾向,岂不是可以发人深醒的吗?

\begin{maonote}
\mnitem{1}全国第一张马列主义大字报和人民日报评论员的评论,一九六六年六月二十五日,北京大学聂元梓等七人贴出反对北大党委的大字报《宋硕、陆平、彭佩云在文化大革命中究竟干了些什么》。经毛泽东的同意,该大字报于一九六六年六月一日在全国广播,六月二日《人民日报》全文刊登,并加了评论员的评论《欢呼北大的一张大字报》,全文如下:

聂元梓等同志的大字报,揭穿了“三家村”黑帮分子的一个大阴谋!

“三家村”黑店的掌柜邓拓被揭露出来了,但是这个反党集团并不甘心自己的失败。他们仍然负隅顽抗,用“三家村”反党集团分子宋硕的话来说,叫作“加强领导,坚守岗位”。

他们“坚守”的是什么“岗位”?他们“坚守”的是他们多年来一直盘踞的堡垒。他们加强的是什么“领导”?就是指挥他们的伙计作垂死挣扎、力图保持他们反党反社会主义的阵地。

宋硕的“加强领导”,“坚守岗位”,这是一个信号。它反映了在这场摧枯拉朽的无产阶级文化大革命中,一切牛鬼蛇神们的动态。他们是一步不让的,寸土必争的,不斗不倒的。

“三家村”黑帮是诡计多端的。在前一个时候,他们采取“牺牲车马,保存主帅”的战术。现在“主帅”垮台了,他们就采取能保存多少车马就保存多少车马的手法。他们妄图保存实力,待机而动。

为陆平、彭珮云等人多年把持的北京大学,是“三家村”黑帮的一个重要据点,是他们反党反社会主义的顽固堡垒。已经到了五月十四日,陆平还传达北京市委大学部副部长宋硕的所谓紧急指示,并手忙脚乱地进行部署,欺骗、蒙蔽和压制广大青年学生和革命干部、革命教师,不许他们响应毛主席和党中央的号召起来革命,彭珮云是一个神秘人物,上窜下跳,拉线搭桥。在这个事件中,她转入地下活动,来往于北京大学历史系住地十三陵和宋硕、陆平之间,出谋划策,秘密指挥。

这一切,都说明“三家村”黑店的分号,“三家村”黑帮的“车马”们,还是有指挥、有组织、有计划地进行顽抗。

陆平以北京大学“党委书记”的身份,以“组织”的名义,对起来革命的学生和干部,进行威吓,说什么不听从他们这一撮人的指挥就是违犯纪律,就是反党。这是“三家村”黑帮反党分子们惯用的伎俩。请问陆平,你们所说的党是什么党?你们的组织是什么组织?你们的纪律是什么纪律?

事实使我们不能不做出这样的回答,你们的“党”不是真共产党,而是假共产党、是修正主义的“党”。你们的“组织”就是反党集团。你们的纪律就是对无产阶级革命派实行残酷无情的打击。

陆平们这一套是骗不了人的。

对于无产阶级革命派来说,我们遵守的是中国共产党的纪律,我们无条件接受的,是以毛主席为首的党中央的领导。

毛泽东思想,是我们各项工作的最高指示。毛主席关于社会主义社会阶级和阶级斗争的学说,关于在意识形态领域中兴无灭资的无产阶级文化大革命的指示,是我们必须坚决遵循的。凡是反对毛主席,反对毛泽东思想,反对毛主席和党中央的指示的,不论他们打着什么旗号,不管他们有多高的职位、多老的资格,他们实际上是代表被打倒了的剥削阶级的利益,全国人民都会起来反对他们,把他们打倒,把他们的黑帮、黑组织、黑纪律彻底摧毁。

人类历史上空前未有的无产阶级文化大革命的浪潮,汹涌澎湃,妄图阻挡这个潮流的小丑们,他们是难逃灭顶之灾的。

工农兵和无产阶级的文化战士,在党中央和毛主席的领导下,以排山倒海之势,正在一个一个地夺取反革命的文化阵地,摧毁反革命的文化堡垒。那些什么“三家村”、“四家村”,不过是纸老虎。他们的“将帅”保不住,他们的“车马”也同样是保不住的。

北京大学的无产阶级革命派,一定能够更高地举起毛泽东思想的伟大红旗,一定能够更加有力地团结群众进行战斗。

一时还看不清楚的人们,一定会迅速地提高自己的觉悟,参加到战斗的行列中来。北京大学广大师生的反对资产阶级代表人物的革命斗争,一定能够胜利。一个欣欣向荣的真正的社会主义的新北大,一定会很快地出现在人民的首都。
\mnitem{2}五十多天,资产阶级专政,一九六六年六月三日,中共中央在刘少奇、邓小平主持下,决定向北京的大学和中学派出工作组,领导各单位的“文化大革命”。刘少奇制订了《中央八条》,包括“内外有别”等政策,限制和镇压学生的自发运动,在全国各地发生了如西安交大六·六事件、清华大学六·七事件、广西六·八事件、北京地院六·二〇事件、北师大六·二〇事件、林院《谈话纪要》事件等学生与工作组对抗、工作组将学生打成反革命的现象。一九六六年七月二十四日毛泽东返回北京后召集中央常委和中央文革小组开会,批评了刘少奇、邓小平,做出了撤销工作组的决定。此后,六六年六月初到七月底被称为“五十多天的资产阶级反动路线”。
\mnitem{3}一九六二年的右倾,一九六二年在“七千人大会”上和一系列政策制订中,刘少奇等执行了经济、文化方面的“调整、恢复、宽松”等右倾路线。如主张放弃人民公社的集体农业,而实行分田单干的个体经济。
\mnitem{4}一九六四年的形“左”而实右,指一九六四年刘少奇制订的四清《后十条》提出“四清四不清的矛盾”、整肃农村基层干部的形“左”实右的路线。
\end{maonote}
