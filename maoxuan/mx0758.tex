
\title{我的一点意见}
\date{一九七〇年八月三十一日}
\thanks{这是毛泽东同志对陈伯达一份报告\mnote{1}的批示。}
\maketitle


这个材料是陈伯达\mnote{2}同志搞的,欺骗了不少同志。

第一,这里没有马克思的话。

第二,只找了恩格斯一句话\mnote{3},而《路易·波拿巴特政变记》这部书不是马克思的主要著作。

第三,找了列宁的有五条\mnote{4}。其中第五条说,要有经过考验、受过专门训练和长期教育,并且彼此能够很好地互相配合的领袖,这里列举了四个条件。别人且不论,就我们中央委员会的同志来说,够条件的不很多。例如,我跟陈伯达这位天才理论家之间,共事三十多年,在一些重大问题上就从来没有配合过,更不去说很好的配合。仅举三次庐山会议\mnote{5}为例。第一次,他跑到彭德怀\mnote{6}那里去了。第二次,讨论工业七十条\mnote{7},据他自己说,上山几天就下山了,也不知道他为了什么原因下山,下山之后跑到什么地方去了。这一次,他可配合得很好了,采取突然袭击,煽风点火,唯恐天下不乱,大有炸平庐山,停止地球转动之势。我这些话,无非是形容我们的天才理论家的心(是什么心我不知道,大概是良心吧,可决不是野心)的广大而已。至于无产阶级的天下是否会乱,庐山能否炸平,地球是否停转,我看大概不会吧。上过庐山的一位古人\mnote{8}说:“杞国无事忧天倾”。我们不要学那位杞国人。最后关于我的话\mnote{9},肯定帮不了他多少忙。我是说主要地不是由于人们的天才,而是由于人们的社会实践。我同林彪\mnote{10}同志交换过意见,我们两人一致认为,这个历史家和哲学史家争论不休的问题,即通常所说的,是英雄创造历史,还是奴隶们创造历史,人的知识(才能也属于知识范畴)是先天就有的,还是后天才有的,是唯心论的先验论,还是唯物论的反映论,我们只能站在马列主义的立场上,而决不能跟陈伯达的谣言和诡辩混在一起。同时我们两人还认为,这个马克思主义的认识论问题,我们自己还要继续研究,并不认为事情已经研究完结。希望同志们同我们一道采取这种态度,团结起来,争取更大的胜利,不要上号称懂得马克思,而实际上根本不懂马克思那样一些人的当。

\begin{maonote}
\mnitem{1}指陈伯达一九七〇年中共九届二中全会期间搜集整理的《恩格斯、列宁、毛主席关于称天才的几段语录》和《林副主席指示》。
\mnitem{2}陈伯达,当时任中共中央政治局常委。原任中央文革小组组长。九大后,中央文革就解散了。
\mnitem{3}指恩格斯为马克思《路易·波拿巴特政变记》(今译为《路易·波拿巴的雾月十八日》)德文第三版写的序言中评价该书的一句话:“这是一部天才的著作。”
\mnitem{4}陈伯达摘录的这五条分别是:

1.“当你读到这些评论的时候,就会觉得自己好像是在亲自听取这位天才思想家讲话一样。”(列宁《卡·马克思致路·库格曼书信集俄译本序言》中对马克思对于各个作家的评论的评价)

2.“马克思的全部天才正在于他回答了人类先进思想已经提出的种种问题。”(列宁《马克思主义的三个来源和三个组成部分》)

3.“马克思的天才就在于他最先从这里得出了全世界历史提示的结论,并且一贯地推行了这个结论。这一结论就是关于阶级斗争的学说。”(列宁《马克思主义的三个来源和三个组成部分》)

4.“这真是多么天才的预见!”(列宁《预言》中对恩格斯谈未来世界大战一段话的评价)。

5.“在现代社会中,假如没有‘十来个’富有天才(而天才人物不是成千成百地产生出来的)、经过考验、受过专门训练和长期教育并且彼此能够很好地互相配合的领袖,无论哪个阶级都无法进行坚持不懈的斗争。”(列宁《怎么办?》)
\mnitem{5}三次庐山会议,指一九五九年七月二日至八月十六日在庐山先后举行的中共中央政治局扩大会议和八届八中全会,一九六一年八月二十三日至九月十六日在庐山举行的中共中央工作会议,一九七〇年八月二十三日至九月六日在庐山举行的中共九届二中全会。
\mnitem{6}彭德怀,原任中共中央政治局委员、中央军委副主席、国务院副总理兼国防部部长。一九五九年七月二日至八月一日在庐山召开的中共中央政治局扩大会议和八月二日至十六日召开的中共八届八中全会,会议通过的《关于以彭德怀同志为首的反党集团的错误的决议》,揭发和批判了彭德怀、张闻天、黄克诚等同志的错误。
\mnitem{7}指《国营工业企业工作条例(草案)》,共七十条,简称工业七十条。
\mnitem{8}指唐代诗人李白。
\mnitem{9}指陈伯达摘录的毛泽东《实践论》中的一段话:“马克思、恩格斯、列宁、斯大林之所以能够作出他们的理论,除了他们的天才条件之外,主要地是他们亲自参加了当时的阶级斗争和科学实验的实践,……”
\mnitem{10}林彪,时任中共中央副主席、中央军委副主席。
\end{maonote}
