
\title{两个中国之命运}
\date{一九四五年四月二十三日}
\thanks{这是毛泽东在中国共产党第七次全国代表大会上的开幕词。}
\maketitle


同志们!中国共产党第七次全国代表大会今天开幕了。

我们这个大会有什么重要意义呢?我们应该讲,我们这次大会是关系全中国四亿五千万人民命运的一次大会。中国之命运有两种:一种是有人已经写了书的\mnote{1};我们这个大会是代表另一种中国之命运,我们也要写一本书出来\mnote{2}。我们这个大会要打倒日本帝国主义,把全中国人民解放出来。这个大会是一个打败日本侵略者、建设新中国的大会,是一个团结全中国人民、团结全世界人民、争取最后胜利的大会。

现在的时机很好。在欧洲,希特勒快要被打倒了。世界反法西斯战争的主要的一部分是在西方,那里的战争很快就要胜利了,这是苏联红军努力的结果。现在柏林已经听到红军的炮声,大概在不久就会打下来。在东方,打倒日本帝国主义的战争也接近着胜利的时节。我们的大会是处在反法西斯战争最后胜利的前夜。

在中国人民面前摆着两条路,光明的路和黑暗的路。有两种中国之命运,光明的中国之命运和黑暗的中国之命运。现在日本帝国主义还没有被打败。即使把日本帝国主义打败了,也还是有这样两个前途。或者是一个独立、自由、民主、统一、富强的中国,就是说,光明的中国,中国人民得到解放的新中国;或者是另一个中国,半殖民地半封建的、分裂的、贫弱的中国,就是说,一个老中国。一个新中国还是一个老中国,两个前途,仍然存在于中国人民的面前,存在于中国共产党的面前,存在于我们这次代表大会的面前。

既然日本现在还没有被打败,既然打败日本之后,还是存在着两个前途,那末,我们的工作应当怎样做呢?我们的任务是什么呢?我们的任务不是别的,就是放手发动群众,壮大人民力量,团结全国一切可能团结的力量,在我们党领导之下,为着打败日本侵略者,建设一个光明的新中国,建设一个独立的、自由的、民主的、统一的、富强的新中国而奋斗。我们应当用全力去争取光明的前途和光明的命运,反对另外一种黑暗的前途和黑暗的命运。我们的任务就是这一个!这就是我们大会的任务,这就是我们全党的任务,这就是全中国人民的任务。

我们的希望能不能实现?我们认为是能够实现的。这个可能性是存在的,因为我们现在已经具备了这样几个条件:

第一,有一个经验丰富和集合了一百二十一万党员的强大的中国共产党;

第二,有一个强大的解放区,这个解放区包括九千五百五十万人口,九十一万军队,二百二十万民兵;

第三,有全国广大人民的援助;

第四,有全世界各国人民特别是苏联的援助。

一个强大的中国共产党,一个强大的解放区,全国人民的援助,国际人民的援助,在这些条件下,我们的希望能不能实现呢?我们认为是能够实现的。这些条件,在中国是从来没有过的。多少年来虽然有了一些条件,但是没有现在这样完备。中国共产党从来没有现在这样强大过,革命根据地从来没有现在这样多的人口和这样大的军队,中国共产党在日本和国民党统治区域的人民中的威信也以现在为最高,苏联和各国人民的革命力量现在也是最大的。在这些条件下,打败侵略者,建设新中国,应当说是完全可能的。

我们需要一个正确的政策。这个政策的基本点,就是放手发动群众,壮大人民的力量,在我们党领导之下,打败侵略者,建设新中国。

中国共产党从一九二一年产生以来,已经二十四年了,其间经过了北伐战争、土地革命战争、抗日战争这样三个英勇奋斗的历史时期,积累了丰富的经验。到了现在,我们的党已经成了中国人民抗日救国的重心,已经成了中国人民解放的重心,已经成了打败侵略者、建设新中国的重心。中国的重心不在任何别的方面,而在我们这一方面。

我们应该谦虚,谨慎,戒骄,戒躁,全心全意地为中国人民服务,在现时,为着团结全国人民战胜日本侵略者,在将来,为着团结全国人民建设新民主主义的国家。只要我们能够这样做,只要我们有正确的政策,只要我们一致努力,我们的任务是必能完成的。

打倒日本帝国主义!

中国人民解放万岁!

中国共产党万岁!

中国共产党第七次全国代表大会万岁!


\begin{maonote}
\mnitem{1}指一九四三年出版的蒋介石《中国之命运》一书。
\mnitem{2}指毛泽东准备在这次大会上作的\mxart{论联合政府}的报告。
\end{maonote}
