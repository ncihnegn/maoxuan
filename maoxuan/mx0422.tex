
\title{中国人民解放军宣言}
\date{一九四七年十月十日}
\thanks{这是毛泽东为中国人民解放军总部所起草的政治宣言。在这个宣言里,分析了当时的国内政治形势,提出了“打倒蒋介石,解放全中国”的口号,宣布了中国人民解放军的也就是中国共产党的八项基本政策。这个宣言在一九四七年十月十日公布,被称为《双十宣言》。宣言是在陕北佳县神泉堡起草的。}
\maketitle


中国人民解放军,在粉碎蒋介石的进攻之后,现已大举反攻。南线我军已向长江流域进击,北线我军已向中长、北宁两路进击。我军所到之处,敌人望风披靡,人民欢声雷动。整个敌我形势,和一年前比较,已经起了基本上的变化。

本军作战目的,迭经宣告中外,是为了中国人民和中华民族的解放。而在今天,则是实现全国人民的迫切要求,打倒内战祸首蒋介石,组织民主联合政府,借以达到解放人民和民族的总目标。

中国人民,为了自己的解放和民族的独立,同日本帝国主义英勇奋战了八年之久。日本投降后,人民渴望和平,蒋介石则破坏人民一切争取和平的努力,而以空前的内战灾难压在人民的头上。这样,就逼得全国各阶层人民,除了团结起来打倒蒋介石以外,再无出路。

蒋介石现在的内战政策,不是偶然的,这是蒋介石及其反动集团一贯反人民政策的必然结果。早在民国十六年(一九二七年),蒋介石就忘恩负义地背叛了国共两党的革命联盟,背叛了孙中山的革命的三民主义和三大政策,从此建立独裁统治,投降帝国主义,打了十年内战,造成日寇侵略。民国二十五年(一九三六年)西安事变\mnote{1}时期,中国共产党以德报怨,协同张学良、杨虎城两将军,释放蒋介石,希望蒋介石悔过自新,共同抗日。但是蒋介石又一次忘恩负义,对于日寇则消极应战,对于人民则积极镇压,对于共产党则极端仇视。前年(一九四五年)日本投降,中国人民又一次宽恕蒋介石,要求蒋介石停止已经发动的内战,实行民主政治,团结各党派和平建国。但是毫无信义的蒋介石,在签订停战协定\mnote{2}、通过政协决议\mnote{3}、宣布四项诺言\mnote{4}以后,随即将其全部推翻。人民方面,虽则再三忍让求全,但是蒋介石在美帝国主义援助之下,决心不顾国家民族的死活,向人民作空前的全面的进攻。从去年(一九四六年)一月停战协定宣布到现在,蒋介石先后动员了二百二十多个正规旅和近百万的杂色部队,向中国人民从日本帝国主义手里用血战夺取过来的解放区,实行大举进攻,先后侵占了沈阳、抚顺、本溪、四平、长春、永吉、承德、集宁、张家口、淮阴、菏泽、临沂、延安、烟台等城市和广大的乡村。蒋军所到之处,杀人放火,奸淫掳掠,实行三光政策,同日本强盗的行为完全一样。去年十一月,蒋介石召集了伪国大\mnote{5},宣布了伪宪法。今年三月,蒋介石驱逐了共产党的代表。今年七月,蒋介石下了反人民的总动员令\mnote{6}。对于全国各地反对内战、反对饥饿、反对美帝国主义侵略的正义的人民运动,对于工人、农民、学生、市民和公教人员的争生存的斗争,蒋介石的方针就是镇压、逮捕和屠杀。对于国内各少数民族,蒋介石的方针就是实施大汉族主义,摧残镇压,无所不至。在一切蒋介石统治区域,贪污遍地,特务横行,捐税繁重,物价高涨,经济破产,百业萧条,征兵征粮,怨声载道,这样就使全国绝大多数人民,处于水深火热之中。而以蒋介石为首的金融寡头,贪官污吏,土豪劣绅,则集中了巨大的财富。这些财富,都是蒋介石等利用其独裁权力横征暴敛、假公济私而来的。蒋介石为着维持独裁,进行内战,不惜出卖国家权利于外国帝国主义,勾结美国军队留驻青岛等地,从美国招致顾问人员,参加内战的指挥和军队的训练,残杀自己的同胞。内战的飞机、坦克、枪炮、弹药,大批从美国运来。内战的经费,大批从美国借来。蒋介石则以出卖军事基地、出卖空海航权、签订奴役性商约\mnote{7}等项比袁世凯卖国行为还要严重多倍的条件,作为酬谢美国帝国主义的礼物。总而言之,蒋介石二十年的统治,就是卖国独裁反人民的统治。到了今天,全国绝大多数人民,地无分南北,年无分老幼,都认识了蒋介石的滔天罪恶,盼望本军从速反攻,打倒蒋介石,解放全中国。

本军是中国人民的军队,一切以中国人民的意志为意志。本军的政策,代表中国人民的迫切要求,主要的有如下各项:

一、联合工农兵学商各被压迫阶级、各人民团体、各民主党派、各少数民族、各地华侨和其它爱国分子,组成民族统一战线,打倒蒋介石独裁政府,成立民主联合政府。

二、逮捕、审判和惩办以蒋介石为首的内战罪犯。

三、废除蒋介石统治的独裁制度,实行人民民主制度,保障人民言论、出版、集会、结社等项自由。

四、废除蒋介石统治的腐败制度,肃清贪官污吏,建立廉洁政治。

五、没收蒋介石、宋子文、孔祥熙、陈立夫兄弟等四大家族和其它首要战犯的财产,没收官僚资本,发展民族工商业,改善职工生活,救济灾民贫民。

六、废除封建剥削制度,实行耕者有其田的制度。

七、承认中国境内各少数民族有平等自治的权利。

八、否认蒋介石独裁政府的一切卖国外交,废除一切卖国条约,否认内战期间蒋介石所借的一切外债。要求美国政府撤退其威胁中国独立的驻华军队,反对任何外国帮助蒋介石打内战和使日本侵略势力复兴。同外国订立平等互惠通商友好条约。联合世界上一切以平等待我之民族共同奋斗。

上述各项,就是本军的基本政策。本军所到之处,立即实施这些政策。这些政策是适合全国百分之九十以上人民的要求的。

本军对于蒋方人员,并不一概排斥,而是采取分别对待的方针。这就是首恶者必办,胁从者不问,立功者受奖。对于罪大恶极的内战祸首蒋介石和一切坚决助蒋为恶、残害人民、而为广大人民所公认的战争罪犯,本军必将追寻他们至天涯海角,务使归案法办。本军警告一切蒋军官兵,蒋政府官员,蒋党党员,凡是尚未沾染无辜人民鲜血的人们,切勿跟那些罪犯们同流合污。凡是已经做过坏事的人们,赶快停止作恶,悔过自新,脱离蒋介石,准其将功赎罪。本军对于放下武器的蒋军官兵,一律不杀不辱,愿留者收容,愿去者遣送。对于起义加入本军的蒋军部队和公开或秘密为本军工作的人们,则给予奖励。

为了早日打倒蒋介石,建立民主联合政府,我们号召全国各界同胞,在本军到达之处,同我们积极合作,肃清反动势力,建立民主秩序。在本军未到之处,则自动拿起武器,实行抗丁抗粮,分田废债,利用敌人空隙,发展游击战争。

为了早日打倒蒋介石,建立民主联合政府,我们号召解放区人民贯彻土地改革,巩固民主基础,发展生产,厉行节约,加强人民武装,肃清敌人残留据点,支援前线作战。

本军全体指挥员、战斗员同志们!我们现在担负了我国革命历史上最重要最光荣的任务,我们应当积极努力,完成自己的任务。我伟大祖国哪一天能由黑暗转入光明,我亲爱同胞哪一天能过人的生活,能按自己的愿望选择自己的政府,依靠我们的努力来决定。我全军将士必须提高军事艺术,在必胜的战争中勇猛前进,坚决彻底干净全部地歼灭一切敌人。必须提高觉悟性,人人学会歼灭敌人、唤起民众两套本领,亲密团结群众,把新区迅速建设成为巩固区。必须提高纪律性,坚决执行命令,执行政策,执行三大纪律八项注意\mnote{8},军民一致,军政一致,官兵一致,全军一致,不允许任何破坏纪律的现象存在。我全军将士必须时刻牢记,我们是伟大的人民解放军,是伟大的中国共产党领导的队伍。只要我们时刻遵守党的指示,我们就一定胜利。

打倒蒋介石!

新中国万岁!


\begin{maonote}
\mnitem{1}参见本书第一卷\mxnote{关于蒋介石声明的声明}{1}。
\mnitem{2}见本卷\mxnote{以自卫战争粉碎蒋介石的进攻}{1}。
\mnitem{3}见本卷\mxnote{以自卫战争粉碎蒋介石的进攻}{2}。
\mnitem{4}四项诺言,指蒋介石一九四六年一月在政治协商会议开幕词中宣布的保障人民自由、保障各党派合法地位、实行普选和释放政治犯。
\mnitem{5}见本卷\mxnote{美国“调解”真相和中国内战前途}{4}。
\mnitem{6}一九四七年七月四日,国民党反动政府通过了蒋介石的“国家总动员提案”,随即下了所谓《戡平共匪叛乱总动员令》。其实,蒋介石的反革命内战的总动员早已实行了。这时,中国人民解放军已经在全国范围内开始转入进攻。蒋介石自己也供认他的统治已到了“严重危机”。这个“总动员令”,只是表示蒋介石的垂死挣扎。
\mnitem{7}奴役性商约,指一九四六年十一月四日国民党政府和美国政府在南京签订的出卖中国主权的所谓《中美友好通商航海条约》。见本卷\mxnote{迎接中国革命的新高潮}{5}。
\mnitem{8}见本卷\mxnote{中国人民解放军总部关于重行颁布三大纪律八项注意的训令}{1}。
\end{maonote}
