
\title{对彭德怀九月九日信\mnote{1}的批示}
\date{一九五九年九月九日}
\thanks{这是毛泽东同志在彭德怀来信上的批示。}
\maketitle


此件即发各级党组织,从中央到支部,印发在北京开会的军事、外事两会议各同志。

我热烈地欢迎彭德怀同志的这封信,认为他的立场和观点是正确的,态度是诚恳的。倘从此彻底转变,不再有大的动摇(小的动摇是必不可免的),那就是“立地成佛”,立地变成一个马克思主义者了。我建议,全党同志都对彭德怀同志此信所表示的态度,予以欢迎。一面严肃地批判他的错误,一面对他的每一个进步都表示欢迎,用这两种态度去帮助这一位同我们有三十一年历史关系的老同志。对其他一切犯错误的同志,只要他们表示愿意改正,都用这两种态度去对待他们。必须坚信,我们的这种政策是能感动人的。人,在一定条件下,是能改变的,除开某些个别的例外的情况不计在内。德怀同志对于他自己在今后一段时间内工作分配的建议,我认为基本上是适当的。读几年书极好。年纪大了,不宜参加体力劳动,每年有一段时间到工厂和农村去做参观和调查研究工作,则是很好的。此事中央将同德怀同志商量,作出适当的决定。

\begin{maonote}
\mnitem{1}指彭德怀给毛泽东的信:

\mxname{主席:}

八届八中全会和军委扩大会议,对我的错误彻底地揭发的批判,消除了制造党内分裂的一个隐患,这是党的伟大胜利,也给了我改正错误的最后机会,我诚恳地感谢你和其他同志对我的耐心教育和帮助,这次党对我的错误进行系统地、历史地批判,是完全必要的,只有这样才能够使我真正认识到错误的极端的危险性,才有可能消除我的错误在党内外的恶劣影响,现在我深刻体会到我的资产阶级世界观、方法论是根深蒂固的,个人主义是极端严重的,现在我认识到党和人民培养我这样一个人付出了多么大的代价,如果不是及时得到彻底揭发和批判,其危险性又是多么可怕,过去由于自己的资产阶级立场作怪,将你对我善意的、诚恳的批评,都当作是对自己的打击,自己都没有受到教育,得到提高,使错误顽症得不到医治,三十余年辜负了你对我的教导和忍耐,使我愧感交集,难以言状,对不起党,对不起人民,也对不起你,今后必须下大功夫继续彻底反省自己的错误,努力学习马列主义理论,来改造自己的思想,保证晚年不再做危害党和人民的事情,为此请求中央考虑,在军委扩大会议以后,请允许我学习,或者离开北京到人民公社中去,一边学习,一边参加劳动,以便在劳动人民集体生活中得到锻炼和思想改造,是否妥当请考虑示复。

敬礼

彭德怀

一九五九年九月九日
\end{maonote}
