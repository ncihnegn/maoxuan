
\title{对中宣部和北京市委的批评}
\date{一九六六年三月、四月}
\thanks{这是毛泽东同志在杭州同部分政治局委员的谈话。}
\maketitle


\section*{(一)一九六六年三月十七日到二十日的谈话}

我们在解放以后,对知识分子实行包下来的政策,有利也有弊。现在学术界和教育界是资产阶级知识分子掌握实权。社会主义革命越深入,他们就越抵抗,就越暴露出他们的反党反社会主义的面目。吴晗和翦伯赞等人是共产党员,也反共,实际上是国民党。现在许多地方对于这个问题认识还很差,学术批判还没有开展起来。各地都要注意学校、报纸、刊物、出版社掌握在什么人手里,要对资产阶级的学术权威进行切实的批判。我们要培养自己的年青的学术权威。不要怕青年人犯“王法”,不要扣压他们的稿件。中宣部不要成为农村工作部\mnote{1}。

《前线》也是吴晗、廖沬沙、邓拓\mnote{2}的,是反党反社会主义的。

文、史、哲、法、经要搞文化大革命,要坚决批判,到底有多少马克思主义?

\section*{(二)一九六六年三月二十八日到三十日的谈话}

八届十中全会作出了进行阶级斗争的决定\mnote{3},吴晗发表这么多文章,从不要打招呼,从不要经过批准,姚文元的文章为什么偏偏要打招呼\mnote{4}?难道中央的决定不算数吗?

什么叫学阀?那些包庇反共知识分子的人就是学阀,包庇吴晗、翦伯赞这些“中学阀”的人是“大学阀”,扣压左派的稿件,包庇右派的大学阀,中宣部是“阎王殿”。要打倒阎王,解放小鬼。

彭真、北京市委、中宣部要是再包庇坏人,中宣部要解散,北京市委要解散,五人小组\mnote{5}要解散。

我历来主张,凡中央机关做坏事,我就号召地方造反,向中央进攻,各地要多出些“孙悟空”,大闹天宫。

去年九月\mnote{6},我问一些同志,中央出了修正主义怎么办?这是很可能的,也是最危险的。要保护左派,在文化大革命中培养左派队伍。

我们都老了,下一代能否顶住修正主义思潮,很难说。文化革命是长期艰巨的任务。我这一辈子完不成,必须进行到底。

\section*{(三)一九六六年四月二十二日的讲话}

我不相信,在文化革命中的问题只是吴晗问题,后面还有一串串“三家村”\mnote{7}。文化革命是触及人们灵魂的革命,是意识形态的斗争,触及的很广泛,涉及面很宽。朝里有人,比如中央宣传都、中央文化部都发生这方面的问题,朝里都有人。各大区、各省市都有。

在党中央各部门,包括大区、名省市,朝里是否那么干净?我不相信。

中国出不出修正主义,两种可能:不出或出,早出或迟出。搞得好可能不早出。早出也好,走向反面。

\section*{(四)一九六六年四月二十九日的讲话}

北京一根针也插不进去,一滴水也滴不进去。彭真要按他的世界观改造党,事物是向他的反面发展的,他自己为自己准备了垮台的条件,对他的错误要彻底攻。这是必然的事,是从偶然中暴露出来的,一步一步深入的。历史教训并不是人人都引以为戒的。这是阶级斗争的规律,是不以人们的意志为转移的。凡是在中央有人搞鬼,我就号召地方起来攻他们,叫孙悟空大闹天宫,并要搞那些保玉皇大帝的人。彭真是混到党内的渺小的人物,没有什么了不起,一个指头就捅倒他。“西风落叶下长安”\mnote{8},告诉同志们不要无穷地忧虑。灰尘不扫不少,阶级敌人不斗不倒。

现象是看得见的,本质是隐蔽的,本质也会通过现象表现出来。彭真的本质隐藏了三十年。

出修正主义不只文化界出,党政军也要出,特别是党军出了修正主义问题就大了。

\begin{maonote}
\mnitem{1}农村工作部,中共中央农村工作部于一九五三年二月组建,一九六二年被撤销。三年困难时期后,为走出困境,中央上层在农村工作上存在两个方向,毛泽东坚决主张巩固和壮大集体经济,刘少奇、陈云、邓小平、邓子恢(农村工作部部长)则坚持分田到户和包产,陈云在一九六二年七月六日面见毛泽东,提出“分田到户不会产生两极分化,不会影响征购,恢复只要四年,否则八年”,对此毛泽东没有盲信,经过详细调查研究后,在一九六二年八月十五日北戴河中心小组会议上,关于农业恢复时间问题,毛泽东说:“瞎指挥我们不干了,高征购改正了,农业恢复的时间会快一些,恐怕再有两年差不多了,主要是今明两年,六四年扫尾。”毛泽东的这个估计,跟后来的实际情况基本符合,到一九六四年底,中国的国民经济就全面好转了。当时刘邓陈邓三个常委都对困难估计得过高,没有看到积极方面,对人民公社失去信心,想退回到到个体的小农经济,后来,陈云曾致信毛泽东,承认自己犯了错误,幸亏主席坚持了正确的方向。

毛泽东认为,农村个体小农经济,同社会主义的工业化是不相适应的;社会主义工业化与农业合作化必须同步;否则,社会主义的工业化,将面临绝大的困难。包产到户、分田到户也不是不能增产,但“增产有限”。而且势必引发贫富两极分化。

一九六五年五月二十五日在井冈山毛泽东对湖南省委第一书记张平化说:

“我为什么把包产到户看得那么严重?中国是个农业大国,农村所有制的基础如果一变,我国以集体经济为服务对象的工业基础就会动摇,工业品卖给谁嘛!工业公有制有一天也会变。两极分化快得很,帝国主义从存在的第一天起,就对中国这个大市场弱肉强食,今天他们在各个领域更是有优势,内外一夹攻,到时候我们共产党怎么保护老百姓的利益,保护工人、农民的利益?!怎么保护和发展自己民族的工商业,加强国防?!中国是个大国、穷国,帝国主义会让中国真正富强吗,那别人靠什么耀武扬威?!仰人鼻息,我们这个国家就不安稳了。”

所以,他坚决反对倒退到个体经济。邓子恢未向中央请示在中央党校和军队系统多次作报告阐述和传播包产到户主张,不仅工作犯错,而且违反组织纪律,一九六二年十一月九日,中共中央下达《关于撤销中共中央农村工作部的决定》。
\mnitem{2}邓拓,任中共北京市委书记处书记兼任中共华北局书记处候补书记,分管思想文化战线工作,主编北京市委理论刊物《前线》,拒绝在北京转载《评新编历史剧“海瑞罢官”》。

邓拓一九五七年曾主持《人民日报》工作,反对鸣放,一九五七年四月十日,毛泽东当面批评了他,“过去我说你是书生办报,不对,应当说是死人办报。”后被调离《人民日报》,邓拓怀恨在心,在与胡绩伟一次谈话中,攻击毛泽东“翻手为云,覆手为雨,自己讲过的话,可以翻脸不认账”,这种思想反映在《伟大的空话》、《专治健忘症》等《前线》的杂文中。

《剑桥中华人民共和国史》是世界上极具影响的国外研究中国历史的权威著作,代表了西方的观点,其中也指出,“邓拓——另一个曾撰文影射毛的领导的北京市官员。”
\mnitem{3}关于阶级斗争的决定,指一九六二年九月二十四日至二十七日在北京举行的中共八届十中全会发布的公报,“八届十中全会指出,在无产阶级革命和无产阶级专政的整个历史时期,在由资本主义过渡到共产主义的整个历史时期(这个时期需要几十年,甚至更多的时间)存在着无产阶级和资产阶级之间的阶级斗争,存在着社会主义和资本主义这两条道路的斗争。被推翻的反动统治阶级不甘心于灭亡,他们总是企图复辟。同时,社会上还存在着资产阶级的影响和旧社会的习惯势力,存在着一部分小生产者的自发的资本主义倾向,因此,在人民中,还有一些没有受到社会主义改造的人,他们人数不多,只占人口的百分之几,但一有机会,就企图离开社会主义道路,走资本主义道路。在这些情况下,阶级斗争是不可避免的。这是马克思列宁主义早就阐明了的一条历史规律,我们千万不要忘记。这种阶级斗争是错综复杂的、曲折的、时起时伏的,有时甚至是很激烈的。这种阶级斗争,不可避免地要反映到党内来。国外帝国主义的压力和国内资产阶级影响的存在,是党内产生修正主义思想的社会根源。在对国内外阶级敌人进行斗争的同时,我们必须及时警惕和坚决反对党内各种机会主义的思想倾向。”
\mnitem{4}打招呼,一九六六年三月十一日,中宣部常务副部长许立群根据彭真的意见在电话中向上海市委宣传部负责人杨永直责问发表姚文元的文章为什么不向中宣部打招呼,彭真说,“过去上海发姚文元的文章,连个招呼都不打,上海市委的党性到哪里去了?”
\mnitem{5}五人小组,一九六四年七月,中共中央成立了一个“五人小组”,在中央政治局、书记处领导下开展文化革命方面的工作,组长彭真(中共中央北京市委第一书记),副组长陆定一(国务院副总理,中宣部长兼文化部长),成员有康生(中共中央书记处书记),周扬(中宣部副部长),吴冷西(新华社社长兼《人民日报》社社长)。这个“五人小组”起初并没有称为“文化革命五人小组”,一直只称为“五人小组”。一九六六年二月十二日,《文化革命五人小组关于当前学术讨论的汇报提纲》(后被称为《二月提纲》)批转全党时,才出现“文化革命五人小组”这个名词。
\mnitem{6}去年九月,中共中央于一九六五年九月十八日至十月十二日在北京举行工作会议,在会议期间,十月十日,毛泽东同大区第一书记谈了话,提出要战备。各省要把“小三线”建设好。不要怕敌人不来,不要怕兵变,不要怕造反。又说,如中央出了修正主义,你们怎么办?很可能出,这是最危险的。中央出了修正主义,你们就造反。各省有了“小三线”,就可以造反嘛。过去有些人就是迷信国际,迷信中央。现在你们要注意,不管谁讲的,中央也好,中央局也好,省委也好,不正确的,你们可以不执行。
\mnitem{7}三家村,北京市委刊物《前线》在一九六一年九月二十日开设《三家村札记》杂文专栏,由吴晗、邓拓、廖沫沙三人轮流撰稿,吴晗出“吴”字,邓拓出“南”字(笔名“马南村”),廖沫沙出“星”(笔名“繁星”),统一署名“吴南星”。这里的“三家村”是指反党反社会主义的修正主义集团。
\mnitem{8}西风落叶下长安,唐朝诗人贾岛在《忆江上吴处士》中有一名句,“秋风生渭水,落叶满长安”,毛泽东同志在一九六三年一月九日写过一首《满江红·和郭沫若同志》:

小小寰球,有几个苍蝇碰壁。

嗡嗡叫,几声凄厉,几声抽泣。

蚂蚁缘槐夸大国,蚍蜉撼树谈何易。

正西风落叶下长安,飞鸣镝。

多少事,从来急;

天地转,光阴迫。

一万年太久,只争朝夕。

四海翻腾云水怒,五洲震荡风雷激。

要扫除一切害人虫,全无敌。

这里指修正主义者处于萧条凄凉的境地,我们进攻的号角就要吹响了,“飞鸣镝”。
\end{maonote}
