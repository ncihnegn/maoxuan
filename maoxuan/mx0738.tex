
\title{关于“九大”和整党问题的谈话}
\date{一九六七年十一月五日}
\thanks{这是毛泽东同志同中央文化革命小组的谈话。}
\maketitle


今天还是谈九大和整党问题。

关于九大问题,第一批反映已经来了,可否综合一下,通报各地,继续征求意见。

各地对整党问题,有些什么意见,也要继续收集。

不少地方的意见还没有反映上来,因为那里还乱着。

打了一年多仗,搞出了不少坏人。现在要打出一个党来。当然,坏人还是搞不干净,一次搞干净,是不可能的。

整党不可能在九大以前统统整好,九大以后,根据新的党纲、党章,继续重新整党、建党。

过去,一是听话(做驯服工具),一是生产好,有这两个条件就可以入党。听话要看听什么话,做驯服工具不行。一个支部书记说了就算,什么事情都不同大家研究,不商量,不向大家征求意见。支部作决议是要征求党员意见的,一次不够,再征求,要让党员讲意见,征求一次不够再征求。但是过去有很多地方不是这样,是个死党,一塘死水。

过去青年人入团可困难啦,比入党更困难。党团组织被一些人把持着。老党员把持党支部,老团员把持团支部。支部书记就象个皇帝。沙石峪大队党支部书记\mnote{1}就是这样,他把持全大队的党、政、财大权,搞贪污,两个孩子都上了大学。只有沙石峪是这样?我就不信。

有些党员官大了,架子就大了,不讲民主,不跟下级商量,遇事不先征求人家意见,自己说一通,喜欢训人。陈老总\mnote{2}就喜欢拿自己多少年的经验包括反对我的经验,去训人。

党员要那种朝气勃勃的,死气沉沉的、暮气沉沉的、就不要加入这个党。革命的党死气沉沉怎么行?有些人二三十岁,年岁不大,暮气却很大,到处住休养所,小病大养,无病也养。这次文化大革命,群众一冲,病没有了,坐喷气式飞机,反而治好了他的病。现在到医院去的很少了,起了很大的变化。才不要都听医生的话,我多少年没量血压了,前几天发烧三十八度五,医生要听心脏,我不让听,吃两片药就好了。当然,有的人是真有病,如心脏病,要治。

我把李大夫放下去支工,可有点意思。他们到北京针织厂。去的时候我就交代了几条,一是先做调查研究,当群众的小学生;二是不轻易表态;三是推动工人群众在革命原则基础上实行大联合。跑到一个地方就哇剌、哇刺干什么?那个厂有两派,斗得很厉害,开始两派都说他们搞阴谋,但他们顶住了,一直做调查研究。后来我说了几句话,就是工人阶级内部没有根本利害的冲突,救了他们,否则,人家要赶他们走。现在两派联合起来了。那个厂一个派一千二百多人,一派八百多人。八百人的一派说一千二百人的那派保得厉害。两派争位置,结果人多的一派采取高姿态,问题反而解决了,一派六个,一派七个,无派的出二人,共十五个人组成革命委员会筹备小组。还有一些人去化工厂,这个厂比较复杂、知识分子多,去的人被围攻。人家出了一条标语,我们去的人乱加评论,讲错了话,结果被人家抓住了。

文化大革命就是整党、整团、整政府、整军队,党、政、军、民、学,都整了。张本\mnote{3}这个人,谁知道她历史有问题。这次文化大革命搞出了很多问题,当然一次搞不干净,但搞出了不少。

我赞成康生同志意见,一个是叛徒、特务,一个是文化大革命当中表观很坏而又死不悔改的人,不能再参加党的组织生活,这就很宽了。

我们的党要吸收新血液。工人、贫农、红卫兵中的积极分子要吸收到党里来。旧血液中二氧化碳太多,要清除掉。一个人有动脉、静脉,通过心脏进行血液循环,还要通过肺部进行呼吸,呼出二氧化碳,吸进新鲜氧气,这就是吐故纳新。一个党也要吐故纳新,不清除废料,就没朝气。

怎样整党,你们拟出几条办法,也是征询意见性质,十一月能够发下去\mnote{6},十二月再发一指示。

恢复组织生活,不要恢复老样子。有些党员对无产阶级文化大革命不积极,一听说要恢复组织生活,他们又神气起来了。我看这些人要检讨。他们对无产阶级专政下的革命,对广大人民群众的革命运动不积极,什么理由?无非要做驯服工具。

这样的人都不要,就太多了,但要做检讨。有的人,不想当党员,就算了嘛。

接受新党员,要经过群众议一下。

党纲要修改。我在看联共党史,他们建党,建了几次。也不要象联共的党纲,写得太长。从政治经济学、商品二重性一直到剩余价值,写得很长,烦琐得很。我们党纲不要写得太长,长了看的人就很少。党纲写成一本书,工人怎么看?农民怎么看?横直是废话,会一散,就完了。我们的党纲,就是党章上开头的那个总纲。

组织纪律性还要有,但我们讲的是自觉的纪律。盲目服从,做驯服工具不行。

组织纪律性要有条件,第一,这个纪律是自觉的,第二,是联系群众的,第三,是在正确的政治路线领导之下的。做陈独秀、瞿秋白的驯服工具就是不行。八七会议我们反陈独秀时,就没有做陈独秀的驯服工具。

组织纪律是有条件的,相对的。无条件不行。列宁在《左派幼稚病》里就讲了三个前提。如果你是革命的,政治路线正确,人家自然就会服从。打了胜仗以后,战士都高兴,对领导上不讲闲话,打了败仗,闲话就多。人民解放军的纪律最好,打开锦州的时候,那么多苹果一个没动,这纪律就是建立在为革命的基础之上的。当然解放军战士也有个别犯纪律的,不是绝对的。

现在报上都讲好的,内部情况报导困难多,才不要相信。我就不相信天下都黑了。

刘邓互相合作,八大决议不经过大会主席团,也不征求我的意见就通过了。刚通过,我就反对。六三年搞了个十条\mnote{4},才隔三个月,他们又开会搞后十条\mnote{5},也不征求我的意见,我也没到会。邓小平要批,请军委准备一篇文章。我的意见还要把他同刘少奇区别一下,怎样把刘邓拆开来。

两件事:

关于九大,第一批反映材料综合一下,通报下去。

关于整党,整个什么样的党,老党员怎样一分为二,新党员入党,恢复组织生活又不要恢复老样子,这些问题怎么办,搞几条办法出来。

废除级别问题,也要谈一谈。

\begin{maonote}
\mnitem{1}沙石峪大队党支部书记,沙石峪位于中国河北省遵化市东南部,张贵顺,一九一四年生于遵化娘娘庄,后落户到沙石峪。一九四一年加入中国共产党,担任沙石峪村党支部书记近二十年。

“他领导人民开山造田。从土改起,通过社会主义合作社,到公社建立止,他在生产斗争的各个时期都领导沙石峪取得了巨大的发展。虽然这个人从一开始就当领导,但由于胜利使他冲昏了头脑,他的思想被腐蚀而停留在某个阶段。在一九六六年文化大革命以前,整个沙石峪就只有他自己的两个孩子上大学。这是一种特殊待遇,而这两个孩子的行动也表现出了他们是特殊的。文化大革命开始时,其中一个快要毕业了。她加入了一个红卫兵组织,并回家保她那掌权的父亲。这使人们非常不满。大队队员起来造反,反对他的领导,我们支持群众的革命行动。群众还反映了这样的意见。当这个村子开始盖房子时,规定每家造两套。这个大队长已有了两套,但他又给自己多造两套。一九六六年五月,当我和阿尔巴尼亚部长会议主席谢胡到那儿去时,我发现了这个情况。我奇怪大队长的房子怎么会比别人的大出两倍。在紧接而来的热潮中,人们揭发了这件事,指责他们的大队长砍了村里的树替自己盖房子。人们指责他这两件事是十分严肃的。大家把他从办公室轰出来,撒了他的党书记的职位。这是正确的。人们又进一步要把他清除出党,但这必须得到遵化县委的同意。六六年冬到六七年间,我第二次去遵化县时,听说他们还未决定他的党籍问题。在他的案件仍在考虑期间,他像大队别的社员一样劳动,要他干什么,他就干什么。然后在一批二斗三改造的运动中和在整党运动时期,他们决定保留他的党籍。最近我听说,人们决定恢复他的大队党支部书记的职务。”——选自《原美中友好协会会长韩丁一九七一年访华期间与周恩来总理的谈话记录》
\mnitem{2}陈老总,指陈毅。一九二九年六月二十二日,红四军前委召开了红四军党的第七次代表大会,由“红四军代理前委书记”陈毅主持,解决当时存在的“是否取消军委”争论,毛泽东明确反对取消军委,会上,陈毅剥夺了毛泽东的军事指挥权,后在实践中认识到错误,向党中央汇报工作后,当年十月份,三次写信请毛泽东回来担任前委书记。
\mnitem{3}张本,原任中华人民共和国科学技术委员会革命委员会主任。《科研批判》一九六八年四、五期合刊发表国家科委齐向红《同国民党反动派的一场殊死搏斗——彻底清算叛徒、特务、现行反革命张本的滔天罪行》,历数张本的反革命罪状。
\mnitem{3}十条,指前十条。一九六三年五月二十日,毛泽东同志主持制定了《中共中央关于目前农村工作中若干问题的决定(草案)》。
\mnitem{4}后十条,一九六三年九月二十日,刘少奇主持制定了《关于农村社会主义教育运动中一些具体政策的规定(草案)》。
\end{maonote}
