
\title{中美联合公报(“上海公报”)中国方面的声明}
\date{一九七二年二月二十八日}
\thanks{这是毛泽东同志审定的中美上海公报的中国立场部分。}
\maketitle


中国方面声明:哪里有压迫,哪里就有反抗。国家要独立,民族要解放,人民要革命,已成为不可抗拒的历史潮流。国家不分大小,应该一律平等,大国不应欺负小国,强国不应欺负弱国。中国决不做超级大国,并且反对任何霸权主义和强权政治。中国方面表示:坚决支持一切被压迫人民和被压迫民族争取自由、解放的斗争;各国人民有权按照自己的意愿,选择本国的社会制度,有权维护本国独立、主权和领土完整,反对外来侵略、干涉、控制和颠覆。一切外国军队都应撤回本国去。

中国方面表示:坚决支持越南、老挝、柬埔寨三国人民为实现自己的目标所作的努力,坚决支持越南南方共和临时革命政府的七点建议\mnote{1}以及在今年二月对其中两个关键问题的说明\mnote{2}和印度支那人民最高级会议联合声明\mnote{3};坚决支持朝鲜民主主义人民共和国政府一九七一年四月十二日提出的朝鲜和平统一的八点方案\mnote{4}和取消“联合国韩国统一复兴委员会”的主张;坚决反对日本军国主义的复活和对外扩张,坚决支持日本人民要求建立一个独立、民主、和平和中立的日本的愿望;坚决主张印度和巴基斯坦按照联合国关系印巴问题的决议,立即把自己的军队全部撤回到本国境内以及查谟和克什米尔停火线的各自一方,坚决支持巴基斯坦政府和人民维护独立、主权的斗争以及查谟和克什米尔人民争取自决权的斗争。

\begin{maonote}
\mnitem{1}一九七一年七月一日越南南方共和临时革命政府在巴黎会议上提出七点和平倡议:

一、如果美国规定撤军的限期,双方的全部战俘就可释放。

二、成立三方民族团结政府。

三、由越南人自己解决南方的越南武装力量问题。

四、北越和南越逐步统一。

五、南越奉行和平和中立的外交政策。

六、美国政府必须对在北越和南越造成的损失和破坏承担全部责任。

七、国际上来保证所达成的协议。指出美国政府必须确定撤军的期限;如果美国政府规定在一九七一年内从越南南方撤走全部美军及其仆从军,各方就可以同时商定关于从越南南方安全撤走全部美军及其仆从军的事宜和关于释放在战争中被俘的各方军事人员和非军事人员的事宜。美国政府还必须尊重越南南方人民的自决权,停止干涉越南南方内政,停止支持以阮文绍为首的傀儡政权。这个倡议,为和平解决越南问题提供了正确的途径。
\mnitem{2}一九七二年二月初,北越就原来的“七点建议”通知美国,准备把释放全部战俘同美军和盟军从南越撤退这两件事协调一致起来,要求美军和盟军在一九七一年内全部撤退完毕,并形成了新的“九点建议”:

一、在一九七一年内全部撤退美国和盟国的军队。

二、在撤军的同时释放俘获的全部士兵与平民,并且在军队全部撤退时完成释放工作。

三、美国停止支持南越阮文绍总统的政权,用一个新的政府代替这个政权,南越临时革命政府可以就这个国家的问题同新政府举行会谈。

四、对在战争中造成的损失,美国向河内和临时革命政府偿付赔款。

五、停止美国的干涉,而且尊重一九五四年和一九六二年关于印度支那问题和老挝问题的日内瓦协议。

六、印度支那各国之间的问题“将由印度支那各方在互相尊重、独立、主权、领土完整和互不干涉内政的基础上加以解决”。

七、在就上述各问题缔结协议后,所有各方将实行停火。

八、将实行国际监督。

九、为了确保印度支那各国人民的基本民族权利以及南越、老挝和柬埔寨的中立受到尊重,为了在这个地区确立持久和平,国际保证是必要的。
\mnitem{3}一九七〇年四月二十四日至二十五日,印度支那人民最高级会议在中国、老挝和越南边境地区举行,柬埔寨、老挝、越南南方共和国、越南民主共和国三国四方领导人各自率代表团出席会议,并发表《联合声明》,谴责美国扩大对越南和老挝的侵略战争,在柬埔寨发动政变,重申四方领导人对解决印度支那问题的严正立场,号召印度支那人民团结起来,捍卫民族权利,打败美国侵略者及其一切走狗,使印度支那成为真正符合三国人民与世界人民愿望的独立和平地区。
\mnitem{4}一九七一年四月十二日,朝鲜最高人民会议提出了《关于统一祖国的八项救国方案》。其主要内容如下:

一、美军撤出南朝鲜;

二、南北方军队均裁减至十万人以下;

三、废除韩美、韩日条约,确立民族自主权;

四、举行南北大选,建立统一的中央政府;

五、保障政治活动自由;

六、作为过渡性措施,实施南北联邦制;

七、广泛进行南北经济、文化、人员交流;

八、召开南北政治协商会议。
\end{maonote}
