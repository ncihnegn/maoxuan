
\title{官僚主义者阶级是革命对象}
\date{一九六五年一月二十九日}
\thanks{这是毛泽东同志在对陈正人\mnote{1}关于社教蹲点情况报告的批语和批注。}
\maketitle


管理也是社教。如果管理人员不到车间小组搞三同,拜老师学一门至几门手艺,那就一辈子会同工人阶级处于尖锐的阶级斗争状态中,最后必然被工人阶级把他们当作资产阶级打倒。不学会技术,长期当外行,管理也搞不好,以其昏昏,使人昭昭,是不行的。

官僚主义者阶级与工人阶级和贫下中农是两个尖锐对立的阶级。这些人\mnote{2}是已经变成或者正在变成吸工人血的资产阶级分子,他们怎么会认识呢?这些人是斗争对象,革命对象,社教运动决不能依靠他们,我们能依靠的,只是那些同工人没有仇恨,而又有革命精神的干部。

\begin{maonote}
\mnitem{1}陈正人,时任农业机械部(一九六五年一月改为第八机械工业部)部长。
\mnitem{2}在印发这段批注的时候,薄一波在“这些人”的后面加了一个注释:“指那些企业领导人中坚决走资本主义道路的人”。
\end{maonote}
