
\title{中华人民共和国国防部告台湾同胞书}
\date{一九五八年十月六日}
\thanks{这篇文章是毛泽东起草的。}
\maketitle


\mxname{台湾、澎湖、金门、马祖军民同胞们:}

我们都是中国人。三十六计,和为上计。金门战斗,属于惩罚性质。你们的领导者们过去长时期间太猖狂了,命令飞机向大陆乱钻,远及云、贵、川、康\mnote{1}、青海,发传单,丢特务,炸福州,扰江浙。是可忍,孰不可忍?因此打一些炮,引起你们注意。台、澎、金、马是中国领土,这一点你们是同意的,见之于你们领导人的文告,确实不是美国人的领土。台、澎、金、马是中国的一部分,不是另一个国家。世界上只有一个中国,没有两个中国。这一点,也是你们同意的,见之于你们领导人的文告。你们领导人与美国人订立军事协定\mnote{2},是片面的,我们不承认,应予废除。美国人总有一天肯定要抛弃你们的。你们不信吗?历史巨人会要出来作证明的。杜勒斯九月三十日的谈话\mnote{3},端倪已见。站在你们的地位,能不寒心?归根结底,美帝国主义是我们的共同敌人。十三万金门军民,供应缺乏,饥寒交迫,难为久计。为了人道主义,我已命令福建前线,从十月六日起,暂以七天为期,停止炮击,你们可以充分地自由地输送供应品,但以没有美国人护航为条件。如有护航,不在此例。你们与我们之间的战争,三十年了,尚未结束,这是不好的。建议举行谈判,实行和平解决。这一点,周恩来总理在几年前已经告诉你们了。这是中国内部贵我两方有关的问题,不是中美两国有关的问题。美国侵占台澎与台湾海峡,这是中美两方有关的问题,应当由两国举行谈判解决,目前正在华沙举行\mnote{4}。美国人总是要走的,不走是不行的。早走于美国有利,因为它可以取得主动。迟走不利,因为它老是被动。一个东太平洋国家,为什么跑到西太平洋来了呢?西太平洋是西太平洋人的西太平洋,正如东太平洋是东太平洋人的东太平洋一样。这一点是常识,美国人应当懂得。中华人民共和国与美国之间并无战争,无所谓停火。无火而谈停火,岂非笑话?台湾的朋友们,我们之间是有战火的,应当停止,并予熄灭。这就需要谈判。当然,再打三十年,也不是什么了不起的大事,但是究竟以早日和平解决较为妥善。何去何从,请你们酌定。

中华人民共和国国防部部长

一九五八年十月六日上午一时

\begin{maonote}
\mnitem{1}康,指西康省,一九五五年撤销。撤销时,原辖区划归四川省。
\mnitem{2}指美国与台湾国民党政府一九五四年签订的《共同防御条约》。
\mnitem{3}美国国务卿杜勒斯一九五八年九月三十日对记者发表谈话,重申美国在台湾问题上所持的国共“双方放弃武力”的立场,批评蒋介石政府在金门、马祖等岛屿上保持大量军队是不明智和不谨慎的,并承认蒋介石反攻大陆是一个“假设成分很大”的计划,认为“只靠他们自己的力量,他们是不会回到那里去的”。当有记者问到如果中国共产党方面作出某些让步,那末美国的对台湾政策是否会有所改变时,杜勒斯说:“我们在这些方面的政策是灵活的,是适应于我们必须应付的局势的。如果我们必须应付的局势改变了,我们的政策也会随之改变。”
\mnitem{4}指中美大使级会谈。一九五五年四月二十三日,周恩来总理在亚非会议八国代表团团长会议上声明:中国政府愿意同美国政府谈判,讨论和缓远东紧张局势问题,特别是和缓台湾地区紧张局势问题。同年七月二十五日,中美双方就举行大使级会谈达成协议,并于八月一日在瑞士日内瓦进行首次会谈。此后由于美方缺乏诚意,会谈中断。一九五八年八月对金门炮击开始后,美国政府公开表示准备恢复会谈,双方随即于九月十五日在波兰华沙恢复会谈。迄至一九七〇年二月二十日,中美大使级会谈共举行了一百三十六次。由于美方坚持干涉中国内政的立场,会谈在和缓和消除台湾地区紧张局势问题上未取得任何进展。
\end{maonote}
