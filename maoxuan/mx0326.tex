
\title{愚公移山}
\date{一九四五年六月十一日}
\thanks{这是毛泽东在中国共产党第七次全国代表大会上的闭幕词。}
\maketitle


我们开了一个很好的大会。我们做了三件事:第一,决定了党的路线,这就是放手发动群众,壮大人民力量,在我党的领导下,打败日本侵略者,解放全国人民,建立一个新民主主义的中国。第二,通过了新的党章。第三,选举了党的领导机关——中央委员会。今后的任务就是领导全党实现党的路线。我们开了一个胜利的大会,一个团结的大会。代表们对三个报告\mnote{1}发表了很好的意见。许多同志作了自我批评,从团结的目标出发,经过自我批评,达到了团结。这次大会是团结的模范,是自我批评的模范,又是党内民主的模范。

大会闭幕以后,很多同志将要回到自己的工作岗位上去,将要分赴各个战场。同志们到各地去,要宣传大会的路线,并经过全党同志向人民作广泛的解释。

我们宣传大会的路线,就是要使全党和全国人民建立起一个信心,即革命一定要胜利。首先要使先锋队觉悟,下定决心,不怕牺牲,排除万难,去争取胜利。但这还不够,还必须使全国广大人民群众觉悟,甘心情愿和我们一起奋斗,去争取胜利。要使全国人民有这样的信心:中国是中国人民的,不是反动派的。中国古代有个寓言,叫做“愚公移山”。说的是古代有一位老人,住在华北,名叫北山愚公。他的家门南面有两座大山挡住他家的出路,一座叫做太行山,一座叫做王屋山。愚公下决心率领他的儿子们要用锄头挖去这两座大山。有个老头子名叫智叟的看了发笑,说是你们这样干未免太愚蠢了,你们父子数人要挖掉这样两座大山是完全不可能的。愚公回答说:我死了以后有我的儿子,儿子死了,又有孙子,子子孙孙是没有穷尽的。这两座山虽然很高,却是不会再增高了,挖一点就会少一点,为什么挖不平呢?愚公批驳了智叟的错误思想,毫不动摇,每天挖山不止。这件事感动了上帝,他就派了两个神仙下凡,把两座山背走了\mnote{2}。现在也有两座压在中国人民头上的大山,一座叫做帝国主义,一座叫做封建主义。中国共产党早就下了决心,要挖掉这两座山。我们一定要坚持下去,一定要不断地工作,我们也会感动上帝的。这个上帝不是别人,就是全中国的人民大众。全国人民大众一齐起来和我们一道挖这两座山,有什么挖不平呢?

昨天有两个美国人要回美国去,我对他们讲了,美国政府要破坏我们,这是不允许的。我们反对美国政府扶蒋反共的政策。但是我们第一要把美国人民和他们的政府相区别,第二要把美国政府中决定政策的人们和下面的普通工作人员相区别。我对这两个美国人说:告诉你们美国政府中决定政策的人们,我们解放区禁止你们到那里去,因为你们的政策是扶蒋反共,我们不放心。假如你们是为了打日本,要到解放区是可以去的,但要订一个条约。倘若你们偷偷摸摸到处乱跑,那是不许可的。赫尔利已经公开宣言不同中国共产党合作\mnote{3},既然如此,为什么还要到我们解放区去乱跑呢?

美国政府的扶蒋反共政策,说明了美国反动派的猖狂。但是一切中外反动派的阻止中国人民胜利的企图,都是注定要失败的。现在的世界潮流,民主是主流,反民主的反动只是一股逆流。目前反动的逆流企图压倒民族独立和人民民主的主流,但反动的逆流终究不会变为主流。现在依然如斯大林很早就说过的一样,旧世界有三个大矛盾:第一个是帝国主义国家中的无产阶级和资产阶级的矛盾,第二个是帝国主义国家之间的矛盾,第三个是殖民地半殖民地国家和帝国主义宗主国之间的矛盾\mnote{4}。这三种矛盾不但依然存在,而且发展得更尖锐了,更扩大了。由于这些矛盾的存在和发展,所以虽有反苏反共反民主的逆流存在,但是这种反动逆流总有一天会要被克服下去。

现在中国正在开着两个大会,一个是国民党的第六次代表大会,一个是共产党的第七次代表大会。两个大会有完全不同的目的:一个要消灭共产党和中国民主势力,把中国引向黑暗;一个要打倒日本帝国主义和它的走狗中国封建势力,建设一个新民主主义的中国,把中国引向光明。这两条路线在互相斗争着。我们坚决相信,中国人民将要在中国共产党领导之下,在中国共产党第七次大会的路线的领导之下,得到完全的胜利,而国民党的反革命路线必然要失败。


\begin{maonote}
\mnitem{1}指在中国共产党第七次全国代表大会上,毛泽东所作的政治报告、朱德所作的军事报告和刘少奇所作的关于修改党章的报告。
\mnitem{2}愚公移山的故事,见《列子·汤问》。原文是:“太行、王屋二山,方七百里,高万仞。本在冀州之南,河阳之北。北山愚公者,年且九十,面山而居。惩山北之塞,出入之迂也,聚室而谋曰:吾与汝毕力平险,指通豫南,达于汉阴,可乎?杂然相许。其妻献疑曰:以君之力,曾不能损魁父之丘,如太行王屋何?且焉置土石?杂曰:投诸渤海之尾,隐土之北。遂率子孙荷担者三夫,叩石垦壤,箕畚运于渤海之尾。邻人京城氏之孀妻,有遗男,始龀,跳往助之。寒暑易节,始一反焉。河曲智叟,笑而止之,曰:甚矣,汝之不惠。以残年余力,曾不能毁山之一毛,其如土石何?北山愚公长息曰:汝心之固,固不可彻,曾不若孀妻弱子。虽我之死,有子存焉;子又生孙,孙又生子;子又有子,子又有孙。子子孙孙,无穷匮也,而山不加增,何苦而不平?河曲智叟亡以应。操蛇之神闻之,惧其不已也,告之于帝。帝感其诚,命夸蛾氏二子负二山,一厝朔东,一厝雍南。自此,冀之南,汉之阴,无陇断焉。”
\mnitem{3}赫尔利(一八八三——一九六三),美国共和党人。他在一九四四年十一月底被任命为美国驻中国大使,因支持蒋介石的反共政策而受到中国人民的坚决反对,于一九四五年十一月被迫宣布离职。一九四五年四月二日他在华盛顿国务院记者招待会上的谈话中,公开宣言不同中国共产党合作。参见本卷\mxart{赫尔利和蒋介石的双簧已经破产}和\mxart{评赫尔利政策的危险}。
\mnitem{4}参见斯大林《论列宁主义基础》第一部分《列宁主义的历史根源》(《斯大林选集》上卷,人民出版社1979年版,第186—187页)。
\end{maonote}
