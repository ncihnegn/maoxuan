
\title{祝贺印度支那三国抗美救国战争取得胜利}
\date{一九七五年四月、十二月}
\thanks{这是毛泽东同志祝贺印度支那三国抗美救国战争\mnote{1}取得胜利的电报。}
\maketitle


\date{一九七五年四月十七日}
\section*{(一)毛泽东等祝贺柬埔寨民族解放人民武装力量完全解放金边的电报}

\mxname{柬埔寨国家元首}

\mxname{柬埔寨民族统一阵线主席}

\mxname{诺罗敦·西哈努克亲王,}

\mxname{柬埔寨民族统一阵线中央政治局主席}

\mxname{柬埔寨王国民族团结政府首相}

\mxname{宾努亲王,}

\mxname{柬埔寨王国民族团结政府副首相兼国防大臣}

\mxname{民族解放人民武装力量总司令}

\mxname{乔拉潘先生阁下:}

在柬埔寨民族解放人民武装力量的强大攻势下,金边已经完全解放。捷报传来,人心振奋。我们代表中国共产党、中国政府和中国人民向你们,向柬埔寨民族统一阵线、柬埔寨王国民族团结政府、英雄的柬埔寨民族解放人民武装力量和全体柬埔寨人民表示最热烈的祝贺和最崇高的敬意。

五年前,美帝国主义阴谋策划了朗诺叛国集团的反动政变,甚至悍然出兵柬埔寨,妄图把一个独立、和平、中立的柬埔寨变为它的殖民地。富有光荣反帝革命传统的柬埔寨人民不畏强暴,揭竿而起,团结在以诺罗敦·西哈努克亲王为主席的柬埔寨民族统一阵线周围,进行了整整五年英勇顽强的战斗,克服了重重困难,终于取得了民族解放战争的决定性胜利。柬埔寨人民通过武装斗争所取得的这一伟大胜利,再一次雄辩地证明,只要坚持正确的道路,弱国就一定能够打败强国,小国就一定能够打败大国。

柬埔寨人民在长期武装斗争中表现出来的顽强革命精神和大无畏英雄气概,为世界革命人民树立了光辉榜样,赢得了各国人民的钦佩和赞扬。柬埔寨人民的伟大胜利,不仅为印度支那人民的反帝斗争作出了重大的贡献,而且有力地鼓舞和推动着全世界一切被压迫民族和被压迫人民的革命斗争。

中柬两国人民是亲如手足的兄弟。我们两国人民在长期的反帝斗争中一贯互相同情、互相鼓舞、互相支持,结成了深厚的战斗友谊。中国人民对柬埔寨人民的胜利就像对自己的胜利一样感到由衷的高兴。兄弟的柬埔寨人民可以相信,在今后的斗争中,中国人民将永远同你们站在一起,共同前进。

金边的解放,标志着柬埔寨民族解放斗争进入了一个新的历史阶段。我们深信,柬埔寨人民在以诺罗敦·西哈努克亲王为主席的柬埔寨民族统一阵线的旗帜下,加强整个民族和全体人民的大团结,继续英勇奋斗,就一定能够不断取得新的更大的胜利。柬埔寨将以崭新的面貌出现在世界的东方。

中国共产党中央委员会主席毛泽东

中华人民共和国全国人民代表大会常务委员会委员长朱德

中华人民共和国国务院总理周恩来

一九七五年四月十七日于北京

\date{一九七五年四月三十日}
\section*{(二)毛泽东等祝贺越南南方军民彻底摧毁南越傀儡政权解放西贡的电报}

\mxname{西贡}

\mxname{越南南方民族解放阵线中央委员会主席团主席阮友寿阁下,}

\mxname{越南南方共和临时革命政府主席黄晋发阁下,}

\mxname{河内}

\mxname{越南民主共和国主席孙德胜同志,}

\mxname{越南劳动党中央委员会第一书记笋同志,}

\mxname{越南民主共和国国会常务委员会主席长征同志,}

\mxname{越南民主共和国政府总理范文同同志:}

越南南方军民经过长期的英勇奋战,彻底摧毁了南越傀儡政权,终于解放了西贡。这是越南人民和印度支那人民坚持长期革命武装斗争的伟大胜利。喜讯传来,中国人民感到无比欢欣鼓舞。我们代表中国共产党、中国政府和中国人民向你们,向越南南方民族解放阵线和越南南方共和临时革命政府,向越南劳动党和越南民主共和国政府,向越南南、北方全体人民,表示最热烈的祝贺和最崇高的敬意。

英雄的越南人民,为了祖国的独立和民族的解放,高举胡志明主席“决战决胜”的光辉旗帜,不屈不挠,前赴后继,进行了几十年艰苦卓绝的斗争,先后打败了美帝国主义及其走狗的“特种战争”、“局部战争”和“越南化”战争,取得了一个又一个的伟大胜利。巴黎协定签订之后,越南人民在新的形势下,对于西贡傀儡集团在美国政府的支持下破坏巴黎协定、拒绝实现民族和睦、疯狂屠杀人民的罪行,进行了坚决的自卫反击战,终于以革命战争打败了反革命战争,取得了完全解放越南南方的辉煌胜利。你们的胜利,开创了越南解放新的时代,具有重大的历史意义和国际意义。你们的胜利极大地鼓舞了一切斗争中的被压迫民族和被压迫人民,为全世界人民的反帝革命事业树立了光辉的榜样。越南人民的胜利,再次有力地证明,一个国家的人民,哪怕是一个小国的人民,在维护祖国独立、自由和争取民族解放的正义事业中,只要敢于起来斗争,敢于拿起武器,用正义战争反对非正义战争,不怕困难,不怕牺牲,不怕挫折,坚持下去,就一定能够打败任何貌似强大的敌人,取得斗争的最终胜利。

中、越两国是唇齿相依的亲密邻邦,两国人民是患难过去的长期革命斗争中,我们两国人民互相支持,互相鼓舞。在今后的岁月里,中国人民仍将坚定不移地同越南人民团结在一起,战斗在一起。我们衷心祝愿越南南方人民在继续完成民族民主革命的斗争中,不断取得新的更大的胜利。一个独立、自由、统一、繁荣的新越南一定实现。

中国共产党中央委员会主席毛泽东

中华人民共和国全国人民代表大会常务委员会委员长朱德

中华人民共和国国务院总理周恩来

一九七五年四月三十日于北京

\date{一九七五年十二月五日}
\section*{(三)毛泽东等祝贺老挝人民民主共和国宣告成立的电报}

\mxname{万象}

\mxname{老挝人民民主共和国主席}

\mxname{老挝最高人民议会主席}

\mxname{苏发努冯阁下,}

\mxname{老挝人民民主共和国政府总理}

\mxname{凯山·丰威汉阁下:}

在老挝人民民主共和国宣告成立及你们荣任共和国主席、最高人民议会主席和政府总理的时候,我们代表中国共产党、中国政府和中国人民向你们,向老挝人民革命党、老挝人民民主共和国政府和兄弟的老挝人民表示热烈的祝贺和崇高的敬意。

老挝人民民主共和国的诞生,是老挝人民长期英勇斗争的光辉成果。几十年来,英雄的老挝人民在老挝人民革命党的领导下,为了祖国的独立和民族的解放,不畏强敌,不怕困难,坚持武装斗争,经过艰苦曲折的道路,终于战胜了帝国主义侵略者和国内极右反动势力,取得了民族民主革命的伟大胜利。老挝革命从此进入了一个崭新的历史阶段。你们的胜利,不仅对印度支那人民的革命事业作出了积极贡献,而且有力地鼓舞了世界被压迫民族和被压迫人民争取独立和解放的正义斗争。

中老两国是亲密的友好邻邦,两国人民在长期的反帝革命斗争中同甘苦,共患难,结成了深厚的战斗友谊。中国人民对老挝人民取得的每一个胜利都看作是自己的胜利,感到由衷的高兴。我们相信,富有革命斗争传统的老挝人民,在老挝人民革命党的领导下,坚持独立自主、自力更生,不断克服前进道路上的困难,必将把老挝建设成独立、民主和繁荣的国家。我们祝愿,中老两国人民的革命友谊不断巩固和发展。

中国共产党中央委员会主席毛泽东

中华人民共和国全国人民代表大会常务委员会委员长朱德

中华人民共和国国务院总理周恩来

\begin{maonote}
\mnitem{1}印度支那三国抗美救国战争,二十世纪六十年代至七十年代中期,越南、柬埔寨、老挝三国人民抗击美国侵略,维护国家独立的民族解放战争。印度支那三国人民抗法斗争胜利不久,美国破坏一九五四年的《日内瓦协议》,于一九五五年十月在越南南方策动亲美势力废黜亲法的保大皇帝,建立吴庭艳傀儡政权,制造了正式分裂越南的局面。一九五九年起南越人民开展武装斗争。一九六〇年上半年,南越西南部游击区已连成一片,农村建立起基层革命政权。同年十二月二十日,越南南方民族解放阵线宣告成立,美国为长期霸占南越,从一九六一年起发动由美国出钱、出枪,越南出人的“特种战争”。南越人民在民族解放阵线领导下,粉碎美伪军的“全面进攻”、“重点扫荡”。到一九六四年底,歼灭和击溃包括三千多名美军在内的几十万敌军,摧毁敌人百分之八十以上的“战略村”,解放三分之二以上的南方地区。

美国发动的“特种战争”彻底失败后,于一九六四年八月初制造“北部湾事件”,以北越鱼雷艇袭击美舰为借口,对越南北方进行狂轰滥炸。翌年起大批美军直接参加侵越战争,将“特种战争”升级为“南打北炸”为特点的“局部战争”。越南北方军民边生产边战斗,全力支援南方同胞抗战。南方军民英勇战斗,连续粉碎敌人的旱季攻势。虽然美国派遣部队陆续增加到五十多万,仍无法取胜。一九六八年三月,约翰逊政府被迫宣布部分停止对北越的轰炸。五月,越美在巴黎举行会谈。十一月,美国宣布完全停止对北越轰炸,“局部战争”失败。

一九六九年起,尼克松政府开始调整对外政策,为在欧洲和中东集中对付苏联,决定收缩力量,从南越逐步撤退,采取“以越南人打越南人”的“越南化”策略。美国侵略越南的同时,在老挝扶植亲美势力,把老挝划入“东南亚集体防务条约组织”的“保护”范围。老挝人民为捍卫民族独立,在爱国战线党领导下,坚持武装斗争,屡次打败敌人武装进攻。一九六四年四月,美国策动右派政变。五月美国飞机轰炸解放区,直接进行武装干涉。一九六五年三月,爱国战线党和爱国中立力量举行全国政治协商会议,号召全国人民奋起抗战。至六十年代末,老挝人民解放军英勇抗战,解放区迅速扩大。一九五四年日内瓦会议后,柬埔寨在西哈努克亲王领导下,坚持推行和平中立的外交政策。“北部湾事件”后,美国开始轰炸柬埔寨,柬美断绝外交关系。一九七〇年三月十八日,美国策动朗诺—施里玛达集团乘西哈努克出国访问之际发动军事政变,颠覆王国政府。四月,派遣近十万名美军和西贡伪军入侵柬埔寨。柬埔寨人民立即拿起武器,奋勇抵抗。至此美国把侵略战火扩大到包括柬埔寨在内的整个印支三国。为了联合抗美,印度支那三国四方领导人于一九七〇年四月举行印度支那人民最高级会议。会议号召印支人民加强团结,英勇战斗,把抗美救国战争进行到底。同年五月,柬埔寨王国民族团结政府在北京成立。经过五年艰苦战斗,柬埔寨爱国军民消灭大量美国侵略军和伪军。一九七五年四月十七日,爱国武装力量解放金边,四月十九日宣告全国解放。柬埔寨抗美救国战争首先取得胜利。

在越南战场上遭到惨败的美国于一九七三年一月二十七日签订《巴黎协定》,同意结束战争,撤出在越南的美军及其盟国全部军队。但美仍在南方支持阮文绍集团。一九七五年四月,南方军民发起总进攻,四月三十日解放西贡。五月一日,十七度线以南的国土全部解放。抗美战争胜利结束,越南举行全国普选,产生了统一的国会。一九七六年七月二日,越南正式宣布南北两方实现统一,改国名为越南社会主义共和国。

老挝爱国军民在坚持进行武装斗争同时,为寻求和平解决问题进行不懈努力。一九七三年二月二十一日,老挝爱国力量代表和万象政府代表签订《万象协定》,老挝人民一面巩固发展解放区,一面打击右派势力军事冒险行动。一九七五年五月以后,老挝人民在越、柬人民胜利的鼓舞下,展开全国范围的夺权斗争。八月底斗争胜利完成。十二月一日,老挝全国人民代表大会在万象举行,宣布废除君主制,成立老挝人民民主共和国。

至此,印支三国人民抗美救国战争取得最后胜利。
\end{maonote}
