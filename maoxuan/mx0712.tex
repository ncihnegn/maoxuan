
\title{工作组要撤,文化革命要依靠革命师生}
\date{一九六六年七月二十四日}
\thanks{这是毛泽东同志在政治局与中央文革的联席会议上的谈话。}
\maketitle


开两个会,一个是大区书记参加的中央工作会,另一个是起草文化大革命文件的会。讲了一些文化大革命的工作,主要讲工作组要撤,要改变派工作组的政策。前几天讲工作组不行,前市委烂了,中宣部烂了,文化部烂了,高教部也烂了,《人民日报》也不行,依靠谁呢?主要依靠广大的革命师生。六月一日公布大字报\mnote{1},就考虑到非如此不可。文化革命就得靠他们去做,不靠他们靠谁?你去,不了解情况,两个月也不了解,半年也不了解,一年也不行。如翦伯赞\mnote{2}写了那么多书,你们谁看过,小平看过吗?

(邓小平:没有)

哦,没有,知道你也没有。那么你能斗得了他吗?只有他们能了解情况,我去也不行,只有依靠革命师生。

(陈伯达:当前抓运动的人有这么几个理论和方针,他们把放手发动群众和党的领导绝对对立起来,认为强调放手发动群众,就是不要党的领导。)

乱弹琴!不懂马列主义的基本常识!少奇、小平开口闭口怕乱,你们就是怕字当头,乱有什么了不起?现在停课又管饭吃,吃了饭要发热,要闹事,不叫闹事干什么?闹事就是革命。只有依靠他们搞。照目前办法搞下去,两个月冷冷清清,搞到何年何月?

昨天说,你们要改变派工作组的政策。现在工作组起了什么作用呢?一起阻碍作用,二不会:一不会斗,二不会改。我也不行。现在无非是搞革命,一是斗坏人,二是革思想。文化大革命,批判资产阶级思想权威,陆平\mnote{3}有多大斗头?李达\mnote{4}有多大斗头?翦伯赞出那么多书,你能斗得了他?群众写对联讲他是:“庙小神灵大,池浅王八多”。搞他你们哪个行?我不行,各省也不行。什么教学改革,我也不懂,只有依靠群众,然后集中起来,所以工作组非撤不可。如果照原来那样搞下去,是搞不出什么名堂来的!

工作组改成联络员或是叫顾问,你们讲顾问权大,那还叫联络员。工作组一个多月,起阻碍革命的作用,实际上是帮了反革命。有的工作组是坐山观虎斗,看着学生斗学生。西安交大限制人家打电话、打电报,限制人家上北京。要在文件上写上,可打电话,可打电报,可派人到中央。党章早就有了嘛!

(康生:工作组公开传达少奇、小平的指示,要大家绝对相信工作组,说反对工作组就是反革命,大搞排除干扰,实际上就是挑动群众斗群众。)

中央好多部,没做多少好事,相反文革小组却做了不少好事,名声很大。据说南京《新华日报》被学生包围了,就不得了了,好像闯了多大的祸,我看可以包围,三天不出报,有什么了不起?你不革命就革到你头上来。为什么不准包围省市委、报馆、国务院?好人来了,你们不见,你们不出去,我去见。你们又派小干部,自己不出去,我出去。总之,你们是怕字当头,怕革命,怕动刀动枪,都不下去,不到有乱的地方看看。李雪峰和吴德\mnote{5}来了吧?

(李雪峰站起来:我在。)

李雪峰、吴德你们不去看大字报,天天忙具体事务,没有感性知识,如何指导运动?南京大学三次大辩论,我看不错。所有到会的人都要到出乱子的地方去。有人怕讲话,讲话有什么了不起?学生们围上来,叫讲话就讲几句:我们是来学习的,是来支持你们革命的,召之即来,随叫随到,以后再来。

(刘少奇:主席,没有工作组了,学校出现乱打人乱斗人怎么办?)

你叫革命的师生一点毛病没有?你搞了一二个月了,却一点感性认识都没有,你去就是叫围嘛,广播学院、北师大打人问题,有人怕挨打,叫工作组保护自己,怕什么,没有死人嘛!左派挨打受锻炼,右派挨打就挨几下嘛,但这不能成为不撤工作组的理由。教科书你工作组能弄出来吗?不行,还得靠本单位的人才能改。这一点大多数人都通了,你还不通?

总之,工作组是一不能斗,二不能改,半年不行,一年也不行。只有本单位的人才能斗,才能改。斗就是破,改就是立。教科书年年编不出来,我看可以去繁就简,错误的去掉,加可能来不及了,要加就加中央社论和通知。

(周恩来:我们在这个问题上听毛主席的,工作组马上撤,越快越好,撤得越快越主动。另外,我提议教科书加上毛主席的著作。)

那个是方向、指南,不能当了教条。如:处理广播学院打人,哪本书上有?哪个将军打仗还翻书?现在这个阶段要把方向转过来。

文化革命委员会,要包括左、中、右,右派也要有几个。如翦伯赞可以被右派用,也可被左派用,是个活字典,但不能集中,象中华书局\mnote{6}那样可搞个训练班,当活字典(只要不是民愤极大的)。代表会、革命委员会都要有个对立面,常委就不能要了。

(李雪峰:我们市委也成立了革命委员会,人数还不少。)

除你外,那个市委,人员不要多,多了他们就要“革命”、打电话、出报表。我这里就一个人嘛,很好嘛。现在部长很多人都有秘书,统统去掉。我到延安前就没有。市委机关可搞个收发。少奇同志,你夫人不要当秘书了,下去劳动嘛。国务院的部有的可改为科,庞大机关,历来没有用。

(邓小平:没有了工作组,黑帮复辟怎么办?右派闹事怎么办?)

有些是要复辟,复辟也不要紧。我们有些部长是不是就那么可靠?有些部、报馆究竟是谁掌握呀?我看还不如有些学校呢!你们没想想,学校的学生一不上课,二管饭吃,三就要闹事嘛,闹事就是革命。工作组使起了阻碍革命的作用,清华、北大的工作组就是这样。我们不是正在制定关于文化大革命的文件吗?我看文件上要写明只有行凶、杀人、放火、放毒的才叫反革命,写大字报、写反动标语的不能抓。有人写“拥护党中央,打倒毛泽东”,你抓他干什么?他还拥护党中央嘛,历史反革命留下用,表现不好的就斗他嘛。不准打人,叫他们放嘛!贴几张大字报、几条反动标语,怕什么?!

总之,工作组要撤,出乱子不可怕。

\begin{maonote}
\mnitem{1}一九六六年五月二十五日下午二时许,北京大学哲学系聂元梓、宋一秀、夏剑豸、杨克明、赵正义、高云鹏、李醒尘七人,在大饭厅东墙上贴出了题为《宋硕、陆平、彭珮云在文化革命中究竟干些什么?》的大字报。毛泽东于一九六六年六月一日批示“此文可以由新华社全文广播,在全国报刊发表,十分必要。北京大学这个反动堡垒从此可以开始打破”。
\mnitem{2}翦伯赞,历史学家,时任北大副校长,著述颇多,主编:《中国史纲要》《中国古代史教学参考资料》《中国近代史资料丛刊》,著作:《中国历史哲学教程》《对处理若干历史问题的初步意见》《目前史学研究中存在的几个问题》《中国史论集》《中国史纲》《历史问题论丛》《先秦史》《秦汉史》等,合著:《中国历史概要》。
\mnitem{3}陆平,原任北京大学校长、党委书记。
\mnitem{4}李达,原任武汉大学校长。曾出席中共一大代表大会,后因故脱党,但仍坚持研究马克思主义,一九四八年底,李达接毛泽东信函,“吾兄系本公司发起人之一,现公司生意兴隆,望速前来参予经营。”一九四九年,又重新入党,毛泽东是他的历史证明人。
\mnitem{5}李雪锋,时任北京市委第一书书记,吴德,时任北京市委第二书记。
\mnitem{6}中华书局是我国整理、编校、出版古籍读物的权威出版机构,在国内外知名度颇高,影响深远。
\end{maonote}
