
\title{人的正确思想是从哪里来的?}
\date{一九六三年五月二十日}
\thanks{这是毛泽东同志主持起草的《中共中央关于目前农村工作中若干问题的决定》(草案)的节选。}
\maketitle


人的正确思想是从哪里来的?是从天上掉下来的吗?不是。是自己头脑里固有的吗?不是。人的正确思想,只能从社会实践中来,只能从社会的生产斗争、阶级斗争和科学实验这三项实践中来。人们的社会存在,决定人们的思想。而代表先进阶级的正确思想,一旦被群众掌握,就会变成改造社会、改造世界的物质力量。人们在社会实践中从事各项斗争,有了丰富的经验,有成功的,有失败的。无数客观外界的现象通过人的眼、耳、鼻、舌、身这五个官能反映到自己的头脑中来,开始是感性认识。这种感性认识的材料积累多了,就会产生一个飞跃,变成了理性认识,这就是思想。这是一个认识过程。这是整个认识过程的第一个阶段,即由客观物质到主观精神的阶段,由存在到思想的阶段。这时候的精神、思想(包括理论、政策、计划、办法)是否正确地反映了客观外界的规律,还是没有证明的,还不能确定是否正确,然后又有认识过程的第二个阶段,即由精神到物质的阶段,由思想到存在的阶段,这就是把第一个阶段得到的认识放到社会实践中去,看这些理论、政策、计划、办法等等是否能得到预期的成功。一般的说来,成功了的就是正确的,失败了的就是错误的,特别是人类对自然界的斗争是如此。在社会斗争中,代表先进阶级的势力,有时候有些失败,并不是因为思想不正确,而是因为在斗争力量的对比上,先进势力这一方,暂时还不如反动势力那一方,所以暂时失败了,但是以后总有一天会要成功的。人们的认识经过实践的考验,又会产生一个飞跃。这次飞跃,比起前一次飞跃来,意义更加伟大。因为只有这一次飞跃,才能证明认识的第一次飞跃,即从客观外界的反映过程中得到的思想、理论、政策、计划、办法等等,究竟是正确的还是错误的,此外再无别的检验真理的办法。而无产阶级认识世界的目的,只是为了改造世界,此外再无别的目的。一个正确的认识,往往需要经过由物质到精神,由精神到物质,即由实践到认识,由认识到实践这样多次的反复,才能够完成。这就是马克思主义的认识论,就是辩证唯物论的认识论。现在我们的同志中,有很多人还不懂得这个认识论的道理。问他的思想、意见、政策、方法、计划、结论、滔滔不绝的演说、大块的文章,是从哪里得来的,他觉得是个怪问题,回答不出来。对于物质可以变成精神,精神可以变成物质这样日常生活中常见的飞跃现象,也觉得不可理解。因此,对我们的同志,应当进行辩证唯物论的认识论的教育,以便端正思想,善于调查研究,总结经验,克服困难,少犯错误,做好工作,努力奋斗,建设一个社会主义的伟大强国,并且帮助世界被压迫被剥削的广大人民,完成我们应当担负的国际主义的伟大义务。
