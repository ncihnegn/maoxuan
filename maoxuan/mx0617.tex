
\title{在八届六中全会上的讲话}
\date{一九五八年十二月九日}
\maketitle


讲些意见,不是结论,决议就是这次会议的结论。

一、人民公社的出现,这是四月成都会议、五月党代表大会没有料到的。其实四月已在河南出现,五、六、七月都不知道,一直到八月才发现,北戴河会议作了决议。这是一件大事。找到了一种建设社会主义的形式,便于由集体所有制过渡到全民所有制,也便于由社会主义的全民所有制过渡到共产主义的全民所有制,便于工农商学兵,规模大,人多,便于办很多事。我们曾经说过,准备发生不吉利的事情,最大的莫过于战争和党的分裂。但也有些好事没料到。如人民公社四月就没料到,八月才作出决议。四个月的时间在全国搭起了架子,现在整顿组织。

二、保护劳动热情问题。犯错误的干部,主要是强迫命令,讲假话,以少报多,以多报少。以多报少危险不大,以少报多就很危险,一百斤报五十斤,不怕,本来是五十斤报一百斤就危险。主要的毛病在于不关心人民的生活,只注意生产。怎么处理?犯错误的人在干部中是少数,对于犯错误的人,百分之九十以上采取耐心说服的方法,一次、二次……不予处分,作自我批评就够了。大家议一议。不能以我一个人的意见,就作为结论。对于严重违法乱纪,脱离群众的干部,约占县、区、乡干部的百分之一、二、三、四、五到此为止。各地情况不同,应加以区别。对这一些人,应该予以处罚,因为他们脱离群众,群众很不喜欢他们。没有对百分之九十以上犯错误的干部采取不处分的方针,就不能保护干部。就会挫伤干部的热情,也会挫伤劳动者的热情。没有对严重违法乱纪的一部分人经过辩论,区分情节,给以轻重不同的处分,也会挫伤群众的热情,有些特别严重的要做刑事处理。总之,要有分析,其中有些是阶级异己分子,有些不是阶级异己分子,但情节恶劣的,如打人、骂人押人、捆人,要给予处分。湖北已经撤了一个县委第一书记,他在旱情严重时,没有抗旱,而谎报抗旱。总之处罚的要极少,教育的要极多,这就是能保护干部的热情。也就保护了劳动者的热情。对群众中间犯错误的人。方针也是如此。

三、苦战三年基本改变全国面貌问题。这个口号是否适当?三年办得到办不到?这个口号首先是河南同志提出来的。开始在南宁会议上我们釆取了这一口号,那时是指农村讲的。后来不知那一天,推广为“苦战三年。改变全国面貌”。曾希圣想说服我,拿出三张河网化地图,说农村可以基本改变。农村也许能够办到,至于全国,我看还要考虑一下。三年之内,大概能够搞到三干到四千万吨钢,六亿五千万人口的大国,搞三、四千万吨钢能说基本改变了面貌?这个标准,我看提的低了一点,不然,以后就没有什么改头了。以后五千万、六千万、一亿、二亿,算什么呢?我看大改还在后头呢!因此三年内还不能说基本改变了全国面貌。到一九六二年大概有五六千万吨钢。那时也许说基本改变了全国面貌。那时就有英美今天的水平了,是不是到那时还不说基本改变。因为六亿多人口的国家,面貌改得这样快,化装都化好了?到底怎么说好,值得商量一下,因为报纸已在大宣传。或者提五年基本改变,十年到十五年彻底改变,如何最好,请同志们考虑,或者超过英国叫基本改变,超过美国叫彻底改变。勉强去超,累得要死,不如稍微从容一点。假如不要这么多年,三年、四年就完成了怎么办?能提前实现也好嘛!提前的时间长一点,结果时间缩短了,我看也不吃亏,曾希圣有一个办法,无非是当“机会主义”。安徽去冬今春水利开始搞八亿土石方。以后翻了一番,变成十六亿。八亿是机会主义,十六亿是马克思主义。没有几天又搞了三十二亿,十六亿就有点“机会主义”了。后来提高到六十四亿了。我们把改变面貌的时间说长一点,无非是当“机会主义”者。这样的机会主义,很有味道,我愿意当,马克思赏识这种机会主义,不会批评我。

四、党内外某些争论问题:围绕人民公社。党内党外有各种议论,大概有几十万、几百万干部在议论,有一大堆问题搞不清楚,一人一说,十人十说,没有作全面分析,深入分析。国际上也有议论,大体上有几说:一说是性急一点,他们有冲天干劲,革命热情很高,非常宝贵,但未作历史分析,形势分析,国际分析,这些人好处是热情高,缺点是太急了,纷纷宣布进入全民所有制,两三年进入共产主义。这次决议的主要锋芒,是对着这一方面讲的。就是说不要太急了。太急了没有好处。有了这个决议,经过这个决议,经过几个星期,几个月,他们在实践中、辩论中可以大体上搞清楚。

五、研究政治经济问题。在这几个月内。读一读斯大林的《论社会主义经济问题》、《政治经济学教科书》(第三版)、《马恩列斯论共产主义社会》,拿出几个月时间,各省要组织一下。为了我们的事业,联系实际研究经济理论问题。目前有很大的理论意义和实际意义。郑州会议我曾经提过这个建议.我写了一封信建议大家研究。

六,研究辩证法问题。郑州会议时。不知是哪位同志提出“大集体,小自由”,这个提法很好。

要抓生产,又要抓生活。两条腿走路是对立统一的学说。都是属于辩证法范畴的。马克思关于对立统一的学说,一九五八年在我国有很大的发展。例如。在优先发展重工业的前提下,工农业同时并举,重工业与轻工业同时并举。中央工业与地方工业同时并举,大中小企业同时并举。小土群与大洋群,土法与洋法,几个并举。还有管理体制——中央统一领导和地方各级分级管理,从中央、省、地,县、公社一直到生产队。都给他一点权。完全无权是不利的。这几种思想,在我们党内已经确立了,这很好。小土群,大洋群也是并举的,还有中洋群,例如唐山、黄石港不是中吗?有没有小洋群?也有。还有洋土结合群。总之,复杂得很,这些事在社会主义阵营,有些国家认为是不合法的。不许可的,我们许可,在我们这里是合法的。许可好还是不许可好?还要看几年。但在我们这样的国家里,啥也没有,穷得要命。搞些小土群也好嘛!专大的太单调。在农业中也是很复杂的,有高产、中产、低产同时存在。实行耕作“三三制”是群众的创造,北戴河会议抓着了提出了三分之一种粮食。三分之一休闲、三分之一种树。这可能是农业革命的方向。又提出“八字宪法”:水肥土种密保工管。人不喝不行。植物不喝也不行。

在社会主义制度方面.在社会主义阶段.有两种所有制同时存在。是对立又是结合,是对立的统一。集体所有制中包含了社会主义全民所有制的核心因素。它的根本性质是集体所有制,并且包含有共产主义全民所有制的因素。尤金最近说,中国提出集体所有制中包含有共产主义因素是对的。说苏联集体所有制和全民所有制中,也包含着共产主义因素。资本主义社会不允许组织社会主义生产方式。但在共产党领导下的国家中,应该也可以允许共产主义因素的增长。斯大林没有解决这个问题,把三种所有制,即集体所有制,社会主义全民所有制,共产主义全民所有制绝对化,截然分开,是不对的。

以上这些可否都讲成辩证法的发展。

郑州会议提出“大集体,小自由”,现在又提出抓革命又抓生活。这都是辩证法的推广。武昌会议又提出实事求是,订计划又热又冷,要雄心很大,但又要有相当的科学分析。当然这个决议,想解决一切问题也不可能。我看这个决议慢一点发表为好。只发表一个公报明年三月人代会上发表,这和我们的雄心大志相符,避免了由于一九五八年大跃进而产生的某些不切合实际的想法,比较有根据。比较有科学分析了。对于钢的问题。明年搞三干万吨钢,我也赞成过。到武昌后,感到不妙。过去我也想过一九六二年搞到一亿或者一亿二千万吨.那时只担心需要不需要的问题。忧虑这些钢谁用,没有考虑到可能性的问题。后来又考虑到可能性的问题。一是可能,一是需要,今年一千零七十万吨累得要死,因而对可能性发生问题。明年三千万吨,后年六千万吨,一九六二年一亿二千万吨,是虚假的可能性,不是现实的可能性。现在,要把空气压缩一下。把盘子放小——一千八百至二千万吨。是否不能超过呢?到明年再看,二千二百至二干三百万吨都可以,行有余力则超过嘛,现在要压缩一下,不一定订那么高。留有余地,让群众的实践去超过我们的计划,这也是一个辩证法的问题。实践。包括我们领导干部的努力和群众的实践在内。提得低,由实践把它提高,这并不是机会主义。从一千一百万吨到二千万吨,成倍的增长,全世界从古以来就没有这样的“机会主义”。这里也要联系到国际主义,要和苏联,和整个社会主义阵营联系起来。要和整个世界工人阶级的国际团体联系起来,在这个问题上不要抢先。现在有些县总是好抢先,要先进入共产主义。其实要先进入共产主义的,应该是鞍钢、抚顺、辽宁、上海、天津。中国先进入共产主义跑到苏联前头,看起来不像样子。有没有可能也是问题。苏联的科学家有一百五十万,高等知识分子几百万,工程师五十万,比美国多。苏联有五千五百万吨钢,我们还只有这么一点。他积蓄的力量大,干部多,我们才开始。所以可能性也是成问题的。赫鲁晓夫提出的七年计划,还是准备进入共产主义,提出两种所有制,逐步合一,这是很好的事。一个不应该,一个不可能.即使我们可能先进也不应该(先进)。十月革命是列宁的事业,我们都不是学习列宁吗?急急忙忙有何意思!无非是到马克思那里去请赏。如果那样搞,可能在国际问题犯错误,要讲辩证法。要注意互相有利,辩证法有很大的发展,就涉及到这个问题。

七、郑州会议搞的十五年纲要,这次搁下没有谈,可能不可能,需要不需要,都缺乏根据,不仅缺乏充分的根据,而且缺乏初步的根据,苏联和美国的经验,都不能证明搞那么多,是不是可能?就是可能了,也找不到买主。因此目前不定这个纲要,我们可以每年到冬季拿出来谈一次。明年,后年,大后年都不作这种长期计划。大概到一九六二年可以作一个长期计划,再早是不行的,全党全民办工业搞了几年,可能和需要的问题也许到那时可以看出一点。这次会议没有谈。收起来了,有些同志失望了。

八、一九五八年军事工作有相当大的发展,一是大整风,二是官长下连队当兵,三是参加生产,四是大办民兵。自从六月在北京开整风会议后,各级一直开下来,到现在可能已经开得差不多了。训练,这件事,也不能丢,如果全去整风,生产、炼钢、搞公社、搞水利,那也不行,军队总是军队,训练是经常任务。

九、关于教育制度的改变。实行教育与劳动生产相结合的制度这也是一件大事,当然也发生了一点问题。例如,有的学生不想读书,劳动搞出味道了,如果很多人不想读书就成了问题。成了问题就开会,开了会又会读书。

十、两种可能性问题。一种事物总有两种对立的东西。我们的党也有两种可能.一是巩固,一是分裂。在上海时,一个中央分裂为两个中央,在长征中与张国焘分裂,高饶事件是部分分裂。部分的分裂是经常的。去年以来。全国有一半的省份在领导集团内发生了分裂。人身上海天都要脱发、脱皮,这就是灭亡一部分细胞。从小孩起就要灭亡一部分细胞,这才有利于生长。如果没有灭亡,人就不能生存。自从孔夫子以来,人要不灭亡那不得了。灭亡有好处,可以做肥料,你说不做,实际做了。精神上要有准备。部分的分裂每天都存在。分裂灭亡总会有的。没有分裂.不利于发展。整个的灭亡,也是历史的必然。整个讲,作为阶级斗争工具的党和国家,是要灭亡的。但在它的历史任务未完成前,是要巩固它,不希望分裂,但要准备分裂。没有准备,就要分裂。有准备。就可避免大分裂。大型、中型的分裂是暂时的。匈牙利事件是大型的,高饶事件、莫洛托失事件是中型的。每个支部都在起变化,有些开除,有些进来,有些工作很好,有些犯错误。永远不起变化是不可能的。列宁经常说:国家总有两种可能。或者胜利,或者灭亡。我们中华人民共和国也有两种可能,胜利下去,或者灭亡。列宁是不隐讳灭亡这种可能性的,我们人民共和国也有两种可能性,不要否定这种可能性。我们手里没有原子弹,打起来,三十六计,走为上计,他占北京、上海、武汉,我们打游击,倒退十几年,二十年,回到延安时代。所以一方面我们要积极准备,大搞钢铁,搞机器,搞铁路,争取三四年内搞几千万吨钢,建立起工业基础,使我们比现在更巩固。我们现在在全世界名声很大,一个是金门打炮,一个是人民公社,还有钢一千零七十万吨这几件大事。我看名声很大,而实力不强。还是“一穷二白”,手无寸铁,一事无成。现在不过有一寸铁而已,国家实际上是弱的,在政治上我们是强国,在军事装备上和经济上是弱国。因此我们目前的任务是由弱变强。苦战三年能否改变?三年恐怕不行。苦战三年,只能改变一部分,不能基本改变。再有四年,共七年时间,就比较好了,就名符其实了。现在名声很大,实力很小,这一点要看清楚。现在外国人吹的很大,许多报纸尽是大话,不要外国人一吹,就神乎其神,飘飘然了。其实今年好钢只有九百万吨,轧成钢材要打七折。只有六百多万吨。不要自己骗自己,粮食是不少。各地打了折扣以后是八千六百亿斤,我们讲七千三百亿斤,即翻一番多点,那一千一百亿斤不算,真有而不算,也不吃亏。东西还存在。我们只怕没有,有没有,没有查过,在座诸公都没有查过。就算有八千六百亿斤,四分之一是薯类。要估计到不高兴的这一面,索性讲清楚,把这些倒霉的事,在省,地、县开个会,吹一吹,有什么不可以,别人讲不爱听,我就到处讲讲倒霉的事,无非是公共食堂、公社垮台。党分裂,脱离群众,被美国占领,国家灭掉,打游击。我们有一条马克恩主义的规律管着,不管怎样,这些倒霉的事总是暂时的、局部的。我们历史上多少次的失败,都证明了这一点。匈牙利事件.万里长征,三十万军队变成两万几。三十万党员变成几万,都是暂时的、局部的。资产阶级的灭亡、帝国主义的灭亡,则是永久的。社会主义的挫折、失败、灭亡是暂时的,不久就要恢复。即使完全失败,也是暂时的,总要恢复的。人皆有死。个别的人总是要死的,而整个人类总是要发展下去的。两种可能性都谈,没有坏处。

十一、关于我不担任共和国主席问题。这次要做个正式决议,希望同志们赞成。要求三天之内,省里开一次电话会议。通知到地、县和人民公社,三天之后发表公报,以免下边感到突如其来。世界上的事就这么怪,能上不能下。估计到可能有一部分人赞成,一部分人不赞成。群众不了解,说大家干劲冲天,你临阵退却。要讲清楚,不是这样。我不退却,要争取超美后才去见马克思嘛!

十二、国际形势。今年这一年有很大的发展。敌人方面乱下去,一天天乱下去,我们方面好起来,一天天好起来。每天的报纸都证明这一点。真正丧气的是帝国主义。他们烂、乱、矛盾重重,四分五裂。他们的日子不好过,好过的日子过去了。他们没有变成帝国主义之前,只有资本主义时代是他们的好日子。现在的时代是帝国主义灭亡的时代,我们的情况会一天比一天好起来。当然,也要估计到还有长期的、曲折的、复杂的斗争.并且要估计到战争的可能性。有那么一些人想冒险,最反动的是垄断资产阶级,大多数是不愿打的。
