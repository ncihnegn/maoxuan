
\title{关于文艺工作的批示}
\date{一九六三年十二月十二日}
\thanks{这是毛泽东同志写在写在中共中央宣传部文艺处一九六三年十二月九日编印的《文艺情况汇报》上的两个批语。}
\maketitle


\section*{一}

\mxname{彭真、刘仁\mnote{1}同志:}

此件\mnote{2}可一看,各种艺术形式——戏剧、曲艺、音乐、美术、舞蹈、电影、诗和文学等等,问题不少,人数很多,社会主义改造在许多部门中,至今收效甚微。许多部门至今还是“死人”统治着。不能低估电影、新诗、民歌、美术、小说的成绩,但其中的问题也不少。至于戏剧等部门,问题就更大了。社会经济基础已经改变了,为这个基础服务的上层建筑之一的艺术部门,至今还是大问题。这需要从调查研究着手,认真地抓起来。

\section*{二}

许多共产党人热心提倡封建主义和资本主义的艺术,却不热心提倡社会主义的艺术,岂非咄咄怪事。

\begin{maonote}
\mnitem{1}彭真,时任中共中央政治局委员、中央书记处书记、北京市委第一书记兼北京市市长。刘仁,时任中共北京市委第二书记。
\mnitem{2}指《文艺情况汇报》上登载的中共上海市委第一书记兼上海市市长柯庆施抓曲艺工作的材料。这份材料说,上海市委很注意曲艺等群众艺术工具,柯庆施曾亲自抓曲艺工作。一个是抓评弹的长篇新书目建设问题。他说,有没有更多的在思想上和艺术上都不错的长篇现代书目,是关系到社会主义文艺能不能占领阵地的问题。另一个是抓故事员的问题。故事员在市郊配合社会主义教育运动大讲革命故事,起到了红色宣传员的作用,很受群众的欢迎。市委要求各级党组织重视这一活动,并可在城市中推广。
\end{maonote}
