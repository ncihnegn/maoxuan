
\title{坚决退赔,刹住“共产风”}
\date{一九六〇年十二月三十日}
\thanks{这是毛泽东同志在中共中央工作会议\mnote{1}上听取汇报时插话的节录。}
\maketitle


退赔问题很重要,一定要认真退赔。大多数都要由各地自己退赔,县、社一定要拿出一部分实物来退赔,现在拿不出实物的,可以给些票子,这就叫兑现。十二条\mnote{2}已经向农民讲了,现在农民鼓着眼睛看着我们能不能兑现,不兑现不行。在两三个月内把兑现问题解决了,农民积极性就来了。

县、社宁可把家业统统赔进去,破产也要赔。因为我们剥夺了农民,这是马列主义完全不许可的。平调农民的劳动果实,比地主、资本家剥削还厉害,资本家还要花点代价,只是不等价,平调却什么都不给。一定要坚决退赔,各部门、各行各业平调的东西都要坚决退赔。赔到什么都没有,公社只要有几个人、几间茅屋能办公就行。不要怕公社没有东西,公社原来就没有东西,它不是白手起家的,是“黑手”起家的。所有县、社的工业,房屋,其他财产等,凡是平调来的,都要退赔,只有退赔光了,才能白手起家。县、社干部可能会不满意,但是只有这样,才能得到群众,得到农民满意,得到工农联盟。我们在井冈山时期红四军的布告中就讲平买平卖,“八项注意”中就有买卖公平这一条。平买平卖就是等价交换。我们历来主张买卖公平,等价交换。公社在短短的时间内,搞来了那么多东西,怎么搞来的?那不是白手起家的,要坚决退赔。县、社两级和有关部门都要退赔,有实物退实物,有钱退钱。大办县、社工业,大办副食品基地,我们都同意过。几个大办一推行就成了一平二调\mnote{3}。县、社干部不满意不要紧,我们得到了农民群众的满意,就得到了一头。机关、学校、工厂、部队,谁平调了谁赔。社、县、省这一头赔了,少了,那一头就有了;这一头空了,那一头就实了。那一头就是几亿农民。要纠正“共产风”,就要真兑现,不受整、不痛一下就得不到教训。苦一下、痛一下,这样才能懂得马克思主义的等价交换这个原则。一平二调是上面搞出来的,谁搞谁负责。退赔兑现了,干部作风才能转变。

退赔也要有界限,在大办水利、大办交通、大办副食品基地等方面,要由国家退赔一部分。应当由国家退赔的,不能让县、社赔。退赔要让县、社两级干部思想上搞通,不通就不能改正。

这几年我们有些东西搞多了,搞快了,自己挨整是必要的。现在看来,建设只能逐步搞,恐怕要搞半个世纪。

看来“五风”\mnote{4}中最普遍、危害最大的是“共产风”和瞎指挥风。首先要把它们整掉。究竟哪些是生产瞎指挥风要搞清楚,不然就会变成无指挥、无计划。

贪污赃款一定要退,一年退不起,两年,三年,不行十年也一定要退。东西赔光了,要以自己的劳动来偿还,不这样贪污现象消灭不了。

机关、部队、学校圈用群众的土地,要坚决退还,机关、工厂的花园,通通都拿来种菜。今后发展副食品生产,只能开荒地,不能占用农民土地。李世民\mnote{5}胜利后封功臣,就是采用圈农民土地的办法。清军入关后也是如此。现在是军队、学校都圈地,又不给人家钱,这实际上是封建残余,一定要纠正。

现在这个时候不要讲九个指头与一个指头的问题。事实上有的地方的缺点、错误不是一个指头的问题,有的是两个指头,有的是三个指头。总之,把问题查清楚了,有多少,讲多少。有的同志提的,有右反右,有“左”反“左”,有什么反什么,有多少反多少,这几句话是好的。把问题弄清楚了,群众也清醒了,九个指头与一个指头的关系也就明白了。

这几年说人家思想混乱,首先是我们自己思想混乱。一方面我们搞了十八条\mnote{6},十四句话\mnote{7},也搞了六条指示\mnote{8},这些就是为了纠正“共产风”,纠正瞎指挥风;另一方面,又来了几个大办,大办钢铁,大办县、社工业,大办交通,大办文教,又大刮起“共产风”。这就是前后矛盾,对不起来。虽然我们没有叫大家去平调,但没有塞死漏洞。总结这些经验教训很重要。以后不要前后矛盾,不要一面反,一面又刮;一面反,一面又提倡。现在值得注意的一个问题是,庐山会议\mnote{9}后,估计今年是好年成。一以为有了郑州会议决议\mnote{10},有了上海会议十八条,“共产风”压下去了,对一个指头的问题作了解决;二以为反了右倾,鼓了干劲;三以为几个大办就解决问题了;四以为年成逢单不利逢双利。没有料到,一九六〇年天灾更大了,人祸也来了。这人祸不是敌人造成的,而是我们自己造成的。今年一平二调比一九五八年厉害,一九五八年只有四五个月,今年是一整年。敌人破坏也增加了,大办也不灵了,“共产风”大刮了。问题最大、最突出的是大搞工业,县以上工业抽调了五千万劳动力。一九五七年是二千四百万,一九五八年是四千四百万,一九五九年和一九六〇年又增加了六百万,合计比一九五七年增加二千六百万。当然,劳动力不完全都是从农村调来的,但是不管从哪里调来,总是影响农业生产的,比如吃粮就增加了嘛!

有几条经验教训:一、“共产风”必须反,不能掠夺农民,这是马列主义不许可的。二、几个大办又刮起了“共产风”,一说老风占的多,一说新风占的多,不管哪个多,总之是大刮,看起来只能中办、小办。三、抽调了大批劳动力,县以上工业就调了几千万。这三条经验教训,是主要的。要承认这些经验教训,不然就改不了。新增加的二千六百万人不回去,怎么得了?压下去是有困难的,但一定要压下去。

今后大办改成中办、小办。农村劳动力要好好组织,专业队砍掉百分之二。再把牲口问题好好研究一下。搞代食品是一条出路,再是从外国买粮,各省要尽可能搞一些外汇。要考虑到明年是不是还有天灾,天的事情我们管不了,不然明年又可能转不过来。

陈云\mnote{11}同志说的几条我都赞成。一是低标准、瓜菜代,今后几年都要注意。总之口粮标准不能高,好日子当穷日子过,有了储备,才能抗御灾害。二是人畜要休息。三是进口粮食。还要加上我刚才说的几条,把领导重点放在农业生产上,吃饭第一,市场第二,建设第三。总的说来,缩短工业战线,延长农业战线,把整风搞好,把抽掉的劳动力压下去,把“共产风”搞掉,把坏人搞掉,几个大办变成中办、小办。这样粮食生产多了,就可以多吃点粮了。还有,多产的要多吃一点,要有差别。

分析起来还是大有希望,过去三年的经验教训很有帮助,吸取这些经验教训,就可以转化为积极因素。

\begin{maonote}
\mnitem{1}这次中央工作会议一九六〇年十二月二十四日至一九六一年一月十三日在北京召开。
\mnitem{2}指一九六〇年十一月三日中共中央《关于农村人民公社当前政策问题的紧急指示信》,内容共十二条。
\mnitem{3}一平二调是人民公社化运动中“共产风”的主要表现,即:在公社范围内实行贫富拉平平均分配;县、社两级无偿调走生产队(包括社员个人)的某些财物。三收款,指银行将过去发放给农村的贷款统统收回。
\mnitem{4}“五风”,指“大跃进”和人民公社化运动中发生的“共产风”、浮夸风、命令风、干部特殊风和对生产瞎指挥风。
\mnitem{5}李世民,即唐太宗(五九九——六四九),六二六年至六四九年在位。
\mnitem{6}指一九五九年三月二十五日至四月一日在上海召开的中共中央政治局扩大会议的会议纪要《关于人民公社的十八个问题》。
\mnitem{7}指在郑州召开的中共中央政治局扩大会议一九五九年三月五日通过的《郑州会议纪要》规定的作为整顿和建设人民公社的方针的十四句话。
\mnitem{8}指毛泽东一九五九年四月二十九日关于农业问题给六级干部的信中讲的六条。
\mnitem{9}指一九五九年七月二日至八月一日在庐山召开的中共中央政治局扩大会议和八月二日至十六日召开的中共八届八中全会。
\mnitem{10}指《郑州会议记录》。这个记录是一九五九年二月二十七日至三月五日在郑州召开的中共中央政治局扩大会议形成并下发的。记录共三部分:(一)《郑州会议纪要》;(二)毛泽东在会议上的讲话;(三)《关于人民公社管理体制的若干规定(草案)》。
\mnitem{11}陈云,时任中共中央副主席、国务院副总理。
\end{maonote}
