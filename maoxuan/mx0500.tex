
\title{出版说明}
\maketitle


伟大的领袖和导师毛泽东主席的著作,是马克思列宁主义的不朽贡献。根据中共中央的决定,《毛泽东选集》第五卷现在出版了,以后各卷也将陆续出版。

过去出版的《毛泽东选集》第一卷至第四卷,是新民主主义革命时期的重要著作。第五卷和以后各卷,是社会主义革命和社会主义建设时期的重要著作。

在中华人民共和国成立以后的新的历史时期,毛泽东同志坚持马克思列宁主义的普遍真理和革命具体实践相结合的一贯原则。领导我党和我国人民,在进行社会主义革命和社会主义建设的斗争中,在反对高饶、彭德怀、刘少奇、林彪、王张江姚的修正主义路线的斗争中,在反对帝国主义和各国反动派的斗争中,在反对以苏修叛徒集团为中心的现代修正主义的斗争中,继承、捍卫和发展了马克思列宁主义。这个时期,毛泽东同志在理论上最伟大的贡献,就是系统的总结了我国的和国际的无产阶级专政的历史经验,运用唯物辩证法的对立统一这个基本观点,分析了社会主义的发展规律,创立了无产阶级专政下继续革命的伟大理论。毛泽东同志关于无产阶级革命和无产阶级专政的这种新思想、新结论,在哲学、政治经济学和科学社会主义方面,极大地丰富了马克思列宁主义的理论宝库。它不仅为我国人民指明了巩固无产阶级专政,防止资本主义复辟,建设社会主义的根本道路,而且具有伟大的深远的世界意义。\\[2\baselineskip]《毛泽东选集》第五卷是一九四九年九月到一九五七年的重要著作。毛泽东同志关于在生产资料所有制的社会主义改造基本完成以后无产阶级和资产阶级、社会主义道路和资本主义道路的斗争还长期存在的科学论断,关于正确区分和处理社会主义社会中敌我矛盾和人民内部矛盾这两类不同性质的矛盾的学说,关于无产阶级专政下继续革命的伟大理论,关于社会主义建设总路线的基本思想,就是在这卷著作中首次提出的。以后,特别是在无产阶级文化大革命中,毛泽东同志根据革命实践经验,不断地充实和发展了这些光辉思想。\\[2\baselineskip]毛泽东同志是当代最伟大的马克思列宁主义者。毛泽东思想是我党我军和我国人民团结战斗、继续革命的胜利旗帜,是国际无产阶级和各国革命人民的共同财富。毛泽东同志的思想和学说是永存的。

收入选集的毛泽东同志在社会主义革命和社会主义建设时期的著作,有一部分公开发表过,有一部分没有公开发表过,包括毛泽东同志起草的文件、手稿和讲话的正式记录。讲话记录在编辑时作了必要的技术性的整理。

\begin{longtable}[]{@{}l@{}}
\toprule
\vtop{\hbox{\strut中共中央毛泽东主席著作编辑出版委员会}\hbox{\strut一九七七年三月一日}}\tabularnewline
\bottomrule
\end{longtable}
