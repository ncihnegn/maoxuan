
\title{论夺权}
\date{一九六七年一月}
\thanks{这是毛泽东同志就夺权问题的一系列指示。}
\maketitle


\date{一九六七年一月十六日}
\section*{(一)在中共中央政治局常委扩大会议上的讲话摘要}

我们的干部十几年来有些人变质了。

左派群众起来夺权,这是好的。右派夺权当然不好。左派的力量发展很快。上海的左派工人去年十一月上旬只有一千多人,今年一月上旬一百万,再加上学生,就是主力。

过去是军队打江山,现在是工农自己打江山,军队帮助。

群众选举新的干部,就让他们选嘛!厂长、书记让他们去选。被接管的地方可以选。

接管很好,只管政务,不管业务,事情还是原来的人去搞,我们只管监督。

\date{一九六七年一月十七日}
\section*{(二)谢富治\mnote{1}对公安干部传达的指示}

接管是不可避免的。

我们这个政府,过去是上面派去少数干部和下面大多数留用人员组成了政府,不是工人、农民起来闹革命夺得了政府,这就很容易产生封建主义、修正主义的东西。

(谢富治:我们老一点的同志,对这个运动不理解,从开始就不理解,到现在还不理解,转不过弯来。)

转不过弯来靠边站,但给饭吃。

\date{一九六七年一月二十三日}
\section*{(三)对张春桥\mnote{2}上海来电请示夺权问题的答复}

如果权落在右派手里,权本来就在右派手里,夺过来。如果再被别人夺过去。仍然在右派手里,没有什么了不起,还可以再夺。

\date{一九六七年一月二十七日}
\section*{(四)在军委扩大会议上的讲话}

一、军队对文化大革命的态度,在运动开始时是不介入的,但实际上已介入了(如材料送到军队上去保管,有的干部去军队)。在现在的形势下,两条路线的斗争非常尖锐的情况下,不能不介人,介入就必须支持左派。

二、老干部的多数到现在对文化大革命还不了解,多数靠吃老本,过去有功劳要很好地在这次运动中锻炼自己,改造自己。要立新功,要立新劳。要坚决站在左派方面,不能和稀泥,坚决支持左派,之后在左派的接管和监督下,搞好工作。

三、关于夺权。报纸上说夺走资本主义道路当权派和坚持资产阶级反动路线顽固分子的权,不是这样的不能夺?现在看来不能仔细分,应该夺来再说,不能形而上学,否则受限制,夺来后,是什么性质的当权派,在运动后期再判断,夺权后报国务院同意。

四、夺权前的老干部和新夺权的干部要共同搞好业务,保守国家机密。

\date{一九六七年一月三十日}
\section*{(五)为《红旗》杂志一九六七年第三期社论《论无产阶级革命派的夺权斗争》所加的话}

无产阶级革命派联合起来,向党内一小撮走资本主义道路当权派夺权!

认为只要是当权派,就一概不相信,这是不对的。不分青红皂白,反对一切,排斥一切,打倒一切,是违背马克思列宁主义、毛泽东思想的阶级观点的。

对于犯有错误的干部,要正确对待,不能一概打倒。只要不是反党反社会主义分子而又坚持不改和累教不改的,就要允许他们改过,鼓励他们将功赎罪。惩前毖后、治病救人,这是党的传统政策。只有这样才能使犯错误的本人心悦诚服,也才能使无产阶级革命派取得大多数人的衷心拥护,使自己立于不败之地,否则是很危险的。

各级干部,都要经受无产阶级文化大革命的考验,都应该为无产阶级文化大革命建立新的功劳,不能躺在过去的成绩上自以为了不起,看轻新起来的革命小将。对自己只看见过去的功劳,而看不见今天的革命大方向。对新的革命小将则又只看见他们的某些缺点错误,而看不见他们的革命大方向是正确的。这样的看法是完全错误的,必须改过来。

在当前无产阶级同资产阶级及其在党内一小撮代理人决战的阶段,坚持反动立场的地主、富农和资产阶级右派分子、坏分子、反革命修正主义分子、美蒋特务,都纷纷出笼。这批牛鬼蛇神,造谣惑众,欺骗、拉拢一些不明真相的人,成立反革命组织,疯狂地进行反革命活动。例如,所谓“中国工农红旗军”,所谓“荣复军”、“联合行动委员会”以及其他一些被修正主义分子组织起来的名为“革命派”,实是保字派的组织,就是这种反动组织。这些组织中的多数群众是受蒙蔽的,是应当争取教育的。但是,这些反动组织的一小撮头头,却处心积虑地炮打无产阶级的革命司令部,向无产阶级革命派夺权,袭击革命群众组织,暗害革命群众,收买职工,停止生产,中断交通,破坏和抢劫国家财产。他们趁火打劫,妄想变天。这种反动组织,有的就是在顽固的走资本主义道路当权派的指挥下进行反革命活动的。这种反动组织,是建筑在沙滩上的,一旦被群众识破,就会立即土崩瓦解,一小撮头头就会被群众揪出来。

\begin{maonote}
\mnitem{1}谢富治,时任公安部部长,国务院副总理。
\mnitem{2}张春桥,时任中央文革副组长,上海革命委员会主任。
\end{maonote}
