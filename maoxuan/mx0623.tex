
\title{关于西藏问题和台湾问题}
\date{一九五九年五月十日}
\thanks{这是毛泽东同志同德意志民主共和国人民议院代表团谈话的主要部分。}
\maketitle


我们有两个问题没有解决,西藏问题和台湾问题。现在开始解决西藏问题。西藏面积不小,有一百二十万平方公里,相当于十二个民主德国。可是西藏地区的人口只有一百二十万。有人问中国共产党为什么长久不解决西藏问题,这主要是因为我们党过去很少与藏族接触,我们有意地把西藏的社会改革推迟。过去我们和达赖喇嘛\mnote{1}达成的口头协议是,在一九六二年以后再对西藏进行民主改革。过早了条件不成熟,这也和西藏的农奴制有关。西藏劳动群众占百分之九十五,剥削者占百分之五,也就是说有六万人是剥削者。我们要分化他们,争取一部分。现在条件成熟了,不要等到一九六三年了。这就要谢谢尼赫鲁\mnote{2}和西藏叛乱分子。他们的武装叛乱为我们提供了现在就在西藏进行改革的理由。叛乱分子拿起枪来打我们,这样就可以看清,他们谁是站在我们这边的,谁是搞叛乱的。

全部藏族人口不是一百二十万,而是三百万。一百二十万在西藏,一百八十万分布在川西、甘南、云南及青海北部。这一百八十万人中也有过叛乱,我们进行了平叛,现在已经基本上解决了问题。外国人武装了藏族反动统治者,很多喇嘛庙都曾经是叛乱者的根据地。现在在这一百八十万藏民聚居的地区建立了党组织,进行了土地改革,解放了农奴,建立了农业生产合作社。过去喇嘛不参加生产,现在百分之九十的喇嘛都参加生产了。人民组织了武装自卫队。

现在西藏问题好解决了,第一步是民主革命,把农奴主的土地分给农奴,第二步再组织合作社。六万农奴主中约有一万人逃到印度了,其余没有走的可分为左、中、右三派,我们将根据他们不同的政治态度来区别对待。对有些人,还要看他们究竟如何,我们可以在斗争中观察他们。总之,我们要争取多数人,使他们赞成改革。

我们在西藏的农村和城市中建立了党组织。藏族人民很好,很勤劳,和人民解放军一起同叛乱分子斗争,很快就能组织起来。我们已培养了近万名藏族干部。过去十年中,我们培养了青年藏族干部,他们学了汉语。在西藏工作的汉族的干部也学了藏语。在西藏,马列主义者和劳动者可以合作,而且合作得很好。

再谈谈台湾问题。

台湾问题暂时不能解决,问题是美国霸占着。

台湾人民很不喜欢美国人,也不喜欢蒋介石。但是要蒋介石好呢,还是不要他好?现在要他好,他是亲美派,但他还想自己统治。另外一批人也是亲美派,但想完全投降美国。

现在的一个具体问题是,蒋介石明年还做不做总统。美国不想让他做,但我们认为他应该做。他想要有自己的军队。你们知道,一九五七年五月二十四日台湾人民打烂了美国“使馆”\mnote{3}。美国人怀疑是蒋介石的儿子蒋经国搞的,他们认为蒋经国不可相信,因为他去苏联住了十来年,娶了苏联老婆。

去年打金门\mnote{4},那里没有美国军队,只有美国一个工作组,十几个人。这个地方和美国没有条约关系,而台湾却和美国有条约关系\mnote{5}。我们打金门是内战问题。杜勒斯\mnote{6}的方针是叫我们和蒋介石都不打。我们说,你们管不着,这是我们中国的地方,我们打不打是我们的事,你们不要多管;我们和你们美国只在一点上有关系,就是要求你们从台湾撤军。正因为这样,我们才在日内瓦、华沙同美国谈判\mnote{7}。美国要签订一个声明,要蒋介石不打我们,要我们也不打蒋介石。我们说不行,金、马、台、澎\mnote{8}问题是我们的内政,你们管不着,唯一的问题就是请你们搬家。

看来我们和美国还得谈下去。它不赞成我们,我们也不赞成它,谈多久我们不知道。已经谈了三年半,恐怕还会谈十年,这是世界上最长的谈判。你们不要怕我们会打台湾。我们打金、马是为了帮助蒋介石,因为美国想把金、马让给我们,自己占据台湾。我们放弃金、马,都给蒋介石。蒋介石一困难,我们就打金、马,美国就可以让蒋介石继续做总统。

美国有“战争边缘政策”\mnote{9},主要是为台湾问题而想出来的。去年我们也采取“边缘政策”。我们打金、马和蒋介石的增援船只,蒋介石就请美国帮助。美国人来了,但只在十二海里以外。我们光打蒋介石的船,不打美国船。美国船升起国旗,叫我们不要打它。美国一炮也没有打我们,我们也没有打它。

所以大家都在战争边缘上。

美国空军很守规矩,它总是和我们的海岸保持一定的距离。有一次我们打下一架美国飞机,因为它越了境,但美国不做声,不要我们赔。美国是强国,霸占的地区太宽,它的十个指头按着十个跳蚤动不了啦,一个跳蚤也都抓不住。力量一分散,事情就难办了。

\begin{maonote}
\mnitem{1}达赖喇嘛,即达赖喇嘛·丹增嘉措。一九三五年生,青海湟中人。原西藏地方宗教和政治领袖之一。曾任全国人大常委会副委员长、西藏自治区筹备委员会主任委员、中国佛教协会名誉会长。一九五九年西藏上层反动集团发动武装叛乱时逃往印度。。
\mnitem{2}尼赫鲁,时任印度总理。
\mnitem{3}一九五七年五月二十日,驻台湾美军士兵雷诺将路过美军住宅的中国人刘自然枪杀。二十三日,雷诺被美军顾问团军事法庭宣判无罪释放。二十四日,台北等地数万民众举行反美示威游行,要求惩办凶手。示威群众冲击并捣毁美国驻台“大使馆”和新闻处,包围美国军事顾问团总部和台北市警察局。
\mnitem{4}指炮击金门,一九五八年七月,台湾国民党当局在美国的支持下叫嚷“反攻大陆”,并不断炮击福建沿海村镇。为严惩国民党军,反对美国侵犯中国主权,人民解放军福建前线部队奉命于八月二十三日开始对国民党军金门防卫部和炮兵阵地等军事目标进行炮击,封锁了金门岛,中断国民党军的补给。九月初,美国向台湾海峡地区大量增兵,派军舰、飞机直接为国民党军运输舰护航,公然入侵中国领海。为打击美国的侵略行径,人民解放军前线部队又于九月八日对金门国民党军和海上舰艇进行全面炮击。至一九五九年一月七日,共进行七次大规模炮击,十三次空战,三次海战,击落击伤国民党军飞机三十六架,击沉击伤军舰十七艘,毙伤国民党军七千余人。
\mnitem{5}指美国和台湾当局签订《共同防御条约》。一九五〇年六月朝鲜战争爆发后,美国总统杜鲁门在公开宣布武装干涉朝鲜内战的同时,命令其海军第七舰队侵入台湾海峡。美国为使侵略中国领土的行为“合法化”,一九五四年十二月二日与台湾当局签订了《共同防御条约》。该条约规定:美国帮助台湾当局维持并发展武装部队;台湾遭到“武装攻击”时,“美国将采取行动”,对付“共同危险”;美国有在台湾、澎湖及其附近部署陆、海、空军的权利,还可扩及到经双方协议所决定的“其他领土”。一九五五年三月三日条约生效。一九七八年十二月十五日,美国政府就美利坚合众国和中华人民共和国建交发表的声明宣布,美台《共同防御条约》将予以终止。一九八〇年一月一日起该条约正式废除。
\mnitem{6}杜勒斯,一九五三年一月至一九五九年四月任美国国务卿。
\mnitem{7}指中美大使级会谈。一九五五年四月二十三日,周恩来总理在亚非会议八国代表团团长会议上声明:中国政府愿意同美国政府谈判,讨论和缓远东紧张局势问题,特别是和缓台湾地区紧张局势问题。同年七月二十五日,中美双方就举行大使级会谈达成协议,并于八月一日在瑞士日内瓦进行首次会谈。此后由于美方缺乏诚意,会谈中断。一九五八年八月对金门炮击开始后,美国政府公开表示准备恢复会谈,双方随即于九月十五日在波兰华沙恢复会谈。迄至一九七一年二月二十日,中美大使级会谈共举行了一百三十六次。由于美方坚持干涉中国内政的立场,会谈在和缓和消除台湾地区紧张局势问题上未取得任何进展。
\mnitem{8}金、马、台、澎,指金门、马祖、台湾和澎湖。
\mnitem{9}一九五六年一月,杜勒斯提出美国“不怕走到战争边缘,但要学会走到战争边缘,又不卷入战争的必要艺术”。这种主张被称为“战争边缘政策”。
\end{maonote}
