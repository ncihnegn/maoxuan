
\title{支持被压迫人民反对帝国主义的战争}
\date{一九六四年六月二十三日}
\thanks{这是毛泽东同志同智利新闻工作者代表团谈话的主要部分。}
\maketitle


打仗对我们没有好处。我们要进行建设,打仗就会把我们进行的建设打烂了。国民党打内战,跟我们打了好多年。后来我们又跟日本打了八年,不是我们打到日本去,而是日本打到中国来。讲长远一点,都是外国打到中国来。中国曾和英国进行了几次战争,如一八四〇年在广东开始的鸦片战争\mnote{1},又如八国联军的战争\mnote{2},英国等八个国家的军队占领了天津,打到北京。中日甲午战争\mnote{3},是一八九四年到一八九五年在旅顺、大连等地打的。以后日本占领了我们东北。在那以前,沙皇俄国同日本还在中国的土地上打过仗,那是在旅顺、辽阳、沈阳一带\mnote{4}。最后是第二次世界大战期间,日本几乎侵占全中国。这些都不是我们打到外国去,都是外国人打到中国来。中国人打到外国去,在古代有过,那是中国的皇帝,打到越南、朝鲜。以后日本占领了朝鲜,法国占领了越南。

一九一一年,我们推翻了清朝皇帝。接着就是各派军阀混战,那时中国还没有共产党。有了共产党以后,就进行了革命战争,那也不是我们要打,是帝国主义、国民党要打。一九二一年,中国成立了共产党,我就变成共产党员了。那时候,我们也没有准备打仗。我是一个知识分子,当一个小学教员,也没学过军事,怎么知道打仗呢?就是由于国民党搞白色恐怖,把工会、农会都打掉了,把五万共产党员杀了一大批,抓了一大批,我们才拿起枪来,上山打游击。后来经过万里长征,跑到北方来。我们的军队原有三十万人,这时只剩下两万多了。恰好是在人数少的时候,我们改正了错误,走上了正确的道路。后来我们的军队又有了发展。日本人走了之后,蒋介石再来打我们的时候,敌人就不行了,我们取得了革命胜利。到现在,我们搞建设只有十五年的时间。要改变中国的落后面貌,不是很短的时间能做到的,至少要几十年的工夫。

中国要和平。凡是讲和平的,我们就赞成。我们不赞成战争。但是,对被压迫人民的反对帝国主义的战争我们是支持的。对古巴、阿尔及利亚的革命战争,我们是支持的;对越南南方人民反对美国帝国主义的战争,我们也是支持的。这些革命是他们自己搞起来的。比如古巴,不是我们叫卡斯特罗起来革命,是他自己起来革命的\mnote{5}。你们相信吗?是美国叫他革命的,是美国走狗叫他革命的。又如阿尔及利亚,是我们叫本·贝拉\mnote{6}革命的吗?以前我们认都不认识这个人,到现在我还没有见过他。是他们自己起来革命的,他们成立了临时政府,我们就承认。他们要求支持,我们就给他们支持。帝国主义说我们是“好战分子”,在某一点上讲也有些道理。因为我们支持卡斯特罗,支持本·贝拉,支持越南南方人民的反美战争。还有一次,一九五〇年到一九五三年美国侵略朝鲜时,我们支持了朝鲜人民反对美帝国主义的战争。我们的这一方针是公开宣布的,是不会放弃它的,就是说,我们要支持各国人民反对帝国主义的战争。我们如果不支持,就会犯错误,就不是共产党员。你们知道,阿联\mnote{7}总统纳赛尔不是共产党员,但他支持过阿尔及利亚革命。他不是共产党员能支持阿尔及利亚,难道我们是共产党员就不能支持阿尔及利亚吗?当一百八十多年以前,华盛顿\mnote{8}起来反对英国的时候,法国支持了华盛顿,难道当时法国人是共产党员吗?那时中国还没有共产党,全世界都还没有共产党。共产党出世是十九世纪的事。大概我们这个“好战分子”的称号还要继续下去。

主要一条还是我们国内问题。在国内,我们把美国走狗蒋介石赶走了,把美国的势力也赶走了。所以美国对我们不那么高兴。我不是指美国人民,而是指美国资本家。在北京也有一些美国人,他们对我们是友好的。

美国要把拉丁美洲变成它的殖民地,这是指在经济上,许多时候也是在政治上。比如说,巴西前总统古拉特,我见过他,他的党是工人党,不是共产党,美国都不能容忍他,把他推翻了。甚至稍微不听美国话的吴庭艳\mnote{9},美国竟把他杀掉了。在美国国内也不是那么和平的。吴庭艳是被美国肯尼迪\mnote{10}政府杀掉的,没过一个月,肯尼迪也见上帝去了。

美国说我们是“侵略者”,我们说它是侵略者;它说我们是“好战分子”,我们说美国政府的大资本家是好战分子。究竟谁是侵略者、好战分子,要叫全世界人民来看。美国在中国周围市满了军事基地,而且侵占了中国的台湾。我们没有占领美国的什么岛屿,没有侵略任何拉丁美洲国家和非洲国家,只“侵略”了亚洲一个国家——中国。我们跟帝国主义打了几十年仗,把它们赶走了。这件事情使美国很不高兴,其他帝国主义也不高兴。不过它们现在没有办法,总不能从地球上把我们搬走,就同不能从地球上把你们智利搬走一样。它们想把古巴搬走也不行,甚至很小的国家比如阿尔巴尼亚,它们要搬走也不行。

美国人说我们政府不是今年要倒台,就是明年要倒台,这件事恐怕不那么真实。看来今年不会倒,明年不会倒,后年呢,我说也不会倒。要把我们政府打倒,需要美国、蒋介石打到我们这里来。即使他们来了,也不一定达到目的。他们曾经来过,可是打输了。现在南越只有一千四百万人口,美国在那里进也不好,退也不好,陷在泥坑里。对拉丁美洲,美国也是感到头痛的。在这一点上,我们是乐观的。全世界人民总要起来,要自己做主人,不要资本家做主人。因为我们相信这一点,并且公开说出这一点,所以那些资本家对我们不是那么有好感。但是,除了美国为什么有那么多资本家跟我们做生意呢?就是因为他们不干涉我们的内政。美国人想跟我们做生意,我们就是不做。他们想派新闻记者来,这也不成。我们认为大问题没有解决以前,这些小问题、个别问题可以不忙着去解决。所以智利新闻工作者代表团能来中国,美国记者就来不了。但是总有一天他们会来的,总有一天两国的关系会正常化的。我看还要十五年,因为已经过了十五年了,再加十五年就是三十年,如果还不够,就再加嘛。

\begin{maonote}
\mnitem{1}鸦片战争,是一八四〇年至一八四二年英国对中国发动的侵略战争。一八四〇年,英国政府因中国反对输入鸦片,借口保护通商,派兵侵略中国。中国军队在两广总督林则徐领导下进行了抵抗。广州人民自发地组织武装抗英团体,打击英国侵略军。福建、浙江等地人民也自发地掀起了抗英斗争。一八四二年英国军队侵入长江,迫使清政府同英国侵略者签订了中国近代史上第一个不平等条约《南京条约》。
\mnitem{2}指一九〇〇年英、美、德、法、俄、日、意、奥八个国家联合出兵侵略中国的战争。中国人民进行了英勇的抵抗。侵略者先后攻陷大沽、天津、北京等地,同时沙俄又单独入侵中国东北,迫使清政府于一九〇一年九月七日同这些国家签订了不平等的《辛丑条约》。
\mnitem{3}中日甲午战争,指一八九四年(甲午年)发生的中日战争。这次战争是日本军国主义者蓄意挑起的。日本军队先向朝鲜发动侵略并对中国的陆海军进行挑衅,随后大举侵入中国的东北。中国军队曾经英勇作战,但是由于清朝政府的腐败及缺乏坚决反对侵略的准备,中国方面遭到了失败。一八九五年,清朝政府被迫同日本签订了不平等的《马关条约》。
\mnitem{4}指一九〇四年至一九〇五年日本同沙俄为争夺在中国东北和朝鲜的权益而进行的一次战争。战场主要在中国东北境内的奉天(今沈阳市)、辽阳地区和旅顺口一带,使中国人民遭受巨大的损失。沙俄在战争中遭到失败,经美国调停,同日本签订《朴次茅斯和约》。日俄战争后,日本取代了沙俄在中国东三省南部的支配地位;日本对于朝鲜的独占地位,也在《朴次茅斯和约》中得到沙俄的承认。
\mnitem{5}一九五九年一月一日,古巴卡斯特罗领导的起义军推翻了巴蒂斯塔任总统的独裁政权,建立了革命政府。
\mnitem{6}本·贝拉,指艾哈迈德·本·贝拉,一九一八年生,阿尔及利亚民族解放阵线领导人之一。一九五六年因积极参与组织发动全国反法武装起义,被法国殖民当局监禁。一九五八年阿尔及利亚临时政府成立时,被缺席推选为第一副总理。一九六二年获释回国,同年九月阿尔及利亚民主人民共和国成立,任政府总理。一九六三年九月当选第一任总统兼武装部队最高统帅。一九六四年四月任民族解放阵线总书记。
\mnitem{7}阿联,阿拉伯联合共和国的简称。一九五八年由埃及、叙利亚合并组成。一九六一年九月叙利亚脱离阿拉伯联合共和国,成立阿拉伯叙利亚共和国。一九七一年阿拉伯联合共和国改名为阿拉伯埃及共和国。
\mnitem{8}华盛顿(一七三二——一七九九),美国第一任总统。一七七五年美国独立战争爆发后,任大陆军总司令,将武装落后、组织松散的地方民军整编训练成为能与英军正面抗衡的正规军,领导美国取得独立战争的胜利。
\mnitem{9}吴庭艳(一九〇一——一九六三),原“越南共和国”总统兼总理和国防部长。一九六三年十一月一日在美国策划的军事政变中,同其弟吴庭儒一起被击毙。
\mnitem{10}肯尼迪,一九六〇年至一九六三年任美国总统。一九六三年十一月二十三日被刺身亡。
\end{maonote}
