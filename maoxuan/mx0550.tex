
\title{加快手工业的社会主义改造}
\date{一九五六年三月四日}
\thanks{这是毛泽东同志在国务院有关部门汇报手工业工作情况时所作指示的一部分。}
\maketitle


(一)个体手工业社会主义改造的速度,我觉得慢了一点。今年一月省市委书记会议的时候,我就说过有点慢。一九五五年底以前只组织了二百万人。今年头两个月就发展了三百万人,今年基本上可以搞完,这很好。手工业的总产值,你们设想在三个五年计划期间平均每年增长百分之十点九,似乎低一点。第一个五年计划定低了,吃了点亏,现在可以不更改,你们要在工作中掌握。

(二)手工业合作社的规模,一般的一百人左右为宜,有的也可以几百人,有的也可以几十人。

(三)组织铁、木业合作社为农业生产服务,下乡修理农具,这个办法很好,农民一定欢迎。中国手工业几千年来就有这样做的。组织合作社以后,提高了技术,就能更好地为农民服务。

(四)你们说,在手工业改造高潮中,修理和服务行业集中生产,撤点过多,群众不满意。这就糟糕!现在怎么办?“天下大势,分久必合,合久必分”。

(五)手工业生产的劳动生产率,同半机械化、机械化生产比较,最高最低相差达三十多倍。每人每年平均产值,国营现代化工业是二万元到三万元,半机械化、机械化的合作社是五千元,百人以上的大型合作社是二千元,小型合作社是一千五百元,个体手工业是八百至九百元。把劳动生产率作一个比较,就清楚了:手工业要向半机械化、机械化方向发展,劳动生产率必须提高。

(六)手工业的各行各业都是做好事的。吃的、穿的、用的都有。还有工艺美术品,什么景泰蓝,什么“葡萄常五处女”\mnote{1}的葡萄。还有烤鸭子可以技术出口。有些服务性行业,串街游乡,修修补补,王大娘补缸,这些人跑的地方多,见识很广。北京东晓市有六千多种产品。

提醒你们,手工业中许多好东西,不要搞掉了。王麻子、张小泉的刀剪一万年也不要搞掉。我们民族好的东西,搞掉了的,一定都要来一个恢复,而且要搞得更好一些。

(七)提高工艺美术品的水平和保护民间老艺人的办法很好,赶快搞,要搞快一些。你们自己设立机构,开办学院,召集会议。杨士惠是搞象牙雕刻的,实际上他是很高明的艺术家。他和我坐在一个桌子上吃饭,看着我,就能为我雕像。我看人家几天,恐怕画都画不出来。

(八)国家调拨物资给合作社,要合理作价,不能按国家调拨价格作价。合作社和国家企业不一样,社会主义集体所有制和社会主义全民所有制有区别。合作社开始时期经济基础不大,需要国家帮助。国家将替换下来的旧机器和公私合营并厂后多余的机器、厂房,低价拨给合作社,很好。“将欲取之,必先与之”。待合作社的基础大了,国家就要多收税,原料还要加价。那时,合作社在形式上是集体所有,在实际上成了全民所有。

国家要帮助合作社半机械化、机械化,合作社本身也要努力发展半机械化、机械化。机械化的速度越快,你们手工业合作社的寿命就越短。你们的“国家”越缩小,我们的事业就越好办了。你们努力快一些机械化,多交一些给国家吧。

(九)手工业产值占全国工业总产值的四分之一,它的供产销为什么没有纳入国家计划?手工业这样大,应当纳入国家计划。

(十)有些地方党委忙,手工业排不上队,这不好。为什么有些干部不大愿意做手工业的工作?我倒很想搞这样的事,很重要嘛!

(十一)你们要在六万多个手工业合作社组织中,选择突出的例子,编写典型材料。各地区、各行各业都要有;好的、坏的,大的、小的,集中的、分散的,半机械化的、机械化的都要有。出一本书,像《中国农村的社会主义高潮》一样。


\begin{maonote}
\mnitem{1}确良指以吹制玻璃葡萄著名的北京手工艺人常家的五位妇女。
\end{maonote}
