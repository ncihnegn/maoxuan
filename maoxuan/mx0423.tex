
\title{中国人民解放军总部关于重行颁布三大纪律八项注意的训令}
\date{一九四七年十月十日}
\maketitle


一、本军三大纪律八项注意,实行多年\mnote{1},其内容各地各军略有出入。现在统一规定,重行颁布。望即以此为准,深入教育,严格执行。至于其它应当注意事项,各地各军最高首长,可根据具体情况,规定若干项目,以命令施行之。

二、三大纪律如下:

(一)一切行动听指挥;(二)不拿群众一针一线;(三)一切缴获要归公。

三、八项注意如下:

(一)说话和气;(二)买卖公平;(三)借东西要还;(四)损坏东西要赔;(五)不打人骂人;(六)不损坏庄稼;(七)不调戏妇女;(八)不虐待俘虏。


\begin{maonote}
\mnitem{1}三大纪律、八项注意,是毛泽东等在第二次国内革命战争中为中国工农红军制订的纪律,其具体内容在不同时候和不同部队略有出入。在红军初创时期,就已提出部队对待群众要说话和气,买卖公平,不拉夫,不打人,不骂人。一九二八年春工农红军在井冈山的时候,规定了三项纪律:第一、行动听指挥;第二、不拿工人农民一点东西;第三、打土豪要归公。一九二八年夏提出了六项注意:一、上门板,二、捆铺草,三、说话和气,四、买卖公平,五、借东西要还,六、损坏东西要赔。一九二九年以后,毛泽东又将三大纪律中的“不拿工人农民一点东西”,改为“不拿群众一针一线”;“打土豪要归公”改为“筹款要归公”,后来又改为“一切缴获要归公”。对于六项注意,增加了“洗澡避女人”和“不搜俘虏腰包”两项内容,从而成为三大纪律、八项注意。这些纪律,曾经是红军以及后来的八路军、新四军、人民解放军政治工作的重要内容,对于人民军队的建设,对于正确处理军队内部关系、团结人民群众和确立人民军队对待俘虏的正确政策,都起了伟大的作用。
\end{maonote}
