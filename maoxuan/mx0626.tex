
\title{如实公开报道灾情}
\date{一九五九年六月二十日、八月二日}
\thanks{这是毛泽东在两文件上写的批语。}
\maketitle


\date{一九五九年六月二十日}
\section{一、如实公开报道灾情}

\mxname{乔木、冷西\mnote{1}同志:}

广东大雨,要如实公开报道。全国灾情,照样公开报道,唤起人民全力抗争。一点也不要隐瞒。政府救济,人民生产自救,要大力报道提倡。工业方面重大事故灾害,也要报道,讲究对策。此件\mnote{2}阅后退回。

\date{一九五九年八月二日}
\section{二、印发河南、湖北两省关于抗旱情况报告的批语}

此两件\mnote{3}印发各同志。今年旱区达五六省,情况严重。广大群众在党领导下的伟大的抗旱斗争,已经起来。吴冷西同志:各省旱情及抗旱斗争,请令新华社如实报导,鼓干劲,一定要把抗旱抗到底,人定胜天,争取丰收。反对一切悲观失望情绪。

\begin{maonote}
\mnitem{1}乔木,即胡乔木(一九一二——一九九二),江苏盐城人,时任中共中央书记处候补书记、毛泽东的秘书。冷西,即吴冷西,一九一九年生,广东新会人,时任新华通讯社社长、《人民日报》总编辑。
\mnitem{2}指新华通讯社一九五九年六月十八日编印的《内部参考》第二八〇一期。这一期登载了《广东水灾继续发展,全省工作中心转入抗洪救灾》和《广州市人民极度关心汛情的发展》两篇报道。
\mnitem{3}指中共河南省委一九五九年七月三十日关于抗旱保苗情况给中央、国务院的简报和中共湖北省委七月二十八日关于抗旱情况给中央的报告。河南省委的简报说,我省进入七月以来,普遍呈现旱象,而且日趋严重。呈现旱象初期,省委就发出了抗旱保苗争取秋季丰收的紧急指示,各级党委立即投入了抗旱斗争。现在群众性的抗旱保苗运动已经形成高潮,取得了一定成绩。目前,旱象仍在向严重方向发展,抗旱保苗斗争进入艰苦阶段,省委要求必须同干旱斗争到底,保证抗旱斗争的彻底胜利。湖北省委的报告说,我省自七月以来,旱情发展非常迅速。省委决心要抗旱保丰收,已于七月十六日发出了关于防旱抗旱的紧急指示,十九日又召开了全省防旱抗旱紧急动员电话会议,明确以抗旱为当前工作的突出中心,提出了战胜灾害确保丰收的几个具体措施。
\end{maonote}
